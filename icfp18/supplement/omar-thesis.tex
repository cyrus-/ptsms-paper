\PassOptionsToPackage{svgnames,dvipsnames,svgnames}{xcolor}

%for a more compact document, add the option openany to avoid
%starting all chapters on odd numbered pages
\documentclass[12pt]{cmuthesis}
%\usepackage[usenames,dvipsnames,svgnames]{xcolor}
\newif\ificfp
\icfptrue
\newcommand{\todolater}[1]{{\color{magenta} TODO (Later): #1}}
\newcommand{\todo}[1]{{\color{red} TODO: #1}}
% some useful packages
\usepackage{times}     % use times font for document
%\usepackage{lmodern}
\usepackage{newtxtt}
\usepackage{bbm}
%\renewcommand{\ttdefault}{txtt} % use txtt for typewriter font
\usepackage{mathpazo}
\usepackage{mathpartir} % use package for inference rules
\usepackage{upgreek} % package for alternative greek letters (\uppi)
\usepackage{fullpage}
\usepackage{colortab}
%\usepackage{graphicx}
\usepackage[labelfont=bf]{caption}

% \usepackage{cleveref}
%\usepackage{MnSymbol}
\usepackage{fancyvrb}
\usepackage{microtype}
\usepackage[framemethod=tikz]{mdframed}

% Get just llangle and rrangle from MnSymbol
\makeatletter
\DeclareFontFamily{OMX}{MnSymbolE}{}
\DeclareSymbolFont{MnLargeSymbols}{OMX}{MnSymbolE}{m}{n}
\SetSymbolFont{MnLargeSymbols}{bold}{OMX}{MnSymbolE}{b}{n}
\DeclareFontShape{OMX}{MnSymbolE}{m}{n}{
    <-6>  MnSymbolE5
   <6-7>  MnSymbolE6
   <7-8>  MnSymbolE7
   <8-9>  MnSymbolE8
   <9-10> MnSymbolE9
  <10-12> MnSymbolE10
  <12->   MnSymbolE12
}{}
\DeclareFontShape{OMX}{MnSymbolE}{b}{n}{
    <-6>  MnSymbolE-Bold5
   <6-7>  MnSymbolE-Bold6
   <7-8>  MnSymbolE-Bold7
   <8-9>  MnSymbolE-Bold8
   <9-10> MnSymbolE-Bold9
  <10-12> MnSymbolE-Bold10
  <12->   MnSymbolE-Bold12
}{}

\let\llangle\@undefined
\let\rrangle\@undefined
\DeclareMathDelimiter{\llangle}{\mathopen}%
                     {MnLargeSymbols}{'164}{MnLargeSymbols}{'164}
\DeclareMathDelimiter{\rrangle}{\mathclose}%
                     {MnLargeSymbols}{'171}{MnLargeSymbols}{'171}
\makeatother

% \usepackage{rotating}
% \usepackage{pdflscape}

\usepackage[colorlinks=true,allcolors=Blue,backref,pageanchor=true,plainpages=false, pdfpagelabels, bookmarks,bookmarksnumbered,
pdfborder={0 0 0},  %removes outlines around hyper links in online display
]{hyperref}

\usepackage{amsmath,amssymb, amsthm}

\allowdisplaybreaks[1]
\newtheorem{theorem}{Theorem}[chapter]
\newtheorem{lemma}[theorem]{Lemma}
\newtheorem{corollary}[theorem]{Corollary}
\newtheorem{definition}[theorem]{Definition}
\newtheorem{assumption}[theorem]{Assumption}
\newtheorem{condition}[theorem]{Condition}

\usepackage{pfsteps}

\usepackage[noabbrev]{cleveref}

\newenvironment{proof-sketch}{\noindent{\emph{Proof Sketch.}}}{\qed}

\makeatletter
\renewenvironment{proof}[1][\proofname]{\par
  \vspace{-\topsep}% remove the space after the theorem
  \pushQED{\qed}%
  \normalfont
  \topsep0pt \partopsep0pt % no space before
  \trivlist
  \item[\hskip\labelsep
        \itshape
    #1\@addpunct{.}]\ignorespaces
}{%
  \popQED\endtrivlist\@endpefalse
  \addvspace{6pt plus 6pt} % some space after
}
\makeatother
\makeatletter
\renewenvironment{proof-sketch}[1][\proofname]{\par
  \vspace{-\topsep}% remove the space after the theorem
  \pushQED{\qed}%
  \normalfont
  \topsep0pt \partopsep0pt % no space before
  \trivlist
  \item[\hskip\labelsep
        \itshape
    Proof Sketch\@addpunct{.}]\ignorespaces
}{%
  \popQED\endtrivlist\@endpefalse
  \addvspace{6pt plus 6pt} % some space after
}
\makeatother




\usepackage{mathtools}
\usepackage{ stmaryrd }
\usepackage[numbers,sort]{natbib}

\usepackage{subfigure}

% Approximately 1" margins, more space on binding side
%\usepackage[letterpaper,twoside,vscale=.8,hscale=.75,nomarginpar]{geometry}
%for general printing (not binding)
\usepackage[letterpaper,twoside,vscale=.8,hscale=.75,nomarginpar,hmarginratio=1:1]{geometry}

\usepackage{listings}
\lstset{tabsize=2, 
basicstyle=\ttfamily\fontsize{11pt}{1em}\selectfont, 
commentstyle=\itshape\ttfamily\color{gray}, 
stringstyle=\ttfamily\color{red},
mathescape=false,escapechar=\#,
numbers=left, numberstyle=\scriptsize\color{gray}\ttfamily, language=ML,moredelim=[il][\sffamily]{?},showspaces=false,showstringspaces=false,xleftmargin=15pt, morekeywords=[1]{tyfam,opfam,let,fn,val,def,casetype,objtype,metadata,of,*,datatype,new,toast,syntax,module,where,import,for,ana,syn,opcon,tycon,metasignature,metamodule,metasig,metamod,static,at,tycase,mod,macro,match,pattern,in,patterns,expressions,implicit,forall,rectype,fold,unfold,inj,by,spliced},deletekeywords={double},classoffset=0,belowskip=\smallskipamount,
moredelim=**[is][\color{red}]{SSTR}{ESTR},
moredelim=**[is][\color{Green}]{SHTML}{EHTML},
moredelim=**[is][\color{purple}]{SCSS}{ECSS},
moredelim=**[is][\color{brown}]{SSQL}{ESQL},
moredelim=**[is][\color{orange}]{SCOLOR}{ECOLOR},
moredelim=**[is][\color{magenta}]{SPCT}{EPCT}, 
moredelim=**[is][\color{gray}]{SNAT}{ENAT}, 
moredelim=**[is][\color{Green}]{SURL}{EURL},
moredelim=**[is][\color{blue}]{SURI}{EURI},
moredelim=**[is][\color{SeaGreen}]{SQT}{EQT},
moredelim=**[is][\color{Periwinkle}]{SGRM}{EGRM},
moredelim=**[is][\color{YellowGreen}]{SID}{EID},
moredelim=**[is][\color{Sepia}]{SUS}{EUS},
deletestring=[d]{"},
}
\lstloadlanguages{Java,VBScript,XML,HTML,ML}
\let\li\lstinline

% http://tex.stackexchange.com/q/43526
% fix the apparently deliberate but undocumented behaviour of disabling escapes other than mathescape in TextStyle (used by \lstinline)
% there may be a good reason why this is disabled by default, so beware in case it causes any problems
\usepackage{etoolbox}
\makeatletter
\patchcmd{\lsthk@TextStyle}{\let\lst@DefEsc\@empty}{}{}{\errmessage{failed to patch}}
\makeatother

% Provides a draft mark at the bottom of the document. 
% \draftstamp{\today}{DRAFT}


% I hate hyphenation.
%\lefthyphenmin=5
\definecolor{light-gray}{gray}{0.95}

% !TEX root = icfp18-supplement.tex
\newcommand{\dolla}{\texttt{\$}}  % used so I don't screw up syntax highlighting when using $ in an identifier inline

% \newcommand{\gheading}[1]{\multicolumn{3}{l}{\textbf{#1}}}

\newcommand{\elided}{{\color{gray}\cdots}}

\newcommand{\simplelang}{\textsf{\textbf{\small Calc}}}
\newcommand{\footnotesimplelang}{\textsf{\textbf{\footnotesize Algor}}}
\newcommand{\tnum}{\texttt{num}}
\newcommand{\numintro}[1]{\texttt{#1}}
\newcommand{\anumintro}[1]{\texttt{num}[#1]}
\newcommand{\aplus}[2]{\abop{plus}{#1; #2}}
\newcommand{\aminus}[2]{\abop{minus}{#1; #2}}
\newcommand{\amult}[2]{\abop{mult}{#1; #2}}
\newcommand{\adiv}[2]{\abop{div}{#1; #2}}
\newcommand{\apow}[2]{\abop{pow}{#1; #2}}

\newcommand{\letplain}[3]{\texttt{let}~#1=#2~\texttt{in}~#3}
\newcommand{\aletplain}[3]{\abop{let}{#1; #2.#3}}

% Calculi Names
\newcommand{\sysFLetter}{\textsf{\textbf{ML}}}
\newcommand{\miniVerseBase}{\sysFLetter}
\newcommand{\miniVerseUE}{\miniVerseBase^\textsf{ELit}}
\newcommand{\miniVersePat}{\miniVerseBase^\textsf{Lit}}
\newcommand{\miniVerseParam}{\miniVerseBase^\textsf{Lit/P}}
\newcommand{\miniVerseTSL}{\mathsf{miniVerse}_\textsf{TSL}}
\newcommand{\miniVerseUB}{\miniVerseBase^\textsf{LitSB}}
\newcommand{\miniVersePH}{\miniVerseBase^\textsf{Lit/PH}}

% General abstract syntax
\newcommand{\aboppz}[2]{\texttt{#1}\texttt{[}#2\texttt{]}}
\newcommand{\abopbz}[2]{\texttt{#1}[#2]}
\newcommand{\abop}[2]{\texttt{#1}\texttt{(}#2\texttt{)}}
\newcommand{\abopi}[3]{\texttt{#1}[#2]\texttt{(}#3\texttt{)}}
\newcommand{\abopii}[4]{\texttt{#1}[#2][#3]\texttt{(}#4\texttt{)}}
\newcommand{\abopic}[4]{\texttt{#1}[#2]\texttt{\{}#3\texttt{\}(}#4\texttt{)}}
\newcommand{\abopp}[3]{\texttt{#1}\texttt{[}#2\texttt{](}#3\texttt{)}}
\newcommand{\abopc}[3]{\texttt{#1}\texttt{\{}#2\texttt{\}(}#3\texttt{)}}
\newcommand{\abopbc}[4]{\texttt{#1}\texttt{[}#2\texttt{]\{}#3\texttt{\}(}#4\texttt{)}}
\newcommand{\abopibc}[5]{\texttt{#1}[#2]\texttt{[}#3\texttt{]\{}#4\texttt{\}(}#5\texttt{)}}
\newcommand{\abopcc}[4]{\texttt{#1}\texttt{\{}#2\texttt{\}\{}#3\texttt{\}(}#4\texttt{)}}
\newcommand{\abopicz}[3]{\texttt{#1}[#2]\texttt{\{}#3\texttt{\}}}

% Types / candidate expansion types
\newcommand{\parr}[2]{#1 \rightharpoonup #2}
\newcommand{\aparr}[2]{\abop{parr}{#1; #2}}
\newcommand{\auparr}[2]{\abop{uparr}{#1; #2}}
\newcommand{\aceparr}[2]{\abop{prparr}{#1; #2}}

\newcommand{\forallt}[2]{\forall #1.#2}
\newcommand{\aall}[2]{\abop{all}{#1.#2}}
\newcommand{\auall}[2]{\abop{uall}{#1.#2}}
\newcommand{\aceall}[2]{\abop{prall}{#1.#2}}

\newcommand{\forallu}[3]{\forall(#1 :: #2).#3}
\newcommand{\aallu}[3]{\abopc{all}{#1}{#2.#3}}
\newcommand{\auallu}[3]{\abopc{uall}{#1}{#2.#3}}
\newcommand{\aceallu}[3]{\abopc{prall}{#1}{#2.#3}}

\newcommand{\rect}[2]{\mu #1.#2}
\newcommand{\arec}[2]{\abop{rec}{#1.#2}}
\newcommand{\aurec}[2]{\abop{urec}{#1.#2}}
\newcommand{\acerec}[2]{\abop{prrec}{#1.#2}}

\newcommand{\prodt}[1]{\langle #1 \rangle}
% \newcommand{\aprod}[2]{\abopi{prod}{#1}{#2}}
\newcommand{\aprod}[2]{\abop{prod}{#2}}
\newcommand{\auprod}[2]{\abopi{uprod}{#1}{#2}}
% \newcommand{\aceprod}[2]{\abopi{prprod}{#1}{#2}}
\newcommand{\aceprod}[2]{\abop{prprod}{#2}}

\newcommand{\sumt}[1]{[#1]}
% \newcommand{\asum}[2]{\abopi{sum}{#1}{#2}}
\newcommand{\asum}[2]{\abop{sum}{#2}}
\newcommand{\ausum}[2]{\abopi{usum}{#1}{#2}}
% \newcommand{\acesum}[2]{\abopi{prsum}{#1}{#2}}
\newcommand{\acesum}[2]{\abop{prsum}{#2}}

% Labels and maps
\newcommand{\labelset}{L}
\newcommand{\smapschema}[3]{\{#1_{#2}\}_{#2 \in #3}}
\newcommand{\mapschema}[3]{\{#2 \hookrightarrow #1_{#2}\}_{#2 \in #3}}
\newcommand{\mapschemab}[4]{\{#3 \hookrightarrow #1_{#3}.#2_{#3}\}_{#3 \in #4}}
\newcommand{\mapschemax}[4]{\{#3 \hookrightarrow #1(#2_{#3})\}_{#3 \in #4}}
\newcommand{\mapschemabx}[5]{\{#4 \hookrightarrow \sigilof{#2_{#4}}.#1(#3_{#4})\}_{#4 \in #5}}
\newcommand{\mapschemacx}[5]{\{#4 \hookrightarrow {#2_{#4}}.#1(#3_{#4})\}_{#4 \in #5}}
\newcommand{\finmap}[1]{#1}
\newcommand{\mapitem}[2]{#1 \hookrightarrow #2}
\newcommand{\lbltxt}[1]{\mathtt{#1}}

% sequences
\newcommand{\seqschema}[4]{\{#1_{#2}\}_{#3 \leq #2 \leq #4}}
\newcommand{\seqschemaijb}[8]{\{\{#1_{#3, #6}\}_{#7 \leq #6 < #8}.#2_{#3}\}_{#4 \leq #3 \leq #5}}
\newcommand{\seqschemaj}[4]{\{#1_{#2}\}_{#3 \leq #2 < #4}}
\newcommand{\seqschemaX}[1]{\seqschema{#1}{i}{1}{n}}
\newcommand{\seqschemaXx}[2]{\{#1(#2_i)\}_{1 \leq i \leq n}}
\newcommand{\seqschemajX}[1]{\seqschema{#1}{j}{1}{n}}

\newcommand{\sseq}[2]{\{#1\}_{0 \leq i < #2}}
\newcommand{\sseqS}[3]{\{#1\}_{#2 \leq i < #3}}

% Expanded/Unexpanded/Candidate expressions
\newcommand{\asc}[2]{#1 : #2}
\newcommand{\auasc}[2]{\abopc{uasc}{#1}{#2}}
\newcommand{\aceasc}[2]{\abopc{prasc}{#1}{#2}}

\newcommand{\letsyn}[3]{\texttt{let\,val}~#1=#2~\texttt{in}~#3}
\newcommand{\aletsyn}[3]{\abop{letval}{#2; #1.#3}}
\newcommand{\auletsyn}[3]{\abop{uletval}{#2; #1.#3}}
\newcommand{\aceletsyn}[3]{\abop{prletval}{#2; #1.#3}}

\newcommand{\analam}[2]{\lambda #1.#2}
\newcommand{\auanalam}[2]{\abop{uanalam}{#1.#2}}
\newcommand{\aceanalam}[2]{\abop{pranalam}{#1.#2}}

\newcommand{\lam}[3]{\lambda #1{:}#2.#3}
\newcommand{\aelam}[3]{\abopc{lam}{#1}{#2.#3}}
\newcommand{\aulam}[3]{\abopc{ulam}{#1}{#2.#3}}
\newcommand{\acelam}[3]{\abopc{prlam}{#1}{#2.#3}}

\newcommand{\ap}[2]{#1(#2)}
\newcommand{\aeap}[2]{\abop{ap}{#1; #2}}
\newcommand{\auap}[2]{\abop{uap}{#1; #2}}
\newcommand{\aceap}[2]{\abop{prap}{#1; #2}}

\newcommand{\Lam}[2]{\Lambda #1.#2}
\newcommand{\aetlam}[2]{\abop{tlam}{#1.#2}}
\newcommand{\autlam}[2]{\abop{utlam}{#1.#2}}
\newcommand{\acetlam}[2]{\abop{prtlam}{#1.#2}}

\newcommand{\App}[2]{#1\texttt{[}#2\texttt{]}}
\newcommand{\aetap}[2]{\abopc{tap}{#2}{#1}}
\newcommand{\autap}[2]{\abopc{utap}{#2}{#1}}
\newcommand{\acetap}[2]{\abopc{prtap}{#2}{#1}}

\newcommand{\clam}[3]{\Lambda #1{::}#2.#3}
\newcommand{\aeclam}[3]{\abopc{clam}{#1}{#2.#3}}
\newcommand{\auclam}[3]{\abopc{uclam}{#1}{#2.#3}}
\newcommand{\aceclam}[3]{\abopc{prclam}{#1}{#2.#3}}

\newcommand{\cAp}[2]{#1\texttt{[}#2\texttt{]}}
\newcommand{\aecap}[2]{\abopc{cap}{#2}{#1}}
\newcommand{\aucap}[2]{\abopc{ucap}{#2}{#1}}
\newcommand{\acecap}[2]{\abopc{prcap}{#2}{#1}}

\newcommand{\foldt}[2]{\texttt{fold}_{#1}(#2)}
\newcommand{\fold}[1]{\texttt{fold}(#1)}
\newcommand{\aefold}[1]{\abop{fold}{#1}}
\newcommand{\aufold}[3]{\abopc{ufold}{#1.#2}{#3}}
\newcommand{\acefold}[1]{\abop{prfold}{#1}}

\newcommand{\auanafold}[1]{\abop{ufold}{#1}}
\newcommand{\aceanafold}[1]{\abop{prfold}{#1}}

\newcommand{\unfold}[1]{\texttt{unfold}(#1)}
\newcommand{\aeunfold}[1]{\abop{unfold}{#1}}
\newcommand{\auunfold}[1]{\abop{uunfold}{#1}}
\newcommand{\aceunfold}[1]{\abop{prunfold}{#1}}

\newcommand{\tpl}[1]{\langle #1\rangle}
% \newcommand{\aetpl}[2]{\abopi{tpl}{#1}{#2}}
\newcommand{\aetpl}[2]{\abop{tpl}{#2}}
\newcommand{\autpl}[2]{\abopc{utpl}{#1}{#2}}
% \newcommand{\acetpl}[2]{\abopc{prtpl}{#1}{#2}}
\newcommand{\acetpl}[2]{\abop{prtpl}{#2}}

\newcommand{\prj}[2]{#1 \cdot #2}
\newcommand{\aepr}[2]{\abopi{prj}{#1}{#2}}
\newcommand{\aupr}[2]{\abopi{uprj}{#1}{#2}}
\newcommand{\acepr}[2]{\abopi{prprj}{#1}{#2}}

\newcommand{\injt}[3]{\texttt{inj}_{#1}~#2 \cdot #3}
\newcommand{\inj}[2]{\texttt{inj}[#1](#2)}
\newcommand{\aein}[2]{\abopi{inj}{#1}{#2}}
\newcommand{\auin}[4]{\abopic{uinj}{#1; #2}{#3}{#4}}
\newcommand{\acein}[2]{\abopi{prinj}{#1}{#2}}

\newcommand{\auanain}[2]{\abopp{uin}{#1}{#2}}
\newcommand{\aceanain}[2]{\abopp{prinj}{#1}{#2}}

\newcommand{\caseof}[2]{\texttt{case}~#1~#2}
% \newcommand{\aecase}[3]{\abopi{case}{#1}{#2; #3}}
\newcommand{\aecase}[3]{\abop{case}{#2; #3}}
\newcommand{\aucase}[4]{\abopic{ucase}{#1}{#2}{#3; #4}}
% \newcommand{\acecase}[3]{\abopi{prcase}{#1}{#2; #3}}
\newcommand{\acecase}[3]{\abop{prcase}{#2; #3}}

% Expanded expressions
\newcommand{\etxt}[1]{e_\text{#1}}

\newcommand{\Uofv}{\mathcal{U}}
\newcommand{\Uof}[1]{\Uofv(#1)}
\newcommand{\sigilof}[1]{\widehat{#1}}
\newcommand{\VTypofv}{\mathcal{V}_\mathsf{Typ}}
\newcommand{\VTypof}[1]{\VTypofv(#1)}
\newcommand{\VExpofv}{\mathcal{V}_\mathsf{Exp}}
\newcommand{\VExpof}[1]{\VExpofv(#1)}
\newcommand{\Cofv}{\mathcal{P}}
\newcommand{\Cof}[1]{\Cofv(#1)}

% Statics of miniVerseU and miniVerseParam expanded expressions
\newcommand{\istypeU}[2]{#1 \vdash #2~\mathsf{type}}
\newcommand{\isctxU}[2]{#1 \vdash #2~\mathsf{ctx}}
\newcommand{\hastypeU}[4]{#1~#2 \vdash #3 : #4}
\newcommand{\hastypeUC}[2]{\vdash #1 : #2}
\newcommand{\hastypeUCO}[3]{#1 \vdash #2 : #3}

%\newcommand{\isctxP}[2]{#1 \vdash #2~\mathsf{ctx}}
\newcommand{\hastypeP}[3]{#1 \vdash #2 : #3}
\newcommand{\hastypePX}[2]{\hastypeP{\uOmega}{#1}{#2}}
\newcommand{\hastypePC}[2]{\vdash #1 : #2}

\newcommand{\Dhyp}[1]{#1~\mathsf{type}}
\newcommand{\Dcons}[2]{{#1}\cup{#2}}
\newcommand{\Ghyp}[2]{#1 : #2}
\newcommand{\Gcons}[2]{{#1}\cup{#2}}
\newcommand{\Gconsi}[2]{\cup_{#1} #2}
\newcommand{\GIconsi}[2]{\uplus_{#1} #2}
\newcommand{\Khyp}[1]{#1~\mathsf{kind}}

% Dynamics of miniVerseU
\newcommand{\isvalU}[1]{#1~\mathsf{val}}
\newcommand{\stepsU}[2]{#1 \mapsto #2}
\newcommand{\multistepU}[2]{#1 \mapsto^{*} #2}
\newcommand{\evalU}[2]{#1 \Downarrow #2}

% Unexpanded types
\newcommand{\utau}{{\hat\tau}}
\newcommand{\ut}{{\hat{t}}}
\newcommand{\uc}{{\hat c}}
\newcommand{\uu}{{\hat u}}

% Unexpanded kinds
\newcommand{\ukappa}{{\hat\kappa}}

% Unexpanded modules
\newcommand{\uM}{{\hat M}}
\newcommand{\usigma}{{\hat\sigma}}

% Unexpanded expressions
\newcommand{\ue}{{\hat e}}
\newcommand{\ux}{{\hat x}}
% \newcommand{\uesyntax}[4]{\texttt{syntax}~#1~\texttt{at}~#2~\texttt{by}~\texttt{static}~#3~\texttt{in}~#4}
\newcommand{\uesyntax}[4]{\texttt{notation}~#1~\texttt{at}~#2~\{~\texttt{expr}~\texttt{parser}~#3;\texttt{expansions}~\texttt{require}~\ue~\}~\texttt{in}~#4}
\newcommand{\uesyntaxq}[4]{\texttt{notation}~#1~\texttt{at}~#2~\{~\texttt{expr}~\texttt{parser}~#3;\texttt{expansions}~\texttt{require}~\ue~\}~\texttt{in}~#4}
\newcommand{\uedep}{\ue_\text{dep}}
\newcommand{\edep}{e_\text{dep}}
\newcommand{\taudep}{\tau_\text{dep}}
\newcommand{\uesyntaxqr}[4]{\texttt{notation}~#1~\texttt{at}~#2~\{~\texttt{expr}~\texttt{parser}~#3;\texttt{expansions}~\texttt{require}~\uedep~\}~\texttt{in}~#4}
% \newcommand{\uesyntaxq}[4]{\texttt{syntax}~#1~\texttt{at}~#2~\texttt{for}~\texttt{expressions}~\texttt{by}~\texttt{static}~#3~\texttt{in}~#4}

\newcommand{\audefuetsm}[4]{\abopcc{usyntaxse}{#2}{#1}{#3.#4}}
\newcommand{\utsmap}[2]{#1~\texttt{\lq(}#2\texttt{)\lq}}
\newcommand{\autsmap}[2]{\texttt{uapsetlm}[#1]\texttt{[}#2\texttt{]}}
\newcommand{\uet}[1]{\ue_\text{#1}}
\newcommand{\ueparse}{\uet{parse}}

% TLM expressions
\newcommand{\tsmv}{\hat{a}}
\newcommand{\utsmdef}[2]{\texttt{syntax}~@~#1~\texttt{\{}#2\texttt{\}}}
\newcommand{\istsmU}[2]{#1 \vdash #2~\mathsf{tlm}}

\newcommand{\uGamma}{\hat{\Gamma}}
\newcommand{\uDelta}{\hat{\Delta}}
\newcommand{\uD}{\mathcal{D}}
\newcommand{\uG}{\mathcal{G}}
\newcommand{\uDD}[2]{\langle #1; #2 \rangle}
\newcommand{\uGG}[2]{\langle #1; #2 \rangle}
\newcommand{\uGammaOK}[1]{\Delta \vdash #1~\mathsf{uctx}}
\newcommand{\uDeltaOK}[1]{#1 \vdash \mathsf{ok}}
\newcommand{\vExpands}[2]{#1 \leadsto #2}
\newcommand{\ctxUpdate}[3]{#1 \uplus \vExpands{#2}{#3}}
\newcommand{\uDhyp}[2]{#1 \leadsto #2~\mathsf{type}}
\newcommand{\uGhyp}[3]{#1 \leadsto #2 : #3}
\newcommand{\uGcons}[2]{#1 \uplus #2}

\newcommand{\uPsi}{\hat{\Psi}}
\newcommand{\uA}{\mathcal{A}}
\newcommand{\uAS}[2]{\langle #1; #2 \rangle}
\newcommand{\uShyp}[4]{#1 \leadsto #2 \hookrightarrow \xuetsmdef{#3}{#4}}
\newcommand{\pShyp}[4]{#1 \leadsto #2 \hookrightarrow \petsmdefn{#3}{#4}}
\newcommand{\uPhyp}[4]{#1 \leadsto #2 \hookrightarrow \xuptsmdef{#3}{#4}}
\newcommand{\pPhyp}[4]{#1 \leadsto #2 \hookrightarrow \pptsmdefn{#3}{#4}}

\newcommand{\tBody}{\mathtt{Body}}
\newcommand{\tParseResultExp}{\mathtt{ParseResultE}} 
\newcommand{\tParseResultF}{\mathtt{ParseResult}}
\newcommand{\tParseResult}[1]{\mathtt{ParseResult}(#1)}
\newcommand{\tCEExp}{\mathtt{PrExpr}} % Typed expansion
\newcommand{\expandsU}[6]{#1~#2 \vdash_{#3} #4 \leadsto #5 : #6} % there's a multiline one in the document done manually
\newcommand{\expandsUX}[3]{\expandsU{\uDelta}{\uGamma}{\uPsi}{#1}{#2}{#3}}
\newcommand{\expandsUC}[3]{\vdash #1 \leadsto #2 : #3}
\newcommand{\expandsP}[6]{#1 \vdash_{#2; #3} #4 \leadsto #5 : #6}
\newcommand{\expandsPX}[3]{\expandsP{\uOmega}{\uPsi}{\uPhi}{#1}{#2}{#3}}
\newcommand{\expandsTU}[3]{#1 \vdash #2 \leadsto #3~\mathsf{type}}
\newcommand{\domof}[1]{\text{dom}(#1)}
\newcommand{\xuetsmdef}[2]{\abop{setlm}{#1;\,#2}}
\newcommand{\xuetsmbnd}[3]{#1 \hookrightarrow \xuetsmdef{#2}{#3}}
\newcommand{\uetsmenv}[2]{#1 \vdash #2~\mathsf{seTLMs}}
\newcommand{\petsmenv}[2]{#1 \vdash #2~\mathsf{peTLMs}}
\newcommand{\pptsmenv}[2]{#1 \vdash #2~\mathsf{ppTLMs}}
\newcommand{\uetsmctx}[2]{#1 \vdash #2~\mathsf{seTLMctx}}
\newcommand{\petsmctx}[2]{#1 \vdash #2~\mathsf{peTLMctx}}
\newcommand{\pptsmctx}[2]{#1 \vdash #2~\mathsf{ppTLMctx}}
\newcommand{\encodeBody}[2]{#1 \downarrow_\mathsf{Body} #2}
\newcommand{\decodeBody}[2]{#1 \uparrow_\mathsf{Body} #2}
\newcommand{\ebody}{\etxt{body}}
\newcommand{\eparse}{\etxt{parse}}
\newcommand{\ecand}{\etxt{proto}}
\newcommand{\decodeCondE}[2]{#1 \uparrow_\mathsf{PrExpr} #2}
\newcommand{\encodeCondE}[2]{#1 \downarrow_\mathsf{PrExpr} #2}

% Candidate Expansions
\newcommand{\ce}{\grave{e}}
\newcommand{\ctau}{\grave{\tau}}

\newcommand{\splicedt}[2]{\texttt{splicedt}[#1; #2]}
\newcommand{\asplicedt}[2]{\texttt{splicedt}[#1; #2]}
\newcommand{\acesplicedt}[2]{\texttt{splicedt}[#1; #2]}
\newcommand{\splicede}[3]{\texttt{splicede}[#1; #2; #3]}
\newcommand{\splicedet}[3]{\texttt{splicede}[#1; #2; #3]}
\newcommand{\asplicede}[2]{\texttt{splicede}[#1; #2]}
\newcommand{\acesplicede}[3]{\texttt{splicede}[#1; #2; #3]}
\newcommand{\acesplicedet}[3]{\texttt{anasplicede}[#1; #2]\texttt{\{}#3\texttt{\}}}
\newcommand{\splicedp}[3]{\texttt{splicedp}[#1; #2; #3]}
\newcommand{\acesplicedp}[3]{\texttt{splicedp}[#1; #2; #3]}
\newcommand{\splicedc}[3]{\texttt{splicedc}[#1; #2; #3]}
\newcommand{\acesplicedc}[3]{\texttt{splicedc}[#1; #2; #3]}
\newcommand{\splicedk}[2]{\texttt{splicedk}[#1; #2]}
\newcommand{\acesplicedk}[2]{\texttt{splicedk}[#1; #2]}

\newcommand{\mtau}{\dot{\tau}}
\newcommand{\mtspliced}[1]{\texttt{spliced}(#1)}

% Candidate expansion validation
\newcommand{\cvalidT}[4]{#1\vdash^{#2} #3 \leadsto #4~\mathsf{type}}
\newcommand{\cvalidE}[6]{#1~#2\vdash^{#3} #4 \leadsto #5 : #6}
\newcommand{\cvalidEP}[5]{#1 \vdash^{#2} #3 \leadsto #4 : #5}
\newcommand{\cvalidEPX}[3]{\cvalidEP{\Omega}{\escenev}{#1}{#2}{#3}}
\newcommand{\csyn}[6]{#1~#2 \vdash^{#3} #4 \leadsto #5 \Rightarrow #6}
\newcommand{\csynX}[3]{\csyn{\Delta}{\Gamma}{\escenev}{#1}{#2}{#3}}
\newcommand{\cana}[6]{#1~#2 \vdash^{#3} #4 \leadsto #5 \Leftarrow #6}
\newcommand{\canaX}[3]{\cana{\Delta}{\Gamma}{\escenev}{#1}{#2}{#3}}
\newcommand{\cvalidEX}[3]{\cvalidE{\Delta}{\Gamma}{\escenev}{#1}{#2}{#3}}
\newcommand{\escenev}{\mathbbmss{E}}
\newcommand{\tscenev}{\mathbbmss{T}}
\newcommand{\esceneU}[4]{#1;\,#2;\,#3;\,#4}
\newcommand{\esceneUP}[5]{#1;\,#2;\,#3;\,#4;\,#5}
\newcommand{\esceneSG}[5]{#1;\,#2;\,#3{\setlength{\fboxsep}{0pt}\colorbox{light-gray}{\fontsize{9pt}{0pt}\selectfont ;\,$#4$}};\,#5}
\newcommand{\esceneSGB}[5]{#1;\,#2;\,#3{\setlength{\fboxsep}{0pt}\colorbox{light-gray}{;\,$#4$}};\,#5}
\newcommand{\tsceneU}[2]{#1;\,#2}
\newcommand{\tsceneUP}[2]{\tsceneU{#1}{#2}}
\newcommand{\tsfrom}[1]{\mathsf{ts}(#1)}
\newcommand{\parseUTypF}[1]{\mathsf{parseUTyp}(#1)}
\newcommand{\parseUExpF}[1]{\mathsf{parseUExp}(#1)}
\newcommand{\parseUTyp}[2]{\mathsf{parseUTyp}(#1)=#2}
\newcommand{\parseUExp}[2]{\mathsf{parseUExp}(#1)=#2}
\newcommand{\bsubseq}[3]{\mathsf{subseq}(#1; #2; #3)}

\newcommand{\sizeof}[1]{\Vert #1 \Vert}

% Pattern matching
\newcommand{\matchwith}[2]{\texttt{match}~#1~#2}
% \newcommand{\aematchwith}[3]{\abopi{match}{#1}{#2; #3}}
\newcommand{\aematchwith}[3]{\abop{match}{#2; #3}}
\newcommand{\aumatchwith}[4]{\abopic{umatch}{#1}{#2}{#3; #4}}
% \newcommand{\acematchwith}[3]{\abopi{prmatch}{#1}{#2; #3}}
\newcommand{\acematchwith}[3]{\abop{prmatch}{#2; #3}}


\newcommand{\aumatchwithb}[3]{\abopi{umatch}{#1}{#2; #3}}
\newcommand{\acematchwithb}[3]{\abopi{prmatch}{#1}{#2; #3}}

\newcommand{\matchrule}[2]{#1 \Rightarrow #2}
\newcommand{\aematchrule}[2]{\abop{rule}{#1.#2}}
\newcommand{\aumatchrule}[2]{\abop{urule}{#1.#2}}
\newcommand{\acematchrule}[2]{\abop{prrule}{#1.#2}}

\newcommand{\dashVx}{\text{\reflectbox{$\Vdash$}}}
\newcommand{\dashV}{~\dashVx~}
\newcommand{\ruleType}[5]{#1~#2 \vdash #3 : #4 \Mapsto #5}
\newcommand{\patType}[3]{#2 : #3 \dashV #1}
\newcommand{\patTypeD}[4]{#1 \vdash #3 : #4 \dashV #2}
\newcommand{\ruleTypeP}[4]{#1 \vdash #2 : #3 \Mapsto #4}
\newcommand{\patTypeP}[3]{\Omega \vdash #2 : #3 \dashV #1}
\newcommand{\patTypePC}[4]{#1 \vdash #3 : #4 \dashV #2}
% \newcommand{\pctx}{\varUpsilon}
\newcommand{\pctx}{\Gamma}

\newcommand{\matchfail}[1]{#1~\mathsf{matchfail}}

\newcommand{\eR}{R}
\newcommand{\uR}{\hat{R}}
\newcommand{\cR}{\grave{R}}

\newcommand{\erv}{r}
\newcommand{\urv}{\hat{r}}
\newcommand{\crv}{\grave{r}}

\newcommand{\epv}{p}
\newcommand{\upv}{\hat{p}}
\newcommand{\cpv}{\grave{p}}

\newcommand{\wildp}{\_}
\newcommand{\aewildp}{\texttt{wildp}}
\newcommand{\auwildp}{\texttt{uwildp}}
\newcommand{\acewildp}{\texttt{prwildp}}

\newcommand{\foldtp}[2]{\texttt{fold}_{#1}(#2)}
\newcommand{\foldp}[1]{\abop{fold}{#1}}
\newcommand{\aefoldp}[1]{\abop{foldp}{#1}}
\newcommand{\aufoldp}[1]{\abop{ufoldp}{#1}}
\newcommand{\acefoldp}[1]{\abop{prfoldp}{#1}}

\newcommand{\tplpt}[2]{\langle #1 \rangle_{#2}}
\newcommand{\tplp}[1]{\langle #1 \rangle}
% \newcommand{\aetplp}[2]{\abopi{tplp}{#1}{#2}}
\newcommand{\aetplp}[2]{\abop{tplp}{#2}}
\newcommand{\autplp}[2]{\abopi{utplp}{#1}{#2}}
\newcommand{\acetplp}[2]{\abopi{prtplp}{#1}{#2}}

\newcommand{\injtp}[3]{\texttt{inj}_{#1}~#2 \cdot #3}
\newcommand{\injp}[2]{\texttt{inj}[#1](#2)}
\newcommand{\aeinjp}[2]{\abopi{injp}{#1}{#2}}
\newcommand{\auinjp}[2]{\abopi{uinjp}{#1}{#2}}
\newcommand{\aceinjp}[2]{\abopi{prinjp}{#1}{#2}}

\newcommand{\usyntaxueP}[4]{\texttt{syntax}~#1~\texttt{at}~#2~\texttt{for expressions}~\{#3\}~\texttt{in}~#4}
\newcommand{\usyntaxup}[4]{\texttt{notation}~#1~\texttt{at}~#2~\{~\texttt{pat}~\texttt{parser}~#3~\}~\texttt{in}~#4}
% \newcommand{\usyntaxup}[4]{\texttt{syntax}~#1~\texttt{at}~#2~\texttt{for}~\texttt{patterns}~\texttt{by}~\texttt{static}~#3~\texttt{in}~#4}
\newcommand{\audefuptsm}[4]{\abopcc{usyntaxsp}{#2}{#1}{#3.#4}}
\newcommand{\auapuptsm}[2]{\texttt{uapsptlm}[#1]\texttt{[}#2\texttt{]}}

\newcommand{\expandsUP}[7]{#1~#2 \vdash_{#3;\,#4} #5 \leadsto #6 : #7} % there's a multiline one in the document done manually
\newcommand{\expandsUPX}[3]{\expandsUP{\uDelta}{\uGamma}{\uPsi}{\uPhi}{#1}{#2}{#3}}

\newcommand{\expandsSG}[7]{#1~#2~{\vdash_{#3}}{\setlength{\fboxsep}{0px}\colorbox{light-gray}{$_{\mathstrut; #4}$}}~ #5 \leadsto #6 : #7}

\newcommand{\ruleExpands}[8]{#1~#2 \vdash_{#3;\,#4} #5 \leadsto #6 : #7 \Mapsto #8}
\newcommand{\patExpands}[5]{\uDelta \vdash_{#2} #3 \leadsto #4 : #5 \dashV #1}
\newcommand{\xuptsmdef}[2]{\abop{sptlm}{#1;\,#2}}
\newcommand{\xuptsmbnd}[3]{#1 \hookrightarrow \xuptsmdef{#2}{#3}}
\newcommand{\uptsmenv}[2]{#1 \vdash #2~\mathsf{spTLMs}}
\newcommand{\uptsmctx}[2]{#1 \vdash #2~\mathsf{spTLMctx}}

\newcommand{\tParseResultPat}{\mathtt{ParseResultP}} 
\newcommand{\tCEPat}{\mathtt{PrPat}} % Typed expansion
\newcommand{\tPCEPat}{\mathtt{PPrPat}}

\newcommand{\decodeCEPat}[2]{#1 \uparrow_\mathsf{PrPat} #2}
\newcommand{\encodeCEPat}[2]{#1 \downarrow_\mathsf{PrPat} #2}
\newcommand{\decodePCEPat}[2]{#1 \uparrow_\mathsf{PPrPat} #2}
\newcommand{\encodePCEPat}[2]{#1 \downarrow_\mathsf{PPrPat} #2}


\newcommand{\cvalidP}[5]{#3 \leadsto #4 : #5~\dashVx^{#2}\,{#1}}
\newcommand{\cvalidR}[7]{#1~#2 \vdash^{#3} #4 \leadsto #5 : #6 \Mapsto #7}
\newcommand{\cvalidRP}[6]{#1 \vdash^{#2} #3 \leadsto #4 : #5 \Mapsto #6}
\newcommand{\crsyn}[7]{#1~#2 \vdash^{#3} #4 \leadsto #5 \Rightarrow #6 \Mapsto #7}
\newcommand{\crsynX}[4]{\crsyn{\Delta}{\Gamma}{\escenev}{#1}{#2}{#3}{#4}}
\newcommand{\crana}[7]{#1~#2 \vdash^{#3} #4 \leadsto #5 \Leftarrow #6 \Mapsto #7}
\newcommand{\cranaX}[4]{\crana{\Delta}{\Gamma}{\escenev}{#1}{#2}{#3}{#4}}
\newcommand{\pscenev}{\mathbbmss{P}}
\newcommand{\pscene}[3]{#1;\,#2;\,#3}

\newcommand{\parseURule}[2]{\mathsf{parseURule}(#1)=#2}
\newcommand{\parseUPat}[2]{\mathsf{parseUPat}(#1)=#2}
\newcommand{\parseUPatF}[1]{\mathsf{parseUPat}(#1)}


\newcommand{\uPhi}{\hat\Phi}
\newcommand{\uAP}[2]{\langle #1; #2 \rangle}
\newcommand{\upctx}{\hat\Gamma}

% Implicits
\newcommand{\implicite}[2]{\texttt{implicit\,syntax}~#1~\texttt{for\,expressions\,in}~#2}
\newcommand{\auimplicite}[2]{\abopp{uimplicite}{#1}{#2}}

\newcommand{\implicitp}[2]{\texttt{implicit\,syntax}~#1~\texttt{for\,patterns\,in}~#2}
\newcommand{\auimplicitp}[2]{\abopp{uimplicitp}{#1}{#2}}

\newcommand{\lit}[1]{{\texttt{/}#1\texttt{/}}}
\newcommand{\auelit}[1]{\abopbz{uelit}{#1}}
\newcommand{\auplit}[1]{\abopbz{uplit}{#1}}

\newcommand{\eana}[7]{#1~#2 \vdash_{#3; #4} #5 \leadsto #6 \Leftarrow #7}
\newcommand{\eanaX}[3]{\eana{\uDelta}{\uGamma}{\uPsi}{\uPhi}{#1}{#2}{#3}}
\newcommand{\esyn}[7]{#1~#2 \vdash_{#3; #4} #5 \leadsto #6 \Rightarrow #7}
\newcommand{\esynX}[3]{\esyn{\uDelta}{\uGamma}{\uPsi}{\uPhi}{#1}{#2}{#3}}
\newcommand{\rana}[8]{#1~#2 \vdash_{#3; #4} #5 \leadsto #6 \Leftarrow #7 \Mapsto #8}
\newcommand{\ranaX}[4]{\rana{\uDelta}{\uGamma}{\uPsi}{\uPhi}{#1}{#2}{#3}{#4}}
\newcommand{\rsyn}[8]{#1~#2 \vdash_{#3; #4} #5 \leadsto #6 \Rightarrow #7 \Mapsto #8}
\newcommand{\rsynX}[4]{\rsyn{\uDelta}{\uGamma}{\uPsi}{\uPhi}{#1}{#2}{#3}{#4}}

\newcommand{\uASI}[3]{\langle #1; #2; #3 \rangle}
\newcommand{\uI}{\mathcal{I}}
\newcommand{\designate}[2]{#1 \hookrightarrow #2}
\newcommand{\uIOK}[2]{#1 \vdash #2~\mathsf{designations}}


% Kinds
\newcommand{\kty}{\texttt{T}}
\newcommand{\akty}{\texttt{Type}}
\newcommand{\aukty}{\texttt{uType}}
\newcommand{\acekty}{\texttt{prType}}

\newcommand{\ksing}[1]{[\texttt{=}#1]}
\newcommand{\aksing}[1]{\abop{S}{#1}}
\newcommand{\auksing}[1]{\abop{uS}{#1}}
\newcommand{\aceksing}[1]{\abop{prS}{#1}}

\newcommand{\kdarr}[3]{(#1 :: #2) \rightarrow #3}
\newcommand{\akdarr}[3]{\abop{darr}{#1; #2.#3}}
\newcommand{\aukdarr}[3]{\abop{udarr}{#1; #2.#3}}
\newcommand{\acekdarr}[3]{\abop{prdarr}{#1; #2.#3}}

\newcommand{\kunit}{\llangle\rrangle}
\newcommand{\akunit}{\texttt{unit}}
\newcommand{\aukunit}{\texttt{uunit}}
\newcommand{\acekunit}{\texttt{prunit}}
\newcommand{\kone}{\mathsf{1}}

\newcommand{\dpitem}[3]{\mapitem{#1 \triangleright #2}{#3}}
\newcommand{\kdprod}[1]{\llangle #1 \rrangle}
\newcommand{\kdprodstd}{\kdprod{\{\dpitem{\ell_i}{u_i}{\kappa_i}\}_{1 \leq i \leq n}}}
\newcommand{\akdprod}[3]{\abopi{dprod}{#1; #2}{#3}}
\newcommand{\akdprodstd}{\akdprod{n}{\seqschemaX{\ell}}{\seqschemaijb{u}{\kappa}{i}{1}{n}{j}{1}{i}}}

\newcommand{\kdbprod}[3]{(#1 :: #2) \times #3}
\newcommand{\akdbprod}[3]{\abop{dprod}{#1; #2.#3}}
\newcommand{\aukdbprod}[3]{\abop{udprod}{#1; #2.#3}}
\newcommand{\acekdbprod}[3]{\abop{prdprod}{#1; #2.#3}}

\newcommand{\iskind}[2]{#1 \vdash #2~\mathsf{kind}}
\newcommand{\iskindX}[1]{\iskind{\Omega}{#1}}

\newcommand{\kequal}[3]{#1 \vdash #2 \equiv #3}
\newcommand{\kequalX}[2]{\kequal{\Omega}{#1}{#2}}

\newcommand{\ksub}[3]{#1 \vdash #2 <:: #3}
\newcommand{\ksubX}[2]{\ksub{\Omega}{#1}{#2}}

% Constructors
\newcommand{\casc}[2]{#1 :: #2}
\newcommand{\aucasc}[2]{\abopc{uasc}{#1}{#2}}
\newcommand{\acecasc}[2]{\abopc{prasc}{#1}{#2}}

\newcommand{\cabs}[2]{\lambda #1.#2}
\newcommand{\acabs}[2]{\abop{abs}{#1.#2}}
\newcommand{\aucabs}[2]{\abop{uabs}{#1.#2}}
\newcommand{\acecabs}[2]{\abop{prabs}{#1.#2}}

\newcommand{\capp}[2]{#1(#2)}
\newcommand{\acapp}[2]{\abop{app}{#1; #2}}
\newcommand{\aucapp}[2]{\abop{uapp}{#1; #2}}
\newcommand{\acecapp}[2]{\abop{prapp}{#1; #2}}

\newcommand{\ctriv}{\llangle\rrangle}
\newcommand{\actriv}{\texttt{triv}}
\newcommand{\auctriv}{\texttt{utriv}}
\newcommand{\acectriv}{\texttt{prtriv}}

\newcommand{\cpair}[2]{\llangle #1, #2 \rrangle}
\newcommand{\acpair}[2]{\abop{pair}{#1; #2}}
\newcommand{\aucpair}[2]{\abop{upair}{#1; #2}}
\newcommand{\acecpair}[2]{\abop{prpair}{#1; #2}}

\newcommand{\dtpl}[1]{\llangle #1 \rrangle}
\newcommand{\dtplX}{\dtpl{\{\dpitem{\ell_i}{u_i}{c_i}\}_{1 \leq i \leq n}}}
\newcommand{\adtpl}[3]{\abopi{dtpl}{#1; #2}{#3}}
\newcommand{\adtplX}{\adtpl{n}{\seqschemaX{\ell}}{\seqschemaijb{u}{c}{i}{1}{n}{j}{1}{i}}}

\newcommand{\cprl}[1]{#1 \cdot \texttt{l}}
\newcommand{\acprl}[1]{\abop{prl}{#1}}
\newcommand{\aucprl}[1]{\abop{uprl}{#1}}
\newcommand{\acecprl}[1]{\abop{prprl}{#1}}

\newcommand{\adprj}[2]{\abopp{prj}{#1}{#2}}

\newcommand{\cprr}[1]{#1 \cdot \texttt{r}}
\newcommand{\acprr}[1]{\abop{prr}{#1}}
\newcommand{\aucprr}[1]{\abop{uprr}{#1}}
\newcommand{\acecprr}[1]{\abop{prprr}{#1}}

\newcommand{\mcon}[1]{#1\cdot\texttt{c}}
\newcommand{\amcon}[1]{\abop{con}{#1}}
\newcommand{\aumcon}[1]{\abop{ucon}{#1}}
\newcommand{\acemcon}[1]{\abop{prcon}{#1}}

\newcommand{\mval}[1]{#1 \cdot \texttt{v}}
\newcommand{\amval}[1]{\abop{val}{#1}}
\newcommand{\aumval}[1]{\abop{uval}{#1}}
\newcommand{\acemval}[1]{\abop{prval}{#1}}

\newcommand{\haskind}[3]{#1 \vdash #2 :: #3}
\newcommand{\haskindX}[2]{\haskind{\Omega}{#1}{#2}}

\newcommand{\cequal}[4]{#1 \vdash #2 \equiv #3 :: #4}
\newcommand{\cequalX}[3]{\cequal{\Omega}{#1}{#2}{#3}}

% Modules and signatures
\newcommand{\signature}[3]{\llbracket #1 :: #2; #3 \rrbracket}
\newcommand{\asignature}[3]{\abopc{sig}{#1}{#2.#3}}
\newcommand{\ausignature}[3]{\abopc{usig}{#1}{#2.#3}}

\newcommand{\struct}[2]{\llbracket #1; #2 \rrbracket}
\newcommand{\astruct}[2]{\abop{struct}{#1; #2}}
\newcommand{\austruct}[2]{\abop{ustruct}{#1; #2}}

\newcommand{\seal}[2]{#1 \upharpoonleft #2}
\newcommand{\aseal}[2]{\abopc{seal}{#1}{#2}}
\newcommand{\auseal}[2]{\abopc{useal}{#1}{#2}}

\newcommand{\mletdecl}[4]{\texttt{let}~#1~\texttt{=}~#2~\texttt{in}~#3}
\newcommand{\umletdecl}[5]{#1~\texttt{let}~#2~\texttt{=}~#3~\texttt{in}~#4}
\newcommand{\mlet}[4]{(\texttt{let}~#1~\texttt{=}~#2~\texttt{in}~#3) : #4}
\newcommand{\staticphase}{\texttt{static}}
\newcommand{\standardphase}{\texttt{standard}}
\newcommand{\mletH}[5]{(#1~\texttt{let}~#2~\texttt{=}~#3~\texttt{in}~#4) : #5}
\newcommand{\amlet}[4]{\abopc{mlet}{#1}{#2; #3.#4}}
\newcommand{\aumlet}[4]{\abopc{umlet}{#1}{#2; #3.#4}}

\newcommand{\syntaxps}[6]{#1~\texttt{syntax}~#2~\texttt{at}~#3~\texttt{for}~#4~\texttt{by}~\texttt{static}~#5~\texttt{in}~#6}
\newcommand{\letsyntaxps}[5]{#1~\texttt{let}~\texttt{syntax}~#2~\texttt{for}~#3=#4~\texttt{in}~#5}

\newcommand{\tsmapp}[2]{\texttt{/}#1\mathtt{/}~#2}

\newcommand{\bindMod}[2]{#1.#2}
\newcommand{\bindTyp}[2]{#1.#2}

\newcommand{\issig}[2]{#1 \vdash #2~\mathsf{sig}}
\newcommand{\issigX}[1]{\issig{\Omega}{#1}}

\newcommand{\sigequal}[3]{#1 \vdash #2 \equiv #3}
\newcommand{\sigequalX}[2]{\sigequal{\Omega}{#1}{#2}}

\newcommand{\sigsub}[3]{#1 \vdash #2 <: #3}
\newcommand{\sigsubX}[2]{\sigsub{\Omega}{#1}{#2}}

\newcommand{\hassig}[3]{#1 \vdash #2 : #3}
\newcommand{\hassigX}[2]{\hassig{\Omega}{#1}{#2}}

\newcommand{\raiseerror}{\mathtt{error}}
\newcommand{\isvalue}[1]{#1~\mathsf{val}}
\newcommand{\iserr}[1]{#1~\mathsf{err}}
\newcommand{\ismval}[2]{#1 \vdash #2~\mathsf{mval}}
\newcommand{\ismvalX}[1]{\ismval{\Omega}{#1}}

\newcommand{\statphase}{\mathtt{static}}
\newcommand{\dynphase}{\mathtt{dynamic}}

\newcommand{\expsortq}{\mathtt{Exp}}
\newcommand{\patsortq}{\mathtt{Pat}}

% Parameterized TLMs
\newcommand{\urho}{{\hat\rho}}
\newcommand{\uepsilon}{{\hat\epsilon}}
\newcommand{\uX}{{\hat X}}

\newcommand{\defpetsm}[4]{\texttt{syntax}~#1~\texttt{at}~#2~\texttt{for}~\texttt{expressions}~\texttt{by}~\texttt{static}~#3~\texttt{in}~#4}
\newcommand{\defpptsm}[4]{\texttt{syntax}~#1~\texttt{at}~#2~\texttt{for}~\texttt{patterns}~\texttt{by}~\texttt{static}~#3~\texttt{in}~#4}

\newcommand{\defpetsmH}[5]{#1~\texttt{syntax}~#2~\texttt{at}~#3~\texttt{for}~\texttt{expressions}~\texttt{by}~\texttt{static}~#4~\texttt{in}~#5}
\newcommand{\defpptsmH}[5]{#1~\texttt{syntax}~#2~\texttt{at}~#3~\texttt{for}~\texttt{patterns}~\texttt{by}~\texttt{static}~#4~\texttt{in}~#5}

\newcommand{\audefpetsm}[4]{\abopcc{usyntaxpe}{#2}{#1}{#3.#4}}
\newcommand{\aumdefpetsm}[4]{\abopcc{umsyntaxpe}{#2}{#1}{#3.#4}}
\newcommand{\auappetsm}[2]{\abopicz{uappetlm}{#1}{#2}}
\newcommand{\uletpetsm}[3]{\texttt{let}~\texttt{syntax}~#1=#2~\texttt{for}~\texttt{expressions}~\texttt{in}~#3}
\newcommand{\auletpetsm}[3]{\abopc{uletpetlm}{#1}{#2.#3}}
\newcommand{\aumletpetsm}[3]{\abopc{umletpetlm}{#1}{#2.#3}}

\newcommand{\uletpetsmH}[4]{#1~\texttt{let}~\texttt{syntax}~#2=#3~\texttt{for}~\texttt{expressions}~\texttt{in}~#4}

\newcommand{\audefpptsm}[4]{\abopcc{usyntaxpp}{#2}{#1}{#3.#4}}
\newcommand{\aumdefpptsm}[4]{\abopcc{umsyntaxpp}{#2}{#1}{#3.#4}}
\newcommand{\auappptsm}[2]{\abopicz{uappptlm}{#1}{#2}}
\newcommand{\uletpptsm}[3]{\texttt{let}~\texttt{syntax}~#1=#2~\texttt{for}~\texttt{patterns}~\texttt{in}~#3}

\newcommand{\uletpptsmH}[4]{#1~\texttt{let}~\texttt{syntax}~#2=#3~\texttt{for}~\texttt{patterns}~\texttt{in}~#4}

\newcommand{\auletpptsm}[3]{\abopc{uletpptlm}{#1}{#2.#3}}
\newcommand{\aumletpptsm}[3]{\abopc{umletpptlm}{#1}{#2.#3}}

\newcommand{\aetype}[1]{\abop{type}{#1}}
\newcommand{\autype}[1]{\abop{utype}{#1}}

\newcommand{\alltypes}[2]{\forall #1.#2}
\newcommand{\aealltypes}[2]{\abop{alltypes}{#1.#2}}
\newcommand{\aualltypes}[2]{\abop{ualltypes}{#1.#2}}

\newcommand{\allmods}[3]{\forall #1{:}#2.#3}
\newcommand{\aeallmods}[3]{\abopc{allmods}{#1}{#2.#3}}
\newcommand{\auallmods}[3]{\abopc{uallmods}{#1}{#2.#3}}

% \newcommand{\typeparam}[2]{\Lambda #1.#2}
% \newcommand{\aetypeparam}[2]{\abop{typeparam}{#1.#2}}
% \newcommand{\autypeparam}[2]{\abop{utypeparam}{#1.#2}}

% \newcommand{\modparam}[3]{\Lambda #1{:}#2.#3}
% \newcommand{\aemodparam}[3]{\abopc{modparam}{#1}{#2.#3}}
% \newcommand{\aumodparam}[3]{\abopc{umodparam}{#1}{#2.#3}}

\newcommand{\aptype}[2]{#1(#2)}
\newcommand{\aeaptype}[2]{\abopc{aptype}{#1}{#2}}
\newcommand{\auaptype}[2]{\abopc{uaptype}{#1}{#2}}

\newcommand{\apmod}[2]{#1(#2)}
\newcommand{\aeapmod}[2]{\abopc{apmod}{#1}{#2}}
\newcommand{\auapmod}[2]{\abopc{uapmod}{#1}{#2}}

\newcommand{\adefref}[1]{\aboppz{defref}{#1}}
\newcommand{\abindref}[1]{\aboppz{bindref}{#1}}

\newcommand{\uOmega}{{\hat\Omega}}

\newcommand{\kExpandsSP}[3]{#1 \vdash #2 \leadsto #3~\mathsf{kind}}
\newcommand{\kExpands}[3]{#1 \vdash #2 \leadsto #3~\mathsf{kind}}
\newcommand{\kExpandsX}[2]{\kExpands{\uOmega}{#1}{#2}}

\newcommand{\cExpandsSP}[4]{#1 \vdash #2 \leadsto #3 :: #4}

\newcommand{\kana}[4]{#1 \vdash #2 \leadsto #3 \Leftarrow #4}
\newcommand{\kanaX}[3]{\kana{\uOmega}{#1}{#2}{#3}}
\newcommand{\ksyn}[4]{#1 \vdash #2 \leadsto #3 \Rightarrow #4}
\newcommand{\ksynX}[3]{\ksyn{\uOmega}{#1}{#2}{#3}}

\newcommand{\cExpands}[4]{#1 \vdash #2 \leadsto #3 :: #4}
\newcommand{\cExpandsX}[3]{\cExpands{\uOmega}{#1}{#2}{#3}}

\newcommand{\eExpandsSP}[6]{#1 \vdash_{#2;#3} #4 \leadsto #5 : #6}

\newcommand{\eanaP}[6]{#1 \vdash_{#2; #3} #4 \leadsto #5 \Leftarrow #6}
\newcommand{\eanaPX}[3]{\eanaP{\uOmega}{\uPsi}{\uPhi}{#1}{#2}{#3}}
\newcommand{\esynP}[6]{#1 \vdash_{#2; #3} #4 \leadsto #5 \Rightarrow #6}
\newcommand{\esynPX}[3]{\esynP{\uOmega}{\uPsi}{\uPhi}{#1}{#2}{#3}}

\newcommand{\rExpandsSP}[7]{#1 \vdash_{#2; #3} #4 \leadsto #5 : #6 \Mapsto #7}
\newcommand{\ranaP}[7]{#1 \vdash_{#2; #3} #4 \leadsto #5 \Leftarrow #6 \Mapsto #7}
\newcommand{\ranaPX}[4]{\ranaP{\uOmega}{\uPsi}{\uPhi}{#1}{#2}{#3}{#4}}
\newcommand{\rsynP}[7]{#1 \vdash_{#2; #3} #4 \leadsto #5 \Rightarrow #6 \Mapsto #7}
\newcommand{\rsynPX}[4]{\rsynP{\uOmega}{\uPsi}{\uPhi}{#1}{#2}{#3}{#4}}

\newcommand{\pExpandsSP}[6]{#1 \vdash_{#2} #3 \leadsto #4 : #5 \dashV #6}
\newcommand{\patExpandsP}[5]{\uOmega \vdash_{#2} #3 \leadsto #4 : #5 \dashV #1}

\newcommand{\uKhyp}[3]{#1 \leadsto #2 :: #3}
\newcommand{\uMhyp}[3]{#1 \leadsto #2 : #3}

\newcommand{\uOmegaEx}[4]{\langle #3; #1; #2; #4 \rangle}
\newcommand{\uMctx}{\mathcal{M}}

\newcommand{\istypeP}[2]{\istypeU{#1}{#2}}
\newcommand{\istypePX}[1]{\istypeP{\Omega}{#1}}

\newcommand{\issubtypeP}[3]{#1 \vdash #2 <: #3}
\newcommand{\issubtypePX}[2]{\issubtypeP{\Omega}{#1}{#2}}

\newcommand{\tequalP}[3]{#1 \vdash #2 \equiv #3~\mathsf{type}}
\newcommand{\tequalPX}[2]{\tequalP{\Omega}{#1}{#2}}

\newcommand{\tExpandsP}[3]{#1 \vdash #2 \leadsto #3~\mathsf{type}}
\newcommand{\tExpandsPX}[2]{\tExpandsP{\uOmega}{#1}{#2}}


\newcommand{\sigExpandsSP}[3]{#1 \vdash #2 \leadsto #3~\mathsf{sig}}

\newcommand{\sigExpandsP}[3]{#1 \vdash #2 \leadsto #3~\mathsf{sig}}
\newcommand{\sigExpandsPX}[2]{\sigExpandsP{\uOmega}{#1}{#2}}

\newcommand{\uSigma}{{\hat\Sigma}}

\newcommand{\uPsiE}{\uPsi_\text{E}}
\newcommand{\uPsiP}{\uPsi_\text{P}}

\newcommand{\mExpandsSP}[7]{#1 \sslash #2 \vdash_{#3; #4} #5 \leadsto #6 : #7}

\newcommand{\msyn}[6]{#1 \vdash_{#2; #3} #4 \leadsto #5 \Rightarrow #6}
\newcommand{\msynX}[3]{\msyn{\uOmega}{\uPsi}{\uPhi}{#1}{#2}{#3}}
\newcommand{\mana}[6]{#1 \vdash_{#2; #3} #4 \leadsto #5 \Leftarrow #6}
\newcommand{\manaX}[3]{\mana{\uOmega}{\uPsi}{\uPhi}{#1}{#2}{#3}}

\newcommand{\mExpandsP}[6]{#1 \vdash_{#2; #3} #4 \leadsto #5 : #6}
\newcommand{\mExpandsPX}[3]{\mExpandsP{\uOmega}{\uPsi}{\uPhi}{#1}{#2}{#3}}

\newcommand{\mExpandsPH}[7]{#1 \vdash_{#2; #3}^{#7} #4 \leadsto #5 : #6}
\newcommand{\mExpandsPHX}[3]{\mExpandsPH{\uOmega}{\uPsi}{\uPhi}{#1}{#2}{#3}{\staticenvv}}

\newcommand{\staticenvv}{\Sigma}
\newcommand{\staticenv}[4]{#1 : #2; #3; #4}


\newcommand{\petsmdefn}[3]{#1 \hookrightarrow \texttt{petlm}(#2; #3)}
\newcommand{\pptsmdefn}[3]{#1 \hookrightarrow \texttt{pptlm}(#2; #3)}

\newcommand{\istsmty}[2]{#1 \vdash #2~\mathsf{tlmty}}
\newcommand{\tsmtyExpands}[3]{#1 \vdash #2 \leadsto #3 ~\mathsf{tlmty}}

\newcommand{\tParseResultPCEExp}{\mathtt{ParseResult}(\mathtt{PPrExpr})}
\newcommand{\tPProtoExpr}{\mathtt{PPrExpr}}

\newcommand{\tsmexpExpandsExp}[5]{#1 \vdash_{#2}^{\mathsf{Exp}} #3 \leadsto #4~@~#5}
\newcommand{\tsmexpExpandsExpX}[3]{\tsmexpExpandsExp{\Omega}{\uPsi}{#1}{#2}{#3}}

\newcommand{\tsmexpExpandsPat}[5]{#1 \vdash_{#2}^{\mathsf{Pat}} #3 \leadsto #4~@~#5}
\newcommand{\tsmexpExpandsPatX}[3]{\tsmexpExpandsPat{\Omega}{\uPsi}{#1}{#2}{#3}}

\newcommand{\hastsmtypeExp}[4]{#1 \vdash_{#2}^{\mathsf{Exp}} #3 ~@~ #4}
\newcommand{\hastsmtypePat}[4]{#1 \vdash_{#2}^{\mathsf{Pat}} #3 ~@~ #4}

\newcommand{\tsmexpNormalExp}[3]{#1 \vdash_{#2}^\mathsf{Exp} #3~\mathsf{normal}}
\newcommand{\tsmexpStepsExp}[4]{#1 \vdash_{#2}^\mathsf{Exp} #3 \mapsto #4}
\newcommand{\tsmexpMultistepsExp}[4]{#1 \vdash_{#2}^\mathsf{Exp} #3 \mapsto^{*} #4}
\newcommand{\tsmexpEvalsExp}[4]{#1 \vdash_{#2}^\mathsf{Exp} #3 \Downarrow #4}

\newcommand{\tsmexpNormalPat}[3]{#1 \vdash_{#2}^\mathsf{Pat} #3~\mathsf{normal}}
\newcommand{\tsmexpStepsPat}[4]{#1 \vdash_{#2}^\mathsf{Pat} #3 \mapsto #4}
\newcommand{\tsmexpMultistepsPat}[4]{#1 \vdash_{#2}^\mathsf{Pat} #3 \mapsto^{*} #4}
\newcommand{\tsmexpEvalsPat}[4]{#1 \vdash_{#2}^\mathsf{Pat} #3 \Downarrow #4}


\newcommand{\tsmdefof}[1]{\mathsf{tlmdef}(#1)}

\newcommand{\decodePCEExp}[2]{#1 \uparrow_\mathsf{PPrExpr} #2}
\newcommand{\encodePCEExp}[2]{#1 \downarrow_\mathsf{PPrExpr} #2}

\newcommand{\pce}{\dot{e}}
\newcommand{\pcp}{\dot{p}}

\newcommand{\prepce}[8]{#1 \vdash_{#2}^\mathsf{Exp} #3 \looparrowright_{#5} #4 ~?~ #6 \dashv #7 : #8}
\newcommand{\prepcp}[8]{#1 \vdash_{#2}^\mathsf{Pat} #3 \looparrowright_{#5} #4 ~?~ #6 \dashv #7 : #8}

%\newcommand{\cvalidT}[4]{#1\vdash^{#2} #3 \leadsto #4~\mathsf{type}}
% \newcommand{\cvalidE}[6]{#1~#2\vdash^{#3} #4 \leadsto #5 : #6}
\newcommand{\csynP}[5]{#1 \vdash^{#2} #3 \leadsto #4 \Rightarrow #5}
\newcommand{\csynPX}[3]{\csynP{\Omega}{\escenev}{#1}{#2}{#3}}
\newcommand{\canaP}[5]{#1 \vdash^{#2} #3 \leadsto #4 \Leftarrow #5}
\newcommand{\canaPX}[3]{\canaP{\Omega}{\escenev}{#1}{#2}{#3}}
%\newcommand{\cvalidEX}[3]{\cvalidE{\Delta}{\Gamma}{\escenev}{#1}{#2}{#3}}
%\newcommand{\escenev}{\mathbbmss{E}}
%\newcommand{\tscenev}{\mathbbmss{T}}
\newcommand{\esceneP}[5]{#1;\,#2;\,#3;\,#4;\,#5}
\newcommand{\tsceneP}[2]{\tsceneU{#1}{#2}}
\newcommand{\psceneP}[4]{#1;\,#2;\,#3;\,#4}

\newcommand{\abstype}[2]{\Lambda #1.#2}
\newcommand{\aeabstype}[2]{\abop{abstype}{#1.#2}}
\newcommand{\auabstype}[2]{\abop{uabstype}{#1.#2}}

\newcommand{\absmod}[3]{\Lambda #1{:}#2.#3}
\newcommand{\aeabsmod}[3]{\abopc{absmod}{#1}{#2.#3}}
\newcommand{\auabsmod}[3]{\abopc{uabsmod}{#1}{#2.#3}}

\newcommand{\tsmapparam}[2]{#1(#2)}

\newcommand{\pcebindtype}[2]{\Lambda #1.#2}
\newcommand{\apcebindtype}[2]{\abop{prbindtype}{#1.#2}}

\newcommand{\pcebindmod}[2]{\Lambda #1.#2}
\newcommand{\apcebindmod}[2]{\abop{prbindmod}{#1.#2}}

\newcommand{\pceexp}[1]{#1}
\newcommand{\apceexp}[1]{\abop{prexp}{#1}}
\newcommand{\apcepat}[1]{\abop{prpat}{#1}}

\newcommand{\cekappa}{{\grave\kappa}}
\newcommand{\cec}{{\grave c}}

\newcommand{\tParseResultCEPat}{\mathtt{ParseResult}(\mathtt{PPrPat})}

\newcommand{\cscenev}{\mathbbmss{C}}
\newcommand{\csceneP}[3]{#1;\,#2;\,#3}
\newcommand{\OParams}{\Omega_\text{params}}

\newcommand{\cvalidK}[4]{#1 \vdash^{#2} #3 \leadsto #4~\mathsf{kind}}
\newcommand{\cvalidKX}[2]{\cvalidK{\Omega}{\cscenev}{#1}{#2}}

\newcommand{\cvalidC}[5]{#1 \vdash^{#2} #3 \leadsto #4 :: #5}
\newcommand{\cvalidCX}[3]{\cvalidC{\Omega}{\cscenev}{#1}{#2}{#3}}
\newcommand{\ccsyn}[5]{#1 \vdash^{#2} #3 \leadsto #4 \Rightarrow #5}
\newcommand{\ccsynX}[3]{\ccsyn{\Omega}{\cscenev}{#1}{#2}{#3}}
\newcommand{\ccana}[5]{#1 \vdash^{#2} #3 \leadsto #4 \Leftarrow #5}
\newcommand{\ccanaX}[3]{\ccana{\Omega}{\cscenev}{#1}{#2}{#3}}

% \newcommand{\cvalidR}[6]{#1 \vdash^{#2} #3 \leadsto #4 : #5 \Mapsto #6}
% \newcommand{\cvalidRX}[4]{\cvalidR{\Omega}{\escenev}{#1}{#2}{#3}{#4}}
\newcommand{\crsynP}[6]{#1 \vdash^{#2} #3 \leadsto #4 \Rightarrow #5 \Mapsto #6}
\newcommand{\crsynPX}[4]{\crsynP{\Omega}{\escenev}{#1}{#2}{#3}{#4}}
\newcommand{\cranaP}[6]{#1 \vdash^{#2} #3 \leadsto #4 \Leftarrow #5 \Mapsto #6}
\newcommand{\cranaPX}[4]{\cranaP{\Omega}{\escenev}{#1}{#2}{#3}{#4}}

\newcommand{\parseUMod}[2]{\mathsf{parseUMod}(#1)=#2}
\newcommand{\parseUModF}[1]{\mathsf{parseUMod}(#1)}
\newcommand{\parseUSigF}[1]{\mathsf{parseUSig}(#1)}
\newcommand{\parseUSig}[2]{\mathsf{parseUSig}(#1)=#2}
\newcommand{\parseUKind}[2]{\mathsf{parseUKind}(#1)=#2}
\newcommand{\parseUKindF}[1]{\mathsf{parseUKind}(#1)}
\newcommand{\parseUConF}[1]{\mathsf{parseUCon}(#1)}
\newcommand{\parseUCon}[2]{\mathsf{parseUCon}(#1)=#2}

\newcommand{\cvalidTP}[4]{#1 \vdash^{#2} #3 \leadsto #4~\mathsf{type}}

\newcommand{\csfrom}[1]{\mathsf{cs}(#1)}

\newcommand{\cvalidPPE}[5]{#3 \leadsto #4 : #5~\dashVx^{#2}{\,#1}}
\newcommand{\cvalidPP}[5]{#3 \leadsto #4 : #5~\dashVx^{#2}{\,#1}}

\newcommand{\isvalP}[1]{#1~\mathsf{val}}

\newcommand{\suc}[1]{\abop{s}{#1}}
\newcommand{\asuc}[1]{\abop{succ}{#1}}

\newcommand{\zero}{\texttt{z}}
\newcommand{\azero}{\texttt{zero}}

\newcommand{\defeq}{\mathrel{\overset{\makebox[0pt]{\mbox{\normalfont\scriptsize def}}}{=}}}

\newcommand{\graybox}[1]{\setlength{\fboxsep}{1.5pt}\colorbox{light-gray}{\vphantom{]}$#1$}}
\newcommand{\yellowbox}[1]{\setlength{\fboxsep}{1.5pt}\colorbox{Yellow}{\vphantom{]}$#1$}}

\newcommand{\graytxtbox}[1]{\setlength{\fboxsep}{1.5pt}\colorbox{light-gray}{\vphantom{]}#1}}

\newmdenv[backgroundcolor=light-gray,hidealllines=true,innerleftmargin=1pt,innerrightmargin=1pt,leftmargin=0pt,rightmargin=0pt,innertopmargin=1.5pt,innerbottommargin=1.5pt]{grayparbox}


\newcommand{\uDOK}[1]{\vdash #1~\mathsf{utctx}}

\newcommand{\segof}[1]{\mathsf{seg}(#1)}
\newcommand{\segOK}[2]{#1~\mathsf{segments}~#2}
\newcommand{\segOKP}[4]{#1 \vdash^{#2} \segOK{#3}{#4}}
\newcommand{\segExp}[2]{\langle #1; #2; \mathsf{UExp} \rangle}
\newcommand{\segTyp}[2]{\langle #1; #2; \mathsf{UTyp} \rangle}
\newcommand{\segPat}[2]{\langle #1; #2; \mathsf{UPat}\rangle}
\newcommand{\segCon}[2]{\langle #1; #2; \mathsf{UCon}\rangle}
\newcommand{\segKind}[2]{\langle #1; #2; \mathsf{UKind} \rangle}

% \newcommand{\summaryOf}[1]{\mathsf{summary}(#1)}

\newcommand{\nty}{n_\text{ty}}
\newcommand{\ntyj}{n_\text{ty,$j$}}
\newcommand{\nexp}{n_\text{exp}}
\newcommand{\nexpj}{n_\text{exp,$j$}}
\newcommand{\nrules}{n}
\newcommand{\npat}{n_\text{pat}}
\newcommand{\npatj}{n_\text{pat,$j$}}
\newcommand{\nkind}{n_\text{kind}}
\newcommand{\ncon}{n_\text{con}}

\newcommand{\fvof}[1]{\mathsf{fv}(#1)}


% allow interrupted equation numbering
% taken from http://tex.stackexchange.com/questions/101002/interrupting-and-resuming-subequations
% \makeatletter
% \def\user@resume{resume}
% \def\user@intermezzo{intermezzo}
% %
% \newcounter{previousequation}
% \newcounter{lastsubequation}
% \newcounter{savedparentequation}
% \setcounter{savedparentequation}{1}
% % 
% \renewenvironment{subequations}[1][]{%
%       \def\user@decides{#1}%
%       \setcounter{previousequation}{\value{equation}}%
%       \ifx\user@decides\user@resume 
%            \setcounter{equation}{\value{savedparentequation}}%
%       \else  
%       \ifx\user@decides\user@intermezzo
%            \refstepcounter{equation}%
%       \else
%            \setcounter{lastsubequation}{0}%
%            \refstepcounter{equation}%
%       \fi\fi
%       \protected@edef\theHparentequation{%
%           \@ifundefined {theHequation}\theequation \theHequation}%
%       \protected@edef\theparentequation{\theequation}%
%       \setcounter{parentequation}{\value{equation}}%
%       \ifx\user@decides\user@resume 
%            \setcounter{equation}{\value{lastsubequation}}%
%          \else
%            \setcounter{equation}{0}%
%       \fi
%       \def\theequation  {\theparentequation  \alph{equation}}%
%       \def\theHequation {\theHparentequation \alph{equation}}%
%       \ignorespaces
% }{%
% %  \arabic{equation};\arabic{savedparentequation};\arabic{lastsubequation}
%   \ifx\user@decides\user@resume
%        \setcounter{lastsubequation}{\value{equation}}%
%        \setcounter{equation}{\value{previousequation}}%
%   \else
%   \ifx\user@decides\user@intermezzo
%        \setcounter{equation}{\value{parentequation}}%
%   \else
%        \setcounter{lastsubequation}{\value{equation}}%
%        \setcounter{savedparentequation}{\value{parentequation}}%
%        \setcounter{equation}{\value{parentequation}}%
%   \fi\fi
% %  \arabic{equation};\arabic{savedparentequation};\arabic{lastsubequation}
%   \ignorespacesafterend
% }
% \makeatother

\begin{document} 
\frontmatter

%initialize page style, so contents come out right (see bot) -mjz
\pagestyle{empty}

\title{\textbf{Reasonably Programmable Syntax}}
\author{Cyrus Omar}
\date{\today}
\Year{2017}
\trnumber{CMU-CS-17-113}

\committee{Jonathan Aldrich, Chair\\
Robert Harper\\
Karl Crary\\
Eric Van Wyk, University of Minnesota}

\support{\scriptsize This research was sponsored by the DOE Computational Science Graduate Fellowship; the NSF Graduate Research Fellowship; the US Department of Defense National Security Agency under grant H9823014C0140; the US Air Force Research Laboratory under grant FA87501620042; the US Army Research Office under grants W911NF0910273 and W911NF1310154; the Boeing Company under grant 1101601762; and the National Science Foundation under grants DGE0750271, DGE1252522, and CCF1116907.  The views and conclusions contained in this document are those of the author and should not be interpreted as representing the official policies, either expressed or implied, of any sponsoring institution, the U.S. government or any other entity.
% Any opinions or recommendations 
%expressed in this material are those of the author and do not necessarily
 %reflect the views of the DOE, NSF, AFRL, DARPA or NSA.
 }

% \disclaimer{}
\permission{This work is licensed under the Creative Commons Attribution 4.0 International License, which can be found at \url{http://creativecommons.org/licenses/by/4.0/}.}

% copyright notice generated automatically from Year and author.
% permission added if \permission{} given.

\keywords{syntax, notation, parsing, type systems, module systems, macro systems, hygiene, pattern matching, bidirectional typechecking, implicit dispatch}

\maketitle

% \begin{dedication}
% Dedicated to the memory of Daniel Schreiber (1986 -- 2010), my friend.
% \end{dedication}

\pagestyle{plain} % for toc, was empty

\begin{abstract}
\noindent
Programming languages commonly provide ``syntactic sugar'' that decreases the syntactic cost of working with certain standard library constructs.   
%, i.e. {derived forms} that decrease the cognitive cost of idioms involving select library constructs. 
For example, Standard ML builds in syntactic sugar for constructing and pattern matching on lists. %This decreases the cognitive cost of working with lists. %Semantically, lists are defined in the SML Basis library (i.e. SML's ``standard library''.)
Third-party library providers are, justifiably, envious of this special arrangement. After all, it is not difficult to find other situations where library-specific syntactic sugar might be useful \cite{TSLs}. For example, (1) clients of a ``collections'' library might like syntactic sugar for finite sets and dictionaries; (2) clients of a ``web programming'' library might like syntactic sugar for HTML and JSON values; (3) a compiler writer might like syntactic sugar for the terms of the object language or various intermediate languages of interest; and (4) clients of a ``chemistry'' library might like syntactic sugar for chemical structures based on the SMILES standard \cite{anderson1987smiles}.

Defining a ``library-specific'' syntax dialect in each of these situations is problematic, because library clients cannot combine dialects like these in a manner that conserves syntactic determinism (i.e. syntactic conflicts can and do arise.) Moreover, it can become difficult for library clients to reason abstractly about types and binding when examining the text of a program that uses unfamiliar forms. Instead, they must reason transparently, about the underlying expansion. Typed, hygienic term-rewriting macro systems, like Scala's macro system \cite{ScalaMacros2013}, while somewhat more reasonable, offer limited syntactic control.

% In other words, there are few clear \emph{abstract reasoning principles} available to client programmers. % As such, the dialect-oriented approach is difficult to reconcile with the best practices of  ``programming in the large.''

This thesis formally introduces \emph{typed literal macros (TLMs)}, which give library providers the ability to programmatically control the parsing and expansion, at ``compile-time'', of expressions and patterns of \emph{generalized literal form}. Library clients can use any combination of TLMs in a program without needing to consider the possibility of syntactic conflicts between them,  because the context-free syntax of the language is never extended (instead, literal forms are  contextually repurposed.) Moreover, the language validates each expansion that a TLM generates in order to maintain useful abstract reasoning principles. Most notably, expansion validation maintains:
\begin{itemize}
\item a \emph{type discipline}, meaning that the client can reason about types while holding the literal expansion abstract; and 
\item a \emph{strictly hygienic binding discipline}, meaning that the client can always be sure that:
  \begin{enumerate}
    \item spliced terms, i.e. terms that appear within literal bodies, cannot capture bindings hidden within the literal expansion; and 
  \item the literal expansion does not refer to definition-site or application-site bindings directly. Instead, all interactions with bindings external to the expansion go explicitly through {spliced terms} or {parameters}.% Support for partial parameter application helps reduce the syntactic cost of this explicit parameter passing style.
  \end{enumerate}
\end{itemize}
\noindent
In short, we formally define a programming language in the ML tradition with a \emph{reasonably} programmable syntax.

%We discuss both explicit application of TLMs (with support for partial parameter application) and implicit, type-directed application of TLMs, which further reduces cognitive cost.
\end{abstract}

\begin{acknowledgments}
I owe a tremendous debt of gratitude to my advisor, Jonathan Aldrich, for being willing to take me on as a na\"ive neuroscience student interested in designing programming languages, and for guiding me patiently through many years of learning, experimentation and refinement. 
Jonathan's depth of expertise and breadth of perspective has been invaluable. Thank you.%My work seeks to apply type theory to solve usability problems, so it has been a great fit.

I would also like to thank Bob Harper and Karl Crary, who both generously served on my thesis committee and substantially influenced my approach. Through their teaching and scholarship, they taught me the type-theoretic foundations of programming languages, and more broadly, they taught me the tremendous value of precision in both formal and informal discourse on language design. These lessons were reinforced during long afternoons discussing theory papers with other POP students in the ConCert reading group,  and during long evenings grading and preparing for 15-150 (Functional Programming) and 15-312 (Principles of Programming Languages) with Dan Licata, Ian Voysey, Bob Harper, Shayak Sen and the rest of the course staff. Thank you all for being uncompromisingly mathematical in your approach.

I have also learned a great deal about the psychological and social aspects of software development from Brad Myers of the HCI Institute and from the faculty and students of the Institute for Software Research (ISR), particularly Thomas LaToza and Joshua Sunshine. In addition, I have collaborated with Alex Potanin, who visited us on sabbatical, and with students Darya Melicher, Ligia Nistor, Benjamin Chung and Chenglong Wang on projects related to the work presented here. Thank you all for broadening my perspective on the art and science of language design. 

During my time in graduate school, I have had the privilege to attend a great many  conferences, workshops and summer schools where I participated in more illuminating conversations than I could  hope to recall here. I am particularly grateful for conversations with Eric Van Wyk, who never failed to appreciate the subtle contours of a design space and graciously served on my thesis committee. I would also like to thank the organizers and participants of the Oregon PL Summer School, where I had an amazing time learning how to properly prove the proper theorems. Finally, I am grateful to have collaborated with Ian Voysey, Michael Hilton and Matthew Hammer on Hazelnut, a side project that quite successfully delayed the completion of this dissertation. With friends and collaborators like these, why graduate?

I would be remiss not to mention Brent Doiron, who took me as a student when I entered graduate school in the Neural Computation PhD program, and Garrett Kenyon,  who was my practicum supervisor during my ``man vs. wild'' stint at Los Alamos National Lab. Both of them left me with a deep appreciation for the  mathematical and computational methods used to study the dynamics of neurobiological systems. I hope some day, far in the future, to return to neuroscience with a pack full of truly modern programming tools.

I also want to explicitly acknowledge Deb Cavlovich, Catherine Copetas, Victoria Poprocky and all of the other staff that have kept things running smoothly around the school, and at conferences and other events. I really appreciate all of the work that you do.

Graduate school can be an emotionally taxing experience, to say the least. Fortunately, my friends were there whenever I came up for air -- sometimes after going under for weeks at a time. Tommy and Sanna, you are beautiful souls and our travels together have been incredibly rejuvenating. D, you have the most exquisite taste and it's so good to know you (yeh yeh yeh.) To the greater Miasma crew, Ian, V, Tom7 and the rest of the thursdz crew, and my former officemate, Harsha: yinz are such fascinating people and I really enjoy the time we spend together. The same goes for so many other individuals that I've connected with personally, whether at conferences, at office hours, through the CNBC, CSGF, LANL, WRCT, SCS, on Twitter, at shows and festivals, in the woods, or in apartments and backyards. You've brought such color and texture to my life.

I especially want to remember Dan Schreiber. Dan was one of my very closest friends, a romantic visionary and the greatest debate partner I have ever had. He died in 2010. I am so glad to have known him and I only wish that I could have heard his take on so many of the topics I've learned about since then -- proof theory, type theory, tech cooperatives, experimental music, long-distance cycling, old books, psychedelic films,  spontaneous theater and colonizing Venus, to name a few topics I suspect he'd have some thoughts about! Dan is truly missed.

Finally, so much of who I am is due to the love and support of my family. The diverse personalities of my aunts, uncles, cousins and their spouses made family gatherings so lively. My sister, Elisha, and now her husband, Pat, have been an endless source of great book recommendations and conversations. And I am forever grateful to Ami and Hibbi Abu, my mother and father, who have given me so much love, offered so many heartfelt prayers and provided me with so much practical assistance and advice throughout my life. From an early age, they encouraged me to maintain an independent mind, to cultivate the highest intellectual standards  and to remember the Big Picture at all times. Those lessons, rooted in the traditions of our family going back generations, have proven incredibly valuable in research, and in life. Thank you.

It's been an unforgettable journey. Thanks, everybody. 
\end{acknowledgments}


\tableofcontents
\listoffigures
%\listoftables

\mainmatter

% The other requirements Catherine has:
%
%  - avoid large margins.  She wants the thesis to use fewer pages, 
%    especially if it requires colour printing.
%
%  - The thesis should be formatted for double-sided printing.  This
%    means that all chapters, acknowledgements, table of contents, etc.
%    should start on odd numbered (right facing) pages.
%
%  - You need to use the department standard tech report title page.  I
%    have tried to ensure that the title page here conforms to this
%    standard.
%
%  - Use a nice serif font, such as Times Roman.  Sans serif looks bad.
%
% Other than that, just make it look good...

% !TEX root = omar-thesis.tex
\chapter{Introduction}\label{chap:intro}
% \vspace{-14px}
\includegraphics[width=\textwidth]{Picasso-Bull-Progression-cropped.png}
\begin{flushright}
\emph{Bull} (plates 3, 6, 9 and 11)\\
Pablo Picasso (1881-1973)\end{flushright}
% http://www.artyfactory.com/art_appreciation/animals_in_art/pablo_picasso.htm
%\vspace{-5px}
% \begin{quote}\textit{The recent development of programming languages suggests that the simul\-taneous achievement of simplicity 
% and generality in language design is a serious unsolved 
% problem.}\begin{flushright}--- John Reynolds (1970) \cite{Reynolds70}\end{flushright}
% \end{quote}
%\begin{quote}
%\textit{Try to imagine that you are a tree. How do you want to look out here?}
%\textit{You want your tree to have some character.}
%\begin{flushright} --- Bob Ross, \emph{The Joy of Painting}\end{flushright}
%\end{quote}

% \vspace{-7px}
\section{Motivation}\label{sec:intro-motivation}
% \vspace{-6px}
%Programming languages come in many sizes. The smallest languages -- for example, the various ``lambda calculi'' -- isolate language primitives of interest for the benefit of students, researchers and language designers interested in studying their mathematical properties. These studies inform the design of ``full-scale'' programming 
%\footnote{Throughout this work, words and phrases that should be read as having an intuitive or informal meaning, rather than a strict mathematical meaning, will be introduced with quotation marks.} 
% languages, which combine several such primitives, or generalizations thereof. Full-scale languages are interesting objects of formal study in their own right. They also serve as useful tools for software developers, allowing them to construct, reason about and modularly organize large software systems.
% A single mathematical structure can often take on many syntactic forms. 
% Formal mathematical structures often come equipped with
% Experienced mathematicians and programmers define formal structures \emph{compositionally}, drawing from libraries by instantiating more abstract structures. This ultimately increases productivity, because clients of these abstract structures do not need to expend effort to establish the associated definitions and proofs anew, for each specialized structure of interest. %Instead, they need only instantiate the definitions and proofs established by a library provider in a more abstract setting.

Experienced mathematicians and programmers define formal structures \emph{compositionally}, drawing from libraries of ``general-purpose'' abstractions. The problem that motivates this work is that the resulting terms are sometimes syntactically unwieldy, and, therefore, cognitively costly. % This can neutralize the cognitive benefits of abstraction and composition. We go, therefore, in search of a mechanism of syntactic control that maintains strong compositional reasoning principles. %This can lower productivity, readability and other quality attributes of interest.  %This saves time, one does not need to establish associated definitions and proofs anew, for each specialized structure of interest.

Consider, for example, natural numbers. It is straightforward to define the natural numbers, $n$, with an inductive structure:
\[ n ::= \textbf{z} ~\vert~ \textbf{s}(n)\]
By defining natural numbers inductively, we immediately inherit a \emph{structural induction principle} -- we can establish that some property $P$ holds over the natural numbers if we establish $P(\textbf{z})$ and $P(\textbf{s}(n))$ assuming $P(n)$. The problem, of course, is that drawing particular natural numbers by repeatedly applying $\textbf{s}$ very quickly becomes syntactically unwieldy (in fact, the syntactic cost of the drawing grows linearly with $n$.)\footnote{We use the word ``drawing'' throughout this document to emphasize that syntactic cost is a property of the visual representation of a structure, rather than a semantic property.}

Similarly, it is easy to define lists of natural numbers with an inductive structure:
\[ \vec{n} ::= \textbf{nil} ~\vert~ \textbf{cons}(n, \vec{n}) \]
The problem once again is that drawings of particular lists quickly become unwieldy, and fail to resemble ``naturally occurring'' drawings of lists of numbers.

Consider a third more sophisticated example (which will be of particular relevance later in this work): when defining a programming language or logic, one often needs various sorts of tree structures equipped with metaoperations\footnote{...so named to distinguish them from the ``object level'' operations of the language being defined.} related to variable binding, e.g. substitution. Repeatedly defining these structures ``from scratch'' is quite tedious, so language designers have instead developed  a more general structure: the \emph{abstract binding tree (ABT)} \cite{Aczel78,pfpl,gabbay2002new}. Briefly, an ABT is an ordered tree structure, classified into one of several \emph{sorts}, where each node is either a \emph{variable}, $x$, or an \emph{operation} of the following form:
%\footnote{Some prior exposure to (single-sorted) ASTs is assumed here. See Sec. \ref{sec:preliminaries} for other preliminaries.} 
\begin{equation*}
\abop{op}{\vec{x}_1.\mathit{a}_1; \ldots; \vec{x}_n.\mathit{a}_n}
\end{equation*} 
where $\texttt{op}$ identifies an \emph{operator} and each of the $n \geq 0$ \emph{arguments} $\vec{x}_i.\mathit{a}_i$ binds the (possibly empty) sequence of variables $\vec{x}_i$ within the subtree $a_i$. The left side of the syntax chart in Figure \ref{fig:simple-example} summarizes the relevant operational forms for a sort called $\mathsf{CalcExp}$. ABTs of this sort are the expressions of a small arithmetic programming language,  $\simplelang$. By using  ABTs as infrastructure in the definition of $\simplelang$, we need not manually define the ``boilerplate'' metaoperations, like substitution, and reasoning principles, like structural induction, that are necessary to define $\simplelang$'s semantics and to prove it correct. {Harper gives a detailed account of ABTs, and many other examples of their use, in his book \cite{pfpl}.} 

 %and reasoning principles, e.g. {structural induction}, so we need not define this  machinery manually. 
% -- the arities of the operators can be read off from these forms ($\anumintro{n}$ is a number-indexed family of nullary operators.) 

\begin{figure}
\hspace{-5px}$\begin{array}{lrlllll}
\textbf{Sort} & & & \textbf{Operational Form} & \textbf{Stylized Form} & \textbf{Textual Form} & \textbf{Description}\\
\mathsf{CalcExp} & e & ::= & x & x & x & \text{variable}\\
&&& \aletplain{e}{x}{e} & \letplain{x}{e}{e} & \letplain{x}{e}{e} & \text{binding}\\
&&& \anumintro{n} & \numintro{n} & \numintro{n} & \text{numbers}\\
&&& \aplus{e}{e} & e + e & e\texttt{ + }e & \text{addition} \\
% &&& \aminus{e}{e} & e - e & e\texttt{ - }e & \text{subtraction}\\
&&& \amult{e}{e} & e \times e & e\texttt{ * }e & \text{multiplication}\\
&&& \adiv{e}{e} & \frac{e}{e} & e\texttt{ / }e & \text{division}\\
&&& \apow{e}{e} & {e}^{e} & e\verb|^|e & \text{exponentiation}\\
\end{array}$
\caption[Syntax of $\simplelang$]{Syntax of $\simplelang$. Metavariable $n$ ranges over natural numbers and $\numintro{n}$ abbreviates the numeral forms (one for each natural number $n$, drawn in \texttt{typewriter} font.) A formal definition of the stylized and textual syntax of $\simplelang$ would require 1) defining these numeral forms explicitly; 2) defining a parenthetical form; 3) defining the precedence and associativity of each infix operator; and 4) defining whitespace conventions.}
\label{fig:simple-example}
% \vspace{-5px}
\end{figure}

% \subsection{Syntax Matters}
The problem with this approach is, again, that drawing a non-trivial $\simplelang$ expression in operational form is syntactically costly. For example, we will consider the following drawing in our discussion below:
\begin{subequations}\label{drawings:simple}\begin{equation}\label{simple-example-op-form}
\adiv{\anumintro{\textbf{s}(\textbf{z})}}{
	\apow{\anumintro{\textbf{s}(\textbf{s}(\textbf{z}))}}{\adiv{\anumintro{\textbf{s}(\textbf{z})}}{\anumintro{\textbf{s}(\textbf{s}(\textbf{z}))}}}
}\end{equation}
% This is an example of a common problem: instantiating a general-purpose abstraction, here for defining ABTs, can be  structurally economical but  \emph{syntactically costly} (or \emph{cognitively costly} in some other sense, as we will discuss in Section \ref{sec:syntactic-properties}.) Mathematics is ultimately a human activity, so these costs are worthy of consideration.

\subsection{Informal Mathematical Practice}
Within a document intended only for human consumption, it is easy to informally outline less costly alternative syntactic forms. 

For example, mathematicians generally use the Western Arabic numeral forms when drawing particular natural numbers, e.g. $2$ is taken as a syntactic alternative to $\textbf{s}(\textbf{s}(\textbf{z}))$. 

Similarly, mathematicians might informally define alternative list forms, e.g. $[0, 1, 2]$ as a syntactic alternative to: 
\[\textbf{cons}(\textbf{z}, \textbf{cons}(\textbf{s}(\textbf{z}), \textbf{cons}(\textbf{s}(\textbf{s}(\textbf{z})), \textbf{nil})))\]

The middle columns of the syntax chart in Figure \ref{fig:simple-example} suggest two alternative forms for every ABT of sort $\mathsf{CalcExp}$. We can draw the ABT from Drawing (\ref{simple-example-op-form}) in an alternative \emph{stylized form}:
% div(num[1]; pow(num[2]; div(num[1]; num[2]))
% \begin{subequations}
% \begin{equation}\label{simple-example-op-form}
% \adiv{\anumintro{1}}{
% 	\apow{\anumintro{2}}{\anumintro{3}}
% }\end{equation}
\begin{equation}\label{simple-example-sty-form}
\frac{\numintro{1}}{{\numintro{2}^{\frac{\numintro{1}}{\numintro{2}}}}}
\end{equation}
or in an alternative \emph{textual form}:
\begin{equation}\label{simple-example-txt-form}
\texttt{1 / 2\textasciicircum(1/2)}
\end{equation}
\end{subequations}

Mathematicians also sometimes supplement alternative primitive forms like these with various \emph{derived forms}, which  identify ABTs indirectly according to stated context-independent \emph{desugaring rules}. For example, the following desugaring rule defines a derived stylized form for square root calculations:
% \begin{subequations}
% \begin{subequations}[intermezzo]
\begin{equation}\label{rule:simplelang-sqrt}
\sqrt{e} \rightarrowtriangle e^{\frac{\numintro{1}}{\numintro{2}}}
\end{equation}
% \end{subequations}
The reader can desugar a drawing of an ABT by recursively applying desugaring rules wherever a syntactic match occurs. A desugared drawing consists only of the {primitive forms} from Figure \ref{fig:simple-example}. 
For example, the following drawing desugars to Drawing (\ref{simple-example-sty-form}), which in turn corresponds to Drawing (\ref{simple-example-op-form}) as discussed above:
\begin{equation*}\tag{\ref*{drawings:simple}d}
\frac{\numintro{1}}{\sqrt{\numintro{2}}}
\end{equation*}
%No new operators are introduced.

% Syntactically, however, this practice has its limits. 

% Mathematicians often invent specialized syntactic forms in order to visually represent the formal structures that they define. 
% % For example, Figure \ref{fig:simple-example} defines three different ways to draw any abstract syntax tree (AST) of sort  $\mathsf{CalcExp}$. ASTs of this sort are the expressions of a small arithmetic programming language, $\simplelang$.\footnote{Some familiarity with abstract syntax trees is preliminary to this work. See Sec. \ref{sec:preliminaries} for citations and a more thorough discussion of preliminaries.}
% % Mathematicians, like painters, exercise creative license when they draw the structures that arise in their work. 
% % Mathematicians, like painters, exercise creativity when they draw the structures that arise in their work. 
% % Mathematicians often define structurally redundant syntactic forms. 
% %, i.e. forms that are syntactically distinct but that identify the same formal structure. 
% % When defining the syntax of a programming language, language designers often define structurally redundant syntactic forms. 
% % Mathematicians, like painters, draws these trees in a variety of styles.  
% % There are many ways to draw trees of this sort. 
% % There are many ways to draw a tree. 
% % Most formal structures are defined as modes of use of more primitive formal structures. For example, the expressions of a variety of programming languages are all defined as particular sorts of \emph{abstract syntax trees}.  
% For example, the three drawings below all identify the same tree structure  of sort $\mathsf{CalcExp}$, differing according to the syntax chart in Figure \ref{fig:simple-example} only in that the first drawing is in a general \emph{operational form}, whereas the second drawing is in a specialized \emph{stylized form} and the third is in a specialized \emph{textual form}:
% %
% % differing only in that Drawing (\ref*{simple-example-op-form}) is in \emph{operational form}, Drawing (\ref*{simple-example-sty-form}) is in \emph{stylized form} and Drawing (\ref*{simple-example-txt-form}) is in \emph{textual form}:
% \begin{subequations}
% \begin{equation}\label{simple-example-op-form}
% \adiv{\anumintro{1}}{
% 	\apow{\anumintro{2}}{\anumintro{3}}
% }\end{equation}
% \begin{equation}\label{simple-example-sty-form}
% \frac{\numintro{1}}{{\numintro{2}^{\numintro{3}}}}
% \end{equation}
% \begin{equation}\label{simple-example-txt-form}
% \texttt{1 / 2\textasciicircum3}
% \end{equation}
% \end{subequations}
% \noindent
% Trees of this sort are the expressions of $\simplelang$, a simple arithmetic programming language. 

% These drawings identify the same AST, meaning that they are all drawn from the same row of the syntax chart at every level. In other words, these drawings are structurally indistinct. 

% In particular, let us consider a simple programming language, $\simplelang$, for performing arithmetic calculations with numbers. The expressions of $\simplelang$ are \emph{abstract syntax trees (ASTs)} of a sort defined by the syntax chart in Figure \ref{fig:simple-example}.\footnote{Familiarity with abstract syntax trees is preliminary to this work (see Sec. \ref{sec:preliminaries} for other preliminaries.)}  For example, the following expression is drawn in stylized form:
% The same expression is drawn in textual form as follows:
% \noindent
% and in operational form as follows:

When defining the semantics of a language like $\simplelang$, it is customary to adopt an \emph{identification convention} whereby drawings that identify the same underlying ABT structure, like Drawings (\ref{drawings:simple}), are considered interchangeable. %only if they are drawn from different rows of the syntax chart in Figure \ref{fig:simple-example}. 
For example, consider the semantic  judgement $\isvalU{e}$, which establishes certain  $\simplelang$ expressions as \emph{values} (as distinct from expressions that can be arithmetically simplified or that are erroneous.) The following inference rule establishes that every number expression is a value:\footnote{Some familiarity with inductively defined judgements and inference rules like these is preliminary to this work. See Sec. \ref{sec:preliminaries} for citations and further discussion of necessary preliminaries.}
\begin{equation}\label{rule:num-val}
\inferrule{ }{
	\isvalU{\anumintro{n}}
}
\end{equation}
Although this rule is drawn using the operational form for number expressions, we can apply it to derive that $\isvalU{\numintro{2}}$, because $\numintro{2}$ and $\anumintro{2}$  identify the same ABT.

% In summary, it is both the case that ``syntax doesn't matter'' (semantically) and that syntax matters (cognitively). %i.e.  because mathematics is a human activity that alternative and derived forms matter (cognitively).%Syntax matters because mathematics and programming are human activities. 


% \subsection{Syntax Doesn't Matter  (Semantically)}
% It is worth emphasizing here that these common syntactic practices are not motivated by semantic considerations. Indeed, 



% \subsection{Syntax Matters (Cognitively)}
% %Our semantics would be no weaker if we had defined only, for example, the operational forms. 
% %The answer, of course, is that from the perspective of a human programmer, syntax \emph{does} matter. 
% % If different drawings of a $\simplelang$ expression are not semantically distinguishable, why did we bother to define alternative syntactic forms at all?
% Syntactic sugar matters because mathematics and programming are human activities. Different visual representations of a formal structure can and must be distinguished by the \emph{cognitive costs} that human programmers incur as they produce or examine them.\footnote{In fact, we should be interested in sensory modalities other than vision, if only because many humans lack sufficient eyesight. Alas, this topic is beyond the scope of our present work.}% Drawings of formal structures serve as \emph{user interfaces} to the underlying structures themselves.

% For example, a human programmer might distinguish drawings of $\simplelang$ expressions in stylized or textual form, like Drawings (\ref{simple-example-sty-form}) through (\ref{simple-example-derived-form}) above, as less ``crowded'' and more ``familiar'' than those in operational form, because they follow the usual arithmetic conventions or close approximations thereof. This might help the human  programmer extract meaning from such drawings more quickly. Similarly, drawings in textual form involve only text, which can lower the costs involved in their production. Of course, drawings in operational form can bring cognitive benefits as well -- dispensing with various syntactic complexities (e.g. related to  precedence and associativity) can simplify metatheoretic reasoning and implementation efforts. 

% We will cover more rigorous operationalizations of the necessarily broad notion of cognitive cost in Section \ref{sec:syntactic-properties}. %Mistakes may also be less frequent when producing drawings in stylized or textual form (for $\simplelang$ expressions, perhaps only because operational forms use more parentheses). 

% \subsection{Derived Forms}
% %The forms defined by the syntax chart in Figure \ref{fig:simple-example} suffice to allow programmers to draw any $\simplelang$ expression. However, 
% In seeking to lower cognitive costs, syntax designers  often  include additional \emph{derived forms}  in a syntax definition. Unlike the \emph{primitive forms}  defined in Figure \ref{fig:simple-example}, which identify trees directly, derived forms identify trees indirectly, through a context-independent {desugaring} rule.   
% %We can define a desugaring transformation by stating a rewrite rule. 
% For example, the following desugaring rule defines a derived stylized form for calculating the square root of a $\simplelang$ expression:
% % \begin{subequations}
% \begin{equation}\label{rule:simplelang-sqrt}
% \sqrt{e} \rightarrowtriangle e^{\frac{\numintro{1}}{\numintro{2}}}
% \end{equation}
% % Similarly, the following rewrite rule, if included in the definition of the textual syntax of expressions, defines a derived form for negating a $\simplelang$ expression:\footnote{Notice that the right-hand side of this rule is in operational form, rather than textual. For $\footnotesimplelang$, it is not necessary to prevent textual and operational forms from being interspersed within a single drawing -- no ambiguities can arise. For richer syntax definitions, this may no longer be the case. The desugaring process must then be modified to first convert the pattern on the righthand side of a desugaring rule like Rule (\ref{rule:simplelang-negate}) to the desired form variant before it is applied.}
% % \begin{equation}\label{rule:simplelang-negate}
% % \texttt{-}e \rightarrowtriangle \amult{e}{\anumintro{-1}}
% % \end{equation}
% % \end{subequations}
% % \noindent 
% Desugaring a drawing of a tree involves first recursively desugaring the drawings of its subtrees. If the drawing is in primitive form, desugaring is complete.  If the drawing is in derived form, we apply the corresponding desugaring rule (here, we  have only one choice.) The desugared drawing will identify a tree immediately, i.e. it will consist only of primitive forms. No new trees are introduced, so the semantics is unchanged. %Derived forms affect only cognitive cost.
%Similarly, we might define a derived form for taking an arbitrary root of an expression as follows:
% \begin{align*}
% \sqrt[e']{e} & \rightarrowtriangle e^{\frac{\numintro{1}}{e'}}
% \end{align*}

\subsection{Derived Forms in General-Purpose Languages}
We would need to define only a few more derived arithmetic forms to satisfyingly capture the  idioms that arise in the limited domains where a simple language of arithmetic operations like $\simplelang$ might be useful. %Consequently, there is little opportunity to go beyond simple derived forms like these. 
However, programming languages in common use today are substantially more semantically expressive. Indeed, many mathematical structures, including natural numbers, lists and ABTs, can be adequately expressed within contemporary ``general-purpose'' programming languages. %Mathematics has become supplanted by particular formal system. 
Consequently, the problems of syntactic cost just discussed at the level of the ambient mathematics also arise ``one level down'', i.e. when writing programs. For example, we want syntactic sugar not only for mathematical natural numbers, lists and $\simplelang$ expressions, but also for \emph{encodings} of these structures within a general-purpose programming language.% .) %https://github.com/jonsterling/sml-abt or https://github.com/RedPRL/sml-typed-abts.) 

We can continue to rely on the informal notational conventions described above only as long as programs are drawn solely for human consumption. These conventions break down when we need drawings of programs to themselves exist as formal structures suitable for consumption by other programs, i.e. \emph{parsers}, which check whether drawings are well-formed relative to a \emph{syntax definition} and produce structures suitable for consumption by yet other programs, e.g. compilers. %This, of course, is the regime of contemporary computer programming.

% The problem is that nearly all contemporary languages are designed so that drawings of programs can be consumption both by humans and by another program -- a parser -- which checks whether drawings are well-formed and produces a structure suitable for consumption by various other useful programs, e.g. editors and compilers.


% Programmers address these problems by informally stating alternative and derived forms, as in handwritten or typeset mathematics, because a drawing of a program must be suitable for consumption both by humans and by % This limits the control that programmers have over syntactic cost.

% In contemporary practice, other programs -- parsers -- consume drawings of programs, which generate corresponding structures for execution by modern computer hardware. As such, we cannot rely on an informal approach that defers ultimately to the intuitions of a human reader. Instead, we must define the syntactic conventions that we wish to use with rigor. 

Constructing a formal syntax definition is not itself an unusually difficult task for an experienced programmer, and there are many \emph{syntax definition systems} that help with this task (Sec. \ref{sec:existing-approaches} will cover several examples.) The problem is that when designing the syntax of a general-purpose language, the language designer cannot hope to anticipate all library constructs for which derived forms might one day be useful. At best, the language designer can bundle certain libraries together into a ``standard library'', and privilege select constructs defined in this library with derived forms. 

% as these ``general-purpose'' languages have evolved, many other derived forms have become incorporated into their syntax definitions.\footnote{The same dynamic is apparent in the progression of ``pen and paper'' and typeset mathematics.}  
For example, the textual syntax of Standard ML (SML), a general-purpose language in the functional tradition, defines derived forms for constructing and pattern matching on lists \cite{mthm97-for-dart,harper1997programming}. In SML, the derived expression form \lstinline{[x, y, z]} desugars to an expression equivalent to:
\begin{lstlisting}[numbers=none]
Cons(x, Cons(y, Cons(z, Nil)))
\end{lstlisting}
assuming \li{Nil} and \li{Cons} stand for the list constructors exported by the SML Basis library (i.e. SML's ``standard library''.)\footnote{The desugaring actually uses unforgeable identifiers bound permanently to the list constructors, to ensure that the desugaring is context independent. We will return to the concept of context independence throughout this work.} Other languages similarly privilege select standard library constructs with derived forms:

\begin{itemize}
\item OCaml \cite{ocaml-manual} defines derived forms for strings (defined as arrays of characters.)
\item Haskell \cite{jones2003haskell} defines derived forms for encapsulated commands (and, more generally, values of any type equipped with monadic structure.)
\item Scala \cite{odersky2008programming} defines derived XML forms as well as string splicing forms, which capture the idioms of string concatenation.
\item F\# \cite{syme2012expert}, Scala \cite{shabalin2013quasiquotes} and various other languages define derived forms for encodings of the language's own terms (these are referred to as \emph{quasiquotation} forms.)
\item Python \cite{python} defines derived forms for mutable sets and dictionaries.
\item Perl \cite{perlre} defines derived regular expression forms.
\end{itemize}

These choices are, fundamentally, made according to \emph{ad hoc} design criteria -- there are no clear semantic criteria that fundamentally distinguish standard library constructs privileged with derived forms from those defined in third-party libraries. 
%Indeed, it is considered a virtue for a standard library can be separated from the language definition. 
Indeed, as the OCaml community has moved away from a single standard library in favor of competing bundles of third-party libraries (e.g. Batteries Included \cite{OCaml-batteries} and Core \cite{OCaml-core}), this approach has become starkly impractical.% This puts into question the practice of bundling a single standard library with  entirely.

\section{Existing Mechanisms of Syntactic Control}
A more parsimonious approach would be to eliminate derived forms  specific to standard library constructs from language definitions in favor of mechanisms that give more syntactic control to third-party library providers.

In this section, we will give a brief overview of existing such mechanisms and speak generally about the problems that they present to motivate our novel contributions in this area. We will return to give a detailed overview of these various existing mechanisms of syntactic control in Section \ref{sec:existing-approaches}. 

\subsection{Syntax Dialects}\label{sec:problems-with-dialects}
One approach that a library provider can take when seeking more syntactic control is to use a syntax definition system to construct a \emph{syntax dialect}, i.e. a new syntax definition that extends the original syntax definition with new derived forms. 
%library-specific (a.k.a. ``domain-specific'') 

For example, Ur/Web extends Ur's textual syntax with derived forms for SQL queries, XHTML elements and other constructs defined in a  web programming library \cite{conf/popl/Chlipala15,conf/pldi/Chlipala10}. Figure \ref{fig:urweb} demonstrates how XHTML expressions that contain strings can be drawn in Ur/Web. The desugaring of this derived form (not shown) is substantially more verbose and, for programmers familiar with the standardized syntax for XHTML \cite{xhtml}, substantially more obscure. % Such dialects are sometimes qualitatively taxonomized as amongst the ``domain-specific language'' for this reason \cite{fowler2010domain}. %Syntactic cost is often assessed qualitatively \cite{green1996usability}, though quantitative metrics can be defined. 
\begin{figure}[h]
\begin{lstlisting}[numbers=none]
val p = SURL<xml><p>Hello, {[EURLjoin " " [first, last]SURL]}!</p></xml>EURL
\end{lstlisting}
\caption{Derived XHTML forms in Ur/Web}
\label{fig:urweb}
\end{figure}                           

Syntax definition systems like Camlp4 \cite{ocaml-manual}, Copper \cite{conf/gpce/WykS07} and SugarJ/Sugar* \cite{erdweg2011sugarj,erdweg2013framework}, which we will discuss in Sec. \ref{sec:syntax-dialects}, have simplified the task of defining ``library-specific'' (a.k.a. ``domain-specific'') syntax dialects like Ur/Web, and have thereby contributed to their ongoing proliferation.
%The desugaring, not shown, is substantially more verbose and, for programmers who are familiar with XHTML forms, substantially more obscure than the drawing above. %We will consider other examples of data structures where syntactic cost becomes a legitimate concern for client programmers in Sec. \ref{sec:motivating-examples}. 
%after first reviewing simpler approaches that also help library providers control syntactic cost, albeit to a more limited extent, 
% Syntax definition systems 
% The most syntactically expressive of the mechanisms that we will detail in Section \ref{sec:existing-approaches} are 


%Full-scale languages are also interesting objects of mathematical study. Uniquely, however, they are also designed for use by humans. Consequently, their designers  typically define both an abstract syntax and a textual syntax. This textual syntax serves as the primary interface between human programmers and the language, so it is common to define various \emph{derived forms}, i.e. forms defined by a context-independent \emph{desugaring} to a set of \emph{base forms}. These serve to decrease the \emph{syntactic cost} or \emph{cognitive cost} of selected idioms. 
%In some cases, a derived form is designed to capture an idiom77Gu that involves only the primitive constructs of the language. 

%The hope amongst some language designers is that a limited number of derived forms like these will suffice to produce a ``general-purpose'' textual syntax, i.e. one that is accepted as suitable for use across a wide variety of application domains. Alas, a stable design that fully achieves this ideal has yet to emerge, as evidenced by the diverse array of \emph{syntax dialects} -- dialects that introduce only new derived forms -- that continue to proliferate around all major contemporary languages. 

%In fact, tools that aid in the construction of so-called  ``domain-specific'' language dialects (DSLs)\footnote{In some parts of the literature, such dialects are called ``external DSLs'', to distinguish them from  ``internal'' or ``embedded DSLs'', which are actually  library interfaces that only ``resemble'' distinct dialects \cite{fowler2010domain}.} seem only to be becoming more prominent over time. 

%\subsection{Why are there so many language dialects?}
%{This calls for an investigation}: why is it that programmers and researchers are still so often unable to satisfyingly express the constructs that they seek in libraries, as modes of use of the ``general-purpose'' primitives already available in major languages today, and instead see a need for new language dialects?

%Perhaps the most common sort of dialect is the \emph{syntax dialect} -- a dialect that introduces only new derived syntactic forms, motivated by a desire to decrease the {syntactic cost} of working with one or more library constructs of interest. 
%Put another way, syntax dialects can be specified by a context-independent expansion to the existing language that they are based on. 
%For example, Ur/Web is a syntax dialect of Ur (a language that itself descends from ML \cite{conf/pldi/Chlipala10}) that builds in derived forms for SQL queries, HTML elements and other datatypes used in the domain of web programming \cite{conf/popl/Chlipala15}. %Syntactic cost is often assessed qualitatively \cite{green1996usability}, though quantitative metrics can be defined. 
%This is not an isolated example -- we will consider a number of additional types of data that similarly stand to benefit from the availability of specialized derived forms in Sec. \ref{sec:motivating-examples}. 
%Tools like Camlp4 \cite{ocaml-manual}, Sugar* \cite{erdweg2011sugarj,erdweg2013framework} and Racket \cite{Flatt:2012:CLR:2063176.2063195}, which we will discuss in Sec. \ref{sec:existing-approaches}, have lowered the engineering costs of constructing syntax dialects in such situations, further contributing to their proliferation. 

%More advanced dialects introduce new type structure, going beyond what is possible with only new derived forms. As a simple example, the static and dynamic semantics of records cannot be expressed by context-independent expansion to a language with only nullary and binary products. Various languages have explored ``record-like'' primitives that go further, supporting functional update operators, width and depth coercions (sometimes implicit)%\cite{Cardelli:1984:SMI:1096.1098}
%, methods, prototypic dispatch and other such ``semantic embellishments'' that in turn cannot be expressed by context-independent expansion to a language with only standard record types (we will detail an  example in Sec. \ref{sec:metamodules-motivating-examples}). OCaml primitively builds in the type structure of polymorphic variants, open datatypes and  operations that use format strings like $\mathtt{sprintf}$ \cite{ocaml-manual}. ReactiveML builds in primitives for functional reactive programming \cite{mandel2005reactiveml}. ML5 builds in high-level primitives for distributed programming based on a modal lambda calculus \cite{Murphy:2007:TDP:1793574.1793585}. Manticore \cite{conf/popl/FluetRRSX07} and AliceML  \cite{AliceLookingGlass} build in parallel programming primitives with a more elaborate type structure than is found in simpler accounts of parallelism. 
%MLj builds in the type structure of the Java object system (motivated by a desire to interface safely and naturally with Java libraries) \cite{Benton:1999:IWW:317636.317791}. Other dialects do the same for other foreign languages, e.g. Furr and Foster describe a dialect of OCaml that builds in the type structure of C \cite{Furr:2005:CTS:1065010.1065019}. Tools like proof assistants and logical frameworks are used to specify and reason metatheoretically about dialects like these, and tools like compiler generators and language frameworks \cite{erdweg2013state} lower their implementation cost, again contributing to their proliferation. 

% \vspace{-5px}
%\subsection{Problems with the Dialect-Oriented Approach}\label{sec:problems-with-dialects}
Many have argued that a proliferation of syntax dialects is harmless or even desirable, because programmers can simply choose the right syntax dialect for each job at hand \cite{journals/stp/Ward94}. However, we argue that this ``dialect-oriented approach'' is difficult to reconcile with the best practices of ``programming in the large''  \cite{DeRemer76}, i.e. developing large programs ``consisting of many small programs (modules), possibly written by different people'' whose interactions are mediated by a reasonable type and binding discipline. The problems that tend to arise are summarized below; a more systematic treatment will follow in  Sec. \ref{sec:syntax-dialects}.

\subsubsection{Problem 1: Conservatively Combining Syntax Dialects}
The first problem with the dialect-oriented approach is that clients  cannot always combine different syntax dialects when they want to use derived forms that they define together. This is problematic because client programs  cannot be expected to fall cleanly into a single preconceived ``problem domain'' -- large programs use many libraries \cite{DBLP:conf/sac/LammelPS11}.

For example, consider a syntax dialect, $\mathcal{H}$, defining derived forms for working with encodings of HTML elements, and another syntax dialect, $\mathcal{R}$,  defining derived forms for working with encodings of regular expressions. Some programs will undoubtedly need to manipulate HTML elements as well as regular expressions, so it would be useful to construct a ``combined dialect'' where all of these derived forms are defined. 

For this notion of ``dialect combination'' to be well-defined at all, we must first have that $\mathcal{H}$ and $\mathcal{R}$ are defined under the same syntax definition system. In practice, there are many useful syntax definition systems, each differing subtly from the others. %If the dialect designers  have not  chosen the same syntax definition system, then ``dialect combination'' is not systematic (in the way that importing different libraries is systematic.)%$\mathcal{H} \cup \mathcal{R}$ is simply undefined.% (e.g. parser combinator libraries like Haskell's \li{parsec} \cite{parsec}.)

If $\mathcal{H}$ and $\mathcal{R}$ are coincidentally defined under the same syntax definition system, we must also have that this system operationalizes the notion of dialect combination, i.e. it must define some operation $\mathcal{H} \cup \mathcal{R}$ that creates a dialect that extends both $\mathcal{H}$ and $\mathcal{R}$, meaning that any form defined by either $\mathcal{H}$ or $\mathcal{R}$ must be defined by $\mathcal{H} \cup \mathcal{R}$. Under systems that do not define such an operation (e.g. Racket's dialect preprocessor \cite{Flatt:2012:CLR:2063176.2063195}), clients can only manually  ``copy-and-paste'' or factor out portions of the constituent dialect definitions to construct the ``combined'' dialect. This is not systematic and, in practice, it can be quite tedious and error-prone. %In both this and the previous case, ``dialect combination'' is a strictly informal notion, left to library clients to operationalize through manual labor (hence the quotes).

Even if we restrict our interest  to dialects defined under a common syntax definition system that does operationalize the notion of dialect combination (or similarly one that allows clients to systematically combine \emph{dialect fragments}), we still have a problem: there is generally no guarantee that the combined dialect will conserve important properties that can be established about the constituent dialects in isolation (i.e. \emph{modularly}.) In other words, establishing $P(\mathcal{H})$ and $P(\mathcal{R})$ is not sufficient to establish $P(\mathcal{H} \cup \mathcal{R})$ for many useful properties $P$. Clients must re-establish such properties for each combined dialect that they construct.%In other words, any putative ``combined language'' must formally be considered a  distinct system for which one must derive essentially all metatheorems of interest anew, guided only informally by those derived for the dialects individually. %There is no well-defined mechanism for constructing such a ``combined language'' in general. 

One important property of interest is \emph{syntactic determinism} -- that every derived form has at most one desugaring. It is not difficult to come up with examples where combining two deterministic syntax dialects produces a non-deterministic dialect. For example, consider two syntax dialects defined under a system like Camlp4: $\mathcal{D}_1$ defines derived forms for sets, and $\mathcal{D}_2$ defines derived forms for finite maps, both delimited by \verb~{<~ and \verb~>}~.\footnote{In OCaml, simple curly braces are already reserved by the language for record types and values.} Though each dialect defines a deterministic grammar, i.e. $\mathrm{det}(\mathcal{D}_1)$ and $\mathrm{det}(\mathcal{D}_2)$, when the grammars are na\"ively combined by Camlp4, we do not have that $\mathrm{det}(\mathcal{D}_1 \cup \mathcal{D}_2)$ (i.e. syntactic ambiguities arise under the combined dialect.) In particular, \verb~{<>}~ can be recognized as either the empty set or the empty finite map. %A recent version of Python added derived forms for mutable sets. Due to a conflict with dictionary syntax, however, there is no derived form for the empty set.)
 %A third syntax dialect might come along that uses the same forms that $\mathcal{D}_2$ defines, but for ordered finite maps.

Schwerdfeger and Van Wyk have developed a modular grammar-based syntax definition system, implemented in Copper \cite{conf/gpce/WykS07}, that guarantees that determinism is conserved when syntax dialects (of a certain restricted class) are combined \cite{conf/pldi/SchwerdfegerW09,schwerdfeger2010context} as long as each constituent dialect prefixes all newly introduced forms with starting tokens drawn from disjoint sets. We will describe the difficulties that this requirement causes in Section \ref{sec:syntax-dialects}.


\subsubsection{Problem 2: Abstract Reasoning About Derived Forms}\label{sec:abs-reasoning-intro}
Even putting aside the difficulties of conservatively combining syntax dialects, there are questions about how \emph{reasonable}  sprinkling library-specific derived forms throughout a large software system might be. 
For example, consider the perspective of a programmer attempting to comprehend (i.e. reason about) the program fragment in Figure \ref{fig:K-dialect}, which is drawn under a syntax dialect constructed by combining a number of dialects of Standard ML's textual syntax.

\begin{figure}[h]
\begin{lstlisting}
val w = compute_w ()
val x = compute_x w
val y = {|(!R)@&{&/x!/:2_!x}'!R}|}
\end{lstlisting}
\caption{An example of unreasonable program text}
\label{fig:K-dialect}
\end{figure}

If the programmer happens to be familiar with the (intentionally terse) syntax of the stack-based database query processing language K \cite{Whitney:2001:LOR:376284.375783}, then Line 3 might pose few difficulties. If the programmer does not recognize this syntax, however, there are no simple, definitive protocols for answering questions like:
\begin{enumerate}
\item \textbf{(Responsibility)} Which constituent dialect defined the derived form that appears on Line 3?
\item \textbf{(Segmentation)} Are the characters \li{x} and \li{R} on Line 3 parsed as spliced expressions \li{x} and \li{R} (i.e. expressions of variable form), or parsed in some other way peculiar to this form?
\item \textbf{(Capture)} If \li{x} is in fact a spliced expression, does it refer to the binding of \li{x} on Line 2? Or might it capture an unseen binding introduced in the desugaring of Line 3?
\item \textbf{(Context Dependence)} If \li{w}, on Line 1, is renamed, could that possibly break the program, or change its meaning? In other words, might the desugaring of Line 3 assume that some variable identified as \li{w} is in scope (even though \li{w} is not mentioned in the text of Line 3)?
\item \textbf{(Typing)} What type does \li{y} have?
\end{enumerate}

In short, syntax dialects do not come with useful principles of \emph{syntactic abstraction}: if the desugaring of the program is held abstract, programmers can no longer reason about types and binding (i.e. answer questions like those above) in the usual disciplined manner. This is burdensome at all scales, but particularly when programming in the large, where it is common to encounter a program fragment drawn by another programmer, or drawn  long ago. Forcing the programmer to examine the desugaring of the drawing in order to reason about types and binding defeats the ultimate purpose of using syntactic sugar -- lowering cognitive cost (we expand on the notion of cognitive cost in Sec. \ref{sec:syntactic-properties}.) 


%In other words, encountering an unfamiliar derived form has made it difficult for the programmer to maintain the usual \emph{type discipline} and \emph{binding discipline}. %Compelling the programmer to examine the desugaring directly defeat the purpose of defining the derived form -- decreasing cognitive cost. Indeed, it substantially increases cognitive cost.

In contrast, when a programmer encounters, for example, a function call like the call to \li{compute_x} on Line 3, the analagous questions can be answered by following clear protocols that become ``cognitive reflexes'' after sufficient experience with the language, even if the programmer has no experience with the library defining \li{compute_x}:
\begin{enumerate}
\item The language's syntax definition determines that \li{compute_x w} is an expression of function application form.
\item Similarly, \li{compute_x} and \li{w} are definitively expressions of variable form.
\item The variable \li{w} can only refer to the binding of \li{w} on Line 1.
\item The variable \li{w} can be renamed without knowing anything about the value that \li{compute_x} stands for.
\item The type of \li{x} can be determined to be \li{B} by determining that the type of \li{compute_x} is \li{A -> B} for some \li{A} and \li{B}, and checking that \li{w} has type \li{A}. Nothing else needs to be known about the value that \li{compute_x} stands for. In Reynolds' words \cite{B304}:
\begin{quote}
\emph{Type structure is a syntactic discipline for enforcing levels of abstraction.}
\end{quote}
\end{enumerate}



% In summary, syntax dialects give library providers so much syntactic control that it creates problems for client programmers.
%A related issue arises when one works within a language with a module system, i.e. a system that supports interacting through a defined interface with various implementations of that interface. For example, consider different regular expression engines that differ only with regard to their performance in various circumstances, or different parser generators that accept the same class of grammar. Ideally, one would like to be able to define derived forms once such that they operate only through the common interface. To do so today requires both an awkward syntactic trick and coordination between library providers, as we will discuss in Sec. \ref{sec:syntax-examples-regexps}. Ideally, this would not be necessary.

%It is thus infeasible to simply allow different contributors to a software system to choose their own favorite dialect for each component they are responsible for. 
%It it clear that dialects are better rhetorical devices than practical engineering artifacts. 

%Due to this paucity of modular reasoning principles, the ``dialect-oriented'' approach is problematic for software development ``in the large''. %Large software projects and software ecosystems must pick a single language that does provide powerful modular reasoning principles and, to benefit from them, stay inside it.

% \subsection{Central Planning Considered Harmful}
% Dialects do sometimes have a less direct influence on large-scale software development: they can help convince the designers in control of comparatively popular languages, like OCaml and Scala, to include some variant of the primitives that they feature into backwards-compatible language revisions. %These decisions are increasingly influenced by community processes, e.g. the Scala Improvement Process.  %This approach concentrates power as well as responsibility over maintaining metatheoretic guarantees in the hands of a small group of language designers, though increasingly influenced by various community processes (e.g. the Scala Improvement Process). 
% %Dialects thus serve the role of rhetorical vehicles for new ideas, rather than direct artifacts. 
% %Over time, accepting such extensions has caused these languages to balloon in size. 
% This \emph{ad hoc} approach is unsustainable, for three main reasons. First, as we will demonstrate in Sec. \ref{sec:motivating-examples}, there are simply too  many potentially useful such primitives, and many of these capture idioms common only in relatively narrow application domains. It is unreasonable to expect language designers to be able to evaluate all of these use cases in a timely and informed manner. Second, primitives introduced earlier in a language's lifespan can end up monopolizing finite ``syntactic resources'', forcing subsequent primitives to use ever more esoteric forms. And third, primitives that prove after some time to be flawed in some way cannot be removed or modified without breaking backwards compatibility. For these reasons, language designers are justifiably reticent to add new primitives to major languages.%Because there is often no empirical data about how useful a construct is in practice until it is available in a major language, decisions about which constructs to include are often informed only by intuition (and are thus)
% %Recalling the words of  Reynolds, which are clearly as relevant today as they were almost half a century ago \cite{Reynolds70}: %This approach is antithetical to the ideal of a truly \emph{general-purpose language} described at the beginning of this section.
% %\newpage

%\subsection{Toward More Reasonable Primitives}
%These 
%This leaves two possible paths forward. One is to simply eschew ``niche'' derived forms and settle on the existing designs, which might be considered to sit at a ``sweet spot'' in the overall language design space (accepting that in some circumstances, this leads to  high cognitive cost). 


%Similarly, it recently introduced ``open datatypes'', which subsume its previous more specialized exception type, and captures many use cases for .

%Viewed ``dually'', one might equivalently ask for a language that builds in a core that is as small as possible, but provides expressive power comparable to languages with much larger cores. This is our goal in the work being proposed

%Similarly, it recently introduced ``open datatypes'', which subsume its previous more specialized exception type, and captures many use cases for .

%Viewed ``dually'', one might equivalently ask for a language that builds in a core that is as small as possible, but provides expressive power comparable to languages with much larger cores. This is our goal in the work being proposed. 

\subsection{Term Rewriting Systems}
An alternative approach that a library provider can consider when seeking to control syntactic cost is to leave the context-free syntax of the language fixed and instead contextually repurpose existing syntactic forms using a \emph{term rewriting system}. We will review various term rewriting systems in detail in Sec. \ref{sec:non-local-term-rewriting} and Sec. \ref{sec:macro-systems}. 

Na\"ive term rewriting systems suffer from problems analagous to those that plague syntax definition systems. In particular, it is difficult to conserve determinism, i.e. separately defined rewriting rules might attempt to rewrite the same term differently. Moreover, it can be difficult to determine which rewriting rule, if any, is responsible for a particular term, and to reason about types and binding given a drawing of a program subject to a large number of rewriting rules without examining the rewritten program.

Modern \emph{term-rewriting macro systems}, however, have made some progress toward addressing these problems. In particular:
\begin{enumerate}
\item Macro systems require that the client explicitly apply the intended rewriting (implemented by a macro) to the term that is to be rewritten, thereby addressing the problems of conflict and determining responsibility. However, it is often unclear whether a given macro  is repurposing the form of a given argument or sub-term thereof, as opposed to treating it parametrically by inserting it unmodified into the generated expansion. This is closely related to the problem of determining a {segmentation}, discussed above.
\item Macro systems that enforce \emph{hygiene}, which we will return to in Sec. \ref{sec:macro-systems}, address many of the problems related to reasoning about binding. 
\item The problem of reasoning about types has been relatively understudied, because most research on macro systems has been for languages in the Lisp tradition that lack rich static type structure \cite{mccarthy1978history}. That said, some progress has also been made on this front with the design of \emph{typed macro systems}, like Scala's macro system \cite{ScalaMacros2013}, where annotations constrain the macro arguments and the generated expansions.
\end{enumerate}

The main problem with term-rewriting macros, then, is that they afford library providers only limited syntactic control -- they must find creative ways to repurpose existing forms. For example, consider the  XHTML and K examples above. In both cases, the syntactic conventions are quite distinct from those of ML-like languages (and, for that matter, languages that use S-expression.) % Moreover, these existing forms normally have other meanings, so contextually repurposing them can be confusing \cite{pane1996usability}.

It is tempting in these situations to consider repurposing string literal forms. For example, we might wish to apply a macro \li{html!} (following Rust's convention of using a post-fix \li{!} to distinguish macro names from variables) to rewrite string literals containing Ur/Web-style XHTML syntax as follows:
\begin{lstlisting}[numbers=none]
  html! "SSTR<p>Hello, {[join " " [first, last]]}!</p>ESTR"
\end{lstlisting}

The problem here is that there is no way to extract the spliced expressions from the supplied string literal forms while satisfying the context independence condition, because variables that come from these spliced terms (e.g. \li{join}) are indistinguishable from variables that inappropriately appear free relative to the expansion. In addition, the problem of segmentation becomes even more pernicious: to a human or tool unaware of Ur/Web's syntax, it is not immediately apparent which particular subsequences of the string literals supplied to \li{html!} are segmented out as spliced expressions. Reader macros have essentially the same problem  \cite{DBLP:journals/jfp/FlattCDF12}.

\section{Contributions}\label{sec:contributions}
%%Our broad aim in the work being proposed is to introduce primitive language mechanisms that give library providers the ability to  express new syntactic expansions as well as new types and operators in a safe and modularly composable manner. 
%To summarize our motivating argument: the widespread proliferation of syntax dialects and syntax definition systems suggests that programmers value library-specific (a.k.a. domain-specific) syntactic sugar. However, the dialect-oriented approach seems to be incompatible with the best practices of programming in the large.  

This work introduces a system of \textbf{typed literal macros (TLMs)} that gives library providers substantially more syntactic control than existing typed term-rewriting macro systems while maintaining the ability to reason abstractly about types, binding and segmentation.% abstract reasoning principles. % comparable to the level of control they have when defining a syntax dialect.

Client programmers apply TLMs to \emph{generalized literal forms}. For example, in Figure \ref{fig:first-tsm-example} we apply a TLM named \li{#\dolla#html} to a generalized literal form delimited by backticks. TLM names are prefixed by \li{#\dolla#} to clearly distinguish TLM application from function application. The semantics delegates control over the parsing and expansion of each literal body to the applied TLM during a semantic phase called \emph{typed expansion}, which generalizes the usual typing phase. 
\begin{figure}[ht!]
\begin{lstlisting}[numbers=none,xleftmargin=0px]
$html `SURL<p>Hello, {[join ($str ' ') ($strlist [first, last])]}</p>EURL`
\end{lstlisting}
\caption[An example of a TLM being applied to a generalized literal form]{An example of a TLM being applied to a generalized literal form. The literal body, in green, is initially left unparsed according to the language's context-free syntax.}
% \vspace{-5px}
\label{fig:first-tsm-example}
\end{figure}

Generalized literal forms subsume a variety of common syntactic forms because the context-free syntax of the language only defines which outer delimiters are available. \emph{Literal bodies} (in green in Figure \ref{fig:first-tsm-example}) are otherwise syntactically unconstrained and left unparsed. For example, the \li{#\dolla#html} TLM is free to use an Ur/Web-inspired HTML syntax (compare Figure \ref{fig:first-tsm-example} to Figure \ref{fig:urweb}.) This choice is not imposed by the language definition. Generalized literal forms have no TLM-independent meaning.
% Because the context-free syntax is never extended, syntactic conflicts are not a concern.

% The semantics delegates control over the parsing and expansion of each literal body to the applied TLM during a semantic phase called \emph{typed expansion}, which generalizes the usual typing phase. %As such, the semantics can take the type and binding structure of the surrounding program into account when validating the expansion that the TLM programmatically generates to ensure that clients can answer critical questions related to types and binding, like those enumerated in Section \ref{sec:abs-reasoning-intro}. Clients need not have knowledge of the implementation of the TLM or of the generated expansion, i.e. there are useful principles of syntactic abstraction.

The primary technical challenge has to do with the fact that the applied TLM needs to be able to parse terms out of the literal body for inclusion in the expansion. We refer to these as \emph{spliced terms}. For example, Figure \ref{fig:first-tsm-example-marked} reveals the locations of the spliced expressions in Figure \ref{fig:first-tsm-example} by coloring them black. We have designed our system so that a figure like this, which presents a \emph{segmentation} of each literal body into spliced terms (in black) and characters parsed in some other way by the applied TLM (in color), can always be automatically generated no matter how each applied TLM has been implemented. 

\begin{figure}[h]
\begin{lstlisting}[numbers=none,xleftmargin=0px]
$html `SURL<p>Hello, {[EURLjoin ($str ' ') ($strlist [firstSCSS,ECSS last])SURL]}</p>EURL`
\end{lstlisting}
\caption{The segmentation of the example from Figure \ref{fig:first-tsm-example}}
\label{fig:first-tsm-example-marked}
\end{figure}

Notice that both arguments to \li{join} are themselves of TLM application form -- the TLMs named \li{#\dolla#str} and \li{#\dolla#strlist} are applied to generalized literal forms  delimited by quotation marks and square brackets, respectively. The bracket-delimited literal form, in turn, contains two spliced expressions of variable form -- \li{first} and \li{last}.
 
 % We design our mechanism such that these locations can easily be determined from the output of the TLM. This is essential for our hygiene mechanism, and it is also useful in that this information can be presented to the user (e.g. as shown in Figure \ref{fig:first-tsm-example-marked}). %As such, we must develop a mechanism where 1) the positions of spliced subterms can be determined without examining the macro implementation (e.g. so that they can be presented to the user differently by an editor or pretty-printer, ;  and 2) the hygiene mechanism must give only portions of the expansion that correspond to these spliced subterms access to the application site context. 



TLMs come equipped with useful principles of syntactic abstraction. We will more precisely characterize these abstract reasoning principles as we proceed. For now, to develop some intuitions, consider Figure \ref{fig:K-tsm-example}, which uses TLMs to express the ``unreasonable'' example from Figure \ref{fig:K-dialect}.
\begin{figure}[h]
\vspace{-3px}
\begin{lstlisting}[numbers=none,xleftmargin=0px]
  val w = compute_w ()
  val x = compute_x w
  val y = $kquery `SURL(!R)@&{&/EURLxSURL!/:2_!EURLxSURL}'!R}EURL`
\end{lstlisting}
\vspace{-5px}
\caption{TLMs make examples like the one from Figure \ref{fig:K-dialect} more reasonable.}
\vspace{-3px}
\label{fig:K-tsm-example}
\end{figure}

\noindent
Without examining the expansion of Line 3, we can reason as follows:

\begin{enumerate}
\item \textbf{(Responsibility)} The applied TLM, \li{$kquery}, is solely responsible for typed expansion of the literal body. 
\item \textbf{(Segmentation)} By examining the segmentation, we know that the two instances of \li{x} on Line 3 are parsed as spliced expressions, whereas \li{R} is parsed in some other way peculiar to this form.
\item \textbf{(Capture)} The system prevents capture, so the spliced expression \li{x} must refer to the binding of \li{x} on Line 2 -- it cannot capture an unseen binding introduced in the expansion of Line 3.
\item \textbf{(Context Dependence)} The system enforces context independence, so the expansion of Line 3 cannot  rely on the fact that, for example, \li{w} is in scope.
\item \textbf{(Typing)} An explicit type annotation on the definition of \li{$kquery} determines the type that every expansion it generates will have. We will see an example of a TLM definition in Chapter \ref{chap:uetsms}. 

Moreover, each segment in the segmentation also comes paired with the type it is expected to have. This information is usually not necessary to reason about typing, but it can be conveyed to the programmer upon request by the program editor if desired. %TLM definitions follow the usual scoping rules, so it is easy to ``jump to the definition'' of \li{$kquery}.
\end{enumerate}

% The primary technical challenge has to do with the fact that the applied TLM needs to be able to parse terms out of the literal body for inclusion in the expansion. We refer to these as \emph{spliced terms}. For example, Figure \ref{fig:first-tsm-example-marked} reveals the locations of the spliced expressions in Figure \ref{fig:first-tsm-example} by coloring them black. We have designed our system so that a figure like this, which presents a \emph{segmentation} of each literal body into spliced terms (in black) and characters parsed in some other way by the applied TLM (in green), can always be automatically generated no matter how each applied TLM has been implemented. 


% \begin{figure}[h]
% \begin{lstlisting}
% PElement Nil Seq(
% 	TextNode "Hello, ", 
% 	Seq(TextNode (join(" ", Cons(first, Cons(second, Nil)))), 
% 	TextNode "!"))
% \end{lstlisting}
% \caption{The desugaring.}
% \end{figure}


% There is also no ambiguity with regard to which TLM has control over each form, and searching for the definition of a TLM is no more difficult than searching for any other binding, i.e. there are well-defined scoping rules.

% In other words, TLMs maintain a useful notion of syntactic abstraction. %More specifically, TLMs maintain a \emph{hygienic binding discipline}, meaning that questions Questions 4 and 5 above were concerned with are disallowed entirely. 
% We will, of course, make this notion more technically precise as we continue.

\subsection{Outline}
% The remainder of this document is organized as follows.

After introducing necessary background material and summarizing the related work in greater detail in Chapter \ref{chap:background}, we formally introduce TLMs in Chapter \ref{chap:uetsms} by integrating them into a simple language of expressions and types. The introductory examples above can be expressed using the language introduced in Chapter \ref{chap:uetsms}. 

% In the remaining chapters, we enrich the language developed in Chapter \ref{chap:uetsms} with advanced features based on those found in full-scale general-purpose languages like ML and Scala, and enhance our TLM mechanism with these new features.

In Chapter \ref{chap:uptsms}, we add structural pattern matching to the language of Chapter \ref{chap:uetsms} and introduce \emph{pattern TLMs}, i.e. TLMs that generate patterns rather than expressions.

% We next develop a more sophisticated account of TLMs in Chapters \ref{chap:ptsms} and \ref{chap:static-eval}, with the aim of working out details necessary to integrate TLMs into full-scale general-purpose languages like ML or Scala. %These two chapters constitute Part \ref{part:parametric-tsms} of our contributions.

In Chapter \ref{chap:ptsms}, we equip the language of Chapter \ref{chap:uptsms} with type functions and an ML-style module system. We then introduce \emph{parametric TLMs}, i.e. TLMs that take type and module parameters. Parameters serve two purposes:
\begin{enumerate}
\item They enable TLMs that operate not just at a single type, but over a type- and module-parameterized family of types. For example, rather than defining a TLM \li{#\dolla#strlist} for string lists and another TLM \li{#\dolla#intlist} for integer lists, we can define a single parametric TLM \li{#\dolla#list} that operates uniformly across the type-parameterized family of list types. 
\item They allow the expansions that TLMs generate to refer to application site bindings in a context independent manner. 
\end{enumerate}
We also demonstrate support for partial parameter application in TLM abbreviations, which decreases the syntactic cost of this explicit parameter passing style. Figure \ref{fig:first-ptsm-example-marked} demonstrates all of these features.

\begin{figure}[h]
\begin{lstlisting}[numbers=none,xleftmargin=0px]
let syntax $strlist = $list string in 
$html `SURL<p>Hello, {[EURLjoin ($str ' ') ($strlist [firstSURL,EURL last])SURL]}</p>EURL`
\end{lstlisting}
\caption{The example from Figure \ref{fig:first-tsm-example-marked} expressed using parametric TLMs}
\label{fig:first-ptsm-example-marked}
\end{figure}

In these first chapters, we assume for the sake of technical simplicity that each TLM definition is self-contained, needing no access to libraries or to other TLMs. This is an impractical assumption in practice. We relax this assumption in Chapter \ref{chap:static-eval}, introducing a \emph{static environment} shared between TLM definitions. We also give examples of TLMs that are useful for defining other TLMs, e.g. TLMs that implement parser generators and quasiquotation.

%\item \textbf{Type-specific languages}, or \textbf{TSLs}. TSLs, described 
In Chapter \ref{chap:tsls}, we develop a mechanism of \emph{TLM implicits} that allows library clients to contextually designate, for any type, a privileged TLM at that type. The semantics applies this privileged TLM implicitly to unadorned literal forms that appear where a term of the associated type is expected. For example, if we designate \li{#\dolla#str} as the privileged TLM at the \li{string} type and \li{#\dolla#strlist} as the privileged TLM at the \li{list(string)} type, we can express the example from Figure \ref{fig:first-tsm-example-marked} instead as shown in Figure \ref{fig:first-tsm-example-implicit} (assuming \li{join} has type \li{string -> list(string) -> string}.) 
\begin{figure}[h]
\begin{lstlisting}[numbers=none]
$html`SURL<p>Hello, {[EURLjoin ' ' [firstSURL,EURL last]SURL]}</p>EURL`
\end{lstlisting}
\caption{The example from Figure \ref{fig:first-tsm-example-marked} drawn to take advantage of TLM implicits}
\label{fig:first-tsm-example-implicit}
\end{figure}

\noindent This approach is competitive in cost with library-specific syntax dialects (e.g. compare Figure \ref{fig:first-tsm-example-implicit} to Figure \ref{fig:urweb}), while maintaining the abstract reasoning principles characteristic of our approach. To further demonstrate the favorable economics of this approach, Figure \ref{fig:big-html-example} gives an example of a function that produces a value of type \li{html}. The body of this function assumes implicit TLM designations at seven different types (the unspliced segments are typeset in a color corresponding to the type that the enclosing literal form is being checked against.) This collection of TLMs, together with the mechanism for applying them implicitly, obviates the need for a web-programming-specific syntax dialect of our language like Ur/Web. An analysis of string literals used in open source projects discovered a wide variety of other examples like this \cite{TSLs}.

\begin{figure}[h]
\begin{lstlisting}[deletekeywords={for}, escapechar=@]
fun resultsFor(searchQuery : string, page : int) : html => 
  let imageBase : url = `SURIimages.example.comEURI` in 
  let bgImage : url = `SURI$EURIimageBaseSURI$/background.pngEURI` in 
  `SHTML<html>
  <head>
    <title>Search Results</title>
    <style>{EHTML{SCSS
      body { background-image: url({ECSSbgImageSCSS})} }
      .search { background-color: {ECSSdarken(`SCOLOR#aabbccECOLOR`, `SPCT10%EPCT`)SCSS} }
    ECSS}SHTML}</style>
  </head><body>
    <h1>Results for {[EHTMLsearchQuerySHTML]}</h1>
    <div class="search">
      Search again: {EHTMLsearchBox "SSTRGo!ESTR"SHTML}
    </div>
    {EHTMLformatResults (db, 
       `SSQLSELECT * FROM products WHERE {ESQLsearchQuerySSQL} iSHTMLEHTMLn titleESQL`,
       10, page)SHTML}
  </body>
  </html>EHTML`
\end{lstlisting}
\caption{A non-trivial example demonstrating implicit TLM application at seven different types: \li{SURIurlEURI}, \li{SHTMLhtmlEHTML}, \li{SCSScssECSS}, \li{SCOLORcolorECOLOR}, \li{SPCTpercentageEPCT}, \li{SSTRstringESTR} and \li{SSQLsqlESQL}}
\label{fig:big-html-example}
\end{figure}
%\item \textbf{Metamodules}, introduced in Sec. \ref{sec:metamodules}, reduce the need to primitively build in the type structure of constructs like records (and variants thereof),  labeled sums and other interesting constructs that we will introduce later by giving library providers programmatic ``hooks'' directly into the semantics, which are specified as a \emph{type-directed translation semantics} targeting a small \emph{typed internal language} (introduced in Sec. \ref{sec:VerseML}). %For example, a library provider can implement the type structure of records with a metamodule that:
%\begin{enumerate}
%\item introduces a type constructor, \lstinline{record}, parameterized by finite mappings from labels to types, and defines, programmatically, a translation to unary and binary product types (which are built in to the internal language); and 
%\item introduces operators used to work with records, minimally record introduction and elimination (but perhaps also various functional update operators), and directly implements the logic governing their typechecking and translation to the IL (which builds in only nullary and binary products). 
%\end{enumerate}
%We will see direct analogies between ML-style modules (which our mechanisms also support) and metamodules later.
%\end{enumerate} 


% As vehicles for this work, we will define a small programming language in each of the three parts just mentioned, each building conceptually upon the previous language. All of our formal contributions are relative to these small languages.

We conclude in Chapter \ref{chap:conclusion} with a discussion of the present limitations of TLMs, and outline various directions for future work.

\subsection{Thesis Statement}
In summary, this work defends the following statement:

\begin{quote}
A programming language (in the ML tradition) can give library providers the ability to %meta\-pro\-gram\-matic\-ally 
programmatically control the parsing and expansion of expressions and patterns of generalized literal form such that clients can reason abstractly about responsibility, segmentation, types and binding. %These  primitives are  expressive enough to subsume the need for a variety of primitives that are, or would need to be, built in to comparable contemporary languages.
\end{quote}

\section{VerseML}

The code examples in this document are written in a new full-scale functional language called VerseML.\footnote{We distinguish VerseML from Wyvern, which is the language described in our prior publications about some of the work that we will describe, because Wyvern is a group effort evolving independently.} VerseML is the language of Chapter \ref{chap:tsls}  extended with some additional conveniences that are commonly found in other functional languages (in particular, in the ML family of languages) and, notionally, orthogonal to TLMs (e.g. higher-rank polymorphism \cite{conf/icfp/DunfieldK13}, signature abbreviations, and syntactic sugar that is not library-specific, e.g. for curried functions.) %VerseML is, as its name suggests, a conceptual descendent of ML. It diverges from other dialects of ML that have a similar type structure in that it has a bidirectional type system \cite{Pierce:2000:LTI:345099.345100} (like, for example, Scala \cite{OdeZenZen01}) for reasons that have to do with the mechanism of TLM implicits described in Chapters \ref{chap:tsls} and \ref{chap:ptsms}. 
%The reason we will not follow Standard ML \cite{mthm97-for-dart} in giving a complete formal definition of VerseML in this work is both to emphasize that the primitives we introduce are ``insensitive'' to the details of the underlying type structure of the language (so TLMs can be considered for inclusion in a variety of languages, not only dialects of ML), and to avoid distracting the reader (and the author) with definitions that are already well-understood in the literature and that are orthogonal to those that are the focus of this work. 
We will not formally define these features mainly to avoid unnecessarily complicating our presentation with details that are not essential to the ideas introduced herein. As such, all examples written in VerseML should be understood to be informal motivating material for the subsequent formal material. %We anticipate that future full-scale language specifications will be able to combine the ideas  in the proposed work without trouble. %The purpose of the work being proposed is to serve as a reference for those interested in the new constructs we introduce, not to serve as a language specification. 
%We will give a brief overview of these languages are organized in Sec. \ref{sec:VerseML}.

%TLMs, like other macro systems, perform \emph{static code generation} (also sometimes called \emph{static} or \emph{compile-time metaprogramming}), meaning that the relevant rules in the static semantics of the language call for the evaluation of \emph{static functions} that generate term encodings. Static functions are functions that are evaluated statically, i.e. during typing. %Library providers write these static functions using the VerseML \emph{static language} (SL).  
%Maintaining a separation between the static (or ``compile-time'') phase and the dynamic (or ``run-time'') phase is an important facet of VerseML's design. % static code generation. %We will  also introduce a simple variant of each of these primitives that leverages VerseML's support for local type inference to further reduce syntactic cost in certain common situations. 

\section{Disclaimers}
Before we continue, it may be prudent to explicitly acknowledge that eliminating the need for syntax dialects would indeed be asking for too much: certain syntax design decisions are fundamentally incompatible with others or require coordination across a language design. We aim only to diminish the need for syntax dialects by finding a reasonable ``sweet spot'' in the design space, not to give control over all design decisions to library providers. %We summarize some of the situations that we explicitly do not consider here in Sec. \ref{sec:future-work}. % out a larger design space within a single language, VerseML.%a subset of constructs that can be specified by a semantics of a certain ``shape'' specified by VerseML (we will make this more specific later). %There is nothing ``universal'' about VerseML.

It may also be prudent to explicitly acknowledge that library providers could use TLMs  to define syntactic forms that are ``in poor taste.'' In practice, programmers should defer to established community guidelines before defining their own TLMs (following the example of languages that support operator overloading or \emph{ad hoc} polymorphism using type classes \cite{Hall:1996:TCH:227699.227700,conf/popl/DreyerHCK07}, which also have some potential for ``abuse'' or ``overuse''.) %For most programmers, using VerseML will not require explicitly defining a TLM on their own.%be substantially different from using a language like ML or one of its dialects. 
The majority of programmers should very rarely need to define a TLM on their own. The reasoning principles that we will develop ensure that even poorly designed TLMs cannot prevent clients from reasoning abstractly about the behavior of a program.

%Finally, VerseML is not designed as a dependently-typed language like Coq, Agda or Idris. %because these languages do not maintain a phase separation between ``compile-time'' and ``run-time.'' This phase separation is useful for programming tasks (where one would like to be able to discover errors before running a program, particularly programs that may have an effect) but less so for theorem proving tasks (where it is mainly the fact that a pure expression is well-typed that is of interest, by the propositions-as-types principle). 


\chapter{Background}\label{chap:background}
\vspace{-6px}
\begin{quote}\textit{The recent development of programming languages suggests that the simul\-taneous achievement of simplicity 
and generality in language design is a serious unsolved 
problem.}\begin{flushright}John Reynolds (1970) \cite{Reynolds70}\end{flushright}
\end{quote}
\vspace{-6px}
%VerseML, like most contemporary full-scale programming languages, has a textual concrete syntax (we will consider the topic of non-textual display forms as future work in Sec. \ref{sec:non-textual-display-forms}).
%\footnote{Although Wyvern specified a layout-sensitive concrete syntax, to avoid unnecessary distractions, we will describe a more conventional layout-insensitive concrete syntax for VerseML.} %We have chosen to specify a layout-sensitive textual concrete syntax (i.e. newlines and indentation are not ignored). This design choice is not  fundamental to our proposed contributions, but it will be useful for cleanly expressing a class of examples that we plan to discuss later. We plan to specify some novel aspects of VerseML's concrete syntax with an \emph{Adams grammar} \cite{Adams:2013:PPI:2429069.2429129} (such a specification for Wyvern, which has a very similar syntax, can be found in \cite{TSLs}), but for the purposes of this proposal, we will simply introduce VerseML's concrete syntax by example as we go on. %For constructs that have an obvious analog in ML, we will omit a detailed explanation.
%Because the purpose of concrete syntax is to serve as the programmer-facing user interface to the language, it is common practice to build in  derived syntactic forms (colloquially, \emph{syntactic sugar}) that capture common idioms more concisely (i.e. at lower \emph{syntactic cost}) or ``naturally'' (i.e. at lower \emph{cognitive cost}, which is usually considered qualitatively \cite{green1996usability}). % (i.e. considering cognitive dimensions \cite{green1996usability}). 
%For example, derived list syntax is built in to many functional languages, so that instead of having to write out 
%\begin{lstlisting}[numbers=none]
%Cons(1, Cons(2, Cons(3, Nil)))
%\end{lstlisting}
%the programmer can equivalently write \lstinline{[1, 2, 3]}. Many languages and dialects thereof go beyond this, building in derived syntax associated with various other types of data, like vectors (the SML/NJ dialect of SML), arrays (OCaml), monadic commands (Haskell), syntax trees (Scala, F\#), XML trees (Scala, Ur/Web) and SQL queries (F\#, Ur/Web). We will begin by describing these and several other examples in more detail in Sec. \ref{sec:motivating-examples}. %This is a rather \emph{ad hoc} process.% discussed previously, the usual approach is to require that the language designer build in new derived syntactic forms. %The desugaring from the latter to the former is specified by the language itself. %Typically, the language designer controls what forms of derived syntax are built in to the language.

%VerseML will take a less \emph{ad hoc} approach -- rather than privileging particular library constructs with primitive syntactic support, VerseML exposes primitives that allow library providers to introduce new expansion logic on their own, in a modular manner. Before describing these primitives in the remaining chapters, we will survey existing approaches to the problem of reducing syntactic cost (and in so doing, highlight some of the problems that we aim to resolve in our work) in Sec. \ref{sec:existing-approaches}. %Lists need no special consideration from the language specification.
%The purpose of this section is to  a and then introduce VerseML's syntax extension mechanisms. %For forms with a clear analogy to a form in Standard ML, we will assume the  semantics are analagous without providing details.

%We will begin in Sec. \ref{sec:examples} by detailing another example for which such a mechanism would be useful: regular expression (regex) patterns expressed using abstract data types. We will refer to this example throughout the proposal. In Sec. \ref{sec:syntax-existing}, we discuss how the usual approach of using dynamic string parsing to introduce regex patterns is not ideal. We also survey existing alternatives to dynamic string parsing, finding that they involve an unacceptable loss of modularity and other undesirable trade-offs. In Secs. \ref{sec:tsms} and \ref{sec:tsls}, we introduce our proposed alternatives -- \emph{typed literal macros} (TLMs) and the related \emph{type-specific languages} (TSLs) -- and discuss how they resolve these issues (as well as some limitations that they have). We  also give an overview of how TLMs are formally specified. We give a concrete timeline for the remaining work in Sec. \ref{sec:syntax-timeline}, and conclude in Sec. \ref{sec:conclusion}.
% !TEX root = omar-thesis.tex

\section{Preliminaries}\label{sec:preliminaries}
\vspace{-3px}
This work is rooted in the tradition of full-scale functional languages like Standard ML, OCaml and Haskell (as might have been obvious from Chapter \ref{chap:intro}.) Familiarity with basic concepts in these languages, e.g. variables, types, polymorphic and recursive functions, tuples, records, recursive datatypes and structural pattern matching, is assumed throughout this work. Readers who are not familiar with these concepts are encouraged to consult the early chapters of an introductory text like Harper's \emph{Programming in Standard ML} \cite{harper1997programming} (a working draft can be found online.) We discuss integrating TLMs into languages from other design traditions in Sec. \ref{sec:integration}.

In Chapter \ref{chap:ptsms} and onward, as well as in some of the motivating examples below, we also assume basic familiarity with ML-style module systems. Readers with experience in a language without such a module system (e.g. Haskell) are also advised to consult the relevant chapters in \emph{Programming in Standard ML} \cite{harper1997programming} as needed. We distinguish \emph{modules}, which are language constructs, from \emph{libraries}, which are extralinguistic packaging constructs managed by some implementation-defined compilation manager (e.g. \li{CM}, distributed with Standard ML of New Jersey (SML/NJ) \cite{DBLP:conf/plilp/AppelM91}.) A library can export modules, signatures and TLM definitions. We return to this distinction in Chapter \ref{chap:static-eval}.

The formal systems that we will consider are defined within the metatheoretic framework of type theory. More specifically, we will assume that abstract binding trees (ABTs, which enrich abstract syntax trees with the notions of binding and scope, as discussed in Chapter \ref{chap:intro}), renaming, alpha-equivalence, substitution, structural induction and rule induction are defined as described in Harper's \emph{Practical Foundations for Programming Languages, Second Edition} (\emph{PFPL}) \cite{pfpl}. Familiarity with other formal accounts of type systems, e.g. Pierce's \emph{Types and Programming Languages} (\emph{TAPL}) \cite{tapl}, should also suffice.% This document is organized so as to be readable even if the sections defining formal systems are skipped entirely, although much precision will, of course, be lost.


% !TEX root = omar-thesis.tex

\section{Cognitive Cost}\label{sec:syntactic-properties}
\begin{quote}
\emph{In the present inquiry, the idea is to adopt a much
wider conception of formal languages so as to investigate more broadly what
exactly is going on when a reasoner puts these tools to use.}

\begin{flushright}Catarina Dutilh Novaes\\
\emph{Formal Languages in Logic: A Philosophical and Cognitive Analysis} \cite{novaes2012formal}
\end{flushright}
\end{quote}

Central to our motivations is the notion that different drawings of a formal structure can and should be distinguished on the basis of the  \emph{cognitive costs} that humans incur as they interact with them. 

The broad notion of cognitive cost must ultimately be understood intuitively, relating as it does to the complexities of the human mind. Cognitive cost is also fundamentally a \emph{subjective} and \emph{situational} notion. 
As such, researchers cannot develop a truly comprehensive formal framework capable of settling questions of cognitive cost.\footnote{The fact that cognitive cost cannot be comprehensively characterized seems itself to create a cognitive hazard, in that those of us who favor comprehensive formal frameworks sometimes devalue or dismiss concerns related to cognitive cost, or consider them in an overly \emph{ad hoc} manner. This tendency must be resisted if programming language design is to progress as a human-oriented design discipline.} However, there are several situationally useful frameworks worth briefly reviewing \cite{box1987empirical}. % operationalize cognitive cost in a simpler and more tractable manner %These can serve as satisfying proxies in many circumstances. %In order to ground this concept, it is common for researchers to  operationalize this notion in order to simplify the arguments that they are making. 

% Notions of cognitive cost can perhaps be understood by informal analogy to notions of \emph{dynamic cost}, which distinguish semantically equivalent expressions based on their consumption of various resources, e.g. time or memory, as they are evaluated. Notions of cognitive cost analagously capture the consumption of human attentional resources as they are being drawn and examined by a human. Human attention resources are, of course, more difficult to quantify.


One useful quantitative framework reduces cognitive cost to \emph{syntactic cost}, which is measured by counting characters (or glyphs, more generally.) This is often a satisfying proxy for cognitive cost, in that smaller drawings are often easier to comprehend and produce. For example, the drawing \li{[x, y, z]} has lower syntactic cost than its desugaring, as discussed in the previous chapter. There is a limit to this approximation, of course. For example, one might argue that the drawings involving the syntax of K, like the drawing from Figure \ref{fig:K-dialect}, have high cognitive cost, despite their low syntactic cost, until one is experienced with the syntax of K. In other words, the relationship between syntactic cost and cognitive cost depends on the subject's progression along some \emph{learning curve}.

A related quantity of interest to human programmers is \emph{edit cost}, measured relative to a program editor as the minimum number of primitive edit actions that must be performed to produce a drawing. For example, when using a text editor (as most professional programmers today do), drawings in textual form typically have lower edit cost, as measured by the minimum number of keystrokes necessary to produce the drawing, than those in operational or stylized forms (indeed, some drawings in stylized form can be understood to have infinite text edit cost.) Edit cost can be modeled using, for example, \emph{keystroke-level models} (KLMs) as described by Card, Moran and Newell \cite{journals/cacm/CardMN80}.%which, for software developers, is their primary mode of interaction with a programming language.%Our choice might also be influenced (or determined) by the tool that we are using to write the program. In particular, stylized forms are suitable for use when typesetting a program, whereas textual forms are necessary for writing programs using a text editor for consumption by an implementation of the semantics on a computer. 

One can also analyze cognitive cost using disciplined qualitative methods. Green's \emph{Cognitive Dimensions of Notations} \cite{Green89,green1996usability} and Pane and Myers' \emph{Usability Issues} \cite{pane1996usability} (both of which synthesized much of the earlier work in the area) are highly cited heuristic frameworks. For example, Green's cognitive dimensions framework gives us a common vocabulary for  comparing the derived list forms described in Chapter \ref{chap:intro} to the primitive list forms. In particular, the derived list forms \emph{map more closely} to other notations used for sequences of elements (e.g. in typeset mathematics, or on a physical notepad) than the primitive list forms. They also make the elements of the list more clearly \emph{visible}, in that the identifier \li{Cons} is not interspersed throughout the term, and they have lower \emph{viscosity} because adding a new item to the middle of a list drawn in derived form requires only a local edit, whereas for a list constructed by applying list constructors in prefix position, one needs also to add a closing parenthesis to the end of the term. (Infix operators for lists, discussed in Sec. \ref{sec:Fixity-directives}, also have low viscosity.)

Finally, one might consider cognitive cost comparatively using quantitative empirical methods, e.g. by conducting randomized control trials to compare forms with respect to task completion time or error rate (for satisfyingly representative tasks.) Stefik et al. have performed many such studies, mainly on novice programmers (these are summarized, along with other such studies, in \cite{journals/jeric/StefikS13}.) Kaijanaho provides another review of evidence-based language design methodologies \cite{kaijanaho2015evidence}.

Our goal in this work is to provide a means by which library providers can introduce alternative syntactic forms of their own design. We leave it up to each library provider to establish the cognitive costs associated with the alternative forms that they introduce, according to whichever operationalization of the concept that they favor. For the examples in this document, we will mainly utilize syntactic cost, because claims about syntactic cost can be evaluated quantitatively. In a few cases, we also make heuristic arguments. 

We claim also that the abstract reasoning principles that TLMs come equipped with serve to limit cognitive costs that a client programmer that encounters an unfamiliar form would otherwise incur when attempting to reason about types and binding. This claim follows from the intuitive assumption that examining only type annotations is less costly than examining the full expansion of an unexpanded term and the logic that produced that expansion. 

% There is much that remains to be understood about cognitive cost, particularly when the subject is an experienced programmer. Many of the difficulties that we will confront in this work have to do with the fact that allowing programmers to add new derived forms unconstrained to a syntax definition can decrease cognitive cost ``in the small'', i.e. for programmers who understand all of the details of the newly introduced desugaring transformations, while increasing cognitive cost ``in the large'' because programmers have few clear modular reasoning principles that they can rely on when they encounter an unfamiliar form. Our aim is to control cognitive cost at all scales. % (Indeed, many of challenges of programming language design might be said to have this flavor.)% Our contributions, however, are not directly in this area, so we will stop here. 

%Put another way, the stylized and textual forms are preferrable when designing a \emph{user interface} of our programming language.

\section{Motivating Definitions}\label{sec:motivating-examples}
In this section, we give a number of VerseML definitions that will serve as the basis for many subsequent examples. This section also serves as an introduction to the textual syntax and semantics of VerseML.

\subsection{Lists}\label{sec:lists}
The Standard ML Basis Library (i.e.  the standard library) defines list types as follows:
\begin{lstlisting}[numbers=none]
datatype 'a list = nil | op:: of 'a * 'a list
\end{lstlisting}
This datatype declaration generates:
\begin{itemize}
\item a type function \li{list} that takes one type parameter; 
\item the value constructors \li{nil : 'a list} and \li{op:: : 'a * 'a list -> 'a list}; and
\item the corresponding list pattern constructors \li{nil} and \li{op::}.
\end{itemize}
We will return to the significance of the identifier \li{op::} in Sec. \ref{sec:Fixity-directives} below.

VerseML does not support SML-style datatype declarations directly. Instead, type functions, recursive types, sum types, product types, value constructors, pattern constructors and type generativity arise through orthogonal mechanisms, as in foundational accounts of these concepts (e.g. \emph{PFPL} \cite{pfpl}.) This is mainly for pedagogical purposes -- it will take until Chapter \ref{chap:ptsms} to build up all of the machinery that would be necessary to integrate TLMs into a language with SML-style datatype declarations. By exposing more granular primitives, we can define sub-languages of VerseML in Chapter \ref{chap:uetsms} and Chapter \ref{chap:uptsms} that communicate certain fundamental ideas more clearly and generally.

With that in mind, the family of list types are defined in VerseML as follows:
\begin{lstlisting}[numbers=none]
type list('a) = rec(self => Nil + Cons of 'a * self)
\end{lstlisting}
Here, \li{list} is a {type function} binding its type parameter to the type variable \li{'a}. We apply parameters in post-fix position (rather than in prefix position, as in SML.) For example, the type of integer lists is \li{list(int)}. This is equivalent, by substitution of \li{int} for \li{'a} on the right side of the definition above, to the following \emph{recursive type}:
\begin{lstlisting}[numbers=none]
rec(self => Nil + Cons of int * self)
\end{lstlisting}
%Here, the type variable \li{self} is bound as a \emph{self reference} in the type body. 

The values of a recursive type \li{T} are \li{fold(e)}, where \li{e} is a value of the \emph{unrolling} of \li{T}. The {unrolling} of a recursive type is determined by substituting the recursive type itself for the self reference in its type body. For example, the unrolling of \li{list(int)} is equivalent, by substitution of \li{list(int)} for \li{self}, to the following \emph{labeled sum type}:
\begin{lstlisting}[numbers=none]
Nil + Cons of int * list(int)
\end{lstlisting}
The values of a labeled sum type \li{T} are injections \li{inj[Lbl](e)}, where \li{Lbl} is a label specified by one of the classes specified by \li{T} and \li{e} is a value of the corresponding type. The {labeled sum type} above specifies two {classes}:
\begin{enumerate}
\item One class, labeled \li{Nil}, takes values of \li{unit} type (we can omit \li{of unit}.) The only value of \li{unit} type is the trivial value \li{()}.  
\item The other class, labeled \li{Cons}, takes values of the \emph{product type} \li{int * list(int)}, the values of which are tuples. 
\end{enumerate}

Let us now define two example values of type \li{list(int)}:
\begin{lstlisting}[numbers=none]
val nil_int : list(int) = fold(inj[Nil] ())
val one_int : list(int) = fold(inj[Cons] (1, nil_int))
\end{lstlisting}
Here, \li{nil_int} is the empty list and \li{one_int} is a list containing a single integer, \li{1}. %We omitted the type ascriptions on the folds and injections because VerseML can infer them.

One way to lower syntactic cost is to define the following polymorphic values, called the \emph{list value constructors}, which abstract away the necessary folds and injections:
\begin{lstlisting}[numbers=none]
val Nil : list('a) = fold(inj[Nil] ())
fun Cons(x : 'a * list('a)) : list('a) => fold(inj[Cons] x)
\end{lstlisting}
In fact, VerseML generates constructors like these automatically.\footnote{A more general mechanism that allows values to be generated from type definitions is beyond the scope of our work on TLMs.} 
Using these list value constructors, we can equivalently express the values above as follows:
\begin{lstlisting}[numbers=none]
val nil_int : list(int) = Nil
val one_int = Cons (1, Nil)
\end{lstlisting}
In SML, constructors like these are the only means by which a value of a datatype can be introduced -- folding and injection operators are not exposed directly to programmers. As such, it is not possible to construct a value of a type like \li{list(int)} in a context-independent manner, i.e. in contexts where the value constructors have been shadowed or are not bound. This will become relevant in the next section and in Chapter \ref{chap:uetsms}. %In Chapter \ref{chap:ptsms}, we will introduce the machinery that would be necessary to take the SML-style approach and suppress mention of \li{fold} and \li{inj} operators entirely.

Values of recursive type, labeled sum type and product type are deconstructed by pattern matching. %\footnote{Readers who are not familiar with structural pattern matching may wish to consult the introduction to Chapter \ref{chap:uptsms} for a somewhat more detailed description.} 
For example, we can write the polymorphic map function, which constructs a  list by applying a given function to each item in a given list, as follows:
\begin{lstlisting}[numbers=none]
fun map (f : 'a -> 'b) (xs : list('a)) : list('b) => 
  match xs with 
  | fold(inj[Nil] ()) => Nil
  | fold(inj[Cons] (y, ys)) => Cons (f y, map f ys)
  end
\end{lstlisting}


The primitive pattern forms above are drawn like the corresponding primitive value forms (though it is important to keep in mind that the syntactic overlap is superficial -- patterns and expressions are distinct sorts of trees.) To lower syntactic cost, VerseML automatically inserts folds, injections and trivial arguments into patterns of constructor form, i.e. those of the form \li{Lbl} and \li{Lbl p} where \li{Lbl} is a capitalized label and \li{p} is another pattern:\footnote{Pattern TLMs, introduced in Chapter \ref{chap:uptsms}, could be used to manually achieve a similar syntax for any particular type, or in Chapter \ref{chap:ptsms}, across a particular family of types, but because this syntactic sugar is useful for all recursive labeled sum types, we build it primitively into VerseML.}
\begin{lstlisting}[numbers=none]
fun map (f : 'a -> 'b) (xs : list('a)) : list('b) => 
  match xs with 
  | Nil => Nil 
  | Cons (y, ys) => Cons (f y, map f ys)
  end
\end{lstlisting}
%To avoid syntactic ambiguity, variables must not begin with an uppercase letter.

We group the type and value definitions above, as well as some other standard utility functions like \li{append}, into a \emph{module} \li{List : LIST}, where \li{LIST} is the \emph{signature} defined in Figure \ref{fig:LIST}. These definitions are not privileged in any way by the language definition. In particular, there are no list-specific derived forms built in to the textual syntax of VerseML. We will show how TLMs allow programmers to achieve a similar syntax for lists over the next several chapters.

\begin{figure}[h!]
\begin{lstlisting}[numbers=none]
signature LIST = 
sig 
  type list('a) = rec(self => Nil + Cons of 'a * self)
  val Nil : list('a)
  val Cons : 'a * list('a) -> list('a)
  val map : ('a -> 'b) -> list('a) -> list('b)
  val append : list('a) -> list('a) -> list('a)
  (* ... *)
end
\end{lstlisting}
\caption{Definition of the \li{LIST} signature}
\label{fig:LIST}
\end{figure}

\subsection{Regular Expressions}\label{sec:syntax-examples-regexps}
A regular expression, or \emph{regex}, is a description of a \emph{regular language} \cite{Thompson:1968:PTR:363347.363387}. Regexes arise with some frequency in fields like natural language processing and bioinformatics.

\paragraph{Recursive Sums}
One way to encode regular expressions in VerseML is as values of the recursive labeled sum type abbreviated \li{rx} in Figure \ref{fig:datatype-rx}.

\begin{figure}[h]
\begin{lstlisting}[numbers=none]
type rx = rec(rx => Empty + Str of string + Seq of rx * rx +
                    Or of rx * rx + Star of rx)
\end{lstlisting}
\caption{Definition of the recursive labeled sum type \li{rx}}
\label{fig:datatype-rx}
\end{figure}
Assuming the automatically generated value constructors as in Sec. \ref{sec:lists}, we can construct a regex that matches the strings \li{"SSTRAESTR"}, \li{"SSTRTESTR"}, \li{"SSTRGESTR"} or \li{"SSTRCESTR"} (i.e. DNA bases) as follows:
\begin{lstlisting}[numbers=none]
Or(Str "SSTRAESTR", Or(Str "SSTRTESTR", Or(Str "SSTRGESTR", Str "SSTRCESTR")))
\end{lstlisting}

Given a value of type \li{rx}, we can deconstruct it by pattern matching, again as in Sec. \ref{sec:lists}. For example, the function \li{is_dna_rx} defined in Figure \ref{fig:is_dna_rx} detects regular expressions that match DNA sequences.

\begin{figure}[h]
\begin{lstlisting}[numbers=none]
fun is_dna_rx(r : rx) : boolean => 
  match r with 
  | Str "SSTRAESTR" => True
  | Str "SSTRTESTR" => True
  | Str "SSTRGESTR" => True
  | Str "SSTRCESTR" => True
  | Seq (r1, r2) => (is_dna_rx r1) andalso (is_dna_rx r2)
  | Or  (r1, r2) => (is_dna_rx r1) andalso (is_dna_rx r2)
  | Star(r') => is_dna_rx r'
  | _ => False 
  end
\end{lstlisting}
\caption{Pattern matching over regexes in VerseML}
\label{fig:is_dna_rx}
\end{figure}


\paragraph{Abstract Types} Encoding regexes as values of type \li{rx} is straightforward, but there are reasons why one might not wish to expose this encoding to clients directly. 

First, regexes are usually identified up to their reduction to a normal form. For example, \li{Seq(Empty, Str "SSTRAESTR")} has normal form \li{Str("SSTRAESTR")}. It can be useful for regexes with the same normal form to be  indistinguishable from the perspective of client code. (The details of regex normalization are not important for our purposes, so we omit them.)

Second, it can be useful for performance reasons to maintain additional data alongside each regex (e.g. a corresponding finite automaton.) In fact, there may be many ways to represent regexes, each with different performance trade-offs, so we would like to provide a choice of representations behind a common interface.

To achieve these goals, we turn to the VerseML module system, which is based directly on the SML module system \cite{mthm97-for-dart,dreyer2005understanding} (which originates in early work by MacQueen \cite{MacQueen:1984:MSM:800055.802036}.) In particular, let us define the {signature} abbreviated \li{RX} in Figure \ref{fig:signature-RX}.
%Notice that it exposes an interface otherwise  to the one available using a case type:

\begin{figure}[ht]
\begin{lstlisting}[deletekeywords={case}]
(* abstract regex unfoldings *)
type u('a) = UEmpty + UStr of string + USeq of 'a * 'a + 
             UOr of 'a * 'a + UStar of 'a

signature RX = 
sig
  type t (* abstract *)

  (* constructors *)
  val Empty : t
  val Str : string -> t
  val Seq : t * t -> t
  val Or : t * t -> t
  val Star : t -> t

  (* produces the normal unfolding *)
  val unfold_norm : t -> u(t)
end

module R1 : RX = struct (* ... *) end
module R2 : RX = struct (* ... *) end
\end{lstlisting}
\vspace{-5px}
\caption{The \lstinline{RX} signature and two example implementations}
\label{fig:signature-RX}
\end{figure}

The clients of any module \lstinline{R} that has been sealed by \lstinline{RX}, e.g. \li{R1} or \li{R2}  in Figure \ref{fig:signature-RX}, manipulate regexes as values of type \li{R.t} using the interface specified by \li{RX}. For example, a client can construct a regex matching DNA bases by projecting the value constructors out of \li{R} and applying them as follows:
\begin{lstlisting}[numbers=none]
R.Or(R.Str "SSTRAESTR", R.Or(R.Str "SSTRTESTR", R.Or (R.Str "SSTRGESTR", R.Str "SSTRCESTR")))
\end{lstlisting}

Because the identity of the representation type \lstinline{R.t} is held abstract by the signature, the only way for a client to construct a value of this type is through the values that \li{RX} specifies (i.e. we have defined an \emph{abstract data type (ADT)}  \cite{liskov1974programming}.) Consequently, representation invariants need only be established locally within each module.




Similarly, clients cannot interrogate the structure of a value \li{r : R.t} directly. Instead, the signature specifies a function \li{R.unfold_norm} that produces the \emph{normal unfolding} of a given regex, i.e. a value of type \li{u(R.t)} that exposes only the outermost form of the regex in normal form (this normal form invariant is specified only as an unenforced side condition that implementations are expected to obey, as is common practice in languages like ML.) Clients can pattern match over the {normal unfolding} in the familiar manner, as shown in Figure \ref{fig:is_dna_rx_prime}. 
\begin{figure}
\begin{lstlisting}[numbers=none]
fun is_dna_rx'(r : R.t) : boolean => 
  match R.unfold_norm r with 
  | UStr "SSTRAESTR" => True
  | UStr "SSTRTESTR" => True
  | UStr "SSTRGESTR" => True
  | UStr "SSTRCESTR" => True
  | USeq (r1, r2) => (is_dna_rx' r1) andalso (is_dna_rx' r2)
  | UOr (r1, r2) => (is_dna_rx' r1) andalso (is_dna_rx' r2)
  | UStar r' => is_dna_rx' r'
  | _ => False
  end
\end{lstlisting}
\vspace{-5px}
\caption{Pattern matching over normal unfoldings of regexes}
\label{fig:is_dna_rx_prime}
\end{figure}

The normal unfolding suffices in situations where a client needs to examine only the outermost structure of a regex. However, in general, a client may want to pattern match more deeply into a regex. There are various ways to approach this problem. 

One approach is to define auxiliary functions that construct $n$-deep unfoldings of \li{r}, where $n$ is the deepest level at which the client wishes to expose the normal structure of the regex. For example, it is easy to define a function \li{unfold_norm2 : R.t -> u(u(R.t))} in terms of \li{R.unfold_norm} that allows pattern matching to depth $2$.\footnote{Defining an unfolding \emph{generic} in $n$ is a more subtle problem that is beyond the scope of this work.} 

Another approach is to \emph{completely unfold} a value of type \li{t} by applying a function \li{view : R.t -> rx} that recursively applies \li{R.unfold_norm} to exhaustion. The type \li{rx} was defined in Figure \ref{fig:datatype-rx}.  Computing the complete unfolding (also called the \emph{view}) can have higher dynamic cost than computing an incomplete unfolding of appropriate depth, but it is also a simpler approach (i.e.   lower cognitive cost can justify higher dynamic cost.)


\begin{figure}[t]
\begin{lstlisting}[numbers=none]
functor RXUtil(R : RX) = 
struct
  fun unfold_norm2(r : R.t) : u(u(R.t)) => (* ... *)

  fun view(r : R.t) : rx => 
    match R.unfold_norm r with 
    | UEmpty => Empty
    | UStr s => Str s
    | USeq (r1, r2) => Seq (view r1, view r2)
    | UOr (r1, r2) => Or (view r1, view r2)
    | UStar r => Star (view r)
    end 

  (* ... *)
end
\end{lstlisting}
\vspace{-5px}
\caption{The definition of \li{RXUtil}}
\vspace{-5px}
\label{fig:RXUtil}
\end{figure}
Typically, utility functions like \li{unfold_norm2} and \li{view} are defined in a \emph{functor} (i.e. a function at the level of modules) like \li{RXUtil} in Figure \ref{fig:RXUtil}, so that they need only be defined once, rather than separately for each module \li{R : RX}. The client can instantiate the functor by applying it to their choice of module as follows:
\begin{lstlisting}[numbers=none]
module RU = RXUtil(R)
\end{lstlisting}
% \subsection{Lists, Sets, Maps, Vectors and Other Containers}\label{sec:syntax-examples-containers}
% \todo{write this (Spring 2016)}
% \subsection{HTML and Other Web Languages}\label{sec:syntax-examples-html}
% \subsection{Dates, URLs and Other Standardized Formats}\label{sec:syntax-examples-dates}
% \subsection{Query Languages} The language of regular expressions can be considered a query language over strings. There are many other query languages that focus on different types of data, e.g. XQuery for XML trees, or that are associated with various database technologies, e.g. SQL for relational databases. \todo{finish this (Spring 2016)} 
% \subsection{Monadic Commands}\label{sec:syntax-examples-monads}
% \todo{write this; cite Bob's blog (Spring 2016)}

% \todo{http://www.cs.umd.edu/~mwh/papers/monadic.pdf}
% \subsection{Quasiquotation and Object Language Syntax}\label{sec:syntax-examples-quasiquotation}
% \todo{write this (Spring 2016)}
% \subsection{Grammars}\label{sec:syntax-examples-grammars}
% \todo{write this (Spring 2016)}
% \subsection{Mathematical and Scientific Notations}\label{sec:syntax-examples-math-science}
% \subsubsection{SMILES: Chemical Notation}
% \todo{write this; cite SMILES \url{https://en.wikipedia.org/wiki/Simplified_molecular-input_line-entry_system} (Spring 2016)}
% \subsubsection{\TeX~Mathematical Formula Notation}
% \todo{write this (Spring 2016)}

% \subsection{Others}

% Get examples from: \url{http://voelter.de/data/pub/mbeddr-cs-oopsla2015-preprint.pdf}


% !TEX root = omar-thesis.tex
\section{Existing Approaches}\label{sec:existing-approaches}
The definitions in the previous section adequately encode the semantics of lists and regular expressions, but they are not particularly convenient. Our task in this section is to consider various mechanisms of syntactic control, i.e. mechanisms that can be deployed to help to decrease the syntactic cost of expressions and patterns involving these constructs (without changing their meaning.)

% There are many approaches that a library provider might consider. These differ with regard to how much control they give the library provider over syntactic form, and the broader cognitive burdens that they impose on client programmers, particularly when  programming in the large. 

We begin in Sec. \ref{sec:standard-abstraction-mechanisms} by considering standard abstraction mechanisms available in languages like ML. We then consider a system of dynamic quotation parsing available in some dialects of ML in Sec. \ref{sec:dynamic-quotation}. 

These methods give library providers only limited control over form and operate at ``run-time.'' To gain more precise control over form at ``compile-time'', a library provider, or another interested party, can define a ``library-specific'' syntax dialect using a \emph{syntax definition system}. The next several sections consider various syntax definition systems:
\begin{itemize}
\item In Sec. \ref{sec:Fixity-directives}, we consider infix operator definition systems.
\item In Sec. \ref{sec:mixfix}, we consider somewhat more expressive mixfix systems.
\item In Sec. \ref{sec:grammars}, we consider grammar-based syntax definition systems.
\item In Sec. \ref{sec:parser-combinators}, we consider parser combinator systems.
\end{itemize}
The systems in Sec. \ref{sec:grammars} and Sec. \ref{sec:parser-combinators} give essentially complete control over form to their users. We give examples of dialects that can be constructed using these systems in Sec. \ref{sec:examples-of-syntax-dialects}. Then, in Sec. \ref{sec:problems-with-syntax-dialects}, we discuss the difficulties that programmers can expect to encounter if they  use these systems when programming in the large (as a follow-up to what was discussed in Section \ref{sec:problems-with-dialects}.)

An alternative approach is to leave the syntax of the language fixed but allow programmers to contextually repurpose existing forms using a \emph{term rewriting system}. We consider {non-local term rewriting systems} in Sec. \ref{sec:non-local-term-rewriting} and {local term rewriting systems}, which are also known as \emph{macro systems}, in Sec. \ref{sec:macro-systems}.

% \subsection*{More General Syntax Definition Systems}\label{sec:syntax-dialects}
% More general syntax definition systems give users more direct control over form than the infix and mixfix systems just described. In the next two subsections, we will begin by describing some notable systems, then give two examples that demonstrate their expressive power. We will 




\subsection{Standard Abstraction Mechanisms}\label{sec:standard-abstraction-mechanisms}
The simplest way to decrease syntactic cost is to capture idioms using the standard abstraction mechanisms of our 
language, e.g. functions and modules. 

We already saw examples of this approach in the previous section. For example, we defined the list value constructors, which capture the idioms of list construction. Such definitions are common enough that VerseML generates them automatically. 
We also defined a utility functor for regexes, \li{RXUtil}, in Figure \ref{fig:RXUtil}. As more idioms involving regexes arise, the library provider can capture them by adding additional definitions to this functor. For example, the library provider might add the definition of a value that matches single digits to \li{RXUtil} as follows:
\begin{lstlisting}[numbers=none]
val digit = R.Or(R.Str "SSTR0ESTR", R.Or(R.Str "SSTR1ESTR", ...))
\end{lstlisting}
Similarly, the library provider might define a function \li{repeat : R.t -> int -> R.t} that constructs a regex by sequentially repeating the given regex a given number of times (not shown.) 
Using these definitions, a client can define a regex that matches U.S. social security numbers (SSNs) as follows:
\begin{lstlisting}[numbers=none]
val dash = R.Str "SSTR-ESTR"
val repeat_d = RU.repeat RU.digit
val ssn = R.Seq(repeat_d 3, R.Seq(dash, R.Seq(repeat_d 2, 
                R.Seq(dash, repeat_d 4))))
\end{lstlisting}
The syntactic cost of this program fragment is lower than the syntactic cost of the equivalent program fragment that applies  the regex value constructors directly. 

One limitation of this approach is that there is no standard way to capture idioms at the level of patterns. Pattern synonyms have been informally explored in some languages, e.g. in an experimental extension of Haskell implemented by GHC \cite{GHC-extensions} and in the $\Omega$mega language \cite{DBLP:conf/cefp/SheardL07}, but these are limited in that arbitrary computations cannot be performed.

Another limitation is that this approach does not give library providers control over form. For example, we cannot ``approximate'' SML-style derived list forms using only auxiliary values like those above. 
Similarly, consider the textual syntax for regexes defined in the POSIX standard \cite{STD95954}. Under this syntax, the regex that matches DNA bases is drawn as follows:
\begin{lstlisting}[numbers=none]
A|T|G|C
\end{lstlisting}
Similarly, the regex that matches SSNs is drawn:
\begin{lstlisting}[numbers=none]
\d\d\d-\d\d-\d\d\d\d
\end{lstlisting}
or
\begin{lstlisting}[numbers=none]
\d{3}-\d{2}-\d{4}
\end{lstlisting}
These drawings have substantially lower syntactic cost than the drawings of the corresponding VerseML encodings shown above. Data suggests that most professional programmers are familiar with POSIX regex forms \cite{Omar:2012:ACC:2337223.2337324}. These programmers would likely agree that the POSIX forms have lower cognitive cost as well. 

\vspace{-6px}
\subsubsection{Dynamic String Parsing}\label{sec:dynamic-string-parsing}
We might attempt to approximate the POSIX standard regex syntax by defining a function \li{parse : string -> R.t} in \li{RXUtil} that parses a VerseML string representation of a POSIX regex form, producing a regular expression value or raising an exception if the input is malformed with respect to the POSIX specification. 
Given this function, a client could construct the regex matching DNA bases as follows:
\begin{lstlisting}[numbers=none]
RU.parse "SSTRA|T|G|CESTR"
\end{lstlisting}
This approach, which we refer to as \emph{dynamic string parsing}, has several limitations:
\begin{enumerate} 
\item First, there are syntactic conflicts between standard string escape sequences and standard regex escape sequences. For example, the following is not a well-formed drawing according to the textual syntax of SML (and many other languages):
\begin{lstlisting}[numbers=none,mathescape=|]
val ssn = RU.parse "SSTR\d\d\d-\d\d-\d\d\d\dESTR" (* ERROR *)
\end{lstlisting}
In practice, most parsers report an error message like the following:\footnote{This is the error message that \texttt{javac} produces. When compiling an analagous expression using SML of New Jersey (SML/NJ), we encounter a more confusing error message: \texttt{Error: unclosed string}.}
\begin{lstlisting}[numbers=none]
error: illegal escape character
\end{lstlisting}
In a small lab study, we observed that even experienced programmers made this class of mistake and could not quickly diagnose the problem and determine a workaround if they had not used a regex library recently  \cite{Omar:2012:ACC:2337223.2337324}.

The workaround -- escaping all backslashes -- nearly doubles syntactic cost here:
\begin{lstlisting}[numbers=none]
val ssn = RU.parse "SSTR\\d\\d\\d-\\d\\d-\\d\\d\\d\\dESTR"
\end{lstlisting}

Some languages build in alternative ``raw'' string forms that leave escape sequences uninterpreted. For example, OCaml supports alternative string literals delimited by matching marked curly braces, e.g. 
\begin{lstlisting}[numbers=none]
val ssn = RU.parse {rx|SSTR\d\d\d-\d\d-\d\d\d\dESTR|rx}
\end{lstlisting}
\item The next limitation is that dynamic string parsing does not capture  the idioms of compositional regex construction. 
For example, the function \li{lookup_rx} in Figure \ref{fig:lookup_rx} constructs a regex from the given string and another regex. We cannot apply \li{RU.parse} to redraw this function equivalently, but at lower syntactic cost. 

\begin{figure}[h]
\begin{lstlisting}[numbers=none]
  fun lookup_rx(name : string) => 
    R.Seq(R.Str name, R.Seq(R.Str "SSTR: ESTR", ssn))
\end{lstlisting}
\caption{Compositional construction of a regex}
\label{fig:lookup_rx}
\end{figure}
%We needed to use both dynamic string parsing and explicit applications of pattern constructors to achieve the intended semantics. 


We will describe derived forms that do capture the idioms of compositional regex construction in Sec. \ref{sec:syntax-dialects} (in particular, we will compare Figure \ref{fig:lookup_rx} to \ref{fig:derived-spliced-subexpressions}.)

Dynamic string parsing cannot capture the idioms of list construction for the same reason -- list expressions can  contain sub-expressions.

%(we will see an example of syntax that does capture such idioms below).

\item Using strings to introduce regexes also creates a \emph{cognitive hazard} for programmers who are coincidentally working with other data of type \li{string}. For example, consider the following na\"ively ``more readable definition of \lstinline{lookup_rx}'', where the infix operator \li{^} means string concatenation:
\begin{lstlisting}[numbers=none,escapechar=~]
fun lookup_rx_insecure(name : string) => 
  RU.parse (name ^ {rx|SSTR: \d\d\d-\d\d-\d\d\d\dESTR|rx})
\end{lstlisting}

or equivalently, given the regex \li{ssn} as above and an auxiliary function \li{RU.to_string} that can compute the string representation of a given regex:
\begin{lstlisting}[numbers=none,escapechar=~]
fun lookup_rx_insecure(name : string) => 
  RU.parse (name ^ "SSTR: ESTR" ^ (RU.to_string ssn))
\end{lstlisting}
%The (unstated) intent here was to treat \lstinline{name} as a sub-pattern matching only itself, but this is not the observed behavior when \lstinline{name} contains special characters that have other meanings in patterns.

Both \lstinline{lookup_rx} and \lstinline{lookup_rx_insecure} have the same type, \li{string -> R.t}, and behave identically at many inputs, particularly the ``typical'' inputs (i.e. alphabetic strings.) It is only when \li{lookup_rx_insecure} is applied to a string that parses as a regex that matches \emph{other} strings that it behaves incorrectly (i.e. differently from \li{lookup_rx}.)

In applications that query sensitive data, mistakes like this lead to \emph{injection attacks}, which are among the most common and catastrophic security threats today \cite{owasp2013}.

This problem is fundamentally attributable to the programmer making a mistake in a misguided effort to decrease syntactic cost. However, the availability of a better approach for decreasing syntactic cost would help make this class of mistakes less common \cite{Bravenboer:2007:PIA:1289971.1289975}. %Given that our design philosophy is explicitly concerned with issues of cognitive cost, it is natural to consider these common cognitive hazards.

%Proving that mistakes like this have not been made involves reasoning about complex run-time data flows. 

 %Ultimately, of course, mistakes like this are the fault of a programmer using a flawed heuristic, and they could be avoided with discipline. The problem is once again that it is difficult to detect violations of this discipline automatically. 

 %Ideally, our library would be able to make it more difficult to inadvertently introduce subtle security bugs like this.
\item The final problem is that regex parsing does not occur until the call to \li{RU.parse} is dynamically evaluated. For example, the malformed regex form in the program fragment below will only trigger an exception when this expression is evaluated during the full moon: %Achieving this goal is an explicit goal of this proposal, so we are obviously not happy with this.

\begin{lstlisting}[numbers=none]
match moon_phase with 
Full => RU.parse "SSTR(GCESTR" | _ => (* ... *)
end
\end{lstlisting}
Malformed string encodings of regexes can sometimes be discovered by testing, though empirical data gathered from large open source projects suggests that  many malformed regexes remain undetected by test suites ``in the wild'' \cite{spishak2012type}.

One workaround is for the programmer to lift all such calls where the argument is a string literal out to the top level of the program, so that the exception is raised every time the program is evaluated. There is a cognitive penalty associated with moving the description of a regex away from its use site (but for statically determined regexes, this might be an acceptable trade-off.) For regexes constructed compositionally, this may not be possible. % Moreover, the dynamic cost of parsing the regex is incurred on every invocation of the program, even when the regex will never be used.
% Statically verifying that pattern formation errors will not dynamically arise requires reasoning about arbitrary dynamic behavior. This is an undecidable verification problem in general and can be difficult to even partially automate. In this example, the verification procedure would first need to be able to establish that the variable \lstinline{rxparse} is equal to the parse function \lstinline{RUtil.parse}. If the string argument had not been written literally but rather computed, e.g. as \lstinline{"SSTR(GESTR" ^ "SSTRCESTR"} where \lstinline{^} is the string concatenation function applied in infix style, it would also need to be able to establish that this expression is equivalent to the string \lstinline{"SSTR(GCESTR"}. For patterns that are dynamically constructed based on input to a function, evaluating the expression statically (or, more generally, in some earlier ``stage'' of evaluation \cite{Jones:Gomard:Sestoft:93:PartialEvaluation}) also does not suffice. 

Another approach is to perform a static analysis that attempts to discover malformed statically determined regexes wherever they appear \cite{spishak2012type}.

\item Finally, to reiterate, this approach is not suitable for abbreviating patterns.
% Of course, asking the client to provide a proof of well-formedness would defeat the purpose of lowering syntactic cost.

% In contrast, were our language to support  derived regex syntax, pattern parsing would occur at compile-time and so malformed patterns would produce a compile-time error, no matter where they appear in a program.

% \item Dynamic string parsing also necessarily incurs dynamic cost. Regular expression patterns are common when processing large datasets, so it is easy to inadvertently incur this cost repeatedly. For example, consider mapping over a list of strings:
% \begin{lstlisting}[numbers=none]
% map exmpl_list (fn s => rxmatch (rxparse "SSTRA|T|G|CESTR") s)
% \end{lstlisting}
% To avoid incurring the parsing cost for each element of \lstinline{exmpl_list}, the programmer or compiler must move the parsing step out of the closure (for example, by eta-reduction in this simple example).\footnote{Anecdotally, in major contemporary compilers, this optimization is not automatic.} If the programmer must do this, it can (in more complex examples) increase syntactic cost and cognitive cost by moving the pattern itself far away from its use site. Alternatively, an appropriately tuned memoization (i.e. caching) strategy could be used to amortize some of this cost, but it is difficult to reason compositionally about performance using such a strategy. %If the programmer does it, it can sometimes make the program more difficult to read. 

% %This too is difficult if a portion of the pattern is dynamically generated. % Regular expressions are often used across large datasets in scientific applications, so the absolute peformance penalty can be non-trivial.

% In contrast, were our language to primitively support derived pattern syntax, the expansion would be computed at compile-time and incur no dynamic cost.
\end{enumerate}

Difficulties like these arise whenever a programmer attempts to deploy dynamic string parsing as a solution to the problem of high syntactic cost. % The reason is that syntactic cost is a property of a drawing of a program, so trying to address it by drawing a different program requires establishing that the alternative program is equivalent to the program that the client would write if syntactic cost was not a consideration (which is, at worst, an ill-posed problem, and at best, a rather difficult problem.) %Moreover, logically equivalent programs can differ in terms of performance.
(There are, of course, legitimate applications of dynamic string parsing that are not motivated by the desire to decrease syntactic cost, e.g. when parsing string encodings of regexes received as dynamic input to the program.)%Strings are, simply put, not ideally suited for this task. 

\subsection{Dynamic Quotation Parsing}\label{sec:dynamic-quotation}
Some syntax dialects of ML, e.g. a syntax dialect that can be activated by toggling a compiler flag in SML/NJ \cite{SML/Quote,conf/icfp/Slind91}, define \emph{quotation literals}, which are derived forms for expressions of type \li{'a frag list} where \li{'a frag} is defined as follows:
\begin{lstlisting}[numbers=none]
datatype 'a frag = QUOTE of string | ANTIQUOTE of 'a
\end{lstlisting}
Quotation literals are delimited by backticks, e.g. \li{`SCSSA|T|G|CECSS`} is the same as writing \li{[QUOTE "SSTRA|T|G|CESTR"]}. Expressions of variable or parenthesized form that appear prefixed by a caret in the body of a quotation literal  are parsed out and appear wrapped in the \li{ANTIQUOTE} constructor, e.g. \li{`SCSSGC^(ECSSdna_rxSCSS)GCECSS`}  is the same as writing:
\begin{lstlisting}[numbers=none]
[QUOTE "SSTRGCESTR", ANTIQUOTE dna_rx, QUOTE "SSTRGCESTR"]
\end{lstlisting}
Unlike dynamic string parsing, \emph{dynamic quotation parsing} allows library providers to capture idioms involving subexpressions. For example:
\begin{itemize}
\item The regex library provider can define a function \li{qparse : R.t frag list -> R.t} in \li{RXUtil} that parses the given fragment list according to the POSIX standard extended to support antiquotation, producing a regex value or raising an exception if the fragment list cannot be parsed. Appyling this function to the examples above produces the corresponding regex values at lower syntactic cost:
\begin{lstlisting}[numbers=none]
val dna = RU.qparse `SCSSA|T|G|CECSS`
val bisI = RU.qparse `SCSSGC^(ECSSdna_rxSCSS)GCECSS`
\end{lstlisting}
\item The list library provider can also define a function \li{qparse : 'a frag list -> 'a list} in the \li{List} module that constructs a list from a quoted list:
\begin{lstlisting}[numbers=none]
List.qparse `SCSS[^(ECSSx + ySCSS), ^ECSSySCSS, ^ECSSzSCSS]ECSS`
\end{lstlisting}
\end{itemize}
%This addresses some of the problems of dynamic string parsing, in that we no longer need to use string concatenation to emulate splicing. 

There remain some problems with dynamic quotation parsing:
\begin{enumerate}
\item The library provider cannot specify alternative outer delimiters or antiquotation delimiters -- backticks and the caret, respectively, are the only choices in SML/NJ. This is problematic for regexes, for example, because the caret has a different meaning in the POSIX standard.

\item Another problem is that all antiquoted values within a quotation literal must be of the same type. If, for example, we sought to support both spliced regexes and spliced strings in quoted regexes, we would need to define an auxiliary sum type in \li{RXUtil} 
and the client would need to wrap each antiquoted expression with a call to the corresponding constructor to mark its type. 
For example, \li{lookup_rx} would be drawn as follows (assuming suitable definitions of \li{RU.QS} and \li{RU.QR}, not shown):
\begin{lstlisting}[numbers=none]
fun lookup_rx(string : name) =>
  RU.qparse' `SCSS^(ECSSRU.QS nameSCSS): ^(ECSSRU.QR readingSCSS)ECSS`
\end{lstlisting}
Similarly, if we sought to support quoted lists where the tail is explicitly given by the client (following OCaml's revised syntax \cite{ocaml-manual}), clients would need to apply marking constructors to each antiquoted expression:
\begin{lstlisting}[numbers=none]
List.qparse `SCSS[^(ECSSList.V xSCSS), ^(ECSSList.V ySCSS) :: ^(ECSSList.VS zsSCSS)]ECSS`
\end{lstlisting}
Marking constructors increase syntactic cost (rather substantially in such examples.)

\item As with dynamic string parsing, parsing occurs dynamically. We cannot use the trick of lifting all calls to \li{qparse} to the top level because the arguments are not closed string literals. At best, we can lift these calls out as far as the binding structure allows, i.e. into the earliest possible ``dynamic phase.'' Parse errors are detected only when this phase is entered, and the dynamic cost of parsing is incurred each time this phase is entered. For example, \li{List.qparse} is called $n$ times below, where $n$ is the length of \li{input}:
\begin{lstlisting}[numbers=none]
List.map (fn x => List.qparse `SCSS[^ECSSxSCSS, ^(ECSS2 * xSCSS)]ECSS`) input
\end{lstlisting}

One way to detect parse errors early and reduce the dynamic cost of parsing is to use a system of \emph{staged partial evaluation} \cite{Jones:Gomard:Sestoft:93:PartialEvaluation}. For example, if we integrated Davies' temporal logic based approach into our language \cite{DBLP:conf/lics/Davies96}, we could rewrite the list example above as follows:
\begin{lstlisting}[numbers=none,morekeywords={next,prev}]
List.map (fn x => prev (List.sqparse 
	`SCSS[^(ECSSnext xSCSS), ^(ECSSnext (2 * x)SCSS)]ECSS`)) input
\end{lstlisting}
Here, the operator \lstinline[morekeywords={next,prev}]{prev} causes the call to \li{List.sqparse} to be evaluated in the previous stage. \li{List.sqparse} differs from \li{List.qparse} in that the antiquoted values in the input must be encapsulated expressions from the next stage, indicated by the \lstinline[morekeywords={next, prev}]{next} operator. The return value is also an encapsulated expression from the next stage. By composing this value with \lstinline[morekeywords={next,prev}]{prev}, we achieve the desired staging. Other systems, e.g. MetaML \cite{Sheard:1999:UMS} and MacroML \cite{ganz2001macros}, provide similar staging primitives.

The main problem with this approach is that it incurs substantial annotation overhead. Here, the staged call to \li{List.sqparse} has higher syntactic cost than if we had simply manually applied \li{Nil} and \li{Cons}. This problem is compounded if marking constructors like those described above are needed.
  % the application of \li{qparse} is evaluated.

\item Finally, quotation parsing, like the other approaches considered so far, helps only with the problem of abbreviating expressions. It provides no solution to the problem of abbreviating patterns (because parse functions compute values, not patterns.)% The reason is simple: these approaches require applying functions, which, by nature, are expressions, not patterns.
\end{enumerate}

Due to these problems, VerseML does not build in quotation literals.\footnote{In fact, quotation syntax can be expressed using parametric TLMs, which are the topic of Chapter \ref{chap:ptsms}.}

% \subsection{Syntax Dialects}
% To gain more precise control over form, a library provider, or another interested party, might also consider defining a syntax dialect. 
% A dialect of a syntax definition, $\mathcal{D}$, is a new syntax definition, $\mathcal{D}'$, that:
% \begin{enumerate}
% \item extends $\mathcal{D}$, meaning that all drawings that are well-formed in $\mathcal{D}$ are well-formed in $\mathcal{D}'$ and identify the same AST; and
% \item defines additional derived forms.
% \end{enumerate}
% We leave the notion of a ``syntax definition'' undefined here for the sake of generality -- there are many different syntax definition systems.

\subsection{Fixity Directives}\label{sec:Fixity-directives}
We will now consider various syntax definition systems.

The simplest syntax definition systems allow programmers to introduce new infix operators. For example, the syntax definition system integrated into Standard ML allows the programmer to designate \li{::} as a right-associative infix operator at precedence level 5 by placing the following directive in the program text:
\begin{lstlisting}[numbers=none]
infixr 5 ::
\end{lstlisting}
This directive causes expressions of the form \li{e1 :: e2} to desugar to \li{op:: (e1, e2)}, i.e. the variable \li{op::} is applied to the pair \li{(e1, e2)}. Given that \li{op::} is a list value constructor in SML, this expression constructs a list with head \li{e1} and tail \li{e2}.

The fixity directive above also causes patterns of the form \li{p1 :: p2} to desugar to \li{op:: (p1, p2)}, i.e. to pattern constructor application. Again, because \li{op::} is a list pattern constructor in SML, the desugaring of this pattern matches lists where the head matches \li{p1} and the tail matches \li{p2}. (If we had used the identifier \li{Cons}, rather than \li{op::}, in the definition of the \li{list} datatype, we would never be able to use the \li{::} operator in list patterns because SML does not support pattern synonyms.)%This is why the SML Basis library uses the label \li{op::} rather than \li{Cons} directly in its definition of the list datatype for this reason.

Figure \ref{fig:infix-RX} shows three fixity declarations related to our regex library together with a functor \li{RXOps} that binds the corresponding identifiers to the appropriate functions. Assuming that a library packaging system has brought the fixity declarations and the definition of \li{RXOps} from Figure \ref{fig:infix-RX} into scope, we can instantiate \li{RXOps} and then \li{open} this instantiated module to bring the necessary bindings into scope as follows:


\begin{figure}
\begin{lstlisting}
infix 5 ::
infix 6 <*>
infix 4 <|>

functor RXOps(R : RX) =
struct 
  structure RU = RXUtil(R)
  val op:: = R.Seq
  val op<*> = RU.repeat
  val op<|> = R.Or
end
\end{lstlisting}
\caption{Fixity declarations and related bindings for \li{RX}}
\label{fig:infix-RX}
\vspace{-5px}
\end{figure}
\begin{lstlisting}[numbers=none]
structure ROps = RXOps(R)
open ROps
\end{lstlisting}
We can now draw the previous examples equivalently as follows:
\begin{lstlisting}[numbers=none]
val dna = (R.Str "SSTRAESTR") <|> (R.Str "SSTRTESTR") <|> (R.Str "SSTRGESTR") <|> 
          (R.Str "SSTRCESTR")
val ssn = (RU.digit)<*>3 :: (RU.digit)<*>2 :: (RU.digit)<*>4
fun lookup_rx(name : string) => 
  (Str name) :: (Str "SSTR: ESTR") :: ssn
\end{lstlisting}

This demonstrates two other problems with this approach. 

First, it grants only limited control over form -- we cannot express the POSIX forms in this way, only \emph{ad hoc} (and in this case, rather poor) approximations thereof. 
% as given cannot be used in patterns.%The desugaring of an infix operator application is always of function application form. In particular, it is always the operator variable (e.g. \li{op::}) applied to two arguments, first the expression on the left, then the expression on the right (with the precedence and associativity determining what these expressions are.) 

Second, there can be syntactic conflicts between libraries. Here, both the list library and the regex library have defined a fixity directive for the \li{::} operator, but each specifies a  different associativity. As such, clients cannot use both forms in the same scope. There is no mechanism that allows a client to explicitly qualify an infix operator as referring to the fixity directive from a particular library -- fixity directives are  not exported from modules or otherwise integrated into the binding structure of SML (libraries are extralinguistic packaging constructs, distinct from modules.) 

%This identifier-oriented approach is also rather \emph{ad hoc}, in that renaming or substituting for an identifier can break or change the meaning of the program.

Formally, each fixity directive induces a dialect of the subset of SML's textual syntax that does not allow the declared identifier to appear in prefix position. When two such dialects are combined, the resulting dialect is not necessarily a dialect of both of the constituent dialects (one fixity declaration overrides the other, according to the order in which the dialects were combined.)

Due to these limitations, VerseML does not inherit this mechanism from SML (the infix operators that are available in VerseML, like \li{^} for string concatenation, have a fixed precedence, associativity and desugaring.)

\subsection{Mixfix Syntax Definitions}\label{sec:mixfix}
Fixity directives do not give direct control over desugaring -- the  desugaring of a binary operator form introduced by a fixity directive is always of function application or pattern constructor application form. ``Mixfix'' syntax definition systems generalize SML-style fixity directives in that newly defined forms can contain any number of sub-trees (rather than just two) and their desugarings are determined by a programmer-defined rewriting. 

The simplest of these systems, e.g. Griffin's system of notational definitions \cite{5134}, later variations on this system with stronger theoretical properties \cite{DBLP:conf/gpce/TahaJ03}, and the syntax definition system integrated into the Agda programming language \cite{DBLP:conf/ifl/DanielssonN08}, support only forms that contain a fixed number of sub-trees, e.g. \li{if _ then _ else _}. We cannot define SML-style derived list forms using these systems, because list forms can contain any number of sub-trees.

More advanced notational definition systems support new forms that contain $n$-ary sequences of sub-trees separated by a given token. For example, Coq's notation system \cite{Coq:manual} can be used to express list syntax as follows:
\begin{lstlisting}[numbers=none]
Notation " [ ] " := nil (format "[ ]") : list_scope.
Notation " [ x ] " := (cons x nil) : list_scope.
Notation " [ x ; y ; .. ; z ] " := 
  (cons x (cons y .. (cons z nil) ..)) : list_scope.
\end{lstlisting}
Here, the final declaration handles a sequence of $n > 1$ semi-colon separated trees.

Even under these systems, we cannot define POSIX-style regex syntax. The problem is that we can only extend the syntax of the existing sorts of trees, e.g. types, expressions and patterns. We cannot define new sorts of trees, with their own distinct syntax. For example, we cannot define a new sort for regular expressions, where sequences of characters are not recognized as Coq identifiers but rather as regex character sequences.

As with other mechanisms for defining syntax dialects, we cannot reason modularly about syntactic determinism. The Coq manual acknowledges this \cite{Coq:manual}: 
\begin{quote}
\emph{Mixing different symbolic notations in [the] same text may cause serious parsing ambiguity.}
\end{quote}

To help library clients manage conflicts when they arise, most of these systems include various precedence mechanisms. For example, Agda supports a system of directed acyclic precedence graphs \cite{DBLP:conf/ifl/DanielssonN08} (this is related to earlier work by Aasa where a complete precedence graph was necessary \cite{DBLP:journals/tcs/Aasa95}.) In Coq, the programmer can associate notation definitions with named ``scopes'', e.g. \li{list_scope} in the example above. A scope can  be activated or deactivated explicitly using scope directives to control the availability of notation definitions. The innermost scope has the highest precedence. In some situations, Coq is able to use type information to activate a scope implicitly. Mixfix syntax definition systems that use types more directly to disambiguate from several possibilities have also been developed \cite{missura1997higher,wieland2009parsing}. These only reduce the likelihood of a conflict -- they do  not eliminate the possibility entirely.

Aasa et al. developed a system whereby each constructor of a datatype definition could have its own syntax \cite{DBLP:conf/lfp/AasaPS88,DBLP:conf/fpca/Aasa93}. This syntax was delimited from the rest of the language using a fixed quotation-antiquotation system like that described in Sec. \ref{sec:dynamic-quotation}. Parsing was integrated into the type inference mechanism of the language. However, this system is also not expressive enough to handle POSIX regex syntax, again because it forces an immediate, one-to-one correspondence between constructors and syntactic forms. For example, it is not possible to treat arbitrary character sequences as regex character sequences, which are governed by the \li{Str} constructor. It is also not possible to capture idioms that do not correspond immediately to datatype constructor application (e.g. idioms involving modules.)

%VerseML does not integrate a  mixfix syntax definition system.


\subsection{Grammar-Based Syntax Definition Systems}\label{sec:grammars}\label{sec:syntax-dialects}
Many syntax definition systems are oriented around \emph{formal grammars} \cite{hopcroft1979introduction}. Formal grammars have been studied since at least the time of P\~anini, who developed a grammar for Sanskrit in or around the 4th century BCE \cite{Ingerman:1967:LFS:363162.363165}. 

\emph{Context-free grammars (CFGs)} were first used to define the textual syntax of a major programming language -- Algol 60 -- by Backus \cite{naur1963revised}. Since then, countless other syntax definition systems oriented around CFGs have emerged. In these systems a syntax definition consists of a CFG (perhaps from some restricted class of CFGs) equipped with various auxiliary definitions (e.g. a lexer definition in many systems) and logic for computing an output value (e.g. a tree) based on the determined form of the source text.% Some of the systems that we will describe  operate as \emph{textual preprocessors}, transforming source text into text accepted by a language's original syntax definition. Others directly generate syntax trees. 

% Some compilers, e.g. the OCaml compiler \cite{ocaml-manual}, integrate preprocessing into the build system. Other systems use a layer of directives placed in the source text to control preprocessor invocation. For example, in Racket's reader macro system, the programmer can direct the lexer (called the ``reader'') to shift control to a given parser when a designated directive or token is seen \cite{Flatt:2012:CLR:2063176.2063195}. A few systems are integrated directly into a language definition -- we will point these out when we introduce them. 

Perhaps the most established CFG-based syntax definition systems within the ML ecosystem are ML-Lex  and ML-Yacc, which are distributed with SML/NJ \cite{TarditiDR:mly}, and Camlp4, which was (until recently)  integrated into  the OCaml system (in recent releases of the OCaml system, it has been deprecated in favor of a simpler system, \li{ppx}, that we discuss in the next section) \cite{ocaml-manual}. In these systems, the output is an ML value computed by ML functions that appear associated with productions in the grammar (these functions are referred to as the \emph{semantic actions}.) 

The \emph{syntax definition formalism  (SDF)}  \cite{journals/sigplan/HeeringHKR89} is a syntactic formalism for describing CFGs. SDF is used by a number of syntax definition systems, e.g. the Spoofax ``language workbench'' \cite{kats2010spoofax}. These systems commonly use Stratego, a rule-based rewriting language, as the language that output logic is written in \cite{Visser-RTA01}. SugarJ is an extension of Java that allows programmers to define and combine fragments of SDF+Stratego-based syntax definitions directly from within the program text \cite{erdweg2011sugarj}. SugarHaskell is a similar system based on Haskell \cite{erdweg2012layout} and Sugar* simplifies the task of defining similar extensions of other languages \cite{erdweg2013framework}. SoundExt and SugarFOmega add the requirement that new derived forms must come equipped with  derived typing rules \cite{conf/icfp/LorenzenE13}. The system must be able to verify that the rewrite rules are sound with respect to these derived typing rules (their verification system defers to the proof search facilities of PLT-Redex \cite{Felleisen-Findler-Flatt09}.) SoundX generalizes this idea to other base languages, and adds the ability to define type-dependent rewritings \cite{conf/popl/LorenzenE16}. We will say more about SoundExt/SugarFOmega and SoundX when we discuss abstract reasoning under syntax dialects below.

 %Erdweg et al. have developed many non-trivial examples, including . \to

Copper implements a CFG-based syntax definition system that uses a context-aware scanner \cite{conf/gpce/WykS07}. We will say more about Copper when we discuss modular reasoning about syntactic determinism below.% Silver is an \emph{attribute grammar} system that incorporates Copper \cite{VanWyk:2010:SEA}. Attribute grammars make it easier to compositionally define output logic that requires information that is not local to each form \cite{knuth1968semantics}.


Some other syntax definition systems are instead oriented  around \emph{parsing expression grammars} (PEGs) \cite{Ford04a}. PEGs are similar to CFGs, distinguished mainly in that they are deterministic by construction (by allowing only for explicitly prioritized choice between alternative parses.) \emph{Packrat parsers} implement PEGs \cite{DBLP:journals/corr/abs-cs-0603077}.

\vspace{-5px}\subsection{Parser Combinator Systems}\label{sec:parser-combinators}\vspace{-4px}
\emph{Parser combinator systems} specify a functional interface for defining parsers, together with various functions that generate new parsers from existing parsers and other values (these functions are referred to as the \emph{parser combinators}) \cite{Hutton1992d}. 
In some cases, the composition of various parser combinators can be taken as definitional (as opposed to the usual view, where a parser is an {implementation} of a syntax definition.)

For example, Hutton describes a system where parsers are functions of some type in the following parametric type family:% \li{'char 'tree parser}, bound in VerseML's syntax:
\begin{lstlisting}[numbers=none]
type parser('c, 't) = list('c) -> list('t * list('c))
\end{lstlisting}
Here, a parser is a function that takes a list of (abstract) characters and returns a list of valid parses, each of which consists of an (abstract) output (e.g. a tree) and a list of the characters that were not consumed. An input is ambiguous if this function returns more than one parse. A deterministic parser is one that never returns more than one parse. The  non-deterministic choice combinator \li{alt} has the following signature: 
\begin{lstlisting}[numbers=none]
val alt : parser('c, 't) -> parser('c, 't) -> parser('c, 't)
\end{lstlisting}
The \li{alt} combinator combines the two given parsers by applying them both to the input and appending the lists that they return.

Various alternative designs that better control dynamic cost or that maintain other useful properties have also been described. For example, Hutton and Meijer describe a parser combinator system in monadic style \cite{hutton1998monadic}. Okasaki has described an alternative design that uses continuations to control cost \cite{Okasaki98b}.

% Given a parser (generated using parser combinators, or by using a CFG processed by a parser generator), the programmer must 

% Some compilers, e.g. the OCaml compiler \cite{ocaml-manual}, integrate preprocessing into the build system. 
Some systems use a layer of directives placed in the source text to control parser invocation. For example, in Racket's reader macro system, the programmer can direct the initial token reader to shift control to a given parser when a designated directive or token is seen \cite{Flatt:2012:CLR:2063176.2063195,DBLP:journals/jfp/FlattCDF12}. Honu is another reader based system, which uses a simple syntactic pattern language to initially ``enforest'' the token stream, i.e. to turn it into a simple tree structure, before passing it to the parser \cite{DBLP:conf/gpce/RafkindF12}. %A few systems are integrated directly into a language definition -- we will point these out when we introduce them. 


%\subsubsection{Preprocessing}
%Syntax dialects implemented using parser combinators, or generated from a grammar-based syntax definition by a \emph{parser generator}, usually 

%The most minimal syntax definition systems, e.g. Racket's dialect preprocessor \cite{Flatt:2012:CLR:2063176.2063195}, take any function of a type like \li{string -> exp}, where \li{exp} is a system-defined encoding of the syntax of expressions (enriched perhaps with source code locations and other ``metadata''), as a syntax definition. Programmers using these systems are free to use an implementation of some other syntax definition system to define this function.

\vspace{-8px}\subsection{Examples of Syntax Dialects}\label{sec:examples-of-syntax-dialects}\vspace{-4px}
Now that we have given an overview of a number of syntax definition systems, let us consider two specific examples of syntax dialects to motivate our subsequent discussion of the problems with the dialect oriented approach.

\vspace{-10px}\subsubsection{Example 1: $\mathcal{V}_\texttt{rx}$}
Using any of the more general syntax definition systems described in the two previous sections, we can define a dialect of VerseML's textual syntax called  $\mathcal{V}_\texttt{rx}$ that builds in derived regex forms.

In particular, $\mathcal{V}_\texttt{rx}$ extends the syntax of expressions with  \emph{derived regex literals}, which are delimited by forward slashes, e.g. \li{/SURLA|T|G|CEURL/}. The desugaring of this form is equivalent to the following if we assume that \li{Or} and \li{Str} stand for the corresponding constructors of the recursive labeled sum type \li{rx} that was defined in Figure \ref{fig:datatype-rx}:
\begin{lstlisting}[numbers=none]
Or(Str "SSTRAESTR", Or (Str "SSTRTESTR", Or (Str "SSTRGESTR", Str "SSTRCESTR")))
\end{lstlisting}
Of course, it is unreasonable to assume that \li{Or} and \li{Str} are bound appropriately at every use site. In order to maintain \emph{context independence}, the desugaring instead applies the explicit \li{fold} and \li{inj} operators as discussed in Sec. \ref{sec:lists}.\footnote{In SML, where datatypes are abstract and explicit fold and injection operators are not exposed, it is more difficult to maintain context independence. We would need to provide a module containing the constructors as a ``syntactic argument'' to each form -- we describe this technique as it relates to our modular encoding of regexes in Example 2 below.}

\begin{figure}
\begin{lstlisting}[numbers=none]
val ssn = SURL/\d\d\d-\d\d-\d\d\d\d/EURL
fun lookup_rx(name : string) => SURL/@EURLnameSURL: %EURLssnSURL/EURL
\end{lstlisting}
\vspace{-5px}
\caption{Derived regex expression forms in $\mathcal{V}_\texttt{rx}$}
\label{fig:derived-spliced-subexpressions}
\end{figure}
\begin{figure}
\begin{lstlisting}[numbers=none]
fun is_dna_rx(r : rx) : boolean => 
  match r with 
  | SURL/A/EURL => True
  | SURL/T/EURL => True
  | SURL/G/EURL => True
  | SURL/C/EURL => True
  | SURL/%(EURLr1SURL)%(EURLr2SURL)/EURL => (is_dna_rx r1) andalso (is_dna_rx r2)
  | SURL/%(EURLr1SURL)|%(EURLr2SURL)/EURL => (is_dna_rx r1) andalso (is_dna_rx r2)
  | SURL/%(EURLrSURL)*/EURL => is_dna_rx r'
  | _ => False
  end
\end{lstlisting}
\vspace{-8px}
\caption{Derived regex pattern forms in $\mathcal{V}_\texttt{rx}$}
\label{fig:derived-pattern-syntax}
\end{figure}


$\mathcal{V}_\texttt{rx}$ also supports regex literals that contain {subexpressions}. These  capture the idioms that arise when constructing regex values compositionally. For example, the definition of \li{lookup_rx} in Figure \ref{fig:derived-spliced-subexpressions} is equivalent to the definition of \li{lookup_rx} that was given in Figure \ref{fig:lookup_rx}. The prefix \li{SURL@EURL} followed by the identifier \li{name} causes the expression \lstinline{name} to appear in the desugaring as if wrapped in the \li{Str} constructor, and the prefix \li{SURL%EURL} 
followed by the identifier \li{ssn} causes \lstinline{ssn} to appear in the desugaring directly. We refer to the expressions that appear inside literal forms as \emph{spliced expressions}. 


%  The body of \li{example_rx} could equivalently be written as follows:
% \begin{lstlisting}[numbers=none]
% Seq(Str(name), Seq(Str "SSTR: ESTR", ssn))
% \end{lstlisting}
% (Again, the desugaring itself must use the explicit \li{fold} and \li{inj} operators to maintain context-independence.)
%Notice that \li{name} appears wrapped in the label \li{Str} because it was prefixed by \li{@}, whereas \li{ssn} appears unadorned because it was prefixed by \li{%}. 

To splice in an expression that is not of variable form, e.g. a function application, we must delimit it with parentheses: \li{/SURL@(EURLcapitalize nameSURL)EURL/}.

Finally, $\mathcal{V}_\texttt{rx}$ extends the syntax of patterns with analagous \emph{derived regex pattern literals}. For example, the definition of \li{is_dna_rx} in Figure \ref{fig:derived-pattern-syntax} is equivalent to the definition of \li{is_dna_rx} that was given in Figure \ref{fig:is_dna_rx}. Notice that the variables bound by the patterns in Figure \ref{fig:derived-pattern-syntax} appear inside \emph{spliced patterns}.

\subsubsection{Example 2: $\mathcal{V}_\text{RX}$}
\begin{figure}
\begin{lstlisting}[numbers=none]
fun is_dna_rx'(r : R.t) : boolean => 
  match R.unfold_norm r with 
  | SURL/A/EURL => True
  | SURL/T/EURL => True
  | SURL/G/EURL => True
  | SURL/C/EURL => True
  | SURL/%(EURLr1SURL)%(EURLr2SURL)/EURL => (is_dna_rx' r1) andalso (is_dna_rx' r2)
  | SURL/%(EURLr1SURL)|%(EURLr2SURL)/EURL => (is_dna_rx' r1) andalso (is_dna_rx' r2)
  | SURL/%(EURLrSURL)*/EURL => is_dna_rx r'
  | _ => False
  end
\end{lstlisting}\vspace{-5px}
\caption{Derived regex unfolding pattern forms in $\mathcal{V}_\text{RX}$}
\label{fig:VRX-pats}
\end{figure}

In Sec. \ref{sec:syntax-examples-regexps}, we also considered a more sophisticated formulation of our regex library organized around the signature \li{RX} defined in Figure \ref{fig:signature-RX}. Let us define another dialect of VerseML's textual syntax called $\mathcal{V}_\text{RX}$ that defines derived forms whose desugarings involve modules that implement \li{RX}. For this to work in a  context-independent manner, these forms must take the particular module that is to appear in the desugaring as a spliced subterm. For example, in the following program fragment, the module \li{R} is ``passed into'' each derived form for use in its desugaring:
\begin{lstlisting}[numbers=none]
val ssn = RSURL./\d\d\d-\d\d\d\d-\d\d\d/EURL
fun lookup_rx'(name : string) => RSURL./@EURLnameSURL: %EURLssnSURL/EURL
\end{lstlisting}
The desugaring of the body of \li{lookup_rx'} is:
\begin{lstlisting}[numbers=none]
R.Seq(R.Str(name), R.Seq(R.Str "SSTR: ESTR", ssn))
\end{lstlisting}
This desugaring logic is context-independent because the constructors are explicitly qualified (i.e. \li{Seq} and \li{Str} are \emph{component labels} here, not variables.) The only variables that appear in the desugaring are \li{R}, \li{name} and \li{ssn}. All of these were specified by the client at the use site, so they are subject to renaming.

Recall that \li{RX} specifies a function \li{unfold_norm : t -> u(t)} for computing the normal unfolding of the given regex. $\mathcal{V}_\text{RX}$ defines derived forms for patterns matching values of types in the type family \li{u('a)}. These are used in the definition of \li{is_dna_rx'} given in Figure \ref{fig:VRX-pats}.




\subsection{Problems with Syntax Dialects}\label{sec:problems-with-syntax-dialects}
\subsubsection{Conservatively Combining Syntax Dialects}

Notice that the derived regex pattern forms that appear in Figure \ref{fig:VRX-pats} are identical to those that appear in Figure \ref{fig:derived-pattern-syntax}. Their desugarings are, however, different. In particular, the patterns in Figure \ref{fig:VRX-pats} match values of type \li{u('a)}, whereas the patterns in Figure \ref{fig:derived-pattern-syntax} match values of type \li{rx}. 

It would be useful to have derived forms for values of type \li{rx} available even when we are working with the modular encoding of regexes, because we have defined a function \li{view : R.t -> rx} in \li{RXUtil}. This brings us to the first of the two main problems with the dialect-oriented approach, already described in Chapter \ref{chap:intro}: there is no good way to conservatively combine $\mathcal{V}_\text{rx}$ and $\mathcal{V}_\text{RX}$. In particular, any such ``combined dialect'' will either fail to conserve determinism (because the forms overlap), or the combined dialect will not be a dialect of both of the constituent dialects, i.e. some of the forms from one dialect will ``shadow'' the overlapping forms from the other dialect (depending on the order in which they were combined \cite{Ford04a}.) 

In response to this problem, Schwerdfeger and Van Wyk have developed a modular analysis that accepts only deterministic extensions of a base LALR(1) grammar where all new forms must start with a ``marking'' terminal symbol and obey certain other constraints related to  the follow sets of the base grammar's non-terminals \cite{conf/pldi/SchwerdfegerW09,schwerdfeger2010context}. By relying on a context-aware scanner (a feature of Copper \cite{conf/gpce/WykS07}) to transfer control when the marking terminals are seen, extensions of a base grammar that pass this analysis and specify disjoint sets of marking terminals can be combined without introducing conflict. %The analysis is ``nearly'' modular in that only a relatively simple ``combine-time'' check that the set of marking terminals is disjoint is necessary.

For the two dialects just considered, these conditions are not satisfied. If we modify the grammar of $\mathcal{V}_\text{RX}$ so that, for example, the regex literal forms are marked with \li{#\dolla#r} and the regex unfolding forms are marked with \li{#\dolla#u}, the analysis will accept both grammars, and the combine-time disjointness check will pass, solving our immediate problem at only a small cost. However, a conflict could still  arise later when a client combines these extensions with another extension that also uses the marking terminals \li{#\dolla#r}, \li{#\dolla#u} or \li{/}. %There is no reason to believe that other dialect providers will avoid these marking terminals.

\begin{figure}
\begin{lstlisting}[numbers=none]
fun is_dna_rx'(r : R.t) : boolean => 
  match R.unfold_norm r with 
  | SURL$cmu_edu_comar_rx $u/A/EURL => True 
  | SURL$cmu_edu_comar_rx $u/T/EURL => True
  | SURL$cmu_edu_comar_rx $u/G/EURL => True
  | SURL$cmu_edu_comar_rx $u/C/EURL => True
  (* and so on *)
  | _ => False
  end
\end{lstlisting}
\vspace{-4px}
\caption{Using URI-based grammar names together with marking tokens to avoid syntactic conflicts}
\label{fig:vanwyk}
\vspace{-6px}
\end{figure}


The solution proposed by Schwerdfeger and Van Wyk \cite{conf/pldi/SchwerdfegerW09,schwerdfeger2010context} is 1) to allow for the grammar's name to be used as an additional syntactic prefix when a conflict arises, and 2) to adopt a naming convention for grammars  based on the Internet domain name system (or some similar coordinating system) that makes conflicts unlikely. For example, Figure \ref{fig:vanwyk} shows how a client would need to draw \li{is_dna_rx'} if a conflict arose. Clearly, this drawing has higher syntactic cost than the drawing in Figure \ref{fig:VRX-pats}. Moreover, there is no simple way for clients to selectively control this cost by defining scoped abbreviations for marking tokens or grammar names (as one does for types, modules or values that are exported from deeply nested modules) because this mechanism is purely syntactic, i.e. agnostic to the binding structure of the language. A facility for defining unscoped abbreviations of marking tokens at combine-time could partially alleviate this cost.

Another approach aimed at making conflicts less likely, though not impossible, is to use types to choose from amongst several possible parses. Some approaches require generating the full \emph{parse forest} before typechecking proceeds, e.g. the \emph{MetaBorg} system \cite{bravenboer2005generalized}. This approach is inefficient, particularly when a large number of grammars have been composed. The method of \emph{type-oriented island parsing} integrates parsing and typechecking so that disambiguation occurs as early as possible \cite{DBLP:conf/sfp/SilkensenS12}.

A more radical approach would be to insist that programmers use a \emph{language composition editor} like Eco \cite{diekmann2014eco}. Language composition editors allow programmers to explicitly switch from one syntax to another with an editor command. This is an instance of the more general concept of \emph{structure editing} (also called \emph{structured editing}, \emph{projectional editing} or \emph{syntax-directed editing}.) This concept, pioneered by the Cornell Program Synthesizer \cite{teitelbaum_cornell_1981}, has various costs and benefits, summarized in \cite{DBLP:conf/sle/VolterSBK14}. In this work, our interest is in text-based syntax, but we consider structure editors as future work in Sec. \ref{sec:future-work}.

\subsubsection{Abstract Reasoning About Derived Forms}
In addition to the difficulties of conservatively combining syntax dialects, there are a  number of other difficulties related to the fact that there is often no useful notion of syntactic abstraction that a programmer can rely on to reason about an unfamiliar derived form. The programmer may need to examine the desugaring, the desugaring logic or even the definitions of all of the constituent dialects, to definitively answer the questions given in Sec. \ref{sec:abs-reasoning-intro}. These questions were stated relative to a particular example involving the query processing language K.  
Here, we generalize from that example to develop an informal classification of the properties that programmers might have difficulty reasoning about in analagous situations. In each case, we will discuss exceptional systems where these difficulties are ameliorated or avoided entirely.% We discuss some exceptions from amongst the related work above:

% \begin{enumerate}
\paragraph{Responsibility} It is not always straightforward to determine which constituent dialect is responsible for any particular derived form.

The system implemented by Copper \cite{conf/pldi/SchwerdfegerW09,schwerdfeger2010context} is an exception, in that the marking terminal (and the grammar name, if necessary) allows clients to search across the constituent dialect definitions for the corresponding declaration without needing to understand any of them deeply.

\paragraph{Segmentation} It is not always possible to segment a derived form such that each segment consists either of a spliced base language term (which we have drawn in black in the examples in this document) or a sequence of characters that are parsed otherwise (which we have drawn in color.) Even when a segmentation exists, determining it is not always straightforward.

For example, consider a production in a grammar that looks like this: 
\begin{lstlisting}[numbers=none]
start <- "%(" verseml_exp ")"
\end{lstlisting}

The name of the non-terminal \li{verseml_exp} suggests that it will match any VerseML expression, but it is not certain that this is the case. Moreover, even if we know that this non-terminal matches VerseML expressions, it is not certain that the output logic will insert that expression as-is into the desugaring -- it may instead only examine its form, or transform it in some way (in which case highlighting it as a spliced expression might be misleading.)

Systems that support the generation of editor plug-ins, such as Spoofax \cite{kats2010spoofax} and Sugarclipse for SugarJ \cite{Erdweg:2012:GLE}, can generate syntax coloring logic from an annotated grammar definition, which often give programmers some indication of where a spliced term occurs. However, there is no definitive information about segmentation in how the editor displays the derived form. (Moreover, these editor plug-ins can themselves conflict, even if the syntax itself is deterministic.)

\paragraph{Capture} The desugaring of a derived form might place spliced terms under binders. These binders are not visible in the program text, but can shadow those that are. As a result, the spliced terms will inadvertently capture these expansion-internal bindings. This significantly obscures the binding structure of the program.

For derived forms that desugar to module-level definitions (e.g. to one or more \li{val} definitions), a desugaring might also introduce exported module components that are similarly invisible in the text. This can cause non-local capture when a client \li{open}s that module into scope.

In most cases, capture is inadvertent. For example, a desugaring might bind an intermediate value to some temporary variable, \li{tmp}. This can cause problems at use sites where \li{tmp} is bound. It is easy to miss this problem in testing (particularly if the types of both bindings are compatible.)

In some syntax dialects, capture is by design. For example, in (Sugar)Haskell, \li{do} notation for monadic values operates as a new binding construct \cite{erdweg2012layout}. For programmers who {are} familiar with \li{do} notation, this can be useful. But when a programmer encounters an unfamiliar form, this forces them to determine whether it similarly is designed as a new binding construct. A simple grammar provides no information about capture.%The point is simply that this is a double-edged sword.

In most systems, it is possible for dialect providers to generate identifiers that are guaranteed to be fresh at the use site. If dialect providers are disciplined about using this mechanism, they can prevent capture. However, this is awkward and most systems provide no guarantee that the dialect provider maintained this freshness discipline \cite{conf/ecoop/ErdwegSD14}.

To enforce a prohibition on capture, the system must be integrated into or otherwise made aware of the binding structure of the language. For example, some of the language-integrated mixfix systems discussed above, e.g. Coq's notation system \cite{Coq:manual}, enforce a prohibition on capture by alpha-renaming desugarings as necessary. Erdweg et al. have developed a formalism for directly describing the ``binding structure'' of program text, as well as contextual transformations that use these descriptions to rename the identifiers that appear in a desugaring to avoid capture \cite{conf/ecoop/ErdwegSD14,conf/sle/RitschelE15}.

\paragraph{Context Dependence} If the desugaring of a derived form assumes that certain identifiers are bound at the application site (e.g. to particular values, or to values of some particular type), we refer to the desugaring as being \emph{context dependent}. 

%One might describe such desugarings as having ``captured'' bindings from the use site. Notice that this is distinct from the situation described above, where it is a term at the use site that ``captures'' bindings from the desugaring. %We will avoid the term ``capture''. %We use the word ``hygiene'' to refer collectively to context independence and shadowing avoidance. 

Context dependent desugarings take control over naming away from clients. Moreover, it is difficult to determine the assumptions that a desugaring is making. As such, it becomes difficult to reason about whether renaming an identifier or moving a binding is a meaning-preserving transformation. 

In our examples above, we maintained context independence as a ``courtesy'' by explicitly applying the \li{fold} and \li{inj} operators, or by taking the module for use in the desugaring as a ``syntactic argument''. 

To enforce context independence, the system must be aware of binding structure and have some way to distinguish those subterms of a desugaring that originate in the text at the use site (which should have access to bindings at the use site) from those that do not (which should only have access to bindings internal to the desugaring.) 
For example, language-integrated mixfix systems, e.g. Coq's notation system, use a simple rewriting system to compute desugarings, so they satisfy these requirements and can enforce context independence. Coq gives desugarings access only to the bindings visible where the notation was defined.

More flexible systems where desugarings are computed functionally, or language-external systems that have no understanding of binding structure, do not enforce context independence.

\paragraph{Typing} Finally, it is not always clear what type an expression drawn in derived form has, or what type of value that a pattern drawn in derived form matches. Similarly, it is not always straightforward to determine what type a spliced expression has, or what type of value that a spliced pattern matches.

SoundExt/SugarFomega \cite{conf/popl/LorenzenE16} and SoundX \cite{conf/sle/RitschelE15} allow dialect providers to define derived typing rules alongside derived forms and desugaring rules. These systems automatically verify that the desugaring rules are sound with respect to these derived typing rules. This ensures that type errors are never reported in terms of the desugaring (which is the stated goal of their work.) However, this helps only to a limited extent in answering the questions just given. In particular, the programmer must first assign Responsibility (which is difficult for the reasons just given.) Next, the programmer must identify the spliced terms (which is difficult because these systems to not make it easy to reason about Segmentation, as just described.) Then, the programmer must construct a derivation using the relevant derived typing rules. Finally, the programmer must traverse the derivation to find out where the spliced terms appear within it to answer questions about their type. Even for relatively simple base languages, like System $\mathbf{F}_\omega$, understanding a typing derivation requires significantly more effort and expertise than programmers usually need.\footnote{At CMU, we teach ML to all first-year students (in 15-150 -- Functional Programming.) However, understanding a judgmental specification of a language like System $\mathbf{F}_\omega$ involves skills that are taught only to some third and fourth year students (in 15-312 -- Principles of Programming Languages.)} For languages like ML, the judgement forms are substantially more complex (no one has yet attempted to apply the SoundX methodology to a language as large as ML.)

Systems like MetaBorg that require that the type of a derived form be known from context so that disambiguation can occur (see above) also address the problem of determining the type of a derived expression or pattern form as a whole. However, it is not always clear what the types of the spliced terms within these derived forms should be.

% As discussed in Sec. \ref{sec:problems-with-dialects}, languages with a rich type and binding structure are designed to minimize or eliminate the cognitive cost of answering analagous questions. These reasoning principles are central to the claim that such languages are suitable for ``programming in the large'' \cite{DeRemer76}. 

% Due to the problems described above, and the problem of conservatively combining syntax dialects, we do not integrate a syntax definition system into VerseML.% (There is, of course, no way to stop programmers from defining dialects of VerseML using any language-external syntax definition system of their choosing. Our goal is only to make this a less compelling option for library providers seeking only to capture idioms like those that we have discussed.) 

 % reasonable to obscure the type bind and binding structure by using these mechanisms.
% One approach would be to define both this encoding and the recursive labeled sum type \li{Rx} and define a parameterized module (i.e. a \emph{functor} in SML) that maps between them, given any module \li{R : RX}: 

% \begin{lstlisting}[numbers=none]
% structure RxHelper(R : RX) = 
% struct
%   fun to_R : Rx -> R.t = (* ... *)
%   fun of_R : R.t -> Rx = (* ... *)
% end
% \end{lstlisting}

% For example, given a particular module \li{R : RX}, we can generate the helper module \li{RH} as follows:

% \begin{lstlisting}[numbers=none]
% structure RH = RxHelper(R)
% \end{lstlisting}

% then, if we are using the VerseML/Rx syntax dialect, we can use the derived forms described previously in the argument position of \li{RH.to_r}:

% \begin{lstlisting}[numbers=none]
% let ssn = RH.to_R /SURL\d\d\d-\d\d\d\d-\d\d\dEURL/
% \end{lstlisting}

% One problem with this approach is that it makes using the spliced forms awkward. For example, consider writing the function \li{example_rx} in this manner:

% \begin{lstlisting}[numbers=none]
% fun example_R(name : string) => RH.to_R /SURL@EURLnameSURL: %(EURLRH.of_R ssnSURL)EURL/\end{lstlisting}

% Notice that we had to transform \li{ssn}, which is of type \li{R.t}, back into a value of type \li{Rx} in order to splice it into the expression above. The value of this expression is then immediately transformed back into a value of type \li{R.t} by \li{RH.to_R}. This is both syntactically awkward and incurs dynamic cost, i.e. it is an $\mathcal{O}(n)$ operation, where $n$ is the size of the regex being spliced. In this particular case, the cost may be negligible, but for large data structures, this may no longer be the case.




% \subsubsection{Direct Syntax Extension}\label{sec:direct-syntax-extension}
% One tempting alternative to dynamic string parsing is to use a system that gives the users of a language the power to directly extend its concrete syntax with new derived forms. %for regular expression patterns.% for patterns.

% The simplest such systems are those where the elaboration of each new syntactic form is defined by a single rewrite rule. For example, Gallina, the ``external language'' of the Coq proof assistant, supports such extensions \cite{Coq:manual}. A formal account of such a system has been developed by Griffin \cite{5134}. Unfortunately, a single equation is not enough to allow us to express pattern syntax following the usual conventions. For example, a system like Coq's cannot handle escape characters, because there is no way to programmatically examine a form when generating its expansion.

% Other syntax extension systems are more flexible. For example, many are based on context-free grammars, e.g.  Sugar* \cite{erdweg2013framework} and Camlp4 \cite{ocaml-manual} (amongst many others). Other systems give library providers direct programmatic access to the parse stream, like Common Lisp's \emph{reader macros} \cite{steele1990common} (which are distinct from its term-rewriting macros, described in Sec. \ref{sec:term-rewriting} below) and Racket's preprocessor \cite{Flatt:2012:CLR:2063176.2063195}. All of these would allow us to add pattern syntax into our language's grammar, perhaps following Unix conventions and supporting splicing syntax as described above:
% \begin{lstlisting}[numbers=none]
% let val ssn = /SURL\d\d\d-\d\d-\d\d\d\dEURL/
% fun example_shorter(name : string) => /SURL@EURLnameSURL: %EURLssn/
% \end{lstlisting}
% %The body of this function elaborates to the body of \lstinline{example_fixed} as shown above. 
% %Had we mistakenly written \lstinline{%name}, we would encounter only a static type error, rather than the  silent injection  vulnerability discussed above. 

% We sidestep the problems of dynamic string parsing described above  when we directly extend the syntax of our language using any of these systems. Unfortunately, direct syntax extension introduces serious new problems. First, the systems mentioned thus far cannot guarantee that {syntactic conflicts} between such extensions will not arise. As stated directly in the  Coq manual: ``mixing different symbolic notations in [the] same text may cause serious parsing ambiguity''. If another library provider used similar syntax for a different implementation or variant of regular expressions, or for some other unrelated construct, then a client could not simultaneously use both libraries at the same time. So properly considered, every combination of extensions introduced using these mechanisms creates a \emph{de facto} syntactic dialect of our language. The benefit of these systems is only that they lower the implementation cost of constructing syntactic dialects. % Resolving such parsing amibiguities is left to each client of the library. 

% In response to this problem, Schwerdfeger and Van Wyk developed a modular analysis that accepts only context-free grammar extensions that begin with an identifying starting token and obey certain constraints on  the follow sets of base language's non-terminals \cite{conf/pldi/SchwerdfegerW09}. Extensions that specify distinct starting tokens and that satisfy these constraints can be used together in any combination without the possibility of syntactic conflict. However, the most natural starting tokens like \lstinline{rx} cannot be guaranteed to be unique. To address this problem, programmers must agree on a convention for defining ``globally unique identifiers'', e.g. the common URI convention used on the web and by the Java packaging system. However, this forces us to use a more verbose token like \lstinline{edu_cmu_VerseML_rx}. There is no simple way for clients of our extension to define scoped abbreviations for starting tokens because this mechanism operates purely at the level of the context-free grammar.

% In particular, if det(H) and det(R) and the set of marking terminals on dialects such that if OK(H) and OK(R) and starttokens(H) disjoint from starttokens(R) then det(H cup R). This is not quite modular, in that we still need to check that the start tokens are disjoint at ``combination-time''. To be confident that this check will not fail, a community might adopt a social convention, e.g. using URIs as start tokens. 


% Putting this aside, we must also consider another modularity-related question: which particular module should the expansion use? Clearly, simply assuming that some module identified as \lstinline{R} matching \lstinline{RX} is in scope is a brittle solution. In fact, we should expect that the system actively prevents such capture of specific variable names to ensure that variables (here, module variables) can be freely renamed. Such a \emph{hygiene discipline} is well-understood only when performing term-to-term rewriting (discussed below) or in simple language-integrated rewrite systems like those found in Coq. For mechanisms that operate strictly at the level of context-free grammars or the parse stream, it is not clear how one could address this issue. The onus is then on the library provider to make no assumptions about variable names and instead require that the client explicitly identify the module they intend to use as an ``argument'' within the newly introduced form:
% \begin{lstlisting}[numbers=none]
% let val ssn = edu_cmu_VerseML_rx R /SURL\d\d\d-\d\d-\d\d\d\dEURL/
% \end{lstlisting}

% Another problem with the approach of direct syntax extension is that, given an unfamiliar piece of syntax, there is no straightforward method for determining what type it will have, causing difficulties for both humans (related to code comprehension) and tools. 

% \todo{Related work I haven't mentioned yet:}
% \begin{itemize}
% \item Fan: http://zhanghongbo.me/fan/start.html
% \item Well-Typed Islands Parse Faster: \\\url{http://www.ccs.neu.edu/home/ejs/papers/tfp12-island.pdf}
% \item User-defined infix operators
% \item SML quote/unquote 
% \item That Modularity paper
% \item Template Haskell and similar
% \end{itemize}
% \subsection{Rewriting Systems}\label{sec:term-rewriting}

\subsection{Non-Local Term Rewriting Systems}\label{sec:non-local-term-rewriting}
Another approach is to leave the textual syntax of the language fixed, but repurpose it for novel ends using a \emph{term rewriting system}. Term rewriting systems transform syntactically well-formed terms into other syntactically well-formed terms (unlike syntax definition systems, which operate on the program text.) 

Non-local term rewriting systems typically operate over an entire compilation unit (e.g. a file). For example, one could define a preprocessor that rewrites every string literal that is followed by the comment \li{(*rx*)} to the corresponding expression (or pattern) of type \li{rx}. For example, the following expression would be rewritten to a regex expression, with \li{dna} treated as a spliced subexpression as described in the previous section:
\begin{lstlisting}[numbers=none]
"SSTRGC%(dna)GCESTR"(*rx*)
\end{lstlisting}

OCaml 4.02 introduced \emph{preprocessor extension (ppx) points} into its textual syntax \cite{ocaml-manual}. Extension points serve as markers for the benefit of a non-local term rewriting system. They are less \emph{ad hoc} than comments, in that each extension point is associated with a single term in a well-defined way, and the compiler gives an error if any extension points remain after preprocessing is complete. For example, in the following program fragment, 
\begin{lstlisting}[numbers=none]
let%lwt (x, y) = f in x + y
\end{lstlisting}
the \li{%lwt} 
annotation on the let expression is recognized by a preprocessor distributed with \li{Lwt}, a lightweight threading library. This preprocessor rewrites this fragment to:
\begin{lstlisting}[numbers=none]
Lwt.bind f (fun (x, y) -> x + y)
\end{lstlisting}
The OCaml system is distributed with a library called \li{ppx_tools} that simplifies the task of writing  preprocessors that operate on terms annotated with extension points.

There are a number of other systems that support non-local term rewriting. For example, the \li{GHC} compiler for Haskell \cite{jones2001playing} and the \li{xoc} compiler for C \cite{conf/asplos/CoxBCKK08} both support user-defined non-local rewritings.

These systems present several difficulties with abstract reasoning, many of which are directly analagous to those that syntax definition systems present:
\begin{enumerate}
\item \textbf{Conflict:} Different preprocessors may recognize the same markers or code patterns.
\item \textbf{Responsibility:} It is not always clear which preprocessor handles each rewritten form.
\item \textbf{Localization:} A non-local term rewriting system might insert code anywhere in the program, complicating reasoning efforts.
\item \textbf{Segmentation:} It is not always clear where spliced terms appear inside rewritten string literal forms.
\item \textbf{Capture:} The rewriting might place terms under binders that shadow bindings visible in the program text.
\item \textbf{Context Dependence:} The rewriting might assume that certain identifiers are bound at particular locations, making it difficult to reason about refactoring.
\item \textbf{Typing:} It is not always clear what type the rewriting of a marked form will have (if indeed the rewriting happens to be local.) Similarly, the type that terms that appear within the rewritten form should have is often unclear.
\end{enumerate}  

\subsection{Term-Rewriting Macro Systems}\label{sec:macro-systems}
Macro systems are language-integrated local term rewriting systems, i.e. they allow programmers to designate functions that implement rewritings as macros. Clients apply macros directly to terms (e.g. expressions, patterns and other sorts of terms.). The rewritten term is known as the \emph{expansion} of the macro application.

Macro systems do not suffer from problems related to reasoning about \textbf{Conflict}, \textbf{Responsibility} and \textbf{Localization} described above because macros are applied explicitly and operate locally.

Na\"ive macro systems, like the earliest variants of the LISP macro system \cite{Hart63a}, early compile-time quotation expanders in ML \cite{mauny1994complete}, Template Haskell macros \cite{SheardPeytonJones:Haskell-02} and GHC quasiquotes \cite{mainland2007s}, do not escape from the remaining problems described above, because they can generate arbitrary code for insertion at the macro application site. For example, it is possible in early LISP dialects and in these other less disciplined modern macro systems to define a macro \li{rx!} that can be applied to rewrite a string form containing a spliced subexpression to a regex:
\begin{lstlisting}[numbers=none]
(rx! "SSTRGC%(dna)GCESTR")
\end{lstlisting}
The problem with these systems is that without examining the macro's implementation or the generated expansion, there is no way to reason about \textbf{Segmentation}, \textbf{Capture}, \textbf{Context Dependence} or \textbf{Typing}.\footnote{It is not enough that the generated expansions be typechecked -- it must be possible for the user to reason about \emph{what the type of the expansion is}.}

The problem of \textbf{Capture} was addressed by the design of Scheme's \emph{hygienic macro system} \cite{Kohlbecker86a,DBLP:conf/popl/Adams15,Herman10:Theory,DBLP:conf/esop/HermanW08,DBLP:journals/lisp/DybvigHB92,DBLP:conf/popl/ClingerR91}, which automatically alpha-renames identifiers bound in the expansion so that they do not shadow those that appear at the macro application site. %modern macro systems (e.g. Racket's macro systems \cite{DBLP:journals/jfp/FlattCDF12} and Scala's macro system \cite{ScalaMacros2013}) incorporate such a hygiene mechanism. %Some of these allow the macro provider to override the automatic renaming in situations where shadowing is explicitly intended, though \todo{some make allowances...}

The problem of \textbf{Context Dependence} is typically confronted by allowing macro expansions to explicitly refer only to those bindings in scope at the macro definition site. These references are preserved even if the identifiers involved have been shadowed at the macro application site \cite{DBLP:conf/popl/ClingerR91,DBLP:journals/lisp/DybvigHB92,DBLP:conf/popl/Adams15}. Any references to application site bindings must originate in one of the macro's arguments. There are two problems with this approach:
\begin{enumerate}
\item It does not make explicit which of the definition site bindings the expansions generated by a macro might refer to, so reasoning abstractly about the renaming of definition site bindings remains problematic.
\item Preventing access to the application site bindings makes defining a macro like \li{rx!} impossible, because spliced subexpressions (like \li{dna} above) do not appear as subexpressions of an argument to \li{rx!} -- they are parsed out of a string literal programmatically. From the perspective of the macro system, such spliced subexpressions are indistinguishable from inappropriate references to bindings tracked by the application site context. 

The only choice, then, is to repurpose other forms that do contain subexpressions. For example, the macro might repurpose infix operators that usually have a different meaning, e.g. \li{^}:
\begin{lstlisting}[numbers=none]
(rx! ("SSTRGCESTR" ^ dna ^ "SSTRGCESTR"))
\end{lstlisting}
This is rather confusing, in that it appears that string concatenation is occuring when that is not the case -- \li{rx!} is simply repurposing the infix \li{^} form.
\end{enumerate}

The problem of reasoning about \textbf{Typing} is relatively understudied, because most research on macro systems has been done in languages in the LISP tradition that do not define a rich static semantics. 

Herman and Wand's calculus of macros \cite{DBLP:conf/esop/HermanW08,Herman10:Theory} does use a type system to reason about the binding structure of the expansion that a macro generates, but the expansions themselves are not written in a language with rich type structure. 

Some macro systems for languages with non-trivial type structure, like Template Haskell \cite{SheardPeytonJones:Haskell-02}, do not support reasoning about types in that the guarantee is only that the expansion is well-typed -- clients cannot reason about \emph{what that type is}. 

Other macro systems, like MacroML \cite{ganz2001macros,Sheard:1999:UMS}, support reasoning about typing, but these systems are \emph{staging macro systems}, rather than \emph{term-rewriting macro systems}, meaning that the macro does not have access to the syntax tree of the arguments at all. Staging macros cannot be used for syntactic control -- macro application syntactically coincides with function application. These macro systems are instead motivated primarily by concerns about performance.

The Scala macro system is a notable example of a term-rewriting macro system that does allow reasoning about typing \cite{ScalaMacros2013}. In particular, Scala's ``black box'' macros include type annotations on the arguments. We are not aware of a typed macro system that has been integrated into a language with an ML-style module system. The main problem with Scala's macro system, then, is that it does not give us enough syntactic control -- we must repurpose Scala's existing syntactic forms, as discussed in point 2 above.

% The most immediate problem with using these for our example is that we are not aware of any such statically-typed macro system that integrates cleanly with an ML-style module system. In other words, macros cannot be parameterized by modules. However, let us imagine such a macro system. We could use it to repurpose string syntax  as follows:
% \begin{lstlisting}[numbers=none]
% let val ssn = rx R {rx|SSTR\d\d\d-\d\d-\d\d\d\dESTR|rx}
% \end{lstlisting}

% The definition of the macro \lstinline{rx} might look like this:
% \begin{lstlisting}
% macro rx[Q : RX](e) at Q.t {
%   static fun f(e : Exp) : Exp => case(e) {
%       StrLit(s) => (* regex parser here *)
%     | BinOp(Caret, e1, e2) => `SQTQ.Seq(Q.Str(%EQTe1SQT), %(EQTf e2SQT))EQT`
%     | BinOp(Plus, e1, e2) => `SQTQ.Seq(%(EQTf e1SQT), %(EQTf e2SQT))EQT`
%     | _ => raise Error
%   }
% }
% \end{lstlisting}

% Here, \lstinline{rx} is a macro parameterized by a module matching \lstinline{rx} (we identify it as \lstinline{Q} to emphasize that there is nothing special about the identifier \lstinline{R}) and taking a single argument, identified as \lstinline{e}. The macro specifies a type annotation, \lstinline{at Q.t}, which imposes the constraint that the expansion the macro statically generates must be of type \lstinline{Q.t} for the provided parameter \lstinline{Q}. This expansion is generated by a \emph{static function} that examines the syntax tree of the provided argument (syntax trees are of a type \lstinline{Exp} defined in the standard library; cf. SML/NJ's visible compiler \cite{SML/VisibleCompiler}). If it is a string literal, as in the example above, it statically parses the literal body to generate an expansion (the details of the parser, elided on line 3, would be entirely standard). 
% By parsing the string statically, we avoid the problems of dynamic string parsing for statically-known patterns. 

% For patterns that are constructed compositionally, we need to get more creative. For example, we might repurpose the infix operators that are normally used for other purposes to support string and pattern splicing, e.g. as follows:

% \begin{lstlisting}[numbers=none,escapechar=|]
% fun example_using_macro(name : string) => 
%   rx R (name ^ "SSTR: ESTR" + ssn)
% \end{lstlisting}

% The binary operator \lstinline{^} is repurposed to indicate a spliced string and \lstinline{+} is repurposed to indicate a spliced pattern. The logic for handling these forms can be seen above on lines 4 and 5, respectively. We assume that there is derived syntax available at the type \lstinline{Exp}, i.e. \emph{quasiquotation} syntax as in Lisp \cite{Bawd99a} and Scala \cite{shabalin2013quasiquotes}, here delimited by backticks and using the prefix \lstinline{%} to indicate a spliced value (i.e. unquote). 

% Having to creatively repurpose existing forms in this way limits the effect a library provider can have on cognitive cost (particularly when it would be desirable to express conventions that are quite different from the conventions adopted by the language). It also can create confusion for readers expecting parenthesized expressions to behave in a consistent manner. However,  this approach is preferable to direct syntax extension because it avoids many of the problems discussed above: there cannot be syntactic conflicts (because the syntax is not extended at all), we can define macro abbreviations because macros are integrated into the language, there is a hygiene discipline that guarantees that the expansion will not capture variables inadvertently, and by using a typed macro system, programmers need not examine the expansion to know what type the expansion produced by a macro must have. 

% \subsection{Active Libraries}
% The design we are proposing also has conceptual roots in earlier work on \emph{active libraries}, which similarly envisioned using compile-time computation to give library providers more control over various aspects of a programming system, including its concrete syntax (but did not take an approach rooted in the study of type systems) \cite{active-libraries-thesis}. 
% TODO FLESH THIS OUT 


% \part{Simple TLMs}\label{part:simple-tsms}
% !TEX root = omar-thesis.tex
\chapter{Simple Expression TLMs (seTLMs)}\label{chap:uetsms}
In the remainder of this work, we will develop a system of \emph{typed literal macros (TLMs)}. Briefly, TLMs offer substantially greater syntactic flexibility as compared to typed term rewriting macros \emph{a la} Scala, while 1) guaranteeing that a segmentation can always be produced; 2) enforcing a prohibition on capture; 3) enforcing a strong form of context independence and 4) maintaining the ability to reason abstractly about types. We will establish these reasoning principles formally, ultimately in a system with an ML-style module system in Chapter \ref{chap:ptsms}. We will begin, however, in this chapter with a simpler calculus of expressions and types. The TLMs available in this calculus are called \emph{simple expression TLMs} (seTLMs).

\section{Simple Expression TLMs By Example}\label{sec:tsms-by-example}
We begin in this section with a ``tutorial-style'' introduction to seTLMs in VerseML. %In particular, we will define an seTLM for constructing values of the recursive labeled sum type \li{rx} that was defined in Figure \ref{fig:datatype-rx}. 
Sec. \ref{sec:tsms-minimal-formalism} then formally defines a reduced dialect of VerseML called $\miniVerseUE$. This will serve as a ``conceptually minimal'' core calculus of TLMs, in the style of the simply typed lambda calculus.   %We conclude in Sec. \ref{sec:uetsms-discussion} 


\subsection{TLM Application}\label{sec:uetsms-usage}
The following VerseML expression, drawn textually, is of \emph{TLM application} form. Here, a TLM named \li{#\dolla#rx} is applied to the \emph{generalized literal form} \li{/SURLA|T|G|CEURL/}:
\begin{lstlisting}[numbers=none,mathescape=|]
$rx /SURLA|T|G|CEURL/
\end{lstlisting}
Generalized literal forms are left unparsed according to the context-free syntax of VerseML. Several other outer delimiters are also available, as summarized in Figure \ref{fig:literal-forms}. The client is free to choose any of these for use with any TLM, as long as the \emph{literal body} (shown in green above) satisfies the requirements stated in Figure \ref{fig:literal-forms}. For example, we could have equivalently written the example above as \li{#\dolla#rx `SURLA|T|G|CEURL`}. (In fact, this would have been convenient if we had wanted to express a regex containing forward slashes but not backticks.) 

It is only during the subsequent \emph{typed expansion} phase that the applied TLM parses the {body} of the literal form to generate a \emph{proto-expansion}. The language then \emph{validates} this proto-expansion according to criteria that we will describe in Sec. \ref{sec:uetsms-validation}. If proto-expansion validation succeeds, the language generates the \emph{final expansion} (or more concisely, simply the \emph{expansion}) of the TLM application. The behavior of the program is determined by its expansion. 

For example, the expansion of the TLM application above is equivalent to the following expression when the regex value constructors \li{Or} and \li{Str} are in scope:
\begin{lstlisting}[numbers=none]
Or(Str "SSTRAESTR", Or(Str "SSTRTESTR", Or(Str "SSTRGESTR", Str "SSTRCESTR")))
\end{lstlisting}
To avoid the assumption that the variables \li{Or} and \li{Str} are in scope at the TLM application site, the expansion actually uses the explicit \li{fold} and \li{inj} operators, as described in Sec. \ref{sec:lists}. In fact, the proto-expansion validation process enforces this notion of context independence -- we will return to proto-expansion validation below. (We will show how TLM parameters can reduce the awkwardness of this requirement in Chapter \ref{chap:ptsms}.)
%The constructors above are those of the type \li{Rx} that was defined in Figure \ref{fig:datatype-rx}.

% A number of literal forms, ,  are available in VerseML's concrete syntax. Any literal form can be used with any TLM,  TLMs have access only to the literal bodies. Because TLMs do not extend the concrete syntax of the language directly, there cannot be syntactic conflicts between TLMs.

 %The form does not directly determine the expansion. 

\begin{figure}
\begin{lstlisting}
'SURLbody cannot contain an apostropheEURL'
`SURLbody cannot contain a backtickEURL`
[SURLbody cannot contain unmatched square bracketsEURL]
{|SURLbody cannot contain unmatched barred curly bracesEURL|}
/SURLbody cannot contain a forward slashEURL/
\SURLbody cannot contain a backslashEURL\
\end{lstlisting}
%SURL<tag>body includes enclosing tags</tag>EURL
\caption[Generalized literal forms available in VerseML]{Generalized literal forms available for use in VerseML's textual syntax. The characters in green indicate the literal bodies and describe how the literal body is constrained by the form shown on that line. The Wyvern language defines additional forms, including whitespace-delimited forms \cite{TSLs} and multipart forms \cite{sac15}, but for simplicity we leave these out of VerseML.}
\label{fig:literal-forms}
\end{figure}
\subsection{TLM Definitions}\label{sec:uetsms-definition}
%The original expression, above, is statically rewritten to this expression.
The definition of \lstinline{#\dolla#rx} takes the following form:
\begin{lstlisting}[numbers=none,mathescape=|]
syntax $rx at rx by 
  static fn(b : body) -> parse_result(proto_expr) => 
    (* regex literal parser here *)
end
\end{lstlisting}
Every seTLM definition consists of a TLM name, here \li{#\dolla#rx}, a \emph{type annotation}, here \lstinline{at rx}, and a \emph{parse function} between \li{by} and \li{end}. TLM definitions follow standard scoping rules -- unless an \li{in} clause is provided, the definition is in scope until the end of the enclosing declaration (e.g. the enclosing function or module.) We will consider how TLM definitions are packaged into libraries in Chapter \ref{chap:static-eval}.

All TLM names must begin with the dollar symbol (\li{#\dolla#}), which distinguishes them from variables. This is inspired by the Rust macro system, which uses post-fix exclamation points (\li{!}) to distinguish macro identifiers \cite{Rust/Macros}.

The {parse function} is a \emph{static function} delegated responsibility over parsing the literal bodies of the literal forms to which the TLM is applied. Static functions, marked by the \li{static} keyword, are applied during the typed expansion process, so they cannot refer to the surrounding variable bindings (because those variables stand for dynamic values.) For now, we will simply assume that static functions are closed and do not themselves make use of TLMs (we will eliminate these impractical limitations in Chapter \ref{chap:static-eval}.)

Every seTLM parse function must have type \li{body -> parse_result(proto_expr)}. The input type, \lstinline{body}, classifies encodings of literal {bodies}. In VerseML, literal bodies are sequences of characters, so it suffices to define \li{body} as an abbreviation for the \li{string} type, as shown in Figure \ref{fig:indexrange-and-parseresult}.\footnote{In languages where the surface syntax is not textual, \li{body} would have a different definition, but we leave explicit consideration of such languages as future work (see Sec. \ref{sec:future-work}.)} The return type is a labeled sum type, defined by applying the type function \li{parse_result} defined in Figure \ref{fig:indexrange-and-parseresult}, that distinguishes between parse errors and successful parses.\footnote{\li{parse_result} is defined as a type function because in Chapter \ref{chap:uptsms}, we will introduce pattern TLMs, which generate patterns rather than expressions.} Let us consider these two possibilities in turn.
\begin{figure}
\begin{lstlisting}[numbers=none]
type body = string

type segment = {startIdx : int, endIdx : int} (* inclusive *)
type parse_result('a) = ParseError of {
                         msg : string, loc : segment
                        }
                      + Success of 'a 
\end{lstlisting}
\caption[Definitions of \li{body}, \li{segment} and \li{parse_result} in VerseML]{Definitions of \li{body}, \li{segment} and \li{parse_result}. These type definitions are given in the VerseML \emph{prelude}, which is a small collection of definitions available ambiently.}
\label{fig:indexrange-and-parseresult}
\end{figure}

\paragraph{Parse Errors} If the parse function determines that the literal body is not well-formed (according to whatever syntax definition that it implements), it returns:
\begin{lstlisting}[numbers=none]
inj[ParseError]({msg=#$e_\text{msg}$#, loc=#$e_\text{loc}$#})
\end{lstlisting}
where $e_\text{msg}$ is an error message and $e_\text{loc}$ is a value of type \li{segment}, defined in Figure \ref{fig:indexrange-and-parseresult}, that designates a segment of the literal body as the location of the error \cite{DBLP:journals/jsc/DeursenKT93}. This information is for use by VerseML compilers when reporting the error to the programmer.

% In languages where the surface syntax is not textual, the definition of \li{loc} would need to designate 


\paragraph{Successful Parses} If parsing succeeds, the parse function returns 
\begin{lstlisting}[numbers=none]
inj[Success](#$\ecand$#)
\end{lstlisting} 
where $\ecand$ is called the \emph{encoding of the proto-expansion}. 

For expression TLMs, proto-expansions are \emph{proto-expressions}, which are encoded as VerseML values of the type \lstinline{proto_expr} defined in Figure \ref{fig:candidate-exp-verseml}.
Most of the variants defined by \li{proto_expr} are individually uninteresting -- they encode VerseML's various expression forms (just as in a compiler, c.f. SML/NJ's Visible Compiler library \cite{SML/VisibleCompiler}.) 
Expressions can mention types, so we also need to define a type \li{proto_typ} in Figure \ref{fig:candidate-exp-verseml}. As we enrich our language in later chapters, we will need to define more encodings like these, for other sorts of trees. The only non-standard classes are \li{SplicedT} and \li{SplicedE} -- these are \emph{references to spliced unexpanded types and expressions}, which we will return to when we consider splicing in Sec. \ref{sec:splicing-and-hygiene} below. 

The definitions of \li{proto_typ} and \li{proto_expr} are recursive labeled sum types to simplify our exposition, but we could have chosen alternative encodings, e.g. based on abstract binding trees \cite{pfpl}, with only minor modifications to our semantics. Indeed, when we formally define seTLMs in Sec. \ref{sec:miniVerseU}, we abstract over the particular encoding.

% We will show a complete encoding when we describe our reduced formal system $\miniVerseUE$ in Sec. \ref{sec:tsms-minimal-formalism}. 
% ; the remaining constructors (some of which are elided for concision) encode the abstract syntax of VerseML expressions and types.  % It is extended with one additional form used to handled spliced subexpressions, 


%Notice that the types just described are those that one would expect to find in a typical parser.

%One would find types analagous to those just described in any parser, so for concision, we elide the details of \li{#\dolla#rx}'s parse function.
%The parse function must treat the TLM parameters parametrically, i.e. it does not have access to any values in the supplied module parameter. Only the expansion the parse function generates can refer to module parameters. 
%For example, the following definition is ill-sorted:
%\begin{lstlisting}[numbers=none]
%syntax pattern_bad[Q : PATTERN] at Q.t {
%  static fn (body : Body) : Exp => 
%    if Q.flag then (* ... *) else (* ... *)
%}
%\end{lstlisting}%So the parse function parses the body of the delimited form to generate an encoding of the elaboration.

\begin{figure}
\begin{lstlisting}[numbers=none]
type proto_typ = rec(proto_typ => 
                 TyVar of var_t 
               + Arrow of proto_typ * proto_typ 
               + (* ... *) 
               + SplicedT of segment)

type proto_expr = rec(proto_expr => 
                  Var of var_t 
                + Fn of var_t * proto_typ * proto_expr
                + Ap of proto_expr * proto_expr
                + (* ... *) 
                + SplicedE of segment * proto_typ)
\end{lstlisting}
\caption[Abbreviated definitions of \li{proto_typ} and \li{proto_expr} in VerseML]{Abbreviated definitions \li{proto_typ} and \li{proto_expr} in the VerseML prelude. We assume some suitable type \li{var_t} exists, not shown.}
\label{fig:candidate-exp-verseml}
% \vspace{-5px}
\end{figure}


\subsection{Splicing}\label{sec:splicing-and-hygiene}
As described thusfar, TLMs operate just like term-rewriting macros over string literals. TLMs therefore do not cause difficulties related to reasoning about \textbf{Conflict}, \textbf{Responsibility} or \textbf{Localization}, for exactly the reasons discussed in Sec. \ref{sec:macro-systems}. TLMs differ from term-rewriting macros in that they support \emph{splicing out arbitrary types and expressions} (including those that may themselves involve TLM applications) from within literal bodies in a reasonable manner. For example, the program fragment from Figure \ref{fig:derived-spliced-subexpressions} can be expressed using the \li{#\dolla#rx} TLM as follows:
\begin{lstlisting}[numbers=none]
val ssn = $rx /SURL\d\d\d-\d\d-\d\d\d\dEURL/ 
fun lookup_rx(name: string) => $rx /SURL@name: %ssnEURL/ 
\end{lstlisting}
The expressions \lstinline{name} and \lstinline{ssn} on the second line appear spliced within the literal body, so we call them \emph{spliced expressions}. 

When \li{#\dolla#rx}'s parse function determines that a subsequence of the literal body should be taken as a spliced expression (here, by recognizing the characters \lstinline{@} or \lstinline{%} followed by a variable or parenthesized expression), it does not directly insert the syntax tree of that expression into the encoding of the expansion. Instead, 
the TLM must refer to the spliced expression \emph{by its relative location} within the literal body using the \li{SplicedE} variant of \li{proto_expr}. 
In particular, the \li{SplicedE} variant requires a value of type \li{segment}, which indicates the zero-indexed location of the spliced expression relative to the start of the literal body provided to the parse function. 
The \li{SplicedE} variant also requires a value of type \li{proto_typ}, which indicates the type that the spliced expression is expected to have. 
For example, the proto-expansion generated by \li{#\dolla#rx} for the literal body on the second line above, if written in a  textual syntax for proto-expressions where references to spliced expressions are \li{spliced<startIdx; endIndex; ty>}, is:
\begin{lstlisting}[numbers=none]
Seq(Str(spliced<1; 4; string>), 
    Seq(Str "SSTR: ESTR", spliced<8; 10; rx>))
\end{lstlisting}
 Here, \li{spliced<1; 4; string>} refers to the spliced string expression \li{name} by location and \li{spliced<8; 10; rx>} refers to the spliced regex expression \li{ssn} by location. 
(For clarity of exposition, we again use the regex value constructors to abbreviate applications of the \li{fold} and \li{inj} operators and use the type abbreviation \li{rx}. In fact, given only the mechanisms introduced in this chapter, these abbreviations would need to be explicitly included in each proto-expansion.)

Proto-types can make reference to spliced types by using the \li{SplicedT} variant of \li{proto_typ} analagously.

Requiring that the TLM refer to spliced terms indirectly in this manner prevents it from ``forging'' a spliced expression (i.e. claiming that an expression is a spliced expression when it does not appear in the literal body.) This will be formally critical to being able to reason abstractly about segmentation, capture and context-independence, as we will detail below.


% The parse function can similarly extract \emph{spliced types} from a literal body using the \li{SplicedT} variant of \li{proto_typ}. %In particular, the parse function must provide the index range of spliced subexpressions to the \li{Spliced} constructor of the type \li{MarkedExp}. %Only subexpressions that actually appear in the body of the literal form can be marked as spliced subexpressions.

%For example, had the  would not be a valid expansion, because the  that are not inside spliced subexpressions:
%\begin{lstlisting}[numbers=none]
%Q.Seq(Q.Str(name), Q.Seq(Q.Str ": ", ssn))
%\end{lstlisting}


\subsection{Segmentations}
The \emph{segmentation} of a proto-expression is the finite set of references to spliced terms within the proto-expression. For example, the summary of the proto-expression above is the finite set containing only \li{spliced<1; 4; string>} and \li{spliced<8; 10; rx>}.% Notice that no information about  the spliced terms appear is communicated by the splice summary.

The semantics checks that all of the locations in the segmentation are 1) in bounds relative to the literal body; 2) non-overlapping; and 3) used at a consistent sort and type. 
This resolves the problem of \textbf{Segmentation} described in Secs. \ref{sec:non-local-term-rewriting}-\ref{sec:macro-systems}, i.e. every literal body in a well-typed program has a well-defined segmentation. 

A program editor or pretty-printer can communicate the segmentation information to the programmer, e.g. by coloring non-spliced segments green as is our convention in this document:
\begin{lstlisting}[numbers=none]
val ssn = $rx /SURL\d\d\d-\d\d-\d\d\d\dEURL/
fun lookup_rx(name: string) => $rx /SURL@EURLnameSURL: %EURLssn/ 
\end{lstlisting}

A program editor or pretty-printer can also communicate the type of each spliced term, as indicated in the segmentation, to the programmer upon request (for example, the Emacs and Vim packages for working with OCaml defer to the Merlin tool when the programmer requests the type of an expression \cite{Merlin}.)


\subsection{Proto-Expansion Validation}\label{sec:uetsms-validation}
Three potential problems described in Secs. \ref{sec:non-local-term-rewriting}-\ref{sec:macro-systems} remain: those related to reasoning abstractly about \textbf{Capture}, \textbf{Context Dependence} and \textbf{Typing}. Addressing these problems is the purpose of the \emph{proto-expansion validation} process.


\subsubsection{Capture}
Proto-expansion validation ensures that spliced terms have access \emph{only} to the bindings that appear at the application site -- spliced terms cannot capture the bindings that appear in the proto-expansion. For example, suppose that  \li{#\dolla#rx} generated a proto-expansion of the following  form (drawn as above):
\begin{lstlisting}[numbers=none]
let tmp = (* ... expansion-internal tmp ... *) in 
Seq(tmp, spliced<1; 3; rx>)
\end{lstlisting}
Na\"ively, the binding of the variable \li{tmp} here could shadow bindings of \li{tmp} that appear at the application site within the indicated spliced expression. For example, consider the following application site:
\begin{lstlisting}[numbers=none]
let tmp = (* ... application site tmp ... *) in 
$rx /SURL%EURLtmp/
\end{lstlisting}
Here, the application site binding of \li{tmp} would be shadowed by the ``invisible'' binding of \li{tmp} in the expansion of the TLM application. % The possibility that this might occur makes it impossible to reason abstractly about binding within a spliced term -- i.e. it is unclear whether \li{tmp}. 

To address this problem, proto-expansion validation enforces a prohibition on capture. This prohibition on capture can be silently enforced by implicitly alpha-varying the bindings in the proto-expansion as needed, as in hygienic term-rewriting macro systems (cf. Sec. \ref{sec:macro-systems}.) For example, the expansion of the example above might take the following form:
\begin{lstlisting}[numbers=none]
let tmp = (* ... application site tmp ... *) in 
let tmp' = (* ... expansion-internal tmp ... *) in 
Seq(tmp', tmp)
\end{lstlisting}
Notice that the expansion-internal binding of \li{tmp} has been alpha-varied to \li{tmp'} to avoid shadowing the application site binding of \li{tmp}. As such, the reference to \li{tmp} in the spliced expression refers, as intended, to the application site binding of \li{tmp}.

For TLM providers, the benefit of this mechanism is that they can name the variables used internally within expansions freely, without worrying about whether their chosen identifiers might shadow those that a client might have used at the application site. There is no need for a user-facing mechanism that generates ``fresh variables''.

TLM clients can, in turn, reliably reason about binding within every spliced expression without examining the expansion that the spliced expression appears within.

The trade-off is that this prevents library providers from defining  alternative binding forms. For example, Haskell's derived form for monadic commands (i.e. \li{do}-notation) supports binding the result of executing a command to a variable that is then available in the subsequent commands in a command sequence. In VerseML, this cannot be expressed in the same way. Values can be communicated from the expansion to a spliced expression only via function arguments. 
%We will show an alternative formulation of Haskell's syntax for monadic commands that uses VerseML's anonymous function syntax to bind variables in Sec. \ref{sec:application-monadic-commands}. 
We will return to this example when we consider other possible points in this design space in Sec. \ref{sec:controlled-binding}.


\subsubsection{Context Dependence}
%The prohibition on shadowing ensures only that variables that appear in spliced terms do not refer to bindings that appear in the surrounding expansion. 
The proto-expansion validation process also ensures that variables that appear in the proto-expansion do not refer to bindings that appear at the TLM definition or application site. In other words, expansions must be completely \emph{context independent} -- a TLM definition can make no assumptions about the application site context.

A minimal example of a ``broken'' TLM that does not generate context-independent proto-expansions is given below:
\begin{lstlisting}[numbers=none]
syntax $bad1 at rx by
  static fn(_) => Success (Var "SSTRxESTR")
end
\end{lstlisting}
The proto-expansion that this TLM generates (for every literal body) refers to a variable \li{x} that is not bound within the expansion. If proto-expansion validation permitted such a proto-expansion, it would be well-typed only under those application site typing contexts where \li{x} is bound. This ``hidden assumption'' makes reasoning about binding and renaming difficult, so this proto-expansion is deemed invalid (even when \li{#\dolla#bad1} is applied where \li{x} is coincidentally bound.)

Of course, this prohibition does not extend into the spliced terms in a proto-expansion -- spliced terms appear at the application site, so they can justifiably refer to application site bindings. The client's ability to hold the expansion abstract is retained. We saw examples of spliced terms that referred to variables bound at the application site  -- \li{name} and \li{ssn} -- in Sec. \ref{sec:splicing-and-hygiene}. Because proto-expansions refer to spliced terms indirectly, and forging is impossible, enforcing context independence is straightforward -- we need only that the proto-expansion itself be closed, without considering the spliced terms.% In the next section, we will formalize this intuition. % The TLM provider can only refer to them opaquely.

This prohibition on context dependence explains why the expansion generated by the TLM application in Sec. \ref{sec:uetsms-usage} cannot make use of the regex value constructors, e.g. \li{Str} and \li{Or}, directly. (In Chapter \ref{chap:ptsms}, we will relax this restriction to allow proto-expansions to access explicit parameters.)

Collectively, we refer to the prohibition on capture and the prohibition on context dependence as \emph{hygiene properties}, by conceptual analogy to corresponding properties in term-rewriting macro systems (see Sec. \ref{sec:macro-systems}.) The novelty here comes from the fact that spliced terms are being extracted from an initially unparsed sequence of characters.
% In the examples in Sec. \ref{sec:uetsms-usage} and Sec. \ref{sec:splicing-and-hygiene}, the expansion used constructors associated with the \li{Rx} type, e.g. \li{Seq} and \li{Str}. This might appear to violate our prohibition on context-dependent expansions. This is not the case only because in VerseML, constructor labels are not variables or scoped symbols. Syntactically, they must begin with a capital letter (like Haskell's datatype constructors). Different labeled sum types can use common constructor labels without conflict because the type the term is being checked against -- e.g. \li{Rx}, due to the type ascription on \li{#\dolla#rx} -- determines which type of value will be constructed. For dialects of ML where datatype definitions do introduce new variables or scoped symbols, we need parameterized TLMs. We will return to this topic in Chapter \ref{chap:ptsms}. % Indeed, we used the label \li{Spliced} for two different recursive labeled sum types in Figure \ref{fig:candidate-exp-verseml}.

\vspace{-8px}
\subsubsection{Typing}\vspace{-4px}
Finally, proto-expansion validation maintains a reasonable \emph{typing discipline} by:
\begin{enumerate}
\item checking each spliced expression against the type indicated in the summary; and 
\item checking to ensure that the generated expansion is of the type specified in the TLM's type annotation. For example, the type annotation on \li{#\dolla#rx} is \li{at rx}, so proto-expansion validation ensures that the final expansion is of type \li{rx}.
\end{enumerate}

 This addresses the problem of reasoning abstractly about \textbf{Typing} described in Secs. \ref{sec:non-local-term-rewriting}-\ref{sec:macro-systems}, i.e.:
 \begin{enumerate}
 \item determining the type that a spliced expression must have requires only the information in the summary of the proto-expansion (rather than complete knowledge of the proto-expansion); and 
 \item determining the type of an expansion requires examining only the type annotation on the TLM definition (much as determining the type of a function application requires examining only the type of the function.)
 \end{enumerate}


% The language \emph{validates} proto-expansions before a final expansion is generated. One aspect of proto-expansion validation is checking  the proto-expansion against the type annotation specified by the TLM, e.g. the type \li{Rx} in the example above. This maintains a \emph{type discipline}: if a programmer sees a TLM being applied when examining a well-typed program, they need only look up the TLM's type annotation to determine the type of the generated expansion. Determining the type does not require examining the expansion directly.


% \subsection{Hygiene}
% The spliced subexpressions that the proto-expansion refers to (by their position within the literal body, cf. above) must be parsed, typed and expanded during the proto-expansion validation process (otherwise, the language would not be able to check the type of the proto-expansion). To maintain a useful \emph{binding discipline}, i.e. to allow programmers to reason also about variable binding without examining expansions directly, the validation process maintains two additional properties related to spliced subexpressions: \textbf{context independent expansion} and \textbf{expansion independent splicing}. These are collectively referred to as the \emph{hygiene properties} (because they are conceptually related to the concept of hygiene in term rewriting macro systems, cf. Sec. \ref{sec:term-rewriting}.) 

% \paragraph{Context Independent Expansion} 

% \paragraph{Expansion Independent Splicing} 
% %These properties suffice to ensure that programmers and tools can freely rename a variable without changing the meaning of the program. The only information that is necessary to perform such a \emph{rename refactoring} is the locations of spliced subexpressions within all the literal forms for which the variable being renamed is in scope; the expansions need not otherwise be examined. It would be straightforward to develop a tool and/or editor plugin to indicate the locations of spliced subexpressions to the user, like we do in this document (by coloring spliced subexpressions black). We discuss tool support as future work in Sec. \ref{sec:interaction-with-tools}.

\subsection{Final Expansion}
The result of proto-expansion validation is the \emph{final expansion}, which is simply the proto-expansion with  references to spliced terms replaced with their own final expansions. For example, the final expansion of the body of \li{lookup_rx} is equivalent to the following, assuming that the regex value constructors were defined (not shown):
\begin{lstlisting}[numbers=none]
Seq(Str(name), Seq(Str "SSTR: ESTR", ssn))
\end{lstlisting}

\subsection{Comparison to the Dialect-Oriented Approach}
Let us compare the VerseML TLM \li{#\dolla#rx} to $\mathcal{V}_\texttt{rx}$, the hypothetical syntactic dialect of VerseML with support for derived forms for values of type \li{rx} described in Sec. \ref{sec:examples-of-syntax-dialects}.

Both $\mathcal{V}_\texttt{rx}$ and \li{#\dolla#rx} give programmers the ability to use the same standard POSIX syntax for constructing regexes, extended with the same syntax for splicing in strings and other regexes. Using \li{#\dolla#rx}, however, we incur the additional syntactic cost of explicitly applying the \li{#\dolla#rx} TLM each time we wish to use regex syntax. This cost does not grow with the size of the regex, so it would only be significant in programs that involve a large number of small regexes (which do, of course, exist.) In Chapter \ref{chap:tsls} we will consider a design where even this syntactic cost can be eliminated in  positions where the type is known to be \li{rx}.

The benefit of the TLM-based approach is that we can easily define other TLMs to use alongside the \li{#\dolla#rx} TLM without needing to consider the possibility of syntactic conflict. Furthermore, programmers can rely on the binding discipline and the typing discipline enforced by proto-expansion validation to reason about programs, including those that contain unfamiliar forms. Put pithily, VerseML helps programmers avoid ``conflict and confusion''. 

% \begin{figure}
% \begin{lstlisting}
% val a = get_a()
% val w = get_w()
% val x = read_data(a)
% val y = $k {|SURL(!R)@&{&/EURLxSURL!/:2_!EURLxSURL}'!R}EURL|}
% \end{lstlisting}
% \caption{The example from Figure \ref{fig:K-dialect}, written using a TLM.}
% \label{fig:K-tsms}
% \end{figure}

% To underline this point, consider the program fragment in Figure \ref{fig:K-tsms}, which is based on the example involving the K query language from Sec. \ref{fig:K-dialect}. The programmer need not be familiar with the syntax of K, or examine the expansion itself, to answer questions corresponding to those posed in Sec. \ref{fig:K-dialect}. In particular, the programmer knows that:
% \begin{enumerate}
% \item The TLM named \li{#\dolla#k} is responsible for parsing the body of the literal on Line 4. 
% \item The character \li{x} inside the literal body is parsed as a ``spliced'' expression, \li{x}, as indicated by our visualization of the segmentation. The other characters, e.g. \li{R}, are definitively not spliced expressions.
% \item The spliced expression \li{x} definitively refers to the binding of \li{x} on the previous line. No other binding of \li{x} could have shadowed this binding, due to the prohibition on capture.
% \item The TLM application on Line 4 must be context-independent, so it cannot have referred to \li{w}.
% \item We need only look at the type annotation on \li{#\dolla#k} to determine the type of \li{y}. For example, if that declaration takes the following form, we know definitively that \li{y} has type \li{kquery} (without examining the elided parse function):
% \begin{lstlisting}[numbers=none]
% syntax $k at kquery by (* ... *) end
% \end{lstlisting}
% The summary gives the types of the spliced expressions.
% \end{enumerate}

\vspace{-3px}
\section{\texorpdfstring{$\miniVerseUE$}{miniVerseSE}}\label{sec:tsms-minimal-formalism}\label{sec:miniVerseU}
\vspace{-3px}
% \begin{figure}[p!]
% $\begin{array}{lllllll}
% \textbf{variables} & \textbf{type variables} & \textbf{labels} & \textbf{label sets} & \textbf{TLM variables} & \textbf{literal bodies} & \textbf{nats}\\
% x & t & \ell & \labelset & \tsmv & b & n\\~\end{array}$\\
% $\begin{array}{ll}
% \textbf{type formation contexts} & \textbf{typing contexts}\\
% \Delta ::= \emptyset ~\vert~ \Delta, t & \Gamma ::= \emptyset ~\vert~ \Gamma, x : \tau\\
% ~
% \end{array}$\\
% ~\\
% $\begin{array}{lcl}
% \gheading{types}\\
% \tau & ::= & t ~\vert~ \parr{\tau}{\tau} ~\vert~ \forallt{t}{\tau} ~\vert~ \rect{t}{\tau} ~\vert~  \prodt{\mapschema{\tau}{i}{\labelset}} ~\vert~ \sumt{\mapschema{\tau}{i}{\labelset}}\\
% ~\\
% \gheading{expanded expressions}\\
% e & ::= & x ~\vert~ \lam{x}{\tau}{e} ~\vert~ \app{e}{e} ~\vert~ \Lam{t}{e} ~\vert~ \App{e}{\tau} ~\vert~ \fold{t}{\tau}{e} ~\vert~ \unfold{e} ~\vert~ \tpl{\mapschema{e}{i}{\labelset}} ~\vert~ \prj{e}{\ell} \\
% & \vert & \inj{\ell}{e} ~\vert~ \caseof{e}{\mapschemab{x}{e}{i}{\labelset}}\\
% ~\\
% \gheading{TLM expressions}\\
% \tsme & ::= & \tsmv ~\vert~ \utsmdef{\tau}{\ue}\\
% ~\\
% \gheading{unexpanded expressions}\\
% \ue & ::= & {x} ~\vert~ \lam{x}{\tau}{\ue} ~\vert~ \ue(\ue) ~\vert~ \Lam{t}{\ue} ~\vert~ \App{\ue}{\tau} ~\vert~ \fold{t}{\tau}{\ue} ~\vert~ \unfold{\ue} ~\vert~ \tpl{\mapschema{\ue}{i}{\labelset}} ~\vert~ \prj{\ue}{\ell} \\
% & \vert & \inj{\ell}{\ue} ~\vert~ \caseof{\ue}{\mapschemab{x}{\ue}{i}{\labelset}}\\
% & \vert & \uesyntax{\tsmv}{\tsme}{\ue} ~\vert~ \utsmapp{\eta}{b}\\
% ~\\
% \gheading{proto-types}\\
% \mtau & ::= & t ~\vert~ \parr{\mtau}{\mtau} ~\vert~ \forallt{t}{\mtau} ~\vert~ \rect{t}{\mtau} ~\vert~ \prodt{\mapschema{\tau}{i}{\labelset}} ~\vert~ \sumt{\mapschema{\mtau}{i}{\labelset}} \\
% & \vert & \mtspliced{\tau}\\
% ~\\
% \gheading{proto-expressions}\\
% \me & ::= & x ~\vert~ \lam{x}{\mtau}{\me} ~\vert~ \app{\me}{\me} ~\vert~ \Lam{t}{\me} ~|~ \App{\me}{\mtau} ~\vert~ \fold{t}{\mtau}{\me} ~\vert~ \unfold{\me} ~\vert~ \tpl{\mapschema{\me}{i}{\labelset}} ~\vert~ \prj{\me}{\ell} \\
% & \vert & \inj{\ell}{\me} ~\vert~ \caseof{\me}{\mapschemab{x}{\me}{i}{\labelset}}\\
% & \vert & \mspliced{e}
% % \\~
% \end{array}$
% \todo{finish breaking this up into syntax tables}
% \caption[Syntax of $\miniVerseUE$]{Syntax of $\miniVerseUE$. The forms $\mapschema{V}{i}{\labelset}$ and $\mapschemab{x}{V}{i}{\labelset}$ where $V$ is a metavariable indicate finite functions from each label $i \in \labelset$ to a term, $V_i$, or binder, $x_i.V_i$, respectively.}
% \label{fig:lambda-tsm-syntax}
% \end{figure}

To make the intuitions developed in the previous section precise, we will now introduce a reduced dialect of VerseML called $\miniVerseUE$ that supports seTLMs. 
The full definition of $\miniVerseUE$ is given in Appendix \ref{appendix:miniVerseSES} for reference. In the exposition below, we will reproduce only particularly noteworthy rules and proof cases. Rule and theorem numbers below refer to corresponding rules and theorems in the appendix.
%For reference, the syntax of $\miniVerseUE$ is specified in Figure \ref{fig:lambda-tsm-syntax}. We will reproduce relevant portions of this specification inline (in tabular form) as we continue. 
%We specify all formal systems in this document within the metatheoretic framework detailed in \emph{PFPL} \cite{pfpl}, and assume familiarity of fundamental background concepts (e.g. abstract binding trees, substitution, implicit identification of terms up to $\alpha$-equivalence, structural induction and rule induction) covered therein. %Familiarity with other accounts of typed lambda calculi should also suffice to understand the formal systems in this document. 

\subsection{Overview}

$\miniVerseUE$ consists of a language of \emph{unexpanded expressions} (the \emph{unexpanded language}, or \emph{UL}) defined by typed expansion to a  language of \emph{expanded expressions} (the \emph{expanded language}, or \emph{XL}.) We will begin with a brief overview of the standard XL before turning our attention to the UL in the remainder of this chapter.

\subsection{Syntax of the Expanded Language}\label{sec:U-expanded-terms}

\begin{figure}
%\hspace{-5px}
\[\begin{array}{lllllll}
\textbf{Sort} & & & \textbf{Operational Form} %& \textbf{Stylized Form} 
& \textbf{Description}\\
\mathsf{Typ} 
& \tau & ::= & t 
%& t 
& \text{variable}\\
&&& \aparr{\tau}{\tau} 
%& \parr{\tau}{\tau} 
& \text{partial function}\\
&&& \aall{t}{\tau} 
%& \forallt{t}{\tau} 
& \text{polymorphic}\\
&&& \arec{t}{\tau} 
%& \rect{t}{\tau} 
& \text{recursive}\\
&&& \aprod{\labelset}{\mapschema{\tau}{i}{\labelset}} 
%& \prodt{\mapschema{\tau}{i}{\labelset}} 
& \text{labeled product}\\
&&& \asum{\labelset}{\mapschema{\tau}{i}{\labelset}} 
%& \sumt{\mapschema{\tau}{i}{\labelset}} 
& \text{labeled sum}\\
\mathsf{Exp} & e & ::= & x 
%& x 
& \text{variable}\\
&&& \aelam{\tau}{x}{e} 
%& \lam{x}{\tau}{e} 
& \text{abstraction}\\
&&& \aeap{e}{e} 
%& \ap{e}{e} 
& \text{application}\\
&&& \aetlam{t}{e} 
%& \Lam{t}{e} 
& \text{type abstraction}\\
&&& \aetap{e}{\tau} 
%& \App{e}{\tau} 
& \text{type application}\\
&&& \aefold{e} 
%& \fold{e} : \tau 
& \text{fold}\\
&&& \aeunfold{e} 
%& \unfold{e} 
& \text{unfold}\\
&&& \aetpl{\labelset}{\mapschema{e}{i}{\labelset}} 
%& \tpl{\mapschema{e}{i}{\labelset}} 
& \text{labeled tuple}\\
&&& \aepr{\ell}{e} 
%& \prj{e}{\ell} 
& \text{projection}\\
&&& \aein{\ell}{e} 
%& \inj{\ell}{e} 
& \text{injection}\\
&&& \aecase{\labelset}{e}{\mapschemab{x}{e}{i}{\labelset}} 
%& \caseof{e}{\mapschemab{x}{e}{i}{\labelset}} 
& \text{case analysis}
\end{array}\]
\caption[Syntax of the $\miniVerseUE$ expanded language (XL)]{Syntax of the $\miniVerseUE$ expanded language (XL)%When using stylized forms, the label set is omitted when it can be inferred, e.g. the labeled product type $\prodt{\finmap{\mapitem{\ell_1}{e_1}, \mapitem{\ell_2}{e_2}}}$ leaves the label set $\{\ell_1, \ell_2\}$ implicit. 
% When we use the stylized forms, we assume that the reader can infer suppressed indices and arguments from the surrounding context.
}
\label{fig:U-expanded-terms}
\end{figure}

\noindent The syntax chart in Figure \ref{fig:U-expanded-terms} defines the syntax of \emph{types}, $\tau$, and \emph{(expanded) expressions}, $e$. Metavariables $x$ range over expression variables, $t$ over type variables, $\ell$ over labels and $\labelset$ over finite sets of labels. Types and expanded expressions are ABTs identified up to $\alpha$-equivalence in the usual manner (our typographic conventions are adapted from \emph{PFPL}, and summarized in Appendix \ref{appendix:typographic-conventions}.) To emphasize that programmers never draw expanded terms directly, and to clearly distinguish expanded terms from unexpanded terms, we do not define a stylized or textual syntax for expanded terms.

The {XL} forms a standard pure functional language with support for partial functions, quantification over types, recursive types, labeled product types and labeled sum types.  The reader is directed to \emph{PFPL} \cite{pfpl} (or another text on type systems, e.g. \emph{TAPL} \cite{tapl}) for a detailed introductory account of these standard constructs. We will tersely summarize the statics and dynamics of the XL in the next two subsections, respectively.


\subsection{Statics of the Expanded Language}
The \emph{statics of the XL} is defined by hypothetical judgements of the following form:

\[\begin{array}{ll}
\textbf{Judgement Form} & \textbf{Description}\\
\istypeU{\Delta}{\tau} & \text{$\tau$ is a type}\\
%\isctxU{\Delta}{\Gamma} & \text{$\Gamma$ is a well-formed typing context assuming $\Delta$}\\
\hastypeU{\Delta}{\Gamma}{e}{\tau} & \text{$e$ is assigned type $\tau$}
\end{array}\]
The \emph{type formation judgement}, $\istypeU{\Delta}{\tau}$, is inductively defined by Rules (\ref{rules:istypeU}). The \emph{typing judgement}, $\hastypeU{\Delta}{\Gamma}{e}{\tau}$, is inductively defined by Rules (\ref{rules:hastypeU}).

\emph{Type formation contexts}, $\Delta$, are finite sets of hypotheses of the form $\Dhyp{t}$. Empty finite sets are written $\emptyset$, or omitted entirely within judgements, and non-empty finite sets are written as comma-separated finite sequences identified up to exchange and contraction. We write $\Delta, \Dhyp{t}$ when $\Dhyp{t} \notin \Delta$ for $\Delta$ extended with the hypothesis $\Dhyp{t}$. %Finite sets are written as finite sequences identified up to exchange.% We write $\Dcons{\Delta}{\Delta'}$ for the union of $\Delta$ and $\Delta'$.

% \begin{subequations}\label{rules:istypeU}
% \begin{equation*}\label{rule:istypeU-var}
% \inferrule{ }{\istypeU{\Delta, \Dhyp{t}}{t}}
% \end{equation*}
% \begin{equation*}\label{rule:istypeU-parr}
% \inferrule{
%   \istypeU{\Delta}{\tau_1}\\
%   \istypeU{\Delta}{\tau_2}
% }{\istypeU{\Delta}{\aparr{\tau_1}{\tau_2}}}
% \end{equation*}
% \begin{equation*}\label{rule:istypeU-all}
%   \inferrule{
%     \istypeU{\Delta, \Dhyp{t}}{\tau}
%   }{
%     \istypeU{\Delta}{\aall{t}{\tau}}
%   }
% \end{equation*}
% \begin{equation*}\label{rule:istypeU-rec}
%   \inferrule{
%     \istypeU{\Delta, \Dhyp{t}}{\tau}
%   }{
%     \istypeU{\Delta}{\arec{t}{\tau}}
%   }
% \end{equation*}
% \begin{equation*}\label{rule:istypeU-prod}
%   \inferrule{
%     \{\istypeU{\Delta}{\tau_i}\}_{i \in \labelset}
%   }{
%     \istypeU{\Delta}{\aprod{\labelset}{\mapschema{\tau}{i}{\labelset}}}
%   }
% \end{equation*}
% \begin{equation*}\label{rule:istypeU-sum}
%   \inferrule{
%     \{\istypeU{\Delta}{\tau_i}\}_{i \in \labelset}
%   }{
%     \istypeU{\Delta}{\asum{\labelset}{\mapschema{\tau}{i}{\labelset}}}
%   }
% \end{equation*}
% \end{subequations}
% Premises of the form $\{{J}_i\}_{i \in \labelset}$ mean that for each $i \in \labelset$, the judgement ${J}_i$ must hold. 

\emph{Typing contexts}, $\Gamma$, are finite functions that map each variable $x \in \domof{\Gamma}$, where $\domof{\Gamma}$ is a finite set of variables, to the hypothesis $\Ghyp{x}{\tau}$, for some $\tau$. Empty typing contexts are written $\emptyset$, or omitted entirely within judgements, and non-empty typing contexts are written as finite sequences of hypotheses identified up to exchange and contraction. We write $\Gamma, \Ghyp{x}{\tau}$, when $x \notin \domof{\Gamma}$, for the extension of $\Gamma$ with a mapping from $x$ to $\Ghyp{x}{\tau}$, and $\Gcons{\Gamma}{\Gamma'}$ when $\domof{\Gamma} \cap \domof{\Gamma'} = \emptyset$ for the typing context mapping each $x \in \domof{\Gamma} \cup \domof{\Gamma'}$ to $x : \tau$ if $x : \tau \in \Gamma$ or $x : \tau \in \Gamma'$. % We write $\isctxU{\Delta}{\Gamma}$ if every type in $\Gamma$ is well-formed relative to $\Delta$.
% \begin{definition}[Typing Context Formation] \label{def:isctxU}
% $\isctxU{\Delta}{\Gamma}$ iff for each hypothesis $x : \tau \in \Gamma$, we have $\istypeU{\Delta}{\tau}$.
% \end{definition}

% \begin{subequations}\label{rules:hastypeU}
% \begin{equation*}\label{rule:hastypeU-var}
%   \inferrule{ }{
%     \hastypeU{\Delta}{\Gamma, \Ghyp{x}{\tau}}{x}{\tau}
%   }
% \end{equation*}
% \begin{equation*}\label{rule:hastypeU-lam}
%   \inferrule{
%     \istypeU{\Delta}{\tau}\\
%     \hastypeU{\Delta}{\Gamma, \Ghyp{x}{\tau}}{e}{\tau'}
%   }{
%     \hastypeU{\Delta}{\Gamma}{\aelam{\tau}{x}{e}}{\aparr{\tau}{\tau'}}
%   }
% \end{equation*}
% \begin{equation*}\label{rule:hastypeU-ap}
%   \inferrule{
%     \hastypeU{\Delta}{\Gamma}{e_1}{\aparr{\tau}{\tau'}}\\
%     \hastypeU{\Delta}{\Gamma}{e_2}{\tau}
%   }{
%     \hastypeU{\Delta}{\Gamma}{\aeap{e_1}{e_2}}{\tau'}
%   }
% \end{equation*}
% \begin{equation*}\label{rule:hastypeU-tlam}
%   \inferrule{
%     \hastypeU{\Delta, \Dhyp{t}}{\Gamma}{e}{\tau}
%   }{
%     \hastypeU{\Delta}{\Gamma}{\aetlam{t}{e}}{\aall{t}{\tau}}
%   }
% \end{equation*}
% \begin{equation*}\label{rule:hastypeU-tap}
%   \inferrule{
%     \hastypeU{\Delta}{\Gamma}{e}{\aall{t}{\tau}}\\
%     \istypeU{\Delta}{\tau'}
%   }{
%     \hastypeU{\Delta}{\Gamma}{\aetap{e}{\tau'}}{[\tau'/t]\tau}
%   }
% \end{equation*}
% \begin{equation*}\label{rule:hastypeU-fold}
%   \inferrule{\
%     \istypeU{\Delta, \Dhyp{t}}{\tau}\\
%     \hastypeU{\Delta}{\Gamma}{e}{[\arec{t}{\tau}/t]\tau}
%   }{
%     \hastypeU{\Delta}{\Gamma}{\aefold{e}}{\arec{t}{\tau}}
%   }
% \end{equation*}
% \begin{equation*}\label{rule:hastypeU-unfold}
%   \inferrule{
%     \hastypeU{\Delta}{\Gamma}{e}{\arec{t}{\tau}}
%   }{
%     \hastypeU{\Delta}{\Gamma}{\aeunfold{e}}{[\arec{t}{\tau}/t]\tau}
%   }
% \end{equation*}
% \begin{equation*}\label{rule:hastypeU-tpl}
%   \inferrule{
%     \{\hastypeU{\Delta}{\Gamma}{e_i}{\tau_i}\}_{i \in \labelset}
%   }{
%     \hastypeU{\Delta}{\Gamma}{\aetpl{\labelset}{\mapschema{e}{i}{\labelset}}}{\aprod{\labelset}{\mapschema{\tau}{i}{\labelset}}}
%   }
% \end{equation*}
% \begin{equation*}\label{rule:hastypeU-pr}
%   \inferrule{
%     \hastypeU{\Delta}{\Gamma}{e}{\aprod{\labelset, \ell}{\mapschema{\tau}{i}{\labelset}; \ell \hookrightarrow \tau}}
%   }{
%     \hastypeU{\Delta}{\Gamma}{\aepr{\ell}{e}}{\tau}
%   }
% \end{equation*}
% \begin{equation*}\label{rule:hastypeU-in}
%   \inferrule{
%     \{\istypeU{\Delta}{\tau_i}\}_{i \in \labelset}\\
%     \istypeU{\Delta}{\tau}\\
%     \hastypeU{\Delta}{\Gamma}{e}{\tau}
%   }{
%     \hastypeU{\Delta}{\Gamma}{\aein{\labelset, \ell}{\ell}{\mapschema{\tau}{i}{\labelset}; \ell \hookrightarrow \tau}{e}}{\asum{\labelset, \ell}{\mapschema{\tau}{i}{\labelset}; \ell \hookrightarrow \tau}}
%   }
% \end{equation*}
% \begin{equation*}\label{rule:hastypeU-case}
%   \inferrule{
%     \hastypeU{\Delta}{\Gamma}{e}{\asum{\labelset}{\mapschema{\tau}{i}{\labelset}}}\\
%     \istypeU{\Delta}{\tau}\\
%     \{\hastypeU{\Delta}{\Gamma, x_i : \tau_i}{e_i}{\tau}\}_{i \in \labelset}
%   }{
%     \hastypeU{\Delta}{\Gamma}{\aecase{\labelset}{e}{\mapschemab{x}{e}{i}{\labelset}}}{\tau}
%   }
% \end{equation*}
% \end{subequations}

%Rules (\ref{rules:istypeU}) and (\ref{rules:hastypeU}) are syntax-directed, so we assume an inversion lemma for each rule as needed without stating it separately. 
% The following standard lemmas also hold. 

These judgements validate standard lemmas, defined in Appendix \ref{appendix:SES-XL}: Weakening, Substitution and Decomposition.

\subsection{Structural Dynamics}\label{sec:dynamics-U}
The \emph{structural dynamics} (a.k.a. the \emph{structural operational semantics} \cite{DBLP:journals/jlp/Plotkin04a}) of $\miniVerseUE$ is defined as a transition system by judgements of the following form:
\[\begin{array}{ll}
\textbf{Judgement Form} & \textbf{Description}\\
\stepsU{e}{e'} & \text{$e$ transitions to $e'$}\\
\isvalU{e} & \text{$e$ is a value}
\end{array}\]
We also define auxiliary judgements for \emph{iterated transition}, $\multistepU{e}{e'}$, and \emph{evaluation}, $\evalU{e}{e'}$.

\begingroup
\def\thetheorem{\ref{defn:iterated-transition-UP}}
\begin{definition}[Iterated Transition] Iterated transition, $\multistepU{e}{e'}$, is the reflexive, transitive closure of the transition judgement, $\stepsU{e}{e'}$.\end{definition}
% \addtocounter{theorem}{-1}
\endgroup

\begingroup
\def\thetheorem{\ref{defn:evaluation-UP}}
\begin{definition}[Evaluation] $\evalU{e}{e'}$ iff $\multistepU{e}{e'}$ and $\isvalU{e'}$. \end{definition}
% \addtocounter{theorem}{-1}
\endgroup

Our subsequent developments do not require making reference to particular rules in the structural dynamics (because TLMs operate statically), so we do not reproduce the rules here. Instead, it suffices to state the following conditions.

The Canonical Forms condition characterizes well-typed values. Satisfying this condition requires an \emph{eager} (a.k.a. \emph{by-value}) formulation of the dynamics. 
\begingroup
\def\thetheorem{\ref{condition:canonical-forms-UP}}
\begin{condition}[Canonical Forms] If $\hastypeUC{e}{\tau}$ and $\isvalU{e}$ then:
\begin{enumerate}
\item If $\tau=\aparr{\tau_1}{\tau_2}$ then $e=\aelam{\tau_1}{x}{e'}$ and $\hastypeUCO{\Ghyp{x}{\tau_1}}{e'}{\tau_2}$.
\item If $\tau=\aall{t}{\tau'}$ then $e=\aetlam{t}{e'}$ and $\hastypeUCO{\Dhyp{t}}{e'}{\tau'}$.
\item If $\tau=\arec{t}{\tau'}$ then $e=\aefold{e'}$ and $\hastypeUC{e'}{[\abop{rec}{t.\tau'}/t]\tau'}$ and $\isvalU{e'}$. 
\item If $\tau=\aprod{\labelset}{\mapschema{\tau}{i}{\labelset}}$ then $e=\aetpl{\labelset}{\mapschema{e}{i}{\labelset}}$ and $\hastypeUC{e_i}{\tau_i}$ and $\isvalU{e_i}$ for each $i \in \labelset$.
\item If $\tau=\asum{\labelset}{\mapschema{\tau}{i}{\labelset}}$ then for some label set $L'$ and label $\ell$ and type $\tau'$, we have that $\labelset=\labelset', \ell$ and $\tau=\asum{\labelset', \ell}{\mapschema{\tau}{i}{\labelset'}; \mapitem{\ell}{\tau'}}$ and $e=\aein{\ell}{e'}$ and $\hastypeUC{e'}{\tau'}$ and $\isvalU{e'}$.
\end{enumerate}\end{condition}
\endgroup

The Preservation condition ensures that evaluation preserve typing.  

\begingroup
\def\thetheorem{\ref{condition:preservation-UP}}
\begin{condition}[Preservation] If $\hastypeUC{e}{\tau}$ and $\multistepU{e}{e'}$ then $\hastypeUC{e'}{\tau}$. \end{condition}
\endgroup
The Progress condition ensures that evaluating a well-typed expanded expression cannot ``get stuck'':
\begingroup
\def\thetheorem{\ref{condition:progress-UP}}
\begin{condition}[Progress] If $\hastypeUC{e}{\tau}$ then either $\isvalU{e}$ or there exists an $e'$ such that $\stepsU{e}{e'}$. \end{condition}
\endgroup
 Together, these two conditions constitute the Type Safety Condition.

\vspace{-8px}
\subsection{Syntax of the Unexpanded Language}\label{sec:syntax-U}
\begin{figure}[t!]
\[\begin{array}{lllllll}
\textbf{Sort} & &  
%&\textbf{Operational Form} 
& \textbf{Stylized Form} & \textbf{Description}\\
\mathsf{UTyp} & \utau & ::= 
% &\ut 
& \ut & \text{identifier}\\
&& 
%& \auparr{\utau}{\utau} 
& \parr{\utau}{\utau} & \text{partial function}\\
&&
%& \auall{\ut}{\utau} 
& \forallt{\ut}{\utau} & \text{polymorphic}\\
&&
%& \aurec{\ut}{\utau} 
& \rect{\ut}{\utau} & \text{recursive}\\
&&
%& \auprod{\labelset}{\mapschema{\utau}{i}{\labelset}} 
& \prodt{\mapschema{\utau}{i}{\labelset}} & \text{labeled product}\\
&&
%& \ausum{\labelset}{\mapschema{\utau}{i}{\labelset}} 
& \sumt{\mapschema{\utau}{i}{\labelset}} & \text{labeled sum}\\
\mathsf{UExp} & \ue & ::= 
%& \ux 
& \ux & \text{identifier}\\
&&
%
& \asc{\ue}{\utau} & \text{ascription}\\
&&
%
& \letsyn{\ux}{\ue}{\ue} & \text{value binding}\\
&&
%& \aulam{\utau}{\ux}{\ue} 
& \lam{\ux}{\utau}{\ue} & \text{abstraction}\\
&&
%& \auap{\ue}{\ue} 
& \ap{\ue}{\ue} & \text{application}\\
&&
%& \autlam{\ut}{\ue} 
& \Lam{\ut}{\ue} & \text{type abstraction}\\
&&
%& \autap{\ue}{\utau} 
& \App{\ue}{\utau} & \text{type application}\\
&&
%& \aufold{\ut}{\utau}{\ue} 
& \fold{\ue} & \text{fold}\\
&&
%& \auunfold{\ue} 
& \unfold{\ue} & \text{unfold}\\
&&
%& \autpl{\labelset}{\mapschema{\ue}{i}{\labelset}} 
& \tpl{\mapschema{\ue}{i}{\labelset}} & \text{labeled tuple}\\
&&
%& \aupr{\ell}{\ue} 
& \prj{\ue}{\ell} & \text{projection}\\
&&
%& \auin{\labelset}{\ell}{\mapschema{\utau}{i}{\labelset}}{\ue} 
& \inj{\ell}{\ue} & \text{injection}\\
&&
%& \aucase{\labelset}{\utau}{\ue}{\mapschemab{\ux}{\ue}{i}{\labelset}} 
& \caseof{\ue}{\mapschemab{\ux}{\ue}{i}{\labelset}} & \text{case analysis}\\
\LCC  &  & 
%& \lightgray 
& \color{Yellow} & \color{Yellow} \\
&&
%& \audefuetsm{\utau}{e}{\tsmv}{\ue} 
& \uesyntax{\tsmv}{\utau}{e}{\ue} & \text{seTLM definition}\\ 
&&
%& \autsmap{b}{\tsmv} 
& \utsmap{\tsmv}{b} & \text{seTLM application}\ECC
\end{array}\]\vspace{-10px}
\caption[Syntax of the $\miniVerseUE$ unexpanded language (UL)]{Syntax of the $\miniVerseUE$ unexpanded language (UL).% Metavariable $\ut$ ranges over type identifiers, $\ux$ ranges over expression identifiers, $\tsmv$  over TLM names and $b$ over character sequences, which, when they appear in an unexpanded expression, are called literal bodies.
}
\label{fig:U-unexpanded-terms}
\end{figure}

A $\miniVerseUE$ program ultimately evaluates as a well-typed expanded expression. However, the programmer does not construct this expanded expression directly. Instead, the programmer constructs an \emph{unexpanded expression}, $\ue$, which might contain \emph{unexpanded types}, $\utau$. Figure \ref{fig:U-unexpanded-terms} defines the relevant forms.

Unexpanded types and expressions are \textbf{not} abstract binding trees -- we do \textbf{not} define notions of renaming, alpha-equivalence or substitution for unexpanded terms. This is because unexpanded expressions remain ``partially parsed'' due to the presence of literal bodies, $b$, from which spliced terms might be extracted during typed expansion. In fact, unexpanded types and expressions do not involve variables at all, but rather \emph{type identifiers}, $\ut$, and \emph{expression identifiers}, $\ux$. Identifiers are given meaning by expansion to variables during typed expansion, as we will see below. This distinction between identifiers and variables will be technically crucial. %We \textbf{cannot} adopt the usual definitions of $\alpha$-renaming of identifiers, because unexpanded types and expressions are still in a ``partially parsed'' state -- the literal bodies, $b$, within an unexpanded expression might contain spliced subterms that are ``surfaced'' by a TLM only during typed expansion, as we will detail below. %identifiers are given meaning by expansion to variables. %In other words, unexpanded expressions are not abstract binding trees, nor sequences of characters, but a ``transitional'' structure with some characteristics of each of these. 
%For this reason, we will need to handle generating fresh variables explicitly at binding sites in our semantics. %To do so, we distinguish \emph{type identifiers}, $\ut$, and \emph{expression identifiers}, $\ux$, from type variables, $t$, and expression variables, $x$. identifiers will be given meaning by expansion to variables (which, in turn, are given meaning by substitution, as described above). 


Most of the unexpanded forms in Figure \ref{fig:U-unexpanded-terms}  mirror the expanded forms. We refer to these as the \emph{common forms}. The mapping from expanded forms to common unexpanded forms is defined explicitly in Appendix \ref{appendix:SES-shared-forms}.

% There are only two unexpanded expression forms, highlighted in gray in Figure \ref{fig:U-unexpanded-terms}, that do not correspond to expanded expression forms -- the seTLM definition form and the seTLM application form. %These are the ``interesting'' forms. % These are the ``interesting'' forms. % Let us define this correspondence by the metafunction $\Uof{e}$:
%\[
%\begin{split}
%\Uof{x} & = x\\
%\Uof{\aelam{\tau}{x}{e}} & = \aulam{\tau}{x}{\Uof{e}}\\
%\Uof{\aeap{e_1}{e_2}} & = \auap{\Uof{e_1}}{\Uof{e_2}}
%\end{split}
%\] and so on for the remaining expanded expression forms.


In addition to the stylized syntax given in Figure \ref{fig:U-unexpanded-terms}, there is also a context-free textual syntax for the UL. 
Giving a complete definition of the context-free textual syntax as, e.g., a context-free grammar, risks digression into details that are not critical to our purposes here. The paper on Wyvern defines a textual syntax for a similar system \cite{TSLs}. Instead, we need only posit the existence of partial metafunctions $\parseUTypF{b}$ and $\parseUExpF{b}$  that go from character sequences, $b$, to unexpanded types and expressions, respectively. 
\begingroup
\def\thetheorem{\ref{condition:textual-representability-SES}}
\begin{condition}[Textual Representability] ~
\begin{enumerate}
\item For each $\utau$, there exists $b$ such that $\parseUTyp{b}{\utau}$. 
\item For each $\ue$, there exists $b$ such that $\parseUExp{b}{\ue}$.
\end{enumerate}
\end{condition}
\endgroup

\subsection{Typed Expansion}\label{sec:typed-expansion-U}
Unexpanded expressions, and the unexpanded types therein, are checked and expanded simultaneously according to the \emph{typed expansion judgements}:
\[\begin{array}{ll}
\textbf{Judgement Form} & \textbf{Description}\\
\expandsTU{\uDelta}{\utau}{\tau} & \text{$\utau$ has well-formed expansion $\tau$}\\
\expandsUX{\ue}{e}{\tau} & \text{$\ue$ has expansion $e$ of type $\tau$}\end{array}\]
%\newcommand{\gray}[1]{{\color{gray} #1}}

\subsubsection{Type Expansion}
\emph{Unexpanded type formation contexts}, $\uDelta$, are of the form $\uDD{\uD}{\Delta}$, i.e. they consist of a \emph{type identifier expansion context}, $\uD$, paired with a standard type formation context, $\Delta$. 

A \emph{type identifier expansion context}, $\uD$, is a finite function that maps each type identifier $\ut \in \domof{\uD}$ to the hypothesis $\vExpands{\ut}{t}$, for some type variable $t$. We write $\ctxUpdate{\uD}{\ut}{t}$ for the type identifier expansion context that maps $\ut$ to $\vExpands{\ut}{t}$ and defers to $\uD$ for all other type identifiers (i.e. the previous mapping is \emph{updated}.) 

We define $\uDelta, \uDhyp{\ut}{t}$ when $\uDelta=\uDD{\uD}{\Delta}$ as an abbreviation of  \[\uDD{\ctxUpdate{\uD}{\ut}{t}}{\Delta, \Dhyp{t}}\]%type identifier expansion context is always extended/updated together with 

The \emph{type expansion judgement}, $\expandsTU{\uDelta}{\utau}{\tau}$, is inductively defined by Rules (\ref{rules:expandsTU}). The first three of these rules are reproduced below:
% \begin{subequations}%\label{rules:expandsTU}
\begin{equation*}\tag{\ref{rule:expandsTU-var}}
\inferrule{ }{\expandsTU{\uDelta, \uDhyp{\ut}{t}}{\ut}{t}}
\end{equation*}
\begin{equation*}\tag{\ref{rule:expandsTU-parr}}
\inferrule{
  \expandsTU{\uDelta}{\utau_1}{\tau_1}\\
  \expandsTU{\uDelta}{\utau_2}{\tau_2}
}{\expandsTU{\uDelta}{\parr{\utau_1}{\utau_2}}{\aparr{\tau_1}{\tau_2}}}
\end{equation*}
\begin{equation*}\tag{\ref{rule:expandsTU-all}}
  \inferrule{
    \expandsTU{\uDelta, \uDhyp{\ut}{t}}{\utau}{\tau}
  }{
    \expandsTU{\uDelta}{\forallt{\ut}{\utau}}{\aall{t}{\tau}}
  }
\end{equation*}
% \begin{equation*}\label{rule:expandsTU-rec}
%   \inferrule{
%     \expandsTU{\uDelta, \uDhyp{\ut}{t}}{\utau}{\tau}
%   }{
%     \expandsTU{\uDelta}{\aurec{\ut}{\utau}}{\arec{t}{\tau}}
%   }
% \end{equation*}
% \begin{equation*}\label{rule:expandsTU-prod}
%   \inferrule{
%     \{\expandsTU{\uDelta}{\utau_i}{\tau_i}\}_{i \in \labelset}
%   }{
%     \expandsTU{\uDelta}{\auprod{\labelset}{\mapschema{\utau}{i}{\labelset}}}{\aprod{\labelset}{\mapschema{\tau}{i}{\labelset}}}
%   }
% \end{equation*}
% \begin{equation*}\label{rule:expandsTU-sum}
%   \inferrule{
%     \{\expandsTU{\uDelta}{\utau_i}{\tau_i}\}_{i \in \labelset}
%   }{
%     \expandsTU{\uDelta}{\ausum{\labelset}{\mapschema{\utau}{i}{\labelset}}}{\asum{\labelset}{\mapschema{\tau}{i}{\labelset}}}
%   }
% \end{equation*}
% \end{subequations}
%We write $\uDeltaOK{\uDelta}$ when $\uDelta=\uDD{\uD}{\Delta}$ and each type variable in $\uD$ also appears in $\Delta$.
%\begin{definition}\label{def:uDeltaOK} $\uDeltaOK{\uDD{\uD}{\Delta}}$ iff for each $\vExpands{\ut}{t} \in \uD$, we have $\Dhyp{t} \in \Delta$.\end{definition}

To develop an intuition for how type identifier expansion operates, it is instructive to inspect the derivation of the expansion of the unexpanded type $\forallt{\ut}{\forallt{\ut}{\ut}}$:
\begin{mathpar}
\inferrule{
  \inferrule{
    \inferrule{ }{
      \expandsTU{\uDD{\vExpands{\ut}{t_2}}{{\Dhyp{t_1}}, {\Dhyp{t_2}}}}{\ut}{t_2}
    }~\text{(\ref*{rule:expandsTU-var})}
  }{
    \expandsTU{\uDD{\vExpands{\ut}{t_1}}{\Dhyp{t_1}}}{\forallt{\ut}{\ut}}{\aall{t_2}{t_2}}
  }~\text{(\ref*{rule:expandsTU-all})}
}{
  \expandsTU{\uDD{\emptyset}{\emptyset}}{\forallt{\ut}{\forallt{\ut}{\ut}}}{\aall{t_1}{\aall{t_2}{t_2}}}
}~\text{(\ref*{rule:expandsTU-all})}
\end{mathpar}
Notice that when Rule (\ref{rule:expandsTU-all}) is applied, the type identifier expansion context is extended (when the outermost binding is encountered) or updated (at all inner bindings) and the type formation context is simultaneously extended with a (necessarily fresh) hypothesis. Without this mechanism, expansions for unexpanded types with shadowing, like this minimal example, would not exist, because by definition we cannot extend a type formation context with a variable it already mentions, nor implicitly $\alpha$-vary the unexpanded type to sidestep this problem in the usual manner.

The Type Expansion Lemma establishes that the expansion of an unexpanded type is a well-formed type.

\begingroup
\def\thetheorem{\ref{lemma:type-expansion-U}}
\begin{lemma}[Type Expansion] If $\expandsTU{\uDD{\uD}{\Delta}}{\utau}{\tau}$ then $\istypeU{\Delta}{\tau}$.\end{lemma}
\begin{proof} By rule induction over Rules (\ref{rules:expandsTU}). In each case, we apply the IH to or over each premise, then apply the corresponding type formation rule in Rules (\ref{rules:istypeU}). \end{proof}
\endgroup

\subsubsection{Typed Expression Expansion}
% \begin{subequations}\label{rules:expandsU}
\emph{Unexpanded typing contexts}, $\uGamma$, are, similarly, of the form $\uGG{\uG}{\Gamma}$, where $\uG$ is an \emph{expression identifier expansion context}, and $\Gamma$ is a typing context. An expression identifier expansion context, $\uG$, is a finite function that maps each expression identifier $\ux \in \domof{\uG}$ to the hypothesis $\vExpands{\ux}{x}$, for some expression variable, $x$. We write $\ctxUpdate{\uG}{\ux}{x}$ for the expression identifier expansion context that maps $\ux$ to $\vExpands{\ux}{x}$ and defers to $\uG$ for all other expression identifiers (i.e. the previous mapping is updated.) %We write $\uGammaOK{\uGamma}$ when $\uGamma=\uGG{\uG}{\Gamma}$ and each expression variable in $\uG$ is assigned a type by $\Gamma$.
%\begin{definition} $\uGammaOK{\uGG{\uG}{\Gamma}}$ iff for each $\vExpands{\ux}{x} \in \uG$, we have $\Ghyp{x}{\tau} \in \Gamma$ for some $\tau$.\end{definition}
%\noindent 
We define $\uGamma, \uGhyp{\ux}{x}{\tau}$ when $\uGamma = \uGG{\uG}{\Gamma}$ as an abbreviation of \[\uGG{\ctxUpdate{\uG}{\ux}{x}}{\Gamma, \Ghyp{x}{\tau}}\]

The \emph{typed expression expansion judgement}, $\expandsUX{\ue}{e}{\tau}$, is inductively defined by Rules (\ref{rules:expandsU}). Before covering these rules, let us state the main theorem of interest: that typed expansion results in a well-typed expanded expression.
\begingroup
\def\thetheorem{\ref{thm:typed-expansion-short-U}}
\begin{theorem}[Typed Expression Expansion] \hspace{-3px}If $\expandsU{\uDD{\uD}{\Delta}\hspace{-3px}}{\uGG{\uG}{\Gamma}\hspace{-3px}}{\uPsi}{\ue}{e}{\tau}$ then $\hastypeU{\Delta}{\Gamma}{e}{\tau}$.
\end{theorem}
\endgroup
%These rules validate the following theorem, which establishes that typed expansion produces an expansion of the assigned type. 
%\begin{theorem}[Typed Expression Expansion] If $\expandsU{\uDD{\uD}{\Delta}}{\uGG{\uG}{\Gamma}}{\uPsi}{\ue}{e}{\tau}$ and $\uetsmenv{\Delta}{\uPsi}$ then $\hastypeU{\Delta}{\Gamma}{e}{\tau}$.\end{theorem}
%\begin{proof} This is the first part of Theorem \ref{thm:typed-expansion-U}, defined and proven below.\end{proof}

\paragraph{Common Forms} Rules (\ref{rule:expandsU-var}) through (\ref{rule:expandsU-case}) handle unexpanded expressions of common form, as well as ascriptions and let binding. The first five of these rules are reproduced below:
%Each of these rules is based on the corresponding typing rule, i.e. Rules (\ref{rule:hastypeU-var}) through (\ref{rule:hastypeU-case}), respectively. For example, the following typed expansion rules are based on the typing rules (\ref{rule:hastypeU-var}), (\ref{rule:hastypeU-lam}) and (\ref{rule:hastypeU-ap}), respectively:% for unexpanded expressions of variable, function and application form, respectively: 
\begin{equation*}\tag{\ref{rule:expandsU-var}}
  \inferrule{ }{\expandsU{\uDelta}{\uGamma, \uGhyp{\ux}{x}{\tau}}{\uPsi}{\ux}{x}{\tau}}
\end{equation*}
\begin{equation*}\tag{\ref{rule:expandsU-asc}}
  \inferrule{
    \expandsTU{\uDelta}{\utau}{\tau}\\
    \expandsU{\uDelta}{\uGamma}{\uPsi}{\ue}{e}{\tau}
  }{
    \expandsU{\uDelta}{\uGamma}{\uPsi}{\asc{\ue}{\utau}}{e}{\tau}
  }
\end{equation*}
\begin{equation*}\tag{\ref{rule:expandsU-letsyn}}
  \inferrule{
    \expandsU{\uDelta}{\uGamma}{\uPsi}{\ue_1}{e_1}{\tau_1}\\
    \expandsU{\uDelta}{\uGamma, \uGhyp{\ux}{x}{\tau_1}}{\uPsi}{\ue_2}{e_2}{\tau_2}
  }{
    \expandsU{\uDelta}{\uGamma}{\uPsi}{\letsyn{\ux}{\ue_1}{\ue_2}}{
      \aeap{\aelam{\tau_1}{x}{e_2}}{e_1}
    }{\tau_2}
  }
\end{equation*}
\begin{equation*}\tag{\ref{rule:expandsU-lam}}
  \inferrule{
    \expandsTU{\uDelta}{\utau}{\tau}\\
    \expandsU{\uDelta}{\uGamma, \uGhyp{\ux}{x}{\tau}}{\uPsi}{\ue}{e}{\tau'}
  }{\expandsUX{\lam{\ux}{\utau}{\ue}}{\aelam{\tau}{x}{e}}{\aparr{\tau}{\tau'}}}
\end{equation*}
\begin{equation*}\tag{\ref{rule:expandsU-ap}}
  \inferrule{
    \expandsUX{\ue_1}{e_1}{\aparr{\tau}{\tau'}}\\
    \expandsUX{\ue_2}{e_2}{\tau}
  }{
    \expandsUX{\ap{\ue_1}{\ue_2}}{\aeap{e_1}{e_2}}{\tau'}
  }
\end{equation*}

The rules for the remaining expressions of common form are entirely straightforward, mirroring the corresponding typing rules, i.e. Rules (\ref{rules:hastypeU}). %In particular, observe that, in each of these rules, the unexpanded and expanded expression forms in the conclusion correspond, and each premise corresponds to a premise of the corresponding typing rule. %Type formation premises in the typing rule give rise to  type expansion premises in the corresponding typed expansion rule, and each typed expression expansion premise in each rule above corresponds to a typing premise in the corresponding typing rule. 
The type assigned in the conclusion of each rule is identical to the type assigned in the conclusion of the corresponding typing rule. The seTLM context, $\uPsi$, passes opaquely through these rules (we will define seTLM contexts below.) As such, the corresponding cases in the proof of Theorem \ref{thm:typed-expansion-short-U} are by application of the induction hypothesis and the  corresponding typing rule. %Rules (\ref{rules:expandsTU}) could similarly have been generated by mechanically transforming Rules (\ref{rules:istypeU}).

% We can express this scheme more precisely with the rule transformation given in Appendix \ref{appendix:SES-uexps}. For each rule in Rules (\ref{rules:istypeU}) and Rules (\ref{rules:hastypeU}),
% \begin{mathpar}
% \refstepcounter{equation}
% % \label{rule:expandsU-tlam}
% % \refstepcounter{equation}
% % \label{rule:expandsU-tap}
% % \refstepcounter{equation}
% \label{rule:expandsU-fold}
% \refstepcounter{equation}
% \label{rule:expandsU-unfold}
% \refstepcounter{equation}
% \label{rule:expandsU-tpl}
% \refstepcounter{equation}
% \label{rule:expandsU-pr}
% \refstepcounter{equation}
% \label{rule:expandsU-in}
% \refstepcounter{equation}
% \label{rule:expandsU-case}
% \inferrule{J_1\\ \cdots \\ J_k}{J}
% \end{mathpar}
% the corresponding typed expansion rule is 
% \begin{mathpar}
% \inferrule{
%   \Uof{J_1} \\
%   \cdots\\
%   \Uof{J_k}
% }{
%   \Uof{J}
% }
% \end{mathpar}
% where
% \[\begin{split}
% \Uof{\istypeU{\Delta}{\tau}} & = \expandsTU{\Uof{\Delta}}{\Uof{\tau}}{\tau} \\
% \Uof{\hastypeU{\Gamma}{\Delta}{e}{\tau}} & = \expandsU{\Uof{\Gamma}}{\Uof{\Delta}}{\uPsi}{\Uof{e}}{e}{\tau}\\
% \Uof{\{J_i\}_{i \in \labelset}} & = \{\Uof{J_i}\}_{i \in \labelset}
% \end{split}\]
% and where:
% \begin{itemize}
% \item $\Uof{\tau}$ is defined as follows:
%   \begin{itemize}
%   \item When $\tau$ is of definite form, $\Uof{\tau}$ is defined as in Sec. \ref{sec:syntax-U}.
%   \item When $\tau$ is of indefinite form, $\Uof{\tau}$ is a uniquely corresponding metavariable of sort $\mathsf{UTyp}$ also of indefinite form. For example, in Rule (\ref{rule:istypeU-parr}), $\tau_1$ and $\tau_2$ are of indefinite form, i.e. they match arbitrary types. The rule transformation simply ``hats'' them, i.e. $\Uof{\tau_1}=\utau_1$ and $\Uof{\tau_2}=\utau_2$.
%   \end{itemize}
% \item $\Uof{e}$ is defined as follows
% \begin{itemize}
% \item When $e$ is of definite form, $\Uof{e}$ is defined as in Sec. \ref{sec:syntax-U}. 
% \item When $e$ is of indefinite form, $\Uof{e}$ is a uniquely corresponding metavariable of sort $\mathsf{UExp}$ also of indefinite form. For example, $\Uof{e_1}=\ue_1$ and $\Uof{e_2}=\ue_2$.
% \end{itemize}
% \item $\Uof{\Delta}$ is defined as follows:
%   \begin{itemize} 
%   \item When $\Delta$ is of definite form, $\Uof{\Delta}$ is defined as above.
%   \item When $\Delta$ is of indefinite form, $\Uof{\Delta}$ is a uniquely corresponding metavariable ranging over unexpanded type formation contexts. For example, $\Uof{\Delta} = \uDelta$.
%   \end{itemize}
% \item $\Uof{\Gamma}$ is defined as follows:
%   \begin{itemize}
%   \item When $\Gamma$ is of definite form, $\Uof{\Gamma}$ produces the corresponding unexpanded typing context as follows:
% \begin{align*}
% \Uof{\emptyset} & = \uGG{\emptyset}{\emptyset}\\
% \Uof{\Gamma, \Ghyp{x}{\tau}} & = \Uof{\Gamma}, \uGhyp{\identifierof{x}}{x}{\tau}
% \end{align*}
%   \item When $\Gamma$ is of indefinite form, $\Uof{\Gamma}$ is a uniquely corresponding metavariable ranging over unexpanded typing contexts. For example, $\Uof{\Gamma} = \uGamma$.
% \end{itemize}
% \end{itemize}

% It is instructive to use this rule transformation to generate Rules (\ref{rules:expandsTU}) and Rules (\ref{rule:expandsU-var}) through (\ref{rule:expandsU-tap}) above. We omit the remaining rules, i.e. Rules (\ref*{rule:expandsU-fold}) through (\ref*{rule:expandsU-case}). By instead defining these rules solely by the rule transformation just described, we avoid having to write down a number of rules that are of limited marginal interest. Moreover, this demonstrates the general technique for generating typed expansion rules for unexpanded types and expressions of common form, so our exposition is somewhat ``robust'' to changes to the inner core. 
%o that when the inner core changes,  typed expansion rules  our exposition somewhat robust to changes to the inner core (though not to changes to the judgement forms in the statics of the inner core).% Even if changes to the judgement forms in the statics of the inner core are needed (e.g. the addition of a symbol context), it is easy to see would correspond to changes in the generic specification above.
% \begin{subequations}\label{rules:expandsU}
% \begin{equation*}\label{rule:expandsU-var}
%   \inferrule{ }{\expandsU{\Delta}{\Gamma, x : \tau}{\uPsi}{x}{x}{\tau}}
% \end{equation*}
% \begin{equation*}\label{rule:expandsU-lam}
%   \inferrule{
%     \istypeU{\Delta}{\tau}\\
%     \expandsU{\Delta}{\Gamma, x : \tau}{\uPsi}{\ue}{e}{\tau'}
%   }{\expandsUX{\aulam{\tau}{x}{\ue}}{\aelam{\tau}{x}{e}}{\aparr{\tau}{\tau'}}}
% \end{equation*}
% \begin{equation*}\label{rule:expandsU-ap}
%   \inferrule{
%     \expandsUX{\ue_1}{e_1}{\aparr{\tau}{\tau'}}\\
%     \expandsUX{\ue_2}{e_2}{\tau}
%   }{
%     \expandsUX{\auap{\ue_1}{\ue_2}}{\aeap{e_1}{e_2}}{\tau'}
%   }
% \end{equation*}
% \begin{equation*}\label{rule:expandsU-tlam}
%   \inferrule{
%     \expandsU{\Delta, \Dhyp{t}}{\Gamma}{\uPsi}{\ue}{e}{\tau}
%   }{
%     \expandsUX{\autlam{t}{\ue}}{\aetlam{t}{e}}{\aall{t}{\tau}}
%   }
% \end{equation*}
% \begin{equation*}\label{rule:expandsU-tap}
%   \inferrule{
%     \expandsUX{\ue}{e}{\aall{t}{\tau}}\\
%     \istypeU{\Delta}{\tau'}
%   }{
%     \expandsUX{\autap{\ue}{\tau'}}{\aetap{e}{\tau'}}{[\tau'/t]\tau}
%   }
% \end{equation*}
% \begin{equation*}\label{rule:expandsU-fold}
%   \inferrule{
%     \istypeU{\Delta, \Dhyp{t}}{\tau}\\
%     \expandsUX{\ue}{e}{[\arec{t}{\tau}/t]\tau}
%   }{
%     \expandsUX{\aufold{t}{\tau}{\ue}}{\aefold{e}}{\arec{t}{\tau}}
%   }
% \end{equation*}
% \begin{equation*}\label{rule:expandsU-unfold}
%   \inferrule{
%     \expandsUX{\ue}{e}{\arec{t}{\tau}}
%   }{
%     \expandsUX{\auunfold{\ue}}{\aeunfold{e}}{[\arec{t}{\tau}/t]\tau}
%   }
% \end{equation*}
% \begin{equation*}\label{rule:expandsU-tpl}
%   \inferrule{
%     \{\expandsUX{\ue_i}{e_i}{\tau_i}\}_{i \in \labelset}
%   }{
%     \expandsUX{\autpl{\labelset}{\mapschema{\ue}{i}{\labelset}}}{\aetpl{\labelset}{\mapschema{e}{i}{\labelset}}}{\aprod{\labelset}{\mapschema{\tau}{i}{\labelset}}}
%   }
% \end{equation*}
% \begin{equation*}\label{rule:expandsU-pr}
%   \inferrule{
%     \expandsUX{\ue}{e}{\aprod{\labelset, \ell}{\mapschema{\tau}{i}{\labelset}; \mapitem{\ell}{\tau}}}
%   }{
%     \expandsUX{\aupr{\ell}{\ue}}{\aepr{\ell}{e}}{\tau}
%   }
% \end{equation*}
% \begin{equation*}\label{rule:expandsU-in}
%   \inferrule{
%     \{\istypeU{\Delta}{\tau_i}\}_{i \in \labelset}\\
%     \istypeU{\Delta}{\tau}\\
%     \expandsUX{\ue}{e}{\tau}
%   }{
%     \left\{\shortstack{$\Delta~\Gamma \vdash_\uPsi \auin{\labelset, \ell}{\ell}{\mapschema{\tau}{i}{\labelset}; \mapitem{\ell}{\tau}}{\ue}$\\$\leadsto$\\$\aein{\labelset, \ell}{\ell}{\mapschema{\tau}{i}{\labelset}; \mapitem{\ell}{\tau}}{e} : \asum{\labelset, \ell}{\mapschema{\tau}{i}{\labelset}; \mapitem{\ell}{\tau}}$\vspace{-1.2em}}\right\}
%   }
% \end{equation*}
% \begin{equation*}\label{rule:expandsU-case}
%   \inferrule{
%     \expandsUX{\ue}{e}{\asum{\labelset}{\mapschema{\tau}{i}{\labelset}}}\\
%     \{\expandsU{\Delta}{\Gamma, \Ghyp{x_i}{\tau_i}}{\uPsi}{\ue_i}{e_i}{\tau}\}_{i \in \labelset}
%   }{
%     \expandsUX{\aucase{\labelset}{\ue}{\mapschemab{x}{\ue}{i}{\labelset}}}{\aecase{\labelset}{e}{\mapschemab{x}{e}{i}{\labelset}}}{\tau}
%   }
% \end{equation*}
% \end{subequations}
\paragraph{seTLM Definition and Application} The two remaining typed expansion rules, Rules (\ref{rule:expandsU-syntax}) and (\ref{rule:expandsU-tsmap}), govern the seTLM definition and application forms. They are defined in the next two subsections, respectively. 

% \begin{equation*}\label{rule:expandsU-syntax}
% \inferrule{
%   \istypeU{\Delta}{\tau}\\
%   \expandsU{\emptyset}{\emptyset}{\emptyset}{\ueparse}{\eparse}{\aparr{\tBody}{\tParseResultExp}}\\\\
%   a \notin \domof{\uPsi}\\
%   \expandsU{\Delta}{\Gamma}{\uPsi, \xuetsmbnd{\tsmv}{\tau}{\eparse}}{\ue}{e}{\tau'}
% }{
%   \expandsUX{\audefuetsm{\tau}{\ueparse}{\tsmv}{\ue}}{e}{\tau'}
% }
% \end{equation*}
% \begin{equation*}\label{rule:expandsU-tsmap}
% \inferrule{
%   \encodeBody{b}{\ebody}\\
%   \evalU{\ap{\eparse}{\ebody}}{\inj{\lbltxt{SuccessE}}{\ecand}}\\
%   \decodeCondE{\ecand}{\ce}\\\\
%   \cvalidE{\emptyset}{\emptyset}{\esceneU{\Delta}{\Gamma}{\uPsi, \xuetsmbnd{\tsmv}{\tau}{\eparse}}{b}}{\ce}{e}{\tau}
% }{
%   \expandsU{\Delta}{\Gamma}{\uPsi, \xuetsmbnd{\tsmv}{\tau}{\eparse}}{\autsmap{b}{\tsmv}}{e}{\tau}
% }
% \end{equation*}
%\end{subequations}

%Notice that each form of expanded expression (Figure \ref{fig:U-expanded-terms}) corresponds to a form of unexpanded expression (Figure \ref{fig:U-unexpanded-terms}). For each typing rule in Rules (\ref{rules:hastypeU}), there is a corresponding typed expansion rule -- Rules (\ref{rule:expandsU-var}) through (\ref{rule:expandsU-case}) -- where the unexpanded and expanded forms correspond. The premises also correspond -- if a typing judgement appears as a premise of a typing rule, then the corresponding premise in the corresponding typed expansion rule is the corresponding typed expansion judgement. The seTLM context is not extended or inspected by these rules (it is only ``threaded through'' them opaquely).

%There are two unexpanded expression forms that do not correspond to an expanded expression form: the seTLM definition form, and the seTLM application form. The rules governing these two forms interact with the seTLM context, and are the topics of the next two subsections, respectively.

\subsection{seTLM Definitions}\label{sec:U-uetsm-definition}
The seTLM definition form is \[\uesyntax{\tsmv}{\utau}{\eparse}{\ue}\] 
%The operational form corresponding to this stylized form is \[\audefuetsm{\utau}{\eparse}{\tsmv}{\ue}\]
An unexpanded expression of this form defines an {seTLM} identified as $\tsmv$ with \emph{unexpanded type annotation} $\utau$ and \emph{parse function} $\eparse$ for use within $\ue$. 

Rule (\ref*{rule:expandsU-syntax}) defines typed expansion of this form:
% \begin{subequations}[resume]
% \begin{equation*}\label{rule:expandsU-syntax}
% \inferrule{
%   \istypeU{\Delta}{\tau}\\
%   \expandsU{\emptyset}{\emptyset}{\emptyset}{\ueparse}{\eparse}{\aparr{\tBody}{\tParseResultExp}}\\\\
%   \expandsU{\Delta}{\Gamma}{\uPsi, \xuetsmbnd{\tsmv}{\tau}{\eparse}}{\ue}{e}{\tau'}
% }{
%   \expandsUX{\audefuetsm{\tau}{\ueparse}{\tsmv}{\ue}}{e}{\tau'}
% }
% \end{equation*}
\begin{equation*}\tag{\ref{rule:expandsU-syntax}}
\inferrule{
  \expandsTU{\uDelta}{\utau}{\tau}\\
  \hastypeU{\emptyset}{\emptyset}{\eparse}{\aparr{\tBody}{\tParseResultExp}}\\\\
  \evalU{\eparse}{\eparse'}\\
  \expandsU{\uDelta}{\uGamma}{\uPsi, \uShyp{\tsmv}{a}{\tau}{\eparse'}}{\ue}{e}{\tau'}
}{
  \expandsUX{\uesyntax{\tsmv}{\utau}{\eparse}{\ue}}{e}{\tau'}
}
\end{equation*}
% \end{subequations}
The premises of this rule can be understood as follows, in order:
\begin{enumerate}
\item The first premise expands the unexpanded type annotation.

\item The second premise checks that the parse function, $\eparse$, is a closed expanded function\footnote{In Chapter \ref{chap:static-eval}, we add the machinery necessary for parse functions that are neither closed nor yet expanded.} of the following type: \[\aparr{\tBody}{\tParseResultExp}\] 

%$\miniVerseUE$.%to generate the \emph{expanded parse function}, $\eparse$. 
 %Notice that this occurs under empty contexts, i.e. parse functions cannot refer to the surrounding bindings. 
%The parse function must be of type $\aparr{\tBody}{\tParseResultExp}$ where the type abbreviations $\tBody$ and $\tParseResultExp$ are defined as follows.

The type abbreviated $\tBody$ classifies encodings of literal bodies, $b$. The mapping from literal bodies to values of type $\tBody$ is defined by the \emph{body encoding judgement} $\encodeBody{b}{\ebody}$. An inverse mapping is defined   by the \emph{body decoding judgement} $\decodeBody{\ebody}{b}$.
\[\begin{array}{ll}
\textbf{Judgement Form} & \textbf{Description}\\
\encodeBody{b}{e} & \text{$b$ has encoding $e$}\\
\decodeBody{e}{b} & \text{$e$ has decoding $b$}
\end{array}\]
Rather than defining $\tBody$ explicitly, and these judgements inductively against that definition (which would be tedious and uninteresting), it suffices to define the following condition, which establishes an isomorphism between literal bodies and values of type $\tBody$ mediated by the judgements above.

\begingroup
\def\thetheorem{\ref{condition:body-isomorphism}}
\begin{condition}[Body Isomorphism] ~
\begin{enumerate}
\item For every literal body $b$, we have that $\encodeBody{b}{\ebody}$ for some $\ebody$ such that $\hastypeUC{\ebody}{\tBody}$ and $\isvalU{\ebody}$.
\item If $\hastypeUC{\ebody}{\tBody}$ and $\isvalU{\ebody}$ then $\decodeBody{\ebody}{b}$ for some $b$.
\item If $\encodeBody{b}{\ebody}$ then $\decodeBody{\ebody}{b}$.
\item If $\hastypeUC{\ebody}{\tBody}$ and $\isvalU{\ebody}$ and $\decodeBody{\ebody}{b}$ then $\encodeBody{b}{\ebody}$. 
\item If $\encodeBody{b}{\ebody}$ and $\encodeBody{b}{\ebody'}$ then $\ebody = \ebody'$.
\item If $\hastypeUC{\ebody}{\tBody}$ and $\isvalU{\ebody}$ and $\decodeBody{\ebody}{b}$ and $\decodeBody{\ebody}{b'}$ then $b=b'$.
\end{enumerate}
\end{condition}
\endgroup

The return type of the parse function, $\tParseResultExp$, abbreviates a labeled sum type that distinguishes parse errors from successful parses:\footnote{In VerseML, the \li{ParseError} constructor of \li{parse_result} required an error message and an error location, but we omit these in our formalization for simplicity.}
\begin{align*}
L_\mathtt{SE} & \defeq \lbltxt{ParseError}, \lbltxt{SuccessE}\\
\tParseResultExp & \defeq \asum{\labelset_\mathtt{SE}}{
  \mapitem{\lbltxt{ParseError}}{\prodt{}}, 
  \mapitem{\lbltxt{SuccessE}}{\tCEExp}
}
\end{align*} %[\mapitem{\lbltxt{ParseError}}{\prodt{}}, \mapitem{\lbltxt{SuccessE}}{\tCEExp}]
% \] 

The type abbreviated $\tCEExp$ classifies encodings of \emph{proto-expressions}, $\ce$ (pronounced ``grave $e$''.) The syntax of proto-expressions, defined in Figure \ref{fig:U-candidate-terms}, will be described when we describe proto-expansion validation in Sec. \ref{sec:ce-syntax-U}. The mapping from proto-expressions to values of type $\tCEExp$ is defined by the \emph{proto-expression encoding judgement}, $\encodeCondE{\ce}{e}$. An inverse mapping is defined by the \emph{proto-expression decoding judgement}, $\decodeCondE{e}{\ce}$.

\[\begin{array}{ll}
\textbf{Judgement Form} & \textbf{Description}\\
\encodeCondE{\ce}{e} & \text{$\ce$ has encoding $e$}\\
\decodeCondE{e}{\ce} & \text{$e$ has decoding $\ce$}
\end{array}\]

Again, rather than picking a particular definition of $\tCEExp$ and defining the judgements above inductively against it, we only state the following condition, which establishes an isomorphism between values of type $\tCEExp$ and proto-expressions.

\begingroup 
\def\thetheorem{\ref{condition:proto-expression-isomorphism}}
\begin{condition}[Proto-Expression Isomorphism] ~
\begin{enumerate}
\item For every $\ce$, we have $\encodeCondE{\ce}{\ecand}$ for some $\ecand$ such that $\hastypeUC{\ecand}{\tCEExp}$ and $\isvalU{\ecand}$.
\item If $\hastypeUC{\ecand}{\tCEExp}$ and $\isvalU{\ecand}$ then $\decodeCondE{\ecand}{\ce}$ for some $\ce$.
\item If $\encodeCondE{\ce}{\ecand}$ then $\decodeCondE{\ecand}{\ce}$.
\item If $\hastypeUC{\ecand}{\tCEExp}$ and $\isvalU{\ecand}$ and $\decodeCondE{\ecand}{\ce}$ then $\encodeCondE{\ce}{\ecand}$.
\item If $\encodeCondE{\ce}{\ecand}$ and $\encodeCondE{\ce}{\ecand'}$ then $\ecand=\ecand'$.
\item If $\hastypeUC{\ecand}{\tCEExp}$ and $\isvalU{\ecand}$ and $\decodeCondE{\ecand}{\ce}$ and $\decodeCondE{\ecand}{\ce'}$ then $\ce=\ce'$.
\end{enumerate}
\end{condition}
\endgroup

\item The third premise of Rule (\ref{rule:expandsU-syntax}) evaluates the parse function to a value.
\item The final premise of Rule (\ref{rule:expandsU-syntax}) extends the seTLM context, $\uPsi$, with the newly determined {seTLM definition}, and proceeds to assign a type, $\tau'$, and expansion, $e$, to $\ue$. The conclusion of Rule (\ref{rule:expandsU-syntax}) assigns this type and expansion to the seTLM definition as a whole.% i.e. TLMs define behavior that is relevant during typed expansion, but not during evaluation. 



\emph{seTLM contexts}, $\uPsi$, are of the form $\uAS{\uA}{\Psi}$, where $\uA$ is a \emph{TLM identifier expansion context} and $\Psi$ is a \emph{seTLM definition context}. 

A \emph{TLM identifier expansion context}, $\uA$, is a finite function mapping each TLM identifier $\tsmv \in \domof{\uA}$ to the \emph{TLM identifier expansion}, $\vExpands{\tsmv}{a}$, for some \emph{TLM name}, $a$. We write $\ctxUpdate{\uA}{\tsmv}{a}$ for the TLM identifier expansion context that maps $\tsmv$ to $\vExpands{\tsmv}{a}$, and defers to $\uA$ for all other TLM identifiers (i.e. the previous mapping is \emph{updated}.)

An \emph{seTLM definition context}, $\Psi$, is a finite function mapping each TLM name $a \in \domof{\Psi}$ to an \emph{expanded seTLM definition}, $\xuetsmbnd{a}{\tau}{\eparse}$, where $\tau$ is the seTLM's type annotation, and $\eparse$ is its parse function. We write $\Psi, \xuetsmbnd{a}{\tau}{\eparse}$ when $a \notin \domof{\Psi}$ for the extension of $\Psi$ that maps $a$ to $\xuetsmbnd{a}{\tau}{\eparse}$. % We write $\uetsmenv{\Delta}{\Psi}$  when all the type annotations in $\Psi$ are well-formed assuming $\Delta$, and the parse functions in $\Psi$ are closed and of type $\parr{\tBody}{\tParseResultExp}$.

We define $\uPsi, \uShyp{\tsmv}{a}{\tau}{\eparse}$, when $\uPsi=\uAS{\uA}{\Psi}$, as an abbreviation of \[\uAS{\ctxUpdate{\uA}{\tsmv}{a}}{\Psi, \xuetsmbnd{a}{\tau}{\eparse}}\]

We distinguish TLM identifiers, $\tsmv$, from TLM names, $a$, for much the same reason that we distinguish type and expression identifiers from type and expression variables: in order to support TLM definitions identified in the same way as a previously defined TLM definition, without an implicit renaming convention. %Moreover, this distinction will be crucial in the semantics of TLM abbreviations in Chapter \ref{chap:ptsms}. 

\end{enumerate}


% \[\begin{array}{ll}
% \textbf{Judgement Form} & \textbf{Description}\\
% \uetsmenv{\Delta}{\uPsi} & \text{$\uPsi$ is well-formed assuming $\Delta$}\end{array}\]
% This judgement is inductively defined by the following rules:
% \begin{subequations}[intermezzo]\label{rules:uetsmenv-U}
% \begin{equation*}\label{rule:uetsmenv-empty}
% \inferrule{ }{\uetsmenv{\Delta}{\emptyset}}
% \end{equation*}
% \begin{equation*}\label{rule:uetsmenv-ext}
% \inferrule{
%   \uetsmenv{\Delta}{\uPsi}\\
%   \istypeU{\Delta}{\tau}\\
%   \hastypeU{\emptyset}{\emptyset}{\eparse}{\aparr{\tBody}{\tParseResultExp}}
% }{
%   \uetsmenv{\Delta}{\uPsi, \xuetsmbnd{\tsmv}{\tau}{\eparse}}
% }
% \end{equation*}
% \end{subequations}

\subsection{seTLM Application}\label{sec:U-uetsm-application}
The unexpanded expression form for applying an seTLM named $\tsmv$ to a literal form with literal body $b$ is:
\[
\utsmap{\tsmv}{b}
\] 
This stylized form uses backticks to delimit the literal body, but other generalized literal forms, like those described in Figure \ref{fig:literal-forms}, could also be included as derived forms in the textual syntax. % (we omit them for simplicity).
%The corresponding operational form is $\autsmap{b}{\tsmv}$. %i.e. for each literal body $b$, the operator $\texttt{uapuetsm}[b]$ is indexed by the TLM name $\tsmv$ and takes no arguments. %\footnote{This is in following the conventions in \emph{PFPL} \cite{pfpl}, where operators parameters allow for the use of metatheoretic objects that are not syntax trees or binding trees, e.g. $\mathsf{str}[s]$ and $\mathsf{num}[n]$.} This operator is indexed by the TLM name $\tsmv$ and takes no arguments. 

The typed expansion rule governing seTLM application is below:
% \begin{subequations}[resume]
% \begin{equation*}\label{rule:expandsU-tsmap}
% \inferrule{
%   \encodeBody{b}{\ebody}\\
%   \evalU{\ap{\eparse}{\ebody}}{\inj{\lbltxt{SuccessE}}{\ecand}}\\
%   \decodeCondE{\ecand}{\ce}\\\\
%   \cvalidE{\emptyset}{\emptyset}{\esceneU{\Delta}{\Gamma}{\uPsi, \xuetsmbnd{\tsmv}{\tau}{\eparse}}{b}}{\ce}{e}{\tau}
% }{
%   \expandsU{\Delta}{\Gamma}{\uPsi, \xuetsmbnd{\tsmv}{\tau}{\eparse}}{\autsmap{b}{\tsmv}}{e}{\tau}
% }
% \end{equation*}
\begin{equation*}\tag{\ref{rule:expandsU-tsmap}}
\inferrule{
  \uPsi = \uPsi', \uShyp{\tsmv}{a}{\tau}{\eparse}\\\\
  \encodeBody{b}{\ebody}\\
  \evalU{\ap{\eparse}{\ebody}}{\aein{\mathtt{SuccessE}}{\ecand}}\\
  \decodeCondE{\ecand}{\ce}\\\\
  \segOK{\segof{\ce}}{b}\\
  \cvalidE{\emptyset}{\emptyset}{\esceneU{\uDelta}{\uGamma}{\uPsi}{b}}{\ce}{e}{\tau}
}{
  \expandsU{\uDelta}{\uGamma}{\uPsi}{\utsmap{\tsmv}{b}}{e}{\tau}
}
\end{equation*}
The premises of Rule (\ref{rule:expandsU-tsmap}) can be understood as follows, in order:
\begin{enumerate}
\item The first premise ensures that $\tsmv$ has been defined and extracts the type annotation and parse function.
\item The second premise determines the encoding of the literal body, $\ebody$. This term is closed per Condition \ref{condition:body-isomorphism}.
\item The third premise applies the parse function $\eparse$ to the encoding of the literal body. The parse function is closed by well-formedness of $\uPsi$ (which, in turn, is maintained by the TLM definition rule, Rule (\ref{rule:expandsU-syntax}), described above). 

If parsing succeeds, i.e. a value of the form $\aein{\mathtt{SuccessE}}{\ecand}$ results from evaluation, then $\ecand$ will be a value of type $\tCEExp$ (assuming a well-formed seTLM context, by application of the Preservation assumption, Assumption \ref{condition:preservation-UP}.) We call $\ecand$ the \emph{encoding of the proto-expansion}.

If the parse function produces a value labeled $\lbltxt{ParseError}$, then typed expansion fails. No rule is necessary to handle this case. 

\item The fourth premise decodes the encoding of the proto-expansion to produce the \emph{proto-expansion}, $\ce$, itself.

\item The fifth premise determines a segmentation, $\segof{\ce}$, and ensures that it is valid with respect to $b$. In particular, the predicate $\segOK{\psi}{b}$ checks that each segment in $\psi$, has non-negative length and is within bounds of $b$, and that the segments in $\psi$ do not overlap and operate at a consistent sort and type. The definition of this predicate is given in Appendix \ref{appendix:segmentations-U}. 

\item The final premise of Rule (\ref{rule:expandsU-tsmap}) \emph{validates} the proto-expansion and simultaneously generates the \emph{final expansion}, $e$, which appears in the conclusion of the rule. The proto-expression validation judgement is discussed next.
\end{enumerate}
\subsection{Syntax of Proto-Expansions}\label{sec:ce-syntax-U}
\begin{figure}
\hspace{-5px}$\arraycolsep=3.5pt\begin{array}{lllllll}
\textbf{Sort} & & & \textbf{Operational Form} & \textbf{Stylized Form} & \textbf{Description}\\
\mathsf{PrTyp} & \ctau & ::= & t & t & \text{variable}\\
&&& \aceparr{\ctau}{\ctau} & \parr{\ctau}{\ctau} & \text{partial function}\\
&&& \aceall{t}{\ctau} & \forallt{t}{\ctau} & \text{polymorphic}\\
&&& \acerec{t}{\ctau} & \rect{t}{\ctau} & \text{recursive}\\
&&& \aceprod{\labelset}{\mapschema{\ctau}{i}{\labelset}} & \prodt{\mapschema{\ctau}{i}{\labelset}} & \text{labeled product}\\
&&& \acesum{\labelset}{\mapschema{\ctau}{i}{\labelset}} & \sumt{\mapschema{\ctau}{i}{\labelset}} & \text{labeled sum}\\
\LCC &&& \color{Yellow} & \color{Yellow} & \color{Yellow}\\
&&& \acesplicedt{m}{n} & \splicedt{m}{n} & \text{spliced type ref.}\\\ECC
\mathsf{PrExp} & \ce & ::= & x & x & \text{variable}\\
&&& \aceasc{\ctau}{\ce} & \asc{\ce}{\ctau} & \text{ascription}\\
&&& \aceletsyn{x}{\ce}{\ce} & \letsyn{x}{\ce}{\ce} & \text{value binding}\\
&&& \acelam{\ctau}{x}{\ce} & \lam{x}{\ctau}{\ce} & \text{abstraction}\\
&&& \aceap{\ce}{\ce} & \ap{\ce}{\ce} & \text{application}\\
&&& \acetlam{t}{\ce} & \Lam{t}{\ce} & \text{type abstraction}\\
&&& \acetap{\ce}{\ctau} & \App{\ce}{\ctau} & \text{type application}\\
&&& \acefold{\ce} & \fold{\ce} & \text{fold}\\
&&& \aceunfold{\ce} & \unfold{\ce} & \text{unfold}\\
&&& \acetpl{\labelset}{\mapschema{\ce}{i}{\labelset}} & \tpl{\mapschema{\ce}{i}{\labelset}} & \text{labeled tuple}\\
&&& \acepr{\ell}{\ce} & \prj{\ce}{\ell} & \text{projection}\\
&&& \acein{\ell}{\ce} & \inj{\ell}{\ce} & \text{injection}\\
&&& \acecase{\labelset}{\ce}{\mapschemab{x}{\ce}{i}{\labelset}} & \caseof{\ce}{\mapschemab{x}{\ce}{i}{\labelset}} & \text{case analysis}\\
\LCC &&& \color{Yellow} & \color{Yellow} & \color{Yellow}\\
&&& \acesplicede{m}{n}{\ctau} & \splicede{m}{n}{\ctau} & \text{spliced expr. ref.}\ECC
\end{array}$
\caption[Syntax of $\miniVerseUE$ proto-types and proto-expressions]{Syntax of $\miniVerseUE$ proto-types and proto-expressions.}
\label{fig:U-candidate-terms}
\end{figure}

Figure \ref{fig:U-candidate-terms} defines the syntax of proto-types, $\ctau$, and proto-expressions, $\ce$. Proto-types and -expressions are ABTs identified up to $\alpha$-equivalence in the usual manner.

Each expanded form maps onto a proto-expansion form. We refer to these as the \emph{common proto-expansion forms}. The mapping is given explicitly in Appendix \ref{appendix:proto-expansions-SES}.

There are two ``interesting'' proto-expansion forms, highlighted in yellow in Figure \ref{fig:U-candidate-terms}: a proto-type form for \emph{references to spliced unexpanded types}, $\acesplicedt{m}{n}$, and a proto-expression form for \emph{references to spliced unexpanded expressions}, $\acesplicede{m}{n}{\ctau}$, where $m$ and $n$ are natural numbers.%TLM utilize these to splice types and unexpanded expressions out of literal bodies.

\subsection{Proto-Expansion Validation}\label{sec:ce-validation-U}



The \emph{proto-expansion validation judgements} validate proto-types and proto-expressions and simultaneously generate their final expansions.% are types and expanded expressions, respectively.
\[\begin{array}{ll}
\textbf{Judgement Form} & \textbf{Description}\\
\cvalidT{\Delta}{\tscenev}{\ctau}{\tau} & \text{$\ctau$ has well-formed expansion $\tau$}\\
\cvalidE{\Delta}{\Gamma}{\escenev}{\ce}{e}{\tau} & \text{$\ce$ has expansion $e$ of type $\tau$}
\end{array}\]
\emph{Type splicing scenes}, $\tscenev$, are of the form $\tsceneU{\uDelta}{b}$ and \emph{expression splicing scenes}, $\escenev$, are of the form $\esceneU{\uDelta}{\uGamma}{\uPsi}{b}$. We write $\tsfrom{\escenev}$ for the type splicing scene constructed by dropping the unexpanded typing context and seTLM context from $\escenev$:
\[\tsfrom{\esceneU{\uDelta}{\uGamma}{\uPsi}{b}} = \tsceneU{\uDelta}{b}\]
The purpose of splicing scenes is to ``remember'', during the proto-expansion validation process, the unexpanded type formation context, $\uDelta$, unexpanded typing context, $\uGamma$, seTLM context, $\uPsi$, and the literal body, $b$, from the seTLM application site (cf. Rule (\ref{rule:expandsU-tsmap}) above.) These structures will be necessary to validate the references to spliced unexpanded types and expressions that appear within the proto-expansion.

\subsubsection{Proto-Type Validation}\label{sec:SE-proto-type-validation}
The \emph{proto-type validation judgement}, $\cvalidT{\Delta}{\tscenev}{\ctau}{\tau}$, is inductively defined by Rules (\ref{rules:cvalidT-U}).

\paragraph{Common Forms} Rules (\ref{rule:cvalidT-U-tvar}) through (\ref{rule:cvalidT-U-sum}) validate proto-types of common form. These rules, like the rules for common unexpanded type forms,  mirror the corresponding type formation rules, i.e. Rules (\ref{rules:istypeU}). The type splicing scene, $\tscenev$, passes opaquely through these rules.  The first three of these are reproduced below.
%Each of these rules is defined based on the corresponding type formation rule, i.e. Rules (\ref{rule:istypeU-var}) through (\ref{rule:istypeU-sum}), respectively. For example, the following proto-types validation rules are based on type formation rules (\ref{rule:istypeU-var}), (\ref{rule:istypeU-parr}) and (\ref{rule:istypeU-all}), respectively: 
% \begin{subequations}%\label{rules:cvalidT-U}
\begin{equation*}\tag{\ref{rule:cvalidT-U-tvar}}
\inferrule{ }{
  \cvalidT{\Delta, \Dhyp{t}}{\tscenev}{t}{t}
}
\end{equation*}
\begin{equation*}\tag{\ref{rule:cvalidT-U-parr}}
  \inferrule{
    \cvalidT{\Delta}{\tscenev}{\ctau_1}{\tau_1}\\
    \cvalidT{\Delta}{\tscenev}{\ctau_2}{\tau_2}
  }{
    \cvalidT{\Delta}{\tscenev}{\aceparr{\ctau_1}{\ctau_2}}{\aparr{\tau_1}{\tau_2}}
  }
\end{equation*}
\begin{equation*}\tag{\ref{rule:cvalidT-U-all}}
  \inferrule {
    \cvalidT{\Delta, \Dhyp{t}}{\tscenev}{\ctau}{\tau}
  }{
    \cvalidT{\Delta}{\tscenev}{\aceall{t}{\ctau}}{\aall{t}{\tau}}
  }
\end{equation*}
% \begin{equation*}\label{rule:cvalidT-U-rec}
%   \inferrule{
%     \cvalidT{\Delta, \Dhyp{t}}{\tscenev}{\ctau}{\tau}
%   }{
%     \cvalidT{\Delta}{\tscenev}{\acerec{t}{\ctau}}{\arec{t}{\tau}}
%   }
% \end{equation*}
% \begin{equation*}\label{rule:cvalidT-U-prod}
%   \inferrule{
%     \{\cvalidT{\Delta}{\tscenev}{\ctau_i}{\tau_i}\}_{i \in \labelset}
%   }{
%     \cvalidT{\Delta}{\tscenev}{\aceprod{\labelset}{\mapschema{\ctau}{i}{\labelset}}}{\aprod{\labelset}{\mapschema{\tau}{i}{\labelset}}}
%   }
% \end{equation*}
% \begin{equation*}\label{rule:cvalidT-U-sum}
%   \inferrule{
%     \{\cvalidT{\Delta}{\tscenev}{\ctau_i}{\tau_i}\}_{i \in \labelset}
%   }{
%     \cvalidT{\Delta}{\tscenev}{\acesum{\labelset}{\mapschema{\ctau}{i}{\labelset}}}{\asum{\labelset}{\mapschema{\tau}{i}{\labelset}}}
%   }
% \end{equation*}


% We can express this scheme more precisely with the following rule transformation. For each rule in Rules (\ref{rules:istypeU}), 
% \begin{mathpar}
% % \refstepcounter{equation}
% % \label{rule:cvalidT-U-rec}
% % \refstepcounter{equation}
% % \label{rule:cvalidT-U-prod}
% % \refstepcounter{equation}
% % \label{rule:cvalidT-U-sum}
% % \inferrule{J_1\\\cdots\\J_k}{J}
% \end{mathpar}
% the corresponding proto-types validation rule is
% \begin{mathpar}
% \inferrule{
%   \VTypof{J_1}\\
%   \cdots\\
%   \VTypof{J_k}
% }{
%   \VTypof{J}
% }
% \end{mathpar}
% where 
% \[\begin{split}
% \VTypof{\istypeU{\Delta}{\tau}} & = \cvalidT{\Delta}{\tscenev}{\VTypof{\tau}}{\tau}\\
% \VTypof{\{J_i\}_{i \in \labelset}} & = \{\VTypof{J_i}\}_{i \in \labelset}
% \end{split}\]
% and where $\VTypof{\tau}$, when $\tau$ is a metapattern of sort $\mathsf{Typ}$, is a metapattern of sort $\mathsf{CETyp}$ defined as follows:
% \begin{itemize}
% \item When $\tau$ is of definite form, $\VTypof{\tau}$ is defined as follows:
% \begin{align*}
% \VTypof{t} & = t\\
% \VTypof{\aparr{\tau_1}{\tau_2}} & = \aceparr{\VTypof{\tau_1}}{\VTypof{\tau_2}}\\
% \VTypof{\aall{t}{\tau}} & = \aceall{t}{\VTypof{\tau}}\\
% \VTypof{\arec{t}{\tau}} & = \acerec{t}{\VTypof{\tau}}\\
% \VTypof{\aprod{\labelset}{\mapschema{\tau}{i}{\labelset}}} & = \aceprod{\labelset}{\mapschemax{\VTypofv}{\tau}{i}{\labelset}}\\
% \VTypof{\asum{\labelset}{\mapschema{\tau}{i}{\labelset}}} & = \acesum{\labelset}{\mapschemax{\VTypofv}{\tau}{i}{\labelset}}
% \end{align*}
% \item When $\tau$ is of indefinite form, $\VTypof{\tau}$ is a uniquely corresponding metapattern also of indefinite form. For example, $\VTypof{\tau_1}=\ctau_1$ and $\VTypof{\tau_2}=\ctau_2$.
% \end{itemize}

% It is instructive to use this rule transformation to generate Rules (\ref{rule:cvalidT-U-tvar}) through (\ref{rule:cvalidT-U-all}) above. We omit the remaining rules, i.e. Rules (\ref*{rule:cvalidT-U-rec}) through (\ref*{rule:cvalidT-U-sum}). 

Notice that in Rule (\ref{rule:cvalidT-U-tvar}), only type variables tracked by $\Delta$, the expansion's local type validation context, are well-formed. Type variables tracked by the application site unexpanded type formation context, which is a component of the type splicing scene, $\tscenev$, are not validated. %Indeed, $\tscenev$ passes opaquely through the rules above. %This achieves \emph{context-independent expansion} as described in Sec. \ref{sec:splicing-and-hygiene} for type variables -- seTLMs cannot impose ``hidden constraints'' on the application site unexpanded type formation context, because the type variables bound at the application site are simply not directly available to proto-types.

\paragraph{References to Spliced Types} The only proto-type form that does not correspond to a type form is $\acesplicedt{m}{n}$, which is a \emph{reference to a spliced unexpanded type}, i.e. it indicates that an unexpanded type should be parsed out from the literal body, which appears in the type splicing scene $\tscenev$, beginning at position $m$ and ending at position $n$, where $m$ and $n$ are natural numbers. Rule (\ref{rule:cvalidT-U-splicedt}) governs this form:
\begin{equation*}\tag{\ref{rule:cvalidT-U-splicedt}}
  \inferrule{
    \parseUTyp{\bsubseq{b}{m}{n}}{\utau}\\
    \expandsTU{\uDD{\uD}{\Delta_\text{app}}}{\utau}{\tau}\\
    \Delta \cap \Delta_\text{app} = \emptyset
  }{
    \cvalidT{\Delta}{\tsceneU{\uDD{\uD}{\Delta_\text{app}}}{b}}{\acesplicedt{m}{n}}{\tau}
  }
\end{equation*}
The first premise of this rule extracts the indicated subsequence of $b$ using the partial metafunction $\bsubseq{b}{m}{n}$ and parses it using the partial metafunction $\mathsf{parseUTyp}(b)$, which was characterized in Sec. \ref{sec:syntax-U}, to produce the spliced unexpanded type itself, $\utau$.

The second premise of Rule (\ref{rule:cvalidT-U-splicedt}) performs type expansion of $\utau$ under the application site unexpanded type formation context, $\uDD{\uD}{\Delta_\text{app}}$, which is a component of the type splicing scene. The hypotheses in the expansion's local type formation context, $\Delta$, are not made available to $\tau$. %This enforces the injunction on shadowing as described in Sec. \ref{sec:splicing-and-hygiene} for type variables that appear in proto-types. 

The third premise of Rule (\ref{rule:cvalidT-U-splicedt}) imposes the constraint that the proto-expansion's type formation context, $\Delta$, be disjoint from the application site type formation context, $\Delta_\text{app}$. This premise can always be discharged by $\alpha$-varying the proto-expansion that the reference to the spliced type appears within. 

Together, these two premises enforce the injunction on type variable capture as described in Sec. \ref{sec:uetsms-validation} -- the TLM provider can choose type variable names freely within a proto-expansion. We will consider this formally in Sec. \ref{sec:SE-metatheory} below. %, because the language prevents them from shadowing type variables at the application site (by $\alpha$-varying the proto-expansion as needed.)%Such a change in bound variable names is possible again because variables bound by the seTLM provider in a proto-expansion cannot ``leak into'' spliced terms because the hypotheses in $\Delta$ are not made available to the spliced type, $\tau$. 

Rules (\ref{rules:cvalidT-U}) validate the following lemma, which establishes that the final expansion of a valid proto-type is a well-formed type under the combined type formation context.
\begingroup
\def\thetheorem{\ref{lemma:candidate-expansion-type-validation}}
\begin{lemma}[Proto-Expansion Type Validation]
If $\cvalidT{\Delta}{\tsceneU{\uDD{\uD}{\Delta_\text{app}}}{b}}{\ctau}{\tau}$ and $\Delta \cap \Delta_\text{app}=\emptyset$ then $\istypeU{\Dcons{\Delta}{\Delta_\text{app}}}{\tau}$.
\end{lemma}
\endgroup

\subsubsection{Proto-Expression Validation}
The \emph{proto-expression validation judgement}, $\cvalidE{\Delta}{\Gamma}{\escenev}{\ce}{e}{\tau}$, is defined mutually inductively with the typed expansion judgement by Rules (\ref{rules:cvalidE-U}) as follows.% This is necessary because a typed expansion judgement appears as a premise in Rule (\ref{rule:cvalidE-U-splicede}) below, and a proto-expression validation judgement appears as a premise in Rule (\ref{rule:expandsU-tsmap}) above.

\paragraph{Common Forms} Rules (\ref{rule:cvalidE-U-var}) through (\ref{rule:cvalidE-U-case}) validate proto-expressions of common form, as well as ascriptions and let binding. Once again, the rules for common forms mirror the typing rules, i.e. Rules (\ref{rules:hastypeU}). The expression splicing scene, $\escenev$, passes opaquely through these rules. The first five of these rules are reproduced below:
%For each expanded expression form defined in Figure \ref{fig:U-expanded-terms}, Figure \ref{fig:U-candidate-terms} defines a corresponding proto-expression form. The validation rules for proto-expressions of these forms are each based on the corresponding typing rule in Rules (\ref{rules:hastypeU}). For example, the validation rules for proto-expressions of variable, function and function application form  are based on Rules (\ref{rule:hastypeU-var}) through (\ref{rule:hastypeU-ap}), respectively:
%\begin{subequations}%\label{rules:cvalidE-U}
\begin{equation*}\tag{\ref{rule:cvalidE-U-var}}
\inferrule{ }{
  \cvalidE{\Delta}{\Gamma, \Ghyp{x}{\tau}}{\escenev}{x}{x}{\tau}
}
\end{equation*}
\begin{equation*}\tag{\ref{rule:cvalidE-U-asc}}
\inferrule{
  \cvalidT{\Delta}{\tsfrom{\escenev}}{\ctau}{\tau}\\
  \cvalidE{\Delta}{\Gamma}{\escenev}{\ce}{e}{\tau}
}{
  \cvalidE{\Delta}{\Gamma}{\escenev}{\aceasc{\ctau}{\ce}}{e}{\tau}
}
\end{equation*}
\begin{equation*}\tag{\ref{rule:cvalidE-U-letsyn}}
  \inferrule{
    \cvalidE{\Delta}{\Gamma}{\escenev}{\ce_1}{e_1}{\tau_1}\\
    \cvalidE{\Delta}{\Gamma, x : \tau_1}{\ce_2}{e_2}{\tau_2}
  }{
    \cvalidE{\Delta}{\Gamma}{\escenev}{\aceletsyn{x}{\ce_1}{\ce_2}}{
      \aeap{\aelam{\tau_1}{x}{e_2}}{e_1}
    }{\tau_2}
  }
\end{equation*}
\begin{equation*}\tag{\ref{rule:cvalidE-U-lam}}
\inferrule{
  \cvalidT{\Delta}{\tsfrom{\escenev}}{\ctau}{\tau}\\
  \cvalidE{\Delta}{\Gamma, \Ghyp{x}{\tau}}{\escenev}{\ce}{e}{\tau'}
}{
  \cvalidE{\Delta}{\Gamma}{\escenev}{\acelam{\ctau}{x}{\ce}}{\aelam{\tau}{x}{e}}{\aparr{\tau}{\tau'}}
}
\end{equation*}
\begin{equation*}\tag{\ref{rule:cvalidE-U-ap}}
  \inferrule{
    \cvalidE{\Delta}{\Gamma}{\escenev}{\ce_1}{e_1}{\aparr{\tau}{\tau'}}\\
    \cvalidE{\Delta}{\Gamma}{\escenev}{\ce_2}{e_2}{\tau}
  }{
    \cvalidE{\Delta}{\Gamma}{\escenev}{\aceap{\ce_1}{\ce_2}}{\aeap{e_1}{e_2}}{\tau'}
  }
\end{equation*}



Notice that in Rule (\ref{rule:cvalidE-U-var}), only variables tracked by the proto-expansion typing context, $\Gamma$, are validated. Variables  in the application site unexpanded typing context, which appears within the expression splicing scene $\escenev$, are not validated. This achieves \emph{context independence} as described in Sec. \ref{sec:uetsms-validation} -- seTLMs cannot impose ``hidden constraints'' on the application site unexpanded typing context, because the variable bindings at the application site are not directly available to proto-expansions. We will consider this formally in Sec. \ref{sec:SE-metatheory} below.

\paragraph{References to Spliced Unexpanded Expressions} The only proto-expression form that does not correspond to an expanded expression form is $\acesplicede{m}{n}{\ctau}$, which is a \emph{reference to a spliced unexpanded expression}, i.e. it indicates that an unexpanded expression should be parsed out from the literal body beginning at position $m$ and ending at position $n$. Rule (\ref{rule:cvalidE-U-splicede}) governs this form:
\begin{equation*}\tag{\ref{rule:cvalidE-U-splicede}}
\inferrule{
  \cvalidT{\emptyset}{\tsfrom{\escenev}}{\ctau}{\tau}\\
  \escenev=\esceneU{\uDD{\uD}{\Delta_\text{app}}}{\uGG{\uG}{\Gamma_\text{app}}}{\uPsi}{b}\\
  \parseUExp{\bsubseq{b}{m}{n}}{\ue}\\
  \expandsU{\uDD{\uD}{\Delta_\text{app}}}{\uGG{\uG}{\Gamma_\text{app}}}{\uPsi}{\ue}{e}{\tau}\\\\
  \Delta \cap \Delta_\text{app} = \emptyset\\
  \domof{\Gamma} \cap \domof{\Gamma_\text{app}} = \emptyset
}{
  \cvalidE{\Delta}{\Gamma}{\escenev}{\acesplicede{m}{n}{\ctau}}{e}{\tau}
}
\end{equation*}
% \begin{equation*}\label{rule:cvalidE-U-splicede}
% \inferrule{
%   \parseUExp{\bsubseq{b}{m}{n}}{\ue}\\\\
%   \expandsU{\Delta_\text{app}}{\Gamma_\text{app}}{\uPsi}{\ue}{e}{\tau}\\
%   \Delta \cap \Delta_\text{app} = \emptyset\\
%   \domof{\Gamma} \cap \domof{\Gamma_\text{app}} = \emptyset
% }{
%   \cvalidE{\Delta}{\Gamma}{\esceneU{\Delta_\text{app}}{\Gamma_\text{app}}{\uPsi}{b}}{\splicede{m}{n}}{e}{\tau}
% }
% \end{equation*}

The premises of this rule can be understood as follows:
\begin{enumerate}
\item The first premise of this rule validates and expands the type annotation. This type must be context independent.

\item The second premise of this rule serves simply to reveal the components of the expression splicing scene.

\item The third premise of this rule extracts the indicated subsequence of $b$ using the partial metafunction $\bsubseq{b}{m}{n}$ and parses it using the partial metafunction $\mathsf{parseUExp}(b)$, characterized in Sec. \ref{sec:syntax-U}, to produce the referenced spliced unexpanded expression, $\ue$.

\item The fourth premise of Rule (\ref{rule:cvalidE-U-splicede}) performs typed expansion of $\ue$ assuming the application site contexts that appear in the expression splicing scene. Notice that the hypotheses in $\Delta$ and $\Gamma$ are not made available to $\ue$. 

\item The fifth premise of Rule (\ref{rule:cvalidE-U-splicede}) imposes the constraint that the proto-expansion's type formation context, $\Delta$, be disjoint from the application site type formation context, $\Delta_\text{app}$. Similarly, the sixth premise requires that the proto-expansion's typing context, $\Gamma$, be disjoint from the application site typing context, $\Gamma_\text{app}$. These two premises can always be discharged by $\alpha$-varying the proto-expression that the reference to the spliced unexpanded expression appears within. 
Together, these premises enforce the prohibition on capture as described in Sec. \ref{sec:uetsms-validation} -- the TLM provider can choose variable names freely within a proto-expansion, because the language prevents them from shadowing those at the application site. Again, we will consider this formally in Sec. \ref{sec:SE-metatheory} below.
\end{enumerate}
%\end{subequations}
% \begin{subequations}\label{rules:cvalidE-U}
% \begin{equation*}\label{rule:cvalidE-U-var}
% \inferrule{ }{
%   \cvalidE{\Delta}{\Gamma, \Ghyp{x}{\tau}}{\escenev}{x}{x}{\tau}
% }
% \end{equation*}
% \begin{equation*}\label{rule:cvalidE-U-lam}
% \inferrule{
%   \cvalidT{\Delta}{\tsfrom{\escenev}}{\ctau}{\tau}\\
%   \cvalidE{\Delta}{\Gamma, \Ghyp{x}{\tau}}{\escenev}{\ce}{e}{\tau'}
% }{
%   \cvalidE{\Delta}{\Gamma}{\escenev}{\acelam{\ctau}{x}{\ce}}{\aelam{\tau}{x}{e}}{\aparr{\tau}{\tau'}}
% }
% \end{equation*}
% \begin{equation*}\label{rule:cvalidE-U-ap}
%   \inferrule{
%     \cvalidE{\Delta}{\Gamma}{\escenev}{\ce_1}{e_1}{\aparr{\tau}{\tau'}}\\
%     \cvalidE{\Delta}{\Gamma}{\escenev}{\ce_2}{e_2}{\tau}
%   }{
%     \cvalidE{\Delta}{\Gamma}{\escenev}{\aceap{\ce_1}{\ce_2}}{\aeap{e_1}{e_2}}{\tau'}
%   }
% \end{equation*}
% \begin{equation*}\label{rule:cvalidE-U-tlam}
%   \inferrule{
%     \cvalidE{\Delta, \Dhyp{t}}{\Gamma}{\escenev}{\ce}{e}{\tau}
%   }{
%     \cvalidEX{\acetlam{t}{\ce}}{\aetlam{t}{e}}{\aall{t}{\tau}}
%   }
% \end{equation*}
% \begin{equation*}\label{rule:cvalidE-U-tap}
%   \inferrule{
%     \cvalidEX{\ce}{e}{\aall{t}{\tau}}\\
%     \cvalidT{\Delta}{\tsfrom{\escenev}}{\ctau'}{\tau'}
%   }{
%     \cvalidEX{\acetap{\ce}{\ctau'}}{\aetap{e}{\tau'}}{[\tau'/t]\tau}
%   }
% \end{equation*}
% \begin{equation*}\label{rule:cvalidE-U-fold}
%   \inferrule{
%     \cvalidT{\Delta, \Dhyp{t}}{\escenev}{\ctau}{\tau}\\
%     \cvalidEX{\ce}{e}{[\arec{t}{\tau}/t]\tau}
%   }{
%     \cvalidEX{\acefold{t}{\ctau}{\ce}}{\aefold{e}}{\arec{t}{\tau}}
%   }
% \end{equation*}
% \begin{equation*}\label{rule:cvalidE-U-unfold}
%   \inferrule{
%     \cvalidEX{\ce}{e}{\arec{t}{\tau}}
%   }{
%     \cvalidEX{\aceunfold{\ce}}{\aeunfold{e}}{[\arec{t}{\tau}/t]\tau}
%   }
% \end{equation*}
% \begin{equation*}\label{rule:cvalidE-U-tpl}
%   \inferrule{
%     \{\cvalidEX{\ce_i}{e_i}{\tau_i}\}_{i \in \labelset}
%   }{
%     \cvalidEX{\acetpl{\labelset}{\mapschema{\ce}{i}{\labelset}}}{\aetpl{\labelset}{\mapschema{e}{i}{\labelset}}}{\aprod{\labelset}{\mapschema{\tau}{i}{\labelset}}}
%   }
% \end{equation*}
% \begin{equation*}\label{rule:cvalidE-U-pr}
%   \inferrule{
%     \cvalidEX{\ce}{e}{\aprod{\labelset, \ell}{\mapschema{\tau}{i}{\labelset}; \mapitem{\ell}{\tau}}}
%   }{
%     \cvalidEX{\acepr{\ell}{\ce}}{\aepr{\ell}{e}}{\tau}
%   }
% \end{equation*}
% \begin{equation*}\label{rule:cvalidE-U-in}
%   \inferrule{
%     \{\cvalidT{\Delta}{\tsfrom{\escenev}}{\ctau_i}{\tau_i}\}_{i \in \labelset}\\
%     \cvalidT{\Delta}{\tsfrom{\escenev}}{\ctau}{\tau}\\
%     \cvalidEX{\ce}{e}{\tau}
%   }{
%     \left\{\shortstack{$\Delta~\Gamma \vdash_\uPsi \acein{\labelset, \ell}{\ell}{\mapschema{\ctau}{i}{\labelset}; \mapitem{\ell}{\ctau}}{\ce}$\\$\leadsto$\\$\aein{\labelset, \ell}{\ell}{\mapschema{\tau}{i}{\labelset}; \mapitem{\ell}{\tau}}{e} : \asum{\labelset, \ell}{\mapschema{\tau}{i}{\labelset}; \mapitem{\ell}{\tau}}$\vspace{-1.2em}}\right\}
%   }
% \end{equation*}
% \begin{equation*}\label{rule:cvalidE-U-case}
%   \inferrule{
%     \cvalidEX{\ce}{e}{\asum{\labelset}{\mapschema{\tau}{i}{\labelset}}}\\
%     \{\cvalidE{\Delta}{\Gamma, \Ghyp{x_i}{\tau_i}}{\escenev}{\ue_i}{e_i}{\tau}\}_{i \in \labelset}
%   }{
%     \cvalidEX{\acecase{\labelset}{\ce}{\mapschemab{x}{\ce}{i}{\labelset}}}{\aecase{\labelset}{e}{\mapschemab{x}{e}{i}{\labelset}}}{\tau}
%   }
% \end{equation*}
% \begin{equation*}\label{rule:cvalidE-U-splicede}
% \inferrule{
%   \parseUExp{\bsubseq{b}{m}{n}}{\ue}\\\\
%   \Delta \cap \Delta_\text{app} = \emptyset\\
%   \domof{\Gamma} \cap \domof{\Gamma_\text{app}} = \emptyset\\
%   \expandsU{\Delta_\text{app}}{\Gamma_\text{app}}{\uPsi}{\ue}{e}{\tau}
% }{
%   \cvalidE{\Delta}{\Gamma}{\esceneU{\Delta_\text{app}}{\Gamma_\text{app}}{\uPsi}{b}}{\acesplicede{m}{n}}{e}{\tau}
% }
% \end{equation*}
% \end{subequations}

% Each form of expanded expression, $e$, corresponds to a form of proto-expression, $\ce$ (compare Figure \ref{fig:U-expanded-terms} and Figure \ref{fig:U-candidate-terms}). For each typing rule in Rules \ref{rules:hastypeU}, there is a corresponding proto-expression validation rule -- Rules (\ref{rule:cvalidE-U-var}) to (\ref{rule:cvalidE-U-case}) -- where the proto-expression and expanded expression correspond. The premises also correspond.


%Candidate expansions cannot themselves define or apply TLMs. This simplifies our metatheory, though it can be inconvenient at times for TLM providers. We discuss adding the ability to use TLMs within proto-expansions in Sec. \ref{sec:tsms-in-expansions}.


\subsection{Metatheory}\label{sec:SE-metatheory}
\subsubsection{Typed Expansion}
Let us now consider Theorem \ref{thm:typed-expansion-short-U}, which was mentioned at the beginnning of Sec. \ref{sec:typed-expansion-U} and is reproduced below:
\begingroup
\def\thetheorem{\ref{thm:typed-expansion-short-U}}
\begin{theorem}[Typed Expression Expansion] \hspace{-3px}If $\expandsU{\uDD{\uD}{\Delta}\hspace{-3px}}{\uGG{\uG}{\Gamma}\hspace{-3px}}{\uPsi}{\ue}{e}{\tau}$ then $\hastypeU{\Delta}{\Gamma}{e}{\tau}$.
\end{theorem}
\endgroup

 To prove this theorem, we must  prove the following stronger theorem, because the proto-expression validation judgement is defined mutually inductively with the typed expansion judgement:

\begingroup
\def\thetheorem{\ref{thm:typed-expansion-full-U}}
\begin{theorem}[Typed Expansion (Full)] ~
\begin{enumerate}
\item If $\expandsU{\uDD{\uD}{\Delta}}{\uGG{\uG}{\Gamma}}{\uAS{\uA}{\Psi}}{\ue}{e}{\tau}$ then $\hastypeU{\Delta}{\Gamma}{e}{\tau}$.
\item If $\cvalidE{\Delta}{\Gamma}{\esceneU{\uDD{\uD}{\Delta_\text{app}}}{\uGG{\uG}{\Gamma_\text{app}}}{\uAS{\uA}{\Psi}}{b}}{\ce}{e}{\tau}$ and $\Delta \cap \Delta_\text{app} = \emptyset$ and $\domof{\Gamma} \cap \domof{\Gamma_\text{app}} = \emptyset$ then $\hastypeU{\Dcons{\Delta}{\Delta_\text{app}}}{\Gcons{\Gamma}{\Gamma_\text{app}}}{e}{\tau}$.
\end{enumerate}
\end{theorem}
\endgroup
\begin{proof}
By mutual rule induction over Rules (\ref{rules:expandsU}) and Rules (\ref{rules:cvalidE-U}). The full proof is given in Appendix \ref{appendix:SES-typed-expression-expansion-metatheory}. We will reproduce the interesting cases below. 

The proof of part 1 proceeds by inducting over the typed expansion assumption. The only interesting cases are those related to seTLM definition and application, reproduced below. In the following cases, let $\uDelta=\uDD{\uD}{\Delta}$ and $\uGamma=\uGG{\uG}{\Gamma}$ and $\uPsi=\uAS{\uA}{\Psi}$.

\begin{byCases}
\item[\text{(\ref{rule:expandsU-syntax})}] We have 
\begin{pfsteps}
  \item \ue=\uesyntax{\tsmv}{\utau'}{\eparse}{\ue'} \BY{assumption}
  \item \expandsTU{\uDelta}{\utau'}{\tau'} \BY{assumption} \pflabel{expandsTU}
 \item \hastypeU{\emptyset}{\emptyset}{\eparse}{\aparr{\tBody}{\tParseResultExp}} \BY{assumption}\pflabel{eparse}
  \item \expandsU{\uDelta}{\uGamma}{\uPsi, \uShyp{\tsmv}{a}{\tau'}{\eparse}}{\ue'}{e}{\tau} \BY{assumption}\pflabel{expandsU}
%  \item \uetsmenv{\Delta}{\Psi} \BY{assumption}\pflabel{uetsmenv1}
 \item \istypeU{\Delta}{\tau'} \BY{Lemma \ref{lemma:type-expansion-U} to \pfref{expandsTU}} \pflabel{istype}
%  \item \uetsmenv{\Delta}{\Psi, \xuetsmbnd{\tsmv}{\tau'}{\eparse}} \BY{Definition \ref{def:seTLM-def-ctx-formation} on \pfref{uetsmenv1}, \pfref{istype} and \pfref{eparse}}\pflabel{uetsmenv3}
  \item \hastypeU{\Delta}{\Gamma}{e}{\tau} \BY{IH, part 1(a) on \pfref{expandsU}}
\end{pfsteps}
\resetpfcounter 

\item[\text{(\ref{rule:expandsU-tsmap})}] We have 
\begin{pfsteps}
  \item \ue=\utsmap{\tsmv}{b} \BY{assumption}
  \item \uA = \uA', \vExpands{\tsmv}{a} \BY{assumption}
  \item \Psi=\Psi', \xuetsmbnd{a}{\tau}{\eparse} \BY{assumption}
  \item \encodeBody{b}{\ebody} \BY{assumption}
  \item \evalU{\eparse(\ebody)}{\aein{\lbltxt{SuccessE}}{\ecand}} \BY{assumption}
  \item \decodeCondE{\ecand}{\ce} \BY{assumption}
  \item \cvalidE{\emptyset}{\emptyset}{\esceneU{\uDelta}{\uGamma}{\uPsi}{b}}{\ce}{e}{\tau} \BY{assumption}\pflabel{cvalidE}
%  \item \uetsmenv{\Delta}{\Psi} \BY{assumption} \pflabel{uetsmenv}
  \item \emptyset \cap \Delta = \emptyset \BY{finite set intersection} \pflabel{delta-cap}
  \item {\emptyset} \cap \domof{\Gamma} = \emptyset \BY{finite set intersection} \pflabel{gamma-cap}
  \item \hastypeU{\emptyset \cup \Delta}{\emptyset \cup \Gamma}{e}{\tau} \BY{IH, part 2 on \pfref{cvalidE}, \pfref{delta-cap}, and \pfref{gamma-cap}} \pflabel{penultimate}
  \item \hastypeU{\Delta}{\Gamma}{e}{\tau} \BY{finite set and finite function identity over \pfref{penultimate}}
\end{pfsteps}
\resetpfcounter
\end{byCases}

The proof of part 2 proceeds by induction over the proto-expression validation assumption. The only interesting case governs references to spliced expressions. In the following cases, let $\uDelta_\text{app}=\uDD{\uD}{\Delta_\text{app}}$ and $\uGamma_\text{app}=\uGG{\uG}{\Gamma_\text{app}}$ and $\uPsi = \uAS{\uA}{\Psi}$.
\begin{byCases}
% \item[\text{(\ref{rule:cvalidE-U-var})}] ~
% \begin{pfsteps*}
%   \item $\ce=x$ \BY{assumption}
%   \item $e=x$ \BY{assumption}
%   \item $\Gamma=\Gamma', \Ghyp{x}{\tau}$ \BY{assumption}
%   \item $\hastypeU{\Dcons{\Delta}{\Delta_\text{app}}}{\Gamma', \Ghyp{x}{\tau}}{x}{\tau}$ \BY{Rule (\ref{rule:hastypeU-var})} \pflabel{hastypeU}
%   \item $\hastypeU{\Dcons{\Delta}{\Delta_\text{app}}}{\Gcons{\Gamma', \Ghyp{x}{\tau}}{\Gamma_\text{app}}}{x}{\tau}$ \BY{Lemma \ref{lemma:weakening-U} over $\Gamma_\text{app}$ to \pfref{hastypeU}}
% \end{pfsteps*}
% \resetpfcounter

% \item[\text{(\ref{rule:cvalidE-U-lam})}] ~
% \begin{pfsteps*}
%   \item $\ce=\acelam{\ctau_1}{x}{\ce'}$ \BY{assumption}
%   \item $e=\aelam{\tau_1}{x}{e'}$ \BY{assumption}
%   \item $\tau=\aparr{\tau_1}{\tau_2}$ \BY{assumption}
%   \item $\cvalidT{\Delta}{\tsceneU{\uDelta_\text{app}}{b}}{\ctau_1}{\tau_1}$ \BY{assumption} \pflabel{cvalidT}
%   \item $\cvalidE{\Delta}{\Gamma, \Ghyp{x}{\tau_1}}{\esceneU{\uDelta_\text{app}}{\uGamma_\text{app}}{\uPsi}{b}}{\ce'}{e'}{\tau_2}$ \BY{assumption} \pflabel{cvalidE}
% %  \item $\uetsmenv{\Delta_\text{app}}{\Psi}$ \BY{assumption} \pflabel{uetsmenv}
%   \item $\Delta \cap \Delta_\text{app}=\emptyset$ \BY{assumption} \pflabel{delta-disjoint}
%   \item $\domof{\Gamma} \cap \domof{\Gamma_\text{app}}=\emptyset$ \BY{assumption} \pflabel{gamma-disjoint}
%   \item $x \notin \domof{\Gamma_\text{app}}$ \BY{identification convention} \pflabel{x-fresh}
%   \item $\domof{\Gamma, x : \tau_1} \cap \domof{\Gamma_\text{app}}=\emptyset$ \BY{\pfref{gamma-disjoint} and \pfref{x-fresh}} \pflabel{gamma-disjoint2}
%   \item $\istypeU{\Dcons{\Delta}{\Delta_\text{app}}}{\tau_1}$ \BY{Lemma \ref{lemma:candidate-expansion-type-validation} on \pfref{cvalidT} and \pfref{delta-disjoint}} \pflabel{istype}
%   \item $\hastypeU{\Dcons{\Delta}{\Delta_\text{app}}}{\Gcons{\Gamma, \Ghyp{x}{\tau_1}}{\Gamma_\text{app}}}{e'}{\tau_2}$ \BY{IH, part 2 on \pfref{cvalidE}, \pfref{delta-disjoint} and \pfref{gamma-disjoint2}} \pflabel{hastype1}
%   \item $\hastypeU{\Dcons{\Delta}{\Delta_\text{app}}}{\Gcons{\Gamma}{\Gamma_\text{app}}, \Ghyp{x}{\tau_1}}{e'}{\tau_2}$ \BY{exchange over $\Gamma_\text{app}$ on \pfref{hastype1}} \pflabel{hastype2}
%   \item $\hastypeU{\Dcons{\Delta}{\Delta_\text{app}}}{\Gcons{\Gamma}{\Gamma_\text{app}}}{\aelam{\tau_1}{x}{e'}}{\aparr{\tau_1}{\tau_2}}$ \BY{Rule (\ref{rule:hastypeU-lam}) on \pfref{istype} and \pfref{hastype2}}
% \end{pfsteps*}
% \resetpfcounter

% \item[\text{(\ref{rule:cvalidE-U-ap})}] ~
% \begin{pfsteps*}
%   \item $\ce=\aceap{\ce_1}{\ce_2}$ \BY{assumption}
%   \item $e=\aeap{e_1}{e_2}$ \BY{assumption}
%   \item $\cvalidE{\Delta}{\Gamma}{\esceneU{\uDelta_\text{app}}{\uGamma_\text{app}}{\uPsi}{b}}{\ce_1}{e_1}{\aparr{\tau_2}{\tau}}$ \BY{assumption} \pflabel{cvalidE1}
%   \item $\cvalidE{\Delta}{\Gamma}{\esceneU{\uDelta_\text{app}}{\uGamma_\text{app}}{\uPsi}{b}}{\ce_2}{e_2}{\tau_2}$ \BY{assumption} \pflabel{cvalidE2}
% %  \item $\uetsmenv{\Delta_\text{app}}{\Psi}$ \BY{assumption} \pflabel{uetsmenv}
%   \item $\Delta \cap \Delta_\text{app}=\emptyset$ \BY{assumption} \pflabel{delta-disjoint}
%   \item $\domof{\Gamma} \cap \domof{\Gamma_\text{app}}=\emptyset$ \BY{assumption} \pflabel{gamma-disjoint}
%   \item $\hastypeU{\Dcons{\Delta}{\Delta_\text{app}}}{\Gcons{\Gamma}{\Gamma_\text{app}}}{e_1}{\aparr{\tau_2}{\tau}}$ \BY{IH, part 2 on \pfref{cvalidE1}, \pfref{delta-disjoint} and \pfref{gamma-disjoint}} \pflabel{hastypeU1}
%   \item $\hastypeU{\Dcons{\Delta}{\Delta_\text{app}}}{\Gcons{\Gamma}{\Gamma_\text{app}}}{e_2}{\tau_2}$ \BY{IH, part 2 on \pfref{cvalidE2}, \pfref{delta-disjoint} and \pfref{gamma-disjoint}} \pflabel{hastypeU2}
%   \item $\hastypeU{\Dcons{\Delta}{\Delta_\text{app}}}{\Gcons{\Gamma}{\Gamma_\text{app}}}{\aeap{e_1}{e_2}}{\tau}$ \BY{Rule (\ref{rule:hastypeU-ap}) on \pfref{hastypeU1} and \pfref{hastypeU2}}
% \end{pfsteps*}
% \resetpfcounter

% \item[\text{(\ref{rule:cvalidE-U-tlam})}] ~
% \begin{pfsteps}
%   \item \ce=\acetlam{t}{\ce'} \BY{assumption}
%   \item e = \aetlam{t}{e'} \BY{assumption}
%   \item \tau = \aall{t}{\tau'}\BY{assumption}
%   \item \cvalidE{\Delta, \Dhyp{t}}{\Gamma}{\esceneU{\uDelta_\text{app}}{\uGamma_\text{app}}{\uPsi}{b}}{\ce'}{e'}{\tau'} \BY{assumption} \pflabel{cvalidE}
% %  \item \uetsmenv{\Delta_\text{app}}{\Psi} \BY{assumption} \pflabel{uetsmenv}
%   \item \Delta \cap \Delta_\text{app}=\emptyset \BY{assumption} \pflabel{delta-disjoint}
%   \item \domof{\Gamma} \cap \domof{\Gamma_\text{app}}=\emptyset \BY{assumption} \pflabel{gamma-disjoint}
%   \item \Dhyp{t} \notin \Delta_\text{app} \BY{identification convention}\pflabel{t-fresh}
%   \item \Delta, \Dhyp{t} \cap \Delta_\text{app} = \emptyset \BY{\pfref{delta-disjoint} and \pfref{t-fresh}}\pflabel{delta-disjoint2}
%   \item \hastypeU{\Dcons{\Delta, \Dhyp{t}}{\Delta_\text{app}}}{\Gcons{\Gamma}{\Gamma_\text{app}}}{e'}{\tau'} \BY{IH, part 2 on \pfref{cvalidE}, \pfref{delta-disjoint2} and \pfref{gamma-disjoint}}\pflabel{hastype1}
%   \item \hastypeU{\Dcons{\Delta}{\Delta_\text{app}, \Dhyp{t}}}{\Gcons{\Gamma}{\Gamma_\text{app}}}{e'}{\tau'} \BY{exchange over $\Delta_\text{app}$ on \pfref{hastype1}}\pflabel{hastype2}
%   \item \hastypeU{\Dcons{\Delta}{\Delta_\text{app}}}{\Gcons{\Gamma}{\Gamma_\text{app}}}{\aetlam{t}{e'}}{\aall{t}{\tau'}} \BY{Rule (\ref{rule:hastypeU-tlam}) on \pfref{hastype2}}
% \end{pfsteps}
% \resetpfcounter

% \item[{\text{(\ref{rule:cvalidE-U-tap})}}~\textbf{through}~{\text{(\ref{rule:cvalidE-U-case})}}] These cases follow analagously, i.e. we apply the IH, part 2 to all proto-expression validation judgements, Lemma \ref{lemma:candidate-expansion-type-validation} to all proto-type validation judgements, the identification convention to ensure that extended contexts remain disjoint, weakening and exchange as needed, and the corresponding typing rule in Rules (\ref{rule:hastypeU-tap}) through (\ref{rule:hastypeU-case}).
% \\

\item[\text{(\ref{rule:cvalidE-U-splicede})}] ~
\begin{pfsteps*}
  \item $\ce=\acesplicede{m}{n}{\ctau}$ \BY{assumption}
  \item $  \escenev=\esceneU{\uDD{\uD}{\Delta_\text{app}}}{\uGG{\uG}{\Gamma_\text{app}}}{\uPsi}{b}$ \BY{assumption}
  \item   $\cvalidT{\emptyset}{\tsfrom{\escenev}}{\ctau}{\tau}$ \BY{assumption}
  \item $\parseUExp{\bsubseq{b}{m}{n}}{\ue}$ \BY{assumption}
  \item $\expandsU{\uDelta_\text{app}}{\uGamma_\text{app}}{\uPsi}{\ue}{e}{\tau}$ \BY{assumption} \pflabel{expands}
%  \item $\uetsmenv{\Delta_\text{app}}{\Psi}$ \BY{assumption} \pflabel{uetsmenv}
  \item $\Delta \cap \Delta_\text{app}=\emptyset$ \BY{assumption} \pflabel{delta-disjoint}
  \item $\domof{\Gamma} \cap \domof{\Gamma_\text{app}}=\emptyset$ \BY{assumption} \pflabel{gamma-disjoint}
  \item $\hastypeU{\Delta_\text{app}}{\Gamma_\text{app}}{e}{\tau}$ \BY{IH, part 1 on \pfref{expands}} \pflabel{hastype}
  \item $\hastypeU{\Dcons{\Delta}{\Delta_\text{app}}}{\Gcons{\Gamma}{\Gamma_\text{app}}}{e}{\tau}$ \BY{Lemma \ref{lemma:weakening-U} over $\Delta$ and $\Gamma$ and exchange on \pfref{hastype}}
\end{pfsteps*}
\resetpfcounter
\end{byCases}

The mutual induction can be shown to be well-founded by showing that the following numeric metric on the judgements that we induct over is decreasing:
\begin{align*}
\sizeof{\expandsU{\uDelta}{\uGamma}{\uPsi}{\ue}{e}{\tau}} & = \sizeof{\ue}\\
\sizeof{\cvalidE{\Delta}{\Gamma}{\esceneU{\uDelta_\text{app}}{\uGamma_\text{app}}{\uPsi}{b}}{\ce}{e}{\tau}} & = \sizeof{b}
\end{align*}
where $\sizeof{b}$ is the length of $b$ and $\sizeof{\ue}$ is the sum of the lengths of the literal bodies in $\ue$ (see Appendix \ref{appendix:SES-body-lengths}.)

The only case in the proof of part 1 that invokes part 2 is Case (\ref{rule:expandsU-tsmap}). There, we have that the metric remains stable: \begin{align*}
 & \sizeof{\expandsU{\uDelta}{\uGamma}{\uPsi}{\utsmap{\tsmv}{b}}{e}{\tau}}\\
=& \sizeof{\cvalidE{\emptyset}{\emptyset}{\esceneU{\uDelta}{\uGamma}{\uPsi}{b}}{\ce}{e}{\tau}}\\
=&\sizeof{b}\end{align*}

The only case in the proof of part 2 that invokes part 1 is Case (\ref{rule:cvalidE-U-splicede}). There, we have that $\parseUExp{\bsubseq{b}{m}{n}}{\ue}$ and the IH is applied to the judgement $\expandsU{\uDelta_\text{app}}{\uGamma_\text{app}}{\uPsi}{\ue}{e}{\tau}$ where $\uDelta_\text{app}=\uDD{\uD}{\Delta_\text{app}}$ and $\uGamma_\text{app}=\uGG{\uG}{\Gamma_\text{app}}$ and $\uPsi=\uAS{\uA}{\Psi}$. Because the metric is stable when passing from part 1 to part 2, we must have that it is strictly decreasing in the other direction:
\[\sizeof{\expandsU{\uDelta_\text{app}}{\uGamma_\text{app}}{\uPsi}{\ue}{e}{\tau}} < \sizeof{\cvalidE{\Delta}{\Gamma}{\esceneU{\uDelta_\text{app}}{\uGamma_\text{app}}{\uPsi}{b}}{\acesplicede{m}{n}{\ctau}}{e}{\tau}}\]
i.e. by the definitions above, 
\[\sizeof{\ue} < \sizeof{b}\]

This is established by appeal to the following two conditions. The first condition states that an unexpanded expression constructed by parsing a textual sequence $b$ is strictly smaller, as measured by the metric defined above, than the length of $b$, because some characters must necessarily be used to invoke a TLM and delimit each literal body.
\begingroup
\def\thetheorem{\ref{condition:body-parsing}}
\begin{condition}[Expression Parsing Monotonicity] If $\parseUExp{b}{\ue}$ then $\sizeof{\ue} < \sizeof{b}$.\end{condition}
\endgroup
The second condition simply states that subsequences of $b$ are no longer than $b$.
\begingroup
\def\thetheorem{\ref{condition:body-subsequences}}
\begin{condition}[Body Subsequencing] If $\bsubseq{b}{m}{n}=b'$ then $\sizeof{b'} \leq \sizeof{b}$. \end{condition}
\endgroup

Combining these two conditions, we have that $\sizeof{\ue} < \sizeof{b}$ as needed.
\end{proof}

% We need to define the following theorem about proto-expression validation mutually with Theorem \ref{thm:typed-expansion-U}. 
% \begin{theorem}[Proto-Expansion Expression Validation]\label{thm:candidate-expansion-validation-U}
% If $\cvalidE{\Delta}{\Gamma}{\esceneU{\Delta_\text{app}}{\Gamma_\text{app}}{\uPsi}{b}}{\ce}{e}{\tau}$ and $\uetsmenv{\Delta_\text{app}}{\uPsi}$ then $\hastypeU{\Dcons{\Delta}{\Delta_\text{app}}}{\Gcons{\Gamma}{\Gamma_\text{app}}}{e}{\tau}$.
% \end{theorem}
% \begin{proof} By rule induction over Rules (\ref{rules:cvalidE-U}).
% \begin{byCases}
% \item[\text{(\ref{rule:cvalidE-U-var})}] ~
% \begin{pfsteps*}
%   \item $\ce=x$ \BY{assumption}
%   \item $e=x$ \BY{assumption}
%   \item $\Gamma=\Gamma', \Ghyp{x}{\tau}$ \BY{assumption}
%   \item $\hastypeU{\Dcons{\Delta}{\Delta_\text{app}}}{\Gamma', \Ghyp{x}{\tau}}{x}{\tau}$ \BY{Rule (\ref{rule:hastypeU-var})} \pflabel{hastypeU}
%   \item $\hastypeU{\Dcons{\Delta}{\Delta_\text{app}}}{\Gcons{\Gamma', \Ghyp{x}{\tau}}{\Gamma_\text{app}}}{x}{\tau}$ \BY{Lemma \ref{lemma:weakening-U} over $\Gamma_\text{app}$ to \pfref{hastypeU}}
% \end{pfsteps*}
% \resetpfcounter

% \item[\text{(\ref{rule:cvalidE-U-lam})}] ~
% \begin{pfsteps*}
%   \item $\ce=\acelam{\ctau_1}{x}{\ce'}$ \BY{assumption}
%   \item $e=\aelam{\tau_1}{x}{e'}$ \BY{assumption}
%   \item $\tau=\aparr{\tau_1}{\tau_2}$ \BY{assumption}
%   \item $\cvalidT{\Delta}{\esceneU{\Delta_\text{app}}{\Gamma_\text{app}}{\uPsi}{b}}{\ctau_1}{\tau_1}$ \BY{assumption} \pflabel{cvalidT}
%   \item $\cvalidE{\Delta}{\Gamma, \Ghyp{x}{\tau_1}}{\esceneU{\Delta_\text{app}}{\Gamma_\text{app}}{\uPsi}{b}}{\ce'}{e'}{\tau_2}$ \BY{assumption} \pflabel{cvalidE}
%   \item $\uetsmenv{\Delta_\text{app}}{\uPsi}$ \BY{assumption} \pflabel{uetsmenv}
%   \item $\istypeU{\Dcons{\Delta}{\Delta_\text{app}}}{\tau_1}$ \BY{Lemma \ref{lemma:candidate-expansion-type-validation} on \pfref{cvalidT}} \pflabel{istype}
%   \item $\hastypeU{\Dcons{\Delta}{\Delta_\text{app}}}{\Gcons{\Gamma, \Ghyp{x}{\tau_1}}{\Gamma_\text{app}}}{e'}{\tau_2}$ \BY{IH on \pfref{cvalidE} and \pfref{uetsmenv}} \pflabel{hastype1}
%   \item $\hastypeU{\Dcons{\Delta}{\Delta_\text{app}}}{\Gcons{\Gamma}{\Gamma_\text{app}}, \Ghyp{x}{\tau_1}}{e'}{\tau_2}$ \BY{exchange over $\Gamma_\text{app}$ on \pfref{hastype1}} \pflabel{hastype2}
%   \item $\hastypeU{\Dcons{\Delta}{\Delta_\text{app}}}{\Gcons{\Gamma}{\Gamma_\text{app}}}{\aelam{\tau_1}{x}{e'}}{\aparr{\tau_1}{\tau_2}}$ \BY{Rule (\ref{rule:hastypeU-lam}) on \pfref{istype} and \pfref{hastype2}}
% \end{pfsteps*}
% \resetpfcounter

% \item[\text{(\ref{rule:cvalidE-U-ap})}] ~
% \begin{pfsteps*}
%   \item $\ce=\aceap{\ce_1}{\ce_2}$ \BY{assumption}
%   \item $e=\aeap{e_1}{e_2}$ \BY{assumption}
%   \item $\cvalidE{\Delta}{\Gamma}{\esceneU{\Delta_\text{app}}{\Gamma_\text{app}}{\uPsi}{b}}{\ce_1}{e_1}{\aparr{\tau_1}{\tau}}$ \BY{assumption} \pflabel{cvalidE1}
%   \item $\cvalidE{\Delta}{\Gamma}{\esceneU{\Delta_\text{app}}{\Gamma_\text{app}}{\uPsi}{b}}{\ce_2}{e_2}{\tau_1}$ \BY{assumption} \pflabel{cvalidE2}
%   \item $\uetsmenv{\Delta_\text{app}}{\uPsi}$ \BY{assumption} \pflabel{uetsmenv}
%   \item $\hastypeU{\Dcons{\Delta}{\Delta_\text{app}}}{\Gcons{\Gamma}{\Gamma_\text{app}}}{e_1}{\aparr{\tau_1}{\tau}}$ \BY{IH on \pfref{cvalidE1} and \pfref{uetsmenv}} \pflabel{hastypeU1}
%   \item $\hastypeU{\Dcons{\Delta}{\Delta_\text{app}}}{\Gcons{\Gamma}{\Gamma_\text{app}}}{e_2}{\tau_1}$ \BY{IH on \pfref{cvalidE2} and \pfref{uetsmenv}} \pflabel{hastypeU2}
%   \item $\hastypeU{\Dcons{\Delta}{\Delta_\text{app}}}{\Gcons{\Gamma}{\Gamma_\text{app}}}{\aeap{e_1}{e_2}}{\tau}$ \BY{Rule (\ref{rule:hastypeU-ap}) on \pfref{hastypeU1} and \pfref{hastypeU2}}
% \end{pfsteps*}
% \resetpfcounter

% \item[\VExpof{\text{\ref{rule:hastypeU-tlam}}}~\text{through}~\VExpof{\text{\ref{rule:hastypeU-case}}}] These cases follow analagously, i.e. we apply the IH to all proto-expression validation premises, Lemma \ref{lemma:candidate-expansion-type-validation} to all proto-types validation premises, weakening and exchange as needed, and then apply the corresponding typing rule.
% \\

% \item[\text{(\ref{rule:cvalidE-U-splicede})}] ~
% \begin{pfsteps*}
%   \item $\ce=\acesplicede{m}{n}$ \BY{assumption}
%   \item $\parseUExp{\bsubseq{b}{m}{n}}{\ue}$ \BY{assumption}
%   \item $\expandsU{\Delta_\text{app}}{\Gamma_\text{app}}{\uPsi}{\ue}{e}{\tau}$ \BY{assumption} \pflabel{expands}
%   \item $\uetsmenv{\Delta_\text{app}}{\uPsi}$ \BY{assumption} \pflabel{uetsmenv}
%   \item $\hastypeU{\Delta_\text{app}}{\Gamma_\text{app}}{e}{\tau}$ \BY{Theorem \ref{thm:typed-expansion-U} on \pfref{expands} and \pfref{uetsmenv}} \pflabel{hastype}
%   \item $\hastypeU{\Dcons{\Delta}{\Delta_\text{app}}}{\Gcons{\Gamma}{\Gamma_\text{app}}}{e}{\tau}$ \BY{Lemma \ref{lemma:weakening-U} on \pfref{hastype}}
% \end{pfsteps*}
% \resetpfcounter
% \end{byCases}
% \end{proof}


%\qed


\subsubsection{Abstract Reasoning Principles}\label{sec:uetsms-reasoning-principles}
The following theorem summarizes the abstract reasoning principles that programmers can rely on when applying an seTLM. A descripition of each named clause is given in-line below. 

\begingroup
\def\thetheorem{\ref{thm:tsc-SES}}
\begin{theorem}[seTLM Abstract Reasoning Principles]
If $\expandsU{\uDD{\uD}{\Delta}}{\uGG{\uG}{\Gamma}}{\uPsi}{\utsmap{\tsmv}{b}}{e}{\tau}$ then:
\begin{enumerate}
\item (\textbf{Typing 1}) $\uPsi = \uPsi', \uShyp{\tsmv}{a}{\tau}{\eparse}$ and $\hastypeU{\Delta}{\Gamma}{e}{\tau}$
  \begin{quote}
    The type of the expansion is consistent with the type annotation on the seTLM definition.
  \end{quote}
\item $\encodeBody{b}{\ebody}$
\item $\evalU{\ap{\eparse}{\ebody}}{\aein{\lbltxt{SuccessE}}{\ecand}}$
\item $\decodeCondE{\ecand}{\ce}$
\item (\textbf{Segmentation}) $\segOK{\segof{\ce}}{b}$
  \begin{quote}
  The segmentation determined by the proto-expansion actually segments the literal body (i.e. each segment is in-bounds and the segments are non-overlapping.)
  \end{quote}
\item $\segof{\ce} = \sseq{\acesplicedt{m'_i}{n'_i}}{\nty} \cup \sseq{\acesplicede{m_i}{n_i}{\ctau_i}}{\nexp}$
\item \textbf{(Typing 2)} $\sseq{
      \expandsTU{\uDD{\uD}{\Delta}}
      {
        \parseUTypF{\bsubseq{b}{m'_i}{n'_i}}
      }{\tau'_i}
    }{\nty}$ and $\sseq{\istypeU{\Delta}{\tau'_i}}{\nty}$
    \begin{quote}
    Each spliced type has a well-formed expansion at the application site.
    \end{quote}
\item \textbf{(Typing 3)} $\sseq{
  \cvalidT{\emptyset}{
    \tsceneUP
      {\uDD
        {\uD}{\Delta}
      }{b}
  }{
    \ctau_i
  }{\tau_i}
}{\nexp}$ and $\sseq{\istypeU{\Delta}{\tau_i}}{\nexp}$
\begin{quote}
  Each type annotation on a reference to a spliced expression has a well-formed expansion at the application site.
\end{quote}
\item \textbf{(Typing 4)} $\sseq{
  \expandsU
    {\uDD{\uD}{\Delta}}
    {\uGG{\uG}{\Gamma}}
    {\uPsi}
    {\parseUExpF{\bsubseq{b}{m_i}{n_i}}}
    {e_i}
    {\tau_i}
}{\nexp}$ and $\sseq{\hastypeU{\Delta}{\Gamma}{e_i}{\tau_i}}{\nexp}$
\begin{quote}
  Each spliced expression has a well-typed expansion consistent with its type annotation.
\end{quote}
\item (\textbf{Capture Avoidance}) $e = [\sseq{\tau'_i/t_i}{\nty}, \sseq{e_i/x_i}{\nexp}]e'$ for some $\sseq{t_i}{\nty}$ and $\sseq{x_i}{\nexp}$ and $e'$
  \begin{quote}
    The final expansion can be decomposed into a  term with variables in place of each spliced type or expression. The expansions of these spliced types and expressions can be substituted into this term in the standard capture avoiding manner.
  \end{quote}
\item (\textbf{Context Independence}) $\mathsf{fv}(e') \subset \sseq{t_i}{\nty} \cup \sseq{x_i}{\nexp}$
  \begin{quote}
    The aforementioned decomposed term makes no mention of bindings in the application site context.
  \end{quote}
  % $\hastypeU
  % {\sseq{\Dhyp{t_i}}{\nty}}
  % {\sseq{x_i : \tau_i}{\nexp}}
  % {e'}{\tau}$
\end{enumerate}
\end{theorem}
\begin{proof} The proof, which involves auxiliary lemmas about the decomposition of proto-types and proto-expressions, is given in Appendix \ref{appendix:SES-reasoning-principles}.
\end{proof}
\endgroup

This style of specifying the hygiene properties builds directly on the standard notion of capture-avoiding substitution for general ABTs. Prior work on hygiene for macro systems has instead explicitly specified how fresh variables are generated during expansion (e.g. \cite{DBLP:conf/esop/HermanW08}.) Our formal approach appears therefore to be more elegant in this regard.
% The following theorem establishes that every valid proto-type generates a final expansion that can be decomposed into a context-independent term and context-dependent sub-terms, all of which arise from references to spliced types as summarized by the splice summary.

% \begingroup
% \def\thetheorem{\ref{thm:proto-type-expansion-decomposition-SES}}
% \begin{theorem}[Proto-Type Expansion Decomposition] 
% If $\cvalidT{\Delta}{\tsceneU{\uDD{\uD}{\Delta_\text{app}}}{b}}{\ctau}{\tau}$ and $\segof{\ctau} = \sseq{\acesplicedt{m_i}{n_i}}{n}$ then all of the following hold:
% \begin{enumerate}
% \item $\sseq{\expandsTU{\uDD{\uD}{\Delta_\text{app}}}{
%   \parseUTypF{\bsubseq{b}{m_i}{n_i}}
% }{\tau_i}}{n}$
% % \item $\sseq{\istypeU{\Delta_\text{app}}{\tau_i}}{n}$
% \item $\tau = [\sseq{\tau_i/t_i}{n}]\tau'$ for some $\sseq{t_i}{n}$ and $\tau'$
% \item $\istypeU{\Delta \cup \sseq{\Dhyp{t_i}}{n}}{\tau'}$
% \end{enumerate}
% \end{theorem}
% \begin{proof}
% By rule induction over Rules (\ref{rules:cvalidT-U}).
% \begin{byCases}
%   \item[\text{(\ref{rule:cvalidT-U-tvar}) \textbf{through} (\ref{rule:cvalidT-U-sum})}] These cases follow by straightforward inductive argument (see appendix.)
%   \item[\text{(\ref{rule:cvalidT-U-splicedt})}] ~
%   \begin{pfsteps}
%   \item \ctau = \acesplicedt{m}{n} \BY{assumption}
%   \item \segof{\acesplicedt{m}{n}} = \{ \acesplicedt{m}{n} \} \BY{definition}
%   \item \parseUTyp{\bsubseq{b}{m}{n}}{\utau} \BY{assumption} \pflabel{parseUTyp}
%   \item \expandsTU{\uDD{\uD}{\Delta_\text{app}}}{\utau}{\tau} \BY{assumption} \pflabel{expandsTU}
%   \item \istypeU{\Delta, \Dhyp{t}}{t} \BY{Rule (\ref{rule:istypeU-var})} \pflabel{istype}
%   \end{pfsteps}
%   The conclusions hold as follows:
%   \begin{enumerate}
%     \item \pfref{parseUTyp} and \pfref{expandsTU}
%     \item Choose $t$ and $t$. Then $\tau = [\tau/t]t$ by definition.
%     \item \pfref{istype}
%   \end{enumerate}
% \end{byCases}
% \end{proof}
% \endgroup

% The following theorem, together with Theorem \ref{thm:typed-expansion-short-U}, establishes \textbf{Typing}, \textbf{Segmentation} and \textbf{Context Independence} as discussed in Sec. \ref{sec:uetsms-validation}.

% \begingroup
% \def\thetheorem{\ref{thm:tsc-SES}}
% \begin{theorem}[seTLM Typing and Context Independence]
% If $\expandsU{\uDelta}{\uGamma}{\uPsi}{\utsmap{\tsmv}{b}}{e}{\tau}$ then:
% \begin{enumerate}
% \item (\textbf{Typing}) $\uPsi = \uPsi', \uShyp{\tsmv}{a}{\tau}{\eparse}$
% \item $\encodeBody{b}{\ebody}$
% \item $\evalU{\ap{\eparse}{\ebody}}{\lbltxt{SuccessE}\cdot\ecand}$
% \item $\decodeCondE{\ecand}{\ce}$
% \item (\textbf{Segmentation}) $\segOK{\segof{\ce}}{b}$
% \item (\textbf{Context Independence}) $\cvalidE{\emptyset}{\emptyset}{\esceneU{\uDelta}{\uGamma}{\uPsi}{b}}{\ce}{e}{\tau}$ 
% \end{enumerate}
% \end{theorem}
% \begin{proof} By rule induction over Rules (\ref{rules:expandsU}). The only rule that applies is Rule (\ref{rule:expandsU-tsmap}). The conclusions of the theorem are the premises of this rule.
% \end{proof}
% \endgroup

% The following theorem establishes a prohibition on \textbf{Shadowing} as discussed in Sec. \ref{sec:uetsms-validation}.
% \begingroup
% \def\thetheorem{\ref{thm:shadowing-prohibition-SES}}
% \begin{theorem}[Shadowing Prohibition] ~
% \begin{enumerate}
% \item If $\cvalidT{\Delta}{\tsceneU{\uDD{\uD}{\Delta_\text{app}}}{b}}{\acesplicedt{m}{n}}{\tau}$ then:\begin{enumerate}
% \item $\parseUTyp{\bsubseq{b}{m}{n}}{\utau}$
% \item $\expandsTU{\uDD{\uD}{\Delta_\text{app}}}{\utau}{\tau}$
% \item $\Delta \cap \Delta_\text{app} = \emptyset$
% \end{enumerate}
% \item If $\cvalidE{\Delta}{\Gamma}{\escenev}{\acesplicede{m}{n}{\ctau}}{e}{\tau}$ then:
% \begin{enumerate}
% \item $\cvalidT{\emptyset}{\tsfrom{\escenev}}{\ctau}{\tau}$
% \item $  \escenev=\esceneU{\uDD{\uD}{\Delta_\text{app}}}{\uGG{\uG}{\Gamma_\text{app}}}{\uPsi}{b}$
% \item $\parseUExp{\bsubseq{b}{m}{n}}{\ue}$
% \item $\expandsU{\uDD{\uD}{\Delta_\text{app}}}{\uGG{\uG}{\Gamma_\text{app}}}{\uPsi}{\ue}{e}{\tau}$
% \item $\Delta \cap \Delta_\text{app} = \emptyset$
% \item $\domof{\Gamma} \cap \domof{\Gamma_\text{app}} = \emptyset$
% \end{enumerate}
% \end{enumerate}
% \end{theorem}
% \begin{proof} ~
% \begin{enumerate}
% \item By rule induction over Rules (\ref{rules:cvalidT-U}). The only rule that applies is Rule (\ref{rule:cvalidT-U-splicedt}). The conclusions are the premises of tihs rule.
% \item By rule induction over Rules (\ref{rules:cvalidE-U}). The only rule that applies is Rule (\ref{rule:cvalidE-U-splicede}). The conclusions are the premises of tihs rule.
% \end{enumerate}
% \end{proof}
% \endgroup

% !TEX root = omar-thesis.tex
\chapter{Simple Pattern TLMs (spTLMs)}\label{chap:uptsms}
In Chapter \ref{chap:uetsms}, our interest was in situations where the programmer needed to \emph{construct} (a.k.a. \emph{introduce}) a value. In this chapter, we consider situations where the programmer needs to \emph{deconstruct} (a.k.a. \emph{eliminate}) a value by pattern matching. For example, consider again the recursive labeled sum type \lstinline{rx} defined in Figure \ref{fig:datatype-rx}. We can pattern match over a value \lstinline{r} of type \lstinline{rx} using VerseML's \lstinline{match} construct as shown in the example below: 
\begin{lstlisting}
fun is_seq(r : rx) => 
  match r with 
    Seq(Str(name), Seq(Str "SSTR: ESTR", ssn)) => Some (name, ssn)
  | _ => None
  end
\end{lstlisting}
% \begin{lstlisting}
% fun is_dna_rx(r : rx) : boolean => 
%   match r with 
%   | Str "SSTRAESTR" => True
%   | Str "SSTRTESTR" => True
%   | Str "SSTRGESTR" => True
%   | Str "SSTRCESTR" => True
%   | Seq (r1, r2) => (is_dna_rx r1) andalso (is_dna_rx r2)
%   | Or  (r1, r2) => (is_dna_rx r1) andalso (is_dna_rx r2)
%   | Star(r') => is_dna_rx r'
%   | _ => False 
%   end
% \end{lstlisting}

Match expressions consist of a \emph{scrutinee}, here \li{r}, and a sequence of \emph{rules} separated by vertical bars, \li{|}, in the textual syntax. Each rule consists of a \emph{pattern} and an {expression} called the corresponding \emph{branch}, separated by a double arrow, \li{=>}, in the textual syntax. During evaluation, the value of the scrutinee is matched against each pattern sequentially. If a match occurs, evaluation proceeds along the corresponding branch. 

A variable can  appear at most once in a valid pattern. In the corresponding branch, the variable stands for the value it matched.  
For example, on Line 3 above, the pattern 
\begin{lstlisting}[numbers=none]
Seq(Str(name), Seq(Str "SSTR: ESTR", ssn))
\end{lstlisting}
matches values with the following structure: 
\begin{lstlisting}[numbers=none]
Seq(Str(#$e_1$#), Seq(Str "SSTR: ESTR", #$e_2$#))
\end{lstlisting}
where $e_1$ is a value of type \li{string} and $e_2$ is a value of type \li{rx}. The variables \li{name} and \li{ssn} stand for the values of $e_1$ and $e_2$, respectively, in the corresponding branch expression. 

On Line 4 above, the pattern \li{_} is the \emph{wildcard pattern} -- it matches any value, like the variable pattern, but binds no variables.

The behavior of the \li{match} construct when no pattern in the rule sequence matches a value is to raise an exception indicating \emph{match failure}. It is possible to statically determine whether match failure is possible (i.e. whether there exist values of the scrutinee that do not match any pattern in the rule sequence.) A rule sequence that cannot lead to match failure is said to be \emph{exhaustive}. Compilers warn the programmer when a rule sequence is non-exhaustive. In the example above, our use of the wildcard pattern ensures that match failure cannot occur. 

It is also possible to statically decide when a rule is \emph{redundant} relative to the preceding rules. For example, if we add  another rule at the end of the match expression above, it will be redundant because all values match the wildcard pattern. Again, compilers warn the programmer when a rule is redundant.

Nested pattern matching generalizes the projection and case analysis operators (i.e. the \emph{eliminators}) for products and sums (cf. $\miniVerseUE$ from the previous section.) 

In Sec. \ref{sec:syntax-examples-regexps}, we considered a hypothetical dialect of VerseML called $\mathcal{V}_\texttt{rx}$ with derived regex pattern forms. In this dialect, we can express the example above at lower syntactic cost using standard POSIX regex syntax extended with pattern splicing forms:

\begin{lstlisting}
fun f(r : rx) => 
  match r with 
    /SURL@EURLnameSURL: %EURLssn/ => Some (name, ssn)
  | _ => None
  end
\end{lstlisting}
\noindent
This dialect-oriented approach has problems, as  discussed in Chapter \ref{sec:problems-with-syntax-dialects}.

% seek language constructs that allow us to decrease the syntactic cost of expressing complex patterns to a similar degree.

Expression TLMs -- introduced in Chapter \ref{chap:uetsms} -- can decrease the syntactic cost of constructing a value, but expressions are syntactically and semantically distinct from patterns, so we cannot simply apply an expression TLM within a pattern.\footnote{The fact that certain concrete expression and pattern forms coincidentally overlap is immaterial to this fundamental distinction.} %For example, the expansion generated by an expression TLM might define or apply a function, but patterns do not contain functions or function applications. 
We need a new (albeit closely related) construct -- the \textbf{pattern TLM}. In this chapter, we consider only \textbf{simple pattern TLMs} (spTLMs), i.e. pattern TLMs that generate patterns that match values of a single specified type, like \li{rx}. In Chapter \ref{chap:ptsms}, we will consider both expression and pattern TLMs that specify type and module parameters (peTLMs and ppTLMs). 

The organization of the remainder of this chapter mirrors that of Chapter \ref{chap:uetsms}. We begin in Sec. \ref{sec:ptsms-by-example} with a ``tutorial-style'' introduction to spTLMs in VerseML. 
%In particular, we  discuss an spTLM for patterns matching values of type \li{rx}. 
Then, in Sec. \ref{sec:miniVerseUP}, we define an extension of $\miniVerseUE$ called $\miniVersePat$ that makes the intuitions developed in Sec. \ref{sec:ptsms-by-example} mathematically precise.

\section{Simple Pattern TLMs By Example}\label{sec:ptsms-by-example}

\subsection{Usage}\label{sec:ptsms-usage}
The VerseML function \li{f} defined at the beginning of this chapter can be expressed at lower syntactic cost by applying an spTLM named \li{#\dolla#rx} as follows:
\begin{lstlisting}
fun f(r : rx) => 
  match r with 
    $rx /SURL@EURLnameSURL: %EURLssn/ => Some (name, ssn)
  | _ => None
  end
\end{lstlisting}
Like expression TLMs, pattern TLMs are applied to \emph{generalized literal forms} (see Figure \ref{fig:literal-forms}.) During the  \emph{typed expansion} phase, the applied pattern TLM parses the body of the literal form to generate a \emph{proto-expansion}. The language validates the proto-expansion according to criteria that we will establish in Sec. \ref{sec:ptsms-validation}. If validation succeeds, the language generates the final expansion (or more concisely, simply the expansion) of the pattern. The expansion of the unexpanded pattern \li{#\dolla#rx /SURL@EURLnameSURL: %EURLssn/} 
from the example above is the following pattern:
\begin{lstlisting}[numbers=none]
Seq(Str(name), Seq(Str "SSTR: ESTR", ssn))
\end{lstlisting}

The checks for exhaustiveness and redundancy are performed post-expansion.

For convenience, the programmer can specify a TLM at the outset of a sequence of rules that is applied to every outermost generalized literal form. For example, the function \li{is_dna_rx} from Figure \ref{fig:is_dna_rx} and Figure \ref{fig:derived-pattern-syntax} can be expressed using the spTLM \li{#\dolla#rx} as follows:
\begin{lstlisting}[morekeywords={using}]
fun is_dna_rx(r : rx) : boolean => 
  match r using $rx with 
  | /SURLAEURL/ => True
  | /SURLTEURL/ => True
  | /SURLGEURL/ => True
  | /SURLCEURL/ => True
  | /SURL%(EURLr1SURL)%(EURLr2SURL)EURL/ => (is_dna_rx r1) andalso (is_dna_rx r2)
  | /SURL%(EURLr1SURL)|%(EURLr2SURL)EURL/ => (is_dna_rx r1) andalso (is_dna_rx r2)
  | /SURL%(EURLrSURL)*EURL/ => is_dna_rx r'
  | _ => False
  end
\end{lstlisting}

\subsection{Definition}\label{sec:ptsms-definition}
The definition of the pattern TLM \li{#\dolla#rx} shown being applied in the examples above takes the following form:
\begin{lstlisting}[numbers=none]
syntax $rx at rx for patterns by 
  static fn(b : body) : parse_result(proto_pat) =>
    (* regex pattern parser here *)
end 
\end{lstlisting}
This definition first names the pattern TLM \li{#\dolla#rx}. Pattern TLM names, like expression TLM names, must begin with the dollar sign (\li{#\dolla#}) to distinguish them from labels. Pattern TLM names and expression TLM names are tracked separately, i.e. an expression TLM and a pattern TLM can have the same name without conflict (as is the case here -- the expression TLM that was described in Sec. \ref{sec:uetsms-definition} is also named \li{#\dolla#rx}.) 

The \emph{sort qualifier} \li{for patterns} indicates that this is a pattern TLM definition, rather than an expression TLM definition (the sort qualifier \li{for expressions} can be written for expression TLMs, though when the sort qualifier is omitted this is the default.) Defining both an expression TLM and a pattern TLM with the same name at the same type is a common idiom, so VerseML defines a derived form for combining their definitions:
\begin{lstlisting}[numbers=none,morekeywords={andfor}]
syntax $rx at rx for expressions by
  static fn(body : body) : parse_result(proto_expr) => 
    (* regex expression parser here *)
for patterns by 
  static fn(body : body) : parse_result(proto_pat) => 
    (* regex pattern parser here *)
end
\end{lstlisting}

Pattern TLMs, like expression TLMs, must specify a static {parse function}. For pattern TLMs, the parse function must be of type \li{body -> parse_result(proto_pat)}, where \li{body} and \li{parse_result} are defined as in Figure \ref{fig:indexrange-and-parseresult}. 

The type \li{proto_pat}, defined in Figure \ref{fig:CEPat}, is analagous to the types \li{proto_expr} and \li{proto_typ} defined in Figure \ref{fig:candidate-exp-verseml}. This type classifies \emph{encodings of proto-patterns}. Every pattern form has a corresponding proto-pattern form, with the exception of variable patterns (for reasons explained in Sec. \ref{sec:ptsms-hygiene} below.) There is also an additional constructor, \li{SplicedP}, to allow a proto-pattern to refer indirectly to spliced patterns by their location within the literal body.

\begin{figure}
\begin{lstlisting}[numbers=none]
type proto_pat = rec(proto_pat => 
                 (* no variable pattern form *) 
                 Wild
               + (* ... *)
               + SplicedP of segment * proto_typ)
\end{lstlisting}
\caption[Abbreviated definition of \li{proto_pat} in VerseML]{Abbreviated definition of \li{proto_pat} in the VerseML prelude.}
\label{fig:CEPat}
\end{figure}

\subsection{Splicing}\label{sec:ptsms-splicing}
Spliced patterns are unexpanded patterns that appear directly within the literal body of another unexpanded pattern. For example, \li{name} and \li{ssn} appear within the unexpanded pattern \li{#\dolla#rx /SURL@EURLnameSURL: %EURLssn/}. 
When the parse function determines that a subsequence of the literal body should be taken as a spliced pattern (here, by recognizing the characters \li{@} or \li{%} followed by a variable or parenthesized pattern), 
it can refer to it within the proto-expansion that it computes using the \li{SplicedP} variant of the \li{proto_pat} type shown in Figure \ref{fig:CEPat}. This variant takes a value of type \li{segment} because proto-patterns refer to spliced patterns indirectly by their position within the literal body. This prevents pattern TLMs from ``forging'' a spliced pattern (i.e. claiming that some pattern is a spliced pattern, even though it does not appear in the literal body.) 
Like references to spliced expressions, each reference to a spliced pattern must also specify a type.

The proto-expansion generated by the pattern TLM \li{#\dolla#rx} for the example above, if written in a hypothetical concrete syntax where references to spliced patterns are written \li{spliced<startIdx; endIdx; ty>}, is:
\begin{lstlisting}[numbers=none]
Seq(Str(spliced<1; 4; string>), 
    Seq(Str "SSTR: ESTR", spliced<8; 10; rx>))
\end{lstlisting}
Here, \li{spliced<1; 4; string>} refers to the string subpattern \li{name} by location, and similarly, \li{spliced<8; 10; rx>} refers to the regex subpattern \li{ssn} by location.

\subsection{Segmentations}
The \emph{segmentation} of a proto-pattern is the finite set of references to spliced types or patterns. As with references to spliced expressions, the language checks that the references to spliced terms in a proto-expansion 1) are within bounds of the literal body; 2) are non-overlapping; and 3) operate at a consistent sort and type.

% An editor or pretty-printer can convey the summ information using color, as shown in the examples above.

\subsection{Proto-Expansion Validation}\label{sec:ptsms-validation}
After the pattern TLM generates a proto-expansion, the language must validate it to generate a final expansion. This also serves to maintain a reasonable type and binding discipline.

\subsubsection{Typing}
To maintain a reasonable type discipline, proto-expansion validation  checks:
\begin{enumerate}
\item that each spliced pattern matches values of the type indicated in the summary; and
\item that the final expansion matches values of the type specified in the type annotation on the pattern TLM definition, e.g. the type \li{rx} above.
\end{enumerate}

\subsubsection{Hidden Bindings}\label{sec:ptsms-hygiene}
%In order to check that the candidate expansion is well-typed, the language must parse, type and expand the spliced subpatterns that the candidate expansion refers to (by their position within the literal body, cf. above). 
To maintain a useful binding discipline, i.e. to allow programmers to reason about variable binding without examining TLM expansions directly, the validation process allows variable patterns to occur only in spliced patterns (just as variables bound at the use site can only appear in spliced expressions when using an expression TLM.) Indeed, there is no constructor for the type \li{proto_pat} corresponding to a variable pattern. This prohibition on ``hidden bindings'' is beneficial because the client can rely on the fact that no variables other than those that appear directly within the pattern at the application site are bound in the corresponding branch expression. This prohibition on hidden bindings is analagous to the prohibition on capture discussed in Sec. \ref{sec:uetsms-validation} (differing in that it is concerned with the bindings visible to the corresponding branch expression, rather than to spliced expressions.)

% \subsubsection{Context Independence}
% In VerseML, patterns are context-independent by construction (i.e. there is no way to refer to the surrounding bindings from within a pattern). It is only in the type annotations on spliced patterns that we need to enforce context independence.  (In languages that support, e.g., arbitrary expressions as \emph{guards} within patterns (e.g. OCaml \cite{ocaml-manual}), or in languages that support pattern synonyms, it would be necessary also to enforce context independence for these constructs as well.) 

\subsection{Final Expansion}\label{sec:ptsms-final-expansion}
If validation succeeds, the semantics generates the \emph{final expansion} of the pattern where the references to spliced patterns in the proto-pattern have been replaced by their respective final expansions. For example, the final expansion of \li{#\dolla#rx /SURL@EURLnameSURL: %EURLssn/} is:
\begin{lstlisting}[numbers=none]
Seq(Str(name), Seq(Str "SSTR: ESTR", ssn))
\end{lstlisting}

\section{\texorpdfstring{$\miniVersePat$}{miniVerseU}}\label{sec:miniVerseUP}
To make the intuitions developed in the previous section about pattern TLMs precise, we  now introduce $\miniVersePat$, a reduced dialect of VerseML with support for both seTLMs and spTLMs. Like $\miniVerseUE$, $\miniVersePat$ consists of an \emph{unexpanded language} (UL) defined by typed expansion to a standard \emph{expanded language} (XL). The full definition of $\miniVersePat$ is given in Appendix \ref{appendix:miniVerseSES} superimposed upon the definition of $\miniVerseUE$. We will focus on the rules specifically related to pattern matching and spTLMs below. 

Our formulation of pattern matching is adapted from  Harper's formulation in \emph{Practical Foundations for Programming Languages, First Edition} \cite{pfple1}.

\subsection{Syntax of the Expanded Language}\label{sec:UP-expanded-terms}\label{sec:inner-core-syntax-UP}
Figure \ref{sec:UP-expanded-terms} defines the syntax of the $\miniVersePat$ \emph{expanded language} (XL), which consists of \emph{types}, $\tau$, \emph{expanded expressions}, $e$, \emph{expanded rules}, $r$, and \emph{expanded patterns}, $p$. The $\miniVersePat$ XL differs from the $\miniVerseUE$ XL only by the addition of the pattern matching operator and related forms.\footnote{The projection and case analysis operators can be defined in terms of the match operator, but to simplify the appendix, we leave them in place.} %\footnote{The chapter on pattern matching has, of this writing, been removed from the draft second edition of \emph{PFPL}, but a copy of the first edition can be found online.}


The main syntactic feature of note is that the rule form places a pattern, $p$, in the binder position:
\[
\aematchrule{p}{e}
\]
This can be understood as binding all of the variables in $p$ for use within $e$. A small technical note: the ABT \emph{renaming} meta-operation (which underlies the notion of alpha-equivalence) requires that these variables appear as a sequence. Rather than redefining this metaoperation explicitly, we implicitly determine such a sequence by performing a depth-first traversal, with traversal of the labeled tuple pattern form, $\aetplp{\labelset}{\mapschema{p}{i}{\labelset}}$, relying on some (arbitrary) total ordering on labels.

\begin{figure}
\[\begin{array}{lllllll}
\textbf{Sort} & & 
& \textbf{Operational Form} 
% & \textbf{Stylized Form} 
& \textbf{Description}\\
\mathsf{Typ} & \tau & ::= 
& \cdots
% & t 
& \text{(see Figure \ref{fig:U-expanded-terms})}\\
% &&& \aall{t}{\tau} & \forallt{t}{\tau} & \text{polymorphic}\\
% &&& \arec{t}{\tau} & \rect{t}{\t au} & \text{recursive}\\
% &&& \aprod{\labelset}{\mapschema{\tau}{i}{\labelset}} & \prodt{\mapschema{\tau}{i}{\labelset}} & \text{labeled product}\\
% &&& \asum{\labelset}{\mapschema{\tau}{i}{\labelset}} & \sumt{\mapschema{\tau}{i}{\labelset}} & \text{labeled sum}\\
\mathsf{Exp} & e & ::= 
& \cdots 
% & x 
& \text{(see Figure \ref{fig:U-expanded-terms})}\\
% &&& \aelam{\tau}{x}{e} & \lam{x}{\tau}{e} & \text{abstraction}\\
% &&& \aeap{e}{e} & \ap{e}{e} & \text{application}\\
% &&& \aetlam{t}{e} & \Lam{t}{e} & \text{type abstraction}\\
% &&& \aetap{e}{\tau} & \App{e}{\tau} & \text{type application}\\
% &&& \aefold{e} & \fold{e} & \text{fold}\\
% &&& \aeunfold{e} & \unfold{e} & \text{unfold}\\
% &&& \aetpl{\labelset}{\mapschema{e}{i}{\labelset}} & \tpl{\mapschema{e}{i}{\labelset}} & \text{labeled tuple}\\
% &&& \aepr{\ell}{e} & \prj{e}{\ell} & \text{projection}\\
% &&& \aein{\ell}{e} & \inj{\ell}{e} & \text{injection}\\
% \LCC \lightgray & \lightgray & \lightgray 
%& \lightgray 
% & \lightgray & \lightgray \\
&&
& \aematchwith{n}{e}{\seqschemaX{r}}
% & \matchwith{e}{\seqschemaX{r}} 
& \text{match}\\
\mathsf{Rule} & r & ::= 
& \aematchrule{p}{e} 
%& \matchrule{p}{e} 
& \text{rule}\\
\mathsf{Pat} & p & ::= 
& x  
%& x 
& \text{variable pattern}\\
&&& \aewildp 
%& \wildp 
& \text{wildcard pattern}\\
&&& \aefoldp{p} 
%& \foldp{p} 
& \text{fold pattern}\\
&&& \aetplp{\labelset}{\mapschema{p}{i}{\labelset}} 
%& \tplp{\mapschema{p}{i}{\labelset}} 
& \text{labeled tuple pattern}\\
&&& \aeinjp{\ell}{p} 
%& \injp{\ell}{p} 
& \text{injection pattern} %\ECC
\end{array}\]
\caption{Syntax of the $\miniVersePat$ expanded language (XL)}
\label{fig:UP-expanded-terms}
\end{figure}


\subsection{Statics of the Expanded Language}\label{sec:inner-core-statics-UP}
The \emph{statics of the XL} is defined by judgements of the following form:
\[\begin{array}{ll}
\textbf{Judgement Form} & \textbf{Description}\\
\istypeU{\Delta}{\tau} & \text{$\tau$ is a well-formed type}\\
%\isctxU{\Delta}{\Gamma} & \text{$\Gamma$ is a well-formed typing context assuming $\Delta$}\\
\hastypeU{\Delta}{\Gamma}{e}{\tau} & \text{$e$ is assigned type $\tau$}\\
\ruleType{\Delta}{\Gamma}{r}{\tau}{\tau'} & \text{$r$ takes values of type $\tau$ to values of type $\tau'$}\\
\patType{\pctx}{p}{\tau} & \text{$p$ matches values of type $\tau$ and generates hypotheses $\pctx$} 
\end{array}\]

The types of $\miniVersePat$ are exactly those of $\miniVerseUE$, described in Sec. \ref{sec:miniVerseU}, so the \emph{type formation judgement}, $\istypeU{\Delta}{\tau}$, is inductively defined by Rules (\ref{rules:istypeU}) as before.

The \emph{typing judgement}, $\hastypeU{\Delta}{\Gamma}{e}{\tau}$, assigns types to expressions and is inductively defined by Rules (\ref{rules:hastypeU}), which consist of:
% \begin{subequations}\label{rules:hastypeUP}
% \refstepcounter{equation}%
\begin{itemize}
% \label{rule:hastypeUP-var}
% \refstepcounter{equation}\label{rule:hastypeUP-lam}
% \refstepcounter{equation}\label{rule:hastypeUP-ap}
% \refstepcounter{equation}\label{rule:hastypeUP-tlam}
% \refstepcounter{equation}\label{rule:hastypeUP-tap}
% \refstepcounter{equation}\label{rule:hastypeUP-fold}
% \refstepcounter{equation}\label{rule:hastypeUP-unfold}
% \refstepcounter{equation}\label{rule:hastypeUP-tpl}
% \refstepcounter{equation}\label{rule:hastypeUP-pr}
% \refstepcounter{equation}\label{rule:hastypeUP-in}
\item The typing rules of $\miniVerseUE$, i.e. Rules (\ref{rule:hastypeU-var}) through (\ref{rule:hastypeU-case}). %Note that we cannot defer directly to the typing rules from Sec. \ref{sec:miniVerseU} because $e$ has been redefined here.
\item The following rule for match expressions: 
\end{itemize}
\begin{equation*}\tag{\ref{rule:hastypeUP-match}}
\inferrule{
  \hastypeU{\Delta}{\Gamma}{e}{\tau}\\
  % \istypeU{\Delta}{\tau'}\\
  \{\ruleType{\Delta}{\Gamma}{r_i}{\tau}{\tau'}\}_{1 \leq i \leq n}\\
}{\hastypeU{\Delta}{\Gamma}{\aematchwith{n}{e}{\seqschemaX{r}}}{\tau'}}
\end{equation*}  
% \end{subequations}
The first premise of Rule (\ref{rule:hastypeUP-match}) assigns a type, $\tau$, to the scrutinee, $e$. The second premise then ensures that each rule $r_i$, for $1 \leq i \leq n$, takes values of type $\tau$ to values of the type of the match expression as a whole, $\tau'$, according to the \emph{rule typing judgement}, $\ruleType{\Delta}{\Gamma}{r}{\tau}{\tau'}$, which is defined mutually with Rules (\ref{rules:hastypeUP}) by the following rule:
\begin{equation*}\tag{\ref{rule:ruleType}}
\inferrule{
  \patType{\pctx'}{p}{\tau}\\
  \hastypeU{\Delta}{\Gcons{\Gamma}{\pctx'}}{e}{\tau'}
}{\ruleType{\Delta}{\Gamma}{\aematchrule{p}{e}}{\tau}{\tau'}}
\end{equation*}
The first premise invokes the \emph{pattern typing judgement}, $\patType{\pctx'}{p}{\tau}$, to check that the pattern, $p$, matches values of type $\tau$ (defined assuming $\Delta$), and to gather the typing hypotheses that the pattern generates in a {typing context}, $\pctx'$. (Algorithmically, the typing context is the ``output'' of the pattern typing judgement.) 
The second premise of Rule (\ref{rule:ruleType}) extends the incoming typing context, $\Gamma$, with the hypotheses generated by pattern typing, $\pctx$, and checks the branch expression, $e$, against the branch type, $\tau'$.%Pattern typing contexts are typing contexts. Algorithmically, however, one should consider the pattern typing context the ``output'' of the pattern typing judgement. 

The pattern typing judgement is inductively defined by Rules (\ref{rules:patType}).
Rule (\ref{rule:patType-var}) specifies that a variable pattern, $x$, matches values of any type, $\tau$, and generates the hypothesis that $x$ has type $\tau$:
\begin{equation*}\tag{\ref{rule:patType-var}}
\inferrule{ }{\patType{\Ghyp{x}{\tau}}{x}{\tau}}
\end{equation*}

Rule (\ref{rule:patType-wild}) specifies that a wildcard pattern also matches values of any type, $\tau$, but wildcard patterns generate no hypotheses:
\begin{equation*}\tag{\ref{rule:patType-wild}}
\inferrule{ }{\patType{\emptyset}{\aewildp}{\tau}}
\end{equation*}

Rule (\ref{rule:patType-fold}) specifies that a fold pattern, $\aefoldp{p}$, matches values of the recursive type $\arec{t}{\tau}$ if $p$ matches values of a single unrolling of the recursive type, $[\arec{t}{\tau}/t]\tau$:
\begin{equation*}\tag{\ref{rule:patType-fold}}
\inferrule{
  \patType{\pctx}{p}{[\arec{t}{\tau}/t]\tau}
}{
  \patType{\pctx}{\aefoldp{p}}{\arec{t}{\tau}}
}
\end{equation*}

Rule (\ref{rule:patType-tpl}) specifies that a labeled tuple pattern matches values of the labeled product type $\aprod{\labelset}{\mapschema{\tau}{i}{\labelset}}$. Labeled tuple patterns, $\aetplp{\labelset}{\mapschema{p}{i}{\labelset}}$, specify a subpattern $p_i$ for each label $i \in \labelset$. The premise checks each subpattern $p_i$ against the corresponding type $\tau_i$, generating hypotheses $\pctx_i$. The conclusion of the rule gathers these hypotheses into a single pattern typing context, $\Gconsi{i \in \labelset}{\pctx_i}$:
\begin{equation*}\tag{\ref{rule:patType-tpl}}
\inferrule{
  \{\patType{\pctx_i}{p_i}{\tau_i}\}_{i \in \labelset}
}{
  \patType{\Gconsi{i \in \labelset}{\pctx_i}}{\aetplp{\labelset}{\mapschema{p}{i}{\labelset}}}{\aprod{\labelset}{\mapschema{\tau}{i}{\labelset}}}
}
\end{equation*}
The definition of typing context extension, applied iteratively here,  implicitly requires that the pattern typing contexts $\pctx_i$ be mutually disjoint, i.e. \[\{\{\domof{\pctx_i} \cap \domof{\pctx_j} = \emptyset\}_{j \in \labelset \setminus i}\}_{i \in \labelset}\]

Finally, Rule (\ref{rule:patType-inj}) specifies that an injection pattern,  $\aeinjp{\ell}{p}$, matches values of labeled sum types of the form $\asum{\labelset, \ell}{\mapschema{\tau}{i}{\labelset}; \mapitem{\ell}{\tau}}$, i.e. labeled sum types that define a case for the label $\ell$. The pattern $p$ must match value of type $\tau$ and generate hypotheses $\pctx$:
\begin{equation*}\tag{\ref{rule:patType-inj}}
\inferrule{
  \patType{\pctx}{p}{\tau}
}{
  \patType{\pctx}{\aeinjp{\ell}{p}}{\asum{\labelset, \ell}{\mapschema{\tau}{i}{\labelset}; \mapitem{\ell}{\tau}}}
}
\end{equation*}


%These judgements obey standard lemmas, defined in Appendix \ref{appendix:SES-XL}: Weakening, Substitution, Decomposition and Regularity.





%\item The second premise of Rule (\ref{rule:ruleType}) ensures that pattern typing of $p$ has generated hypotheses for all of the variables that the branch expression, $e$, binds. This is merely a matter of ``metatheoretic bookkeeping''. In the stylized form for rules, $\matchrule{p}{e}$, the variables bound in $e$ are, implicitly, exactly those mentioned in p.% The bindings for $e$ would be extracted from the pattern implicitly. 

\subsection{Structural Dynamics}\label{sec:dynamics-UP}
The \emph{structural dynamics of }$\miniVersePat$ is defined as a transition system, and is organized around judgements of the following form:
\[\begin{array}{ll}
\textbf{Judgement Form} & \textbf{Description}\\
\stepsU{e}{e'} & \text{$e$ transitions to $e'$}\\
\isvalU{e} & \text{$e$ is a value}\\
\matchfail{e} & \text{$e$ raises match failure}
\end{array}\]
We also define auxiliary judgements for \emph{iterated transition}, $\multistepU{e}{e'}$, and \emph{evaluation}, $\evalU{e}{e'}$.

\begingroup
\def\thetheorem{\ref{defn:iterated-transition-UP}}
\begin{definition}[Iterated Transition] Iterated transition, $\multistepU{e}{e'}$, is the reflexive, transitive closure of the transition judgement, $\stepsU{e}{e'}$.\end{definition}
% \addtocounter{theorem}{-1}
\endgroup

\begingroup
\def\thetheorem{\ref{defn:evaluation-UP}}
\begin{definition}[Evaluation] $\evalU{e}{e'}$ iff $\multistepU{e}{e'}$ and $\isvalU{e'}$. \end{definition}
% \addtocounter{theorem}{-1}
\endgroup

As in Sec. \ref{sec:dynamics-U}, our subsequent developments do not make mention of particular rules in the dynamics, nor do they make mention of other judgements, not listed above, that are used only for defining the dynamics of the match operator, so we do not produce these details here. Instead, it suffices to state the following conditions.

The Canonical Forms condition, which characterizes well-typed values, is identical to the corresponding condition in the structural dynamics of $\miniVerseUE$, i.e. Condition \ref{condition:canonical-forms-UP}. 

The Preservation condition ensures that evaluation preserves typing, and is again identical to the corresponding condition in the structural dynamics of $\miniVerseUE$.
\begingroup
\def\thetheorem{\ref{condition:preservation-UP}}
\begin{condition}[Preservation] If $\hastypeUC{e}{\tau}$ and $\stepsU{e}{e'}$ then $\hastypeUC{e'}{\tau}$. \end{condition}
% \addtocounter{theorem}{-1}
\endgroup

The Progress condition ensures that evaluation of a well-typed expanded expression cannot ``get stuck''. We must consider the possibility of match failure in this condition.
\begingroup
\def\thetheorem{\ref{condition:progress-UP}}
\begin{condition}[Progress] If $\hastypeUC{e}{\tau}$ then either $\isvalU{e}$ or $\matchfail{e}$ or there exists an $e'$ such that $\stepsU{e}{e'}$. \end{condition}
% \addtocounter{theorem}{-1}
\endgroup

Together, these two conditions constitute the Type Safety Condition.
%\noindent
%Condition \ref{condition:preservation-UP} is identical to Condition \ref{condition:preservation-U}, while Condition \ref{condition:progress-UP} modifies Condition \ref{condition:progress-U} to allow for match failure. 

We do not define the semantics of exhaustiveness and redundancy checking here, because these can be checked post-expansion (but see \cite{pfple1} for a formal account.)


\subsection{Syntax of the Unexpanded Language}\label{sec:syntax-UP}
The syntax of the $\miniVersePat$ unexpanded language (UL) extends the syntax of the $\miniVerseUE$ unexpanded language as shown in Figure \ref{fig:UP-unexpanded-terms}.

\begin{figure}[h!]
\[\arraycolsep=4pt\begin{array}{lllllll}
\textbf{Sort} & & 
%& \textbf{Operational Form} 
& \textbf{Stylized Form} & \textbf{Description}\\
\mathsf{UTyp} & \utau & ::= 
% & ... 
& \cdots & \text{(see Figure \ref{fig:U-unexpanded-terms})}\\
\mathsf{UExp} & \ue & ::= 
%& \ux 
& \cdots 
& \text{(see Figure \ref{fig:U-unexpanded-terms})}\\
&&
%& \aumatchwith{n}{\utau}{\ue}{\seqschemaX{\urv}} 
& \matchwith{\ue}{\seqschemaX{\urv}} & \text{match}\\
% %\LCC &&& \gray & \gray & \gray \\
% &&
%& \audefuetsm{\utau}{e}{\tsmv}{\ue}
%& \texttt{syntax}~\tsmv~\texttt{at}~\utau~\texttt{for} & \text{seTLM definition}\\
% &&&                                    & \texttt{expressions}~\{e\}~\texttt{in}~\ue\\
% &&& \autsmap{b}{\tsmv} & \utsmap{\tsmv}{b} & \text{seTLM application}\\%\ECC
\LCC &&& \color{Yellow} & \color{Yellow}\\
&&
%& \audefuptsm{\utau}{e}{\tsmv}{\ue} 
& \usyntaxup{\tsmv}{\utau}{e}{\ue}
& \text{spTLM definition}\ECC\\
% &&&                                    & \texttt{patterns}~\{e\}~\texttt{in}~\ue\\\ECC
\mathsf{URule} & \urv & ::= 
%& \aumatchrule{\upv}{\ue} 
& \matchrule{\upv}{\ue} & \text{match rule}\\
\mathsf{UPat} & \upv & ::= 
%& \ux 
& \ux & \text{identifier pattern}\\
&&
%& \auwildp 
& \wildp & \text{wildcard pattern}\\
&&
%& \aufoldp{\upv} 
& \foldp{\upv} & \text{fold pattern}\\
&&
%& \autplp{\labelset}{\mapschema{\upv}{i}{\labelset}} 
& \tplp{\mapschema{\upv}{i}{\labelset}} & \text{labeled tuple pattern}\\
&&
%& \auinjp{\ell}{\upv} 
& \injp{\ell}{\upv} & \text{injection pattern}\\
\LCC &&& \color{Yellow} & \color{Yellow}\\
&&
%& \auapuptsm{b}{\tsmv} 
& \utsmap{\tsmv}{b} & \text{spTLM application}\ECC
\end{array}\]
\caption[Syntax of the $\miniVersePat$ unexpanded language]{Syntax of the $\miniVersePat$ unexpanded language}
\label{fig:UP-unexpanded-terms}
\end{figure}

As in $\miniVerseUE$, each expanded form has a corresponding unexpanded form. We refer to these as the \emph{common forms}. The correspondence is defined in Appendix \ref{appendix:SES-shared-forms}. There are two forms related specifically to spTLMs, highlighted in yellow above: the spTLM definition form and the spTLM application form.

In addition to the stylized syntax given in Figure \ref{fig:U-unexpanded-terms}, there is also a context-free textual syntax for the UL. Again, we need only posit the existence of partial metafunctions $\parseUTypF{b}$, $\parseUExpF{b}$ and $\parseUPatF{b}$ that go from character sequences, $b$, to unexpanded types, expressions and patterns, respectively. 
\begingroup
\def\thetheorem{\ref{condition:textual-representability-SES}}
\begin{condition}[Textual Representability] ~
\begin{enumerate}
\item For each $\utau$, there exists $b$ such that $\parseUTyp{b}{\utau}$. 
\item For each $\ue$, there exists $b$ such that $\parseUExp{b}{\ue}$.
\item For each $\upv$, there exists $b$ such that $\parseUPat{b}{\upv}$.
\end{enumerate}
\end{condition}
\endgroup

\subsection{Typed Expansion}\label{sec:typed-expansion-UP}
Unexpanded terms are checked and expanded simultaneously according to the \emph{typed expansion judgements}:

\vspace{10px}
\noindent$\arraycolsep=2pt\begin{array}{ll}
\textbf{Judgement Form} & \textbf{Description}\\
\expandsTU{\uDelta}{\utau}{\tau} & \text{$\utau$ has well-formed expansion $\tau$}\\
\expandsUP{\uDelta}{\uGamma}{\uPsi}{\uPhi}{\ue}{e}{\tau} & \text{$\ue$ has expansion $e$ of type $\tau$}\\
\ruleExpands{\uDelta}{\uGamma}{\uPsi}{\uPhi}{\urv}{r}{\tau}{\tau'} & \text{$\urv$ has expansion $r$ taking values of type $\tau$ to values of type $\tau'$}\\
\patExpands{\upctx}{\uPhi}{\upv}{p}{\tau} & \text{$\upv$ has expansion $p$ matching against $\tau$ generating hypotheses $\upctx$}
% & \text{hypotheses $\upctx$}\\
\end{array}$

\subsubsection{Type Expansion}
The \emph{type expansion judgement}, $\expandsTU{\uDelta}{\utau}{\tau}$, is inductively defined by Rules (\ref{rules:expandsTU}) as before.

\subsubsection{Typed Expression, Rule and Pattern Expansion}
%\emph{Unexpanded pattern typing contexts}, $\upctx$, are defined identically to unexpanded typing contexts (i.e. we only use a distinct metavariable to emphasize their distinct roles in the judgements above). % and the definition of seTLM definition contexts is reproduced below. \emph{spTLM contexts}, $\uPhi$, are defined in Sec. \ref{sec:uptsm-definition} below.

The \emph{typed expression expansion} judgement, $\expandsUP{\uDelta}{\uGamma}{\uPsi}{\uPhi}{\ue}{e}{\tau}$, and the \emph{typed rule expansion judgement}, $\ruleExpands{\uDelta}{\uGamma}{\uPsi}{\uPhi}{\urv}{r}{\tau}{\tau'}$ are defined mutually inductively by Rules (\ref{rules:expandsU}) and Rule (\ref{rule:ruleExpands}). The \emph{typed pattern expansion judgement}, $\patExpands{\upctx}{\uPhi}{\upv}{p}{\tau}$, is inductively defined by Rules (\ref{rules:patExpands}).

Rules (\ref{rule:expandsU-var}) through (\ref{rule:expandsU-tsmap}) are adapted directly from $\miniVerseUE$, differing only in that the {spTLM context}, $\uPhi$, passes opaquely through them. 

There is one new common unexpanded expression form in $\miniVersePat$: the unexpanded match form. Rule (\ref{rule:expandsU-match}) governs this form:
\begin{equation*}\tag{\ref{rule:expandsU-match}}
\inferrule{
  % \uDelta = \uDD{\uD}{\Delta}\\\\
  \expandsUP{\uDelta}{\uGamma}{\uPhi}{\uPsi}{\ue}{e}{\tau}\\
  % \istypeU{\Delta}{\tau'}\\
  \{\ruleExpands{\uDelta}{\uGamma}{\uPsi}{\uPhi}{\urv_i}{r_i}{\tau}{\tau'}\}_{1 \leq i \leq n}\\
}{
  \expandsUP
    {\uDelta}{\uGamma}{\uPsi}{\uPhi}
    {\matchwith
      {\ue}
      {\seqschemaX{\urv}}
    }{\aematchwith
      {n}
      % {\tau'}
      {e}
      {\seqschemaX{r}}
    }{\tau'}
}
\end{equation*}  

% We can express this scheme more precisely with the following rule transformation. For each rule in Rules (\ref{rules:hastypeUP}),
% \begin{mathpar}
% %\refstepcounter{equation}
% %\label{rule:expandsU-case}
% \inferrule{J_1\\ \cdots \\ J_k}{J}
% \end{mathpar}
% the corresponding typed expansion rule is 
% \begin{mathpar}
% \inferrule{
%   \Uof{J_1} \\
%   \cdots\\
%   \Uof{J_k}
% }{
%   \Uof{J}
% }
% \end{mathpar}
% where
% \[\begin{split}
% \Uof{\istypeU{\Delta}{\tau}} & = \expandsTU{\Uof{\Delta}}{\Uof{\tau}}{\tau} \\
% \Uof{\hastypeU{\Gamma}{\Delta}{e}{\tau}} & = \expandsUP{\Uof{\Gamma}}{\Uof{\Delta}}{\uPsi}{\uPhi}{\Uof{e}}{e}{\tau}\\
% \Uof{\ruleType{\Gamma}{\Delta}{r}{\tau}{\tau'}} & = \ruleExpands{\Uof{\Gamma}}{\Uof{\Delta}}{\uPsi}{\uPhi}{\Uof{r}}{r}{\tau}{\tau'}\\
% \Uof{\{J_i\}_{i \in \labelset}} & = \{\Uof{J_i}\}_{i \in \labelset}
% \end{split}\]
% and where $\Uof{\Delta}$, $\Uof{\Gamma}$ and $\Uof{\tau}$ are defined as in Sec. \ref{sec:typed-expansion-U} and:
% \begin{itemize}
% \item $\Uof{e}$ is defined as follows
% \begin{itemize}
% \item When $e$ is of definite form, $\Uof{e}$ is defined as in Sec. \ref{sec:syntax-UP}. 
% \item When $e$ is of indefinite form, $\Uof{e}$ is a uniquely corresponding metavariable of sort $\mathsf{UExp}$ also of indefinite form. For example, $\Uof{e_1}=\ue_1$ and $\Uof{e_2}=\ue_2$.
% \end{itemize}
% \item $\Uof{r}$ is defined as follows:
% \begin{itemize}
% \item When $r$ is of definite form, $\Uof{r}$ is defined as in Sec. \ref{sec:syntax-UP}.
% \item When $e$ is of indefinite form, $\Uof{r}$ is a uniquely corresponding metavariable of sort $\mathsf{URule}$ also of indefinite form.
% \end{itemize}
% \end{itemize}

% It is instructive to use this rule transformation to generate Rules (\ref{rule:expandsUP-var}) through (\ref{rule:expandsUP-tap}) and Rule (\ref{rule:expandsUP-match}) above. We omit the remaining rules generated by this transformation, i.e. Rules (\ref*{rule:expandsUP-tlam}) through (\ref*{rule:expandsUP-in}). 

The typed rule expansion judgement is defined by Rule (\ref{rule:ruleExpands}), below:
\begin{equation*}\tag{\ref{rule:ruleExpands}}
\inferrule{
  \patExpands{\uAS{\uG'}{\pctx'}}{\uPhi}{\upv}{p}{\tau}\\
  \expandsUP{\uDelta}{\uGG{\uGcons{\uG}{\uG'}}{\Gcons{\Gamma}{\pctx'}}}{\uPsi}{\uPhi}{\ue}{e}{\tau'} 
}{
  \ruleExpands{\uDelta}{\uGG{\uG}{\Gamma}}{\uPsi}{\uPhi}{\aumatchrule{\upv}{\ue}}{\aematchrule{p}{e}}{\tau}{\tau'}
}
\end{equation*}
Because unexpanded terms mention only expression identifiers, which are given meaning by expansion to variables, the pattern typing rules must generate both an identifier expansion context, $\uG'$, and a typing context, $\pctx'$. %The second and third premises check that the domains of $\uG'$ and $\pctx$ correspond to the bindings in the unexpanded and expanded rule, respectively, with the second and third premise. 
In the second premise of the rule above, we update the ``incoming'' identifier expansion context, $\uG$, with the new identifier expansions, $\uG'$, and correspondingly, extend the ``incoming'' typing context, $\Gamma$, with the new typing hypotheses, $\pctx'$. 

Rules (\ref{rule:patExpands-var}) through (\ref{rule:patExpands-in}), reproduced below, define typed expansion  of unexpanded patterns of common form.
\begin{equation*}\tag{\ref{rule:patExpands-var}}
{\inferrule{ }{
  \patExpands{\uGG{\vExpands{\ux}{x}}{\Ghyp{x}{\tau}}}{\uPhi}{\ux}{x}{\tau}
}}
\end{equation*}
\begin{equation*}\tag{\ref{rule:patExpands-wild}}
{\inferrule{ }{
  \patExpands{\uGG{\emptyset}{\emptyset}}{\uPhi}{\wildp}{\aewildp}{\tau}
}}
\end{equation*}
\begin{equation*}\tag{\ref{rule:patExpands-fold}}
{\inferrule{ 
  \patExpands{\upctx}{\uPhi}{\upv}{p}{[\arec{t}{\tau}/t]\tau}
}{
  \patExpands{\upctx}{\uPhi}{\foldp{\upv}}{\aefoldp{p}}{\arec{t}{\tau}}
}}
\end{equation*}
\begin{equation*}\tag{\ref{rule:patExpands-tpl}}
{
  \inferrule{
    \tau = \aprod{\labelset}{\mapschema{\tau}{i}{\labelset}}\\\\
    \{\patExpands{{\upctx_i}}{\uPhi}{\upv_i}{p_i}{\tau_i}\}_{i \in \labelset}
  }{
    % \left(\shortstack{
    %   $\Delta \vdash_{\uPhi} \tplp{\mapschema{\upv}{i}{\labelset}}$\\
    %   $\leadsto$\\
    %   $\aetplp{\labelset}{\mapschema{p}{i}{\labelset}} : \aprod{\labelset}{\mapschema{\tau}{i}{\labelset}}$\vspace{-1.2em}}\right)
    \patExpands{\GIconsi{i \in \labelset}{\upctx_i}}{\uPhi}{\tplp{\mapschema{\upv}{i}{\labelset}}}{\aetplp{\labelset}{\mapschema{p}{i}{\labelset}}}{\tau}
  }
}
% \graybox{\inferrule{
%   \{\patExpands{{\upctx_i}}{\uPhi}{\upv_i}{p_i}{\tau_i}\}_{i \in \labelset}\\
% }{
%   % \patExpands{\Gconsi{i \in \labelset}{\pctx_i}}{\Phi}{
%   %   \autplp{\labelset}{\mapschema{\upv}{i}{\labelset}}
%   % }{
%   %   \aetplp{\labelset}{\mapschema{p}{i}{\labelset}}
%   % }{
%   %   \aprod{\labelset}{\mapschema{\tau}{i}{\labelset}}
%   % } %{\autplp{\labelset}{\mapschema{\upv}{i}{\labelset}}}{\aetplp{\labelset}{\mapschema}{p}{i}{\labelset}}{...}
%   \left(\shortstack{$\Delta \vdash_{\uPhi} \autplp{\labelset}{\mapschema{\upv}{i}{\labelset}}$\\$\leadsto$\\$\aetplp{\labelset}{\mapschema{p}{i}{\labelset}} : \aprod{\labelset}{\mapschema{\tau}{i}{\labelset}} \dashV \Gconsi{i \in \labelset}{\upctx_i}$\vspace{-1.2em}}\right)
% }}
\end{equation*}
\begin{equation*}\tag{\ref{rule:patExpands-in}}
{\inferrule{
  \patExpands{\upctx}{\uPhi}{\upv}{p}{\tau}
}{
  \patExpands{\upctx}{\uPhi}{\injp{\ell}{\upv}}{\aeinjp{\ell}{p}}{\asum{\labelset, \ell}{\mapschema{\tau}{i}{\labelset}; \mapitem{\ell}{\tau}}}
}}
\end{equation*}
Again, the unexpanded and expanded pattern forms in the conclusion correspond and the premises correspond to those of the corresponding pattern typing rule, i.e. Rules (\ref{rule:patType-var}) through (\ref{rule:patType-inj}), respectively. The spTLM context, $\uPhi$, passes through these rules opaquely. In Rule (\ref{rule:patExpands-tpl}), the conclusion of the rule collects all of the identifier expansions and hypotheses generated by the subpatterns. We define $\upctx_i$ as shorthand for $\uGG{\uG_i}{\pctx_i}$ and $\GIconsi{i \in \labelset}{\upctx_i}$ as shorthand for \[\uGG{\GIconsi{i \in \labelset}{\uG_i}}{\Gconsi{i \in \labelset}{\pctx_i}}\] By the definition of iterated extension of finite functions, we implicitly have that no identifiers or variables can be duplicated, i.e. that 
\[\{\{\domof{\uG_i} \cap \domof{\uG_j} = \emptyset\}_{j \in \labelset \setminus i}\}_{i \in \labelset}\]
and
\[\{\{\domof{\pctx_i} \cap \domof{\pctx_j} = \emptyset\}_{j \in \labelset \setminus i}\}_{i \in \labelset}\]
%By instead defining these rules by the rule transformation just described, we avoid having to list a number of rules that are individually uninteresting. Moreover, this approach makes our exposition somewhat robust to changes to the inner core (though not to changes to the judgement forms in the statics of the inner core).

\paragraph{spTLM Definition and Application}
Two rules remain: Rules (\ref{rule:expandsU-defuptsm}) and (\ref{rule:patExpands-apuptsm}), which define spTLM definition and application, respectively. These rules are  defined in the next two subsections, respectively.





\subsection{spTLM Definition}\label{sec:uptsm-definition}

The spTLM definition form is \[\usyntaxup{\tsmv}{\utau}{\eparse}{\ue}\] 
%The operational form corresponding to this stylized form is \[\audefuetsm{\utau}{\eparse}{\tsmv}{\ue}\]
An unexpanded expression of this form defines a {spTLM} identified as $\tsmv$ with \emph{unexpanded type annotation} $\utau$ and \emph{parse function} $\eparse$ for use within $\ue$. 

%The parse function is an expanded expression because parse functions are applied statically (i.e. during typed expansion of $\ue$), as we will discuss when describing seTLM application below, and evaluation is defined only for closed expanded expressions. This construction simplifies our exposition, though it is not entirely practical because it provides no way for TLM providers to share values between parse functions, nor any way to use TLMs when defining other TLMs. We discuss enriching the language to eliminate these limitations in Sec. \ref{sec:uetsms-static-language}, but it is pedagogically simpler to leave the necessary machinery out of our calculus for now.%$\miniVerseUE$.

Rule (\ref{rule:expandsU-defuptsm}) defines typed expansion of spTLM definitions:
% \begin{equation}\label{rule:expandsU-syntax}
% \inferrule{
%   \istypeU{\Delta}{\tau}\\
%   \expandsU{\emptyset}{\emptyset}{\emptyset}{\ueparse}{\eparse}{\aparr{\tBody}{\tParseResultExp}}\\\\
%   \expandsU{\Delta}{\Gamma}{\uPsi, \xuetsmbnd{\tsmv}{\tau}{\eparse}}{\ue}{e}{\tau'}
% }{
%   \expandsUX{\audefuetsm{\tau}{\ueparse}{\tsmv}{\ue}}{e}{\tau'}
% }
% \end{equation}
\begin{equation*}\tag{\ref{rule:expandsU-defuptsm}}
\inferrule{
  \expandsTU{\uDelta}{\utau}{\tau}\\
  \hastypeU{\emptyset}{\emptyset}{\eparse}{\aparr{\tBody}{\tParseResultPat}}\\\\
  \evalU{\eparse}{\eparse'}\\
  \expandsUP{\uDelta}{\uGamma}{\uPsi}{\uPhi, \uPhyp{\tsmv}{a}{\tau}{\eparse'}}{\ue}{e}{\tau'}
}{
  \expandsUPX{\usyntaxup{\tsmv}{\utau}{\eparse}{\ue}}{e}{\tau'}
}
\end{equation*}
This rule is similar to Rule (\ref{rule:expandsU-syntax}), which governs seTLM definitions. The premises of this rule can be understood as follows, in order:
\begin{enumerate}
\item The first premise expands the unexpanded type annotation.

\item The second premise checks that the parse function, $\eparse$, is a closed expanded function of the following type: \[\aparr{\tBody}{\tParseResultExp}\] %to generate the \emph{expanded parse function}, $\eparse$. 
 %Notice that this occurs under empty contexts, i.e. parse functions cannot refer to the surrounding bindings. 
%The parse function must be of type $\aparr{\tBody}{\tParseResultExp}$ where the type abbreviations $\tBody$ and $\tParseResultExp$ are defined as follows.

The assumed type $\tBody$ is characterized as before by Condition \ref{condition:body-isomorphism}.

$\tParseResultPat$, like $\tParseResultExp$, abbreviates a labeled sum type that distinguishes parse errors from successful parses:
\begin{align*}
L_\mathtt{SP} & \defeq \lbltxt{ParseError}, \lbltxt{SuccessP}\\
\tParseResultPat & \defeq \asum{L_\mathtt{SP}}{
  \mapitem{\lbltxt{ParseError}}{\prodt{}}, 
  \mapitem{\lbltxt{SuccessP}}{\tCEPat}
}
\end{align*} %[\mapitem{\lbltxt{ParseError}}{\prodt{}}, \mapitem{\lbltxt{SuccessE}}{\tCEExp}]

The type abbreviated $\tCEPat$ classifies encodings of \emph{proto-patterns}, $\cpv$. The syntax of proto-patterns, defined in Figure \ref{fig:UP-candidate-terms}, will be described when we describe proto-expansion validation in Sec. \ref{sec:ce-syntax-UP}. The mapping from proto-patterns to values of type $\tCEPat$ is defined by the \emph{proto-pattern encoding judgement}, $\encodeCEPat{\cpv}{e}$. An inverse mapping is defined by the \emph{proto-pattern decoding judgement}, $\decodeCEPat{e}{\cpv}$.

\[\begin{array}{ll}
\textbf{Judgement Form} & \textbf{Description}\\
\encodeCEPat{\cpv}{e} & \text{$\cpv$ has encoding $e$}\\
\decodeCEPat{e}{\cpv} & \text{$e$ has decoding $\cpv$}
\end{array}\]

Again, rather than picking a particular definition of $\tCEPat$ and defining the judgements above inductively against it, we only state the following condition, which establishes an isomorphism between values of type $\tCEPat$ and proto-patterns.

\begingroup
\def\thetheorem{\ref{condition:proto-pattern-isomorphism}}
\begin{condition}[Proto-Pattern Isomorphism] ~
\begin{enumerate}
\item For every $\cpv$, we have $\encodeCEPat{\cpv}{\ecand}$ for some $\ecand$ such that $\hastypeUC{\ecand}{\tCEPat}$ and $\isvalU{\ecand}$.
\item If $\hastypeUC{\ecand}{\tCEPat}$ and $\isvalU{\ecand}$ then $\decodeCEPat{\ecand}{\cpv}$ for some $\cpv$.
\item If $\encodeCEPat{\cpv}{\ecand}$ then $\decodeCEPat{\ecand}{\cpv}$.
\item If $\hastypeUC{\ecand}{\tCEPat}$ and $\isvalU{\ecand}$ and $\decodeCEPat{\ecand}{\cpv}$ then $\encodeCEPat{\cpv}{\ecand}$.
\item If $\encodeCEPat{\cpv}{\ecand}$ and $\encodeCEPat{\cpv}{\ecand'}$ then $\ecand=\ecand'$.
\item If $\hastypeUC{\ecand}{\tCEPat}$ and $\isvalU{\ecand}$ and $\decodeCEPat{\ecand}{\cpv}$ and $\decodeCEPat{\ecand}{\cpv'}$ then $\cpv=\cpv'$.
\end{enumerate}
\end{condition}
\endgroup

\item The third premise of Rule (\ref{rule:expandsU-defuptsm}) evaluates the parse function to a value.
\item The final premise of Rule (\ref{rule:expandsU-defuptsm}) extends the spTLM context, $\uPhi$, with the newly determined {spTLM definition}, and proceeds to assign a type, $\tau'$, and expansion, $e$, to $\ue$. The conclusion of Rule (\ref{rule:expandsU-defuptsm}) assigns this type and expansion to the spTLM definition as a whole.% i.e. TLMs define behavior that is relevant during typed expansion, but not during evaluation. 



\emph{spTLM contexts}, $\uPhi$, are of the form $\uAS{\uA}{\Phi}$, where $\uA$ is a {TLM identifier expansion context}, defined previously, and $\Phi$ is a \emph{spTLM definition context}. 

%A \emph{TLM naming context}, $\uA$, is a finite function mapping each TLM name $\tsmv \in \domof{\uA}$ to the \emph{TLM name-symbol mapping}, $\vExpands{\tsmv}{a}$, for some \emph{symbol}, $a$. We write $\ctxUpdate{\uA}{\tsmv}{a}$ for the seTLM naming context that maps $\tsmv$ to $\vExpands{\tsmv}{a}$, and defers to $\uA$ for all other TLM names (i.e. the previous mapping, if it exists, is updated).

An \emph{spTLM definition context}, $\Phi$, is a finite function mapping each TLM name $a \in \domof{\Phi}$ to an \emph{expanded spTLM definition}, $\xuptsmbnd{a}{\tau}{\eparse}$, where $\tau$ is the spTLM's type annotation, and $\eparse$ is its parse function. We write $\Phi, \xuptsmbnd{a}{\tau}{\eparse}$ when $a \notin \domof{\Phi}$ for the extension of $\Phi$ that maps $a$ to $\xuptsmbnd{a}{\tau}{\eparse}$. %We write $\uptsmenv{\Delta}{\Phi}$  when all the type annotations in $\Phi$ are well-formed assuming $\Delta$, and the parse functions in $\Phi$ are closed and of type $\parr{\tBody}{\tParseResultPat}$.
%\begin{definition}[spTLM Definition Context Formation]\label{def:spTLM-def-ctx-formation} $\uptsmenv{\Delta}{\Phi}$ iff for each $\xuptsmbnd{a}{\tau}{\eparse} \in \Phi$, we have $\istypeU{\Delta}{\tau}$ and $\hastypeU{\emptyset}{\emptyset}{\eparse}{\parr{\tBody}{\tParseResultPat}}$.\end{definition}
We define $\uPhi, \uPhyp{\tsmv}{a}{\tau}{\eparse}$, when $\uPhi=\uAS{\uA}{\Phi}$, as an abbreviation of \[\uAS{\ctxUpdate{\uA}{\tsmv}{a}}{\Phi, \xuptsmbnd{a}{\tau}{\eparse}}\]
% and $\uPhi \cup \uPhi'$ when $\uPhi=\uAS{\uA}{\Phi}$ and $\uPhi'=\uAS{\uA'}{\Phi'}$ as an abbreviation of \[\uAS{\uA \cup \uA'}{\Phi \cup \Phi'}\]
\end{enumerate}
\subsection{spTLM Application}\label{sec:uptsm-application}
The unexpanded pattern form for applying an spTLM named $\tsmv$ to a literal form with literal body $b$ is:
\[
\utsmap{\tsmv}{b}
\] 
This stylized form is identical to the stylized form for seTLM application, differing in that appears within the syntax of unexpanded patterns, $\upv$, rather than unexpanded expressions, $\ue$. %It uses forward slashes as delimiters, though stylized variants of any of the literal forms specified in Figure \ref{fig:literal-forms} would be straightforward to add to the syntax table in Figure \ref{fig:UP-unexpanded-terms} (we omit them for simplicity). 
% The corresponding operatio?nal form is $\auapuptsm{b}{\tsmv}$.%, i.e. there is an operator $\texttt{uapuptsm}[b]$ for each literal body $b$ indexed by the TLM name $\tsmv$ and taking no arguments.

Rule (\ref{rule:patExpands-apuptsm}), below, governs spTLM application. 
\begin{equation*}\tag{\ref{rule:patExpands-apuptsm}}
\inferrule{
  \uPhi = \uPhi', \uPhyp{\tsmv}{a}{\tau}{\eparse}\\\\
  \encodeBody{b}{\ebody}\\
  \evalU{\ap{\eparse}{\ebody}}{\aein{\mathtt{SuccessP}}{\ecand}}\\
  \decodeCEPat{\ecand}{\cpv}\\\\
  \segOK{\segof{\cpv}}{b}\\
  \cvalidP{\upctx}{\pscene{\uDelta}{\uPhi}{b}}{\cpv}{p}{\tau}
}{
  \patExpands{\upctx}{\uPhi}{\utsmap{\tsmv}{b}}{p}{\tau}
}
\end{equation*}
This rule is similar to Rule (\ref{rule:expandsU-tsmap}), which governs seTLM application. Its premises can be understood as follows, in order:
\begin{enumerate}
\item The first premise ensures that $\tsmv$ has been defined and extracts the type annotation and parse function.
\item The second premise determines the encoding of the literal body, $\ebody$.
\item The third premise applies the parse function $\eparse$ to the encoding of the literal body. If parsing succeeds, then $\ecand$ will be a value of type $\tCEPat$ (assuming a well-formed spTLM context, by application of the Preservation assumption, Assumption \ref{condition:preservation-UP}.) We call $\ecand$ the \emph{encoding of the proto-expansion}.

If the parse function produces a value labeled $\lbltxt{ParseError}$, then typed expansion fails. No rule is necessary to handle this case. 

\item The fourth premise decodes the encoding of the proto-expansion to produce the \emph{proto-expansion}, $\cpv$, itself.

\item The fifth premise ensures that the proto-expansion induces a valid segmentation of $b$, i.e. that the spliced pattern locations are within bounds and non-overlapping.

\item The final premise of Rule (\ref{rule:expandsU-tsmap}) \emph{validates} the proto-expansion and simultaneously generates the \emph{final expansion}, $e$, and generates hypotheses $\uGamma$, which appear in the conclusion of the rule. The proto-pattern validation judgement is discussed next.
\end{enumerate}

\subsection{Syntax of Proto-Expansions}\label{sec:ce-syntax-UP}

\begin{figure}[h!]
\hspace{-5px}$\arraycolsep=4pt\begin{array}{lllllll}
\textbf{Sort} & & & \textbf{Operational Form} & \textbf{Stylized Form} & \textbf{Description}\\
\mathsf{PrTyp} & \ctau & ::= & \cdots & \cdots & \text{(see Figure \ref{fig:U-candidate-terms})}\\
\mathsf{PrExp} & \ce & ::= & \cdots & \cdots &\text{(see Figure \ref{fig:U-candidate-terms})}\\
&&& \acematchwith{n}{\ce}{\seqschemaX{\crv}} & \matchwith{\ce}{\seqschemaX{\crv}} & \text{match}\\
%\LCC &&& \gray & \gray & \gray\\
\mathsf{PrRule} & \crv & ::= & \acematchrule{p}{\ce} & \matchrule{p}{\ce} & \text{rule}\\
\mathsf{PrPat} & \cpv & ::= & \acewildp & \wildp & \text{wildcard pattern}\\
&&& \acefoldp{p} & \foldp{p} & \text{fold pattern}\\
&&& \acetplp{\labelset}{\mapschema{\cpv}{i}{\labelset}} & \tplp{\mapschema{\cpv}{i}{\labelset}} & \text{labeled tuple pattern}\\
&&& \aceinjp{\ell}{\cpv} & \injp{\ell}{\cpv} & \text{injection pattern}\\
\LCC &&& \color{Yellow} & \color{Yellow} & \color{Yellow}\\
&&& \acesplicedp{m}{n}{\ctau} & \splicedp{m}{n}{\ctau} & \text{spliced pattern ref.}\ECC
\end{array}$
\caption{Syntax of $\miniVersePat$ proto-expansions}
\label{fig:UP-candidate-terms}
\end{figure}

Figure \ref{fig:UP-candidate-terms} defines the syntax of proto-types, $\ctau$, proto-expressions, $\ce$, proto-rules, $\crv$, and proto-patterns, $\cpv$. %The syntax of ce-types is identical to that given in Figure \ref{fig:U-candidate-terms}, which was described in Sec. \ref{sec:ce-syntax-U}. 
Proto-expansion terms are identified up to $\alpha$-equivalence in the usual manner.

Each expanded form, with the exception of the variable pattern form, maps onto a proto-expansion form. We refer to these collectively as the \emph{common proto-expansion forms}. The mapping is given explicitly in Appendix \ref{appendix:proto-expansions-SES}. 

The main proto-expansion form of interest here, highlighted in yellow, is the proto-pattern form for \emph{references to spliced unexpanded patterns}.

% Notice that patterns, $p$, rather than proto-patterns, $\upv$, appear in the match proto-expression form. This is because proto-expressions arise by the action of seTLMs. It would not be sensible for an seTLM to spliced a pattern out of a literal body.

\subsection{Proto-Expansion Validation}\label{sec:ce-validation-UP}
The \emph{proto-expansion validation judgements} validate proto-expansion terms and simultaneously generate their final expansions.

\vspace{10px}\noindent$\arraycolsep=2pt\begin{array}{ll}
\textbf{Judgement Form} & \textbf{Description}\\
\cvalidT{\Delta}{\tscenev}{\ctau}{\tau} & \text{$\ctau$ has well-formed expansion $\tau$}\\
\cvalidE{\Delta}{\Gamma}{\escenev}{\ce}{e}{\tau} & \text{$\ce$ has expansion $e$ of type $\tau$}\\
\cvalidR{\Delta}{\Gamma}{\escenev}{\crv}{r}{\tau}{\tau'} & \text{$\crv$ has expansion $r$ taking values of type $\tau$ to values of type $\tau'$}\\
\cvalidP{\upctx}{\pscenev}{\cpv}{p}{\tau} & \text{$\cpv$ has expansion $p$ matching against $\tau$ generating assumptions $\upctx$}
\end{array}$\vspace{10px}

\emph{Type splicing scenes}, $\tscenev$, are of the form $\tsceneUP{\uDelta}{b}$. \emph{Expression splicing scenes}, $\escenev$, are of the form $\esceneUP{\uDelta}{\uGamma}{\uPsi}{\uPhi}{b}$. \emph{Pattern splicing scenes}, $\pscenev$, are of the form $\pscene{\uDelta}{\uPhi}{b}$. As in $\miniVerseUE$, their purpose is to ``remember'', during proto-expansion validation, the contexts and the literal body from the TLM application site (cf. Rules (\ref{rule:expandsU-tsmap}) and (\ref{rule:patExpands-apuptsm})), because these are necessary to validate references to spliced terms. We write $\tsfrom{\escenev}$ for the type splicing scene constructed by dropping unnecessary contexts from $\escenev$:
\[\tsfrom{\esceneUP{\uDelta}{\uGamma}{\uPsi}{\uPhi}{b}} = \tsceneUP{\uDelta}{b}\]

\subsubsection{Proto-Type Validation}
The \emph{proto-type validation judgement}, $\cvalidT{\Delta}{\tscenev}{\ctau}{\tau}$, is inductively defined by Rules (\ref{rules:cvalidT-U}), which were already described in Sec. \ref{sec:SE-proto-type-validation}.

\subsubsection{Proto-Expansion Expression and Rule Validation}
The \emph{proto-expression validation judgement}, $\cvalidE{\Delta}{\Gamma}{\escenev}{\ce}{e}{\tau}$, and the \emph{proto-rule validation judgement}, $\cvalidR{\Delta}{\Gamma}{\escenev}{\crv}{r}{\tau}{\tau'}$, are defined mutually inductively with Rules (\ref{rules:expandsU}) and Rule (\ref{rule:ruleExpands}) by Rules (\ref{rules:cvalidE-U}) and Rule (\ref{rule:cvalidR-UP}), respectively.

Rules (\ref{rule:cvalidE-U-var}) through (\ref{rule:cvalidE-U-splicede}) were described in Sec. \ref{sec:ce-validation-U}. Rule (\ref{rule:cvalidE-U-match}) governs match proto-expressions:
\begin{equation*}\tag{\ref{rule:cvalidE-U-match}}
\inferrule{
  \cvalidE{\Delta}{\Gamma}{\escenev}{\ce}{e}{\tau}\\
  % \cvalidT{\Delta}{\tsfrom{\escenev}}{\ctau'}{\tau'}\\\\
  \{\cvalidR{\Delta}{\Gamma}{\escenev}{\crv_i}{r_i}{\tau}{\tau'}\}_{1 \leq i \leq n}
}{\cvalidE{\Delta}{\Gamma}{\escenev}{\acematchwith{n}{\ce}{\seqschemaX{\crv}}}{\aematchwith{n}{e}{\seqschemaX{r}}}{\tau'}}
\end{equation*}
Rule (\ref{rule:cvalidR-UP}) governs proto-rules:
\begin{equation*}\tag{\ref{rule:cvalidR-UP}}
\inferrule{
  \patType{\pctx}{p}{\tau}\\
  \cvalidE{\Delta}{\Gcons{\Gamma}{\pctx}}{\escenev}{\ce}{e}{\tau'}
}{
  \cvalidR{\Delta}{\Gamma}{\escenev}{\acematchrule{p}{\ce}}{\aematchrule{p}{e}}{\tau}{\tau'}
}
\end{equation*}
Notice that proto-rules bind expanded patterns, rather than proto-patterns. This is because proto-rules appear in proto-expressions, which are generated by seTLMs. Proto-patterns are generated exclusively by spTLMs.

% \begin{equation}\label{rule:cvalidE-UP-var}
% \inferrule{ }{
%   \cvalidE{\Delta}{\Gamma, \Ghyp{x}{\tau}}{\escenev}{x}{x}{\tau}
% }
% \end{equation}
% \begin{equation}\label{rule:cvalidE-UP-lam}
% \inferrule{
%   \cvalidT{\Delta}{\tsfrom{\escenev}}{\ctau}{\tau}\\
%   \cvalidE{\Delta}{\Gamma, \Ghyp{x}{\tau}}{\escenev}{\ce}{e}{\tau'}
% }{
%   \cvalidE{\Delta}{\Gamma}{\escenev}{\acelam{\ctau}{x}{\ce}}{\aelam{\tau}{x}{e}}{\aparr{\tau}{\tau'}}
% }
% \end{equation}
% \begin{equation}\label{rule:cvalidE-UP-ap}
%   \inferrule{
%     \cvalidE{\Delta}{\Gamma}{\escenev}{\ce_1}{e_1}{\aparr{\tau}{\tau'}}\\
%     \cvalidE{\Delta}{\Gamma}{\escenev}{\ce_2}{e_2}{\tau}
%   }{
%     \cvalidE{\Delta}{\Gamma}{\escenev}{\aceap{\ce_1}{\ce_2}}{\aeap{e_1}{e_2}}{\tau'}
%   }
% \end{equation}
% \begin{equation}\label{rule:cvalidE-UP-tlam}
%   \inferrule{
%     \cvalidE{\Delta, \Dhyp{t}}{\Gamma}{\escenev}{\ce}{e}{\tau}
%   }{
%     \cvalidEX{\acetlam{t}{\ce}}{\aetlam{t}{e}}{\aall{t}{\tau}}
%   }
% \end{equation}
% \begin{equation}\label{rule:cvalidE-UP-tap}
%   \inferrule{
%     \cvalidEX{\ce}{e}{\aall{t}{\tau}}\\
%     \cvalidT{\Delta}{\tsfrom{\escenev}}{\ctau'}{\tau'}
%   }{
%     \cvalidEX{\acetap{\ce}{\ctau'}}{\aetap{e}{\tau'}}{[\tau'/t]\tau}
%   }
% \end{equation}
% \begin{equation}\label{rule:cvalidE-UP-fold}
%   \inferrule{
%     \cvalidT{\Delta, \Dhyp{t}}{\escenev}{\ctau}{\tau}\\
%     \cvalidEX{\ce}{e}{[\arec{t}{\tau}/t]\tau}
%   }{
%     \cvalidEX{\acefold{t}{\ctau}{\ce}}{\aefold{e}}{\arec{t}{\tau}}
%   }
% \end{equation}
% \begin{equation}\label{rule:cvalidE-UP-unfold}
%   \inferrule{
%     \cvalidEX{\ce}{e}{\arec{t}{\tau}}
%   }{
%     \cvalidEX{\aceunfold{\ce}}{\aeunfold{e}}{[\arec{t}{\tau}/t]\tau}
%   }
% \end{equation}
% \begin{equation}\label{rule:cvalidE-UP-tpl}
%   \inferrule{
%     \{\cvalidEX{\ce_i}{e_i}{\tau_i}\}_{i \in \labelset}
%   }{
%     \cvalidEX{\acetpl{\labelset}{\mapschema{\ce}{i}{\labelset}}}{\aetpl{\labelset}{\mapschema{e}{i}{\labelset}}}{\aprod{\labelset}{\mapschema{\tau}{i}{\labelset}}}
%   }
% \end{equation}
% \begin{equation}\label{rule:cvalidE-UP-pr}
%   \inferrule{
%     \cvalidEX{\ce}{e}{\aprod{\labelset, \ell}{\mapschema{\tau}{i}{\labelset}; \mapitem{\ell}{\tau}}}
%   }{
%     \cvalidEX{\acepr{\ell}{\ce}}{\aepr{\ell}{e}}{\tau}
%   }
% \end{equation}
% \begin{equation}\label{rule:cvalidE-UP-in}
%   \inferrule{
%     \{\cvalidT{\Delta}{\tsfrom{\escenev}}{\ctau_i}{\tau_i}\}_{i \in \labelset}\\
%     \cvalidT{\Delta}{\tsfrom{\escenev}}{\ctau}{\tau}\\
%     \cvalidEX{\ce}{e}{\tau}
%   }{
%     \left\{\shortstack{$\Delta~\Gamma \vdash_\uPsi \acein{\labelset, \ell}{\ell}{\mapschema{\ctau}{i}{\labelset}; \mapitem{\ell}{\ctau}}{\ce}$\\$\leadsto$\\$\aein{\labelset, \ell}{\ell}{\mapschema{\tau}{i}{\labelset}; \mapitem{\ell}{\tau}}{e} : \asum{\labelset, \ell}{\mapschema{\tau}{i}{\labelset}; \mapitem{\ell}{\tau}}$\vspace{-1.2em}}\right\}
%   }
% \end{equation}
% \begin{equation}\label{rule:cvalidE-UP-case}
%   \inferrule{
%     \cvalidEX{\ce}{e}{\asum{\labelset}{\mapschema{\tau}{i}{\labelset}}}\\
%     \{\cvalidE{\Delta}{\Gamma, \Ghyp{x_i}{\tau_i}}{\escenev}{\ue_i}{e_i}{\tau}\}_{i \in \labelset}
%   }{
%     \cvalidEX{\acecase{\labelset}{\tau}{\ce}{\mapschemab{x}{\ce}{i}{\labelset}}}{\aecase{\labelset}{e}{\mapschemab{x}{e}{i}{\labelset}}}{\tau}
%   }
% \end{equation}
% \end{subequations}
%The \emph{ce-rule validation judgement}, $\cvalidR{\Delta}{\Gamma}{\escenev}{\crv}{r}{\tau}{\tau'}$, is defined mutually inductively with Rules (\ref{rules:cvalidE-UP}) by 


\subsubsection{Proto-Pattern Validation}
spTLMs generate candidate expansions of proto-pattern form, as described in Sec. \ref{sec:uptsm-application}. The \emph{proto-pattern validation judgement}, $\cvalidP{\upctx}{\pscenev}{\cpv}{p}{\tau}$, which appears as the final premise of Rule (\ref{rule:patExpands-apuptsm}), validates proto-patterns and simultaneously generates the final expansion, $p$, and the unexpanded typing hypotheses $\upctx$.

The proto-pattern validation judgement is defined mutually inductively with Rules (\ref{rules:patExpands}) by Rules (\ref{rules:cvalidP-UP}), reproduced below.
\begin{equation*}\tag{\ref{rule:cvalidP-UP-wild}}
\inferrule{ }{
  \cvalidP{\uGG{\emptyset}{\emptyset}}{\pscenev}{\acewildp}{\aewildp}{\tau}
}
\end{equation*}
\begin{equation*}\tag{\ref{rule:cvalidP-UP-fold}}
\inferrule{
  \cvalidP{\upctx}{\pscenev}{\cpv}{p}{[\arec{t}{\tau}/t]\tau}
}{
  \cvalidP{\upctx}{\pscenev}{\acefoldp{\cpv}}{\aefoldp{p}}{\arec{t}{\tau}}
}
\end{equation*}
\begin{equation*}\tag{\ref{rule:cvalidP-UP-tpl}}
\inferrule{
  \tau = \aprod{\labelset}{\mapschema{\tau}{i}{\labelset}}\\\\
  \{\cvalidP{\upctx_i}{\pscenev}{\cpv_i}{p_i}{\tau_i}\}_{i \in \labelset}
}{
% \left(\shortstack{$\vdash^{\pscenev} \acetplp{\labelset}{\mapschema{\cpv}{i}{\labelset}}$\\$\leadsto$\\$\aetplp{\labelset}{\mapschema{p}{i}{\labelset}} : \aprod{\labelset}{\mapschema{\tau}{i}{\labelset}}~\dashVx^{\,\Gconsi{i \in \labelset}{\upctx_i}}$\vspace{-1.2em}}\right)
  \cvalidP{\GIconsi{i \in \labelset}{\upctx_i}}{\pscenev}{\acetplp{\labelset}{\mapschema{\cpv}{i}{\labelset}}}{\aetplp{\labelset}{\mapschema{p}{i}{\labelset}}}{\tau}
}
\end{equation*}
\begin{equation*}\tag{\ref{rule:cvalidP-UP-in}}
\inferrule{
  \cvalidP{\upctx}{\pscenev}{\cpv}{p}{\tau}
}{
  \cvalidP{\upctx}{\pscenev}{\aceinjp{\ell}{\cpv}}{\aeinjp{\ell}{p}}{\asum{\labelset, \ell}{\mapschema{\tau}{i}{\labelset}; \mapitem{\ell}{\tau}}}
}
\end{equation*}
\begin{equation*}\tag{\ref{rule:cvalidP-UP-spliced}}
\inferrule{
  \cvalidT{\emptyset}{\tsceneUP{\uDelta}{b}}{\ctau}{\tau}\\
  \parseUPat{\bsubseq{b}{m}{n}}{\upv}\\
  \patExpands{\upctx}{\uPhi}{\upv}{p}{\tau}
}{
  \cvalidP{\upctx}{\pscene{\uDelta}{\uPhi}{b}}{\acesplicedp{m}{n}{\ctau}}{p}{\tau}
}
\end{equation*}

Rules (\ref{rule:cvalidP-UP-wild}) through (\ref{rule:cvalidP-UP-in}) govern proto-patterns of common form, and behave like the corresponding pattern typing rules, i.e. Rules (\ref{rule:patType-wild}) through (\ref{rule:patType-inj}). Rule (\ref{rule:cvalidP-UP-spliced}) governs references to spliced unexpanded patterns. The first premise validates the type annotation. The second premise parses the indicated subsequence of the literal body, $b$, to produce the referenced unexpanded pattern, $\upv$, and the third premise types and expands $\upv$ under the spTLM context $\uPhi$ from the spTLM application site, generating the hypotheses $\upctx$. These are the hypotheses generated in the conclusion of the rule.

Hypotheses can be generated only by spliced subpatterns, so there is no proto-pattern form corresponding to variable patterns. This achieves the prohibition on hidden bindings described in Sec. \ref{sec:ptsms-hygiene}. We consider this invariant formally below.

\subsection{Metatheory}
The following theorem establishes that typed pattern expansion produces an expanded pattern that matches values of the specified type and generates the specified hypotheses. We must mutually state the corresponding proposition about proto-patterns, because the relevant judgements are mutually defined.
\begingroup
\def\thetheorem{\ref{thm:typed-pattern-expansion}}
\begin{theorem}[Typed Pattern Expansion] ~
\begin{enumerate}
  \item If $\pExpandsSP{\uDD{\uD}{\Delta}}{\uAS{\uA}{\Phi}}{\upv}{p}{\tau}{\uGG{\uG}{\pctx}}$ then $\patType{\pctx}{p}{\tau}$.
  \item If $\cvalidP{\uGG{\uG}{\pctx}}{\pscene{\uDD{\uD}{\Delta}}{\uAP{\uA}{\Phi}}{b}}{\cpv}{p}{\tau}$ then $\patType{\pctx}{p}{\tau}$.
\end{enumerate}
\end{theorem}
\begin{proof}
  By mutual rule induction on Rules (\ref{rules:patExpands}) and Rules (\ref{rules:cvalidP-UP}). The full proof is given in Appendix \ref{appendix:SES-typed-pattern-expansion}. We will reproduce only the interesting cases below.
  \begin{enumerate}
  \item The only interesting case in the proof of part 1 is the case for spTLM application. In the following, let $\uDelta=\uDD{\uD}{\Delta}$ and $\upctx=\uGG{\uG}{\pctx}$ and $\uPhi=\uAP{\uA}{\Phi}$.
  \begin{byCases}
%     \item[\text{(\ref{rule:patExpands-var})}] ~
%       \begin{pfsteps*}
%         \item $\upv=\ux$ \BY{assumption}
%         \item $p=x$ \BY{assumption}
%         \item $\pctx=\Ghyp{x}{\tau}$ \BY{assumption}
%         \item $\patType{\Ghyp{x}{\tau}}{x}{\tau}$ \BY{Rule (\ref{rule:patType-var})}
%       \end{pfsteps*}
%       \resetpfcounter
%     \item[\text{(\ref{rule:patExpands-wild})}] ~
%       \begin{pfsteps*}
%         \item $p=\aewildp$ \BY{assumption}
%         \item $\pctx = \emptyset$ \BY{assumption}
%         \item $\patType{\emptyset}{\aewildp}{\tau}$ \BY{Rule (\ref{rule:patType-wild})}
%       \end{pfsteps*}
%       \resetpfcounter
%     \item[\text{(\ref{rule:patExpands-fold})}] ~
%       \begin{pfsteps*}
%         \item $\upv=\foldp{\upv'}$ \BY{assumption}
%         \item $p=\aefoldp{p'}$ \BY{assumption}
%         \item $\tau=\arec{t}{\tau'}$ \BY{assumption}
%         %\item $\uptsmenv{\Delta}{\Phi}$ \BY{assumption} \pflabel{env}
%         \item $\patExpands{\upctx}{\uPhi}{\upv'}{p'}{[\arec{t}{\tau'}/t]\tau'}$ \BY{assumption} \pflabel{patExpands}
%         \item $\patType{\pctx}{p'}{[\arec{t}{\tau'}/t]\tau'}$ \BY{IH, part 1 on \pfref{patExpands}} \pflabel{patType}
%         \item $\patType{\pctx}{\aefoldp{p'}}{\arec{t}{\tau'}}$ \BY{Rule (\ref{rule:patType-fold}) on \pfref{patType}}
%       \end{pfsteps*}
%       \resetpfcounter
%     \item[\text{(\ref{rule:patExpands-tpl})}] ~
%       \begin{pfsteps*}
%         \item $\upv=\tplp{\mapschema{\upv}{i}{\labelset}}$ \BY{assumption}
%         \item $p=\aetplp{\labelset}{\mapschema{p}{i}{\labelset}}$ \BY{assumption}
%         \item $\tau=\aprod{\labelset}{\mapschema{\tau}{i}{\labelset}}$ \BY{assumption}
%         \item $\{\patExpands{\uGG{\uG_i}{\pctx_i}}{\uPhi}{\upv_i}{p_i}{\tau_i}\}_{i \in \labelset}$ \BY{assumption} \pflabel{patExpands}
%         \item $\pctx = \Gconsi{i \in \labelset}{\pctx_i}$ \BY{assumption}
%         %\item $\uptsmenv{\Delta}{\Phi}$ \BY{assumption} \pflabel{env}
%         \item $\{\patType{\pctx_i}{p_i}{\tau_i}\}_{i \in \labelset}$ \BY{IH, part 1 over \pfref{patExpands}}\pflabel{patType}
%         \item $\patType{\Gconsi{i \in \labelset}{\pctx_i}}{\aetplp{\labelset}{\mapschema{p}{i}{\labelset}}}{\aprod{\labelset}{\mapschema{\tau}{i}{\labelset}}}$ \BY{Rule (\ref{rule:patType-tpl}) on \pfref{patType}}
%       \end{pfsteps*}
%       \resetpfcounter
%     \item[\text{(\ref{rule:patExpands-in})}] ~
%       \begin{pfsteps*}
%         \item $\upv=\injp{\ell}{\upv'}$ \BY{assumption}
%         \item $p=\aeinjp{\ell}{p'}$ \BY{assumption}
%         \item $\tau=\asum{\labelset, \ell}{\mapschema{\tau}{i}{\labelset}; \mapitem{\ell}{\tau'}}$ \BY{assumption}
%         \item $\patExpands{\upctx}{\uPhi}{\upv'}{p'}{\tau'}$ \BY{assumption} \pflabel{patExpands}
% %        \item $\uptsmenv{\Delta}{\Phi}$ \BY{assumption} \pflabel{env}
%         \item $\patType{\pctx}{p'}{\tau'}$ \BY{IH, part 1 on \pfref{patExpands}} \pflabel{patType}
%         \item $\patType{\pctx}{\aeinjp{\ell}{p'}}{\asum{\labelset, \ell}{\mapschema{\tau}{i}{\labelset}; \mapitem{\ell}{\tau'}}}$ \BY{Rule (\ref{rule:patType-inj}) on \pfref{patType}}
%       \end{pfsteps*}
%       \resetpfcounter
    \item[\text{(\ref{rule:patExpands-apuptsm})}] ~
      \begin{pfsteps*}
        \item $\upv=\utsmap{\tsmv}{b}$ \BY{assumption}
        \item $\uA=\uA', \vExpands{\tsmv}{a}$ \BY{assumption}
        \item $\Phi=\Phi', \xuptsmbnd{a}{\tau}{\eparse}$ \BY{assumption}
        \item $\encodeBody{b}{\ebody}$ \BY{assumption}
        \item $\evalU{\eparse(\ebody)}{\aein{\mathtt{SuccessP}}{\ecand}}$ \BY{assumption}
        \item $\decodeCEPat{\ecand}{\cpv}$ \BY{assumption}
        \item $\cvalidP{\uGG{\uG}{\pctx}}{\pscene{\uDelta}{\uAP{\uA}{\Phi}}{b}}{\cpv}{p}{\tau}$ \BY{assumption} \pflabel{cvalidP}
%        \item $\uptsmenv{\Delta}{\Phi', \xuptsmbnd{a}{\tau}{\eparse}}$ \BY{assumption} \pflabel{env}
        \item $\patType{\pctx}{p}{\tau}$ \BY{IH, part 2 on \pfref{cvalidP}}
      \end{pfsteps*}
      \resetpfcounter
  \end{byCases}

  \item The only interesting case in the proof of part 2 is the case for spliced patterns. In the following, let $\upctx=\uGG{\uG}{\pctx}$ and $\uDelta=\uDD{\uD}{\Delta}$ and $\uPhi=\uAP{\uA}{\Phi}$.
  \begin{byCases}
%     \item[\text{(\ref{rule:cvalidP-UP-wild})}] ~
%       \begin{pfsteps*}
%         \item $p=\aewildp$ \BY{assumption}
%         \item $\pctx=\emptyset$ \BY{assumption}
%         \item $\patType{\emptyset}{\aewildp}{\tau}$ \BY{Rule (\ref{rule:patType-wild})}
%       \end{pfsteps*}
%       \resetpfcounter
%     \item[\text{(\ref{rule:cvalidP-UP-fold})}] ~
%       \begin{pfsteps*}
%         \item $\cpv=\acefoldp{\cpv'}$ \BY{assumption}
%         \item $p=\aefoldp{p'}$ \BY{assumption}
%         \item $\tau=\arec{t}{\tau'}$ \BY{assumption}
%         % \item $\uptsmenv{\Delta}{\Phi}$ \BY{assumption} \pflabel{env}
%         \item $\cvalidP{\upctx}{\pscene{\Delta}{\uPhi}{b}}{\cpv'}{p'}{[\arec{t}{\tau'}/t]\tau'}$ \BY{assumption} \pflabel{cvalidP}
%         \item $\patType{\pctx}{p'}{[\arec{t}{\tau'}/t]\tau'}$ \BY{IH, part 2 on \pfref{cvalidP}} \pflabel{patType}
%         \item $\patType{\pctx}{\aefoldp{p'}}{\arec{t}{\tau'}}$ \BY{Rule (\ref{rule:patType-fold}) on \pfref{patType}}
%       \end{pfsteps*}
%       \resetpfcounter
%     \item[\text{(\ref{rule:cvalidP-UP-tpl})}] ~
%       \begin{pfsteps*}
%         \item $\cpv=\acetplp{\labelset}{\mapschema{\cpv}{i}{\labelset}}$ \BY{assumption}
%         \item $p=\aetplp{\labelset}{\mapschema{p}{i}{\labelset}}$ \BY{assumption}
%         \item $\tau=\aprod{\labelset}{\mapschema{\tau}{i}{\labelset}}$ \BY{assumption}
%         \item $\{\cvalidP{\uGG{\uG_i}{\pctx_i}}{\pscene{\Delta}{\uPhi}{b}}{\cpv_i}{p_i}{\tau_i}\}_{i \in \labelset}$ \BY{assumption} \pflabel{cvalidP}
%         \item $\pctx = \Gconsi{i \in \labelset}{\pctx_i}$ \BY{assumption}
%         %\item $\uptsmenv{\Delta}{\Phi}$ \BY{assumption} \pflabel{env}
%         \item $\{\patType{\pctx_i}{p_i}{\tau_i}\}_{i \in \labelset}$ \BY{IH, part 2 over \pfref{cvalidP}}\pflabel{patType}
%         \item $\patType{\Gconsi{i \in \labelset}{\pctx_i}}{\aetplp{\labelset}{\mapschema{p}{i}{\labelset}}}{\aprod{\labelset}{\mapschema{\tau}{i}{\labelset}}}$ \BY{Rule (\ref{rule:patType-tpl}) on \pfref{patType}}
%       \end{pfsteps*}
%       \resetpfcounter
%     \item[\text{(\ref{rule:cvalidP-UP-in})}] ~
%       \begin{pfsteps*}
%         \item $\cpv=\aceinjp{\ell}{\cpv'}$ \BY{assumption}
%         \item $p=\aeinjp{\ell}{p'}$ \BY{assumption}
%         \item $\tau=\asum{\labelset, \ell}{\mapschema{\tau}{i}{\labelset}; \mapitem{\ell}{\tau'}}$ \BY{assumption}
%         \item $\cvalidP{\upctx}{\pscene{\Delta}{\uPhi}{b}}{\cpv'}{p'}{\tau'}$ \BY{assumption} \pflabel{cvalidP}
% %        \item $\uptsmenv{\Delta}{\Phi}$ \BY{assumption} \pflabel{env}
%         \item $\patType{\pctx}{p'}{\tau'}$ \BY{IH, part 2 on \pfref{cvalidP}} \pflabel{patType}
%         \item $\patType{\pctx}{\aeinjp{\ell}{p'}}{\asum{\labelset, \ell}{\mapschema{\tau}{i}{\labelset}; \mapitem{\ell}{\tau'}}}$ \BY{Rule (\ref{rule:patType-inj}) on \pfref{patType}}
%       \end{pfsteps*}
%       \resetpfcounter
    \item[\text{(\ref{rule:cvalidP-UP-spliced})}] ~
      \begin{pfsteps*}
        \item $\cpv=\acesplicedp{m}{n}{\ctau}$ \BY{assumption}
        \item $\cvalidT{\emptyset}{\tsceneUP{\uDelta}{b}}{\ctau}{\tau}$ \BY{assumption}
        \item $\parseUExp{\bsubseq{b}{m}{n}}{\upv}$ \BY{assumption}
        \item $\patExpands{\upctx}{\uPhi}{\upv}{p}{\tau}$ \BY{assumption} \pflabel{patExpands}
        \item $\patType{\pctx}{p}{\tau}$ \BY{IH, part 1 on \pfref{patExpands}}
      \end{pfsteps*}
      \resetpfcounter
  \end{byCases}
  \end{enumerate}
The mutual induction can be shown to be well-founded by showing that the following numeric metric on the judgements that we induct on is decreasing:
\begin{align*}
\sizeof{\patExpands{\upctx}{\uPhi}{\upv}{p}{\tau}} & = \sizeof{\upv}\\
\sizeof{{\cvalidP{\upctx}{\pscene{\uDelta}{\uPhi}{b}}{\cpv}{p}{\tau}}} & = \sizeof{b}
\end{align*}
where $\sizeof{b}$ is the length of $b$ and $\sizeof{\upv}$ is the sum of the lengths of the literal bodies in $\upv$ (see Appendix \ref{appendix:SES-body-lengths}.)

The only case in the proof of part 1 that invokes part 2 is Case (\ref{rule:patExpands-apuptsm}). There, we have that the metric remains stable: \begin{align*}
 & \sizeof{\patExpands{\upctx}{\uPhi}{\utsmap{\tsmv}{b}}{p}{\tau}}\\
=& \sizeof{{\cvalidP{\upctx}{\pscene{\uDelta}{\uPhi}{b}}{\cpv}{p}{\tau}}}\\
=&\sizeof{b}\end{align*}

The only case in the proof of part 2 that invokes part 1 is Case (\ref{rule:cvalidP-UP-spliced}). There, we have that $\parseUPat{\bsubseq{b}{m}{n}}{\upv}$ and the IH is applied to the judgement $\patExpands{\upctx}{\uPhi}{\upv}{p}{\tau}$. Because the metric is stable when passing from part 1 to part 2, we must have that it is strictly decreasing in the other direction:
\[\sizeof{\patExpands{\upctx}{\uPhi}{\upv}{p}{\tau}} < \sizeof{{\cvalidP{\upctx}{\pscene{\uDelta}{\uPhi}{b}}{\acesplicedp{m}{n}{\ctau}}{p}{\tau}}}\]
i.e. by the definitions above, 
\[\sizeof{\upv} < \sizeof{b}\]

This is established by appeal to Condition \ref{condition:body-subsequences}, which states that subsequences of $b$ are no longer than $b$, and the following condition, which states that an unexpanded pattern constructed by parsing a textual sequence $b$ is strictly smaller, as measured by the metric defined above, than the length of $b$, because some characters must necessarily be used to apply the pattern TLM and delimit each literal body.
\begingroup
\def\thetheorem{\ref{condition:pattern-parsing}}
\begin{condition}[Pattern Parsing Monotonicity] If $\parseUPat{b}{\upv}$ then $\sizeof{\upv} < \sizeof{b}$.\end{condition}
\endgroup

Combining Conditions \ref{condition:body-subsequences} and \ref{condition:pattern-parsing}, we have that $\sizeof{\ue} < \sizeof{b}$ as needed.
\end{proof}
\endgroup

Finally, the following theorem establishes that typed expression and rule expansion produces expanded expressions and rules of the same type under the same contexts. Again, it must be stated mutually with the corresponding theorem about candidate expansion expressions and rules because the judgements are mutually defined.
\begin{theorem}[Typed Expansion] ~
\begin{enumerate}
  \item \begin{enumerate}
    \item If $\expandsUP{\uDD{\uD}{\Delta}}{\uGG{\uG}{\Gamma}}{\uPsi}{\uPhi}{\ue}{e}{\tau}$ then $\hastypeU{\Delta}{\Gamma}{e}{\tau}$.
    \item If $\ruleExpands{\uDD{\uD}{\Delta}}{\uGG{\uG}{\Gamma}}{\uPsi}{\uPhi}{\urv}{r}{\tau}{\tau'}$  then $\ruleType{\Delta}{\Gamma}{r}{\tau}{\tau'}$.
  \end{enumerate}
  \item \begin{enumerate}
    \item If $\cvalidE{\Delta}{\Gamma}{\esceneUP{\uDD{\uD}{\Delta_\text{app}}}{\uGG{\uG}{\Gamma_\text{app}}}{\uPsi}{\uPhi}{b}}{\ce}{e}{\tau}$ and $\Delta \cap \Delta_\text{app}=\emptyset$ and $\domof{\Gamma} \cap \domof{\Gamma_\text{app}}=\emptyset$ then $\hastypeU{\Dcons{\Delta}{\Delta_\text{app}}}{\Gcons{\Gamma}{\Gamma_\text{app}}}{e}{\tau}$. 
    \item If $\cvalidR{\Delta}{\Gamma}{\esceneUP{\uDD{\uD}{\Delta_\text{app}}}{\uGG{\uG}{\Gamma_\text{app}}}{\uPsi}{\uPhi}{b}}{\crv}{r}{\tau}{\tau'}$ and $\Delta \cap \Delta_\text{app}=\emptyset$ and $\domof{\Gamma} \cap \domof{\Gamma_\text{app}}=\emptyset$ then $\ruleType{\Dcons{\Delta}{\Delta_\text{app}}}{\Gcons{\Gamma}{\Gamma_\text{app}}}{r}{\tau}{\tau'}$.
  \end{enumerate}
\end{enumerate}
\end{theorem}
\begin{proof}
By mutual rule induction on Rules (\ref{rules:expandsU}), Rule (\ref{rule:ruleExpands}), Rules (\ref{rules:cvalidE-U}) and Rule (\ref{rule:cvalidR-UP}). The full proof is given in Appendix \ref{appendix:metatheory-SES}. We will reproduce only the cases that have to do with pattern matching below.

\begin{enumerate}
\item In the following cases, let $\uDelta=\uDD{\uD}{\Delta}$ and $\uGamma=\uGG{\uG}{\Gamma}$.
  \begin{enumerate}
  \item The only  cases in the proof of part 1(a) that have to do with pattern matching are the cases involving the unexpanded match expression and spTLM definition. 
  \begin{byCases}
    \item[\text{(\ref{rule:expandsU-match})}] ~
      \begin{pfsteps*}
        \item $\ue=\matchwith{\ue'}{\seqschemaX{\urv}}$ \BY{assumption}
        \item $e=\aematchwith{n}{e'}{\seqschemaX{r}}$ \BY{assumption}
        \item $\expandsUP{\uDelta}{\uGamma}{\uPsi}{\uPhi}{\ue'}{e'}{\tau'}$ \BY{assumption} \pflabel{expandsUP}
        % \item $\istypeU{\Delta}{\tau}$ \BY{assumption}\pflabel{istype}
        % \item $\expandsTU{\uDelta}{\utau}{\tau}$ \BY{assumption} \pflabel{expandsTU}
        \item $\{\ruleExpands{\uDelta}{\uGamma}{\uPsi}{\uPhi}{\urv_i}{r_i}{\tau'}{\tau}\}_{1 \leq i \leq n}$ \BY{assumption}\pflabel{ruleExpands}
        \item $\hastypeU{\Delta}{\Gamma}{e'}{\tau'}$ \BY{IH, part 1(a) on \pfref{expandsUP}}\pflabel{hasType}
        \item $\{\ruleType{\Delta}{\Gamma}{r_i}{\tau'}{\tau}\}_{1 \leq i \leq n}$ \BY{IH, part 1(b) over \pfref{ruleExpands}}\pflabel{ruleType}
        \item $\hastypeU{\Delta}{\Gamma}{\aematchwith{n}{\tau}{e'}{\seqschemaX{r}}}{\tau}$ \BY{Rule (\ref{rule:hastypeUP-match}) on \pfref{hasType} and \pfref{ruleType}}
      \end{pfsteps*}
      \resetpfcounter

    \item[\text{(\ref{rule:expandsU-defuptsm})}] ~
      \begin{pfsteps}
          \item \ue=\usyntaxup{\tsmv}{\utau'}{\eparse}{\ue'} \BY{assumption}
          \item \expandsTU{\uDelta}{\utau'}{\tau'} \BY{assumption} \pflabel{expandsTU}
         \item \hastypeU{\emptyset}{\emptyset}{\eparse}{\aparr{\tBody}{\tParseResultPat}} \BY{assumption}\pflabel{eparse}
          \item \expandsUP{\uDelta}{\uGamma}{\uPsi}{\uPhi, \uPhyp{\tsmv}{a}{\tau'}{\eparse}}{\ue'}{e}{\tau} \BY{assumption}\pflabel{expandsU}
        %  \item \uetsmenv{\Delta}{\Psi} \BY{assumption}\pflabel{uetsmenv1}
         \item \istypeU{\Delta}{\tau'} \BY{Lemma \ref{lemma:type-expansion-U} to \pfref{expandsTU}} \pflabel{istype}
        %  \item \uetsmenv{\Delta}{\Psi, \xuetsmbnd{\tsmv}{\tau'}{\eparse}} \BY{Definition \ref{def:seTLM-def-ctx-formation} on \pfref{uetsmenv1}, \pfref{istype} and \pfref{eparse}}\pflabel{uetsmenv3}
          \item \hastypeU{\Delta}{\Gamma}{e}{\tau} \BY{IH, part 1(a) on \pfref{expandsU}}
        \end{pfsteps}
        \resetpfcounter 
  \end{byCases}
  \item There is only one case.
  \begin{byCases}
    \item[\text{(\ref{rule:ruleExpands})}] ~
      \begin{pfsteps*}
        \item $\urv=\matchrule{\upv}{\ue}$ \BY{assumption}
        \item $r=\aematchrule{p}{e}$ \BY{assumption}
        \item $\patExpands{\uGG{\uA'}{\pctx'}}{\uPhi}{\upv}{p}{\tau}$ \BY{assumption} \pflabel{patExpands}
        \item $\expandsUP{\uDelta}{\uGG{{\uA}\uplus{\uA'}}{\Gcons{\Gamma}{\pctx'}}}{\uPsi}{\uPhi}{\ue}{e}{\tau'}$ \BY{assumption} \pflabel{expandsUP}
        \item $\patType{\pctx'}{p}{\tau}$ \BY{Theorem \ref{thm:typed-pattern-expansion}, part 1 on \pfref{patExpands}}\pflabel{patType}
        \item $\hastypeU{\Delta}{\Gcons{\Gamma}{\pctx'}}{e}{\tau'}$ \BY{IH, part 1(a) on \pfref{expandsUP}} \pflabel{hasType}
        \item $\ruleType{\Delta}{\Gamma}{\aematchrule{p}{e}}{\tau}{\tau'}$ \BY{Rule (\ref{rule:ruleType}) on \pfref{patType} and \pfref{hasType}}
      \end{pfsteps*}
      \resetpfcounter
  \end{byCases}
  \end{enumerate}
\item In the following, let $\uDelta=\uDD{\uD}{\Delta_\text{app}}$ and $\uGamma=\uGG{\uG}{\Gamma_\text{app}}$. \begin{enumerate}
  \item The only case in the proof of part 2(a) that has to do with pattern matching is the case involving the match proto-expression. 
  \begin{byCases}
    \item[\text{(\ref{rule:cvalidE-U-match})}] ~
      \begin{pfsteps*}
        \item $\ce=\acematchwith{n}{\ce'}{\seqschemaX{\crv}}$ \BY{assumption}
        \item $e=\aematchwith{n}{e'}{\seqschemaX{r}}$ \BY{assumption}
        \item $\cvalidE{\Delta}{\Gamma}{\esceneUP{\uDelta}{\uGamma}{\uPsi}{\uPhi}{b}}{\ce'}{e'}{\tau'}$ \BY{assumption} \pflabel{cvalidE}
        % \item $\cvalidT{\Delta}{\tsceneUP{\uDelta}{b}}{\ctau}{\tau}$ \BY{assumption} \pflabel{cvalidT}
        \item $\{\cvalidR{\Delta}{\Gamma}{\esceneUP{\uDelta}{\uGamma}{\uPsi}{\uPhi}{b}}{\crv_i}{r_i}{\tau'}{\tau}\}_{1 \leq i \leq n}$ \BY{assumption} \pflabel{cvalidR}
        \item $\Delta \cap \Delta_\text{app} = \emptyset$ \BY{assumption} \pflabel{delta-disjoint}
        \item $\domof{\Gamma} \cap \domof{\Gamma_\text{app}} = \emptyset$ \BY{assumption} \pflabel{gamma-disjoint}
        \item $\hastypeU{\Delta \cup \Delta_\text{app}}{\Gamma \cup \Gamma_\text{app}}{e'}{\tau'}$ \BY{IH, part 2(a) on \pfref{cvalidE}, \pfref{delta-disjoint} and \pfref{gamma-disjoint}} \pflabel{hastype}
        % \item $\istypeU{\Delta \cup \Delta_\text{app}}{\tau}$ \BY{Lemma \ref{lemma:candidate-expansion-type-validation} on \pfref{cvalidT}} \pflabel{istype}
        \item $\ruleType{\Delta \cup \Delta_\text{app}}{\Gamma \cup \Gamma_\text{app}}{r}{\tau'}{\tau}$ \BY{IH, part 2(b) on \pfref{cvalidR}, \pfref{delta-disjoint} and \pfref{gamma-disjoint}} \pflabel{ruleType}
        \item $\hastypeU{\Delta \cup \Delta_\text{app}}{\Gamma \cup \Gamma_\text{app}}{\aematchwith{n}{e'}{\seqschemaX{r}}}{\tau}$ \BY{Rule (\ref{rule:hastypeUP-match}) on \pfref{hastype} and \pfref{ruleType}}
      \end{pfsteps*}
      \resetpfcounter
    % \item[\text{(\ref{rule:cvalidE-U-splicede})}] ~
    %   \begin{pfsteps*}
    %     \item $\ce=\acesplicede{m}{n}$ \BY{assumption}
    %     \item $\parseUExp{\bsubseq{b}{m}{n}}{\ue}$ \BY{assumption}
    %     \item $\expandsUP{\uDelta}{\uGamma}{\uPsi}{\uPhi}{\ue}{e}{\tau}$ \BY{assumption} \pflabel{expands}
    %   %  \item $\uetsmenv{\Delta_\text{app}}{\Psi}$ \BY{assumption} \pflabel{uetsmenv}
    %     \item $\Delta \cap \Delta_\text{app}=\emptyset$ \BY{assumption} \pflabel{delta-disjoint}
    %     \item $\domof{\Gamma} \cap \domof{\Gamma_\text{app}}=\emptyset$ \BY{assumption} \pflabel{gamma-disjoint}
    %     \item $\hastypeU{\Delta_\text{app}}{\Gamma_\text{app}}{e}{\tau}$ \BY{IH, part 1(a) on \pfref{expands}} \pflabel{hastype}
    %     \item $\hastypeU{\Dcons{\Delta}{\Delta_\text{app}}}{\Gcons{\Gamma}{\Gamma_\text{app}}}{e}{\tau}$ \BY{Lemma \ref{lemma:weakening-UP} over $\Delta$ and $\Gamma$ and exchange on \pfref{hastype}}
    %   \end{pfsteps*}
    %   \resetpfcounter

  \end{byCases}
  \item There is only one case.
  \begin{byCases}
     \item[\text{(\ref{rule:cvalidR-UP})}] ~
      \begin{pfsteps*}
        \item $\crv=\acematchrule{p}{\ce}$ \BY{assumption}
        \item $r=\aematchrule{p}{e}$ \BY{assumption}
        \item $\patType{\pctx}{p}{\tau}$ \BY{assumption} \pflabel{patType}
        \item $\cvalidE{\Delta}{\Gcons{\Gamma}{\pctx}}{\esceneUP{\uDelta}{\uGamma}{\uPsi}{\uPhi}{b}}{\ce}{e}{\tau'}$ \BY{assumption} \pflabel{cvalidE}
        \item $\Delta \cap \Delta_\text{app} = \emptyset$ \BY{assumption}\pflabel{delta-disjoint}
        \item $\domof{\Gamma} \cap \domof{\pctx} = \emptyset$ \BY{identification convention}\pflabel{gamma-disjoint1}
        \item $\domof{\Gamma_\text{app}} \cap \domof{\pctx} = \emptyset$ \BY{identification convention}\pflabel{gamma-disjoint2}
        \item $\domof{\Gamma} \cap \domof{\Gamma_\text{app}} = \emptyset$ \BY{assumption}\pflabel{gamma-disjoint3}
        \item $\domof{\Gcons{\Gamma}{\pctx}} \cap \domof{\Gamma_\text{app}} = \emptyset$ \BY{standard finite set definitions and identities on \pfref{gamma-disjoint1}, \pfref{gamma-disjoint2} and \pfref{gamma-disjoint3}}\pflabel{gamma-disjoint4}
        \item $\hastypeU{\Dcons{\Delta}{\Delta_\text{app}}}{\Gcons{\Gcons{\Gamma}{\pctx}}{\Gamma_\text{app}}}{e}{\tau'}$ \BY{IH, part 2(a) on \pfref{cvalidE}, \pfref{delta-disjoint} and \pfref{gamma-disjoint4}}\pflabel{hastype}
        \item $\hastypeU{\Dcons{\Delta}{\Delta_\text{app}}}{\Gcons{\Gcons{\Gamma}{\Gamma_\text{app}}}{\pctx}}{e}{\tau'}$ \BY{exchange of $\pctx$ and $\Gamma_\text{app}$ on \pfref{hastype}}\pflabel{hastype2}
        \item $\ruleType{\Dcons{\Delta}{\Delta_\text{app}}}{\Gcons{\Gamma}{\Gamma_\text{app}}}{\aematchrule{p}{e}}{\tau}{\tau'}$ \BY{Rule (\ref{rule:ruleType}) on \pfref{patType} and \pfref{hastype2}}
      \end{pfsteps*}
      \resetpfcounter
   \end{byCases} 
\end{enumerate}
\end{enumerate}

The mutual induction can be shown to be well-founded essentially as described in Sec. \ref{sec:SE-metatheory}. Appendix \ref{appendix:metatheory-SES} gives the complete details.
\end{proof}

\subsubsection{Abstract Reasoning Principles}\label{sec:uptsms-abstract-reasoning-principles}
The following theorem summarizes the abstract reasoning principles available to programmers when applying an spTLM. Descriptions of labeled clauses are given inline.
\begingroup
\def\thetheorem{\ref{thm:spTLM-Typing-Segmentation}}
\begin{theorem}[spTLM Abstract Reasoning Principles]
% \label{thm:spTLM-Typing-Segmentation}
If $\patExpands{\upctx}{\uPhi}{\utsmap{\tsmv}{b}}{p}{\tau}$ where $\uDelta=\uDD{\uD}{\Delta}$ and $\uGamma=\uGG{\uG}{\Gamma}$ then all of the following hold:
\begin{enumerate}
        \item (\textbf{Typing 1}) $\uPhi=\uPhi', \uPhyp{\tsmv}{a}{\tau}{\eparse}$ and $\patType{\pctx}{p}{\tau}$
          \begin{quote}
            The final expansion matches values of the type specified by the spTLM's type annotation.
          \end{quote}
        \item $\encodeBody{b}{\ebody}$
        \item $\evalU{\eparse(\ebody)}{\aein{\mathtt{SuccessP}}{\ecand}}$
        \item $\decodeCEPat{\ecand}{\cpv}$
        \item (\textbf{Segmentation}) $\segOK{\segof{\cpv}}{b}$
          \begin{quote}
            The segmentation determined by the proto-expansion actually segments the literal body (i.e. each segment is in-bounds and the segments are non-overlapping.)
          \end{quote}
        \item $\segof{\cpv} = \sseq{\acesplicedt{n'_i}{m'_i}}{\nty} \cup \sseq{\acesplicedp{m_i}{n_i}{\ctau_i}}{\npat}$
        \item (\textbf{Typing 2}) $\sseq{
              \expandsTU{\uDelta}
              {
                \parseUTypF{\bsubseq{b}{m'_i}{n'_i}}
              }{\tau'_i}
            }{\nty}$ and $\sseq{\istypeU{\Delta}{\tau'_i}}{\nty}$
            \begin{quote}
              Each spliced type has a well-formed expansion at the application site.
            \end{quote}
        \item (\textbf{Typing 3}) $\sseq{
          \cvalidT{\emptyset}{
            \tsceneUP
              {\uDelta}{b}
          }{
            \ctau_i
          }{\tau_i}
        }{\npat}$ and $\sseq{\istypeU{\Delta}{\tau_i}}{\npat}$
          \begin{quote}
            Each type annotation on a reference to a spliced pattern has a well-formed expansion at the application site.
          \end{quote}
        \item (\textbf{Typing 4}) $\sseq{
          \patExpands
            {\uGG{\uG_i}{\pctx_i}}
            {\uPhi}
            {\parseUPatF{\bsubseq{b}{m_i}{n_i}}}
            {p_i}
            {\tau_i}
        }{\npat}$  and $\sseq{\patType{\pctx_i}{p_i}{\tau_i}}{\npat}$
          \begin{quote}
            Each spliced pattern has a well-typed expansion that matches values of the type indicated by the corresponding type annotation in the segmentation.
          \end{quote}
      \item (\textbf{No Hidden Bindings}) $\uG = \biguplus_{0 \leq i < \npat} \uG_i$ and $\Gamma = \bigcup_{0 \leq i < \npat} \pctx_i$
        \begin{quote}
          The hypotheses generated by the TLM application are exactly those generated by the spliced patterns.
        \end{quote}
  \end{enumerate}
\end{theorem}
\begin{proof} The proof relies on a lemma about decomposing proto-patterns. The proof is given in Appendix \ref{appendix:SES-reasoning-principles}.\end{proof}
\endgroup


% The following theorem, together with Theorem \ref{thm:typed-pattern-expansion} part 1, establishes \textbf{Typing} and \textbf{Segmentation}, as discussed in Sec. \ref{sec:ptsms-validation}.

% \begingroup
% \def\thetheorem{\ref{thm:spTLM-Typing-Segmentation}}
% \begin{theorem}[spTLM Typing and Segmentation]
% If $\patExpands{\upctx}{\uPhi}{\utsmap{\tsmv}{b}}{p}{\tau}$ then 
% \begin{enumerate}
%         \item (\textbf{Typing}) $\uPhi=\uPhi', \uPhyp{\tsmv}{a}{\tau}{\eparse}$
%         \item $\encodeBody{b}{\ebody}$
%         \item $\evalU{\eparse(\ebody)}{{\lbltxt{SuccessP}}\cdot{\ecand}}$
%         \item $\decodeCEPat{\ecand}{\cpv}$
%         \item (\textbf{Segmentation}) $\segOK{\segof{\cpv}}{b}$
% \end{enumerate}
% \end{theorem}
% \begin{proof} By rule induction over Rules (\ref{rules:patExpands}). The only rule that applies is Rule (\ref{rule:patExpands-apuptsm}). The conclusions are premises of this rule.
% \end{proof}
% \endgroup


%\part[Parametric TLMs]{Parametric TLMs
% ~\\ ~\\ ~\\ 
% \begin{center}
%                      \begin{minipage}[l]{11cm}
%                        \textnormal{\normalsize
% }
%                      \end{minipage}
%                   \end{center}
% }\label{part:parametric-tsms}

% !TEX root = omar-thesis.tex

\chapter{Parametric TLMs (pTLMs)}\label{chap:ptsms}
\begin{quote}\textit{You know me, I gotta put in a big tree.}
\begin{flushright} --- Bob Ross, \emph{The Joy of Painting}\end{flushright}
\end{quote}
% \begin{quote}\textit{The recent development of programming languages suggests that the simul\-taneous achievement of simplicity 
% and generality in language design is a serious unsolved 
% problem.}\begin{flushright}--- John Reynolds (1970) \cite{Reynolds70}\end{flushright}
% \end{quote}
% ~\\
~\\
\noindent
This chapter introduces \emph{parametric TLMs} (pTLMs). Parametric TLMs can be defined over a parameterized family of types, rather than just a single type, and the expansions that they generate can refer to  supplied type and module parameters. 

This chapter is organized like the preceding chapters. We begin in Sec. \ref{sec:parameterized-tsms-by-example} by introducing parametric TLMs by example in VerseML. In particular, we discuss type parameters in Sec. \ref{sec:type-parameters} and module parameters in Sec. \ref{sec:module-parameters}. We then develop a reduced calculus of parametric TLMs, $\miniVerseParam$, in Sec. \ref{sec:miniVerseP}.

\section{Parametric TLMs By Example}\label{sec:parameterized-tsms-by-example}

\subsection{Type Parameters}\label{sec:type-parameters}
Recall from Sec. \ref{sec:lists} the definition of the type-parameterized family of list types:
\begin{lstlisting}[numbers=none]
type list('a) = rec(self => Nil + Cons of 'a * self)
\end{lstlisting}

% \emph{Kinds} classify construction expressions, much like types classify expressions. Types are construction expressions of kind \li{T}, and type constructions are construction expressions of arrow kind. Here, \li{list} takes a single type parameter, so it has arrow kind \li{T -> T}.

% ML dialects commonly define derived syntactic forms for constructing and pattern matching over values of list type. VerseML, in contrast, does not build in derived list forms. Instead, 
Figure \ref{fig:petsm-list} defines a \emph{parametric expression TLM} (peTLM) and a \emph{parametric pattern TLM} (ppTLM), both named \li{#\dolla#list}. These TLMs operate uniformly over this family of types.
\begin{figure}[h]
\begin{lstlisting}
syntax $list('a) at list('a) for expressions by 
  static fn(b : body) : parse_result(proto_expr) => (* ... *)
and for patterns by 
  static fn(b : body) : parse_result(proto_pat) => (* ... *) 
end
\end{lstlisting}
\caption{The type-parameterized \texttt{\$list} TLMs.}
\label{fig:petsm-list}
\end{figure}

Line 1 specifies a single type parameter, \li{'a}. This type parameter appears in the type annotation, which establishes that:
\begin{enumerate}
\item The peTLM \li{#\dolla#list}, when applied to a type \li{T} and a generalized literal form, can only generate expansions of type \li{list(T)}.
\item The ppTLM \li{#\dolla#list}, when applied to a type \li{T} and a generalized literal form, can only generate expansions that match values of type \li{list(T)}.
\end{enumerate}
For example, we can apply \li{#\dolla#list} to \li{int} and a generalized literal form delimited by square brackets as follows:
% val y = $list int [3SURL, EURL4SURL, EURL5]
\begin{lstlisting}[numbers=none]
val x = $list int [xSURL, EURLySURL :: EURLxs]
\end{lstlisting}
The parse function (elided above for concision) segments the literal body into  spliced expressions. The trailing spliced expression is prefixed by two colons (\li{SURL::EURL}), which the TLM takes to mean that it should be the tail of the list. The final expansion of the example above is equivalent to the following when the list value constructors are in scope:
\begin{lstlisting}[numbers=none]
val x = Cons(x, Cons(y, xs))
\end{lstlisting}
As in the preceding chapters, the expansion itself must use the explicit \li{fold} and \li{inj} operators rather than the list value constructors \li{Cons} and \li{Nil} due to the prohibition on context dependence.

\subsection{Module Parameters}\label{sec:module-parameters}
We can finally address the inconvenience of needing to use explicit \li{fold} and \li{inj}  operators by  defining a module-parameterized TLM.

Recall that in Figure \ref{fig:LIST}, we defined a signature \li{LIST} that exported the definition of \li{list} and specified the list value constructors (and some other values.) The definition of \li{#\dolla#list'} shown in Figure \ref{fig:ptsm-listprime} takes modules matching this signature as an additional parameter.

\begin{figure}[h]
\begin{lstlisting}[numbers=none]
syntax $list' (L : LIST) 'a at 'a L.list for expressions by 
  static fn(b : body) : parse_result(proto_expr) => (* ... *)
for patterns by 
  static fn(b : body) : parse_result(proto_pat) => (* ... *)
end
\end{lstlisting}
\caption{The type- and module-parameterized \texttt{\$list'} TLMs}
\label{fig:ptsm-listprime}
\end{figure}
% differing only in that any type parameter that the peTLM specifies can appear free in the generated expansion.

We can apply \li{#\dolla#list'} to the module \li{List} and the type \li{int} as follows:
\begin{lstlisting}[numbers=none]
val y = $list' List int [3SURL, EURL4SURL, EURL5]
val x = $list' List int [1SURL, EURL2SURL :: EURLy]
\end{lstlisting}
The expansion is:
\begin{lstlisting}[numbers=none]
val y = List.Cons(3, List.Cons(4, List.Cons(5, List.Nil)))
val x = List.Cons(1, List.Cons(2, y))
\end{lstlisting}
There is no need to use explicit \li{fold} and \li{inj} operators in this expansion, because the expansion projects the constructors out of the provided module parameter. The TLM itself did not assume that the module would be named \li{List} (internally, the proto-expansion refers to it as \li{L}.)

This makes matters simpler for the TLM provider, but there is a syntactic cost associated with supplying a module parameter at each TLM application site. To reduce this cost, VerseML supports partial parameter application in TLM abbreviations. For example, we can define \li{#\dolla#list} by partially applying \li{#\dolla#list'} as follows:
\begin{lstlisting}[numbers=none]
let syntax $list = $list' List
\end{lstlisting}
(This abbreviates both the expression and pattern TLMs -- sort qualifiers can be added to restrict the abbreviation if desired.)


% Similarly, in lieu of derived list pattern forms, we define the following \emph{parameterized pattern TLM} (ppTLM):
% \begin{lstlisting}[numbers=none]
% syntax $list('a) at list('a) for patterns {
%   static fn(body : Body) : ParseResult(CEPat) => (* ... *)
% }
% \end{lstlisting}
% Again, Line 1 names the ppTLM \li{#\dolla#list} and specifies a single type parameter, \li{'a}. This type parameter appears in the type annotation, which specifies that \li{#\dolla#list}, when apply to a type \li{'a} and a generalized literal form, will only generate patterns that match values of type \li{list('a)}. 

% For example, we can apply the ppTLM \li{#\dolla#list} and the \li{#\dolla#list} to define the polymorphic map function as follows.
% \begin{lstlisting}[numbers=none]
% fun map (f : 'a -> 'b) (x : list('a)) => match x { 
%   $list('a) [] => $list('b) []
% | $list('a) [hdSURL :: EURLtl] => $list('b) [f hdSURL :: EURLmap f tl]
% }
% \end{lstlisting}
% The expansion of this function definition, written textually, is:
% \begin{lstlisting}[numbers=none]
% fun map (f : 'a -> 'b) (x : list('a)) : 'b list => match x { 
%   Nil => Nil
% | Cons(hd, tl) => Cons(f hd, map f tl)
% }
% \end{lstlisting}
% This is somewhat unsatisfying, however, because the expansion is more concise than the unexpanded definition of \li{map}. To further reduce syntactic cost, we can designate \li{#\dolla#list} as the implicit TLM for both expressions and patterns at all types \li{'a list} around our definition of \li{map} as follows.
% \begin{lstlisting}[numbers=none]
% implicit syntax $list('a) in
%   fun map (f : 'a -> 'b) (x : 'a list) : 'b list => match x {
%     [] => []
%   | [hdSURL :: EURLtl] => [f hdSURL :: EURLmap f tl]
%   }
% end
% \end{lstlisting}
% By designating an implicit TLM, we no longer need to explicitly apply \li{#\dolla#list} within expressions in analytic position or patterns.

% When designating an implicit TLM, we assume that free type variables in the type annotation, e.g. here \li{'a}, range over all types. We can make this more explicit by specifying a type parameter explicitly as follows:
% \begin{lstlisting}[numbers=none]
% implicit syntax('a) $list('a) at list('a) in
% 	(* ... *)
% end
% \end{lstlisting}
% All type parameters must appear in the type annotation.
% \subsection{More Examples}
Module parameters also allow us to define TLMs that operate uniformly over module-parameterized families of abstract types. For example, the module-parameterized TLM \texttt{\$r} defined in Figure \ref{fig:param-tsm-r} supports the POSIX regex syntax for any type \li{R.t} where \li{R : RX}. 

\begin{figure}[h]
\begin{lstlisting}
syntax $r(R : RX) at R.t by 
  static fn(b : body) : parse_result(proto_expr) => (* ... *)
end
\end{lstlisting}
\vspace{-8px}
\caption{The module-parameterized TLM \texttt{\$r}}
\label{fig:param-tsm-r}
\vspace{-4px}
\end{figure}
\noindent
For example, given \li{R1 : RX}, we can apply \li{#\dolla#r} as follows:
\begin{lstlisting}[numbers=none]
let dna = $r R1 /SURLA|T|G|CEURL/
\end{lstlisting}
The final expansion of this term is:
\begin{lstlisting}[numbers=none]
let dna = R1.Or(R1.Str "SSTRAESTR", R1.Or(R1.Str "SSTRTESTR", 
	        R1.Or(R1.Str "SSTRGESTR", R1.Str "SSTRCESTR")))
\end{lstlisting}

To be clear: parameters are available to the generated expansion, but they are not available to the parse function that generates the expansion. For example, the following TLM definition is not well-typed because it refers to \li{M} from within the parse function:
\begin{lstlisting}[numbers=none]
syntax $badM(M : A) at T by 
  static fn(b : body) => let x = M.x in (* ... *)
end
\end{lstlisting}
(In the next chapter, we will define a mechanism that gives parse functions access to a common static environment.)
% \subsubsection{Queues}
% Consider the following signature for working with persistent queues:
% \begin{lstlisting}[numbers=none]
% signature QUEUE = sig
%   type queue('a)
%   val empty  : queue('a)
%   val insert : 'a * queue('a) -> queue('a)
%   val remove : queue('a) -> option('a * queue('a))
% end 
% \end{lstlisting}
% Structures that match this signature must define a type constructor \li{queue} of kind \li{T -> T} and three values -- \li{empty} introduces the empty queue, \li{insert} inserts a value onto the back of a queue, and \li{remove} removes the element at the front of the queue and returns it and the remaining queue, or \li{None} if the queue is empty.%one for inserting an item into a queue, and one for removing a value from a queue.

% There are many possible structures that implement this signature. For example, we can define a structure \li{ListQueue} that represents queues internally as lists, where the head of the list is the back of the queue. With this representation, \li{insert} is a constant time operation, but \li{remove} is a linear time operation. Alternatively, we might define a structure \li{TwoListQueue} that represents queues internally as a pair of lists, maintaining the invariant that one is the reverse of the other, so that both \li{insert} and \li{remove} are constant time operations (see \cite{harper1997programming} for the details of this and other possibilities). 

% Regardless of the implementation that the client chooses, we would like for the client to be able to introduce queues more naturally and at lower syntactic cost than is possible by directly applying the functions specified by the signature above. In VerseML, we can give clients of structures matching the signature \li{QUEUE} this ability by defining the following parameterized expression TLM:
% \begin{lstlisting}[numbers=none]
% syntax $queue(Q : QUEUE)('a) at Q.queue('a) {
%   static fn(body : Body) : ParseResult(CEExp) => (* ... *)
% }
% \end{lstlisting}
% This peTLM specifies one module parameter, \li{Q}, which must match the signature \li{QUEUE}, and one type parameter, \li{'a} (implicitly of kind \li{T}). These appear in the type annotation, which specifies that expansions that arise from applying \li{#\dolla#queue} to a module \li{Q : QUEUE} and a type \li{'a} will be of type \li{Q.queue('a)}. For example:
% \begin{lstlisting}
% val q = $queue TwoListQueue int [SURL> EURL1SURL, EURL2SURL, EURL3]
% val q' = $queue TwoListQueue int [qSURL > EURL4SURL, EURL5]
% \end{lstlisting}
% On Line 1, the initial angle bracket (\li{SURL>EURL}) indicates that the items are inserted in left-to-right order. The items in the queue are given as spliced subexpressions separated by commas. Line 2 inserts two additional items onto the back of the queue \li{q}. The expansion of this example, written textually, is:
% \begin{lstlisting}
% val q : TwoListQueue.queue(int) = 
%   TwoListQueue.insert(1, 
%     TwoListQueue.insert(2, 
%       TwoListQueue.insert(3, 
%         TwoListQueue.empty)))
% val q' : TwoListQueue.queue(int) = 
%   TwoListQueue.insert(4, TwoListQueue.insert(5, q))
% \end{lstlisting}
% Notice that the expansion can refer to the module parameter \li{TwoListQueue}.

% We can further reduce syntactic cost by defining a synonym for the partial application of \li{#\dolla#queue} to the module parameter \li{TwoListQueue}:
% \begin{lstlisting}[numbers=none]
% syntax $tlq = $queue TwoListQueue
% val q = $tlq int [SURL> EURL1SURL, EURL2SURL, EURL3]
% \end{lstlisting}
% We can further define a synonym for the partial application of \li{#\dolla#tlq} to a type parameter:
% \begin{lstlisting}[numbers=none]
% syntax $tlqi = $tlq int (* = $queue TwoListQueue int *)
% val q' = $tlqi [qSURL > EURL4SURL, EURL5]
% \end{lstlisting}
% \subsection{Module Parameters}
% VerseML also provides a module language based on the Standard ML module language \cite{MacQueen:1984:MSM:800055.802036}. The module language consists of \emph{module expressions} classified by \emph{signatures}. %Signatures specify type components, which may be opaque or transparent, value components, and module components.h

% %In Sec. \ref{sec:motivating-examples}, we gave several examples of signatures and discussed how one might introduce derived forms that  across a module-parameterized family of types.



% Another way to reduce syntactic cost is by designating \li{#\dolla#queue Q 'a} the implicit TLM at all types of the form \li{Q.queue('a)} where \li{Q : QUEUE}. This is written as follows:
% \begin{lstlisting}[numbers=none]
% implicit syntax (Q : QUEUE) ('a) => $queue Q 'a in
%   val q : TwoListQueue.queue(int) = [SURL> EURL1SURL, EURL2SURL, EURL3]
%   val q' : TwoListQueue.queue(int) = [qSURL > EURL4SURL, EURL5]
% end
% \end{lstlisting}
% This designation is particularly useful for clients who need to construct a queue as an argument to a function. For example, consider a function 
% \begin{lstlisting}[numbers=none]
% enqueue_jobs : Q.queue(Job) -> Ticket
% \end{lstlisting}
% for some module \li{Q : QUEUE} and types \li{Job} and \li{Ticket}. We can enqueue a sequence of jobs \li{j1} through \li{j4} under the TLM designation above as follows:
% \begin{lstlisting}[numbers=none]
% enqueue_jobs [SURL> EURLj1SURL, EURLj2SURL, EURLj3SURL, EURLj4]
% \end{lstlisting}

\vspace{-5px}
\section{\texorpdfstring{$\miniVerseParam$}{miniVerseP}}\label{sec:miniVerseP}
We will now define a reduced dialect of VerseML called $\miniVerseParam$ that supports parametric expression and pattern TLMs (peTLMs and ppTLMs.) This language, like $\miniVersePat$, consists of an unexpanded language (UL) defined by typed expansion to an expanded language (XL). The full definition of $\miniVerseParam$ is given in Appendix \ref{appendix:miniVerseParam} -- we will detail only  particularly interesting constructs below.

\subsection{Syntax of the Expanded Language (XL)}\label{sec:P-expanded-terms}

\begin{figure}[p] 
\[\begin{array}{lllllll}
\textbf{Sort} & & & \textbf{Operational Form} 
%& \textbf{Stylized Form} 
& \textbf{Description}\\
\mathsf{Sig} & \sigma & ::= & \asignature{\kappa}{u}{\tau} 
%& \signature{u}{\kappa}{\tau} 
& \text{signature}\\
\mathsf{Mod} & M & ::= & X 
%& X 
& \text{module variable}\\
&&& \astruct{c}{e} 
%& \struct{c}{e} 
& \text{structure}\\
&&& \aseal{\sigma}{M} 
%& \seal{M}{\sigma} 
& \text{seal}\\
&&& \amlet{\sigma}{M}{X}{M} %& \mlet{X}{M}{M}{\sigma} 
& \text{definition}
\end{array}\]
\caption[Syntax of signatures and module expressions in $\miniVerseParam$]{Syntax of signatures and module expressions in $\miniVerseParam$}
\label{fig:P-modules-signatures}
\end{figure}


\begin{figure}[p] 
\[\begin{array}{lrlllll}
\textbf{Sort} & & & \textbf{Operational Form} 
%& \textbf{Stylized Form} 
& \textbf{Description}\\
\mathsf{Kind} & \kappa & ::= & k & \text{kind variable}\\
&&& \akdarr{\kappa}{u}{\kappa} 
%& \kdarr{u}{\kappa}{\kappa} 
& \text{dependent function}\\
&&& \akunit 
%& \kunit 
& \text{nullary product}\\
&&& \akdbprod{\kappa}{u}{\kappa} 
%& \kdbprod{u}{\kappa}{\kappa} 
& \text{dependent product}\\
%&&& \akdprodstd & \kdprodstd & \text{labeled dependent product}\\
&&& \akty 
%& \kty
& \text{type}\\
&&& \aksing{\tau} 
%& \ksing{\tau} 
& \text{singleton}\\
\mathsf{Con} & c, \tau & ::= & u 
%& u 
& \text{construction variable}\\
&&& t 
%& t 
& \text{type variable}
\\
&&& \acabs{u}{c} 
%& \cabs{u}{c} 
& \text{abstraction}\\
&&& \acapp{c}{c} 
%& \capp{c}{c} 
& \text{application}\\
&&& \actriv 
%& \ctriv 
& \text{trivial}\\
&&& \acpair{c}{c}
% & \cpair{c}{c} 
& \text{pair}\\
&&& \acprl{c} 
%& \cprl{c} 
& \text{left projection}\\
&&& \acprr{c} 
%& \cprr{c} 
& \text{right projection}\\
%&&& \adtplX & \dtplX & \text{labeled dependent tuple}\\
%&&& \adprj{\ell}{c} & \prj{c}{\ell} & \text{projection}\\
&&& \aparr{\tau}{\tau} 
%& \parr{\tau}{\tau} 
& \text{partial function}\\
&&& \aallu{\kappa}{u}{\tau} 
%& \forallu{u}{\kappa}{\tau} 
& \text{polymorphic}\\
&&& \arec{t}{\tau} 
%& \rect{t}{\tau} 
& \text{recursive}\\
&&& \aprod{\labelset}{\mapschema{\tau}{i}{\labelset}} 
%& \prodt{\mapschema{\tau}{i}{\labelset}} 
& \text{labeled product}\\
&&& \asum{\labelset}{\mapschema{\tau}{i}{\labelset}} 
%& \sumt{\mapschema{\tau}{i}{\labelset}} 
& \text{labeled sum}\\
&&& \amcon{M} 
%& \mcon{M} 
& \text{construction component}
\end{array}\]
\caption[Syntax of kinds and constructions in $\miniVerseParam$]{Syntax of kinds and constructions in $\miniVerseParam$. By convention, we choose the metavariable $\tau$ for constructions that, in well-formed terms, must necessarily be of kind $\kty$, and the metavariable $c$ otherwise. Similarly, we use construction variables $t$ to stand for constructions of kind $\kty$, and construction variables $u$ otherwise. Kind variables, $k$, are necessary only for the metatheory.}
\label{fig:P-kinds-constructors}
\end{figure}

\begin{figure}
\[\begin{array}{lllllll}
\textbf{Sort} & & & \textbf{Operational Form} 
%& \textbf{Stylized Form} 
& \textbf{Description}\\
\mathsf{Exp} & e & ::= & x 
%& x 
& \text{variable}\\
&&& \aelam{\tau}{x}{e} 
%& \lam{x}{\tau}{e} 
& \text{abstraction}\\
&&& \aeap{e}{e} 
%& \ap{e}{e} 
& \text{application}\\
&&& \aeclam{\kappa}{u}{e} %& \clam{u}{\kappa}{e} 
& \text{construction abstraction}\\
&&& \aecap{e}{c} %& \cAp{e}{c} 
& \text{construction application}\\
&&& \aefold{e} %& \fold{e} 
& \text{fold}\\
&&& \aeunfold{e} %& \unfold{e} 
& \text{unfold}\\
&&& \aetpl{\labelset}{\mapschema{e}{i}{\labelset}} 
%& \tpl{\mapschema{e}{i}{\labelset}} 
& \text{labeled tuple}\\
&&& \aepr{\ell}{e} 
%& \prj{e}{\ell} 
& \text{projection}\\
&&& \aein{\ell}{e} 
%& \inj{\ell}{e} 
& \text{injection}\\
&&& \aematchwith{n}{e}{\seqschemaX{r}} 
%& \matchwith{e}{\seqschemaX{r}} 
& \text{match}\\
&&& \amval{M} 
%& \mval{M} 
& \text{value component}\\
\mathsf{Rule} & r & ::= & \aematchrule{p}{e} 
%& \matchrule{p}{e} 
& \text{rule}\\
\mathsf{Pat} & p & ::= & x 
%& x 
& \text{variable pattern}\\
&&& \aewildp 
%& \wildp 
& \text{wildcard pattern}\\
&&& \aefoldp{p} 
%& \foldp{p} 
& \text{fold pattern}\\
&&& \aetplp{\labelset}{\mapschema{p}{i}{\labelset}} 
%& \tplp{\mapschema{p}{i}{\labelset}} 
& \text{labeled tuple pattern}\\
&&& \aeinjp{\ell}{p} 
%& \injp{\ell}{p} 
& \text{injection pattern}
\end{array}\]
\caption[Syntax of expanded expressions, rules and patterns in $\miniVerseParam$]{Syntax of expanded expressions, rules and patterns in $\miniVerseParam$}
\label{fig:P-expanded-terms}
\end{figure}


Figure \ref{fig:P-modules-signatures} defines the syntax of the \emph{expanded module language}. Figure \ref{fig:P-kinds-constructors} defines the syntax of the \emph{expanded type construction language}. Figure \ref{fig:P-expanded-terms} defines the syntax of the \emph{expanded expression language}.


\subsection{Statics of the Expanded Language}
The module and type construction languages are based closely on those defined by Harper in \emph{PFPL} \cite{pfpl}. These languages, in turn, are based on the languages developed by Lee et al. \cite{conf/popl/LeeCH07}, and also by Dreyer \cite{dreyer2005understanding}. All of these incorporate Stone and Harper's \emph{dependent singleton kinds} formalism to track type identity \cite{stone2006extensional}. The expression language is similar to that of $\miniVersePat$, defined in Chapter \ref{chap:uptsms}.

The \emph{statics of the expanded language} is defined by a collection of judgements that we organize into three groups. 

The first group of judgements, which we refer to as the \emph{statics of the expanded module language}, define the statics of expanded signatures and module expressions.

\vspace{5px}
$\begin{array}{ll}
\textbf{Judgement Form} & \textbf{Description}\\
\issigX{\sigma} & \text{$\sigma$ is a signature }\\
\sigequalX{\sigma}{\sigma'} & \text{$\sigma$ and $\sigma'$ are definitionally equal signatures}\\
\sigsubX{\sigma}{\sigma'} & \text{$\sigma$ is a sub-signature of $\sigma'$}\\
\hassigX{M}{\sigma} & \text{$M$ matches $\sigma$}\\
\ismvalX{M} & \text{$M$ is, or stands for, a module value}
\end{array}$
\vspace{5px}

The second group of judgements, which we refer to as the \emph{statics of the expanded type construction language}, define the statics of expanded kinds and constructions.

\vspace{5px}
$\begin{array}{ll}
\textbf{Judgement Form} & \textbf{Description}\\
\iskindX{\kappa} & \text{$\kappa$ is a kind}\\
\kequalX{\kappa}{\kappa'} & \text{$\kappa$ and $\kappa'$ are definitionally equal kinds}\\
\ksubX{\kappa}{\kappa'} & \text{$\kappa$ is a subkind of $\kappa'$}\\
\haskindX{c}{\kappa} & \text{$c$ has kind $\kappa$}\\
\cequalX{c}{c'}{\kappa} & \text{$c$ and $c'$ are equivalent as constructions of kind $\kappa$}
\end{array}$
\vspace{5px}

The third group of judgements, which we refer to as the \emph{statics of the expanded expression language}, define the statics of types, expanded expressions, rules and patterns. Types are constructions of kind $\akty$. We use the metavariable $\tau$ rather than $c$ for types.

\vspace{5px}
$\begin{array}{ll}
\textbf{Judgement Form} & \textbf{Description}\\
% \istypeP{\Omega}{\tau} & \text{$\tau$ is a well-formed type}\\
% \tequalPX{\tau}{\tau'} & \text{$\tau$ and $\tau'$ are definitionally equal types}\\
\issubtypePX{\tau}{\tau'} & \text{$\tau$ is a subtype of $\tau'$}\\
\hastypeP{\Omega}{e}{\tau} & \text{$e$ is assigned type $\tau$}\\
\ruleTypeP{\Omega}{r}{\tau}{\tau'} & \text{$r$ takes values of type $\tau$ to values of type $\tau'$}\\
\patTypeP{\Omega'}{p}{\tau} & \text{$p$ matches values of type $\tau$ and generates hypotheses $\Omega'$} 
\end{array}$
\vspace{5px}


A \emph{unified context}, $\Omega$, is a finite function over module, expression and construction variables. 
We write
\begin{itemize}
\item $\Omega, X : \sigma$ when $X \notin \domof{\Omega}$ and $\issigX{\sigma}$ for the extension of $\Omega$ with a mapping from $X$ to the hypothesis $X : \sigma$.
\item $\Omega, x : \tau$ when $x \notin \domof{\Omega}$ and $\haskindX{\tau}{\akty}$ for the extension of $\Omega$ with a mapping from $x$ to the hypothesis $x : \tau$
\item $\Omega, u :: \kappa$ when $u \notin \domof{\Omega}$ and $\iskindX{\kappa}$ for the extension of $\Omega$ with a mapping from $u$ to the hypothesis $u :: \kappa$
\end{itemize}
A well-formed unified context is one that can be constructed by some sequence of such extensions, starting from the empty context, $\emptyset$. We identify unified contexts up to exchange and contraction in the usual manner.

The complete set of rules is given in Appendix \ref{appendix:P-statics}. A comprehensive introductory account of these constructs is beyond the scope of this work (see \cite{pfpl}.) Instead, let us summarize the key features of the expanded language by example. 

Modules take the form $\astruct{c}{e}$, following a \emph{phase-splitting} approach -- the construction components of the module are ``tupled'' into a single construction component, $c$, and the value components of the module are ``tupled'' into a single value component, $e$ \cite{harper1989higher}. Signatures, $\sigma$, are also split in this way -- a single \emph{kind}, $\kappa$,  classifies the construction component and a single type, $\tau$, classifies the value component of the classified module. The type can refer to the construction component through a mediating construction variable, $u$. The key rule is reproduced below:
\begin{equation*}\tag{\ref{rule:hassig-struct}}
\inferrule{
  \haskindX{c}{\kappa}\\
  \hastypeP{\Omega}{e}{[c/u]\tau}
}{
  \hassigX{\astruct{c}{e}}{\asignature{\kappa}{u}{\tau}}
}
\end{equation*}

For example, consider the VerseML signature and the corresponding $\miniVerseParam$ signature in Figure \ref{fig:corresponding-signatures}. The kind on the right (Lines 1-3) is a \emph{dependent product kind} and the type (Lines 4-5) is a product type. Let us consider these in turn.

\begin{figure}
\begin{minipage}{0.35\textwidth}
\begin{lstlisting}
sig
  type t
  type t' = t * t
  val x : t
  val y : t -> t'
end
\end{lstlisting}
\end{minipage}
\begin{minipage}{0.5\textwidth}\vspace{3px}
{\footnotesize\[
\begin{array}{l}
\asignature{\akdbprod{\\
\quad\quad \akty}{t}{\\
\quad\quad \aksing{
  \aprod{\lbltxt{1}; \lbltxt{2}}{
    \mapitem{\lbltxt{1}}{t}; \mapitem{\lbltxt{2}}{t}
  }
}}\\}{u}{
 \aprod{\lbltxt{x}; \lbltxt{y}}{
  \mapitem{\lbltxt{x}}{\acprl{u}}; \\ 
\quad\quad \mapitem{\lbltxt{y}}{\aparr{
    \acprl{u}
  }{
    \acprr{u}
  }}
}\\
}
\end{array}
\]}
\end{minipage}
\caption{A VerseML signature and the corresponding $\miniVerseParam$ signature}
\label{fig:corresponding-signatures}
\vspace{-10px}
\end{figure}



On Lines 2-3 (left), we specified an abstract type component \li{t}, and then a translucent type component \li{t'} equal to \li{t * t}. Abstract type components have kind $\akty$, so the first component of the dependent product kind is $\akty$ (Line 2, right). The construction variable $t$ stands for the first component in the second component of the dependent product kind. The second component is not held abstract, so it is classified by a corresponding \emph{singleton kind}, rather than by the kind $\akty$ (Line 3, right). A singleton kind $\aksing{\tau}$ classifies only those types definitionally equal to $\tau$. A subkinding system is necessary to ensure that constructions of singleton kind can appear where a construction of kind $\akty$ is needed -- the key rule is reproduced below:
\begin{equation*}\tag{\ref{rule:ksub-sing}}
\inferrule{
  \haskindX{\tau}{\akty}
}{
  \ksubX{\aksing{\tau}}{\akty}
}
\end{equation*}

Lines 4-5 (right) define a product type that classifies the value component of matching modules. The construction variable \li{u} stands for the construction component of the matching module. The left- and right-projection operations $\acprl{c}$ and $\acprr{c}$ on the right correspond to \li{t} and \li{t'} on the left. (In practice, we would use labeled dependent product kinds, but for simplicity, we stick to binary dependent product kinds here.)

Consider another example: the VerseML \li{LIST} signature from Figure \ref{fig:LIST}, partially reproduced below:
\begin{lstlisting}
sig 
  type list('a) = rec(self => Nil + Cons of 'a * self)
  val Nil : list('a)
  val Cons : 'a * list('a) -> list('a)
  (* ... *)
end
\end{lstlisting}
This signature corresponds to the $\miniVerseParam$ signature $\sigma_\texttt{LIST}$ defined in Figure \ref{fig:LIST-mini}. 

\begin{figure}
\[
\arraycolsep=1px\begin{array}{ll}

\sigma_\texttt{LIST} & \defeq \asignature{\kappa_\texttt{LIST}}{list}{\tau_\texttt{LIST}}\\
\kappa_\texttt{LIST} & \defeq \akdarr{\akty}{\alpha}{\aksing{
  \arec{self}{
    \asum{L_\texttt{list}}{\\
    & \quad\quad 
      \mapitem{\lbltxt{Nil}}{\aprod{}{}}; \\
    & \quad\quad 
      \mapitem{\lbltxt{Cons}}{
        \aprod{\lbltxt{1}; \lbltxt{2}}{
          \mapitem{\lbltxt{1}}{\alpha}; 
          \mapitem{\lbltxt{2}}{self}
        }
      }
    }
  } 
}}\\
L_\texttt{list} & \defeq \lbltxt{Nil}, \lbltxt{Cons}\\
\tau_\texttt{LIST} & \defeq \aprod{L_\texttt{list}}{\\&
  \quad\quad \mapitem{\lbltxt{Nil}}{
    \aallu{\akty}{\alpha}{\acapp{list}{\alpha}}
  }; \\&
  \quad\quad \mapitem{\lbltxt{Cons}}{
    \aallu{\akty}{\alpha}{
      \aparr{\\&\quad\quad\quad
        \aprod{\lbltxt{1}; \lbltxt{2}}{
          \mapitem{\lbltxt{1}}{\alpha}; 
          \mapitem{\lbltxt{2}}{\acapp{list}{\alpha}}
        }
      }{\\&\quad\quad\quad
        \acapp{list}{\alpha}
      }
    }
  }
}
\end{array}
\]
\caption{The $\miniVerseParam$ encoding of the \texttt{LIST} signature}
\label{fig:LIST-mini}
\end{figure}
Here, the signature specifies only a single construction component, so no tupling of the construction component is necessary. This single construction component is a type function, so it has dependent function kind: the argument kind is $\akty$ and the return kind is a singleton kind, because the type function is not abstract. (Had we held the type function abstract, its kind would instead be $\akdarr{\akty}{\_}{\akty}$.)
%A well-formed unified inner context is one where there are no cycles in the dependency graph between the hypotheses (constructed in the obvious manner) and for each hypothesis, the construction, kinds or signature involved is well-formed relative to the unified inner context.

At the top level, a program consists of a module expression, $M$. The module let binding form allows the programmer to bind a module to a module variable, $X$:
\begin{equation*}\tag{\ref{rule:hassig-let}}
\inferrule{
  \hassigX{M}{\sigma}\\
  \issigX{\sigma'}\\
  \hassig{\Omega, X : \sigma}{M'}{\sigma'}  
}{
  \hassigX{\amlet{\sigma'}{M}{X}{M'}}{\sigma'}
}
\end{equation*}

The construction projection form, $\amcon{M}$, allows us to refer to the construction component of $M$ within a construction appearing in $M'$. The kinding rule for this form is reproduced below:
\begin{equation*}\tag{\ref{rule:haskind-stat}}
\inferrule{
  \ismvalX{M}\\
  \hassigX{M}{\asignature{\kappa}{u}{\tau}}
}{
  \haskindX{\amcon{M}}{\kappa}
}
\end{equation*}
Similarly, the value projection form, $\amval{M}$, projects out the value component of $M$ within an expression appearing in $M'$. The typing rule for this form is reproduced below:
\begin{equation*}\tag{\ref{rule:hastypeP-dyn}}
\inferrule{
  \ismvalX{M}\\
  \hassigX{M}{\asignature{\kappa}{u}{\tau}}
}{
  \hastypeP{\Omega}{\amval{M}}{[\amcon{M}/u]\tau}
}
\end{equation*}
The first premise of both of these rules requires that $M$ be, or stand for, a \emph{module value}, according to the following rules:
\begin{equation*}\tag{\ref{rule:ismval-struct}}
\inferrule{ }{
  \ismvalX{\astruct{c}{e}}
}
\end{equation*}
\begin{equation*}\tag{\ref{rule:ismval-var}}
\inferrule{ }{
  \ismval{\Omega, X : \sigma}{X}
}
\end{equation*}
The reason for this restriction has to do with the \emph{sealing} operation:
\begin{equation*}\tag{\ref{rule:hassig-seal}}
\inferrule{
  \issigX{\sigma}\\
  \hassigX{M}{\sigma}
}{
  \hassigX{\aseal{\sigma}{M}}{\sigma}
}
\end{equation*}
Sealing enforces \emph{representation independence} -- the abstract construction components of a sealed module are not treated as equivalent to those of any other sealed module within the program. In other words, sealing is \emph{generative}. The module value restriction above achieves this behavior by simple syntactic means -- a sealed module is not a module value, so all sealed modules have to be bound to distinct module variables.

The judgements above obey standard lemmas, including Weakening, Substitution and Decomposition (see Appendix \ref{appendix:P-statics}.)

We omit certain features of the ML module system in  $\miniVerseParam$, such as its support for hierarchical modules and functors. Our formulation also does not support ``width'' subtyping and subkinding for simplicity. These are straightforward extensions of $\miniVerseParam$, but because their inclusion would not change the semantics of parametric TLMs, we did not include them (see \cite{pfpl} for a discussion of these features.)

\subsection{Structural Dynamics}
The structural dynamics of modules is defined as a transition system, and is organized around judgements of the following form:

\vspace{10px}
$\begin{array}{ll}
\textbf{Judgement Form} & \textbf{Description}\\
\stepsU{M}{M'} & \text{$M$ transitions to $M'$}\\
\isvalP{M} & \text{$M$ is a module value}\\
\matchfail{M} & \text{$M$ raises match failure}
\end{array}$
\vspace{10px}

The structural dynamics of expressions is also defined as a transition system, and is organized around judgements of the following form:

\vspace{10px}
$\begin{array}{ll}
\textbf{Judgement Form} & \textbf{Description}\\
\stepsU{e}{e'} & \text{$e$ transitions to $e'$}\\
\isvalP{e} & \text{$e$ is a value}\\
\matchfail{e} & \text{$e$ raises match failure}
\end{array}$
\vspace{10px}

We also define auxiliary judgements for \emph{iterated transition}, $\multistepU{e}{e'}$, and \emph{evaluation}, $\evalU{e}{e'}$, of expressions.

\begingroup
\def\thetheorem{\ref{defn:iterated-transition-P}}
\begin{definition}[Iterated Transition] Iterated transition, $\multistepU{e}{e'}$, is the reflexive, transitive closure of the transition judgement, $\stepsU{e}{e'}$.\end{definition}
% \addtocounter{theorem}{-1}
\endgroup

\begingroup
\def\thetheorem{\ref{defn:evaluation-P}}
\begin{definition}[Evaluation] $\evalU{e}{e'}$ iff $\multistepU{e}{e'}$ and $\isvalU{e'}$. \end{definition}
% \addtocounter{theorem}{-1}
\endgroup

As in previous chapters, our subsequent developments do not make mention of particular rules in the dynamics, so we do not produce these details here. Instead, it suffices to state the following conditions.

The Preservation condition ensures that evaluation preserves typing.
\begingroup
\def\thetheorem{\ref{condition:preservation-P}}
\begin{condition}[Preservation] ~
\begin{enumerate}
\item If $\hassig{}{M}{\sigma}$ and $\stepsU{M}{M'}$ then $\hassig{}{M}{\sigma}$.
\item If $\hastypeUC{e}{\tau}$ and $\stepsU{e}{e'}$ then $\hastypeUC{e'}{\tau}$.
\end{enumerate}
\end{condition}
\endgroup

The Progress condition ensures that evaluation of a well-typed expanded expression cannot ``get stuck''. We must consider the possibility of match failure in this condition.
\begingroup
\def\thetheorem{\ref{condition:progress-P}}
\begin{condition}[Progress] ~
\begin{enumerate}
\item If $\hassig{}{M}{\sigma}$ then either $\isvalU{M}$ or $\matchfail{M}$ or there exists an $M'$ such that $\stepsU{M}{M'}$.
\item If $\hastypeUC{e}{\tau}$ then either $\isvalU{e}$ or $\matchfail{e}$ or there exists an $e'$ such that $\stepsU{e}{e'}$.
\end{enumerate}
\end{condition}
% \addtocounter{theorem}{-1}
\endgroup

Together, these two conditions constitute the Type Safety Condition.

\begin{figure}[p] \vspace{-15px}
$\arraycolsep=4pt\begin{array}{lllllll}
\textbf{Sort} & & 
%& \textbf{Operational Form} 
& \textbf{Stylized Form} & \textbf{Description}\\
\mathsf{USig} & \usigma & ::= 
%& \ausignature{\ukappa}{\uu}{\utau} 
& \signature{\uu}{\ukappa}{\utau} & \text{signature}\\
\mathsf{UMod} & \uM & ::= 
%& \uX 
& \uX & \text{module identifier}\\
&&
%& \austruct{\uc}{\ue} 
& \struct{\uc}{\ue} & \text{structure}\\
&&
%& \auseal{\usigma}{\uM} 
& \seal{\uM}{\usigma} & \text{seal}\\
&&
%& \aumlet{\usigma}{\uM}{\uX}{\uM} 
& \mlet{\uX}{\uM}{\uM}{\usigma} & \text{definition}\\
\LCC &&
%& \lightgray 
& \color{Yellow} & \color{Yellow}\\
&&
%& \aumdefpetsm{\urho}{e}{\tsmv}{\uM} 
& \defpetsm{\tsmv}{\urho}{e}{\uM} & \text{peTLM definition}\\
%&&&                                    & \texttt{expressions}~\{e\}~\texttt{in}~\uM\\
&&
%& \aumletpetsm{\uepsilon}{\tsmv}{\uM} 
& \uletpetsm{\tsmv}{\uepsilon}{\uM} & \text{peTLM binding}\\
% &&&                                  & \texttt{expressions}~\texttt{in}~\uM\\
% &&& ... & ... & \text{peTLM designation}\\
&&
%& \audefpptsm{\urho}{e}{\tsmv}{\uM} 
& \defpptsm{\tsmv}{\urho}{e}{\uM} & \text{ppTLM definition}\\
% &&&                                    & \texttt{patterns}~\{e\}~\texttt{in}~\uM\\
&&
%& \auletpptsm{\uepsilon}{\tsmv}{\uM} 
& \uletpptsm{\tsmv}{\uepsilon}{\uM} & \text{ppTLM binding}\ECC%
% &&& & \texttt{patterns}~\texttt{in}~\uM\\
% &&& ... & ... & \text{ppTLM designation}\ECC
\end{array}$%\vspace{-5px}
\caption[Syntax of unexpanded module expressions and signatures in $\miniVerseParam$]{Syntax of unexpanded module expressions and signatures in $\miniVerseParam$}%\vspace{-5px}
\label{fig:P-unexpanded-modules-signatures}
\end{figure}
\begin{figure}[p] \vspace{-10px}
\[\begin{array}{lrlllll}
\textbf{Sort} & & 
%& \textbf{Operational Form} 
& \textbf{Stylized Form} & \textbf{Description}\\
\mathsf{UKind} & \ukappa & ::= 
%& \aukdarr{\ukappa}{\uu}{\ukappa} 
& \kdarr{\uu}{\ukappa}{\ukappa} & \text{dependent function}\\
&&
%& \aukunit 
& \kunit & \text{nullary product}\\
&&
%& \aukdbprod{\ukappa}{\uu}{\ukappa} 
& \kdbprod{\uu}{\ukappa}{\ukappa} & \text{dependent product}\\
%&&& \akdprodstd & \kdprodstd & \text{labeled dependent product}\\
&&
%& \aukty 
& \kty & \text{type}\\
&&
%& \auksing{\utau} 
& \ksing{\utau} & \text{singleton}\\
\mathsf{UCon} & \uc, \utau & ::= 
%& \uu 
& \uu & \text{construction identifier}\\
&&
%& \ut 
& \ut & \\
&&
%& \aucasc{\ukappa}{\uc} 
& \casc{\uc}{\ukappa} & \text{ascription}\\
&&
%& \aucabs{\uu}{\uc} 
& \cabs{\uu}{\uc} & \text{abstraction}\\
&&
%& \aucapp{c}{c} 
& \capp{c}{c} & \text{application}\\
&&
%& \auctriv 
& \ctriv & \text{trivial}\\
&&
%& \aucpair{\uc}{\uc} 
& \cpair{\uc}{\uc} & \text{pair}\\
&&
%& \aucprl{\uc} 
& \cprl{\uc} & \text{left projection}\\
&&
%& \aucprr{\uc} 
& \cprr{\uc} & \text{right projection}\\
%&&& \adtplX & \dtplX & \text{labeled dependent tuple}\\
%&&& \adprj{\ell}{c} & \prj{c}{\ell} & \text{projection}\\
&&
%& \auparr{\utau}{\utau} 
& \parr{\utau}{\utau} & \text{partial function}\\
&&
%& \auallu{\ukappa}{\uu}{\utau} 
& \forallu{\uu}{\ukappa}{\utau} & \text{polymorphic}\\
&&
%& \aurec{\ut}{\utau} 
& \rect{\ut}{\utau} & \text{recursive}\\
&&
%& \auprod{\labelset}{\mapschema{\utau}{i}{\labelset}} 
& \prodt{\mapschema{\utau}{i}{\labelset}} & \text{labeled product}\\
&&
%& \ausum{\labelset}{\mapschema{\utau}{i}{\labelset}} 
& \sumt{\mapschema{\utau}{i}{\labelset}} & \text{labeled sum}\\
&&
%& \aumcon{\uX} 
& \mcon{\uX} & \text{construction component}
\end{array}\]%\vspace{-5px}
\caption[Syntax of unexpanded kinds and constructions in $\miniVerseParam$]{Syntax of unexpanded kinds and constructions in $\miniVerseParam$}\vspace{-10px}
\label{fig:P-unexpanded-kinds-constructors}
\end{figure}

% \clearpage
\begin{figure}[p]
\[\begin{array}{lllllll}
\textbf{Sort} & & 
%& \textbf{Operational Form} 
& \textbf{Stylized Form} & \textbf{Description}\\
\mathsf{UExp} & \ue & ::= 
%& \ux 
& \ux & \text{identifier}\\
&&
% & \auasc{\utau}{\ue} 
& \asc{\ue}{\utau} & \text{ascription}\\
&&
% & \auletsyn{\ux}{\ue}{\ue} 
& \letsyn{\ux}{\ue}{\ue} & \text{value binding}\\
% &&
%& \auanalam{\ux}{\ue} 
% & \analam{\ux}{\ue} & \text{abstraction (unannotated)}\\
&&
%& \aulam{\utau}{\ux}{\ue} 
& \lam{\ux}{\utau}{\ue} & \text{abstraction}\\
&&
%& \auap{\ue}{\ue} 
& \ap{\ue}{\ue} & \text{application}\\
&&
%& \auclam{\ukappa}{\uu}{\ue} 
& \clam{\uu}{\ukappa}{\ue} & \text{construction abstraction}\\
&&
%& \aucap{\ue}{\uc} 
& \cAp{\ue}{\uc} & \text{construction application}\\
&&
%& \auanafold{\ue} 
& \fold{\ue} & \text{fold}\\
&&
%& \auunfold{\ue} 
& \unfold{\ue} & \text{unfold}\\
&&
%& \autpl{\labelset}{\mapschema{\ue}{i}{\labelset}} 
& \tpl{\mapschema{\ue}{i}{\labelset}} & \text{labeled tuple}\\
&&
%& \aupr{\ell}{\ue} 
& \prj{\ue}{\ell} & \text{projection}\\
&&
%& \auanain{\ell}{\ue} 
& \inj{\ell}{\ue} & \text{injection}\\
&&
%& \aumatchwithb{n}{\ue}{\seqschemaX{\urv}} 
& \matchwith{\ue}{\seqschemaX{\urv}} & \text{match}\\
&&
%& \aumval{\uX} 
& \mval{\uX} & \text{value component}\\
\LCC &&
% %& \color{Yellow} 
& \color{Yellow} & \color{Yellow} \\
% &&& \audefpetsm{\urho}{e}{\tsmv}{\ue} & \texttt{syntax}~\tsmv~\texttt{at}~\urho~\texttt{for} & \text{peTLM definition}\\
% &&&                                    & \texttt{expressions}~\{e\}~\texttt{in}~\ue\\
% &&& \auletpetsm{\uepsilon}{\tsmv}{\ue} & \texttt{let}~\texttt{syntax}~\tsmv=\uepsilon~\texttt{for} & \text{peTLM binding}\\
% &&&                                  & \texttt{expressions}~\texttt{in}~\ue\\
% &&& ... & ... & \text{peTLM designation}\\
&&
%& \auappetsm{b}{\uepsilon} 
& \utsmap{\uepsilon}{b} & \text{peTLM application}\ECC\\%\ECC
% &&& \auelit{b} & {\lit{b}}  & \text{peTLM unadorned literal}\\
% &&& \audefpptsm{\urho}{e}{\tsmv}{\ue} & \texttt{syntax}~\tsmv~\texttt{at}~\urho~\texttt{for} & \text{ppTLM definition}\\
% &&&                                    & \texttt{patterns}~\{e\}~\texttt{in}~\ue\\
% &&& \auletpptsm{\uepsilon}{\tsmv}{\ue} & \texttt{let}~\texttt{syntax}~\tsmv=\uepsilon~\texttt{for} & \text{ppTLM binding}\\
% &&& & \texttt{patterns}~\texttt{in}~\ue\\
% &&& ... & ... & \text{ppTLM designation}\\\ECC
\mathsf{URule} & \urv & ::= 
%& \aumatchrule{\upv}{\ue} 
& \matchrule{\upv}{\ue} & \text{match rule}\\
\mathsf{UPat} & \upv & ::= 
%& \ux 
& \ux & \text{identifier pattern}\\
&&
%& \auwildp 
& \wildp & \text{wildcard pattern}\\
&&
%& \aufoldp{\upv} 
& \foldp{\upv} & \text{fold pattern}\\
&&
%& \autplp{\labelset}{\mapschema{\upv}{i}{\labelset}} 
& \tplp{\mapschema{\upv}{i}{\labelset}} & \text{labeled tuple pattern}\\
&&
% & \auinjp{\ell}{\upv} 
& \injp{\ell}{\upv} 
& \text{injection pattern}\\
\LCC &&
%& \lightgray 
& \color{Yellow} & \color{Yellow}\\
&&
%& \auappptsm{b}{\uepsilon} 
& \utsmap{\uepsilon}{b} & \text{ppTLM application}\ECC
% &&& \auplit{b} & \lit{b} & \text{ppTLM unadorned literal}\ECC
\end{array}\]
\caption[Syntax of unexpanded expressions, rules and patterns in $\miniVerseParam$]{Syntax of unexpanded expressions, rules and patterns in $\miniVerseParam$}
\label{fig:P-unexpanded-terms}
\end{figure}

% \clearpage



\begin{figure}[p]
\[\begin{array}{lllllll}
\textbf{Sort} & & 
%& \textbf{Operational Form} 
& \textbf{Stylized Form} 
& \textbf{Description}\\
\LCC \color{Yellow}&\color{Yellow}& \color{Yellow}
%& \lightgray 
& \color{Yellow} & \color{Yellow}\\
\mathsf{UMType} & \urho & ::= 
%& \autype{\utau} 
& \utau & \text{type annotation}\\
% &&
%& \aualltypes{\ut}{\urho} 
% & \alltypes{\ut}{\urho} & \text{type parameterization}\\
&&
%& \auallmods{\usigma}{\uX}{\urho} 
& \allmods{\uX}{\usigma}{\urho} & \text{module parameterization}\\
\mathsf{UMExp} & \uepsilon & ::= 
%& \abindref{\tsmv} 
& \tsmv & \text{TLM identifier reference}\\
% &&
%& \auabstype{\ut}{\uepsilon} 
% & \abstype{\ut}{\uepsilon} & \text{type abstraction}\\
&&
%& \auabsmod{\usigma}{\uX}{\uepsilon} 
& \absmod{\uX}{\usigma}{\uepsilon} & \text{module abstraction}\\
% &&
%& \auaptype{\utau}{\uepsilon} 
% & \aptype{\uepsilon}{\utau} & \text{type application}\\
&&
%& \auapmod{\uM}{\uepsilon} 
& \apmod{\uepsilon}{\uX} & \text{module application}\ECC
\end{array}
\]
\caption{Syntax of unexpanded TLM types and expressions  in $\miniVerseParam$}
\label{fig:P-macro-expressions-types-u}
\end{figure}
\begin{figure}[t]
\[\begin{array}{lllllll}
\textbf{Sort} & & & \textbf{Operational Form} 
%& \textbf{Stylized Form} 
& \textbf{Description}\\
\LCC \color{Yellow}&\color{Yellow}& \color{Yellow}
%& \lightgray 
& \color{Yellow} & \color{Yellow}\\
\mathsf{MType} & \rho & ::= & \aetype{\tau} 
%& \tau 
& \text{type annotation}\\
% &&& \aealltypes{t}{\rho} 
%& \alltypes{t}{\rho} 
% & \text{type parameterization}\\
&&& \aeallmods{\sigma}{X}{\rho} 
%& \allmods{X}{\sigma}{\rho} 
& \text{module parameterization}\\
\mathsf{MExp} & \epsilon & ::= & \adefref{a} 
%& a 
& \text{TLM definition reference}\\
% &&& \aeabstype{t}{\epsilon} 
%& \abstype{t}{\epsilon} 
% & \text{type abstraction}\\
&&& \aeabsmod{\sigma}{X}{\epsilon} 
%& \absmod{X}{\sigma}{\epsilon} 
& \text{module abstraction}\\
% &&& \aeaptype{\tau}{\epsilon} 
%& \aptype{\epsilon}{\tau} 
% & \text{type application}\\
&&& \aeapmod{M}{\epsilon} 
%& \aptype{\epsilon}{M} 
& \text{module application}\ECC
\end{array}\]
\caption[Syntax of TLM types and expressions in $\miniVerseParam$]{Syntax of TLM types and expressions  in $\miniVerseParam$}
\label{fig:P-macro-expressions-types}
\end{figure}
\subsection{Syntax of the Unexpanded Language}
The syntax of the unexpanded language is defined in Figures \ref{fig:P-unexpanded-modules-signatures} through \ref{fig:P-macro-expressions-types}.

Each expanded form, with three exceptions, has a corresponding unexpanded form. We refer to these as the \emph{common forms}. The correspondence is defined in Appendix \ref{appendix:P-shared-forms}.

Kind variables, $k$, are one exception. Kind variables are used only in the metatheory.

The other two exceptions are constructions of the form $\amcon{M}$ and expressions of the form $\amval{M}$ where $M$ is of the form $\astruct{c}{e}$. Projection out of a module expression of the form $\astruct{c}{e}$ was supported in the XL only because this is needed to give the language  a conventional structural dynamics. Programmers refer to modules exclusively through module identifiers in unexpanded programs. 

In addition to the common forms, there are several forms related to pTLMs, highlighted in yellow in these figures. We need syntax for unexpanded TLM types, $\urho$, and unexpanded TLM expressions, $\uepsilon$, to support parameterization and parameter application. Internally, these expand to TLM expressions, $\epsilon$, and TLM types, $\rho$, respectively.

There is also a context-free textual syntax for the UL. For our purposes, we need only posit the existence of partial metafunctions that satisfy the following condition. 
\begingroup
\def\thetheorem{\ref{condition:textual-representability-P}}
\begin{condition}[Textual Representability] All of the following must hold:
\begin{enumerate}
% \item For each $\usigma$, there exists $b$ such that $\parseUSig{b}{\usigma}$.
% \item For each $\uM$, there exists $b$ such that $\parseUMod{b}{\uM}$.
\item For each $\ukappa$, there exists $b$ such that $\parseUKind{b}{\ukappa}$.
\item For each $\uc$, there exists $b$ such that $\parseUCon{b}{\uc}$.
\item For each $\ue$, there exists $b$ such that $\parseUExp{b}{\ue}$.
\item For each $\upv$, there exists $b$ such that $\parseUPat{b}{\upv}$.
\end{enumerate}
\end{condition}
\endgroup


\subsection{Typed Expansion}

Typed expansion is defined by six groups of judgements. In these judgements, \emph{unexpanded unified contexts}, $\uOmega$, take the form $\uOmegaEx{\uD}{\uG}{\uMctx}{\Omega}$, where $\uMctx$ is a \emph{module identifier expansion context}, $\uD$ is a \emph{construction identifier expansion context}, $\uG$ is an \emph{expression identifier expansion context} and $\Omega$ is a unified context. Identifier expansion contexts are defined in Appendix \ref{appendix:u-unified-ctxs} and conceptually operate as described in Sec. \ref{sec:miniVerseU}, mapping identifiers to variables.

The first group of judgements defines signature and module expansion.

\vspace{6px}
$\begin{array}{ll}
\textbf{Judgement Form} & \textbf{Description}\\
\sigExpandsPX{\usigma}{\sigma} & \text{$\usigma$ has well-formed expansion $\sigma$}\\
\mExpandsPX{\uM}{M}{\sigma} & \text{$\uM$ has expansion $M$ matching $\sigma$}
\end{array}$
\vspace{6px}

The second group of judgements defines kind and construction expansion.

\vspace{6px}
$\begin{array}{ll}
\textbf{Judgement Form} & \textbf{Description}\\
\kExpandsX{\ukappa}{\kappa} & \text{$\ukappa$ has well-formed expansion $\kappa$}\\
\cExpandsX{\uc}{c}{\kappa} & \text{$\uc$ has expansion $c$ of kind $\kappa$}
\end{array}$
\vspace{6px}

The third group of judgements defines expression, rule and pattern expansion.

\vspace{6px}
$\begin{array}{ll}
\textbf{Judgement Form} & \textbf{Description}\\
% \tExpandsPX{\utau}{\tau} & \text{$\utau$ has well-formed expansion $\tau$}\\
\expandsPX{\ue}{e}{\tau} & \text{$\ue$ has expansion $e$ of type $\tau$}\\
% \eanaPX{\ue}{e}{\tau} & \text{$\ue$ has expansion $e$ when analyzed against type $\tau$}\\
\rExpandsSP{\uOmega}{\uPsi}{\uPhi}{\urv}{r}{\tau}{\tau'} & \text{$\urv$ has expansion $r$ taking values of type $\tau$ to values of type $\tau'$}\\
% & \text{synthesized type $\tau'$}\\
% \ranaPX{\urv}{r}{\tau}{\tau'} & \text{$\urv$ has expansion $r$ and takes values of type $\tau$ to values of}\\
% & \text{type $\tau'$ when $\tau's$ is provided for analysis}\\
\patExpandsP{\uOmega'}{\uPhi}{\upv}{p}{\tau} & \text{$\upv$ has expansion $p$ matching at $\tau$  generating hypotheses $\uOmega'$}
\end{array}$
\vspace{6px}

The judgements above are defined by the rules given in Appendix \ref{appendix:typed-expansion-P}. Most of these rules simply serve to ``mirror'' corresponding rules in the statics of the XL, as was described in Sec. \ref{sec:miniVerseU}. The interesting rules, governing the forms highlighted in yellow, will be reproduced as we discuss them below.

The remaining judgements assign meaning to TLM types and expressions. We will detail these below. In particular, the fourth group of judgements define TLM type and expression expansion.

\vspace{6px}
$\begin{array}{ll}
\textbf{Judgement Form} & \textbf{Description}\\
\tsmtyExpands{\uOmega}{\urho}{\rho} & \text{$\urho$ has well-formed expansion $\rho$}\\
\tsmexpExpandsExp{\uOmega}{\uPsi}{\uepsilon}{\epsilon}{\rho} & \text{$\uepsilon$ has peTLM expression expansion $\epsilon$ at $\rho$}\\
\tsmexpExpandsPat{\uOmega}{\uPsi}{\uepsilon}{\epsilon}{\rho} & \text{$\uepsilon$ has ppTLM expression expansion $\epsilon$ at $\rho$}
\end{array}$
\vspace{6px}

The fifth group of judgements define the statics of TLM expressions.

\vspace{6px}
$\begin{array}{ll}
\textbf{Judgement Form} & \textbf{Description}\\
\istsmty{\Omega}{\rho} & \text{$\rho$ is a TLM type}\\
\hastsmtypeExp{\Omega}{\Psi}{\epsilon}{\rho} & \text{$\epsilon$ is a peTLM expression at $\rho$}\\
\hastsmtypePat{\Omega}{\Phi}{\epsilon}{\rho} & \text{$\epsilon$ is a ppTLM expression at $\rho$}
\end{array}$
\vspace{6px}

The sixth group of judgements define the dynamics of TLM expressions.

\vspace{6px}
$\begin{array}{ll}
\textbf{Judgement Form} & \textbf{Description}\\
\tsmexpStepsExp{\Omega}{\Psi}{\epsilon}{\epsilon'} & \text{peTLM expression $\epsilon$ transitions to $\epsilon'$}\\
\tsmexpStepsPat{\Omega}{\Psi}{\epsilon}{\epsilon'} & \text{ppTLM expression $\epsilon$ transitions to $\epsilon'$}\\
\tsmexpNormalExp{\Omega}{\Psi}{\epsilon} & \text{$\epsilon$ is a normal peTLM expression}\\
\tsmexpNormalPat{\Omega}{\Psi}{\epsilon} & \text{$\epsilon$ is a normal ppTLM expression}
\end{array}$
\vspace{6px}

We define the multi-step transition judgements $\tsmexpMultistepsExp{\Omega}{\Psi}{\epsilon}{\epsilon'}$ and $\tsmexpMultistepsPat{\Omega}{\Phi}{\epsilon}{\epsilon'}$ as the reflexive transitive closures of the corresponding transition judgements. We also define the peTLM expression normalization judgement $\tsmexpEvalsExp{\Omega}{\Psi}{\epsilon}{\epsilon'}$ iff $\tsmexpMultistepsExp{\Omega}{\Psi}{\epsilon}{\epsilon'}$ and $\tsmexpNormalExp{\Omega}{\Psi}{\epsilon'}$. Similarly, we define the ppTLM expression normalization judgement $\tsmexpEvalsPat{\Omega}{\Phi}{\epsilon}{\epsilon'}$ iff $\tsmexpMultistepsPat{\Omega}{\Phi}{\epsilon}{\epsilon'}$ and $\tsmexpNormalPat{\Omega}{\Phi}{\epsilon'}$.

\subsection{TLM Definitions}
TLMs are scoped to module expressions. (Adding support for TLM definitions scoped to a single expression would be a straightforward exercise, so we omit the details for simplicity.)

\subsubsection{peTLM Definitions}
The rule governing peTLM definitions is reproduced below:
\begin{equation*}\tag{\ref{rule:mExpandsP-syntaxpe}}
\inferrule{
  \tsmtyExpands{\uOmega}{\urho}{\rho}\\
  \hastypeP{\emptyset}{\eparse}{\aparr{\tBody}{\tParseResultPCEExp}}\\\\
  \evalU{\eparse}{\eparse'}\\
  \mExpandsP{\uOmega}{\uAS{\uA \uplus \mapitem{\tsmv}{\adefref{a}}}{\Psi, \petsmdefn{a}{\rho}{\eparse'}}}{\uPhi}{\uM}{M}{\sigma}
}{
  \mExpandsP{\uOmega}{\uAS{\uA}{\Psi}}{\uPhi}{\defpetsm{\tsmv}{\urho}{\eparse}{\uM}}{M}{\sigma}
}
\end{equation*}

peTLM definitions differ from ueTLM definitions in that the unexpanded type annotation is an \emph{unexpanded TLM type}, $\urho$, rather than an unexpanded type, $\utau$. This unexpanded TLM type determines the parameterization of the TLM. The first premise of the rule above expands the unexpanded TLM type to produce a \emph{TLM type}, $\rho$. The straightforward rules governing TLM type expansion are reproduced below.
\begin{equation*}\tag{\ref{rule:tsmtyExpands-type}}
\inferrule{
  \cExpandsX{\utau}{\tau}{\akty}
}{
  \tsmtyExpands{\uOmega}{{\utau}}{\aetype{\tau}}
}
\end{equation*}
% \begin{equation*}\tag{\ref{rule:tsmtyExpands-alltypes}}
% \inferrule{
%   \tsmtyExpands{\uOmega, \uKhyp{\ut}{t}{\akty}}{\urho}{\rho}
% }{
%   \tsmtyExpands{\uOmega}{\alltypes{\ut}{\urho}}{\aealltypes{t}{\rho}}
% }
% \end{equation*}
\begin{equation*}\tag{\ref{rule:tsmtyExpands-allmods}}
\inferrule{
  \sigExpandsPX{\usigma}{\sigma}\\
  \tsmtyExpands{\uOmega, \uMhyp{\uX}{X}{\sigma}}{\urho}{\rho}
}{
  \tsmtyExpands{\uOmega}{\allmods{\uX}{\usigma}{\urho}}{\aeallmods{\sigma}{X}{\rho}}
}
\end{equation*}
Rule (\ref{rule:tsmtyExpands-type}) defines quantification over modules matching a given signature. There is no mechanism for quantification over types in the calculus because it can be understood as quantification over a module with a single type component.

The second premise of Rule (\ref{rule:mExpandsP-syntaxpe}) checks that the parse function is of the appropriate type. The types $\tBody$ and $\tParseResultPCEExp$ are characterized in Appendix \ref{appendix:typed-expansion-P}. The type $\tPProtoExpr$ classifies \emph{encodings of parameterized proto-expressions}, which we will return to when we discuss TLM application below.

The third premise of Rule (\ref{rule:mExpandsP-syntaxpe}) evaluates the parse function to a value.

The final premise of Rule (\ref{rule:mExpandsP-syntaxpe}) extends the \emph{peTLM context}, $\uPsi$, which consists of a \emph{TLM identifier expansion context}, $\uA$, and a \emph{peTLM definition context}, $\Psi$. A peTLM definition context maps TLM names, $a$, to an expanded peTLM definition, $\petsmdefn{a}{\rho}{\eparse}$, where $\rho$ is the TLM type determined from the annotation and $\eparse$ is its parse function. A TLM identifier context maps TLM identifiers, $\tsmv$, to \emph{TLM expressions}, $\epsilon$. In this case, the TLM expression is simply a reference to the newly introduced TLM definition, $\adefref{a}$. We discuss the other TLM expression forms when we discuss TLM abbreviations below.

\subsubsection{ppTLM Definitions}
The rule governing ppTLM definitions is similar, and is reproduced below:
\begin{equation*}\tag{\ref{rule:mExpandsP-syntaxpp}}
\inferrule{ 
  \tsmtyExpands{\uOmega}{\urho}{\rho}\\
  \hastypeP{\emptyset}{\eparse}{\aparr{\tBody}{\tParseResultCEPat }}\\\\
  \evalU{\eparse}{\eparse'}\\
  \mExpandsP{\uOmega}{\uPsi}{\uAS{\uA \uplus \mapitem{\tsmv}{\adefref{a}}}{\Phi, \pptsmdefn{a}{\rho}{\eparse'}}}{\uM}{M}{\sigma}
}{
  \mExpandsP{\uOmega}{\uPsi}{\uAS{\uA}{\Phi}}{\defpptsm{\tsmv}{\urho}{\eparse}{\uM}}{M}{\sigma}
}
\end{equation*}
This rule differs from Rule (\ref{rule:mExpandsP-syntaxpe}) in the type of the parse function and in the fact that the \emph{ppTLM context}, $\uPhi$, rather than the peTLM context, is updated.

\subsection{TLM Abbreviations}
It is possible to abbreviate a complex TLM expression by binding it to a TLM identifier.

\subsubsection{peTLM Abbreviations}
The rule governing peTLM abbreviations is reproduced below:
\begin{equation}\tag{\ref{rule:mExpandsP-letpetsm}}
\inferrule{
  \tsmexpExpandsExp{\uOmega}{\uAS{\uA}{\Psi}}{\uepsilon}{\epsilon}{\rho}\\
  \mExpandsP{\uOmega}{\uAS{\uA\uplus\mapitem{\tsmv}{\epsilon}}{\Psi}}{\uPhi}{\uM}{M}{\sigma}
}{
  \mExpandsP{\uOmega}{\uAS{\uA}{\Psi}}{\uPhi}{\uletpetsm{\tsmv}{\uepsilon}{\uM}}{M}{\sigma}
}
\end{equation}
Here, $\uepsilon$ is an \emph{unexpanded TLM expression}. The first premise of the rule above expands it, producing a TLM expression $\epsilon$ at TLM type $\rho$. The second premise updates the peTLM identifier expansion context with this TLM expression.

The rules below govern peTLM expression expansion. The first rule handles the base case, when the unexpanded TLM expression is a TLM identifier, $\tsmv$, by looking it up in $\uA$ and determining its TLM type according to the TLM expression typing judgement, $\hastsmtypeExp{\Omega}{\Psi}{\epsilon}{\rho}$ (which mirrors the rules below, and is defined in Appendix \ref{appendix:typed-expansion-P}.)
\begin{equation*}\tag{\ref{rule:tsmexpExpandsExp-bindref}}
\inferrule{
  \hastsmtypeExp{\Omega}{\Psi}{\epsilon}{\rho}  
}{
  \tsmexpExpandsExp{\uOmegaEx{\uD}{\uG}{\uMctx}{\Omega}}{\uAS{\uA, \mapitem{\tsmv}{\epsilon}}{\Psi}}{{\tsmv}}{\epsilon}{\rho}
}
\end{equation*}

The following rule allows a peTLM expression to itself abstract over a module. (This is necessary to support abbreviated application of parameters other than the first.)
% \begin{equation*}\tag{\ref{rule:tsmexpExpandsExp-abstype}}
% \inferrule{
%   \tsmexpExpandsExp{\uOmega, \uKhyp{\ut}{t}{\akty}}{\uPsi}{\uepsilon}{\epsilon}{\rho}
% }{
%   \tsmexpExpandsExp{\uOmega}{\uPsi}{\abstype{\ut}{\uepsilon}}{\aeabstype{t}{\epsilon}}{\aealltypes{t}{\rho}}
% }
% \end{equation*}
\begin{equation*}\tag{\ref{rule:tsmexpExpandsExp-absmod}}
\inferrule{
  \sigExpandsPX{\usigma}{\sigma}\\
  \tsmexpExpandsExp{\uOmega, \uMhyp{\uX}{X}{\sigma}}{\uPsi}{\uepsilon}{\epsilon}{\rho}
}{
  \tsmexpExpandsExp{\uOmega}{\uPsi}{\absmod{\uX}{\usigma}{\uepsilon}}{\aeabsmod{\sigma}{X}{\epsilon}}{\aeallmods{\sigma}{X}{\rho}}
}
\end{equation*}

The final rule defines the semantics of parameter application.
% \begin{equation*}\tag{\ref{rule:tsmexpExpandsExp-aptype}}
% \inferrule{
%   \tsmexpExpandsExp{\uOmega}{\uPsi}{\uepsilon}{\epsilon}{\aealltypes{t}{\rho}}\\
%   \cExpandsX{\utau}{\tau}{\akty}
% }{
%   \tsmexpExpandsExp{\uOmega}{\uPsi}{\aptype{\uepsilon}{\utau}}{\aeaptype{\tau}{\epsilon}}{[\tau/t]\rho} 
% }
% \end{equation*}
\begin{equation*}\tag{\ref{rule:tsmexpExpandsExp-apmod}}
\inferrule{
  \tsmexpExpandsExp{\uOmega}{\uPsi}{\uepsilon}{\epsilon}{\aeallmods{\sigma}{X'}{\rho}}\\
  \mExpandsPX{\uX}{X}{\sigma}
}{
  \tsmexpExpandsExp{\uOmega}{\uPsi}{\apmod{\uepsilon}{\uX}}{\aeapmod{X}{\epsilon}}{[X/X']\rho}
}
\end{equation*}

\subsubsection{ppTLM Abbreviations}
The rule governing ppTLM abbreviations is analagous:
\begin{equation*}\tag{\ref{rule:mExpandsP-letpptsm}}
\inferrule{
  \tsmexpExpandsPat{\uOmega}{\uAS{\uA}{\Phi}}{\uepsilon}{\epsilon}{\rho}\\
  \mExpandsP{\uOmega}{\uPsi}{\uAS{\uA\uplus\mapitem{\tsmv}{\epsilon}}{\Phi}}{\uM}{M}{\sigma}
}{
  \mExpandsP{\uOmega}{\uPsi}{\uAS{\uA}{\Phi}}{\uletpptsm{\tsmv}{\uepsilon}{\uM}}{M}{\sigma}
}
\end{equation*}
The ppTLM expression expansion judgement appearing as the first premise is defined analagously to the peTLM expression expansion judgement defined above, differing only in that the rule for TLM identifiers consults the ppTLM context rather than the peTLM context. The rules are reproduced in Appendix \ref{appendix:typed-expansion-P}.

\subsection{TLM Application}

\subsubsection{peTLM Application}
The rule for applying an unexpanded peTLM expression $\uepsilon$ to a generalized literal form with body $b$ is reproduced below:
\begin{equation*}\tag{\ref{rule:expandsP-apuetsm}}
\inferrule{
  \uOmega = \uOmegaEx{\uD}{\uG}{\uMctx}{\Omega_\text{app}}\\
  \uPsi=\uAS{\uA}{\Psi}\\\\
  \tsmexpExpandsExp{\uOmega}{\uPsi}{\uepsilon}{\epsilon}{\aetype{\tau_\text{final}}}\\
  \tsmexpEvalsExp{\Omega_\text{app}}{\Psi}{\epsilon}{\epsilon_\text{normal}}\\\\
  \tsmdefof{\epsilon_\text{normal}}=a\\
  \Psi = \Psi', \petsmdefn{a}{\rho}{\eparse}\\\\
  \encodeBody{b}{\ebody}\\
  \evalU{\ap{\eparse}{\ebody}}{\aein{\mathtt{SuccessE}}{e_\text{pproto}}}\\
  \decodePCEExp{e_\text{pproto}}{\pce}\\\\
  \prepce{\Omega_\text{app}}{\Psi}{\pce}{\ce}{\epsilon_\text{normal}}{\aetype{\tau_\text{proto}}}{\omega}{\Omega_\text{params}}\\\\
     \segOKP{\OParams}{\csceneP{\omega : \OParams}{\uOmega}{b}}{\segof{\ce}}{b}\\
  \cvalidEP{\Omega_\text{params}}{\esceneP{\omega : \OParams}{\uOmega}{\uPsi}{\uPhi}{b}}{\ce}{e}{\tau_\text{proto}}
}{
  \expandsP{\uOmega}{\uPsi}{\uPhi}{\utsmap{\uepsilon}{b}}{[\omega]e}{[\omega]\tau_\text{proto}}
}
\end{equation*}

The first two premises simply deconstruct $\uOmega$ and $\uPsi$. Next, we expand $\uepsilon$ according to the unexpanded peTLM expression expansion rules that we already described above. The resulting TLM expression, $\epsilon$, must be defined at a type (i.e. no quantification must remain.)

The fourth premise performs \emph{peTLM expression normalization}. Normalization is defined in terms of a simple structural dynamics with two stepping rules:
% \begin{equation*}\tag{\ref{rule:tsmexpEvalsExp}}
% \inferrule{
%   \tsmexpMultistepsExp{\Omega}{\Psi}{\epsilon}{\epsilon'}\\
%   \tsmexpNormalExp{\Omega}{\Psi}{\epsilon'}
% }{
%   \tsmexpEvalsExp{\Omega}{\Psi}{\epsilon}{\epsilon'}
% }
% \end{equation*}
% where the multistep judgement, $\tsmexpMultistepsExp{\Omega}{\Psi}{\epsilon}{\epsilon'}$, is defined as the reflexive, transitive closure of the stepping judgement defined by the following rules:
% \begin{equation*}\tag{\ref{rule:tsmexpStepsExp-aptype-1}}
% \inferrule{
%   \tsmexpStepsExp{\Omega}{\Psi}{\epsilon}{\epsilon'}
% }{
%   \tsmexpStepsExp{\Omega}{\Psi}{\aeaptype{\tau}{\epsilon}}{\aeaptype{\tau}{\epsilon'}}
% }
% \end{equation*}
% \begin{equation*}\tag{\ref{rule:tsmexpStepsExp-aptype-2}}
% \inferrule{ }{
%   \tsmexpStepsExp{\Omega}{\Psi}{\aeaptype{\tau}{\aeabstype{t}{\epsilon}}}{[\tau/t]\epsilon}
% }
% \end{equation*}
\begin{equation*}\tag{\ref{rule:tsmexpStepsExp-apmod-1}}
\inferrule{
  \tsmexpStepsExp{\Omega}{\Psi}{\epsilon}{\epsilon'}
}{
  \tsmexpStepsExp{\Omega}{\Psi}{\aeapmod{X}{\epsilon}}{\aeapmod{X}{\epsilon'}}
}
\end{equation*}
\begin{equation*}\tag{\ref{rule:tsmexpStepsExp-apmod-2}}
\inferrule{ }{
  \tsmexpStepsExp{\Omega}{\Psi}{\aeapmod{X}{\aeabsmod{\sigma}{X'}{\epsilon}}}{[X/X']\epsilon}
}
\end{equation*}
The peTLM expression normal forms are defined as follows:
\begin{equation*}\tag{\ref{rule:tsmexpNormalExp-defref}}
\inferrule{ }{
  \tsmexpNormalExp{\Omega}{\Psi, \petsmdefn{a}{\rho}{\eparse}}{\adefref{a}}
}
\end{equation*}
% \begin{equation*}\tag{\ref{rule:tsmexpNormalExp-abstype}}
% \inferrule{ }{
%   \tsmexpNormalExp{\Omega}{\Psi}{\aeabstype{t}{\epsilon}}
% }
% \end{equation*}
\begin{equation*}\tag{\ref{rule:tsmexpNormalExp-absmod}}
\inferrule{ }{
  \tsmexpNormalExp{\Omega}{\Psi}{\aeabsmod{\sigma}{X}{\epsilon}}
}
\end{equation*}
% \begin{equation*}\tag{\ref{rule:tsmexpNormalExp-aptype}}
% \inferrule{
%   \epsilon \neq \aeabstype{t}{\epsilon'}\\
%   \tsmexpNormalExp{\Omega}{\Psi}{\epsilon}
% }{
%   \tsmexpNormalExp{\Omega}{\Psi}{\aeaptype{\tau}{\epsilon}}
% }
% \end{equation*}
\begin{equation*}\tag{\ref{rule:tsmexpNormalExp-apmod}}
\inferrule{
  \epsilon \neq \aeabsmod{\sigma}{X'}{\epsilon'}\\
  \tsmexpNormalExp{\Omega}{\Psi}{\epsilon}
}{
  \tsmexpNormalExp{\Omega}{\Psi}{\aeapmod{X}{\epsilon}}
}
\end{equation*}
Normalization leaves only those parameter applications that cannot be reduced away immediately, i.e. those specified by the original TLM definition.

The TLM definition at the root of the normalized TLM expression is extracted by the third row of premises in Rule (\ref{rule:expandsP-apuetsm}). The first of these appeals to the following metafunction to produce the TLM definition's name.
\begin{align}
\tsmdefof{\adefref{a}} & = a \tag{\ref{eqn:tsmdefof-adefref}}\\
% \tsmdefof{\aeabstype{t}{\epsilon}} & = \tsmdefof{\epsilon} \tag{\ref{eqn:tsmdefof-abstype}}\\
\tsmdefof{\aeabsmod{\sigma}{X}{\epsilon}} & = \tsmdefof{\epsilon} \tag{\ref{eqn:tsmdefof-absmod}}\\
% \tsmdefof{\aeaptype{\tau}{\epsilon}} & = \tsmdefof{\epsilon} \tag{\ref{eqn:tsmdefof-aptype}}\\
\tsmdefof{\aeapmod{X}{\epsilon}} & = \tsmdefof{\epsilon} \tag{\ref{eqn:tsmdefof-apmod}}
\end{align}
The second premise on the third row then looks up this name within $\Psi$.

The fourth row of premises in Rule (\ref{rule:expandsP-apuetsm}) 1) encode the body as a value of the type $\tBody$; 2) apply the parse function; and 3) decode the result, producing a \emph{parameterized proto-expression}, $\pce$. Parameterized proto-expressions, $\pce$, are ABTs that serve to introduce the parameter bindings into a proto-expression, $\ce$. The operational and stylized syntax of parameterized proto-expression is given in Figure \ref{fig:P-pceexp}. 

\begin{figure}[h]
\[\begin{array}{lllllll}
\textbf{Sort} & & & \textbf{Operational Form} & \textbf{Stylized Form} & \textbf{Description}\\
\LCC \color{Yellow}&\color{Yellow}&\color{Yellow}& \color{Yellow} & \color{Yellow} & \color{Yellow}\\
\mathsf{PPrExpr} & \pce & ::= & \apceexp{\ce} & \pceexp{\ce} & \text{proto-expression}\\
% &&& \apcebindtype{t}{\pce} & \pcebindtype{t}{\pce} & \text{type binding}\\
&&& \apcebindmod{X}{\pce} & \pcebindmod{X}{\pce} & \text{module binding}\ECC
\end{array}\]
\caption[Syntax of parameterized proto-expressions in $\miniVerseParam$]{Syntax of parameterized proto-expressions  in $\miniVerseParam$}
\label{fig:P-pceexp}
\end{figure}
\noindent 
There must be one binder in $\pce$ for each TLM parameter specified by $\tsmdefof{\epsilon_\text{normal}}$. (VerseML inserts these binders automatically as a convenience, but we consider only the underlying mechanism in this core calculus.) The judgement on the fifth row of Rule (\ref{rule:expandsP-apuetsm}) then \emph{deparameterizes} $\pce$ by peeling away these binders to produce 1) the underlying proto-expression, $\ce$, with the variables that stand for the parameters free; 2) a corresponding deparameterized type, $\tau_\text{proto}$, that uses the same free variables to stand for the parameters; 3) a \emph{substitution}, $\omega$, that pairs the applied parameters from $\epsilon_\text{normal}$ with the corresponding variables generated when peeling away the binders in $\pce$; and 4) a corresponding \emph{parameter context}, $\Omega_\text{params}$, that tracks the signatures of these variables. The two rules governing the proto-expression deparameterization judgement are reproduced below:
\begin{equation*}\tag{\ref{rule:prepce-ceexp}}
\inferrule{ }{
  \prepce{\Omega_\text{app}}{\Psi, \petsmdefn{a}{\rho}{\eparse}}{\apceexp{\ce}}{\ce}{\adefref{a}}{\rho}{\emptyset}{\emptyset}
}
\end{equation*}
% \begin{equation*}\tag{\ref{rule:prepce-alltypes}}
% \inferrule{
%   \prepce{\Omega_\text{app}}{\Psi}{\pce}{\ce}{\epsilon}{\aealltypes{t}{\rho}}{\omega}{\Omega}\\
%   t \notin \domof{\Omega_\text{app}}
% }{
%   \prepce{\Omega_\text{app}}{\Psi}{\apcebindtype{t}{\pce}}{\ce}{\aeaptype{\tau}{\epsilon}}{\rho}{\omega, \tau/t}{\Omega, t :: \akty}
% }
% \end{equation*}
\begin{equation*}\tag{\ref{rule:prepce-allmods}}
\inferrule{
  \prepce{\Omega_\text{app}}{\Psi}{\pce}{\ce}{\epsilon}{\aeallmods{\sigma}{X}{\rho}}{\omega}{\Omega}\\
  X \notin \domof{\Omega_\text{app}}
}{
  \prepce{\Omega_\text{app}}{\Psi}{\apcebindmod{X}{\pce}}{\ce}{\aeapmod{X'}{\epsilon}}{\rho}{(\omega, X'/X)}{(\Omega, X : \sigma)}
}
\end{equation*}
This judgement can be pronounced ``when applying peTLM $\epsilon$, $\pce$ has deparameterization $\ce$ leaving $\rho$ with parameter substitution $\omega$''. Notice from Rule (\ref{rule:prepce-allmods}) that every module binding in $\pce$ must pair with a corresponding module parameter application. Moreover, the variables standing for parameters must not appear in $\Omega_\text{app}$, i.e. $\domof{\Omega_\text{params}}$ must be disjoint from $\domof{\Omega_\text{app}}$ (this requirement can always be discharged by alpha-variation.)

The final row of premises in Rule (\ref{rule:expandsP-apuetsm}) performs proto-expansion validation. This involves first checking that the segmentation of $\ce$ is valid. Segmentation validation is contextual because kind and type equivalence are contextual (see Appendix \ref{appendix:segmentations-P} for details.) After checking the segmentation, the next premise checks that the proto-expansion is well-typed under the parameter context, $\Omega_\text{param}$ (rather than the empty context, as was the case in $\miniVersePat$.) The conclusion of the rule applies the parameter substitution, $\omega$, to the resulting expression and the deparameterized type it was checked against.

\subsubsection{ppTLM Application}

The rule governing ppTLM application is similar:
\begin{equation*}\tag{\ref{rule:patExpandsP-apuptsm}}
\inferrule{
  \uOmega=\uOmegaEx{\uD}{\uG}{\uMctx}{\Omega_\text{app}}\\
  \uPhi=\uAS{\uA}{\Phi}\\\\
  \tsmexpExpandsPat{\uOmega}{\uPhi}{\uepsilon}{\epsilon}{\aetype{\tau_\text{final}}}\\
  \tsmexpEvalsPat{\Omega_\text{app}}{\Phi}{\epsilon}{\epsilon_\text{normal}}\\\\
  \tsmdefof{\epsilon_\text{normal}}=a\\
  \Phi = \Phi', \pptsmdefn{a}{\rho}{\eparse}\\\\
  \encodeBody{b}{\ebody}\\
  \evalU{\ap{\eparse}{\ebody}}{\aein{\mathtt{SuccessP}}{e_\text{pproto}}}\\
  \decodePCEPat{e_\text{pproto}}{\pcp}\\\\
  \prepcp{\Omega_\text{app}}{\Phi}{\pcp}{\cpv}{\epsilon_\text{normal}}{\aetype{\tau_\text{proto}}}{\omega}{\Omega_\text{params}}\\\\
     \segOKP{\OParams}{\csceneP{\omega : \OParams}{\uOmega}{b}}{\segof{\cpv}}{b}\\
  \cvalidPP{\uOmega'}{\psceneP{\omega : \Omega_\text{params}}{\uOmega}{\uPhi}{b}}{\cpv}{p}{\tau_\text{proto}}
}{
  \patExpandsP{\uOmega'}{\uPhi}{\utsmap{\uepsilon}{b}}{p}{[\omega]\tau_\text{proto}}
}
\end{equation*}

Although patterns themselves cannot make reference to surrounding bindings, the type annotations on spliced patterns can, so we need the notion of a \emph{parameterized proto-pattern}, $\pcp$, and a corresponding deparameterization judgement. The necessary definitions, which are analagous to those given above for peTLMs, are given in Appendix \ref{appendix:typed-expansion-P}.

\subsection{Syntax of Proto-Expansions}\label{sec:ce-syntax-P}

\begin{figure}[p] 
\[\begin{array}{lrlllll}
\textbf{Sort} & & & \textbf{Operational Form} & \textbf{Stylized Form} & \textbf{Description}\\
\mathsf{PrKind} & \cekappa & ::= & \acekdarr{\cekappa}{u}{\cekappa} & \kdarr{u}{\cekappa}{\cekappa} & \text{dependent function}\\
&&& \acekunit & \kunit & \text{nullary product}\\
&&& \acekdbprod{\cekappa}{u}{\cekappa} & \kdbprod{u}{\cekappa}{\cekappa} & \text{dependent product}\\
%&&& \akdprodstd & \kdprodstd & \text{labeled dependent product}\\
&&& \acekty & \kty & \text{type}\\
&&& \aceksing{\ctau} & \ksing{\ctau} & \text{singleton}\\
\LCC &&& \color{Yellow} & \color{Yellow} & \color{Yellow}\\
&&& \acesplicedk{m}{n} & \splicedk{m}{n} & \text{spliced kind}\ECC\\
\mathsf{PrCon} & \cec, \ctau & ::= & u & u & \text{construction variable}\\
&&& t & t & \text{type variable}\\
% &&& \acecasc{\cekappa}{\cec} & \casc{\cec}{\cekappa} & \text{ascription}\\
&&& \acecabs{u}{\cec} & \cabs{u}{\cec} & \text{abstraction}\\
&&& \acecapp{\cec}{\cec} & \capp{\cec}{\cec} & \text{application}\\
&&& \acectriv & \ctriv & \text{trivial}\\
&&& \acecpair{\cec}{\cec} & \cpair{\cec}{\cec} & \text{pair}\\
&&& \acecprl{\cec} & \cprl{\cec} & \text{left projection}\\
&&& \acecprr{\cec} & \cprr{\cec} & \text{right projection}\\
%&&& \adtplX & \dtplX & \text{labeled dependent tuple}\\
%&&& \adprj{\ell}{c} & \prj{c}{\ell} & \text{projection}\\
&&& \aceparr{\ctau}{\ctau} & \parr{\ctau}{\ctau} & \text{partial function}\\
&&& \aceallu{\cekappa}{u}{\ctau} & \forallu{u}{\cekappa}{\ctau} & \text{polymorphic}\\
&&& \acerec{t}{\ctau} & \rect{t}{\ctau} & \text{recursive}\\
&&& \aceprod{\labelset}{\mapschema{\ctau}{i}{\labelset}} & \prodt{\mapschema{\ctau}{i}{\labelset}} & \text{labeled product}\\
&&& \acesum{\labelset}{\mapschema{\ctau}{i}{\labelset}} & \sumt{\mapschema{\ctau}{i}{\labelset}} & \text{labeled sum}\\
&&& \acemcon{X} & \mcon{X} & \text{construction component}\\
\LCC &&& \color{Yellow} & \color{Yellow} & \color{Yellow}\\
&&& \acesplicedc{m}{n}{\cekappa} & \splicedc{m}{n}{\cekappa} & \text{spliced construction}\ECC
\end{array}\]
\caption[Syntax of proto-kinds and proto-constructions in $\miniVerseParam$]{Syntax of proto-kinds and proto-constructions in $\miniVerseParam$}
\label{fig:P-ce-kinds-constructors}
\end{figure}

\begin{figure}[p]
\[\arraycolsep=4pt\begin{array}{lllllll}
\textbf{Sort} & & & \textbf{Operational Form} & \textbf{Stylized Form} & \textbf{Description}\\
\mathsf{PrExp} & \ce & ::= & x & x & \text{variable}\\
&&& \aceasc{\ctau}{\ce} & \asc{\ce}{\ctau} & \text{ascription}\\
&&& \aceletsyn{x}{\ce}{\ce} & \letsyn{x}{\ce}{\ce} & \text{value binding}\\
% &&& \aceasc{\ctau}{\ce} & \asc{\ce}{\ctau} & \text{ascription}\\
% &&& \aceletsyn{x}{\ce}{\ce} & \letsyn{x}{\ce}{\ce} & \text{value binding}\\
% &&& \aceanalam{x}{\ce} & \analam{x}{\ce} & \text{abstraction (unannotated)}\\
&&& \acelam{\ctau}{x}{\ce} & \lam{x}{\ctau}{\ce} & \text{abstraction}\\
&&& \aceap{\ce}{\ce} & \ap{\ce}{\ce} & \text{application}\\
&&& \aceclam{\cekappa}{u}{\ce} & \clam{u}{\cekappa}{\ce} & \text{construction abstraction}\\
&&& \acecap{\ce}{\cec} & \cAp{\ce}{\cec} & \text{construction application}\\
&&& \acefold{\ce} & \fold{\ce} & \text{fold}\\
&&& \aceunfold{\ce} & \unfold{\ce} & \text{unfold}\\
&&& \acetpl{\labelset}{\mapschema{\ce}{i}{\labelset}} & \tpl{\mapschema{\ce}{i}{\labelset}} & \text{labeled tuple}\\
&&& \acepr{\ell}{\ce} & \prj{\ce}{\ell} & \text{projection}\\
&&& \aceanain{\ell}{\ce} & \inj{\ell}{\ce} & \text{injection}\\
&&& \acematchwith{n}{\ce}{\seqschemaX{\urv}} & \matchwith{\ce}{\seqschemaX{\crv}} & \text{match}\\
&&& \acemval{X} & \mval{X} & \text{value component}\\
\LCC &&& \color{Yellow} & \color{Yellow} & \color{Yellow}\\
&&& \acesplicede{m}{n}{\ctau} & \splicede{m}{n}{\ctau} & \text{spliced expression}\ECC\\
\mathsf{PrRule} & \crv & ::= & \acematchrule{p}{\ce} & \matchrule{p}{\ce} & \text{rule}\\
\mathsf{PrPat} & \cpv & ::= & \acewildp & \wildp & \text{wildcard pattern}\\
&&& \acefoldp{p} & \foldp{p} & \text{fold pattern}\\
&&& \acetplp{\labelset}{\mapschema{\cpv}{i}{\labelset}} & \tplp{\mapschema{\cpv}{i}{\labelset}} & \text{labeled tuple pattern}\\
&&& \aceinjp{\ell}{\cpv} & \injp{\ell}{\cpv} & \text{injection pattern}\\
&&& \acemval{X} & \mval{X} & \text{value component}\\
\LCC &&& \color{Yellow} & \color{Yellow} & \color{Yellow}\\
&&& \acesplicedp{m}{n}{\ctau} & \splicedp{m}{n}{\ctau} & \text{spliced pattern} \ECC
\end{array}\]
\caption[Syntax of proto-expressions, proto-rules and proto-patterns in $\miniVerseParam$]{Syntax of proto-expressions, proto-rules and proto-patterns in $\miniVerseParam$}
\label{fig:P-candidate-terms}
\end{figure}
Figure \ref{fig:P-ce-kinds-constructors} defines the syntax of proto-kinds, $\cekappa$ and proto-constructions, $\cec$. Figure \ref{fig:P-candidate-terms} defines the syntax of proto-expressions, $\ce$, proto-rules, $\crv$, and proto-patterns, $\cpv$. All of these are ABTs. %The syntax of ce-types is identical to that given in Figure \ref{fig:U-candidate-terms}, which was described in Sec. \ref{sec:ce-syntax-U}. 

The mapping from expanded forms to proto-expansion forms is given in Appendix \ref{appendix:P-proto-expansion-validation}. The only ``interesting'' forms are the forms for references to spliced unexpanded terms, highlighted in yellow in Figure \ref{fig:P-ce-kinds-constructors} and Figure \ref{fig:P-candidate-terms}.

\subsection{Proto-Expansion Validation}
Proto-expansion validation operates essentially as described in Sec. \ref{sec:ce-validation-U}. It is governed by two groups of judgements. The first group of judgements defines proto-kind and proto-construction validation.

\vspace{10px}\noindent
$\begin{array}{ll}
\textbf{Judgement Form} & \textbf{Description}\\
\cvalidKX{\cekappa}{\kappa} & \text{$\cekappa$ has well-formed expansion $\kappa$}\\
\cvalidCX{\cec}{c}{\kappa} & \text{$\cec$ has expansion $c$ of kind $\kappa$}\\
\end{array}$
\vspace{10px}

The second group of judgements defines proto-expression, proto-rule and proto-pattern validation.

\vspace{10px}\noindent
$\arraycolsep=4pt\begin{array}{ll}
\textbf{Judgement Form} & \textbf{Description}\\
% \cvalidTP{\Omega}{\cscenev}{\ctau}{\tau} & \text{$\ctau$ has expansion $\tau$}\\
\cvalidEPX{\ce}{e}{\tau} & \text{$\ce$ has expansion $e$ of type $\tau$}\\
\cvalidRP{\Omega}{\escenev}{\crv}{r}{\tau}{\tau'} & \text{$\crv$ has expansion $r$ taking values of type $\tau$ to values of type $\tau'$}\\
\cvalidPPE{\uOmega}{\pscenev}{\cpv}{p}{\tau} & \text{$\cpv$ has expansion $p$ matching against $\tau$ generating hypotheses $\uOmega$}
\end{array}$
\vspace{10px}

\emph{Expression splicing scenes}, $\escenev$, are of the form $\esceneP{\omega : \Omega_\text{params}}{\uOmega}{\uPsi}{\uPhi}{b}$, \emph{construction splicing scenes}, $\cscenev$, are of the form $\csceneP{\omega : \Omega_\text{params}}{\uOmega}{b}$, and \emph{pattern splicing scenes}, $\pscenev$, are of the form $\psceneP{\omega : \Omega_\text{params}}{\uOmega}{\uPhi}{b}$. Their purpose is to ``remember'', during proto-expansion validation, the contexts and literal bodies from the TLM application site (cf. Rules (\ref{rule:expandsP-apuetsm}) and (\ref{rule:patExpandsP-apuptsm}) above), because these are necessary to validate references to spliced terms. They also keep around the parameter substitution and corresponding context, $\omega : \Omega_\text{params}$, because type/kind annotations on spliced terms need to be able to access parameters (but not expansion-local bindings.) 
We write $\csfrom{\escenev}$ for the construction splicing scene constructed by dropping the TLM contexts from $\escenev$:
\[\csfrom{\esceneP{\omega : \OParams}{\uOmega}{\uPsi}{\uPhi}{b}} = \csceneP{\omega : \OParams}{\uOmega}{b}\]

The rules governing references to spliced terms are reproduced below:
\begin{equation*}\tag{\ref{rule:cvalidK-spliced}}
\inferrule{
  \parseUKind{\bsubseq{b}{m}{n}}{\ukappa}\\
  \kExpands{\uOmega}{\ukappa}{\kappa}\\\\
  \uOmega=\uOmegaEx{\uD}{\uG}{\uMctx}{\Omega_\text{app}}\\
  \domof{\Omega} \cap \domof{\Omega_\text{app}} = \emptyset
}{
  \cvalidK{\Omega}{\csceneP{\omega : \OParams}{\uOmega}{b}}{\acesplicedk{m}{n}}{\kappa}
}
\end{equation*}
\begin{equation*}\tag{\ref{rule:cvalidC-spliced}}
\inferrule{
  \cscenev=\csceneP{\omega : \OParams}{\uOmega}{b}\\
  \cvalidK{\OParams}{\cscenev}{\cekappa}{\kappa}\\\\
  \parseUCon{\bsubseq{b}{m}{n}}{\uc}\\
  \cExpands{\uOmega}{\uc}{c}{[\omega]\kappa}\\\\
  \uOmega=\uOmegaEx{\uD}{\uG}{\uMctx}{\Omega_\text{app}}\\
  \domof{\Omega} \cap \domof{\Omega_\text{app}} = \emptyset
}{
  \cvalidC{\Omega}{\cscenev}{\acesplicedc{m}{n}{\cekappa}}{c}{\kappa}
}
\end{equation*}
\begin{equation*}\tag{\ref{rule:cvalidE-P-splicede}}
\inferrule{
  \escenev = \esceneP{\omega : \OParams}{\uOmega}{\uPsi}{\uPhi}{b}\\
  \cvalidC{\OParams}{\csfrom{\escenev}}{\ctau}{\tau}{\akty}\\\\
  \parseUExp{\bsubseq{b}{m}{n}}{\ue}\\
  \expandsP{\uOmega}{\uPsi}{\uPhi}{\ue}{e}{[\omega]\tau}\\\\
  \uOmega=\uOmegaEx{\uD}{\uG}{\uMctx}{\Omega_\text{app}}\\
  \domof{\Omega} \cap \domof{\Omega_\text{app}} = \emptyset
}{
  \cvalidEP{\Omega}{\escenev}{\acesplicede{m}{n}{\ctau}}{e}{\tau}
}
\end{equation*}
\begin{equation*}\tag{\ref{rule:cvalidPP-spliced}}
\inferrule{
  \cvalidC{\OParams}{\csceneP{\omega : \OParams}{\uOmega}{b}}{\ctau}{\tau}{\akty}\\
  \parseUPat{\bsubseq{b}{m}{n}}{\upv}\\
  \patExpandsP{\uOmega'}{\uPhi}{\upv}{p}{[\omega]\tau}
}{
  \cvalidPP{\uOmega'}{\psceneP{\omega : \Omega_\text{params}}{\uOmega}{\uPhi}{b}}{\acesplicedp{m}{n}{\ctau}}{p}{\tau}
}
\end{equation*}


Notice that the kind/type annotations on spliced terms can refer to the provided parameters, but not to bindings local to the expansion. The parameter substitution, $\omega$, must be applied after expanding the annotations because the parameter names are not bound at the application site.

\subsection{Metatheory}
A more detailed account of the metatheory is given in Appendix \ref{appendix:metatheory-P}. We will summarize the key theorems below.

\subsubsection{TLM Expression Evaluation}
The following theorems establish a notion of TLM type safety based on preservation and progress for TLM expression evaluation.

\begingroup
\def\thetheorem{\ref{thm:peTLM-preservation}}
\begin{theorem}[peTLM Preservation]
% \label{thm:peTLM-preservation}
If $\hastsmtypeExp{\Omega}{\Psi}{\epsilon}{\rho}$ and $\tsmexpStepsExp{\Omega}{\Psi}{\epsilon}{\epsilon'}$ then $\hastsmtypeExp{\Omega}{\Psi}{\epsilon'}{\rho}$.
\end{theorem}
\endgroup

\begingroup
\def\thetheorem{\ref{thm:ppTLM-preservation}}
\begin{theorem}[ppTLM Preservation]
% \label{thm:ppTLM-preservation}
If $\hastsmtypePat{\Omega}{\Phi}{\epsilon}{\rho}$ and $\tsmexpStepsPat{\Omega}{\Phi}{\epsilon}{\epsilon'}$ then $\hastsmtypePat{\Omega}{\Phi}{\epsilon'}{\rho}$.
\end{theorem}
\endgroup

\begingroup
\def\thetheorem{\ref{thm:peTLM-progress}}
\begin{theorem}[peTLM Progress]
% \label{thm:peTLM-progress}
If $\hastsmtypeExp{\Omega}{\Psi}{\epsilon}{\rho}$ then either $\tsmexpStepsExp{\Omega}{\Psi}{\epsilon}{\epsilon'}$ for some $\epsilon'$ or $\tsmexpNormalExp{\Omega}{\Psi}{\epsilon}$.
\end{theorem}
\endgroup

\begingroup
\def\thetheorem{\ref{thm:ppTLM-progress}}
\begin{theorem}[ppTLM Progress]
% \label{thm:ppTLM-progress}
If $\hastsmtypePat{\Omega}{\Phi}{\epsilon}{\rho}$ then either $\tsmexpStepsPat{\Omega}{\Phi}{\epsilon}{\epsilon'}$ for some $\epsilon'$ or $\tsmexpNormalPat{\Omega}{\Phi}{\epsilon}$.
\end{theorem}
\endgroup

\subsubsection{Typed Expansion}
There are also a number of theorems that establish that typed expansion generates a well-typed expansion.

The top-level theorem is the typed expansion theorem for modules. 

\begingroup
\def\thetheorem{\ref{thm:module-expansion-P}}
\begin{theorem}[Module Expansion]
% \label{thm:module-expansion-P}
If $\mExpandsP{\uOmegaEx{\uD}{\uG}{\uMctx}{\Omega}}{\uPsi}{\uPhi}{\uM}{M}{\sigma}$ then $\hassig{\Omega}{M}{\sigma}$.
\end{theorem}
\endgroup

(The proof of this theorem requires proving the corresponding theorems about the other typed expansion judgements, as well as the proto-expansion validation judgements -- see Appendix \ref{appendix:metatheory-P}.)

\subsubsection{peTLM Abstract Reasoning Principles}
The following theorem summarizes the abstract reasoning principles available to programmers when applying a peTLM. Descriptions of labeled clauses are given inline.

\begingroup
\def\thetheorem{\ref{thm:petsm-abstract-reasoning-principles}}
\begin{theorem}[peTLM Abstract Reasoning Principles]
If $\expandsP{\uOmega}{\uPsi}{\uPhi}{\utsmap{\uepsilon}{b}}{e}{\tau}$ then:
\begin{enumerate}
	\item $\uOmega=\uOmegaEx{\uD}{\uG}{\uMctx}{\Omega_\text{app}}$
	\item $\uPsi=\uAS{\uA}{\Psi}$
  \item (\textbf{Typing 1}) $\tsmexpExpandsExp{\uOmega}{\uPsi}{\uepsilon}{\epsilon}{\aetype{\tau'}}$ and $\hastypeP{\Omega_\text{app}}{e}{\tau'}$ for $\tau'$ such that $\issubtypeP{\Omega_\text{app}}{\tau'}{\tau}$
		\begin{quote}
			The type of the expansion is consistent with the type annotation on the peTLM definition.
		\end{quote}
	\item $\tsmexpEvalsExp{\Omega_\text{app}}{\Psi}{\epsilon}{\epsilon_\text{normal}}$
	\item $\tsmdefof{\epsilon_\text{normal}}=a$
	\item $\Psi = \Psi', \petsmdefn{a}{\rho}{\eparse}$
	\item $\encodeBody{b}{\ebody}$
  	\item $\evalU{\ap{\eparse}{\ebody}}{\aein{\mathtt{SuccessE}}{e_\text{pproto}}}$
	\item $\decodePCEExp{e_\text{pproto}}{\pce}$
	\item $\prepce{\Omega_\text{app}}{\Psi}{\pce}{\ce}{\epsilon_\text{normal}}{\aetype{\tau_\text{proto}}}{\omega}{\Omega_\text{params}}$
	\item (\textbf{Segmentation}) $\segOKP{\OParams}{\csceneP{\omega : \OParams}{\uOmega}{b}}{\segof{\ce}}{b}$
		\begin{quote}
			The segmentation determined by the proto-expansion actually segments the literal body (i.e. each segment is in-bounds and the segments are non-overlapping and operate at consistent sorts, kinds and types.)
		\end{quote}
	\item $\cvalidEP{\Omega_\text{params}}{\esceneP{\omega : \OParams}{\uOmega}{\uPsi}{\uPhi}{b}}{\ce}{e'}{\tau_\text{proto}}$
	\item $e = [\omega]e'$
	\item $\tau = [\omega]\tau_\text{proto}$
	\item $
		\segof{\ce} = \sseq{\acesplicedk{m_i}{n_i}}{\nkind} \cup \sseq{\acesplicedc{m'_i}{n'_i}{\cekappa'_i}}{\ncon} \cup\\
					     \sseq{\acesplicede{m''_i}{n''_i}{\ctau_i}}{\nexp}
		$
	\item (\textbf{Kinding 1}) $\sseq{\kExpands{\uOmega}{\parseUKindF{\bsubseq{b}{m_i}{n_i}}}{\kappa_i}}{\nkind}$ and $\sseq{\iskind{\Omega_\text{app}}{\kappa_i}}{\nkind}$
		\begin{quote}
			Each spliced kind has a well-formed expansion at the application site.
		\end{quote}
	\item (\textbf{Kinding 2}) $\sseq{\cvalidK{\OParams}{\csceneP{\omega : \OParams}{\uOmega}{b}}{\cekappa'_i}{\kappa'_i}}{\ncon}$ and $\sseq{\iskind{\Omega_\text{app}}{[\omega]\kappa'_i}}{\ncon}$
		\begin{quote}
			Each kind annotation on a spliced construction has a well-formed expansion at the application site.
		\end{quote}
	\item (\textbf{Kinding 3}) $\sseq{\cExpands{\uOmega}{\parseUConF{\bsubseq{b}{m'_i}{n'_i}}}{c_i}{[\omega]\kappa'_i}}{\ncon}$ and $\sseq{\haskind{\Omega_\text{app}}{c_i}{[\omega]\kappa'_i}}{\ncon}$
		\begin{quote}
			Each spliced construction is well-kinded consistent with its kind annotation.
		\end{quote}
	\item (\textbf{Kinding 4}) $\sseq{\cvalidC{\OParams}{\csceneP{\omega : \OParams}{\uOmega}{b}}{\ctau_i}{\tau_i}{\akty}}{\nexp}$ and $\sseq{\haskind{\Omega_\text{app}}{[\omega]\tau_i}{\akty}}{\nexp}$
		\begin{quote}
			Each type annotation on a spliced expression has a well-formed expansion at the application site.
		\end{quote}
	\item (\textbf{Typing 2}) $\sseq{\expandsP{\uOmega}{\uPsi}{\uPhi}{\parseUExpF{\bsubseq{b}{m''_i}{n''_i}}}{e_i}{[\omega]\tau_i}}{\nexp}$ and $\sseq{\hastypeP{\Omega_\text{app}}{e_i}{[\omega]\tau_i}}{\nexp}$
		\begin{quote}
			Each spliced expression is well-typed consistent with its type annotation.
		\end{quote}
	\item (\textbf{Capture Avoidance}) $e = [\sseq{\kappa_i/k_i}{\nkind}, \sseq{c_i/u_i}{\ncon}, \sseq{e_i/x_i}{\nexp}, \omega]e''$ for some $e''$ and fresh $\sseq{k_i}{\nkind}$ and fresh $\sseq{u_i}{\ncon}$ and fresh $\sseq{x_i}{\nexp}$
		\begin{quote}
			The final expansion can be decomposed into a term with variables in place of each spliced kind, construction, expression and parameter. The expansions of these spliced kinds, constructions and expressions, as well as the provided parameters, can be substituted into this term in the standard capture avoiding manner.
		\end{quote}
	\item (\textbf{Context Independence}) \[\mathsf{fv}(e'') \subset \sseq{k_i}{\nkind} \cup \sseq{u_i}{\ncon} \cup \sseq{x_i}{\nexp} \cup \domof{\OParams}\]
		\begin{quote}
			The decomposed term is independent of the application site context.
		\end{quote}
	% $\hastypeP{\sseq{\Khyp{k_i}}{\nkind} \cup \sseq{u_i :: [\omega]\kappa'_i}{\ncon} \cup \sseq{x_i : [\omega]\tau_i}{\nexp}}{[\omega]e''}{\tau}$\todo{maybe restate this in terms of free variables of e'' here and elsewhere, because context isn't technically well-formed here?}
\end{enumerate}
\end{theorem}
\endgroup

\subsubsection{ppTLM Abstract Reasoning Principles}
The following theorem summarizes the abstract reasoning principles available to programmers when applying a ppTLM. Descriptions of labeled clauses are given inline.

\begingroup
\def\thetheorem{\ref{thm:pptsm-abstract-reasoning-principles}}
\begin{theorem}[ppTLM Abstract Reasoning Principles]
If $\patExpandsP{\uOmega'}{\uPhi}{\utsmap{\uepsilon}{b}}{p}{\tau}$ then:
\begin{enumerate}
  \item $\uOmega=\uOmegaEx{\uD}{\uG}{\uMctx}{\Omega_\text{app}}$
  \item $\uPhi=\uAS{\uA}{\Phi}$
  \item (\textbf{Typing 1}) $\tsmexpExpandsPat{\uOmega}{\uPhi}{\uepsilon}{\epsilon}{\aetype{\tau'}}$ and $\patTypePC{\Omega_\text{app}}{\uOmega'}{p}{\tau'}$ for $\tau'$ such that $\issubtypeP{\Omega_\text{app}}{\tau'}{\tau}$
  	\begin{quote}
  		The final expansion matches values of the type specified by the ppTLM's type annotation.
  	\end{quote}
  \item $\tsmexpEvalsPat{\Omega_\text{app}}{\Phi}{\epsilon}{\epsilon_\text{normal}}$
  \item $\tsmdefof{\epsilon_\text{normal}}=a$
  \item $\Phi = \Phi', \pptsmdefn{a}{\rho}{\eparse}$
  \item $\encodeBody{b}{\ebody}$
  \item $\evalU{\ap{\eparse}{\ebody}}{\aein{\mathtt{SuccessP}}{e_\text{pproto}}}$
  \item $\decodePCEPat{e_\text{pproto}}{\pcp}$
  \item $\prepcp{\Omega_\text{app}}{\Phi}{\pcp}{\cpv}{\epsilon_\text{normal}}{\aetype{\tau_\text{proto}}}{\omega}{\Omega_\text{params}}$
  \item (\textbf{Segmentation}) $\segOKP{\OParams}{\csceneP{\omega : \OParams}{\uOmega}{b}}{\segof{\cpv}}{b}$
  	\begin{quote}
  		The segmentation determined by the proto-expansion actually segments the literal body (i.e. each segment is in-bounds and the segments are non-overlapping and operate at consistent sorts, kinds and types.)
  	\end{quote}
    \item $\cvalidPP{\uOmega'}{\psceneP{\omega : \Omega_\text{params}}{\uOmega}{\uPhi}{b}}{\cpv}{p}{\tau_\text{proto}}$
      \item $\tau'=[\omega]\tau_\text{proto}$
	\item $
	\segof{\ce} = \sseq{\acesplicedk{m_i}{n_i}}{\nkind} \cup \sseq{\acesplicedc{m'_i}{n'_i}{\cekappa'_i}}{\ncon} \cup\\
				     \sseq{\acesplicedp{m''_i}{n''_i}{\ctau_i}}{\npat}
	$
	\item (\textbf{Kinding 1}) $\sseq{\kExpands{\uOmega}{\parseUKindF{\bsubseq{b}{m_i}{n_i}}}{\kappa_i}}{\nkind}$ and $\sseq{\iskind{\Omega_\text{app}}{\kappa_i}}{\nkind}$
		\begin{quote}
			Each spliced kind has a well-formed expansion at the application site.
		\end{quote}
	\item (\textbf{Kinding 2}) $\sseq{\cvalidK{\OParams}{\csceneP{\omega : \OParams}{\uOmega}{b}}{\cekappa'_i}{\kappa'_i}}{\ncon}$ and $\sseq{\iskind{\Omega_\text{app}}{[\omega]\kappa'_i}}{\ncon}$
		\begin{quote}
			Each kind annotation on a spliced construction has a well-formed expansion at the application site.
		\end{quote}
	\item (\textbf{Kinding 3}) $\sseq{\cExpands{\uOmega}{\parseUConF{\bsubseq{b}{m'_i}{n'_i}}}{c_i}{[\omega]\kappa'_i}}{\ncon}$ and $\sseq{\haskind{\Omega_\text{app}}{c_i}{[\omega]\kappa'_i}}{\ncon}$
		\begin{quote}
			Each spliced construction is well-kinded consistent with its kind annotation.
		\end{quote}
	\item (\textbf{Kinding 4}) $\sseq{\cvalidC{\OParams}{\csceneP{\omega : \OParams}{\uOmega}{b}}{\ctau_i}{\tau_i}{\akty}}{\npat}$ and $\sseq{\haskind{\Omega_\text{app}}{[\omega]\tau_i}{\akty}}{\npat}$
		\begin{quote}
			Each type annotation on a spliced expression has a well-formed expansion at the application site.
		\end{quote}
  \item (\textbf{Typing 2}) $\sseq{\patExpandsP{\uOmegaEx{\emptyset}{\uG_i}{\emptyset}{\Omega_i}}{\uPhi}{\parseUPatF{\bsubseq{b}{m''_i}{n''_i}}}{p_i}{[\omega]\tau_i}}{\npat}$ and $\sseq{\patTypePC{\Omega_\text{app}}{\Omega_i}{p_i}{[\omega]\tau_i}}{\npat}$
  	\begin{quote}
  		Each spliced pattern has a well-typed expansion that matches values of the type indicated by the corresponding type annotation in the splice summary.
  	\end{quote}
      \item (\textbf{No Hidden Bindings}) $\uOmega'=\uOmegaEx{\emptyset}{\biguplus_{0 \leq i < \npat} \uG_i}{\emptyset}{\bigcup_{0 \leq i < \npat} \Omega_i}$
      	\begin{quote}
      		The hypotheses generated by the TLM application are exactly those generated by the spliced patterns.
      	\end{quote}
\end{enumerate}
\end{theorem}
\endgroup


% !TEX root = ./pldi18-supplement.tex
% !TEX root = omar-thesis.tex
% \ificfp \else  
\chapter{Static Evaluation}\label{chap:static-eval}
In the previous chapters, we have assumed that the parse functions in TLM definitions are closed expanded expressions. This is unrealistic in practice -- writing a parser generally requires access to various libraries. Moreover, the parse function might itself be written more concisely using TLMs. In this chapter, we address these problems by introducing a \emph{static environment} shared between parse functions.

\section{Static Values}
Figure \ref{fig:static-module-example} shows an example of a module, \li{ParserCombos}\ificfp \else (see Sec. \ref{sec:parser-combinators})\fi. The \li{static} qualifier indicates that this module is bound for use within the parse functions of the subsequent TLM definitions.
\begin{figure}[h]
\begin{lstlisting}
static module ParserCombos = 
struct 
  type parser('c, 't) = list('c) -> list('t * list('c))
  val alt : parser('c, 't) -> parser('c, 't) -> parser('c, 't)
  (* ... *)
end

syntax $a at T by 
  static fn(b) => 
  	(* ... *) ParserCombos.alt (* ... *)
end

syntax $b at T' by 
  static fn(b) => 
    (* ... *) ParserCombos.alt (* ... *)
end

val y = (* ParserCombos CANNOT be used here *)
\end{lstlisting}
\caption{Binding a static module for use within parse functions}
\label{fig:static-module-example}
\end{figure}
\clearpage

The values that arise during the the evaluation of parse functions do not need to persist from ``compile-time'' to ``run-time'', so we do not need a full staged computation system \cite{Taha99multi-stageprogramming:}. Instead, a sequence of static bindings operates like a lexically-scoped read-evaluate-print loop (REPL), in that each static expression is evaluated immediately and the evaluated values are tracked by a \emph{static environment}.


\section{Applying TLMs Within TLM Definitions}\label{sec:tsms-for-tsms}
TLMs and TLM abbreviations can also be qualified with the \li{static} keyword, which marks them for use within subsequent static expressions and patterns. Let us consider some examples of particular relevance to TLM providers.

\subsection{Quasiquotation}
TLMs must generate values of type \li{proto_expr} or \li{proto_pat}. Constructing values of these types explicitly can have high syntactic cost. To decrease this cost, we can define TLMs that provide support for \emph{quasiquotation syntax} (similar to that built in to languages like Lisp \cite{Bawd99a} and Scala \cite{shabalin2013quasiquotes}):
\begin{lstlisting}[numbers=none]
static syntax $proto_expr at proto_expr by static fn(b) => 
  (* proto-expression quasiquotation parser here *)
end

static syntax $proto_typ at proto_typ by static fn(b) => 
  (* proto-type quasiquotation parser here *)
end
\end{lstlisting}
For example, the following expression:
\begin{lstlisting}[numbers=none]
val gx = $proto_expr `SQTg(x)EQT`
\end{lstlisting}
is more concise than its expansion:
\begin{lstlisting}[numbers=none]
val gx = App(Var 'SSTRgESTR', Var 'SSTRxESTR')
\end{lstlisting}
Anti-quotation, i.e. splicing in an expression of type \li{proto_expr} (or \li{proto_pat}), is performed by prefixing a variable or parenthesized expression with \li{%}:
\begin{lstlisting}[numbers=none]
val fgx = $proto_expr `SQTf(%EQTgxSQT)EQT`
\end{lstlisting}
The expansion of this term is:
\begin{lstlisting}[numbers=none]
val fgx = App(Var 'SSTRfESTR', gx)
\end{lstlisting}

% A similar approach can be taken for working with encodings of terms of other languages (e.g. when writing an interpretter or compiler in VerseML.)

\subsection{Grammar-Based Parser Generators}
\ificfp \else In Sec. \ref{sec:grammars}, we discussed a number of grammar-based parser generators. \fi Abstractly, a parser generator is a module matching the signature \li{PARSEGEN} defined in Figure \ref{fig:PARSEGEN}. Let us assume a module \li{P : PARSEGEN} and a grammar of spliced unexpanded expressions that have a given type annotation, \li{spliced_uexp : proto_typ -> P.grammar(proto_expr)}, in the discussion below.



\begin{figure}
\begin{lstlisting}
signature PARSEGEN = sig 
  type grammar('a)
  (* ... operations on grammars ... *)
  type parser('a) = string -> parse_result('a)
  val generate : grammar('a) -> parser('a)
end
\end{lstlisting}
\vspace{-8px}
\caption[A signature for parser generators]{A signature for parser generators. \ificfp \else The type function \li{parse_result} was defined in Figure \ref{fig:candidate-exp-verseml}.\fi}
\vspace{-8px}
\label{fig:PARSEGEN}
\end{figure}

Rather than constructing a grammar using various operations (whose specifications are elided in \li{PARSEGEN}), we wish to use a syntax for grammars that follows standard conventions. We can do so by defining a static parametric TLM \li{#\dolla#grammar}:
\begin{lstlisting}[numbers=none]
static syntax $grammar (P : PARSEGEN) 'a at P.grammar('a) by 
  static fn(b) => (* ... *)
end
\end{lstlisting}

Using these definitions, we might define a TLM for regexes (implementing a subset of the POSIX regex syntax for simplicity) as shown in Figure \ref{fig:rx-grammar-based}.

\begin{figure}[h!]
\vspace{-5px}
\begin{lstlisting}[deletekeywords={as}]
static module RS : RX = (* ... *)
static module RU = RXUtil(RS)
syntax $rx(R : RX) at R.t by static 
  P.generate ($grammar P proto_expr {|SHTML #\label{line:rx_parse_fn_start}#
    start <- ""
      EHTMLfn () => $proto_expr `SCSSR.EmptyECSS`SHTML
    start <- "(" start ")"
      EHTMLfn e => eSHTML
    token str_tok #\label{line:str_tok_start}#
      EHTMLRU.parse "SSTR[^(@$]+ESTR" (* cannot use $rx within its own def *)SHTML #\label{line:str_tok_end}#
    start <- str_tok
      EHTMLfn s => $proto_expr `SCSSR.Str %(ECSSstr_to_proto_strlit sSCSS)ECSS`SHTML
    start <- start start
      EHTMLfn e1 e2 => $proto_expr `SCSSR.Seq (%ECSSe1SCSS, %ECSSe2SCSS)ECSS`SHTML
    start <- start "|" start 
      EHTMLfn e1 e2 => $proto_expr `SCSSR.Or (%ECSSe1SCSS, %ECSSe2SCSS)ECSS`SHTML
    start <- start "*"
      EHTMLfn e => $proto_expr `SCSSR.Star %ECSSe`SHTML

    using EHTMLspliced_uexp ($proto_typ `SCSSR.tECSS`) SHTMLas spliced_rx #\label{line:splicede_using}#
    start <- "%{" spliced_rx "}" #\label{line:splicing-start}#
      EHTMLfn e => eSHTML

    using EHTMLspliced_uexp ($proto_typ `SCSSstringECSS`) SHTMLas spliced_str
    start <- "${" spliced_str "}"
      EHTMLfn e => $proto_expr `SCSSR.Str %(ECSSeSCSS)ECSS`SHTML #\label{line:splicing-end}#
  EHTML|})
end #\label{line:rx_parse_fn_end}#
\end{lstlisting}
\vspace{-12px}
\caption{A grammar-based definition of \texttt{\$rx}}
\vspace{-15px}
\label{fig:rx-grammar-based}
\end{figure}


\section{Library Management}
In the examples above, we explicitly qualified various definitions with the \li{static} keyword to make them available within static values. This captures the essential nature of the problem of static evaluation, but in practice, we would like to be able to use libraries within both static values and standard values as needed without duplicating code. This can be achieved by a library manager.

For example, a language-external library manager \ificfp for VerseML \else \fi similar to SML/NJ's CM \cite{blume:smlnj-cm} could support a \li{static} qualifier on imported libraries, which would place the definitions exported by the imported library into the static phase of the library being defined. In particular, a library definition in such a compilation manager might look like this:
\begin{lstlisting}[numbers=none,morekeywords={Library,is}]
Library 
  (* ... exported module, signature and TLM names ... *)
is 
  (* ... files defining those exports ... *)

  (* imports: *)
  static parsegen.cm 
\end{lstlisting}

A similar approach could be taken for languages the incorporate library management directly into the syntax of programs, e.g. Scala \cite{odersky2008programming}:
\begin{lstlisting}[numbers=none]
static import edu.cmu.comar.parsegen
\end{lstlisting}

\ificfp \else For the sake of generality and simplicity, we will leave the details of library and compilation management out of our formal developments (following the approach taken in the definition of Standard ML \cite{mthm97-for-dart}.) The problem of packaging macros into components has been studied for term-rewriting macros \cite{culpepper2005syntactic}. \fi

An alternative design that allows for the explicit lowering of standard-phase bindings to the static phase has been proposed for OCaml \cite{Ocaml/macros}. \ificfp We are closely following this work in our implementation. \else \fi

TLMs definitions can be exported from the top level of packages, but they cannot be exported from within ML-style modules because that would require that they also appear in signatures, and that, in turn, would complicate reasoning about signature equivalence, since TLM definitions contain arbitrary parse functions. 
%It would also bring in confusion about whether the generated expansions can use private knowledge about type identity. 
That said, it should be possible to export TLM \emph{abbreviations} from modules, since they refer to TLM definitions only through symbolic names. We have not yet formalized this intuition, but the work of \citet{culpepper2005syntactic,culpepper2007advanced} considered a closely related question: how should Typed Scheme's macros interact with its unit (i.e. package) system.


% \fi

\ificfp
\section{\texorpdfstring{Formalizing Static Evaluation}{Formalizing Static Evaluation}}
\else
\section{\texorpdfstring{$\miniVersePH$}{miniVersePH}}
\fi
\ificfp

We will now formalize the mechanisms just discussed by developing a reduced calculus, $\miniVersePH$. This calculus is defined identically to $\miniVerseParam$ with the exception of the syntax and semantics of unexpanded module expressions, $\uM$, so we assume all of the definitions that were given in Appendix \ref{appendix:miniVerseParam} without restating them. 

% We will now formalize the mechanisms outlined in Sec. 6 of the paper by developing a reduced calculus, $\miniVersePH$. This calculus is defined identically to $\miniVerseParam$ with the exception of the syntax and semantics of unexpanded module expressions, $\uM$, so we assume all of the definitions that were given in Appendix \ref{appendix:miniVerseParam} without restating them. 

\else 

We will now formalize the mechanisms just discussed by developing a reduced calculus, $\miniVersePH$. This calculus is defined identically to $\miniVerseParam$ with the exception of the syntax and semantics of unexpanded module expressions, $\uM$, so we assume all of the definitions that were given in Appendix \ref{appendix:miniVerseParam} without restating them. 

\fi

\ificfp
\subsection{Syntax of Unexpanded Modules}
\else
\subsection{Syntax of Unexpanded Modules}
\fi
The syntax of unexpanded modules is defined in Figure \ref{fig:syntax-uM-PH}. The parts of this figure that differ from \ificfp $\miniVerseParam$ \else Figure \ref{fig:P-unexpanded-modules-signatures} \fi are highlighted in yellow. Each binding form has a \emph{phase} annotation, $\varphi$, and parse functions are now unexpanded expressions, $\ue$, rather than expanded expressions, $e$. In the textual syntax, the phase annotation $\standardphase$ is assumed when no phase annotation has been given.

\begin{figure}[t]
\[\arraycolsep=3pt\begin{array}{lllllll}
\textbf{Sort} & & 
%& \textbf{Operational Form} 
& \textbf{Stylized Form} & \textbf{Description}\\
\LCC \color{Yellow} & \color{Yellow} & \color{Yellow} & \color{Yellow} & \color{Yellow}\\
\mathsf{Phase} & \varphi & ::= & \standardphase & \text{standard phase}\\
& & & \staticphase & \text{static phase}\\\ECC
\mathsf{UMod} & \uM & ::= 
%& \uX 
& \uX & \text{module identifier}\\
&&
%& \austruct{\uc}{\ue} 
& \struct{\uc}{\ue} & \text{structure}\\
&&
%& \auseal{\usigma}{\uM} 
& \seal{\uM}{\usigma} & \text{seal}\\
&&
%& \aumlet{\usigma}{\uM}{\uX}{\uM} 
& \mletH{\yellowbox{\varphi}}{\uX}{\uM}{\uM}{\usigma} & \text{definition}\\
% \LCC &&
%& \lightgray 
% & \color{Yellow} & \color{Yellow}\\
&&
%& \aumdefpetsm{\urho}{e}{\tsmv}{\uM} 
& \defpetsmH{\yellowbox{\varphi}}{\tsmv}{\urho}{\yellowbox{\ue}}{\uM} & \text{peTLM definition}\\
%&&&                                    & \texttt{expressions}~\{e\}~\texttt{in}~\uM\\
&&
%& \aumletpetsm{\uepsilon}{\tsmv}{\uM} 
& \uletpetsmH{\yellowbox{\varphi}}{\tsmv}{\uepsilon}{\uM} & \text{peTLM binding}\\
% &&&                                  & \texttt{expressions}~\texttt{in}~\uM\\
% &&& ... & ... & \text{peTLM designation}\\
&&
%& \audefpptsm{\urho}{e}{\tsmv}{\uM} 
& \defpptsmH{\yellowbox{\varphi}}{\tsmv}{\urho}{\yellowbox{\ue}}{\uM} & \text{ppTLM definition}\\
% &&&                                    & \texttt{patterns}~\{e\}~\texttt{in}~\uM\\
&&
%& \auletpptsm{\uepsilon}{\tsmv}{\uM} 
& \uletpptsmH{\yellowbox{\varphi}}{\tsmv}{\uepsilon}{\uM} & \text{ppTLM binding}%\ECC%
% &&& & \texttt{patterns}~\texttt{in}~\uM\\
% &&& ... & ... & \text{ppTLM designation}\ECC
\end{array}\]
\caption{Syntax of unexpanded modules in $\miniVersePH$}
\label{fig:syntax-uM-PH}
\end{figure}

\ificfp
\subsection{Module Expansion}
\else
\subsection{Module Expansion}
\fi
The module expansion judgement in $\miniVersePH$ takes the following form:

\vspace{10px}
$\begin{array}{ll}
\textbf{Judgement Form} & \textbf{Description}\\
\mExpandsPHX{\uM}{M}{\sigma} & \text{$\uM$ has expansion $M$ matching $\sigma$}
\end{array}$
\vspace{10px}

The difference here is that there is now a \emph{static environment}, $\Sigma$. Static environments take the form $\staticenv{\omega}{\uOmega}{\uPsi}{\uPhi}$, where $\omega$ is a substitution.

The static environment passes opaquely through the subsumption rule and the rules governing module identifiers, structures and sealing:
\begin{subequations}\label{rules:mExpandsPH}
\begin{equation}\label{rule:mExpandsPH-subsumes}
\inferrule{
  \mExpandsPHX{\uM}{M}{\sigma}\\
  \sigsub{\uOmega}{\sigma}{\sigma'}
}{
  \mExpandsPHX{\uM}{M}{\sigma'}
}
\end{equation}
\begin{equation}\label{rule:mExpandsPH-var}
\inferrule{ }{
  \mExpandsPH{\uOmega, \uMhyp{\uX}{X}{\sigma}}{\uPsi}{\uPhi}{\uX}{X}{\sigma}{\Sigma}
}
\end{equation}
\begin{equation}\label{rule:mExpandsPH-struct}
\inferrule{
  \cExpandsX{\uc}{c}{\kappa}\\
  \expandsPX{\ue}{e}{[c/u]\tau}
}{
  \mExpandsPHX{\struct{\uc}{\ue}}{\astruct{c}{e}}{\asignature{\kappa}{u}{\tau}}
}
\end{equation}
\begin{equation}\label{rule:mExpandsPH-seal}
\inferrule{
  \sigExpandsPX{\usigma}{\sigma}\\
  \mExpandsPHX{\uM}{M}{\sigma}
}{
  \mExpandsPHX{\seal{\uM}{\usigma}}{\aseal{\sigma}{M}}{\sigma} 
}
\end{equation}

Each binding form in the syntax of $\uM$ is qualified with a \emph{phase}, $\varphi$, which is either $\standardphase$ or $\staticphase$. The static environment passes opaquely through the standard phase module let binding construct:
\begin{equation}\label{rule:mExpandsPH-mlet-standard}
\inferrule{
  \mExpandsPHX{\uM}{M}{\sigma}\\
  \sigExpandsPX{\usigma'}{\sigma'}\\\\
  \mExpandsPH{\uOmega, \uMhyp{\uX}{X}{\sigma}}{\uPsi}{\uPhi}{\uM'}{M'}{\sigma'}{\Sigma}
}{
  \mExpandsPHX{\mletH{\standardphase}{\uX}{\uM}{\uM'}{\usigma'}}{\amlet{\sigma'}{M}{X}{M'}}{\sigma'}
}
\end{equation}

The rule for the $\staticphase$ phase module let binding construct, on the other hand, calls for the module expression being bound to be evaluated to a module value under the current environment. The substitution and corresponding unexpanded context is then extended with this module value:
\begin{equation}\label{rule:mExpandsPH-mlet-static}
\inferrule{
  \Sigma = \staticenv{\omega}{\uOmega_S}{\uPsi_S}{\uPhi_S}\\\\
  \mExpandsPH{\uOmega_S}{\uPsi_S}{\uPhi_S}{\uM}{M}{\sigma}{\Sigma}\\
  \evalU{[\omega]M}{M'}\\\\
  \sigExpandsP{\uOmega}{\usigma'}{\sigma'}\\
  \mExpandsPH{\uOmega}{\uPsi}{\uPhi}{\uM'}{M'}{\sigma'}{
  	\staticenv{\omega, M'/X}{\uOmega_S, \uMhyp{\uX}{X}{\sigma}}{\uPsi_S}{\uPhi_S}
  }
}{
  \mExpandsPH{\uOmega}{\uPsi}{\uPhi}{\mletH{\staticphase}{\uX}{\uM}{\uM'}{\usigma'}}{M'}{\sigma'}{\Sigma}
}
\end{equation}

The standard peTLM definition construct is governed by the following rule:
\begin{equation}\label{rule:mExpandsPH-syntaxpe-standard}
\inferrule{
  \tsmtyExpands{\uOmega}{\urho}{\rho}\\
  \Sigma = \staticenv{\omega}{\uOmega_S}{\uPsi_S}{\uPhi_S}\\\\
  \expandsP{\uOmega_S}{\uPsi_S}{\uPhi_S}{\ueparse}{\eparse}{\aparr{\tBody}{\tParseResultPCEExp}}\\\\
  \evalU{[\omega]\eparse}{\eparse'}\\
  \mExpandsPH{\uOmega}{\uAS{\uA \uplus \mapitem{\tsmv}{\adefref{a}}}{\Psi, \petsmdefn{a}{\rho}{\eparse'}}}{\uPhi}{\uM}{M}{\sigma}{\Sigma}
}{
  \mExpandsPH{\uOmega}{\uAS{\uA}{\Psi}}{\uPhi}{\defpetsmH{\standardphase}{\tsmv}{\urho}{\ueparse}{\uM}}{M}{\sigma}{\Sigma}
}
\end{equation}
The difference here is that the parse function is an unexpanded (rather than an expanded) expression. It is expanded under the static environment's unified context, $\uOmega_S$. Then the substitution, $\omega$, is applied to the resulting expanded parse function before it is added to the peTLM context.

The static peTLM definition construct operates similarly, differing only in that the static environment's peTLM context is extended, rather than the standard peTLM context:
\begin{equation}\label{rule:mExpandsPH-syntaxpe-static}
\inferrule{
  \tsmtyExpands{\uOmega}{\urho}{\rho}\\
  \Sigma = \staticenv{\omega}{\uOmega_S}{\uPsi_S}{\uPhi_S}\\
  \uPsi_S = \uAS{\uA_S}{\Psi_S}\\\\
  \expandsP{\uOmega_S}{\uPsi_S}{\uPhi_S}{\ueparse}{\eparse}{\aparr{\tBody}{\tParseResultPCEExp}}\\\\
  \evalU{[\omega]\eparse}{\eparse'}\\
  \mExpandsPH{\uOmega}{\uPsi}{\uPhi}{\uM}{M}{\sigma}{\staticenv{\omega}{\uOmega_S}{\uAS{\uA_S \uplus \mapitem{\tsmv}{\adefref{a}}}{\Psi_S, \petsmdefn{a}{\rho}{\eparse'}}}{\uPhi_S}}
}{
  \mExpandsPH{\uOmega}{\uPsi}{\uPhi}{\defpetsmH{\staticphase}{\tsmv}{\urho}{\ueparse}{\uM}}{M}{\sigma}{\Sigma}
}
\end{equation}


The static environment passes opaquely through the standard peTLM abbreviation construct:
\begin{equation}\label{rule:mExpandsPH-letpetsm-standard}
\inferrule{
  \tsmexpExpandsExp{\uOmega}{\uAS{\uA}{\Psi}}{\uepsilon}{\epsilon}{\rho}\\
  \mExpandsPH{\uOmega}{\uAS{\uA\uplus\mapitem{\tsmv}{\epsilon}}{\Psi}}{\uPhi}{\uM}{M}{\sigma}{\Sigma}
}{
  \mExpandsPH{\uOmega}{\uAS{\uA}{\Psi}}{\uPhi}{\uletpetsmH{\standardphase}{\tsmv}{\uepsilon}{\uM}}{M}{\sigma}{\Sigma}
}
\end{equation}

The static peTLM abbreviation construct updates the static peTLM identifier expansion context, $\uA_S$:
\begin{equation}\label{rule:mExpandsPH-letpetsm-static}
\inferrule{
  \tsmexpExpandsExp{\uOmega}{\uAS{\uA}{\Psi}}{\uepsilon}{\epsilon}{\rho}\\\\
  \Sigma = \staticenv{\omega}{\uOmega_S}{\uPsi_S}{\uPhi_S}\\
  \uPsi_S = \uAS{\uA_S}{\Psi_S}\\\\
  \mExpandsPH{\uOmega}{\uPsi}{\uPhi}{\uM}{M}{\sigma}{\staticenv{\omega}{\Omega_S}{\uAS{\uA_S\uplus\mapitem{\tsmv}{\epsilon}}{\Psi_S}}{\uPhi_S}}
}{
  \mExpandsPH{\uOmega}{\uPsi}{\uPhi}{\uletpetsmH{\staticphase}{\tsmv}{\uepsilon}{\uM}}{M}{\sigma}{\Sigma}
}
\end{equation}

The rules governing ppTLM definitions and abbreviations are analagous:
\begin{equation}\label{rule:mExpandsPH-syntaxpp-standard}
\inferrule{ 
  \tsmtyExpands{\uOmega}{\urho}{\rho}\\
    \Sigma = \staticenv{\omega}{\uOmega_S}{\uPsi_S}{\uPhi_S}\\\\
  \expandsP{\uOmega_S}{\uPsi_S}{\uPhi_S}{\ueparse}{\eparse}{\aparr{\tBody}{\tParseResultCEPat }}\\\\
  \evalU{[\omega]\eparse}{\eparse'}\\
  \mExpandsPH{\uOmega}{\uPsi}{\uAS{\uA \uplus \mapitem{\tsmv}{\adefref{a}}}{\Phi, \pptsmdefn{a}{\rho}{\eparse'}}}{\uM}{M}{\sigma}{\Sigma}
}{
  \mExpandsPH{\uOmega}{\uPsi}{\uAS{\uA}{\Phi}}{\defpptsmH{\standardphase}{\tsmv}{\urho}{\ueparse}{\uM}}{M}{\sigma}{\Sigma}
}
\end{equation}
\begin{equation}\label{rule:mExpandsPH-syntaxpp-static}
\inferrule{ 
  \tsmtyExpands{\uOmega}{\urho}{\rho}\\
  \Sigma = \staticenv{\omega}{\uOmega_S}{\uPsi_S}{\uPhi_S}\\
  \uPhi_S = \uAS{\uA_S}{\Phi_S}\\\\
  \expandsP{\uOmega_S}{\uPsi_S}{\uPhi_S}{\ueparse}{\eparse}{\aparr{\tBody}{\tParseResultCEPat }}\\\\
  \evalU{[\omega]\eparse}{\eparse'}\\
  \mExpandsPH{\uOmega}{\uPsi}{\uPhi}{\uM}{M}{\sigma}{
  	\staticenv{\omega}{\uOmega_S}{\uPsi_S}{\uAS{\uA_S \uplus \mapitem{\tsmv}{\adefref{a}}}{\Phi_S, \pptsmdefn{a}{\rho}{\eparse}}}
  }
}{
  \mExpandsPH{\uOmega}{\uPsi}{\uPhi}{\defpptsmH{\staticphase}{\tsmv}{\urho}{\ueparse}{\uM}}{M}{\sigma}{\Sigma}
}
\end{equation}
\begin{equation}\label{rule:mExpandsPH-letpptsm-standard}
\inferrule{
  \tsmexpExpandsPat{\uOmega}{\uAS{\uA}{\Phi}}{\uepsilon}{\epsilon}{\rho}\\
  \mExpandsPH{\uOmega}{\uPsi}{\uAS{\uA\uplus\mapitem{\tsmv}{\epsilon}}{\Phi}}{\uM}{M}{\sigma}{\Sigma}
}{
  \mExpandsPH{\uOmega}{\uPsi}{\uAS{\uA}{\Phi}}{\uletpptsmH{\standardphase}{\tsmv}{\uepsilon}{\uM}}{M}{\sigma}{\Sigma}
}
\end{equation}
\begin{equation}\label{rule:mExpandsPH-letpptsm-static}
\inferrule{
  \tsmexpExpandsPat{\uOmega}{\uAS{\uA}{\Phi}}{\uepsilon}{\epsilon}{\rho}\\\\
  \Sigma = \staticenv{\omega}{\uOmega_S}{\uPsi_S}{\uPhi_S}\\
  \uPhi_S = \uAS{\uA_S}{\Phi_S}\\\\
    \mExpandsPH{\uOmega}{\uPsi}{\uPhi}{\uM}{M}{\sigma}{
    	\staticenv{\omega}{\uOmega_S}{\uPsi_S}{\uAS{\uA_S\uplus\mapitem{\tsmv}{\epsilon}}{\Phi_S}}
    }
}{
  \mExpandsPH{\uOmega}{\uPsi}{\uPhi}{\uletpptsmH{\staticphase}{\tsmv}{\uepsilon}{\uM}}{M}{\sigma}{\Sigma}
}
\end{equation}
\end{subequations}

\ificfp
\subsection{Metatheory}
\else
\subsection{Metatheory}
\fi
The metatheorem having to do with unexpanded module expressions was the Module Expansion theorem, Theorem \ref{thm:module-expansion-P}. This theorem continues to hold in $\miniVersePH$.
\begin{theorem}[Module Expansion]
\label{thm:module-expansion-PH}
If $\mExpandsPH{\uOmegaEx{\uD}{\uG}{\uMctx}{\Omega}}{\uPsi}{\uPhi}{\uM}{M}{\sigma}{\Sigma}$ then $\hassig{\Omega}{M}{\sigma}$.
\end{theorem}
\begin{proof} 
By rule induction over Rules (\ref{rules:mExpandsPH}). In the following, let $\uOmega=\uOmegaEx{\uD}{\uG}{\uMctx}{\Omega}$.

\begin{byCases}
  \item[\text{(\ref{rule:mExpandsPH-subsumes})}] ~
    \begin{pfsteps*}
      \item $\mExpandsPX{\uM}{M}{\sigma'}$ \BY{assumption} \pflabel{mexpands}
      \item $\sigsub{\uOmega}{\sigma'}{\sigma}$ \BY{assumption} \pflabel{sigsub}
      \item $\hassig{\Omega}{M}{\sigma'}$ \BY{IH on \pfref{mexpands}} \pflabel{hassig}
      \item $\hassig{\Omega}{M}{\sigma}$ \BY{Rule (\ref{rule:hassig-subsume}) on \pfref{hassig} and \pfref{sigsub}}
    \end{pfsteps*}
    \resetpfcounter
  \item[\text{(\ref{rule:mExpandsPH-var}) \textbf{through} (\ref{rule:mExpandsPH-mlet-static})}] In each of these cases, we apply the IH over each module expansion premise, Theorem \ref{thm:typed-expression-expansion-P} over each expression expansion premise and Theorem \ref{thm:kind-and-constructor-expansion-P} over each construction expansion premise, then apply the corresponding signature matching rule in Rules (\ref{rules:hassig}) and weakening as needed.
  \item[\text{(\ref{rule:mExpandsPH-syntaxpe-standard}) \textbf{through} (\ref{rule:mExpandsPH-letpptsm-static})}] In each of these cases, we apply the IH to the module expansion premise.
\end{byCases}
\end{proof}
The rest of the metatheory is identical to that of $\miniVerseParam$.


% \part{TLM Implicits}\label{part:implicits}
% !TEX root = omar-thesis.tex
\chapter{TLM Implicits}\label{chap:tsls}
When applying a TLM, library clients must explicitly prefix each literal form with a TLM name and, in some cases, several parameters. In situations where the client is repeatedly applying a TLM to small literal forms, this can itself be costly. For example, list literals are often small, so applying \li{#\dolla#intlist} repeatedly can be distracting and syntactically costly.

To further lower the syntactic cost of using TLMs, so that it compares to the syntactic cost of using derived forms built primitively into a language, VerseML allows clients to designate, for any type, one expression TLM and one pattern TLM as that type's \emph{designated TLMs} within a delimited scope. When VerseML's \emph{local type inference} system encounters a generalized literal form not prefixed by a TLM name (an \emph{unadorned literal form}), it implicitly applies the TLM designated at the type that the expression or pattern is being checked against. 
This chapter will introduce {TLM implicits} first by example in Sec. \ref{sec:tsm-implicits-by-example} and then formally in Sec. \ref{sec:b-miniverse}. %Simple TLM implicits operate at a single specified type. In the next chapter, we will consider \emph{parametric TLM implicits}, which operate across a parameterized family of types.

\section{TLM Implicits By Example}\label{sec:tsm-implicits-by-example}
\subsection{Designation and Usage}
On Lines 1-2 of Figure \ref{fig:implicits-example}, the client has \emph{designated} the expression TLM \li{#\dolla#rx} for implicit application to \emph{unadorned literal forms} being checked against type \li{rx}, like the unadorned literal form on Line 5. 

Similarly, on Line 3 of Figure \ref{fig:implicits-example} the client has designated the pattern TLM \li{#\dolla#rx} for implicit application to unadorned pattern literal forms matching values of type \li{rx}, like the pattern form on Line 8.

\begin{figure}[t]
\begin{lstlisting}
implicit syntax 
  $rx at rx for expressions
  $rx at rx for patterns
in
  val ssn : rx = /SURL\d\d\d-\d\d-\d\d\d\dEURL/
  fun name_from_example_rx(r : rx) : option(string) => 
    match r with 
      /SURL@EURLnameSURL: %EURL_/ => Some name
    | _ => None
end
\end{lstlisting}
\caption{An example of simple TLM implicits in VerseML}
\label{fig:implicits-example}
\end{figure}


Type annotations on TLM designations are technically redundant -- the definition of the designated TLM determines the designated type. Annotations are included in our examples for readability.

Expression and pattern TLMs need not be designated together, nor have the same name if they are. However, this is a common idiom, so for convenience, VerseML also provides a derived designation form that combines the two designations in Figure \ref{fig:implicits-example}:
\begin{lstlisting}[numbers=none]
implicit syntax $rx at rx in (* ... *) end 
\end{lstlisting}



\subsection{Analytic and Synthetic Positions}
During typed expansion of a subexpression, $e'$, of an expresssion, $e$, we say that $e'$ appears in \emph{analytic position} if the type that $e'$ must have is determined by  the surrounding context and its position within $e$. For example, an expression appearing as a function argument is in analytic position because the function's type determines the argument's type. Similarly, an expression may appear in analytic position due to a \emph{type ascription}, either directly on the expression, or on a binding, as on Line 5 above.

If the type that $e'$ must be assigned is not determined by the surrounding context -- i.e. $e'$ must be examined to synthesize its type -- we instead say that the expression appears in a \emph{synthetic position}. For example, a top-level expression, or an expression being bound without a type ascription, appears in synthetic position.

An expression of unadorned literal form is valid only in analytic position, because its type must be known to be able to determine the designated TLM that will control its expansion. For example, typed expansion of the following expression will fail because an expression of unadorned literal form appears in synthetic position:
\begin{lstlisting}[numbers=none]
val ssn = /SURL\d\d\d-\d\d-\d\d\d\dEURL/ (* INVALID *)
\end{lstlisting}

Patterns can always be of unadorned literal form in VerseML, because the scrutinee of a match expression is always in synthetic position, and so the type of value that each pattern appearing within the match expression must match is always known. 

\section{\texorpdfstring{Bidirectional $\miniVersePat$}{Bidirectional miniVerseS}}\label{sec:b-miniverse}
To formalize TLM implicits, we will now develop a reduced calculus called \emph{Bidirectional $\miniVersePat$}. The full definition of this calculus is given in Appendix \ref{appendix:simple-implicits}. We choose to base our calculus on the simpler $\miniVersePat$ calculus, rather than $\miniVerseParam$, to communicate the essential character of TLM implicits. Section \ref{sec:parametric-simple-implicits} briefly considers the small changes that would be necessary to incorporate the same mechanism into a bidirectionally typed variant of $\miniVerseParam$.

\subsection{Expanded Language}
The Bidirectional $\miniVersePat$ expanded language (XL) is the same as the  $\miniVersePat$ XL, which was described in Sections \ref{sec:inner-core-syntax-UP} through \ref{sec:dynamics-UP}. %It consists of types, $\tau$, expanded expressions, $e$, expanded rules, $r$, and expanded patterns, $p$.

\subsection{Syntax of the Unexpanded Language}

\begin{figure}
\[\begin{array}{lllllll}
\textbf{Sort} & & 
%& \textbf{Operational Form} 
& \textbf{Stylized Form} & \textbf{Description}\\
\mathsf{UTyp} & \utau & ::= 
%& \cdots 
& \cdots & \text{(as in $\miniVersePat$)}\\
\mathsf{UExp} & \ue & ::= 
%& \cdots 
& \cdots & \text{(as in $\miniVersePat$)}\\
% &&
%& \auasc{\utau}{\ue} 
% & \asc{\ue}{\utau} & \text{ascription}\\
% &&
%& \auletsyn{\ux}{\ue}{\ue} 
% & \letsyn{\ux}{\ue}{\ue} & \text{value binding}\\
&&& \implicite{\tsmv}{\ue} & \text{seTLM designation}\\
&&& \implicitp{\tsmv}{\ue} & \text{spTLM designation}\\
&&& \lit{b} & \text{seTLM unadorned literal}\\
% &&& \auanalam{\ux}{\ue} & \analam{\ux}{\ue} & \text{abstraction (unannotated)}\\
% &&& \aulam{\utau}{\ux}{\ue} & \lam{\ux}{\utau}{\ue} & \text{abstraction (annotated)}\\
% &&& \auap{\ue}{\ue} & \ap{\ue}{\ue} & \text{application}\\
% &&& \autlam{\ut}{\ue} & \Lam{\ut}{\ue} & \text{type abstraction}\\
% &&& \autap{\ue}{\utau} & \App{\ue}{\utau} & \text{type application}\\
% &&& \auanafold{\ue} & \fold{\ue} & \text{fold}\\
% &&& \auunfold{\ue} & \unfold{\ue} & \text{unfold}\\
% &&& \autpl{\labelset}{\mapschema{\ue}{i}{\labelset}} & \tpl{\mapschema{\ue}{i}{\labelset}} & \text{labeled tuple}\\
% &&& \aupr{\ell}{\ue} & \prj{\ue}{\ell} & \text{projection}\\
% &&& \auanain{\ell}{\ue} & \inj{\ell}{\ue} & \text{injection}\\
% &&& \aumatchwithb{n}{\ue}{\seqschemaX{\urv}} & \matchwith{\ue}{\seqschemaX{\urv}} & \text{match}\\
% &&& \audefuetsm{\utau}{e}{\tsmv}{\ue} & \texttt{syntax}~\tsmv~\texttt{at}~\utau~\texttt{for} & \text{ueTLM definition}\\
% &&&                                    & \texttt{expressions}~\{e\}~\texttt{in}~\ue\\
% \LCC &&& \lightgray & \lightgray & \lightgray \\
% &&& \auimplicite{\tsmv}{\ue} & \texttt{implicit\,syntax}~\tsmv~\texttt{for} & \text{ueTLM designation}\\
% &&&                          & \texttt{expressions\,in}~\ue\\ \ECC
% &&& \autsmap{b}{\tsmv} & \utsmap{\tsmv}{b} & \text{ueTLM application}\\%\ECC
% \LCC &&& \lightgray & \lightgray & \lightgray \\
% &&& \auelit{b} & {\lit{b}}  & \text{ueTLM unadorned literal}\\\ECC
% &&& \audefuptsm{\utau}{e}{\tsmv}{\ue} & \texttt{syntax}~\tsmv~\texttt{at}~\utau~\texttt{for} & \text{upTLM definition}\\
% &&&                                    & \texttt{patterns}~\{e\}~\texttt{in}~\ue\\
% \LCC &&& \lightgray & \lightgray & \lightgray \\
% &&& \auimplicitp{\tsmv}{\ue} & \texttt{implicit\,syntax}~\tsmv~\texttt{for} & \text{upTLM designation}\\
% &&&                          & \texttt{patterns\,in}~\ue\\ \ECC
\mathsf{URule} & \urv & ::= 
%& \aumatchrule{\upv}{\ue} 
& \cdots & \text{(as in $\miniVersePat$)}\\
\mathsf{UPat} & \upv & ::= 
%& \ux 
& \cdots & \text{(as in $\miniVersePat$)}\\
&&& \lit{b} & \text{spTLM unadorned literal}
% &&& \auwildp & \wildp & \text{wildcard pattern}\\
% &&& \aufoldp{\upv} & \foldp{\upv} & \text{fold pattern}\\
% &&& \autplp{\labelset}{\mapschema{\upv}{i}{\labelset}} & \tplp{\mapschema{\upv}{i}{\labelset}} & \text{labeled tuple pattern}\\
% &&& \auinjp{\ell}{\upv} & \injp{\ell}{\upv} & \text{injection pattern}\\
% &&& \auapuptsm{b}{\tsmv} & \utsmap{\tsmv}{b} & \text{upTLM application}\\
% \LCC &&& \lightgray & \lightgray & \lightgray\\
% &&& \auplit{b} & \lit{b} & \text{upTLM unadorned literal}\ECC
\end{array}\]\vspace{-8px}
\caption[Syntax of unexpanded terms in Bidirectional $\miniVersePat$]{Syntax of unexpanded terms in Bidirectional $\miniVersePat$}
\vspace{-5px}
\label{fig:B-unexpanded-terms}
\end{figure}
The syntax of the Bidirectional $\miniVersePat$ unexpanded language (UL) extends the syntax of the $\miniVersePat$ UL as shown in Figure \ref{fig:B-unexpanded-terms}.

As in $\miniVersePat$, there is also a textual syntax for the UL, characterized by the following condition:
\begingroup
\def\thetheorem{\ref{condition:textual-representability-BS}}
\begin{condition}[Textual Representability] ~
\begin{enumerate}
\item For each $\utau$, there exists $b$ such that $\parseUTyp{b}{\utau}$. 
\item For each $\ue$, there exists $b$ such that $\parseUExp{b}{\ue}$.
% \item For each $\urv$, there exists $b$ such that $\parseURule{b}{\urv}$.
\item {For each $\upv$, there exists $b$ such that $\parseUPat{b}{\upv}$.}
\end{enumerate}
\end{condition}
\endgroup

% Each inner core form (defined in Figure \ref{fig:UP-expanded-terms}) maps onto an outer surface form. In particular:
% \begin{itemize}
% \item Each type variable, $t$, maps onto a unique {type sigil}, written $\sigilof{t}$. %(pronounced ``sigil of $t$''). %Notice the distinction between $\ut$, which is a metavariable ranging over type sigils, and $\sigilof{t}$, which is a metafunction, written in stylized form, applied to a type variable to produce a type sigil.
% \item Each type form, $\tau$, maps onto an unexpanded type form, $\Uof{\tau}$, according to the definition of $\Uof{\tau}$ in Sec. \ref{sec:syntax-U}.
% \item Each expression variable, $x$, maps onto a unique expression sigil, written $\sigilof{x}$. %Again, notice the distinction between $\ux$ and $\sigilof{x}$.
% \item Each expanded expression form, $e$, maps onto an unexpanded expression form $\Uof{e}$ as follows:
% \begin{align*}
% \Uof{x} & = \sigilof{x}\\
% \Uof{\aelam{\tau}{x}{e}} & = \aulam{\Uof{\tau}}{\sigilof{x}}{\Uof{e}}\\
% \Uof{\aeap{e_1}{e_2}} & = \auap{\Uof{e_1}}{\Uof{e_2}}\\
% \Uof{\aetlam{t}{e}} & = \autlam{\sigilof{t}}{\Uof{e}}\\
% \Uof{\aetap{e}{\tau}} & = \autap{\Uof{e}}{\Uof{\tau}}\\
% \Uof{\aefold{t}{\tau}{e}} & = \auasc{\aurec{\sigilof{t}}{\Uof{\tau}}}{\auanafold{\Uof e}}\\
% \Uof{\aeunfold{e}} & = \auunfold{\Uof{e}}\\
% \Uof{\aetpl{\labelset}{\mapschema{e}{i}{\labelset}}} & = \autpl{\labelset}{\mapschemax{\Uofv}{e}{i}{\labelset}}\\
% \Uof{\aein{\ell}{e}} &= \auasc{\ausum{\labelset}{\mapschemax{\Uofv}{\tau}{i}{\labelset}}}{\auanain{\ell}{\Uof{e}}}\\
% \Uof{\aematchwith{n}{\tau}{e}{\seqschemaX{r}}} &= \auasc{{\Uof{\tau}}}{\aumatchwithb{n}{\Uof{e}}{\seqschemaXx{\Uofv}{r}}}\\
% \end{align*}
% Notice that some type arguments that appear in $e$ appear within a type ascription in $\Uof{e}$. 
% \item The expanded rule form maps onto the unexpanded rule form as follows:
% \begin{align*}
% \Uof{\aematchrule{p}{e}} & = \aumatchrule{\Uof{p}}{\Uof{e}}
% \end{align*}
% \item Each expanded pattern form, $p$, maps onto the unexpanded pattern form $\Uof{p}$ as follows:
% \begin{align*}
% \Uof{x} & = \sigilof{x}\\
% \Uof{\aewildp} &= \auwildp\\
% \Uof{\aefoldp{p}} &= \aufoldp{\Uof{p}}\\
% \Uof{\aetplp{\labelset}{\mapschema{p}{i}{\labelset}}} & = \autplp{\labelset}{\mapschemax{\Uofv}{p}{i}{\labelset}}\\
% \Uof{\aeinjp{\ell}{p}} & = \auinjp{\ell}{\Uof{p}}
% \end{align*}
% \end{itemize}

% %Eight unexpanded forms relate to TLMs: the unexpanded expression forms for ueTLM definition, ueTLM designation, ueTLM application, ueTLM unadorned literals, upTLM definition and upTLM designation, and the unexpanded pattern forms for upTLM application and upTLM undorned literals. 
% The forms related to TLM implicits are highlighted in gray in Figure \ref{fig:B-unexpanded-terms}.

\subsection{Bidirectionally Typed Expansion}
Unexpanded terms are checked and expanded simultaneously according to the \emph{bidirectionally typed expansion judgements}:
\[\begin{array}{ll}
\textbf{Judgement Form} & \textbf{Description}\\
\expandsTU{\uDelta}{\utau}{\tau} & \text{$\utau$ has well-formed expansion $\tau$}\\
\esyn{\uDelta}{\uGamma}{\uPsi}{\uPhi}{\ue}{e}{\tau} & \text{$\ue$ has expansion $e$ synthesizing type $\tau$}\\
\eana{\uDelta}{\uGamma}{\uPsi}{\uPhi}{\ue}{e}{\tau} & \text{$\ue$ has expansion $e$ when analyzed against type $\tau$}\\
% \rsyn{\uDelta}{\uGamma}{\uPsi}{\uPhi}{\urv}{r}{\tau}{\tau'} & \text{$\urv$ has expansion $r$ and takes values of type $\tau$ to values of}\\
% & \text{synthesized type $\tau'$}\\
\rana{\uDelta}{\uGamma}{\uPsi}{\uPhi}{\urv}{r}{\tau}{\tau'} & \text{$\urv$ has expansion $r$ and takes values of type $\tau$ to values of}\\
& \text{type $\tau'$ when $\tau's$ is provided for analysis}\\
\patExpands{\upctx}{\uPhi}{\upv}{p}{\tau} & \text{$\upv$ has expansion $p$ and type $\tau$ and generates hypotheses $\upctx$}
\end{array}\]

\subsubsection{Type Expansion}
\emph{Unexpanded type formation contexts}, $\uDelta$, were defined in Sec. \ref{sec:typed-expansion-U}. The \emph{type expansion judgement}, $\expandsTU{\uDelta}{\utau}{\tau}$, is inductively defined as in $\miniVersePat$ by Rules (\ref{rules:expandsTU}).

\subsubsection{Bidirectionally Typed Expression and Rule Expansion}
In order to clearly define the semantics of TLM implicits, we must make a judgmental distinction between \emph{type synthesis} and \emph{type analysis}. In the former, the type is determined from the term, while in the latter, the type is presumed known. Type systems that make this distinction are called \emph{bidirectional type systems} \cite{Pierce:2000:LTI:345099.345100}. (Pierce characterizes the idea as folklore predating his paper.)

The \emph{typed expression expansion judgements}, $\esynX{\ue}{e}{\tau}$, for type synthesis, and $\eanaX{\ue}{e}{\tau}$, for type analysis, and the typed rule expansion judgement, $\rana{\uDelta}{\uGamma}{\uPsi}{\uPhi}{\urv}{r}{\tau}{\tau'}$, are defined mutually inductively by Rules (\ref{rules:esyn-S}),  Rules (\ref{rules:eana-S}) and Rule (\ref{rule:rana-S}), respectively. We will reproduce only certain ``interesting'' rules below -- the appendix gives the complete set of rules.

\paragraph{Subsumption} Type analysis subsumes type synthesis according to the following \emph{rule of subsumption}:
\begin{equation*}\tag{\ref{rule:eana-S-subsume}}
  \inferrule{
    \esynX{\ue}{e}{\tau}
  }{
    \eanaX{\ue}{e}{\tau}
  }
\end{equation*}
In other words, when a type can be synthesized for an unexpanded expression, that unexpanded expression can also be analyzed against that type, producing the same expansion.


\paragraph{Type Ascription} 
A \emph{type ascription} can be placed on an unexpanded expression to specify the type that it should be analyzed against. 
\begin{equation*}\tag{\ref{rule:esyn-S-asc}}
  \inferrule{
    \expandsTU{\uDelta}{\utau}{\tau}\\
    \eanaX{\ue}{e}{\tau}
  }{
    \esynX{\asc{\ue}{\utau}}{e}{\tau}
  }
\end{equation*}

\paragraph{Variables} \emph{Unexpanded typing contexts}, $\uGamma$, were defined in Sec. \ref{sec:typed-expansion-U}. Identifiers that appear in $\uGamma$ have the expansion and synthesize the type that $\uGamma$ assigns to them.
\begin{equation*}\tag{\ref{rule:esyn-S-var}}
  \inferrule{ }{ 
    \esyn{\uDelta}{\uGamma, \uGhyp{\ux}{x}{\tau}}{\uPsi}{\uPhi}{\ux}{x}{\tau}
  }
\end{equation*}

\paragraph{Value Binding} We define let-binding of a value in synthetic or analytic position primitively in Bidirectional $\miniVersePat$. The following rules govern this construct.
\begin{equation*}\tag{\ref{rule:esyn-S-let}}
  \inferrule{
    \esynX{\ue}{e}{\tau}\\
    \esyn{\uDelta}{\uGamma, \uGhyp{\ux}{x}{\tau}}{\uPsi}{\uPhi}{\ue'}{e'}{\tau'}
  }{
    \esynX{\letsyn{\ux}{\ue}{\ue'}}{\aeap{\aelam{\tau}{x}{e'}}{e}}{\tau'}
  }
\end{equation*}
\begin{equation*}\tag{\ref{rule:eana-S-let}}
  \inferrule{
    \esynX{\ue}{e}{\tau}\\
    \eana{\uDelta}{\uGamma, \uGhyp{\ux}{x}{\tau}}{\uPsi}{\uPhi}{\ue'}{e'}{\tau'}
  }{
    \eanaX{\letsyn{\ux}{\ue}{\ue'}}{\aeap{\aelam{\tau}{x}{e'}}{e}}{\tau'}
  }
\end{equation*}

\paragraph{Functions} Functions with an argument type annotation can appear in synthetic position.
\begin{equation*}\tag{\ref{rule:esyn-S-lam}}
  \inferrule{
    \expandsTU{\uDelta}{\utau_1}{\tau_1}\\
    \esyn{\uDelta}{\uGamma, \uGhyp{\ux}{x}{\tau_1}}{\uPsi}{\uPhi}{\ue}{e}{\tau_2}
  }{
    \esynX{\lam{\ux}{\utau_1}{\ue}}{\aelam{\tau_1}{x}{e}}{\aparr{\tau_1}{\tau_2}}
  }
\end{equation*}
(In addition to such ``half annotated'' functions \cite{DBLP:conf/tldi/ChlipalaPH05}, it would be straightforward to include unannotated functions, $\lambda \ux.\ue$, which can appear only in analytic position. We leave these out for simplicity.)

Function applications can appear in synthetic position. The argument is analyzed against the argument type synthesized by the function.
\begin{equation*}\tag{\ref{rule:esyn-S-ap}}
  \inferrule{
    \esynX{\ue_1}{e_1}{\aparr{\tau_2}{\tau}}\\
    \eanaX{\ue_2}{e_2}{\tau_2}
  }{
    \esynX{\ap{\ue_1}{\ue_2}}{\aeap{e_1}{e_2}}{\tau}
  }
\end{equation*}

\paragraph{Pattern Matching}
The following rule governs match expressions, which  must appear in analytic position.
\begin{equation*}\tag{\ref{rule:eana-S-match}}
  \inferrule{
    \esynX{\ue}{e}{\tau}\\
    \{\ranaX{\urv_i}{r_i}{\tau}{\tau'}\}_{1 \leq i \leq n}
  }{
    \eanaX{\matchwith{\ue}{\seqschemaX{\urv}}}{\aematchwith{n}{e}{\seqschemaX{r}}}{\tau'}
  }
\end{equation*}
The typed rule expansion judgement is defined by the following rule:
\begin{equation*}\tag{\ref{rule:rana-S}}
  \inferrule{
    \patExpands{\uGG{\uG'}{\Gamma'}}{\uPhi}{\upv}{p}{\tau}\\
    \eana{\uDD{\uD}{\Delta}}{\uGG{\uG \uplus \uG'}{\Gamma \cup \Gamma'}}{\uPsi}{\uPhi}{\ue}{e}{\tau'}
  }{
    \rana{\uDD{\uD}{\Delta}}{\uGG{\uG}{\Gamma}}{\uPsi}{\uPhi}{\matchrule{\upv}{\ue}}{\aematchrule{p}{e}}{\tau}{\tau'}
  }
\end{equation*}
(In this simple calculus, it would also be possible to allow match expressions to appear in synthetic position -- all of the branches would need to synthesize the same type. In a language with richer notions of type equality and subtyping, this requires greater care. To avoid this orthogonal concern, we do not formally consider this case.)

The pattern expansion judgement, $\patExpands{\upctx}{\uPhi}{\upv}{p}{\tau}$, is inductively defined by Rules (\ref{rules:patExpands-B}), and operates as described in Chapter \ref{chap:uptsms}. There is one new rule, governing the newly introduced unadorned pattern literal form. We will return to this rule below.

\paragraph{Other Shared Forms} Other constructs of shared form have similar bidirectional rules, given in the appendix.

\paragraph{TLMs} seTLM contexts, $\uPsi$, take the form 
\[ 
\uASI{\uA}{\Psi}{\uI}
\]
and spTLM contexts, $\uPhi$, take the form
\[
\uASI{\uA}{\Phi}{\uI}
\]
where TLM identifier expansion contexts, $\uA$, seTLM definition contexts, $\Psi$, and spTLM definition contexts, $\Phi$, are defined as in $\miniVersePat$. \emph{TLM implicit designation contexts}, $\uI$, are new to Bidirectional $\miniVersePat$ and defined below.

Before considering TLM implicits, let us briefly review the rules for defining and explicitly applying TLMs. These rules are nearly identical to their counterparts in $\miniVersePat$, differing only in that they have been made bidirectional.

TLMs can be defined in synthetic or analytic position. The rules for seTLMs are reproduced below (the rules for spTLMs are analagous -- see appendix.)
\begin{equation*}\tag{\ref{rule:esyn-defuetsm}}
\inferrule{
  \expandsTU{\uDelta}{\utau}{\tau}\\
  \hastypeU{\emptyset}{\emptyset}{\eparse}{\aparr{\tBody}{\tParseResultExp}}\\\\
  \evalU{\eparse}{\eparse'}\\
  \esyn{\uDelta}{\uGamma}{\uPsi, \uShyp{\tsmv}{a}{\tau}{\eparse'}}{\uPhi}{\ue}{e}{\tau'}
}{
  \esynX{\usyntaxueP{\tsmv}{\utau}{\eparse}{\ue}}{e}{\tau'}
}
\end{equation*}
\begin{equation*}\tag{\ref{rule:eana-S-defuetsm}}
\inferrule{
  \expandsTU{\uDelta}{\utau}{\tau}\\
  \hastypeU{\emptyset}{\emptyset}{\eparse}{\aparr{\tBody}{\tParseResultExp}}\\\\
  \evalU{\eparse}{\eparse'}\\
  \eana{\uDelta}{\uGamma}{\uPsi, \uShyp{\tsmv}{a}{\tau}{\eparse'}}{\uPhi}{\ue}{e}{\tau'}
}{
  \eanaX{\usyntaxueP{\tsmv}{\utau}{\eparse}{\ue}}{e}{\tau'}
}
\end{equation*}

The rule for explicitly applying an seTLM is reproduced below:
\begin{equation*}\tag{\ref{rule:esyn-S-apuetsm}}
\inferrule{
  \uPsi = \uPsi', \uShyp{\tsmv}{a}{\tau}{\eparse}\\\\
  \encodeBody{b}{\ebody}\\
  \evalU{\ap{\eparse}{\ebody}}{\lbltxt{SuccessE}\cdot\ecand}\\
  \decodeCondE{\ecand}{\ce}\\\\
    \segOK{\segof{\ce}}{b}\\
  \cana{\emptyset}{\emptyset}{\esceneUP{\uDelta}{\uGamma}{\uPsi}{\uPhi}{b}}{\ce}{e}{\tau}
}{
  \esyn{\uDelta}{\uGamma}{\uPsi}{\uPhi}{\utsmap{\tsmv}{b}}{e}{\tau}
}
\end{equation*}

Similarly, the rule for explicitly applying an spTLM is reproduced below:

\begin{equation*}\tag{\ref{rule:patExpands-B-apuptsm}}
\inferrule{
  \uPhi = \uPhi', \uPhyp{\tsmv}{a}{\tau}{\eparse}\\\\
  \encodeBody{b}{\ebody}\\
  \evalU{\ap{\eparse}{\ebody}}{{\lbltxt{SuccessP}}\cdot{\ecand}}\\
  \decodeCEPat{\ecand}{\cpv}\\\\
    \segOK{\segof{\cpv}}{b}\\
  \cvalidP{\upctx}{\pscene{\Delta}{\uPhi}{b}}{\cpv}{p}{\tau}
}{
  \patExpands{\upctx}{\uPhi}{\utsmap{\tsmv}{b}}{p}{\tau}
}
\end{equation*}


\paragraph{TLM Implicits}

\emph{TLM implicit designation contexts}, $\uI$, are finite functions that map each type $\tau \in \domof{\uI}$ to the \emph{TLM designation} $\designate{\tau}{a}$, for some TLM name $a$. We write $\uI \uplus \designate{\tau}{a}$ for the TLM designation context that maps $\tau$ to $\designate{\tau}{a}$ and defers to $\uI$ for all other types (i.e. the previous designation, if any, is updated). 

The following rules governs seTLM designation in synthetic and analytic position, respectively:% We write $\uIOK{\Delta}{\uI}$ when each type in $\uI$ is well-formed assuming $\Delta$.
%\begin{definition}[TLM Designation Context Well-Formedness] $\uIOK{\Delta}{{\uI}$ iff for each $\designate{\tau}{a}$ we have $\istypeU{\Delta}{\tau}$.\end{definition}

\begin{equation*}\tag{\ref{rule:esyn-S-implicite}}
  \inferrule{
    \uPsi = \uASI{\uA \uplus \vExpands{\tsmv}{a}}{\Psi, \xuetsmbnd{a}{\tau}{\eparse}}{\uI}\\\\
    \esyn{\uDelta}{\uGamma}{\uASI{\uA \uplus \vExpands{\tsmv}{a}}{\Psi, \xuetsmbnd{a}{\tau}{\eparse}}{\uI \uplus \designate{\tau}{a}}}{\uPhi}{\ue}{e}{\tau'}
  }{
    \esyn{\uDelta}{\uGamma}{\uPsi}{\uPhi}{\implicite{\tsmv}{\ue}}{e}{\tau'}
  }
\end{equation*}

\begin{equation*}\tag{\ref{rule:eana-S-implicite}}
  \inferrule{
    \uPsi = \uASI{\uA \uplus \vExpands{\tsmv}{a}}{\Psi, \xuetsmbnd{a}{\tau}{\eparse}}{\uI}\\\\
    \eana{\uDelta}{\uGamma}{\uASI{\uA \uplus \vExpands{\tsmv}{a}}{\Psi, \xuetsmbnd{a}{\tau}{\eparse}}{\uI \uplus \designate{\tau}{a}}}{\uPhi}{\ue}{e}{\tau'}
  }{
    \eana{\uDelta}{\uGamma}{\uPsi}{\uPhi}{\implicite{\tsmv}{\ue}}{e}{\tau'}
  }
\end{equation*}

Similarly, the following rules govern spTLM designation in synthetic and analytic position, respectively:
\begin{equation*}\tag{\ref{rule:esyn-S-implicitp}}
  \inferrule{
    \uPhi = \uASI{\uA\uplus\vExpands{\tsmv}{a}}{\Phi, \xuptsmbnd{a}{\tau}{\eparse}}{\uI}\\\\
    \esyn{\uDelta}{\uGamma}{\uPsi}{\uASI{\uA\uplus\vExpands{\tsmv}{a}}{\Phi, \xuptsmbnd{a}{\tau}{\eparse}}{\uI \uplus \designate{\tau}{a}}}{\ue}{e}{\tau'}
  }{
    \esyn{\uDelta}{\uGamma}{\uPsi}{\uPhi}{\implicitp{\tsmv}{\ue}}{e}{\tau'}
  }
\end{equation*}
\begin{equation*}\tag{\ref{rule:eana-S-implicitp}}
  \inferrule{
    \uPhi = \uASI{\uA\uplus\vExpands{\tsmv}{a}}{\Phi, \xuptsmbnd{a}{\tau}{\eparse}}{\uI}\\\\
    \eana{\uDelta}{\uGamma}{\uPsi}{\uASI{\uA\uplus\vExpands{\tsmv}{a}}{\Phi, \xuptsmbnd{a}{\tau}{\eparse}}{\uI \uplus \designate{\tau}{a}}}{\ue}{e}{\tau'}
  }{
    \eana{\uDelta}{\uGamma}{\uPsi}{\uPhi}{\implicitp{\tsmv}{\ue}}{e}{\tau'}
  }
\end{equation*}

The following rule determines the TLM designated at the type that the expression of unadorned literal form is being analyzed against and applies it implicitly:
\begin{equation*}\tag{\ref{rule:eana-S-lit}}
  \inferrule{
    \uPsi = \uASI{\uA}{\Psi, \xuetsmbnd{a}{\tau}{\eparse}}{\uI \uplus \designate{\tau}{a}}\\\\
  \encodeBody{b}{\ebody}\\
  \evalU{\ap{\eparse}{\ebody}}{\lbltxt{SuccessE}\cdot\ecand}\\
  \decodeCondE{\ecand}{\ce}\\\\
    \segOK{\segof{\ce}}{b}\\
  \cana{\emptyset}{\emptyset}{\esceneUP{\uDelta}{\uGamma}{\uPsi}{\uPhi}{b}}{\ce}{e}{\tau}
  }{
    \eana{\uDelta}{\uGamma}{\uPsi}{\uPhi}{\lit{b}}{e}{\tau}
  }
\end{equation*}

Similarly, the following rule determines the TLM designated at the type that the pattern of unadorned literal form is matching against and applies it implicitly:
\begin{equation*}\tag{\ref{rule:patExpands-B-lit}}
\inferrule{
  \uPhi = \uASI{\uA}{\Phi, \xuptsmbnd{a}{\tau}{\eparse}}{\uI, \designate{\tau}{a}}\\\\
  \encodeBody{b}{\ebody}\\
  \evalU{\ap{\eparse}{\ebody}}{{\lbltxt{SuccessP}}\cdot{\ecand}}\\
  \decodeCEPat{\ecand}{\cpv}\\\\
    \segOK{\segof{\cpv}}{b}\\
  \cvalidP{\upctx}{\pscene{\uDelta}{\uPhi}{b}}{\cpv}{p}{\tau}
}{
  \patExpands{\upctx}{\uPhi}{\lit{b}}{p}{\tau}
}
\end{equation*}






% \subsection{Syntax of Proto-Expansions}\label{sec:ce-syntax-B}
% \begin{figure}
% \[\begin{array}{lllllll}
% \textbf{Sort} & & & \textbf{Operational Form} & \textbf{Stylized Form} & \textbf{Description}\\
% \mathsf{PrTyp} & \ctau & ::= & \cdots & \cdots & \text{(as in $\miniVersePat$)}\\
% % &&& \aceparr{\ctau}{\ctau} & \parr{\ctau}{\ctau} & \text{partial function}\\
% % &&& \aceall{t}{\ctau} & \forallt{t}{\ctau} & \text{polymorphic}\\
% % &&& \acerec{t}{\ctau} & \rect{t}{\ctau} & \text{recursive}\\
% % &&& \aceprod{\labelset}{\mapschema{\ctau}{i}{\labelset}} & \prodt{\mapschema{\ctau}{i}{\labelset}} & \text{labeled product}\\
% % &&& \acesum{\labelset}{\mapschema{\ctau}{i}{\labelset}} & \sumt{\mapschema{\ctau}{i}{\labelset}} & \text{labeled sum}\\
% %\LCC &&& \gray & \gray & \gray\\
% % &&& \acesplicedt{m}{n} & \splicedt{m}{n} & \text{spliced}\\%\ECC
% \mathsf{PrExp} & \ce & ::= & \cdots & \cdots & \text{(as in $\miniVersePat$)}\\
% &&& \aceasc{\ctau}{\ce} & \asc{\ce}{\ctau} & \text{ascription}\\
% &&& \aceletsyn{x}{\ce}{\ce} & \letsyn{x}{\ce}{\ce} & \text{value binding}\\
% % &&& \aceanalam{x}{\ce} & \analam{x}{\ce} & \text{abstraction (unannotated)}\\
% % &&& \acelam{\ctau}{x}{\ce} & \lam{x}{\ctau}{\ce} & \text{abstraction (annotated)}\\
% % &&& \aceap{\ce}{\ce} & \ap{\ce}{\ce} & \text{application}\\
% % &&& \acetlam{t}{\ce} & \Lam{t}{\ce} & \text{type abstraction}\\
% % &&& \acetap{\ce}{\ctau} & \App{\ce}{\ctau} & \text{type application}\\
% % &&& \aceanafold{\ce} & \fold{\ce} & \text{fold}\\
% % &&& \aceunfold{\ce} & \unfold{\ce} & \text{unfold}\\
% % &&& \acetpl{\labelset}{\mapschema{\ce}{i}{\labelset}} & \tpl{\mapschema{\ce}{i}{\labelset}} & \text{labeled tuple}\\
% % &&& \acepr{\ell}{\ce} & \prj{\ce}{\ell} & \text{projection}\\
% % &&& \aceanain{\ell}{\ce} & \inj{\ell}{\ce} & \text{injection}\\
% % &&& \acematchwithb{n}{\ce}{\seqschemaX{\urv}} & \matchwith{\ce}{\seqschemaX{\crv}} & \text{match}\\%\LCC &&& \gray & \gray & \gray\\
% % &&& \acesplicede{m}{n} & \splicede{m}{n} & \text{spliced}\\%\ECC
% % &&& \acesplicedet{m}{n}{\ctau} & \splicedet{m}{n}{\ctau} & \text{spliced (analytic)}\\
% \mathsf{PrRule} & \crv & ::= & \cdots & \cdots & \text{(as in $\miniVersePat$)}\\
% \mathsf{PrPat} & \cpv & ::= & \cdots & \cdots & \text{(as in $\miniVersePat$)}\\
% % &&& \acefoldp{p} & \foldp{p} & \text{fold pattern}\\
% % &&& \acetplp{\labelset}{\mapschema{\cpv}{i}{\labelset}} & \tplp{\mapschema{\cpv}{i}{\labelset}} & \text{labeled tuple pattern}\\
% % &&& \aceinjp{\ell}{\cpv} & \injp{\ell}{\cpv} & \text{injection pattern}\\
% % &&& \acesplicedp{m}{n} & \splicedp{m}{n} & \text{spliced}
% \end{array}\]
% \caption[Syntax of proto-terms in Bidirectional $\miniVersePat$]{Syntax of proto-types, proto-expressions, proto-rules and proto-patterns in Bidirecitonal $\miniVersePat$.}
% \label{fig:B-candidate-terms}
% \end{figure}

\subsection{Bidirectional Proto-Expansion Validation}\label{sec:ce-validation-B}
The syntax of proto-expansions was defined in Sec. \ref{sec:ce-syntax-UP}.

The \emph{bidirectional proto-expansion validation judgements} validate proto-terms and simultaneously generate their final expansions.

\vspace{10px}\noindent$\arraycolsep=2pt\begin{array}{ll}
\textbf{Judgement Form} & \textbf{Description}\\
\cvalidT{\Delta}{\tscenev}{\ctau}{\tau} & \text{$\ctau$ has well-formed expansion $\tau$}\\
\csynX{\ce}{e}{\tau} & \text{$\ce$ has expansion $e$ synthesizing type $\tau$}\\
\canaX{\ce}{e}{\tau} & \text{$\ce$ has expansion $e$ when analyzed against type $\tau$}\\
\crana{\Delta}{\Gamma}{\escenev}{\crv}{r}{\tau}{\tau'} & \text{$\crv$ has expansion $r$ taking values of type $\tau$ to values of type $\tau'$}\\
\cvalidP{\upctx}{\pscenev}{\cpv}{p}{\tau} & \text{$\cpv$ has expansion $p$ matching against $\tau$ generating assumptions $\upctx$}
\end{array}$\vspace{10px}

These judgements are defined by rules given in Appendix \ref{appendix:proto-expansion-validation-BS}. Most rules follow the corresponding typed expansion rule. The main rule of interest here is the rule governing references to spliced expressions, reproduced below:
\begin{equation*}\tag{\ref{rule:csyn-splicede}}
\inferrule{
  \cvalidT{\emptyset}{\tsfrom{\escenev}}{\ctau}{\tau}\\
  \escenev=\esceneUP{\uDD{\uD}{\Delta_\text{app}}}{\uGG{\uG}{\Gamma_\text{app}}}{\uPsi}{\uPhi}{b}\\
  \parseUExp{\bsubseq{b}{m}{n}}{\ue}\\
  \eana{\uDD{\uD}{\Delta_\text{app}}}{\uGG{\uG}{\Gamma_\text{app}}}{\uPsi}{\uPhi}{\ue}{e}{\tau}\\\\
  \Delta \cap \Delta_\text{app} = \emptyset\\
  \domof{\Gamma} \cap \domof{\Gamma_\text{app}} = \emptyset
}{
  \csyn{\Delta}{\Gamma}{\escenev}{\acesplicede{m}{n}{\ctau}}{e}{\tau}
}
\end{equation*}
This rule is similar to Rule (\ref{rule:cvalidE-U-splicede}), which governed references to spliced expressions in $\miniVersePat$. Notice that here, the unexpanded expression $\ue$ is analyzed against the type $\tau$.

\subsection{Metatheory}
Bidirectional $\miniVersePat$ enjoys metatheoretic properties analagous to those established for $\miniVersePat$. We state these properties below -- the proofs are given in Appendix \ref{appendix:B-metatheory}.

The following theorem establishes that typed pattern expansion produces an expanded pattern that matches values of the specified type and generates the same hypotheses. It must be stated mutually with the corresponding theorem about proto-patterns, because the judgements are mutually defined.
\begingroup
\def\thetheorem{\ref{thm:typed-pattern-expansion-B}}
\begin{theorem}[Typed Pattern Expansion] ~
\begin{enumerate}
  \item If $\pExpandsSP{\uDD{\uD}{\Delta}}{\uASI{\uA}{\Phi}{\uI}}{\upv}{p}{\tau}{\uGG{\uG}{\pctx}}$ then $\patType{\pctx}{p}{\tau}$.
  \item If $\cvalidP{\uGG{\uG}{\pctx}}{\pscene{\uDD{\uD}{\Delta}}{\uAP{\uA}{\Phi}}{b}}{\cpv}{p}{\tau}$ then $\patType{\pctx}{p}{\tau}$.
\end{enumerate}
\end{theorem}
\endgroup

The following theorem establishes that bidirectionally typed expression and rule expansion produces expanded expressions and rules of the appropriate type under the appropriate contexts. These statements must be stated mutually with the corresponding statements about birectional proto-expression and proto-rule validation because the judgements are mutually defined. 

\begingroup
\def\thetheorem{\ref{thm:typed-expansion-full-B}}
\begin{theorem}[Typed Expression and Rule Expansion] ~
\begin{enumerate}
  \item \begin{enumerate}
    \item If $\esyn{\uDD{\uD}{\Delta}}{\uGG{\uG}{\Gamma}}{\uPsi}{\uPhi}{\ue}{e}{\tau}$ then $\hastypeU{\Delta}{\Gamma}{e}{\tau}$.
    \item If $\eana{\uDD{\uD}{\Delta}}{\uGG{\uG}{\Gamma}}{\uPsi}{\uPhi}{\ue}{e}{\tau}$ and $\istypeU{\Delta}{\tau}$ then $\hastypeU{\Delta}{\Gamma}{e}{\tau}$.
    \item If $\rana{\uDD{\uD}{\Delta}}{\uGG{\uG}{\Gamma}}{\uPsi}{\uPhi}{\urv}{r}{\tau}{\tau'}$ and $\istypeU{\Delta}{\tau'}$ then $\ruleType{\Delta}{\Gamma}{r}{\tau}{\tau'}$.
  \end{enumerate}
  \item \begin{enumerate}
    \item If $\csyn{\Delta}{\Gamma}{\esceneUP{\uDD{\uD}{\Delta_\text{app}}}{\uGG{\uG}{\Gamma_\text{app}}}{\uPsi}{\uPhi}{b}}{\ce}{e}{\tau}$ and $\Delta \cap \Delta_\text{app}=\emptyset$ and $\domof{\Gamma} \cap \domof{\Gamma_\text{app}}=\emptyset$ then $\hastypeU{\Dcons{\Delta}{\Delta_\text{app}}}{\Gcons{\Gamma}{\Gamma_\text{app}}}{e}{\tau}$. 
    \item If $\cana{\Delta}{\Gamma}{\esceneUP{\uDD{\uD}{\Delta_\text{app}}}{\uGG{\uG}{\Gamma_\text{app}}}{\uPsi}{\uPhi}{b}}{\ce}{e}{\tau}$ and $\istypeU{\Delta}{\tau}$ and $\Delta \cap \Delta_\text{app}=\emptyset$ and $\domof{\Gamma} \cap \domof{\Gamma_\text{app}}=\emptyset$ then $\hastypeU{\Dcons{\Delta}{\Delta_\text{app}}}{\Gcons{\Gamma}{\Gamma_\text{app}}}{e}{\tau}$. 
    \item If $\crana{\Delta}{\Gamma}{\esceneUP{\uDD{\uD}{\Delta_\text{app}}}{\uGG{\uG}{\Gamma_\text{app}}}{\uPsi}{\uPhi}{b}}{\crv}{r}{\tau}{\tau'}$ and $\istypeU{\Delta}{\tau'}$ and $\Delta \cap \Delta_\text{app}=\emptyset$ and $\domof{\Gamma} \cap \domof{\Gamma_\text{app}}=\emptyset$ then $\ruleType{\Dcons{\Delta}{\Delta_\text{app}}}{\Gcons{\Gamma}{\Gamma_\text{app}}}{r}{\tau}{\tau'}$.
  \end{enumerate}
\end{enumerate}
\end{theorem}

The following theorem establishes abstract reasoning principles for implicitly applied expression TLMs. These are analagous to those described in Section \ref{sec:uetsms-reasoning-principles} for explicitly applied expression TLMs.
\begingroup
\def\thetheorem{\ref{thm:tsc-B}}
\begin{theorem}[seTLM Abstract Reasoning Principles - Implicit Application]
If \[\eana{\uDD{\uD}{\Delta}}{\uGG{\uG}{\Gamma}}{\uPsi}{\uPhi}{\lit{b}}{e}{\tau}\] then:
\begin{enumerate}
\item (\textbf{Typing 1}) $\uPsi = \uASI{\uA}{\Psi, \xuetsmbnd{a}{\tau}{\eparse}}{\uI \uplus \designate{\tau}{a}}$ and $\hastypeU{\Delta}{\Gamma}{e}{\tau}$
\item $\encodeBody{b}{\ebody}$
\item $\evalU{\ap{\eparse}{\ebody}}{\aein{\mathtt{SuccessE}}{\ecand}}$
\item $\decodeCondE{\ecand}{\ce}$
\item (\textbf{Segmentation}) $\segOK{\segof{\ce}}{b}$
\item $\segof{\ce} = \sseq{\acesplicedt{m'_i}{n'_i}}{\nty} \cup \sseq{\acesplicede{m_i}{n_i}{\ctau_i}}{\nexp}$
\item \textbf{(Typing 2)} $\sseq{
      \expandsTU{\uDD{\uD}{\Delta}}
      {
        \parseUTypF{\bsubseq{b}{m'_i}{n'_i}}
      }{\tau'_i}
    }{\nty}$ and $\sseq{\istypeU{\Delta}{\tau'_i}}{\nty}$
\item \textbf{(Typing 3)} $\sseq{
  \cvalidT{\emptyset}{
    \tsceneUP
      {\uDD
        {\uD}{\Delta}
      }{b}
  }{
    \ctau_i
  }{\tau_i}
}{\nexp}$ and $\sseq{\istypeU{\Delta}{\tau_i}}{\nexp}$
\item \textbf{(Typing 4)} $\sseq{
  \eana
    {\uDD{\uD}{\Delta}}
    {\uGG{\uG}{\Gamma}}
    {\uPsi}
    {\uPhi}
    {\parseUExpF{\bsubseq{b}{m_i}{n_i}}}
    {e_i}
    {\tau_i}
}{\nexp}$ and $\sseq{\hastypeU{\Delta}{\Gamma}{e_i}{\tau_i}}{\nexp}$
\item (\textbf{Capture Avoidance}) $e = [\sseq{\tau'_i/t_i}{\nty}, \sseq{e_i/x_i}{\nexp}]e'$ for some $\sseq{t_i}{\nty}$ and $\sseq{x_i}{\nexp}$ and $e'$
\item (\textbf{Context Independence}) $\mathsf{fv}(e') \subset \sseq{t_i}{\nty} \cup \sseq{x_i}{\nexp}$
  % $\hastypeU
  % {\sseq{\Dhyp{t_i}}{\nty}}
  % {\sseq{x_i : \tau_i}{\nexp}}
  % {e'}{\tau}$
\end{enumerate}
\end{theorem}
\endgroup

Similarly, the following theorem establishes abstract reasoning principles for implicitly applied pattern TLMs. These are analagous to those described in Sec. \ref{sec:uptsms-abstract-reasoning-principles} for explicitly applied pattern TLMs.

\begingroup
\def\thetheorem{\ref{thm:spTLM-Typing-Segmentation-Implicit-B}}
\begin{theorem}[spTLM Abstract Reasoning Principles - Implicit Application]
If \[\patExpands{\upctx}{\uPhi}{\lit{b}}{p}{\tau}\] where $\uDelta=\uDD{\uD}{\Delta}$ and $\uGamma=\uGG{\uG}{\Gamma}$ then all of the following hold:
\begin{enumerate}
        \item (\textbf{Typing 1}) $\uPhi = \uASI{\uA}{\Phi, \xuptsmbnd{a}{\tau}{\eparse}}{\uI, \designate{\tau}{a}}$ and $\patType{\pctx}{p}{\tau}$
        \item $\encodeBody{b}{\ebody}$
        \item $\evalU{\eparse(\ebody)}{\aein{\mathtt{SuccessP}}{\ecand}}$
        \item $\decodeCEPat{\ecand}{\cpv}$
        \item (\textbf{Segmentation}) $\segOK{\segof{\cpv}}{b}$
        \item $\segof{\cpv} = \sseq{\acesplicedt{n'_i}{m'_i}}{\nty} \cup \sseq{\acesplicedp{m_i}{n_i}{\ctau_i}}{\npat}$
        \item (\textbf{Typing 2}) $\sseq{
              \expandsTU{\uDelta}
              {
                \parseUTypF{\bsubseq{b}{m'_i}{n'_i}}
              }{\tau'_i}
            }{\nty}$ and $\sseq{\istypeU{\Delta}{\tau'_i}}{\nty}$
        \item (\textbf{Typing 3}) $\sseq{
          \cvalidT{\emptyset}{
            \tsceneUP
              {\uDelta}{b}
          }{
            \ctau_i
          }{\tau_i}
        }{\npat}$ and $\sseq{\istypeU{\Delta}{\tau_i}}{\npat}$
        \item (\textbf{Typing 4}) $\sseq{
          \patExpands
            {\upctx_i}
            {\uPhi}
            {\parseUPatF{\bsubseq{b}{m_i}{n_i}}}
            {p_i}
            {\tau_i}
        }{\npat}$ 
      \item (\textbf{No Hidden Bindings}) $\upctx = \biguplus_{0 \leq i < \npat} \upctx_i$
\end{enumerate}
\end{theorem}
\endgroup

% \begin{proof} By mutual rule induction over Rules (\ref{rules:esyn}), Rules (\ref{rules:eana}), Rule (\ref{rule:rsyn}), Rule (\ref{rule:rana}), Rules (\ref{rules:csyn}), Rules (\ref{rules:cana}), Rule (\ref{rule:crsyn}) and Rule (\ref{rule:crana}). In the following, we refer to the induction hypothesis applied to the assumption $\uetsmenv{\Delta}{\Psi}$ as simply the ``IH''. When we apply the induction hypothesis to a different argument, we refer to it as the ``Outer IH''.

% \begin{enumerate}
%   \item In the following, let $\uDelta=\uDD{\uD}{\Delta}$ and $\uGamma=\uGG{\uG}{\Gamma}$. We have:
%   \begin{enumerate}
%     \item \begin{enumerate}
%       \item We induct on the assumption.
%         \begin{byCases}
%           \item[\text{(\ref{rule:esyn-var})}] We have:
%             \begin{pfsteps*}
%               \item $e=x$ \BY{assumption}
%               \item $\Gamma=\Gamma', \Ghyp{x}{\tau}$ \BY{assumption}
%               \item $\hastypeU{\Delta}{\Gamma', \Ghyp{x}{\tau}}{x}{\tau}$ \BY{Rule (\ref{rule:hastypeUP-var})}
%             \end{pfsteps*}
%             \resetpfcounter
%           \item[\text{(\ref{rule:esyn-asc})}] We have:
%             \begin{pfsteps*}
%                \item $\ue=\auasc{\utau}{\ue'}$ \BY{assumption}
%                \item $\expandsTU{\uDelta}{\utau}{\tau}$ \BY{assumption}\pflabel{expandsTU}
%                \item $\eanaX{\ue'}{e}{\tau}$ \BY{assumption}\pflabel{eanaX}
%                \item $\istypeU{\Delta}{\tau}$ \BY{Lemma \ref{lemma:type-expansion-U} on \pfref{expandsTU}}\pflabel{istype}
%                \item $\hastypeU{\Delta}{\Gamma}{e}{\tau}$ \BY{IH, part 1(b)(i) to \pfref{eanaX} and \pfref{istype}}
%              \end{pfsteps*}
%              \resetpfcounter
%           \item[\text{(\ref{rule:esyn-let}) through (\ref{rule:esyn-match})}] In each of these cases, we apply:
%             \begin{itemize}
%               \item Lemma \ref{lemma:type-expansion-U} to or over all type expansion premises.
%               \item The IH, part 1(a)(i) to or over all synthetic typed expression expansion premises.
%               \item The IH, part 1(a)(ii) to or over all synthetic rule expansion premises.
%               \item The IH, part 1(b)(i) to or over all analytic typed expression expansion premises.
%             \end{itemize}
%             We then derive the conclusion by applying Rules (\ref{rules:hastypeUP}) and Rule (\ref{rule:ruleType}) as needed.
%           \item[\text{(\ref{rule:esyn-defuetsm})}] We have:
%             \begin{pfsteps*}
%               \item $\ue=\audefuetsm{\utau'}{\eparse}{\tsmv}{\ue'}$ \BY{assumption}
%               \item $\expandsTU{\uDelta}{\utau'}{\tau'}$ \BY{assumption} \pflabel{expandsTU}
%               \item $\hastypeU{\emptyset}{\emptyset}{\eparse}{\aparr{\tBody}{\tParseResultExp}}$ \BY{assumption}\pflabel{eparse}
%               \item $\esyn{\uDelta}{\uGamma}{\uASI{\ctxUpdate{\uA}{\tsmv}{a}}{\Psi, \xuetsmbnd{a}{\tau'}{\eparse}}{\uI}}{\uPhi}{\ue'}{e}{\tau}$ \BY{assumption}\pflabel{expandsU}
%               \item $\uetsmenv{\Delta}{\Psi}$ \BY{assumption}\pflabel{uetsmenv1}
%               \item $\istypeU{\Delta}{\tau'}$ \BY{Lemma \ref{lemma:type-expansion-U} to \pfref{expandsTU}} \pflabel{istype}
%               \item $\uetsmenv{\Delta}{\Psi, \xuetsmbnd{\tsmv}{\tau'}{\eparse}}$ \BY{Definition \ref{def:ueTLM-def-ctx-formation-UP} on \pfref{uetsmenv1}, \pfref{istype} and \pfref{eparse}}\pflabel{uetsmenv3}
%               \item $\hastypeU{\Delta}{\Gamma}{e}{\tau}$ \BY{Outer IH, part 1(a)(i) on \pfref{uetsmenv3} and \pfref{expandsU}}
%             \end{pfsteps*}
%             \resetpfcounter
%           \item[\text{(\ref{rule:esyn-apuetsm})}] We have:
%             \begin{pfsteps*}
%               \item $\ue=\autsmap{b}{\tsmv}$ \BY{assumption}
%               \item $\uPsi = \uASI{\ctxUpdate{\uA'}{\tsmv}{a}}{\Psi', \xuetsmbnd{a}{\tau}{\eparse}}{\uI}$ \BY{assumption}
%               \item $\encodeBody{b}{\ebody}$ \BY{assumption}
%               \item $\evalU{\eparse(\ebody)}{\inj{\lbltxt{Success}}{\ecand}}$ \BY{assumption}
%               \item $\decodeCondE{\ecand}{\ce}$ \BY{assumption}
%               \item $\cana{\emptyset}{\emptyset}{\esceneUP{\uDelta}{\uGamma}{\uPsi}{\uPhi}{b}}{\ce}{e}{\tau}$ \BY{assumption}\pflabel{cvalidE}
%               \item $\uetsmenv{\Delta}{\Psi}$ \BY{assumption} \pflabel{uetsmenv}
%               \item $\istypeU{\Delta}{\tau}$ \BY{Definition \ref{def:ueTLM-def-ctx-formation-UP} on \pfref{uetsmenv}} \pflabel{istype}
%               \item $\emptyset \cap \Delta = \emptyset$ \BY{finite set intersection identity} \pflabel{delta-cap}
%               \item ${\emptyset} \cap \domof{\Gamma} = \emptyset$ \BY{finite set intersection identity} \pflabel{gamma-cap}
%               \item $\hastypeU{\emptyset \cup \Delta}{\emptyset \cup \Gamma}{e}{\tau}$ \BY{IH, part 2(a)(i) on \pfref{cvalidE}, \pfref{delta-cap}, \pfref{gamma-cap} and \pfref{istype}} \pflabel{penultimate}
%               \item $\hastypeU{\Delta}{\Gamma}{e}{\tau}$ \BY{definition of finite set and finite function union over \pfref{penultimate}}               
%              \end{pfsteps*} 
%              \resetpfcounter
%           \item[\text{(\ref{rule:esyn-implicite})}] We have:
%             \begin{pfsteps*}
%               \item $\ue=\auimplicite{\tsmv}{\ue}$ \BY{assumption}
%               \item $\uPsi=\uASI{\uA' \uplus \vExpands{\tsmv}{a}}{\Psi', \xuetsmbnd{a}{\tau'}{\eparse}}{\uI}$ \BY{assumption}
%               \item $\esyn{\uDelta}{\uGamma}{\uASI{\uA' \uplus \vExpands{\tsmv}{a}}{\Psi', \xuetsmbnd{a}{\tau'}{\eparse}}{\uI \uplus \designate{\tau}{a}}}{\uPhi}{\ue}{e}{\tau}$ \BY{assumption} \pflabel{esyn}
%               \item $\hastypeU{\Delta}{\Gamma}{e}{\tau}$ \BY{IH, part 1(a)(i) on \pfref{esyn}}
%             \end{pfsteps*}
%             \resetpfcounter
%           \item[\text{(\ref{rule:esyn-defuptsm})}] We have:
%             \begin{pfsteps*}
%               \item $\ue=\audefuptsm{\utau'}{\eparse}{\tsmv}{\ue'}$ \BY{assumption}
%               \item $\expandsTU{\uDelta}{\utau'}{\tau'}$ \BY{assumption} \pflabel{expandsTU}
%             %  \item \hastypeU{\emptyset}{\emptyset}{\eparse}{\aparr{\tBody}{\tParseResultExp}} \BY{assumption}\pflabel{eparse}
%               \item $\esyn{\uDelta}{\uGamma}{\uPsi}{\uPhi, \uPhyp{\tsmv}{a}{\tau'}{\eparse}}{\ue'}{e}{\tau}$ \BY{assumption}\pflabel{expandsU}
%             %  \item \uetsmenv{\Delta}{\Psi} \BY{assumption}\pflabel{uetsmenv1}
%             %  \item \istypeU{\Delta}{\tau'} \BY{Lemma \ref{lemma:type-expansion-U} to \pfref{expandsTU}} \pflabel{istype}
%             %  \item \uetsmenv{\Delta}{\Psi, \xuetsmbnd{\tsmv}{\tau'}{\eparse}} \BY{Definition \ref{def:ueTLM-def-ctx-formation} on \pfref{uetsmenv1}, \pfref{istype} and \pfref{eparse}}\pflabel{uetsmenv3}
%               \item $\hastypeU{\Delta}{\Gamma}{e}{\tau}$ \BY{IH, part 1(a)(i) on \pfref{expandsU}}
%             \end{pfsteps*}
%             \resetpfcounter
%           \item[\text{(\ref{rule:esyn-implicitp})}] We have:
%             \begin{pfsteps*}
%               \item $\ue=\auimplicitp{\tsmv}{\ue}$ \BY{assumption}
%               \item $\uPhi=\uASI{\uA \uplus \vExpands{\tsmv}{a}}{\Phi, \xuptsmbnd{a}{\tau'}{\eparse}}{\uI}$ \BY{assumption}
%               \item $\esyn{\uDelta}{\uGamma}{\uPsi}{\uASI{\uA \uplus \vExpands{\tsmv}{a}}{\Phi, \xuptsmbnd{a}{\tau'}{\eparse}}{\uI \uplus \designate{\tau}{a}}}{\ue}{e}{\tau}$ \BY{assumption} \pflabel{esyn}
%               \item $\hastypeU{\Delta}{\Gamma}{e}{\tau}$ \BY{IH, part 1(a)(i) on \pfref{esyn}}
%             \end{pfsteps*}
%             \resetpfcounter
%         \end{byCases}
%       \item We induct on the assumption. There is one case.
%         \begin{byCases}
%           \item[\text{(\ref{rule:rsyn})}] We have:
%             \begin{pfsteps*}
%               \item $\urv=\aumatchrule{\upv}{\ue}$ \BY{assumption}
%               \item $r=\aematchrule{p}{e}$ \BY{assumption}
%               \item $\patExpands{\uGG{\uA'}{\pctx}}{\uPhi}{\upv}{p}{\tau}$ \BY{assumption} \pflabel{patExpands}
%               \item $\esyn{\uDelta}{\uGG{{\uA}\uplus{\uA'}}{\Gcons{\Gamma}{\pctx}}}{\uPsi}{\uPhi}{\ue}{e}{\tau'}$ \BY{assumption} \pflabel{expandsUP}
%               \item $\patType{\pctx}{p}{\tau}$ \BY{Theorem \ref{thm:typed-pattern-expansion-B}, part 1 on \pfref{patExpands}}\pflabel{patType}
%               \item $\hastypeU{\Delta}{\Gcons{\Gamma}{\pctx}}{e}{\tau'}$ \BY{IH, part 1(a)(i) on \pfref{expandsUP}} \pflabel{hasType}
%               \item $\ruleType{\Delta}{\Gamma}{\aematchrule{p}{e}}{\tau}{\tau'}$ \BY{Rule (\ref{rule:ruleType}) on \pfref{patType} and \pfref{hasType}}
%             \end{pfsteps*}
%             \resetpfcounter
%         \end{byCases}
%     \end{enumerate}
%     \item \begin{enumerate}
%       \item We induct on the assumption.
%         \begin{byCases}
%           \item[\text{(\ref{rule:eana-subsume})}] We have:
%             \begin{pfsteps*}
%               \item $\esynX{\ue}{e}{\tau}$ \BY{assumption} \pflabel{esyn}
%               \item $\hastypeU{\Delta}{\Gamma}{e}{\tau}$ \BY{IH, part 1(a)(i) on \pfref{esyn}}
%             \end{pfsteps*}
%           \item[\text{(\ref{rule:eana-let}) through (\ref{rule:eana-match})}] In each of these cases, we apply:
%             \begin{itemize}
%               \item Lemma \ref{lemma:type-expansion-U} to or over all type expansion premises.
%               \item The IH, part 1(a)(i) to or over all synthetic typed expression expansion premises.
%               \item The IH, part 1(a)(ii) to or over all synthetic rule expansion premises.
%               \item The IH, part 1(b)(i) to or over all analytic typed expression expansion premises.
%             \end{itemize}
%             We then derive the conclusion by applying Rules (\ref{rules:hastypeUP}) and Rule (\ref{rule:ruleType}) as needed. 
%           \item[\text{(\ref{rule:eana-defuetsm})}] We have:
%             \begin{pfsteps*}
%               \item $\ue=\audefuetsm{\utau'}{\eparse}{\tsmv}{\ue'}$ \BY{assumption}
%               \item $\expandsTU{\uDelta}{\utau'}{\tau'}$ \BY{assumption} \pflabel{expandsTU}
%               \item $,$ \BY{assumption}\pflabel{eparse}
%               \item $\eana{\uDelta}{\uGamma}{\uPsi, \uShyp{\tsmv}{a}{\tau'}{\eparse}}{\uPhi}{\ue'}{e}{\tau}$ \BY{assumption}\pflabel{expandsU}
%               \item $\uetsmenv{\Delta}{\Psi}$ \BY{assumption}\pflabel{uetsmenv1}
%               \item $\istypeU{\Delta}{\tau'}$ \BY{Lemma \ref{lemma:type-expansion-U} to \pfref{expandsTU}} \pflabel{istype}
%               \item $\uetsmenv{\Delta}{\Psi, \xuetsmbnd{\tsmv}{\tau'}{\eparse}}$ \BY{Definition \ref{def:ueTLM-def-ctx-formation-UP} on \pfref{uetsmenv1}, \pfref{istype} and \pfref{eparse}}\pflabel{uetsmenv3}
%             %  \item \uetsmenv{\Delta}{\Psi} \BY{assumption}\pflabel{uetsmenv1}
%             %  \item \istypeU{\Delta}{\tau'} \BY{Lemma \ref{lemma:type-expansion-U} to \pfref{expandsTU}} \pflabel{istype}
%             %  \item \uetsmenv{\Delta}{\Psi, \xuetsmbnd{\tsmv}{\tau'}{\eparse}} \BY{Definition \ref{def:ueTLM-def-ctx-formation} on \pfref{uetsmenv1}, \pfref{istype} and \pfref{eparse}}\pflabel{uetsmenv3}
%               \item $\hastypeU{\Delta}{\Gamma}{e}{\tau}$ \BY{IH, part 1(b)(i) on \pfref{expandsU}}
%             \end{pfsteps*}
%             \resetpfcounter
%           \item[\text{(\ref{rule:eana-implicite})}] We have:
%             \begin{pfsteps*}
%               \item $\ue=\autsmap{b}{\tsmv}$ \BY{assumption}
%               \item $\uPsi = \uPsi', \uShyp{\tsmv}{a}{\tau}{\eparse}$ \BY{assumption}
%               \item $\encodeBody{b}{\ebody}$ \BY{assumption}
%               \item $\evalU{\eparse(\ebody)}{\inj{\lbltxt{Success}}{\ecand}}$ \BY{assumption}
%               \item $\decodeCondE{\ecand}{\ce}$ \BY{assumption}
%               \item $\cana{\emptyset}{\emptyset}{\esceneUP{\uDelta}{\uGamma}{\uPsi}{\uPhi}{b}}{\ce}{e}{\tau}$ \BY{assumption}\pflabel{cvalidE}
%             %  \item \uetsmenv{\Delta}{\Psi} \BY{assumption} \pflabel{uetsmenv}
%               \item $\emptyset \cap \Delta = \emptyset$ \BY{finite set intersection identity} \pflabel{delta-cap}
%               \item ${\emptyset} \cap \domof{\Gamma} = \emptyset$ \BY{finite set intersection identity} \pflabel{gamma-cap}
%               \item $\hastypeU{\emptyset \cup \Delta}{\emptyset \cup \Gamma}{e}{\tau}$ \BY{IH, part 2(b)(i) on \pfref{cvalidE}, \pfref{delta-cap}, and \pfref{gamma-cap}} \pflabel{penultimate}
%               \item $\hastypeU{\Delta}{\Gamma}{e}{\tau}$ \BY{definition of finite set union over \pfref{penultimate}}               
%              \end{pfsteps*} 
%              \resetpfcounter
%           \item[\text{(\ref{rule:eana-lit})}] We have:
%             \begin{pfsteps*}
%               \item $\ue=\auelit{b}$ \BY{assumption}
%               \item $\uPsi=\uASI{\uA}{\Psi, \xuetsmbnd{a}{\tau}{\eparse}}{\uI \uplus \designate{\tau}{a}}$ \BY{assumption}
%               \item $\encodeBody{b}{\ebody}$ \BY{assumption}
%               \item $\evalU{\ap{\eparse}{\ebody}}{\inj{\lbltxt{Success}}{\ecand}}$ \BY{assumption}
%               \item $\decodeCondE{\ecand}{\ce}$ \BY{assumption}
%               \item $\cana{\emptyset}{\emptyset}{\esceneUP{\uDelta}{\uGamma}{\uASI{\uA}{\Psi, \xuetsmbnd{a}{\tau}{\eparse}}{\uI \uplus \designate{\tau}{a}}}{\uPhi}{b}}{\ce}{e}{\tau}$ \BY{assumption} \pflabel{cvalidE}
%               \item $\emptyset \cap \Delta = \emptyset$ \BY{finite set intersection identity} \pflabel{delta-cap}
%               \item ${\emptyset} \cap \domof{\Gamma} = \emptyset$ \BY{finite set intersection identity} \pflabel{gamma-cap}
%               \item $\hastypeU{\emptyset \cup \Delta}{\emptyset \cup \Gamma}{e}{\tau}$ \BY{IH, part 2(a)(i) on \pfref{cvalidE}, \pfref{delta-cap}, and \pfref{gamma-cap}} \pflabel{penultimate}
%               \item $\hastypeU{\Delta}{\Gamma}{e}{\tau}$ \BY{definition of finite set union over \pfref{penultimate}}
%             \end{pfsteps*}
%             \resetpfcounter
%           \item[\text{(\ref{rule:eana-defuptsm})}] We have:
%             \begin{pfsteps*}
%               \item $\ue=\audefuptsm{\utau'}{\eparse}{\tsmv}{\ue'}$ \BY{assumption}
%               \item $\expandsTU{\uDelta}{\utau'}{\tau'}$ \BY{assumption} \pflabel{expandsTU}
%             %  \item \hastypeU{\emptyset}{\emptyset}{\eparse}{\aparr{\tBody}{\tParseResultExp}} \BY{assumption}\pflabel{eparse}
%               \item $\eana{\uDelta}{\uGamma}{\uPsi}{\uPhi, \uPhyp{\tsmv}{a}{\tau'}{\eparse}}{\ue'}{e}{\tau}$ \BY{assumption}\pflabel{expandsU}
%             %  \item \uetsmenv{\Delta}{\Psi} \BY{assumption}\pflabel{uetsmenv1}
%             %  \item \istypeU{\Delta}{\tau'} \BY{Lemma \ref{lemma:type-expansion-U} to \pfref{expandsTU}} \pflabel{istype}
%             %  \item \uetsmenv{\Delta}{\Psi, \xuetsmbnd{\tsmv}{\tau'}{\eparse}} \BY{Definition \ref{def:ueTLM-def-ctx-formation} on \pfref{uetsmenv1}, \pfref{istype} and \pfref{eparse}}\pflabel{uetsmenv3}
%               \item $\hastypeU{\Delta}{\Gamma}{e}{\tau}$ \BY{IH, part 1(b)(i) on \pfref{expandsU}}
%             \end{pfsteps*}
%             \resetpfcounter
%           \item[\text{(\ref{rule:eana-implicitp})}] We have:
%             \begin{pfsteps*}
%               \item $\ue=\auimplicitp{\tsmv}{\ue}$ \BY{assumption}
%               \item $\uPhi=\uASI{\uA \uplus \vExpands{\tsmv}{a}}{\Phi, \xuptsmbnd{a}{\tau'}{\eparse}}{\uI}$ \BY{assumption}
%               \item $\eana{\uDelta}{\uGamma}{\uPsi}{\uASI{\uA \uplus \vExpands{\tsmv}{a}}{\Phi, \xuptsmbnd{a}{\tau'}{\eparse}}{\uI \uplus \designate{\tau}{a}}}{\ue}{e}{\tau}$ \BY{assumption} \pflabel{esyn}
%               \item $\hastypeU{\Delta}{\Gamma}{e}{\tau}$ \BY{IH, part 1(b)(i) on \pfref{esyn}}
%             \end{pfsteps*}
%             \resetpfcounter
%         \end{byCases}
%       \item We induct on the assumption. There is one case.
%         \begin{byCases}
%           \item[\text{(\ref{rule:rana})}] We have:
%             \begin{pfsteps*}
%               \item $\urv=\aumatchrule{\upv}{\ue}$ \BY{assumption}
%               \item $r=\aematchrule{p}{e}$ \BY{assumption}
%               \item $\patExpands{\uGG{\uA'}{\pctx}}{\uPhi}{\upv}{p}{\tau}$ \BY{assumption} \pflabel{patExpands}
%               \item $\eana{\uDelta}{\uGG{{\uA}\uplus{\uA'}}{\Gcons{\Gamma}{\pctx}}}{\uPsi}{\uPhi}{\ue}{e}{\tau'}$ \BY{assumption} \pflabel{expandsUP}
%               \item $\patType{\pctx}{p}{\tau}$ \BY{Theorem \ref{thm:typed-pattern-expansion-B}, part 1 on \pfref{patExpands}}\pflabel{patType}
%               \item $\hastypeU{\Delta}{\Gcons{\Gamma}{\pctx}}{e}{\tau'}$ \BY{IH, part 1(b)(i) on \pfref{expandsUP}} \pflabel{hasType}
%               \item $\ruleType{\Delta}{\Gamma}{\aematchrule{p}{e}}{\tau}{\tau'}$ \BY{Rule (\ref{rule:ruleType}) on \pfref{patType} and \pfref{hasType}}
%             \end{pfsteps*}
%             \resetpfcounter
%         \end{byCases}
%     \end{enumerate}
%   \end{enumerate}
%   \item In the following, let $\uDelta=\uDD{\uD}{\Delta_\text{app}}$ and $\uGamma=\uGG{\uG}{\Gamma_\text{app}}$ and $\escenev=\esceneUP{\uDelta}{\uGamma}{\uPsi}{\uPhi}{b}$.
%   \begin{enumerate}
%     \item \begin{enumerate}
%       \item We induct on the assumption.
%         \begin{byCases}
%           \item[\text{(\ref{rule:csyn-var})}] We have:
%             \begin{pfsteps*}
%               \item $e=x$ \BY{assumption}
%               \item $\Gamma=\Gamma', \Ghyp{x}{\tau}$ \BY{assumption}
%               \item $\hastypeU{\Delta}{\Gamma', \Ghyp{x}{\tau}}{x}{\tau}$ \BY{Rule (\ref{rule:hastypeUP-var})}
%             \end{pfsteps*}
%             \resetpfcounter 
%           \item[\text{(\ref{rule:csyn-asc})}] We have:
%             \begin{pfsteps*}
%                \item $\ce=\aceasc{\ctau}{\ce'}$ \BY{assumption}
%                \item $\Delta \cap \Delta_\text{app}=\emptyset$ \BY{assumption} \pflabel{delta-disjoint}
%                \item $\domof{\Gamma} \cap \domof{\Gamma_\text{app}}=\emptyset$ \BY{assumption} \pflabel{gamma-disjoint}
%                \item $\cvalidT{\Delta}{\tsfrom{\escenev}}{\ctau}{\tau}$ \BY{assumption}\pflabel{expandsTU}
%                \item $\canaX{\ce'}{e}{\tau}$ \BY{assumption}\pflabel{eanaX}
%                \item $\istypeU{\Delta \cup \Delta_\text{app}}{\tau}$ \BY{Lemma \ref{lemma:candidate-expansion-type-validation} on \pfref{expandsTU}}\pflabel{istype}
%                \item $\hastypeU{\Delta}{\Gamma}{e}{\tau}$ \BY{IH, part 2(b)(i) to \pfref{eanaX}, \pfref{delta-disjoint}, \pfref{gamma-disjoint} and  \pfref{istype}}
%              \end{pfsteps*}
%              \resetpfcounter
%           \item[\text{(\ref{rule:csyn-let}) through (\ref{rule:csyn-match})}] In each of these cases, we apply:
%             \begin{itemize}
%               \item Lemma \ref{lemma:candidate-expansion-type-validation} to or over all ce-type validation premises.
%               \item The IH, part 2(a)(i) to or over all synthetic ce-expression validation premises.
%               \item The IH, part 2(a)(ii) to or over all synthetic ce-rule validation premises.
%               \item The IH, part 2(b)(i) to or over all analytic ce-expression validation premises.
%             \end{itemize}
%             We then derive the conclusion by applying Rules (\ref{rules:hastypeUP}), Rule (\ref{rule:ruleType}), Lemma \ref{lemma:weakening-UP},  the identification convention and exchange as needed.
%           \item[\text{(\ref{rule:csyn-splicede})}] We have:
%             \begin{pfsteps*}
%               \item $\ce=\acesplicede{m}{n}$ \BY{assumption}
%               \item $\parseUExp{\bsubseq{b}{m}{n}}{\ue}$ \BY{assumption}
%               \item $\esyn{\uDelta}{\uGamma}{\uPsi}{\uPhi}{\ue}{e}{\tau}$ \BY{assumption} \pflabel{expands}
%             %  \item $\uetsmenv{\Delta_\text{app}}{\Psi}$ \BY{assumption} \pflabel{uetsmenv}
%               \item $\Delta \cap \Delta_\text{app}=\emptyset$ \BY{assumption} \pflabel{delta-disjoint}
%               \item $\domof{\Gamma} \cap \domof{\Gamma_\text{app}}=\emptyset$ \BY{assumption} \pflabel{gamma-disjoint}
%               \item $\hastypeU{\Delta_\text{app}}{\Gamma_\text{app}}{e}{\tau}$ \BY{IH, part 1(a)(i) on \pfref{expands}} \pflabel{hastype}
%               \item $\hastypeU{\Dcons{\Delta}{\Delta_\text{app}}}{\Gcons{\Gamma}{\Gamma_\text{app}}}{e}{\tau}$ \BY{Lemma \ref{lemma:weakening-UP} over $\Delta$ and $\Gamma$ and exchange on \pfref{hastype}}
%             \end{pfsteps*}
%             \resetpfcounter
%         \end{byCases}
%       \item We induct on the assumption. There is one case.
%         \begin{byCases}
%           \item[\text{(\ref{rule:crsyn})}] We have:
%             \begin{pfsteps*}
%               \item $\crv=\acematchrule{p}{\ce}$ \BY{assumption}
%               \item $r=\aematchrule{p}{e}$ \BY{assumption}
%               \item $\patType{\pctx}{p}{\tau}$ \BY{assumption} \pflabel{patType}
%               \item $\csyn{\Delta}{\Gcons{\Gamma}{\pctx}}{\esceneUP{\uDelta}{\uGamma}{\uPsi}{\uPhi}{b}}{\ce}{e}{\tau'}$ \BY{assumption} \pflabel{cvalidE}
%               \item $\Delta \cap \Delta_\text{app} = \emptyset$ \BY{assumption}\pflabel{delta-disjoint}
%               \item $\domof{\Gamma} \cap \domof{\pctx} = \emptyset$ \BY{identification convention}\pflabel{gamma-disjoint1}
%               \item $\domof{\Gamma_\text{app}} \cap \domof{\pctx} = \emptyset$ \BY{identification convention}\pflabel{gamma-disjoint2}
%               \item $\domof{\Gamma} \cap \domof{\Gamma_\text{app}} = \emptyset$ \BY{assumption}\pflabel{gamma-disjoint3}
%               \item $\domof{\Gcons{\Gamma}{\pctx}} \cap \domof{\Gamma_\text{app}} = \emptyset$ \BY{standard finite set definitions and identities on \pfref{gamma-disjoint1}, \pfref{gamma-disjoint2} and \pfref{gamma-disjoint3}}\pflabel{gamma-disjoint4}
%               \item $\hastypeU{\Dcons{\Delta}{\Delta_\text{app}}}{\Gcons{\Gcons{\Gamma}{\pctx}}{\Gamma_\text{app}}}{e}{\tau'}$ \BY{IH, part 2(a)(i) on \pfref{cvalidE}, \pfref{delta-disjoint} and \pfref{gamma-disjoint4}}\pflabel{hastype}
%               \item $\hastypeU{\Dcons{\Delta}{\Delta_\text{app}}}{\Gcons{\Gcons{\Gamma}{\Gamma_\text{app}}}{\pctx}}{e}{\tau'}$ \BY{exchange of $\pctx$ and $\Gamma_\text{app}$ on \pfref{hastype}}\pflabel{hastype2}
%               \item $\ruleType{\Dcons{\Delta}{\Delta_\text{app}}}{\Gcons{\Gamma}{\Gamma_\text{app}}}{\aematchrule{p}{e}}{\tau}{\tau'}$ \BY{Rule (\ref{rule:ruleType}) on \pfref{patType} and \pfref{hastype2}}
%             \end{pfsteps*}
%             \resetpfcounter
%         \end{byCases}
%     \end{enumerate}
%     \item  \begin{enumerate}
%       \item We induct on the assumption.
%         \begin{byCases}
%           \item[\text{(\ref{rule:cana-subsume})}] We have:
%             \begin{pfsteps*}
%               \item $\csynX{\ce}{e}{\tau}$ \BY{assumption} \pflabel{esyn}
%               \item $\hastypeU{\Delta}{\Gamma}{e}{\tau}$ \BY{IH, part 2(a)(i) on \pfref{esyn}}
%             \end{pfsteps*}
%           \item[\text{(\ref{rule:cana-let}) through (\ref{rule:eana-match})}] In each of these cases, we apply:
%             \begin{itemize}
%               \item Lemma \ref{lemma:candidate-expansion-type-validation} to or over all ce-type validation premises.
%               \item The IH, part 2(a)(i) to or over all synthetic ce-expression validation premises.
%               \item The IH, part 2(a)(ii) to or over all synthetic ce-rule validation premises.
%               \item The IH, part 2(b)(i) to or over all analytic ce-expression validation premises.
%             \end{itemize}
%             We then derive the conclusion by applying Rules (\ref{rules:hastypeUP}), Rule (\ref{rule:ruleType}), Lemma \ref{lemma:weakening-UP},  the identification convention and exchange as needed.
%           \item[\text{(\ref{rule:cana-splicede})}] We have:
%             \begin{pfsteps*}
%               \item $\ce=\acesplicede{m}{n}$ \BY{assumption}
%               \item $\parseUExp{\bsubseq{b}{m}{n}}{\ue}$ \BY{assumption}
%               \item $\eana{\uDelta}{\uGamma}{\uPsi}{\uPhi}{\ue}{e}{\tau}$ \BY{assumption} \pflabel{expands}
%               \item $\istypeU{\Delta \cup \Delta_\text{app}}{\tau}$ \BY{assumption} \pflabel{istype}
%             %  \item $\uetsmenv{\Delta_\text{app}}{\Psi}$ \BY{assumption} \pflabel{uetsmenv}
%               \item $\Delta \cap \Delta_\text{app}=\emptyset$ \BY{assumption} \pflabel{delta-disjoint}
%               \item $\domof{\Gamma} \cap \domof{\Gamma_\text{app}}=\emptyset$ \BY{assumption} \pflabel{gamma-disjoint}
%               \item $\hastypeU{\Delta_\text{app}}{\Gamma_\text{app}}{e}{\tau}$ \BY{IH, part 1(b)(i) on \pfref{expands}, \pfref{delta-disjoint}, \pfref{gamma-disjoint} and \pfref{istype}} \pflabel{hastype}
%               \item $\hastypeU{\Dcons{\Delta}{\Delta_\text{app}}}{\Gcons{\Gamma}{\Gamma_\text{app}}}{e}{\tau}$ \BY{Lemma \ref{lemma:weakening-UP} over $\Delta$ and $\Gamma$ and exchange on \pfref{hastype}}
%             \end{pfsteps*}
%             \resetpfcounter
%         \end{byCases}
%       \item We induct on the assumption. There is one case.
%         \begin{byCases}
%           \item[\text{(\ref{rule:crana})}] We have:    
%             \begin{pfsteps*}
%                 \item $\crv=\acematchrule{p}{\ce}$ \BY{assumption}
%                 \item $r=\aematchrule{p}{e}$ \BY{assumption}
%                 \item $\patType{\pctx}{p}{\tau}$ \BY{assumption} \pflabel{patType}
%                 \item $\cana{\Delta}{\Gcons{\Gamma}{\pctx}}{\esceneUP{\uDelta}{\uGamma}{\uPsi}{\uPhi}{b}}{\ce}{e}{\tau'}$ \BY{assumption} \pflabel{cvalidE}
%                 \item $\istypeU{\Delta \cup \Delta_\text{app}}{\tau'}$ \BY{assumption} \pflabel{istype}
%                 \item $\domof{\Gamma} \cap \domof{\Gamma_\text{app}} = \emptyset$ \BY{assumption}\pflabel{gamma-disjoint3}
%                 \item $\Delta \cap \Delta_\text{app} = \emptyset$ \BY{assumption}\pflabel{delta-disjoint}
%                 \item $\domof{\Gamma} \cap \domof{\pctx} = \emptyset$ \BY{identification convention}\pflabel{gamma-disjoint1}
%                 \item $\domof{\Gamma_\text{app}} \cap \domof{\pctx} = \emptyset$ \BY{identification convention}\pflabel{gamma-disjoint2}
%                 \item $\domof{\Gcons{\Gamma}{\pctx}} \cap \domof{\Gamma_\text{app}} = \emptyset$ \BY{standard finite set definitions and identities on \pfref{gamma-disjoint1}, \pfref{gamma-disjoint2} and \pfref{gamma-disjoint3}}\pflabel{gamma-disjoint4}
%                 \item $\hastypeU{\Dcons{\Delta}{\Delta_\text{app}}}{\Gcons{\Gcons{\Gamma}{\pctx}}{\Gamma_\text{app}}}{e}{\tau'}$ \BY{IH, part 2(b)(i) on \pfref{cvalidE}, \pfref{delta-disjoint}, \pfref{gamma-disjoint4} and \pfref{istype}}\pflabel{hastype}
%                 \item $\hastypeU{\Dcons{\Delta}{\Delta_\text{app}}}{\Gcons{\Gcons{\Gamma}{\Gamma_\text{app}}}{\pctx}}{e}{\tau'}$ \BY{exchange of $\pctx$ and $\Gamma_\text{app}$ on \pfref{hastype}}\pflabel{hastype2}
%                 \item $\ruleType{\Dcons{\Delta}{\Delta_\text{app}}}{\Gcons{\Gamma}{\Gamma_\text{app}}}{\aematchrule{p}{e}}{\tau}{\tau'}$ \BY{Rule (\ref{rule:ruleType}) on \pfref{patType} and \pfref{hastype2}}
%               \end{pfsteps*}
%               \resetpfcounter

%         \end{byCases}
%     \end{enumerate}
%   \end{enumerate}
% \end{enumerate}

% We must now show that the induction is well-founded. All applications of the IH are on subterms except the following.  

% \begin{itemize}
% \item The only cases in the proof of part 1 that invoke the IH, part 2 are Case (\ref{rule:esyn-apuetsm}) in the proof of part 1(a)(i) and Case (\ref{rule:eana-lit}) in the proof of part 1(b)(i). The only cases in the proof of part 2 that invoke the IH, part 1 are Case (\ref{rule:csyn-splicede}) in the proof of part 2(a)(i) and Case (\ref{rule:cana-splicede}) in the proof of part 2(b)(i). We can show that the following metric on the judgements that we induct on is stable in one direction and strictly decreasing in the other direction:
% \begin{align*}
% \sizeof{\esyn{\uDelta}{\uGamma}{\uPsi}{\uPhi}{\ue}{e}{\tau}} & = \sizeof{\ue}\\
% \sizeof{\eana{\uDelta}{\uGamma}{\uPsi}{\uPhi}{\ue}{e}{\tau}} & = \sizeof{\ue}\\
% \sizeof{\csyn{\Delta}{\Gamma}{\esceneUP{\uDelta}{\uGamma}{\uPsi}{\uPhi}{b}}{\ce}{e}{\tau}} & = \sizeof{b}\\
% \sizeof{\cana{\Delta}{\Gamma}{\esceneUP{\uDelta}{\uGamma}{\uPsi}{\uPhi}{b}}{\ce}{e}{\tau}} & = \sizeof{b}
% \end{align*}
% where $\sizeof{b}$ is the length of $b$ and $\sizeof{\ue}$ is the sum of the lengths of the ueTLM literal bodies in $\ue$,
% \begin{align*}
% \sizeof{\ux} & = 0\\
% \sizeof{\auasc{\utau}{\ue}} & = \sizeof{\ue}\\
% \sizeof{\auletsyn{\ux}{\ue}{\ue'}} & = \sizeof{\ue} + \sizeof{\ue'}\\
% \sizeof{\auanalam{\ux}{\ue}} & = \sizeof{\ue}\\
% \sizeof{\aulam{\utau}{\ux}{\ue}} &= \sizeof{\ue}\\
% \sizeof{\auap{\ue_1}{\ue_2}} & = \sizeof{\ue_1} + \sizeof{\ue_2}\\
% \sizeof{\autlam{\ut}{\ue}} & = \sizeof{\ue}\\
% \sizeof{\autap{\ue}{\utau}} & = \sizeof{\ue}\\
% \sizeof{\auanafold{\ue}} & = \sizeof{\ue}\\
% \sizeof{\auunfold{\ue}} & = \sizeof{\ue}\\
% %\end{align*}
% %\begin{align*}
% \sizeof{\autpl{\labelset}{\mapschema{\ue}{i}{\labelset}}} & = \sum_{i \in \labelset} \sizeof{\ue_i}\\
% \sizeof{\aupr{\ell}{\ue}} & = \sizeof{\ue}\\
% \sizeof{\auanain{\ell}{\ue}} & = \sizeof{\ue}\\
% %\sizeof{\aucase{\labelset}{\utau}{\ue}{\mapschemab{\ux}{\ue}{i}{\labelset}}} & = \sizeof{\ue} + \sum_{i \in \labelset} \sizeof{\ue_i}\\
% \sizeof{\aumatchwithb{n}{\ue}{\seqschemaX{\urv}}} & = \sizeof{\ue} + \sum_{1 \leq i \leq n} \sizeof{r_i}\\
% \sizeof{\audefuetsm{\utau}{\eparse}{\tsmv}{\ue}} & = \sizeof{\ue}\\
% \sizeof{\auimplicite{\tsmv}{\ue}} & = \sizeof{\ue}\\
% \sizeof{\autsmap{b}{\tsmv}} & = \sizeof{b}\\
% \sizeof{\auelit{b}} & = \sizeof{b}\\
% \sizeof{\audefuptsm{\utau}{\eparse}{\tsmv}{\ue}} & = \sizeof{\ue}\\
% \sizeof{\auimplicitp{\tsmv}{\ue}} & = \sizeof{\ue}
% \end{align*}
% and $\sizeof{r}$ is defined as follows:
% \begin{align*}
% \sizeof{\aumatchrule{\upv}{\ue}} & = \sizeof{\ue}
% \end{align*}

% Going from part 1 to part 2, the metric remains stable:
% \begin{align*}
%  & \sizeof{\esyn{\uDelta}{\uGamma}{\uPsi}{\uPhi}{\autsmap{b}{\tsmv}}{e}{\tau}}\\
% =& \sizeof{\eana{\uDelta}{\uGamma}{\uPsi}{\uPhi}{\auelit{b}}{e}{\tau}}\\
% =& \sizeof{\cana{\emptyset}{\emptyset}{\esceneUP{\uDelta}{\uGamma}{\uPsi}{\uPhi}{b}}{\ce}{e}{\tau}}\\
% =&\sizeof{b}\end{align*}

% Going from part 2 to part 1, in each case we have that $\parseUExp{\bsubseq{b}{m}{n}}{\ue}$ and the IH is applied to the judgements $\esyn{\uDelta}{\uGamma}{\uPsi}{\uPhi}{\ue}{e}{\tau}$ and $\eana{\uDelta}{\uGamma}{\uPsi}{\uPhi}{\ue}{e}{\tau}$, respectively. Because the metric is stable when passing from part 1 to part 2, we must have that it is strictly decreasing in the other direction:
% \[\sizeof{\esyn{\uDelta}{\uGamma}{\uPsi}{\uPhi}{\ue}{e}{\tau}} < \sizeof{\csyn{\Delta}{\Gamma}{\esceneUP{\uDelta}{\uGamma}{\uPsi}{\uPhi}{b}}{\acesplicede{m}{n}}{e}{\tau}}\]
% and
% \[\sizeof{\eana{\uDelta}{\uGamma}{\uPsi}{\uPhi}{\ue}{e}{\tau}} < \sizeof{\cana{\Delta}{\Gamma}{\esceneUP{\uDelta}{\uGamma}{\uPsi}{\uPhi}{b}}{\acesplicede{m}{n}}{e}{\tau}}\]
% i.e. by the definitions above, 
% \[\sizeof{\ue} < \sizeof{b}\]

% This is established by appeal to Condition \ref{condition:body-subsequences}, which states that subsequences of $b$ are no longer than $b$, and the following condition, which states that an unexpanded expression constructed by parsing a textual sequence $b$ is strictly smaller, as measured by the metric defined above, than the length of $b$, because some characters must necessarily be used to delimit each literal body.
% \begin{condition}[Expression Parsing Monotonicity]\label{condition:body-parsing-B} If $\parseUExp{b}{\ue}$ then $\sizeof{\ue} < \sizeof{b}$.\end{condition}

% Combining Conditions \ref{condition:body-subsequences} and \ref{condition:body-parsing-B}, we have that $\sizeof{\ue} < \sizeof{b}$ as needed.
% \item In Case (\ref{rule:eana-subsume}) of the proof of part 1(b)(i), we apply the IH, part 1(a)(i), with $\ue=\ue$. This is well-founded because all applications of the IH, part 1(b)(i) elsewhere in the proof are on strictly smaller terms.
% \item Similarly, in Case (\ref{rule:cana-subsume}) of the proof of part 2(b)(i), we apply the IH, part 2(a)(i), with $\ce=\ce$. This is well-founded because all applications of the IH, part 2(b)(i) elsewhere in the proof are on strictly smaller terms.
% \end{itemize}
% \end{proof} 
\endgroup

\section{Parametric TLM Implicits}\label{sec:parametric-simple-implicits}
Incorporating simple implicits into a bidirectionally typed dialect of $\miniVerseParam$ would require that the implicit context, $\uI$, be a finite function from {equivalence classes} of types to TLM expressions, $\epsilon$ (rather than from syntactic types, $\tau$, to TLM names, $a$.)

We consider a more sophisticated mechanism that allows a TLM implicit designation itself  to operate over a parameterized family of types as future work in Sec. \ref{sec:parametric-designations}. 



% \part{Conclusion}
% !TEX root = omar-thesis.tex
\chapter{Discussion \& Conclusion}\label{chap:conclusion}
% \begin{quote}
% \emph{Any representation of reality we develop can be only partial. There is no finality, sometimes no single best representation. There is only deeper understanding, more revealing and enveloping representations.
% }

% -- Carl R. Woese \cite{woese}
% \end{quote}

\vspace{-4px}
\section{Summary of Contributions}

In summary, typed literal macros (TLMs) allow library providers to programmatically control the parsing and expansion of generalized expression and pattern literal forms. A proto-expansion validation step allows a TLM client to reason about types, binding and segmentation \emph{abstractly}, i.e. without examining the macro definition or the generated expansion in complete detail. Instead, the client needs only to be made aware of the type annotation on the applied TLM and the \emph{splice summary} of the generated proto-expansion, which locates and assigns a type to each spliced term within the literal body. This information can be communicated to the client using secondary notation (e.g. colors) together with a simple type inspection service similar to that available in many program editors today.

This work developed a series of progressively more sophisticated core calculi in order to formally characterize the mechanisms of TLM definition and application and the associated abstract reasoning principles. In particular:
\begin{enumerate}
\item Chapters \ref{chap:uetsms} and \ref{chap:uptsms} developed $\miniVersePat$, which communicates the central concepts of typed expansion and proto-expansion validation for simple expression and pattern TLMs, respectively. It also showed how we can formally establish the abstract reasoning principles available to TLM clients. The reasoning principles related to binding relied on the standard notion of capture-avoiding substitution (for ABTs), rather than on a macro-specific mechanism for fresh variable generation as in other formal accounts of macro systems (notably, the work of Herman and Wand \cite{Herman10:Theory,DBLP:conf/esop/HermanW08}.) 

\item Chapter \ref{chap:ptsms} then developed $\miniVerseParam$, which added type and module parameters. These enrich the TLM type discipline to allow a single TLM to operate across a parameterized family of types. They also allow TLMs to refer to external bindings through module parameters, rather than requiring that all external references go through spliced terms. Support for partial application in higher-order abbreviations lowers the syntactic cost of this explicit parameter-passing style. 

\item Chapter \ref{chap:static-eval} developed $\miniVersePH$, which adds a static environment shared between TLM definitions. This makes the job of the TLM provider easier, by giving them access to libraries, including those that themselves export TLM definitions. We gave examples of TLMs that are useful for defining other TLMs, including quasiquotation TLMs and a grammar-based parser generator. We also briefly discussed how TLMs interact with library management systems.

\item Finally, Chapter \ref{chap:tsls} developed Bidirectional $\miniVersePat$, which supports a mechanism of TLM implicits that further decreases the syntactic cost of applying a TLM while retaining the essential reasoning principles developed in the previous chapters.
\end{enumerate}

\vspace{-10px}
\section{Summary of Related Work}
\vspace{-4px}
Chapter \ref{chap:background} gave a detailed account of the various existing mechanisms of syntactic control. In this section, we briefly compare TLMs to this related work more directly.

Compared to dialect-oriented mechanisms of syntactic control, i.e. mechanisms that allow  providers to form a new syntax dialect by extending the context-free syntax of a language, TLMs provide clients with stronger abstract reasoning principles. In particular:
\begin{enumerate}
\item \textbf{Conflict}: Clients attempting to combine general syntax dialects that have been found to be individually free of syntactic conflicts cannot  be sure that the combined dialect will be free of syntactic conflicts. Only a system described by Schwerdfeger and Van Wyk allows reasoning modularly about conflicts, but for a restricted class of syntax dialects \cite{conf/pldi/SchwerdfegerW09,schwerdfeger2010context}. In contrast, TLM providers can reason modularly about syntactic conflicts because the context-free syntax of the language is fixed.
\item \textbf{Responsibility}: Clients using a combined syntax dialect cannot easily determine which constituent dialect is responsible for any given form. In contrast, TLM clients can reason about responsibility by following the binding structure of the language in the usual manner.
\item \textbf{Segmentation}: Clients of a syntax dialect cannot accurately determine which segments of the program text appear directly in the desugaring. In contrast, TLM clients can reason about segmentation by inspecting the splice summary. The information in a splice summary can be communicated straightforwardly by a program editor using secondary notation (e.g. colors.) The system guarantees that spliced segments are non-overlapping.
\item \textbf{Typing}: Clients of a syntax dialect cannot reason abstractly about types. In contrast, TLM clients clients can determine the type of any expansion by inspecting the parameter and type declarations on the TLM definition. The parse function and the expansion itself need not be inspected, i.e. these details can be held abstract. The splice summary also gives types for each spliced expression or pattern. This information can be communicated upon request by the type inspection service of a program editor.
\item \textbf{Binding}: Clients of a context-free syntax dialect cannot be sure that the desugaring is context-independent and that spliced terms are capture avoiding. In contrast, TLM clients can be sure that the expansion is completely context-independent and that spliced terms are capture-avoiding.
\end{enumerate}

Some existing typed term-rewriting macro systems, notably the Scala macro system \cite{ScalaMacros2013}, address most of these problems but sacrifice syntactic control by requiring that the macro repurpose what existing syntactic forms are otherwise defined. String literal forms cannot be repurposed flexibly because terms cannot be spliced hygienically out of string literal bodies. TLMs solve this problem by  distinguishing spliced segments explicitly within the generated proto-expansions. This allows the language to selectively give spliced terms access to application-site bindings during proto-expansion validation, while requiring that the remainder of the proto-expansion be context-independent.

This work also provides the first  type-theoretic account of a hygienic typed macro system integrated into a language with support for structural pattern matching and an ML-style module system. Let us consider some related work to justify this assertion.
\begin{itemize}
\item The Scala macro system has not been formally specified as of this writing. 
\item The macro calculus of  Herman and Wand \cite{DBLP:conf/esop/HermanW08,Herman10:Theory} uses types only to make manifest the binding structure of the generated expansion. The language that expansions are written in does not have an ML-style type structure. 

Their calculus uses tree locations, which originate in the early work of Gorn \cite{gorn1965explicit}, to identify macro arguments by their location within the macro argument list. These tree locations are conceptually reminiscent of our spliced segment references, which identify spliced terms by their lexical location within the literal body. However, the tree locations serve a different purpose in Herman and Wand's calculus -- they distinguish symbolic arguments that will appear in binding positions in the expansion. In our system, there is no way for spliced identifiers to appear in binding position (but see Sec. \ref{sec:controlled-binding} for further discussion of this point.) Spliced segment references instead serve to distinguish segments that appear as sub-terms in the proto-expansion from those parsed in some other way. 

Additionally, their calculus explicitly specifies the mechanism for fresh variable generation during expansion, whereas our approach relies on the standard type-theoretic notions of implicit alpha-variation and capture-avoiding substitution (for general ABTs.) This significantly simplifies our calculi. 

Finally, their calculus defines term representations and quasiquotation primitively because the binding structure of the expansion must be manifest in the type of the macro. In our work, the parse functions can perform arbitrary computations on simpler term representations -- there is no need to define quasiquotation primitively or encode the binding structure of  term representations in their types. This is possible because we enforce a stronger notion of context-independence -- not even definition-site bindings are available directly to expansions.

\item It is also worth mentioning MacroML \cite{ganz2001macros}. MacroML is only a staging macro system, i.e. it does not give the macro access to the syntax (parsed or unparsed) of the provided arguments but rather requires that it treat them parametically. As such, MacroML does not qualify as a mechanism of syntactic control (staging is motivated primarily by performance considerations.) MacroML also does not support pattern matching or ML-style modules.

\item Finally, the predecessor to the systems introduced in this work was the system of \emph{type-specific languages} (TSLs) that was defined in a previous paper \cite{TSLs}. That work introduced generalized literal forms and gave a bidirectionally typed protocol for type-directed parse function invocation. This work progresses beyond that work in several ways.

Most obviously, this work allows explicit TLM application. Different TLMs can therefore be defined at the same type without conflict. A subsequent short paper proposed explicit application of simple expression TLMs in a bidirectional typed setting, but did not provide any formal details \cite{sac15}. We do not assume a bidirectionally typed language when discussing explicit application (see Sec. \ref{sec:type-inference} below for more discussion on type inference.) 
Moreover, TLMs are not associated directly with named type definitions, like TSLs,  but rather can operate at any type.  

Our mechanism of TLM implicits, described in Chapter \ref{chap:tsls}, also differs from TSLs in that implicit designations are orthogonal to the type abbreviation mechanism. Clients are free to differentially control implicit designations as desired within a lexically scoped portion of a program.

Another distinction is that the metatheory presented in the earlier work establishes only the ability to reason abstractly about types. It does not establish the remaining abstract reasoning principles that have been a major focus of this work. In particular, there is no formal notion of hygiene (though it is discussed informally), and the mechanism does not guarantee that a valid segmentation will exist, nor associate types with segments.

Finally, the prior work on TSLs did not consider the mechanisms of pattern matching, parameterized types, modules, parametric expansions or static evaluation. All of these are important for integration into an ML-like functional language.

\end{itemize}

\vspace{-8px}
\section{Limitations \& Future Research Directions}\label{sec:future-work}
Let us conclude with a discussion of the present limitations of our work. When possible, we outline future research that might resolve these limitations.

\vspace{-4px}
\subsection{Integration Into a Full-Scale Functional Language Definition}
We left many orthogonal language features out of our calculi, for the sake of simplicity and to focus our exposition on our novel contributions. We leave the work of integrating TLMs into a full-scale functional language definition as future work. We hope that this work will serve to guide a variety of different efforts in this direction.

\vspace{-4px}
\subsection{Integration Into Languages From Other Design Traditions}
\label{sec:integration}
We conjecture that all of the mechanisms we have described could be integrated into dependently typed functional languages, e.g. Coq \cite{Coq:manual}, but leave the necessary technical developments as future work.

The mechanisms described in Chapter \ref{chap:uetsms} and Chapter \ref{chap:static-eval} could also be adapted for use in languages from the imperative and object-oriented traditions without difficulty. These languages do not typically support structural pattern matching, so the mechanism in Chapter \ref{chap:uptsms} may not be relevant. Parametric TLMs as described in Chapter \ref{chap:ptsms} require an ML-style module system, i.e. a system where types and values are packaged into modules, and these modules are classified by signatures. However, it may be possible to define a variant of this mechanism that treats type and value parameters separately (the difficulty being that the type of value parameters may mention prior type parameters.)

The mechanism of TLM implicits introduced in Chapter \ref{chap:tsls} assumes a bidirectionally typed unexpanded language. A number of full-scale languages are bidirectionally typed, notably including Scala \cite{odersky2008programming}. However, Scala, like many other object-oriented languages, supports subtyping. Subtyping would complicate the question of implicit TLM dispatch, because there may be designations at multiple supertypes of a given type.

Various forms of \emph{ad hoc} polymorphism, e.g. function/method overloading or type classes \cite{Hall:1996:TCH:227699.227700}, would also complicate the question of implicit TLM dispatch. One approach that may be worth exploring in future work would associate implicits with a type class, rather than with an individual type.

TLMs could also be adapted for use in dynamic languages like Racket  by eliminating the type annotation (thereby treating the language as ``uni-typed'' \cite{scott1980lambda,pfpl}.) In such a language (and  even in a language with richer type structure), it could be useful to allow TLM providers to annotate a TLM with a dynamic contract governing all generated expansions \cite{DBLP:conf/icfp/FindlerF02}. 

\vspace{-4px}
\subsection{Constraint-Based Type Inference}\label{sec:type-inference}
ML uses a constraint-based type inference system \emph{a la} Damas-Milner \cite{damas1982principal}. We conjecture that the mechanisms described up through Chapter \ref{chap:static-eval} are compatible with such a system -- most simply, by generating the constraints from the final expansion. Actually, it should be possible to perform type inference abstractly, using only the information in the splice summary. We leave the evaluation of this conjecture as future work.

The mechanism of TLM implicits developed in Chapter \ref{chap:tsls} assumed a bidirectional type system, i.e. one that only locally infers types \cite{Pierce:2000:LTI:345099.345100}. It may be possible to integrate implicit TLM dispatch with a constraint-based type inference system, but the approach to take is less clear. We leave the exploration of this question as future work.

\vspace{-4px}
\subsection{Module Expression Syntax Macros}
We did not consider situations where a library client wants to define syntactic sugar for  module expressions matching a given signature. It should be possible to ``replicate'' the mechanisms developed up through Chapter \ref{chap:static-eval} at the level of module expressions without difficulty. Implicit dispatch would be problematic at the module level because signatures are related by a notion of subtyping.

A similar approach might be considered for syntax macros at the level of types (or, more generally, constructions.) However, for an ML-like language where all types are classified by a single kind, type-level TLMs would obscure the identity of the type. This, in turn, might obscure the type structure of the program too much to be useful.

% We have explored syntax for types in a recent short paper \cite{sac15}. Care must be taken here, however, because knowing merely that the %We also did not consider situations where a library client wants to generate an expression or a pattern based on the structure of a type (e.g. automatiCarec generation of equality comparisons.) Finally, we do not consider situations that require modifications to the underlying type structure of a language (though ``reasonably programmable type structure'' is a rich avenue for future work.)


\subsection{Parameterized Implicit Designations}\label{sec:parametric-designations}
The mechanism of TLM implicits developed in Chapter \ref{chap:tsls} allows the client to designate a TLM at a single type. A more sophisticated mechanism would allow for parameterization of the TLM implicit designation itself, so that it could operate across a parameterized family of types. For example, we may want to be able to implicitly apply the parametric TLM \li{#\dolla#list'} at \emph{all} list types, with the parameters determined implicitly from the type supplied for analysis.

Na\"ively, we might imagine a designation that quantifies over modules, \li{L}, and types, \li{'a}, like this:
\begin{lstlisting}[numbers=none]
implicit syntax (L : LIST) 'a => $list' L 'a in (* ... *) end 
\end{lstlisting}
When encountering an unadorned literal form, the implicit dispatch mechanism must  instantiate each implicit parameter from the type provided for analysis. The problem is that there does not always exist a unique such instantiation. For example, the type expression \li{list(int)} makes no reference to any module matching the \li{LIST} signature, and there may be many such modules in context, so we cannot uniquely instantiate \li{L}.

To solve this problem, we would need to define a unique \emph{normal form} that serves as a representative for each equivalence class of types. A designation is deemed invalid if the normal form of its type does not mention every designation parameter. For example, the normal form of \li{L.list('a)} does not mention \li{L} (because \li{LIST.list} is not abstract), so the implicit designation above can simply be deemed invalid.

The following designation does not require instantiating a module variable, so it would be valid under this restriction (recalling that \li{#\dolla#list} was defined as a synonym for \li{#\dolla#list' List}):
\begin{lstlisting}[numbers=none]
implicit syntax 'a => $list 'a in (* ... *) end
\end{lstlisting}

The normal form of an abstract type would necessarily mention a module path, so the following parametric designation would also be valid:
\begin{lstlisting}[numbers=none]
implicit syntax (R : RX) => $r R in (* ... *) end 
\end{lstlisting}

It may be possible to use Crary's method for representing terms of the dependent singleton calculus in a beta-normal, eta-long form for this purpose \cite{DBLP:conf/lfmtp/Crary09}. Defining pattern matching over types of normal form, and incorporating this mechanism into the implicit dispatch mechanism, is left as future work.

% \subsection{TLM Packaging}\label{sec:tsm-packaging}

% In the exposition thusfar, we have assumed that TLMs have delimited scope. However, ideally, we would like to be able to define TLMs within a module:
% \begin{lstlisting}[numbers=none]
% structure Rxlib = struct 
%   type Rx = (* ... *)
%   syntax $rx at Rx { (* ... *) }
%   (* ... *)
% end
% \end{lstlisting}
% However, this leads to an important question: how can we write down a signature for the module \li{Rxlib}? One approach would be to simply duplicate the full definition of the TLM in the signature, but this leads to inelegant code duplication and raises the difficult question of how the language should decide whether one TLM is a duplicate of another. For this reason, in VerseML, a signature can only refer to a previously defined TLM. So, for example, we can write down a signature for \li{Rxlib} after it has been defined:

% \begin{lstlisting}[numbers=none]
% signature RXLIB = sig 
%   type Rx = (* ... *)
%   syntax $rx = Rxlib.$rx
%   (* ... *)
% end
% Rxlib : RXLIB (* check Rxlib against RXLIB after the fact *)
% \end{lstlisting}

% Alternatively, we can define the type \li{Rx} and the TLM \li{#\dolla#rx} before defining \li{Rxlib}:
% \begin{lstlisting}[numbers=none]
% local 
%   type Rx = (* ... *)
%   syntax $rx at Rx { (* ... *) }
% in 
%   structure Rxlib : 
%   sig 
%     type Rx = Rx
%     syntax $rx = $rx
%     (* ... *)
%   end = struct 
%     type Rx = Rx 
%     syntax $rx = $rx
%     (* ... *)
%   end
% end 
% \end{lstlisting}

% Another important question is: how does a TLM defined within a module at a type that is held abstract outside of that module operate? For example, consider the following:
% \begin{lstlisting}[numbers=none]
% local 
%   type Rx = (* ... *)
%   syntax $rx at Rx { (* ... *) }
% in 
%   structure Rxlib : 
%   sig 
%     type Rx (* held abstract *)
%     syntax $rx = $rx
%     (* ... *)
%   end = struct 
%     type Rx = Rx
%     syntax $rx = $rx
%     (* ... *)
%   end
% end 
% \end{lstlisting}
% If we apply \li{Rxlib.#\dolla#rx}, it may generate an expansion that uses the constructors of the \li{Rx} type. However, because the type is being held abstract, these constructors may not be visible at the application site. \todo{actually, this is why doing this is a bad idea. export TLMs only from units, not modules}

% \subsection{TSLs}

% \section{pTSLs By Example}
% For example, a module \lstinline{P} can associate the TLM \lstinline{rx} defined in the previous section with the abstract type \lstinline{R.t} by qualifying the definition of the sealed module it is defined by as follows:
% \begin{lstlisting}[numbers=none]
% module R = mod {
%   type t = (* ... *)
%   (* ... *)
% } :> RX with syntax rx
% \end{lstlisting}
% More generally, when sealing a module expression against a signature, the programmer can specify, for each abstract type that is generated, at most one previously defined TLMs. This TLM must take as its first parameter the module being sealed.

% The following function has the same expansion as \lstinline{example_using_tsm} but, by using the TSL just defined, it is more concise. Notice the return type annotation, which is necessary to ensure that the TSL can be unambiguously determined:
% \begin{lstlisting}[numbers=none]
% fun example_using_tsl(name : string) : R.t => /SURL@EURLnameSURL: %EURLssn/
% \end{lstlisting}

% As another example, let us consider the standard list datatype. We can use TSLs to express derived list syntax, for both expressions and patterns:
% \begin{lstlisting}[numbers=none]
% datatype list('a) { Nil | Cons of 'a * list('a) } with syntax {
%   static fn (body : Body) => 
%     (* ... comma-delimited spliced exps ... *)
% } with pattern syntax {
%   static fn (body : Body) : Pat => 
%     (* ... list pattern parser ... *)
% }
% \end{lstlisting}
% Together with the TSL for regular expression patterns, this allows us to write lists like this:
% \begin{lstlisting}[numbers=none]
% let val x : list(R.t) = [/SURL\dEURL/SHTML, EHTML/SURL\d\dEURL/SHTML, EHTML/SURL\d\d\dEURL/]
% \end{lstlisting}
% From the client's perspective, it is essentially as if the language had built in derived syntax for lists and regular expression patterns directly.%However, we did not need to build in this syntax primitively.%The only constraint is that this syntax must be used in an analytic position, which we argue is actually better for code compren when encountering unfamiliar syntax.

\subsection{Exportable Implicit Designations}
Implicit designations cannot be exported from a library, because different libraries might define conflicting designations. This can be inconvenient for clients.

This restriction is perhaps too severe in cases where the designation is at an abstract type generated from within the same library. In such situation, it would be safe to export an implicit designation because no other library could do the same.

% \todo{revise this} TSLs can be associated with abstract types that are generated by parameterized modules (i.e. generative functors in Standard ML) as well. For example, consider a trivially parameterized module that creates modules sealed against \lstinline{RX}:
% \begin{lstlisting}[numbers=none]
% module F() => mod {
%   type t = (* ... *)
%   (* ... *)
% } :> RX with syntax rx 
% \end{lstlisting}
% Each application of \lstinline{F} generates a distinct abstract type. The semantics associates the appropriately parameterized TLM with each of these as they are generated:
% \begin{lstlisting}[numbers=none]
% module F1 = F() (* F1.t has TSL rx(F1) *)
% module F2 = F() (* F2.t has TSL rx(F2) *)
% \end{lstlisting}

% As a more complex example, let us define two signatures, \lstinline{A} and \lstinline{B}, a TLM \texttt{\$G} and a parameterized module \lstinline{G : A -> B}:
% \begin{lstlisting}[numbers=none,mathescape=|]
% signature A = sig { type t; val x : t }
% signature B = sig { type u; val y : u }
% syntax $G(M : A)(G : B) at G.u { (* ... *) }
% module G(M : A) => mod { 
%   type u = M.t; val y = M.x } :> B with syntax $G(M)
% \end{lstlisting}
% Both \lstinline{G} and \texttt{\$G} take a parameter \lstinline{M : A}. We associate the partially applied TLM \texttt{\$G(M)} with the abstract type that \lstinline{G} generates. Again, this satisfies the requirement that one must be able to apply the TLM being associated with the abstract type to the module being sealed. 

% Only fully abstract types can have TSLs associated with them. Within the definition of \lstinline{G}, type \lstinline{u} does not have a TSL available to it because it is synonymous to \lstinline{M.t}. More generally, TSL lookup respects type equality, so any synonyms of a type with a TSL will also have that TSL. We can see this in the following example, where the type \lstinline{u} has a different TSL associated with it inside and outside the definition of the module \lstinline{N}:
% \begin{lstlisting}[numbers=none,mathescape=|]
% module M : A = mod { type t = int; val x = 0 }
% module G1 = G(M) (* G1.t has TSL $G(M), per above *)
% module N = mod { 
%   type u = G1.t (* u = G1.t in this scope, so u also has TSL $G(M) *)
%   val y = /asdf/ (* we can use it to create a value of that type *) 
% } :> B (* did not specify a TSL for N.u at the point where it is sealed, 
%             so N.u has no TSL in the outer scope *)
% val z : N.u = /asdf/ (* ERROR: no TSL for type N.u *)
% \end{lstlisting}

% A formal specification of TSLs in a language that supports only non-parametric datatypes is available in a paper published in ECOOP 2014 \cite{TSLs}. %We will add support for parameterized TSLs in the dissertation (see Sec. %\ref{sec:syntax-timeline}).

% \subsection{TLMs and TSLs In Candidate Expansions}\label{sec:tsms-in-expansions}


% Candidate expansions cannot themselves define or apply TLMs. This simplifies our metatheory, though it can be inconvenient at times for TLM providers. We discuss adding the ability to use TLMs within candidate expansions here.\todo{write this}

% \subsection{Pattern Matching Over Values of Abstract Type}\label{sec:patterns-for-abstract-types}
% ML does not presently support pattern matching over values of an abstract data type. However, there have been proposals for adding support for pattern matching over abstract data types defined by modules having a ``datatype-like'' shape, e.g. those that define a case analysis function like the one specified by \lstinline{RX}, %shown in Sec. \ref{sec:examples}. %We leave further discussion of such a facility and of parameterized TPSMs also as remaining work (see Sec. \ref{sec:syntax-timeline}). 

\subsection{Controlled Capture}\label{sec:controlled-binding}

The prohibition on capture makes it impossible to define new binding constructs. For example, consider Haskell's \textbf{do}-notation for monadic structures:
\begin{lstlisting}[numbers=none]
do x1 <- action1
   x2 <- action2
   action3 x1 x2
\end{lstlisting}
This desugars to:
\begin{lstlisting}[numbers=none]
action1 >>= \ x1 -> action2 >>= \ x2 -> action3 x1 x2
\end{lstlisting}
where \li{>>=} is infix application of the  \emph{bind} function and \li{\ x1 -> e} is Haskell's syntax for lambda abstraction.

If we na\"ively attempted to define something like \textbf{do}-notation using a TLM, the prohibition on capture would prevent \li{x1} from being visible within \li{action2}.

Completely relaxing the prohibition on capture would be unreasonable. Instead, we conjecture that there are two important constraints that need to be enforced:
\begin{enumerate}
  \item Identifiers that can be captured by spliced terms must themselves appear in the literal body. This leaves control over naming entirely to the client programmer.
  \item The splice summary must state which of these are available to each spliced term.
\end{enumerate}

The prior work of Herman and Wand developed mechanisms to support these sorts of examples \cite{DBLP:conf/esop/HermanW08,Herman10:Theory}. We leave integrating related mechanisms into a system of TLMs as future work. We also leave open the question of how an editor should best convey the set of available identifiers within a spliced term to the programmer.

\subsection{Type-Aware Splicing}\label{sec:type-aware-splicing}
Although typed expansion generalizes typing, there is no mechanism by which the expansion can branch based on the type synthesized by a spliced term. Recent work by Lorenzen and Erdweg on type-dependent desugarings gives some examples where branching according to the type of a spliced term might be useful \cite{conf/popl/LorenzenE16}. Developing a reasonable extension of our TLM mechanism for doing so is left as future work.

\subsection{TLM Application in Proto-Expansions}
Proto-expansions are abstract binding trees in our work, so it is not possible to apply TLMs within proto-expansions. This might occasionally be inconvenient for TLM providers. Developing the machinery necessary to be able to take TLMs as parameters and apply TLMs in proto-expansions is left as future work.

\subsection{Mechanically Reasoning About Parse Functions}\label{sec:verifying-tsms}
A correct parse function never returns an encoding of a proto-expansion that fails proto-expansion validation. This invariant cannot be enforced by the simple type systems we have considered in this work. Using a proof assistant, it would be possible to verify that a parse function generates only encodings of valid proto-expansions. Alternatively, in a dependently typed setting, the type of the parse function itself could be enriched so as to enforce this invariant intrinsically. We leave the details of this approach as future work.

A related problem is that a parse function might diverge. Again, we can either prove that the parse function does not diverge extrinsically, or define the parse function using a total language.

Parse functions implement some intended syntax definition. It would be useful to be able to state the syntax definition separately as a formal structure, and then prove that the parse function implements it correctly. In fact, in most cases, it should be possible to generate the parse function directly from the syntax definition using a parser generator. In this case, it would be useful to be able to mechanically prove the parser generator correct.

It would also be useful to develop the notion of an \emph{splice summary specification}, i.e. a specification of the splice summary that should result from a well-formed string. This could be combined with a grammar like the one shown in Figure \ref{fig:rx-grammar-based}, with the non-terminals representing spliced terms annotated with types.
% \subsection{Improved Error Reporting}\label{sec:error-handling}
% \todo{write this}

\subsection{Refactoring Unexpanded Terms}\label{sec:refactoring}
A crucial distinction is between identifiers, which appear in unexpanded terms, and variables, which appear in expanded terms. Variables are given meaning by substitution, and ABTs are identified only up to renaming of bound variables. In contrast, identifiers are given meaning only by expansion to variables and there is no notion of renaming or substitution. Unexpanded terms are not evaluated directly, so there is no need for these operations to assign static and dynamic meaning to programs.

It would, however, be useful to support identifier renaming and substitution operations for the purposes of automatic refactoring \cite{mens2004survey}. The simplest solution would be to use the splice summaries to locate spliced terms, and then perform the renaming directly within the literal body. The problem is that there is no guarantee that the parse function will produce an alpha-equivalent expansion after such a renaming operation has been performed. Similar concerns about invariance come up for other kinds of refactorings.

There are three approaches one might take to avoid this problem. The simplest approach is for the renaming operation to re-run the parse function and check that the expansion it generates is related to the previously generated expansion as expected. Alternatively, we might seek to mechanically verify that the parse function is invariant to refactoring, either intrinsically or extrinsically as discussed in Sec. \ref{sec:verifying-tsms}. 
A third approach would be to require that the TLM be defined using a grammar formalism that precludes inspection of the form of spliced expressions by construction. Exploration of these approaches is a promising avenue for future work.

\subsection{Integration with Editor Services}\label{sec:editor-integration}
Program editors often seek to provide feedback to the programmer about the syntax and semantics of the program being written. Questions remain about how various editor services should interact with TLMs. In the examples in this document, we colored spliced terms black and all other segments of a literal body some other uniform color (e.g. green.) A more sophisticated approach would allow the TLM to define its own syntax highlighting logic governing these non-spliced segments.

Another concern has to do with performance: na\"ively, a program editor would need to re-run the corresponding parse function on each edit that modified a literal body. Ideally, it should be possible to incrementally compute the resulting change to the splice summary as a function of the change to the literal body \cite{Ghezzi:1979:IP:357062.357066}. The generated expansions are context-independent, so there should be ample opportunity for caching.

Finally, we considered only reasoning principles for well-formed, well-typed programs. However, programmers often produce ill-formed or ill-typed programs. It is often useful to have error recovery heuristics that can be applied to provide useful feedback to the programmer in these circumstances. Error recovery within generalized literal forms might need the assistance of the TLM.


\subsection{Pretty Printing}\label{sec:resugaring}
We considered only the task of defining expressions and patterns using alternative syntactic forms. It is also generally useful to be able to ``pretty print'' (or ``unparse'') values of a given type using the same syntactic conventions, e.g. when using a REPL. It may be useful to explicitly associate a pretty-printer with a type using a mechanism closely related to our mechanism of TLM implicits. To avoid inconsistencies between the parsed syntax and the pretty-printed syntax, it is useful to generate both a parser and a pretty-printer from the same syntax definition, e.g. as described by van den Brand and Visser \cite{DBLP:journals/tosem/BrandV96} and implemented by many syntax definition systems. 

There has also been some prior work on \emph{resugaring}, i.e. retaining syntactic sugar while stepping through the evaluation of an expression \cite{DBLP:conf/pldi/PombrioK14,DBLP:conf/icfp/PombrioK15}. Extending these techniques to support syntactic forms defined via TLMs is left as future work.

\subsection{Structure Editing}
In this work, we assumed have that the surface syntax that the programmer interacts with is textual. However, there is an alternative approach:  \emph{structure editors} (a.k.a. structured editors, syntax-directed editors, projectional editors)  allow for a surface syntax that is tree-shaped, with holes standing for branches of the tree that have yet to be constructed. The programmer interacts with a projection of this tree structure using a language of edit actions. There are a number of prominent examples of structure editors, starting with the Cornell Program Synthesizer \cite{teitelbaum_cornell_1981}. In recent work, we have developed a type-theoretic foundation for structure editing by assigning static meaning to terms with holes, and formally defining a type-aware action semantics \cite{DBLP:conf/popl/OmarVHAH17}. TLMs could be incorporated into such a system with some modifications. In particular, the TLM would need to define a visual representation as well as an interaction model for a projection of an expression or pattern. Every state that the projection can be in would need to map onto an expansion. Spliced segments in such a system would correspond to holes that appear within the projection. Detailing this system of ``typed projection macros'' is left as future work. We elaborated on this idea in a recent ``vision paper'' \cite{snapl17}.

\section{Concluding Remarks}
Strong abstract reasoning principles dramatically increase the usability of a programming language by allowing the programmer to ignore (i.e. hold abstract) certain details when reasoning about the behavior of a program. Similarly, syntactic sugar that captures the idioms common to an application domain more concisely or naturally can dramatically increase the usability of a programming language by decreasing the cognitive cost of producing and examining programs. This work aimed to show that these considerations need not be in opposition -- it is possible, if formal care is taken, to define a programming language with a \emph{reasonably programmable syntax}. 

\newpage
\section*{\LaTeX~Sources and Errata}
\addcontentsline{toc}{chapter}{\LaTeX~Source Code and Updates}
\noindent
The \LaTeX~sources for this document can be found at the following URL:
\begin{center}
\url{https://github.com/cyrus-/thesis}
\end{center}
The latest version of this document can be downloaded from the following URL:
\begin{center}
\url{http://www.cs.cmu.edu/~comar/omar-thesis.pdf}
\end{center}
Any errors or omissions can be reported using GitHub's issue tracker, or by sending an email to the author:
\begin{center}
\url{comar@cs.cmu.edu}
\end{center}
Any changes to this document that occur after the final dissertation  has been submitted to the university will be summarized below.

\begin{enumerate}
	\item There was a minor syntax highlighting mistake in Figure \ref{fig:big-html-example}.
\end{enumerate}




% !TEX root = ./icfp18-supplement.tex
% !TEX root = omar-thesis.tex
\ificfp\else \part*{Appendix} \fi

\ificfp
\chapter{\texorpdfstring{$\miniVersePat$: A Calculus of Simple TLMs}{A Calculus of Simple TLMs}}
\else
\chapter{\texorpdfstring{$\miniVerseUE$ and $\miniVersePat$}{miniVerseSE and miniVerseS}}
\fi
\label{appendix:miniVerseSES}

\ificfp
This section defines $\miniVersePat$, the calculus of simple expression and pattern TLMs. For some readers, it might be useful to snip out pattern matching to get a language strictly of expression TLMs. To support that, one can omit the segments typeset in gray backgrounds below to recover $\miniVerseUE$, a calculus of simple expression TLMs. We have included the necessary eliminators below (they are technically redundant with pattern matching, but don't hurt things so they're left in white.)
\else

This section defines $\miniVersePat$, the language of Chapter \ref{chap:uptsms}. The language of Chapter \ref{chap:uetsms}, $\miniVerseUE$, can be recovered by omitting the segments typeset in a gray backgrounds below.
\fi

\clearpage

\section{Typographic Conventions}\label{appendix:typographic-conventions}
We adopt \emph{PFPL}'s typographic conventions for abstract binding trees \cite{pfpl}. In particular, the names of operators and indexed families of operators are written in $\texttt{typewriter font}$, indexed families of operators specify indices within $[\text{braces}]$ (except when the index is a label set, $\labelset$, or natural number, $n$, in which case it is omitted). Term arguments are grouped roughly by sort using \texttt{\{}curly braces\texttt{\}} and \texttt{(}rounded braces\texttt{)}. We write $p.e$ for expressions binding the variables that appear in the pattern $p$. The variables in a pattern must be distinct. 

We write $\mapschema{\tau}{i}{\labelset}$ for an unordered collection of type arguments $\tau_i$, one for each $i\in \labelset$, and similarly for arguments of other sorts. Similarly, we write $\mapschema{J}{i}{\labelset}$ for the finite set of derivations $J_i$ for each $i \in \labelset$.

We write $\seqschemaX{r}$ for sequences of $n \geq 0$ rule arguments, and similarly for other finite sequences. 

Empty finite sets and finite functions are written $\emptyset$, or omitted entirely within judgements, and non-empty finite sets and finite functions are written as comma-separated sequences identified up to exchange and contraction.


\section{Core Language}\label{appendix:SES-XL}
\subsection{Syntax}
% \begin{figure}[h!]
\[\begin{array}{lllllll}
\textbf{Sort} & & 
& \textbf{Operational Form} 
% & \textbf{Stylized Form} 
& \textbf{Description}\\
\mathsf{Typ} & \tau & ::= & t 
%& t 
& \text{variable}\\
&&& \aparr{\tau}{\tau} 
%& \parr{\tau}{\tau} 
& \text{partial function}\\
&&& \aall{t}{\tau} 
%& \forallt{t}{\tau} 
& \text{polymorphic}\\
&&& \arec{t}{\tau} 
%& \rect{t}{\tau} 
& \text{recursive}\\
&&& \aprod{\labelset}{\mapschema{\tau}{i}{\labelset}} 
%& \prodt{\mapschema{\tau}{i}{\labelset}} 
& \text{labeled product}\\
&&& \asum{\labelset}{\mapschema{\tau}{i}{\labelset}} 
%& \sumt{\mapschema{\tau}{i}{\labelset}} 
& \text{labeled sum}\\
\mathsf{Exp} & e & ::= & x 
%& x 
& \text{variable}\\
&&& \aelam{\tau}{x}{e} 
%& \lam{x}{\tau}{e} 
& \text{abstraction}\\
&&& \aeap{e}{e} 
%& \ap{e}{e} 
& \text{application}\\
&&& \aetlam{t}{e} 
%& \Lam{t}{e} 
& \text{type abstraction}\\
&&& \aetap{e}{\tau} 
%& \App{e}{\tau} 
& \text{type application}\\
&&& \aefold{e} 
%& \fold{e} : \tau 
& \text{fold}\\
&&& \aeunfold{e} 
%& \unfold{e} 
& \text{unfold}\\
&&& \aetpl{\labelset}{\mapschema{e}{i}{\labelset}} 
%& \tpl{\mapschema{e}{i}{\labelset}} 
& \text{labeled tuple}\\
&&& \aepr{\ell}{e} 
%& \prj{e}{\ell} 
& \text{projection}\\
&&& \aein{\ell}{e} 
%& \inj{\ell}{e} 
& \text{injection}\\
&&& \aecase{\labelset}{e}{\mapschemab{x}{e}{i}{\labelset}} 
%& \caseof{e}{\mapschemab{x}{e}{i}{\labelset}} 
& \text{case analysis}\\
\LCC \color{light-gray} & \color{light-gray} & \color{light-gray} 
% & \color{light-gray} 
& \color{light-gray} & \color{light-gray} \\
&&
& \aematchwith{n}{e}{\seqschemaX{r}}
% & \matchwith{e}{\seqschemaX{r}} 
& \text{match}\\
\mathsf{Rule} & r & ::= 
& \aematchrule{p}{e} 
%& \matchrule{p}{e} 
& \text{rule}\\
\mathsf{Pat} & p & ::= 
& x  
%& x 
& \text{variable pattern}\\
&&& \aewildp 
%& \wildp 
& \text{wildcard pattern}\\
&&& \aefoldp{p} 
%& \foldp{p} 
& \text{fold pattern}\\
&&& \aetplp{\labelset}{\mapschema{p}{i}{\labelset}} 
%& \tplp{\mapschema{p}{i}{\labelset}} 
& \text{labeled tuple pattern}\\
&&& \aeinjp{\ell}{p} 
%& \injp{\ell}{p} 
& \text{injection pattern} \ECC
\end{array}\]
% \caption{Syntax of the $\miniVersePat$ expanded language (XL).}
% \end{figure}

\subsection{Static Semantics}
\emph{Type formation contexts}, $\Delta$, are finite sets of hypotheses of the form $\Dhyp{t}$. We write $\Delta, \Dhyp{t}$ when $\Dhyp{t} \notin \Delta$ for $\Delta$ extended with the hypothesis $\Dhyp{t}$. %Finite sets are written as finite sequences identified up to exchange.% We write $\Dcons{\Delta}{\Delta'}$ for the union of $\Delta$ and $\Delta'$.

\emph{Typing contexts}, $\Gamma$, are finite functions that map each variable $x \in \domof{\Gamma}$, where $\domof{\Gamma}$ is a finite set of variables, to the hypothesis $\Ghyp{x}{\tau}$, for some $\tau$. We write $\Gamma, \Ghyp{x}{\tau}$, when $x \notin \domof{\Gamma}$, for the extension of $\Gamma$ with a mapping from $x$ to $\Ghyp{x}{\tau}$, and $\Gcons{\Gamma}{\Gamma'}$ when $\domof{\Gamma} \cap \domof{\Gamma'} = \emptyset$ for the typing context mapping each $x \in \domof{\Gamma} \cup \domof{\Gamma'}$ to $x : \tau$ if $x : \tau \in \Gamma$ or $x : \tau \in \Gamma'$. We write $\isctxU{\Delta}{\Gamma}$ if every type in $\Gamma$ is well-formed relative to $\Delta$.
\begin{definition}[Typing Context Formation] \label{def:isctxU}
$\isctxU{\Delta}{\Gamma}$ iff for each hypothesis $x : \tau \in \Gamma$, we have $\istypeU{\Delta}{\tau}$.
\end{definition}

\noindent\fbox{\strut$\istypeU{\Delta}{\tau}$}~~$\tau$ is a well-formed type
\begin{subequations}\label{rules:istypeU}
\begin{equation}\label{rule:istypeU-var}
\inferrule{ }{\istypeU{\Delta, \Dhyp{t}}{t}}
\end{equation}
\begin{equation}\label{rule:istypeU-parr}
\inferrule{
  \istypeU{\Delta}{\tau_1}\\
  \istypeU{\Delta}{\tau_2}
}{\istypeU{\Delta}{\aparr{\tau_1}{\tau_2}}}
\end{equation}
\begin{equation}\label{rule:istypeU-all}
  \inferrule{
    \istypeU{\Delta, \Dhyp{t}}{\tau}
  }{
    \istypeU{\Delta}{\aall{t}{\tau}}
  }
\end{equation}
\begin{equation}\label{rule:istypeU-rec}
  \inferrule{
    \istypeU{\Delta, \Dhyp{t}}{\tau}
  }{
    \istypeU{\Delta}{\arec{t}{\tau}}
  }
\end{equation}
\begin{equation}\label{rule:istypeU-prod}
  \inferrule{
    \{\istypeU{\Delta}{\tau_i}\}_{i \in \labelset}
  }{
    \istypeU{\Delta}{\aprod{\labelset}{\mapschema{\tau}{i}{\labelset}}}
  }
\end{equation}
\begin{equation}\label{rule:istypeU-sum}
  \inferrule{
    \{\istypeU{\Delta}{\tau_i}\}_{i \in \labelset}
  }{
    \istypeU{\Delta}{\asum{\labelset}{\mapschema{\tau}{i}{\labelset}}}
  }
\end{equation}
\end{subequations}

\noindent\fbox{\strut$\hastypeU{\Delta}{\Gamma}{e}{\tau}$}~~$e$ is assigned type $\tau$
\begin{subequations}\label{rules:hastypeU}\label{rules:hastypeUP}
\begin{equation}\label{rule:hastypeU-var}
  \inferrule{ }{
    \hastypeU{\Delta}{\Gamma, \Ghyp{x}{\tau}}{x}{\tau}
  }
\end{equation}
\begin{equation}\label{rule:hastypeU-lam}
  \inferrule{
    \istypeU{\Delta}{\tau}\\
    \hastypeU{\Delta}{\Gamma, \Ghyp{x}{\tau}}{e}{\tau'}
  }{
    \hastypeU{\Delta}{\Gamma}{\aelam{\tau}{x}{e}}{\aparr{\tau}{\tau'}}
  }
\end{equation}
\begin{equation}\label{rule:hastypeU-ap}
  \inferrule{
    \hastypeU{\Delta}{\Gamma}{e_1}{\aparr{\tau}{\tau'}}\\
    \hastypeU{\Delta}{\Gamma}{e_2}{\tau}
  }{
    \hastypeU{\Delta}{\Gamma}{\aeap{e_1}{e_2}}{\tau'}
  }
\end{equation}
\begin{equation}\label{rule:hastypeU-tlam}
  \inferrule{
    \hastypeU{\Delta, \Dhyp{t}}{\Gamma}{e}{\tau}
  }{
    \hastypeU{\Delta}{\Gamma}{\aetlam{t}{e}}{\aall{t}{\tau}}
  }
\end{equation}
\begin{equation}\label{rule:hastypeU-tap}
  \inferrule{
    \hastypeU{\Delta}{\Gamma}{e}{\aall{t}{\tau}}\\
    \istypeU{\Delta}{\tau'}
  }{
    \hastypeU{\Delta}{\Gamma}{\aetap{e}{\tau'}}{[\tau'/t]\tau}
  }
\end{equation}
\begin{equation}\label{rule:hastypeU-fold}
  \inferrule{\
    % \istypeU{\Delta, \Dhyp{t}}{\tau}\\
    \hastypeU{\Delta}{\Gamma}{e}{[\arec{t}{\tau}/t]\tau}
  }{
    \hastypeU{\Delta}{\Gamma}{\aefold{e}}{\arec{t}{\tau}}
  }
\end{equation}
\begin{equation}\label{rule:hastypeU-unfold}
  \inferrule{
    \hastypeU{\Delta}{\Gamma}{e}{\arec{t}{\tau}}
  }{
    \hastypeU{\Delta}{\Gamma}{\aeunfold{e}}{[\arec{t}{\tau}/t]\tau}
  }
\end{equation}
\begin{equation}\label{rule:hastypeU-tpl}
  \inferrule{
    \{\hastypeU{\Delta}{\Gamma}{e_i}{\tau_i}\}_{i \in \labelset}
  }{
    \hastypeU{\Delta}{\Gamma}{\aetpl{\labelset}{\mapschema{e}{i}{\labelset}}}{\aprod{\labelset}{\mapschema{\tau}{i}{\labelset}}}
  }
\end{equation}
\begin{equation}\label{rule:hastypeU-pr}
  \inferrule{
    \hastypeU{\Delta}{\Gamma}{e}{\aprod{\labelset, \ell}{\mapschema{\tau}{i}{\labelset}; \ell \hookrightarrow \tau}}
  }{
    \hastypeU{\Delta}{\Gamma}{\aepr{\ell}{e}}{\tau}
  }
\end{equation}
\begin{equation}\label{rule:hastypeU-in}
  \inferrule{
    % \{\istypeU{\Delta}{\tau_i}\}_{i \in \labelset}\\
    % \istypeU{\Delta}{\tau}\\
    \hastypeU{\Delta}{\Gamma}{e}{\tau}
  }{
    \hastypeU{\Delta}{\Gamma}{\aein{\ell}{e}}{\asum{\labelset, \ell}{\mapschema{\tau}{i}{\labelset}; \ell \hookrightarrow \tau}}
  }
\end{equation}
\begin{equation}\label{rule:hastypeU-case}
  \inferrule{
    \hastypeU{\Delta}{\Gamma}{e}{\asum{\labelset}{\mapschema{\tau}{i}{\labelset}}}\\
    % \istypeU{\Delta}{\tau}\\
    \{\hastypeU{\Delta}{\Gamma, x_i : \tau_i}{e_i}{\tau}\}_{i \in \labelset}
  }{
    \hastypeU{\Delta}{\Gamma}{\aecase{\labelset}{e}{\mapschemab{x}{e}{i}{\labelset}}}{\tau}
  }
\end{equation}
\begin{grayparbox}
\begin{equation}\label{rule:hastypeUP-match}
\graybox{\inferrule{
  \hastypeU{\Delta}{\Gamma}{e}{\tau}\\
  % \istypeU{\Delta}{\tau'}\\
  \{\ruleType{\Delta}{\Gamma}{r_i}{\tau}{\tau'}\}_{1 \leq i \leq n}\\
}{\hastypeU{\Delta}{\Gamma}{\aematchwith{n}{e}{\seqschemaX{r}}}{\tau'}}}
\end{equation}
\end{grayparbox}
\end{subequations}

\vspace{-5px}\begin{grayparbox}
\vspace{5px}\noindent\fcolorbox{black}{light-gray}{\strut$\ruleType{\Delta}{\Gamma}{r}{\tau}{\tau'}$}~~$r$ takes values of type $\tau$ to values of type $\tau'$
\begin{equation}\label{rule:ruleType}
\graybox{\inferrule{
  \patType{\pctx'}{p}{\tau}\\
  \hastypeU{\Delta}{\Gcons{\Gamma}{\pctx'}}{e}{\tau'}
}{\ruleType{\Delta}{\Gamma}{\aematchrule{p}{e}}{\tau}{\tau'}}}
\end{equation}
Rule (\ref{rule:ruleType}) is defined mutually inductively with Rules (\ref{rules:hastypeUP}).

\noindent\fcolorbox{black}{light-gray}{\strut$\patType{\Gamma}{p}{\tau}$}~~$p$ matches values of type $\tau$ and generates hypotheses $\pctx$
\begin{subequations}\label{rules:patType}
\begin{equation}\label{rule:patType-var}
\graybox{\inferrule{ }{\patType{\Ghyp{x}{\tau}}{x}{\tau}}}
\end{equation}
\begin{equation}\label{rule:patType-wild}
\graybox{\inferrule{ }{\patType{\emptyset}{\aewildp}{\tau}}}
\end{equation}
\begin{equation}\label{rule:patType-fold}
\graybox{\inferrule{
  \patType{\pctx}{p}{[\arec{t}{\tau}/t]\tau}
}{
  \patType{\pctx}{\aefoldp{p}}{\arec{t}{\tau}}
}}
\end{equation}
\begin{equation}\label{rule:patType-tpl}
\graybox{\inferrule{
  \{\patType{\pctx_i}{p_i}{\tau_i}\}_{i \in \labelset}
}{
  \patType{\Gconsi{i \in \labelset}{\pctx_i}}{\aetplp{\labelset}{\mapschema{p}{i}{\labelset}}}{\aprod{\labelset}{\mapschema{\tau}{i}{\labelset}}}
}}
\end{equation}
\begin{equation}\label{rule:patType-inj}
\graybox{\inferrule{
  \patType{\pctx}{p}{\tau}
}{
  \patType{\pctx}{\aeinjp{\ell}{p}}{\asum{\labelset, \ell}{\mapschema{\tau}{i}{\labelset}; \mapitem{\ell}{\tau}}}
}}
\end{equation}
\end{subequations}
\end{grayparbox}

\subsubsection{Metatheory}
The rules above are syntax-directed, so we assume an inversion lemma for each rule as needed without stating it separately or proving it explicitly. The following standard lemmas also hold.

The Weakening Lemma establishes that extending the context with unnecessary hypotheses preserves well-formedness and typing.
\begin{lemma}[Weakening]\label{lemma:weakening-UP}\label{lemma:weakening-U} ~
\begin{enumerate} 
\item If $\istypeU{\Delta}{\tau}$ then $\istypeU{\Delta, \Dhyp{t}}{\tau}$.
%\item If $\isctxU{\Delta}{\Gamma}$ then $\isctxU{\Delta, \Dhyp{t}}{\Gamma}$.
\item \begin{enumerate}
  \item If $\hastypeU{\Delta}{\Gamma}{e}{\tau}$ then $\hastypeU{\Delta, \Dhyp{t}}{\Gamma}{e}{\tau}$.
  \item \graytxtbox{If $\ruleType{\Delta}{\Gamma}{r}{\tau}{\tau'}$ then $\ruleType{\Delta, \Dhyp{t}}{\Gamma}{r}{\tau}{\tau'}$.}
  \end{enumerate}
\item \begin{enumerate}
  \item If $\hastypeU{\Delta}{\Gamma}{e}{\tau}$ and $\istypeU{\Delta}{\tau''}$ then $\hastypeU{\Delta}{\Gamma, \Ghyp{x}{\tau''}}{e}{\tau}$.
  \item \graytxtbox{If $\ruleType{\Delta}{\Gamma}{r}{\tau}{\tau'}$ and $\istypeU{\Delta}{\tau''}$ then $\ruleType{\Delta}{\Gamma, \Ghyp{x}{\tau''}}{r}{\tau}{\tau'}$.}
  \end{enumerate}
\item \graytxtbox{If $\patType{\pctx}{p}{\tau}$ then $\patTypeD{\Delta, \Dhyp{t}}{\pctx}{p}{\tau}$.}
\end{enumerate}
\end{lemma}
\begin{proof-sketch} ~
\begin{enumerate}
\item By rule induction over Rules (\ref{rules:istypeU}).
%\item By rule induction over Rules (\ref{rules:isctxU}).
\item By \graytxtbox{mutual} rule induction over Rules (\ref{rules:hastypeUP}) \graytxtbox{and Rule (\ref{rule:ruleType})}, and part 1.
\item By \graytxtbox{mutual} rule induction over Rules (\ref{rules:hastypeUP}) \graytxtbox{and Rule (\ref{rule:ruleType})}, and part 1.
\item \graytxtbox{By rule induction over Rules (\ref{rules:patType}).}
\end{enumerate}
\end{proof-sketch}

\begin{grayparbox}
Note clause 4, which allows weakening of $\Delta$ but requires that the {pattern typing judgement} is \emph{linear} in the pattern typing context, i.e. it does \emph{not} obey weakening of the pattern typing context. This is to ensure that the pattern typing context captures exactly those hypotheses generated by a pattern, and no others.
\end{grayparbox}

The Substitution Lemma establishes that substitution of a well-formed type for a type variable, or an expanded expression of the appropriate type for an expanded expression variable, preserves well-formedness and typing.
\begin{lemma}[Substitution]\label{lemma:substitution-UP} ~
\begin{enumerate}
\item If $\istypeU{\Delta, \Dhyp{t}}{\tau}$ and $\istypeU{\Delta}{\tau'}$ then $\istypeU{\Delta}{[\tau'/t]\tau}$.
%\item If $\isctxU{\Delta, \Dhyp{t}}{\Gamma}$ and $\istypeU{\Delta}{\tau'}$ then $\isctxU{\Delta}{[\tau'/t]\Gamma}$.
\item \begin{enumerate}
  \item If $\hastypeU{\Delta, \Dhyp{t}}{\Gamma}{e}{\tau}$ and $\istypeU{\Delta}{\tau'}$ then $\hastypeU{\Delta}{[\tau'/t]\Gamma}{[\tau'/t]e}{[\tau'/t]\tau}$.
  \item \begin{grayparbox} 
  {If} $\ruleType{\Delta, \Dhyp{t}}{\Gamma}{r}{\tau}{\tau''}$ and $\istypeU{\Delta}{\tau'}$ then $\ruleType{\Delta}{[\tau'/t]\Gamma}{[\tau'/t]r}{[\tau'/t]\tau}{[\tau'/t]\tau''}$.
  \end{grayparbox}
  \end{enumerate}
\item \begin{enumerate}
  \item If $\hastypeU{\Delta}{\Gamma, \Ghyp{x}{\tau'}}{e}{\tau}$ and $\hastypeU{\Delta}{\Gamma}{e'}{\tau'}$ then $\hastypeU{\Delta}{\Gamma}{[e'/x]e}{\tau}$.
  \item \graytxtbox{
  If $\ruleType{\Delta}{\Gamma, \Ghyp{x}{\tau'}}{r}{\tau}{\tau''}$ and $\hastypeU{\Delta}{\Gamma}{e'}{\tau''}$ then $\ruleType{\Delta}{\Gamma}{[e'/x]r}{\tau}{\tau''}$.}
  \end{enumerate}
\end{enumerate}\end{lemma}
\begin{proof-sketch} ~
\begin{enumerate}
\item By rule induction over Rules (\ref{rules:istypeU}).
\item By \graytxtbox{mutual} rule induction over Rules (\ref{rules:hastypeUP}) \graytxtbox{and Rule (\ref{rule:ruleType})}.
\item By \graytxtbox{mutual} rule induction over Rules (\ref{rules:hastypeUP}) \graytxtbox{and Rule (\ref{rule:ruleType})}.
\end{enumerate}
\end{proof-sketch}

The Decomposition Lemma is the converse of the Substitution Lemma.
\begin{lemma}[Decomposition]\label{lemma:decomposition-UP} ~
\begin{enumerate}
\item If $\istypeU{\Delta}{[\tau'/t]\tau}$ and $\istypeU{\Delta}{\tau'}$ then $\istypeU{\Delta, \Dhyp{t}}{\tau}$.
%\item If $\isctxU{\Delta}{[\tau'/t]\Gamma}$ and $\istypeU{\Delta}{\tau'}$ then $\isctxU{\Delta, \Dhyp{t}}{\Gamma}$.
\item \begin{enumerate}
  \item If $\hastypeU{\Delta}{[\tau'/t]\Gamma}{[\tau'/t]e}{[\tau'/t]\tau}$ and $\istypeU{\Delta}{\tau'}$ then $\hastypeU{\Delta, \Dhyp{t}}{\Gamma}{e}{\tau}$.
  \item \begin{grayparbox}
  If $\ruleType{\Delta}{[\tau'/t]\Gamma}{[\tau'/t]r}{[\tau'/t]\tau}{[\tau'/t]\tau''}$ and $\istypeU{\Delta}{\tau'}$ then $\ruleType{\Delta, \Dhyp{t}}{\Gamma}{r}{\tau}{\tau''}$.
  \end{grayparbox}
  \end{enumerate}
\item \begin{enumerate}
  \item If $\hastypeU{\Delta}{\Gamma}{[e'/x]e}{\tau}$ and $\hastypeU{\Delta}{\Gamma}{e'}{\tau'}$ then $\hastypeU{\Delta}{\Gamma, \Ghyp{x}{\tau'}}{e}{\tau}$.
  \item \graytxtbox{If $\ruleType{\Delta}{\Gamma}{[e'/x]r}{\tau}{\tau''}$ and $\hastypeU{\Delta}{\Gamma}{e'}{\tau'}$ then $\ruleType{\Delta}{\Gamma, \Ghyp{x}{\tau'}}{r}{\tau}{\tau''}$.}
  \end{enumerate}
\end{enumerate}\end{lemma}
\begin{proof-sketch} ~
\begin{enumerate}
\item By rule induction over Rules (\ref{rules:istypeU}) and case analysis over the definition of substitution. In all cases, the derivation of $\istypeU{\Delta}{[\tau'/t]\tau}$ does not depend on the form of $\tau'$.
%\item Context formation of $[\tau'/t]\Gamma$ does not depend on the structure of $\tau'$.
\item By \graytxtbox{mutual} rule induction over Rules (\ref{rules:hastypeUP}) \graytxtbox{and Rule (\ref{rule:ruleType})} and case analysis over the definition of substitution. In all cases, the derivation of $\hastypeU{\Delta}{[\tau'/t]\Gamma}{[\tau'/t]e}{[\tau'/t]\tau}$ \graytxtbox{or $\ruleType{\Delta}{[\tau'/t]\Gamma}{[\tau'/t]r}{[\tau'/t]\tau}{[\tau'/t]\tau''}$} does not depend on the form of $\tau'$.
\item By \graytxtbox{mutual} rule induction over Rules (\ref{rules:hastypeUP}) \graytxtbox{and Rule (\ref{rule:ruleType})} and case analysis over the definition of substitution. In all cases, the derivation of $\hastypeU{\Delta}{\Gamma}{[e'/x]e}{\tau}$ \graytxtbox{or $\ruleType{\Delta}{\Gamma}{[e'/x]r}{\tau}{\tau''}$} does not depend on the form of $e'$.
\end{enumerate}
\end{proof-sketch}

\begin{grayparbox}
% The Pattern Regularity Lemma establishes that the hypotheses generated by checking a pattern against a well-formed type involve only well-formed types.
\begin{lemma}[Pattern Regularity]\label{lemma:pattern-regularity-UP} 
If $\patType{\pctx}{p}{\tau}$ and $\istypeU{\Delta}{\tau}$ then $\isctxU{\Delta}{\pctx}$ and $\mathsf{patvars}({p}) = \domof{\pctx}$.
\end{lemma}
\begin{proof} By rule induction over Rules (\ref{rules:patType}).
\begin{byCases}
\item[\text{(\ref{rule:patType-var})}] ~
\begin{pfsteps*}
  \item $p=x$ \BY{assumption}
  \item $\pctx=x : \tau$ \BY{assumption}
  \item $\istypeU{\Delta}{\tau}$ \BY{assumption}\pflabel{istypeU}
  \item $\isctxU{\Delta}{\Ghyp{x}{\tau}}$ \BY{Definition \ref{def:isctxU} on \pfref{istypeU}}
  \item $\fvof{p} = \domof{\Gamma} = \{x\}$ \BY{definition}
 \end{pfsteps*}
 \resetpfcounter
\item[\text{(\ref{rule:patType-wild})}] ~
\begin{pfsteps}
\item p = \aewildp \BY{assumption}
\item \pctx=\emptyset \BY{assumption}
\item \isctxU{\Delta}{\emptyset} \BY{Definition \ref{def:isctxU}}
\item \mathsf{patvars}({p}) = \domof{\Gamma} = \emptyset \BY{definition}
\end{pfsteps}
\resetpfcounter

\item[\text{(\ref{rule:patType-tpl})}] ~
\begin{pfsteps*}
  \item $p=\aetplp{\labelset}{\mapschema{p}{i}{\labelset}}$ \BY{assumption}
  \item $\tau=\aprod{\labelset}{\mapschema{\tau}{i}{\labelset}}$ \BY{assumption}
  \item $\pctx=\cup_{i \in \labelset} \pctx_i$ \BY{assumption}
  \item $\{\patType{\pctx_i}{p_i}{\tau_i}\}_{i \in \labelset}$ \BY{assumption}\pflabel{patType}
  \item $\istypeU{\Delta}{\aprod{\labelset}{\mapschema{\tau}{i}{\labelset}}}$ \BY{assumption} \pflabel{istypeU}
  \item $\{\istypeU{\Delta}{\tau_i}\}_{i \in \labelset}$ \BY{Inversion of Rule (\ref{rule:istypeU-prod}) on \pfref{istypeU}}\pflabel{istypeU-each}
  \item $\{\isctxU{\Delta}{\pctx_i}\}_{i \in \labelset}$ \BY{IH over \pfref{patType} and \pfref{istypeU-each}} \pflabel{biggy}
  \item $\{\mathsf{patvars}({p_i}) = \domof{\pctx_i}\}_{i \in \labelset}$ \BY{IH over \pfref{patType} and \pfref{istypeU-each}} \pflabel{biggy2}
  \item $\isctxU{\Delta}{\cup_{i \in \labelset} \pctx_i}$ \BY{Definition \ref{def:isctxU} over \pfref{biggy}, then Definition \ref{def:isctxU} iteratively}
  \item $\mathsf{patvars}({p}) = \domof{\Gamma} = \emptyset$ \BY{definition and \pfref{biggy2}}
\end{pfsteps*}
\resetpfcounter

\item[\text{(\ref{rule:patType-inj})}] ~
\begin{pfsteps*}
  \item $p=\aeinjp{\ell}{p'}$ \BY{assumption}
  \item $\tau=\asum{\labelset, \ell}{\mapschema{\tau}{i}{\labelset}; \mapitem{\ell}{\tau'}}$ \BY{assumption}
  \item $\istypeU{\Delta}{\asum{\labelset, \ell}{\mapschema{\tau}{i}{\labelset}; \mapitem{\ell}{\tau'}}}$ \BY{assumption} \pflabel{istype}
  \item $\patType{\pctx}{p'}{\tau'}$ \BY{assumption} \pflabel{patType}
  \item $\istypeU{\Delta}{\tau'}$ \BY{Inversion of Rule (\ref{rule:istypeU-sum}) on \pfref{istype}} \pflabel{istypeTwo} 
  \item $\isctxU{\Delta}{\pctx}$ \BY{IH on \pfref{patType} and \pfref{istypeTwo}}
  \item $\mathsf{patvars}({p'}) = \domof{\pctx}$
  \BY{IH on \pfref{patType} and \pfref{istypeTwo}} \pflabel{fv1}
  \item $\mathsf{patvars}({p}) = \domof{\pctx}$ \BY{definition and \pfref{fv1}}
\end{pfsteps*}
\resetpfcounter
\end{byCases}
\end{proof}
\end{grayparbox}

% Finally, the Regularity Lemma establishes that the type assigned to an expression under a well-formed typing context is well-formed. 
% \begin{lemma}[Regularity]\label{lemma:regularity-UP} ~
% \begin{enumerate}
% \item If $\hastypeU{\Delta}{\Gamma}{e}{\tau}$ and $\isctxU{\Delta}{\Gamma}$ then $\istypeU{\Delta}{\tau}$.
% \item \graytxtbox{If $\ruleType{\Delta}{\Gamma}{r}{\tau}{\tau'}$ and $\isctxU{\Delta}{\Gamma}$ then $\istypeU{\Delta}{\tau'}$.}
% \end{enumerate}
% \end{lemma}
% \begin{proof-sketch} By \graytxtbox{mutual} rule induction over Rules (\ref{rules:hastypeUP}) \graytxtbox{and Rule (\ref{rule:ruleType})}, and Lemma \ref{lemma:substitution-UP} \graytxtbox{and Lemma \ref{lemma:pattern-regularity-UP}}.
% \end{proof-sketch}

\subsection{Structural Operational Semantics}\vspace{-4px}
The \emph{structural operational semantics} is specified as a transition system, and is organized around judgements of the following form:
\vspace{-4px}\[\begin{array}{ll}
\textbf{Judgement Form} & \textbf{Description}\\
\stepsU{e}{e'} & \text{$e$ transitions to $e'$}\\
\isvalU{e} & \text{$e$ is a value}\\
\LCC \color{light-gray} & \color{light-gray} \\
\matchfail{e} & \text{$e$ raises match failure} \ECC
\end{array}\]\vspace{-4px}
We also define auxiliary judgements for \emph{iterated transition}, $\multistepU{e}{e'}$, and \emph{evaluation}, $\evalU{e}{e'}$.


\begin{definition}[Iterated Transition]\label{defn:iterated-transition-UP} Iterated transition, $\multistepU{e}{e'}$, is the reflexive, transitive closure of the transition judgement, $\stepsU{e}{e'}$.\end{definition}


\begin{definition}[Evaluation]\label{defn:evaluation-UP}  $\evalU{e}{e'}$ iff $\multistepU{e}{e'}$ and $\isvalU{e'}$. \end{definition}

Our subsequent developments do not make mention of particular rules in the dynamic semantics, nor do they make mention of other judgements, not listed above,  that are used only for defining the dynamics of the match operator, so we do not produce these details here. Instead, it suffices to state the following conditions.

\begin{condition}[Canonical Forms]\label{condition:canonical-forms-UP} If $\hastypeUC{e}{\tau}$ and $\isvalU{e}$ then:
\begin{enumerate}
\item If $\tau=\aparr{\tau_1}{\tau_2}$ then $e=\aelam{\tau_1}{x}{e'}$ and $\hastypeUCO{\Ghyp{x}{\tau_1}}{e'}{\tau_2}$.
\item If $\tau=\aall{t}{\tau'}$ then $e=\aetlam{t}{e'}$ and $\hastypeUCO{\Dhyp{t}}{e'}{\tau'}$.
\item If $\tau=\arec{t}{\tau'}$ then $e=\aefold{e'}$ and $\hastypeUC{e'}{[\abop{rec}{t.\tau'}/t]\tau'}$ and $\isvalU{e'}$. 
\item If $\tau=\aprod{\labelset}{\mapschema{\tau}{i}{\labelset}}$ then $e=\aetpl{\labelset}{\mapschema{e}{i}{\labelset}}$ and $\hastypeUC{e_i}{\tau_i}$ and $\isvalU{e_i}$ for each $i \in \labelset$.
\item If $\tau=\asum{\labelset}{\mapschema{\tau}{i}{\labelset}}$ then for some label set $L'$ and label $\ell$ and type $\tau'$, we have that $\labelset=\labelset', \ell$ and $\tau=\asum{\labelset', \ell}{\mapschema{\tau}{i}{\labelset'}; \mapitem{\ell}{\tau'}}$ and $e=\aein{\ell}{e'}$ and $\hastypeUC{e'}{\tau'}$ and $\isvalU{e'}$.
\end{enumerate}\end{condition}


\begin{condition}[Preservation]\label{condition:preservation-UP} If $\hastypeUC{e}{\tau}$ and $\stepsU{e}{e'}$ then $\hastypeUC{e'}{\tau}$. \end{condition}

\begin{condition}[Progress]\label{condition:progress-UP} If $\hastypeUC{e}{\tau}$ then either $\isvalU{e}$ \graytxtbox{or $\matchfail{e}$} or there exists an $e'$ such that $\stepsU{e}{e'}$. \end{condition}

\section{Unexpanded Language (UL)}\label{appendix:SES-uexps}
\subsection{Syntax}\label{appendix:SES-syntax}\label{appendix:SES-shared-forms}
\subsubsection{Stylized Syntax}
\[\begin{array}{lllllll}
\textbf{Sort} & &  
%&\textbf{Operational Form} 
& \textbf{Stylized Form} & \textbf{Description}\\
\mathsf{UTyp} & \utau & ::= 
% &\ut 
& \ut & \text{identifier}\\
&& 
%& \auparr{\utau}{\utau} 
& \parr{\utau}{\utau} & \text{partial function}\\
&&
%& \auall{\ut}{\utau} 
& \forallt{\ut}{\utau} & \text{polymorphic}\\
&&
%& \aurec{\ut}{\utau} 
& \rect{\ut}{\utau} & \text{recursive}\\
&&
%& \auprod{\labelset}{\mapschema{\utau}{i}{\labelset}} 
& \prodt{\mapschema{\utau}{i}{\labelset}} & \text{labeled product}\\
&&
%& \ausum{\labelset}{\mapschema{\utau}{i}{\labelset}} 
& \sumt{\mapschema{\utau}{i}{\labelset}} & \text{labeled sum}\\
\mathsf{UExp} & \ue & ::= 
%& \ux 
& \ux & \text{identifier}\\
&&
%
& \asc{\ue}{\utau} & \text{ascription}\\
&&
%
& \letsyn{\ux}{\ue}{\ue} & \text{value binding}\\
&&
%& \aulam{\utau}{\ux}{\ue} 
& \lam{\ux}{\utau}{\ue} & \text{abstraction}\\
&&
%& \auap{\ue}{\ue} 
& \ap{\ue}{\ue} & \text{application}\\
&&
%& \autlam{\ut}{\ue} 
& \Lam{\ut}{\ue} & \text{type abstraction}\\
&&
%& \autap{\ue}{\utau} 
& \App{\ue}{\utau} & \text{type application}\\
&&
%& \aufold{\ut}{\utau}{\ue} 
& \fold{\ue} & \text{fold}\\
&&
%& \auunfold{\ue} 
& \unfold{\ue} & \text{unfold}\\
&&
%& \autpl{\labelset}{\mapschema{\ue}{i}{\labelset}} 
& \tpl{\mapschema{\ue}{i}{\labelset}} & \text{labeled tuple}\\
&&
%& \aupr{\ell}{\ue} 
& \prj{\ue}{\ell} & \text{projection}\\
&&
%& \auin{\labelset}{\ell}{\mapschema{\utau}{i}{\labelset}}{\ue} 
& \inj{\ell}{\ue} & \text{injection}\\
&&
%& \aucase{\labelset}{\utau}{\ue}{\mapschemab{\ux}{\ue}{i}{\labelset}} 
& \caseof{\ue}{\mapschemab{\ux}{\ue}{i}{\labelset}} & \text{case analysis}\\
&&
%& \audefuetsm{\utau}{e}{\tsmv}{\ue} 
& \uesyntax{\tsmv}{\utau\\&&&}{e}{\ue} & \text{seTLM definition}\\ 
&&
%& \autsmap{b}{\tsmv} 
& \utsmap{\tsmv}{b} & \text{seTLM application}\\%\ECC
\LCC  \color{light-gray} & \color{light-gray} & \color{light-gray}
& \color{light-gray} 
& \color{light-gray} & \color{light-gray} \\
&&
%& \aumatchwith{n}{\utau}{\ue}{\seqschemaX{\urv}} 
& \matchwith{\ue}{\seqschemaX{\urv}} & \text{match}\\
&&
%& \audefuptsm{\utau}{e}{\tsmv}{\ue} 
& \usyntaxup{\tsmv}{\utau}{e}{\ue}
& \text{spTLM definition}\\
\mathsf{URule} & \urv & ::= 
%& \aumatchrule{\upv}{\ue} 
& \matchrule{\upv}{\ue} & \text{match rule}\\
\mathsf{UPat} & \upv & ::= 
%& \ux 
& \ux & \text{identifier pattern}\\
&&
%& \auwildp 
& \wildp & \text{wildcard pattern}\\
&&
%& \aufoldp{\upv} 
& \foldp{\upv} & \text{fold pattern}\\
&&
%& \autplp{\labelset}{\mapschema{\upv}{i}{\labelset}} 
& \tplp{\mapschema{\upv}{i}{\labelset}} & \text{labeled tuple pattern}\\
&&
%& \auinjp{\ell}{\upv} 
& \injp{\ell}{\upv} & \text{injection pattern}\\
% \LCC &&& \color{light-gray} & \color{light-gray} & \color{light-gray}\\
&&
%& \auapuptsm{b}{\tsmv} 
& \utsmap{\tsmv}{b} & \text{spTLM application}\ECC
\end{array}\]

\clearpage

\paragraph{Body Lengths}\label{appendix:SES-body-lengths}
We write $\sizeof{b}$ for the length of $b$. The metafunction $\sizeof{\ue}$ computes the sum of the lengths of expression literal bodies in $\ue$:
\[
\begin{array}{ll}
\sizeof{\ux} & = 0\\
\sizeof{\asc{\ue}{\utau}} & = \sizeof{\ue}\\
\sizeof{\letsyn{\ux}{\ue_1}{\ue_2}} & = \sizeof{\ue_1} + \sizeof{\ue_2}\\
\sizeof{\lam{\ux}{\utau}{\ue}} &= \sizeof{\ue}\\
\sizeof{\ap{\ue_1}{\ue_2}} & = \sizeof{\ue_1} + \sizeof{\ue_2}\\
\sizeof{\Lam{\ut}{\ue}} & = \sizeof{\ue}\\
\sizeof{\App{\ue}{\utau}} & = \sizeof{\ue}\\
\sizeof{\fold{\ue}} & = \sizeof{\ue}\\
\sizeof{\unfold{\ue}} & = \sizeof{\ue}\\
%\end{align*}
%\begin{align*}
\sizeof{\tpl{\mapschema{\ue}{i}{\labelset}}} & = \sum_{i \in \labelset} \sizeof{\ue_i}\\
\sizeof{\prj{\ell}{\ue}} & = \sizeof{\ue}\\
\sizeof{\inj{\ell}{\ue}} & = \sizeof{\ue}\\
\sizeof{\caseof{\ue}{\mapschemab{\ux}{\ue}{i}{\labelset}}} & = \sizeof{\ue} + \sum_{i \in \labelset} \sizeof{\ue_i}\\
\sizeof{\uesyntax{\tsmv}{\utau}{e}{\ue'}} & = \sizeof{\ue} + \sizeof{\ue'}\\
\sizeof{\utsmap{\tsmv}{b}} & = \sizeof{b}\\
\LCC \color{light-gray} & \color{light-gray}\\
\sizeof{\matchwith{\ue}{\seqschemaX{\urv}}} & = \sizeof{\ue} + \sum_{1 \leq i \leq n} \sizeof{r_i}\\
\sizeof{\usyntaxup{\tsmv}{\utau}{e}{\ue}} & = \sizeof{\ue}\ECC
\end{array}
\]
\vspace{-3px}\begin{grayparbox}\vspace{3px}and $\sizeof{\urv}$ computes the sum of the lengths of expression literal bodies in $\urv$:
\begin{align*}
\sizeof{\matchrule{\upv}{\ue}} & = \sizeof{\ue}
\end{align*}
Similarly, the metafunction $\sizeof{\upv}$ computes the sum of the lengths of the pattern literal bodies in $\upv$:
\begin{align*}
\sizeof{\ux} & = 0\\
\sizeof{\foldp{\upv}} & = \sizeof{\upv}\\
\sizeof{\tplp{\mapschema{\upv}{i}{\labelset}}} & = \sum_{i \in \labelset} \sizeof{\upv_i}\\
\sizeof{\injp{\ell}{\upv}} & = \sizeof{\upv}\\
\sizeof{\utsmap{\tsmv}{b}} & = \sizeof{b}
\end{align*}
\end{grayparbox}

\paragraph{Common Unexpanded Forms} Each expanded form maps onto an unexpanded form. We refer to these as the \emph{common forms}. In particular:
\begin{itemize}
\item Each type variable, $t$, maps onto a unique {type identifier}, written $\sigilof{t}$.
\item Each type, $\tau$, maps onto an unexpanded type, $\Uof{\tau}$, as follows: 
  \begin{align*}
  \Uof{t} &= \sigilof{t}\\
  \Uof{\aparr{\tau_1}{\tau_2}} & = \parr{\Uof{\tau_1}}{\Uof{\tau_2}}\\
  \Uof{\aall{t}{\tau}} & = \forallt{\sigilof{t}}{\Uof{\tau}}\\
  \Uof{\arec{t}{\tau}} & = \rect{\sigilof{t}}{\Uof{\tau}}\\
  \Uof{\aprod{\labelset}{\mapschema{\tau}{i}{\labelset}}} & = \prodt{\mapschemax{\Uofv}{\tau}{i}{\labelset}}\\
  \Uof{\asum{\labelset}{\mapschema{\tau}{i}{\labelset}}} & = \sumt{\mapschemax{\Uofv}{\tau}{i}{\labelset}}
  \end{align*}
\item Each expression variable, $x$, maps onto a unique expression identifier, written $\sigilof{x}$.
\item Each core language expression, $e$, maps onto an unexpanded expression, $\Uof{e}$, as follows:
\[\arraycolsep=1pt\begin{array}{rl}
\Uof{x} & = \sigilof{x}\\
\Uof{\aelam{\tau}{x}{e}} & = \lam{\sigilof{x}}{\Uof{\tau}}{\Uof{e}}\\
\Uof{\aeap{e_1}{e_2}} & = \ap{\Uof{e_1}}{\Uof{e_2}}\\
\Uof{\aetlam{t}{e}} & = \Lam{\sigilof{t}}{\Uof{e}}\\
\Uof{\aetap{e}{\tau}} & = \App{\Uof{e}}{\Uof{\tau}}\\
\Uof{\aefold{e}} & = \fold{\Uof e}\\
\Uof{\aeunfold{e}} & = \unfold{\Uof{e}}\\
\Uof{\aetpl{\labelset}{\mapschema{e}{i}{\labelset}}} & = \tpl{\mapschemax{\Uofv}{e}{i}{\labelset}}\\
\Uof{\aepr{\ell}{e}} & = \prj{\Uof{e}}{\ell}\\
\Uof{\aein{\ell}{e}} &= \inj{\ell}{\Uof{e}}\\
\LCC \color{light-gray} & \color{light-gray} \\
\Uof{\aematchwith{n}{e}{\seqschemaX{r}}} & = \matchwith{\Uof{e}}{\seqschemaXx{\Uofv}{r}}\ECC
\end{array}\]
\end{itemize}
\begin{grayparbox}
\begin{itemize}
\item Each core language rule, $r$, maps onto an unexpanded rule, $\Uof{r}$, as follows:
\[\arraycolsep=1pt\begin{array}{rl}
\LCC \color{light-gray} & \color{light-gray} \\
\Uof{\aematchrule{p}{e}} & = \aumatchrule{\Uof{p}}{\Uof{e}}\ECC
\end{array}\]
\item Each core language pattern, $p$, maps onto the unexpanded pattern, $\Uof{p}$, as follows:
\[\arraycolsep=1pt\begin{array}{rl}
\LCC \color{light-gray} & \color{light-gray} \\
\Uof{x} & = \sigilof{x}\\
\Uof{\aewildp} &= \auwildp\\
\Uof{\aefoldp{p}} &= \aufoldp{\Uof{p}}\\
\Uof{\aetplp{\labelset}{\mapschema{p}{i}{\labelset}}} & = \autplp{\labelset}{\mapschemax{\Uofv}{p}{i}{\labelset}}\\
\Uof{\aeinjp{\ell}{p}} & = \auinjp{\ell}{\Uof{p}}\ECC
\end{array}\]
\end{itemize}
\end{grayparbox}
\vspace{-10px}
\subsubsection{Textual Syntax}\vspace{-3px} In addition to the stylized syntax, there is also a context-free textual syntax for the UL. For our purposes, we need only posit the existence of partial metafunctions $\parseUTypF{b}$ and $\parseUExpF{b}$\graytxtbox{~and $\parseUPatF{b}$}. 

\begin{condition}[Textual Representability]\label{condition:textual-representability-SES} ~
\begin{enumerate}
\item For each $\utau$, there exists $b$ such that $\parseUTyp{b}{\utau}$. 
\item For each $\ue$, there exists $b$ such that $\parseUExp{b}{\ue}$.
% \item For each $\urv$, there exists $b$ such that $\parseURule{b}{\urv}$.
\item \graytxtbox{For each $\upv$, there exists $b$ such that $\parseUPat{b}{\upv}$.}
\end{enumerate}
\end{condition}

We also impose the following technical condition\graytxtbox{s}.

\begin{condition}[Expression Parsing Monotonicity]\label{condition:body-parsing} If $\parseUExp{b}{\ue}$ then $\sizeof{\ue} < \sizeof{b}$.\end{condition}

\begin{grayparbox}\begin{condition}[Pattern Parsing Monotonicity]\label{condition:pattern-parsing} If $\parseUPat{b}{\upv}$ then $\sizeof{\upv} < \sizeof{b}$.\end{condition}\end{grayparbox}

\subsection{Type Expansion}
\emph{Unexpanded type formation contexts}, $\uDelta$, are of the form $\uDD{\uD}{\Delta}$, i.e. they consist of a \emph{type identifier expansion context}, $\uD$, paired with a type formation context, $\Delta$. 

A \emph{type identifier expansion context}, $\uD$, is a finite function that maps each type identifier $\ut \in \domof{\uD}$ to the hypothesis $\vExpands{\ut}{t}$, for some type variable $t$. We write $\ctxUpdate{\uD}{\ut}{t}$ for the type identifier expansion context that maps $\ut$ to $\vExpands{\ut}{t}$ and defers to $\uD$ for all other type identifiers (i.e. the previous mapping is \emph{updated}.) 

We define $\uDelta, \uDhyp{\ut}{t}$ when $\uDelta=\uDD{\uD}{\Delta}$ as an abbreviation of  \[\uDD{\ctxUpdate{\uD}{\ut}{t}}{\Delta, \Dhyp{t}}\]%type identifier expansion context is always extended/updated together with 

\begin{definition}[Unexpanded Type Formation Context Formation] $\uDOK{\uDD{\uD}{\Delta}}$ iff for each $\uDhyp{\ut}{t} \in \uD$ we have $\Dhyp{t} \in \Delta$. \end{definition}

\vspace{10px}\noindent\fbox{\strut$\expandsTU{\uDelta}{\utau}{\tau}$}~~$\utau$ has well-formed expansion $\tau$
\begin{subequations}\label{rules:expandsTU}
\begin{equation}\label{rule:expandsTU-var}
\inferrule{ }{\expandsTU{\uDelta, \uDhyp{\ut}{t}}{\ut}{t}}
\end{equation}
\begin{equation}\label{rule:expandsTU-parr}
\inferrule{
  \expandsTU{\uDelta}{\utau_1}{\tau_1}\\
  \expandsTU{\uDelta}{\utau_2}{\tau_2}
}{\expandsTU{\uDelta}{\auparr{\utau_1}{\utau_2}}{\aparr{\tau_1}{\tau_2}}}
\end{equation}
\begin{equation}\label{rule:expandsTU-all}
  \inferrule{
    \expandsTU{\uDelta, \uDhyp{\ut}{t}}{\utau}{\tau}
  }{
    \expandsTU{\uDelta}{\auall{\ut}{\utau}}{\aall{t}{\tau}}
  }
\end{equation}
\begin{equation}\label{rule:expandsTU-rec}
  \inferrule{
    \expandsTU{\uDelta, \uDhyp{\ut}{t}}{\utau}{\tau}
  }{
    \expandsTU{\uDelta}{\aurec{\ut}{\utau}}{\arec{t}{\tau}}
  }
\end{equation}
\begin{equation}\label{rule:expandsTU-prod}
  \inferrule{
    \{\expandsTU{\uDelta}{\utau_i}{\tau_i}\}_{i \in \labelset}
  }{
    \expandsTU{\uDelta}{\auprod{\labelset}{\mapschema{\utau}{i}{\labelset}}}{\aprod{\labelset}{\mapschema{\tau}{i}{\labelset}}}
  }
\end{equation}
\begin{equation}\label{rule:expandsTU-sum}
  \inferrule{
    \{\expandsTU{\uDelta}{\utau_i}{\tau_i}\}_{i \in \labelset}
  }{
    \expandsTU{\uDelta}{\ausum{\labelset}{\mapschema{\utau}{i}{\labelset}}}{\asum{\labelset}{\mapschema{\tau}{i}{\labelset}}}
  }
\end{equation}
\end{subequations}
% \emph{Unexpanded type formation contexts}, $\uDelta$, are of the form $\uDD{\uD}{\Delta}$, where $\uD$ is a \emph{type identifier expansion context}, and $\Delta$ is a type formation context. A type identifier expansion context, $\uD$, is a finite function that maps each type identifier $\ut \in \domof{\uD}$ to the hypothesis $\vExpands{\ut}{t}$, for some type variable $t$. We write $\ctxUpdate{\uD}{\ut}{t}$ for the type identifier expansion context that maps $\ut$ to $\vExpands{\ut}{t}$ and defers to $\uD$ for all other type identifiers (i.e. the previous mapping, if it exists, is updated). 
% We define $\uDelta, \uDhyp{\ut}{t}$ when $\uDelta=\uDD{\uD}{\Delta}$ as an abbreviation of  \[\uDD{\ctxUpdate{\uD}{\ut}{t}}{\Delta, \Dhyp{t}}\]%type identifier expansion context is always extended/updated together with 
% %We write $\uDeltaOK{\uDelta}$ when $\uDelta=\uDD{\uD}{\Delta}$ and each type variable in $\uD$ also appears in $\Delta$.
% %\begin{definition}\label{def:uDeltaOK} $\uDeltaOK{\uDD{\uD}{\Delta}}$ iff for each $\vExpands{\ut}{t} \in \uD$, we have $\Dhyp{t} \in \Delta$.\end{definition}

\subsection{Typed Expression Expansion}\label{appendix:typed-expression-expansion-S}
\subsubsection{Contexts}
\emph{Unexpanded typing contexts}, $\uGamma$, are, similarly, of the form $\uGG{\uG}{\Gamma}$, where $\uG$ is an \emph{expression identifier expansion context}, and $\Gamma$ is a typing context. An expression identifier expansion context, $\uG$, is a finite function that maps each expression identifier $\ux \in \domof{\uG}$ to the hypothesis $\vExpands{\ux}{x}$, for some expression variable, $x$. We write $\ctxUpdate{\uG}{\ux}{x}$ for the expression identifier expansion context that maps $\ux$ to $\vExpands{\ux}{x}$ and defers to $\uG$ for all other expression identifiers (i.e. the previous mapping is updated.) 
%We write $\uGammaOK{\uGamma}$ when $\uGamma=\uGG{\uG}{\Gamma}$ and each expression variable in $\uG$ is assigned a type by $\Gamma$.
%\noindent 

We define $\uGamma, \uGhyp{\ux}{x}{\tau}$ when $\uGamma = \uGG{\uG}{\Gamma}$ as an abbreviation of \[\uGG{\ctxUpdate{\uG}{\ux}{x}}{\Gamma, \Ghyp{x}{\tau}}\]

\begin{definition}[Unexpanded Typing Context Formation] $\uGammaOK{\uGG{\uG}{\Gamma}}$ iff $\isctxU{\Delta}{\Gamma}$ and for each $\vExpands{\ux}{x} \in \uG$, we have $x \in \domof{\Gamma}$.\end{definition}


\subsubsection{Body Encoding and Decoding}
An assumed type abbreviated $\tBody$ classifies encodings of literal bodies, $b$. The mapping from literal bodies to values of type $\tBody$ is defined by the \emph{body encoding judgement} $\encodeBody{b}{\ebody}$. An inverse mapping is defined   by the \emph{body decoding judgement} $\decodeBody{\ebody}{b}$.
\[\begin{array}{ll}
\textbf{Judgement Form} & \textbf{Description}\\
\encodeBody{b}{e} & \text{$b$ has encoding $e$}\\
\decodeBody{e}{b} & \text{$e$ has decoding $b$}
\end{array}\]
The following condition establishes an isomorphism between literal bodies and values of type $\tBody$ mediated by the judgements above.
\begin{condition}[Body Isomorphism]\label{condition:body-isomorphism} ~
\begin{enumerate}
\item For every literal body $b$, we have that $\encodeBody{b}{\ebody}$ for some $\ebody$ such that $\hastypeUC{\ebody}{\tBody}$ and $\isvalU{\ebody}$.
\item If $\hastypeUC{\ebody}{\tBody}$ and $\isvalU{\ebody}$ then $\decodeBody{\ebody}{b}$ for some $b$.
\item If $\encodeBody{b}{\ebody}$ then $\decodeBody{\ebody}{b}$.
\item If $\hastypeUC{\ebody}{\tBody}$ and $\isvalU{\ebody}$ and $\decodeBody{\ebody}{b}$ then $\encodeBody{b}{\ebody}$. 
\item If $\encodeBody{b}{\ebody}$ and $\encodeBody{b}{\ebody'}$ then $\ebody = \ebody'$.
\item If $\hastypeUC{\ebody}{\tBody}$ and $\isvalU{\ebody}$ and $\decodeBody{\ebody}{b}$ and $\decodeBody{\ebody}{b'}$ then $b=b'$.
\end{enumerate}
\end{condition}
We also assume a partial metafunction, $\bsubseq{b}{m}{n}$, which extracts a subsequence of $b$ starting at position $m$ and ending at position $n$, inclusive, where $m$ and $n$ are natural numbers. The following condition is technically necessary.
\begin{condition}[Body Subsequencing]\label{condition:body-subsequences} If $\bsubseq{b}{m}{n}=b'$ then $\sizeof{b'} \leq \sizeof{b}$. \end{condition}

\subsubsection{Parse Results}
 The type abbreviated $\tParseResultExp$, and an auxiliary abbreviation used below, is defined as follows:
\begin{align*}
L_\mathtt{SE} & \defeq \lbltxt{Error}, \lbltxt{SuccessE}\\
\tParseResultExp & \defeq \asum{L_\mathtt{SE}}{
  \mapitem{\lbltxt{Error}}{\prodt{}}, 
  \mapitem{\lbltxt{SuccessE}}{\tCEExp}
}\\
\end{align*} %[\mapitem{\lbltxt{Error}}{\prodt{}}, \mapitem{\lbltxt{SuccessE}}{\tCEExp}]

\begin{grayparbox}
 The type abbreviated $\tParseResultPat$, and an auxiliary abbreviation used below, is defined as follows:
\begin{align*}
L_\mathtt{SP} & \defeq \lbltxt{Error}, \lbltxt{SuccessP}\\
\tParseResultExp & \defeq \asum{L_\mathtt{SP}}{
  \mapitem{\lbltxt{Error}}{\prodt{}}, 
  \mapitem{\lbltxt{SuccessP}}{\tCEPat}
}\\
\end{align*} %[\mapitem{\lbltxt{Error}}{\prodt{}}, \mapitem{\lbltxt{SuccessE}}{\tCEExp}]
\end{grayparbox}

\subsubsection{seTLM Contexts}

\emph{seTLM contexts}, $\uPsi$, are of the form $\uAS{\uA}{\Psi}$, where $\uA$ is a \emph{TLM identifier expansion context} and $\Psi$ is a \emph{seTLM definition context}. 

A \emph{TLM identifier expansion context}, $\uA$, is a finite function mapping each TLM identifier $\tsmv \in \domof{\uA}$ to the \emph{TLM identifier expansion}, $\vExpands{\tsmv}{x}$, for some variable $x$. We write $\ctxUpdate{\uA}{\tsmv}{x}$ for the TLM identifier expansion context that maps $\tsmv$ to $\vExpands{\tsmv}{x}$, and defers to $\uA$ for all other TLM identifiers (i.e. the previous mapping is \emph{updated}.)

An \emph{seTLM definition context}, $\Psi$, is a finite function mapping each variable $x \in \domof{\Psi}$ to an \emph{expanded seTLM definition}, $\xuetsmbnd{x}{\tau}{\eparse}$, where $\tau$ is the seTLM's type annotation, and $\eparse$ is its parse function. We write $\Psi, \xuetsmbnd{x}{\tau}{\eparse}$ when $x \notin \domof{\Psi}$ for the extension of $\Psi$ that maps $x$ to $\xuetsmbnd{x}{\tau}{\eparse}$. We write $\uetsmenv{\Delta}{\Psi}$  when all the type annotations in $\Psi$ are well-formed assuming $\Delta$, and the parse functions in $\Psi$ are closed and of the appropriate type.

\begin{definition}[seTLM Definition Context Formation]\label{def:seTLM-def-ctx-formation} $\uetsmenv{\Delta}{\Psi}$ iff for each $\xuetsmbnd{x}{\tau}{\eparse} \in \Psi$, we have $\istypeU{\Delta}{\tau}$ and $\hastypeU{\emptyset}{\emptyset}{\eparse}{\aparr{\tBody}{\tParseResultExp}}$.\end{definition}

\begin{definition}[seTLM Context Formation] $\uetsmctx{\Delta}{\uAS{\uA}{\Psi}}$ iff $\uetsmenv{\Delta}{\Psi}$ and for each $\vExpands{\tsmv}{x} \in \uA$ we have $x \in \domof{\Psi}$.
\end{definition}

We define $\uPsi, \uShyp{\tsmv}{x}{\tau}{\eparse}$, when $\uPsi=\uAS{\uA}{\Phi}$, as an abbreviation of \[\uAS{\ctxUpdate{\uA}{\tsmv}{x}}{\Psi, \xuetsmbnd{x}{\tau}{\eparse}}\]
%\vspace{10px}

\begin{grayparbox}\vspace{-15px}\subsubsection{spTLM Contexts}
\emph{spTLM contexts}, $\uPhi$, are of the form $\uAS{\uA}{\Phi}$, where $\uA$ is a {TLM identifier expansion context}, defined above, and $\Phi$ is a \emph{spTLM definition context}. 

An \emph{spTLM definition context}, $\Phi$, is a finite function mapping each variable $x \in \domof{\Phi}$ to an \emph{expanded seTLM definition}, $\xuptsmbnd{a}{\tau}{\eparse}$, where $\tau$ is the spTLM's type annotation, and $\eparse$ is its parse function. We write $\Phi, \xuptsmbnd{a}{\tau}{\eparse}$ when $a \notin \domof{\Phi}$ for the extension of $\Phi$ that maps $x$ to $\xuptsmbnd{a}{\tau}{\eparse}$. We write $\uptsmenv{\Delta}{\Phi}$  when all the type annotations in $\Phi$ are well-formed assuming $\Delta$, and the parse functions in $\Phi$ are closed and of the appropriate type.

\begin{definition}[spTLM Definition Context Formation]\label{def:spTLM-def-ctx-formation} $\uptsmenv{\Delta}{\Phi}$ iff for each $\xuptsmbnd{a}{\tau}{\eparse} \in \Phi$, we have $\istypeU{\Delta}{\tau}$ and $\hastypeU{\emptyset}{\emptyset}{\eparse}{\aparr{\tBody}{\tParseResultPat}}$.\end{definition}

\begin{definition}[spTLM Context Formation] $\uptsmctx{\Delta}{\uAS{\uA}{\Phi}}$ iff $\uptsmenv{\Delta}{\Phi}$ and for each $\vExpands{\tsmv}{x} \in \uA$ we have $x \in \domof{\Phi}$.
\end{definition}

We define $\uPhi, \uPhyp{\tsmv}{x}{\tau}{\eparse}$, when $\uPhi=\uAS{\uA}{\Phi}$, as an abbreviation of \[\uAS{\ctxUpdate{\uA}{\tsmv}{x}}{\Phi, \xuptsmbnd{a}{\tau}{\eparse}}\]
\end{grayparbox}

\subsubsection{Typed Expression Expansion}\label{appendix:typed-expression-expansion-SES}
\vspace{8px}\noindent\fbox{\strut$\expandsSG{\uDelta}{\uGamma}{\uPsi}{\uPhi}{\ue}{e}{\tau}$}~~$\ue$ has expansion $e$ of type $\tau$
\begin{subequations}\label{rules:expandsU}
\begin{equation}\label{rule:expandsU-var}
  \inferrule{ }{
    \expandsSG{\uDelta}{\uGamma, \uGhyp{\ux}{x}{\tau}}{\uPsi}{\uPhi}{\ux}{x}{\tau}
  }
\end{equation}
\begin{equation}\label{rule:expandsU-asc}
  \inferrule{
    \expandsTU{\uDelta}{\utau}{\tau}\\
    \expandsSG{\uDelta}{\uGamma}{\uPsi}{\uPhi}{\ue}{e}{\tau}
  }{
    \expandsSG{\uDelta}{\uGamma}{\uPsi}{\uPhi}{\asc{\ue}{\utau}}{e}{\tau}
  }
\end{equation}
\begin{equation}\label{rule:expandsU-letsyn}
  \inferrule{
    \expandsSG{\uDelta}{\uGamma}{\uPsi}{\uPhi}{\ue_1}{e_1}{\tau_1}\\
    \expandsSG{\uDelta}{\uGamma, \uGhyp{\ux}{x}{\tau_1}}{\uPsi}{\uPhi}{\ue_2}{e_2}{\tau_2}
  }{
    \expandsSG{\uDelta}{\uGamma}{\uPsi}{\uPhi}{\letsyn{\ux}{\ue_1}{\ue_2}}{
      \aeap{\aelam{\tau_1}{x}{e_2}}{e_1}
    }{\tau_2}
  }
\end{equation}
\begin{equation}\label{rule:expandsU-lam}
  \inferrule{
    \expandsTU{\uDelta}{\utau}{\tau}\\
    \expandsSG{\uDelta}{\uGamma, \uGhyp{\ux}{x}{\tau}}{\uPsi}{\uPhi}{\ue}{e}{\tau'}
  }{
    \expandsSG{\uDelta}{\uGamma}{\uPsi}{\uPhi}{\lam{\ux}{\utau}{\ue}}{\aelam{\tau}{x}{e}}{\aparr{\tau}{\tau'}}
  }
\end{equation}
\begin{equation}\label{rule:expandsU-ap}
  \inferrule{
    \expandsSG{\uDelta}{\uGamma}{\uPsi}{\uPhi}{\ue_1}{e_1}{\aparr{\tau}{\tau'}}\\
    \expandsSG{\uDelta}{\uGamma}{\uPsi}{\uPhi}{\ue_2}{e_2}{\tau}
  }{
    \expandsSG{\uDelta}{\uGamma}{\uPsi}{\uPhi}{\ap{\ue_1}{\ue_2}}{\aeap{e_1}{e_2}}{\tau'}
  }
\end{equation}
\begin{equation}\label{rule:expandsU-tlam}
  \inferrule{
    \expandsSG{\uDelta, \uDhyp{\ut}{t}}{\uGamma}{\uPsi}{\uPhi}{\ue}{e}{\tau}
  }{
    \expandsSG{\uDelta}{\uGamma}{\uPsi}{\uPhi}{\Lam{\ut}{\ue}}{\aetlam{t}{e}}{\aall{t}{\tau}}
  }
\end{equation}
\begin{equation}\label{rule:expandsU-tap}
  \inferrule{
    \expandsSG{\uDelta}{\uGamma}{\uPsi}{\uPhi}{\ue}{e}{\aall{t}{\tau}}\\
    \expandsTU{\uDelta}{\utau'}{\tau'}
  }{
    \expandsSG{\uDelta}{\uGamma}{\uPsi}{\uPhi}{\App{\ue}{\utau'}}{\aetap{e}{\tau'}}{[\tau'/t]\tau}
  }
\end{equation}
\begin{equation}\label{rule:expandsU-fold}
  \inferrule{
    % \istypeU{\Delta, \Dhyp{t}}{\tau}\\
    \expandsSG{\uDelta}{\uGamma}{\uPsi}{\uPhi}{\ue}{e}{[\arec{t}{\tau}/t]\tau}
  }{
    \expandsSG{\uDelta}{\uGamma}{\uPsi}{\uPhi}{\fold{\ue}}{\aefold{e}}{\arec{t}{\tau}}
  }
\end{equation}
\begin{equation}\label{rule:expandsU-unfold}
  \inferrule{
    \expandsSG{\uDelta}{\uGamma}{\uPsi}{\uPhi}{\ue}{e}{\arec{t}{\tau}}
  }{
    \expandsSG{\uDelta}{\uGamma}{\uPsi}{\uPhi}{\unfold{\ue}}{\aeunfold{e}}{[\arec{t}{\tau}/t]\tau}
  }
\end{equation}
\begin{equation}\label{rule:expandsU-tpl}
  \inferrule{
    \{\expandsSG{\uDelta}{\uGamma}{\uPsi}{\uPhi}{\ue_i}{e_i}{\tau_i}\}_{i \in \labelset}
  }{
    \expandsSG{\uDelta}{\uGamma}{\uPsi}{\uPhi}{\tpl{\mapschema{\ue}{i}{\labelset}}}{\aetpl{\labelset}{\mapschema{e}{i}{\labelset}}}{\aprod{\labelset}{\mapschema{\tau}{i}{\labelset}}}
  }
\end{equation}
\begin{equation}\label{rule:expandsU-pr}
  \inferrule{
    \expandsSG{\uDelta}{\uGamma}{\uPsi}{\uPhi}{\ue}{e}{\aprod{\labelset, \ell}{\mapschema{\tau}{i}{\labelset}; \ell \hookrightarrow \tau}}
  }{
    \expandsSG{\uDelta}{\uGamma}{\uPsi}{\uPhi}{\prj{\ue}{\ell}}{\aepr{\ell}{e}}{\tau}
  }
\end{equation}
\begin{equation}\label{rule:expandsU-in}
  \inferrule{
   % \{\istypeU{\Delta}{\tau_i}\}_{i \in \labelset}\\
    % \istypeU{\Delta}{\tau'}\\
    \expandsSG{\uDelta}{\uGamma}{\uPsi}{\uPhi}{\ue}{e}{\tau'}
  }{
    % \left(\shortstack{
    %   $\uDelta~\uGamma~{\vdash_{\uPhi}}{\setlength{\fboxsep}{0px}\colorbox{light-gray}{$_{\mathstrut; \uPsi}$}}~ \inj{\ell}{\ue}$\\
    %   $\leadsto$\\
    %   $\aein{\ell}{e} : \asum{\labelset, \ell}{\mapschema{\tau}{i}{\labelset}; \ell \hookrightarrow \tau}$\vspace{-1.2em}}\right)
    \expandsSG{\uDelta}{\uGamma}{\uPsi}{\uPhi}{\inj{\ell}{\ue}}{\aein{\ell}{e}}{\asum{\labelset, \ell}{\mapschema{\tau}{i}{\labelset}; \ell \hookrightarrow \tau'}}
  }
\end{equation}
\begin{equation}\label{rule:expandsU-case}
  \inferrule{
    \expandsSG{\uDelta}{\uGamma}{\uPsi}{\uPhi}{\ue}{e}{\asum{\labelset}{\mapschema{\tau}{i}{\labelset}}}\\
    % \istypeU{\Delta}{\tau}\\
    \{\expandsSG{\uDelta}{\uGamma, \uGhyp{\ux_i}{x_i}{\tau_i}}{\uPsi}{\uPhi}{\ue_i}{e_i}{\tau}\}_{i \in \labelset}
  }{
    \expandsSG{\uDelta}{\uGamma}{\uPsi}{\uPhi}{\caseof{\ue}{\mapschemab{\ux}{\ue}{i}{\labelset}}}{\aecase{\labelset}{e}{\mapschemab{x}{e}{i}{\labelset}}}{\tau}
  }
\end{equation}
\begin{equation}\label{rule:expandsU-syntax}
\inferrule{
  \expandsTU{\uDelta}{\utau}{\tau}\\\\
  \hastypeU{\emptyset}{\emptyset}{\eparse}{\aparr{\tBody}{\tParseResultExp}}\\
  \expandsSG{\uDelta}{\uGamma}{\uPsi}{\uPhi}{\uedep}{\edep}{\taudep}\\\\
  \uGamma = \uGG{\uG}{\Gamma}\\
  \expandsSG{\uDelta}{\uGG{\uG}{\Gamma, x : \taudep}}{\uPsi, \uShyp{\tsmv}{x}{\tau}{\eparse}}{\uPhi}{\ue}{e}{\tau'}\\\\
  e_\text{defn} = \aeap{\aelam{\taudep}{x}{e}}{\edep}
}{
  \expandsSG{\uDelta}{\uGamma}{\uPsi}{\uPhi}{\uesyntax{\tsmv}{\utau}{\eparse}{\ue}}{e_\text{defn}}{\tau'}
}
\end{equation}
\begin{equation}\label{rule:expandsU-tsmap}
\inferrule{
  \uPsi = \uPsi', \uShyp{\tsmv}{x}{\tau}{\eparse}\\
        \uGamma = \uGG{\uG}{\Gamma, x : \taudep}\\\\
  \encodeBody{b}{\ebody}\\
 \evalU{\ap{\eparse}{\ebody}}{\aein{\mathtt{SuccessE}}{\ecand}}\\
   \decodeCondE{\ecand}{\ce}\\\\
    \segOK{\segof{\ce}}{b}\\
  \cvalidE{\emptyset}{\emptyset}{\esceneSG{\uDelta}{\uGamma}{\uPsi}{\uPhi}{b}}{\ce}{e}{\aparr{\taudep}{\tau}}
}{
  \expandsSG{\uDelta}{\uGamma}{\uPsi}{\uPhi}{\utsmap{\tsmv}{b}}{\aeap{e}{x}}{\tau}
}
\end{equation}
\begin{grayparbox}
\begin{equation}\label{rule:expandsU-match}
\inferrule{
  \expandsSG{\uDelta}{\uGamma}{\uPsi}{\uPhi}{\ue}{e}{\tau}\\
  % \istypeU{\Delta}{\tau'}\\
  \{\ruleExpands{\uDelta}{\uGamma}{\uPsi}{\uPhi}{\urv_i}{r_i}{\tau}{\tau'}\}_{1 \leq i \leq n}\\
}{
  \expandsSG
    {\uDelta}{\uGamma}{\uPsi}{\uPhi}
    {\matchwith
      {\ue}
      {\seqschemaX{\urv}}
    }{\aematchwith
      {n}
      {e}
      {\seqschemaX{r}}
    }{\tau'}
}
\end{equation}
\begin{equation}\label{rule:expandsU-defuptsm}
\graybox{\inferrule{
  \expandsTU{\uDelta}{\utau}{\tau}\\
  \hastypeU{\emptyset}{\emptyset}{\eparse}{\aparr{\tBody}{\tParseResultPat}}\\\\
  \expandsUP{\uDelta}{\uGamma}{\uPsi}{\uPhi, \uPhyp{\tsmv}{x}{\tau}{\eparse'}}{\ue}{e}{\tau'}
}{
  \expandsUPX{\usyntaxup{\tsmv}{\utau}{\eparse}{\ue}}{e}{\tau'}
}}
\end{equation}
\end{grayparbox}
\end{subequations}

% \begin{subequations}\label{rules:expandsU}
% Rules (\ref*{rule:expandsU-var}) through (\ref*{rule:expandsU-case}) handle unexpanded expressions of common form. The first five of these rules are defined below:
% %Each of these rules is based on the corresponding typing rule, i.e. Rules (\ref{rule:hastypeU-var}) through (\ref{rule:hastypeU-case}), respectively. For example, the following typed expansion rules are based on the typing rules (\ref{rule:hastypeU-var}), (\ref{rule:hastypeU-lam}) and (\ref{rule:hastypeU-ap}), respectively:% for unexpanded expressions of variable, function and application form, respectively: 
% \begin{equation}\label{rule:expandsU-var}
%   \inferrule{ }{\expandsU{\uDelta}{\uGamma, \uGhyp{\ux}{x}{\tau}}{\uPsi}{\ux}{x}{\tau}}
% \end{equation}
% \begin{equation}\label{rule:expandsU-lam}
%   \inferrule{
%     \expandsTU{\uDelta}{\utau}{\tau}\\
%     \expandsU{\uDelta}{\uGamma, \uGhyp{\ux}{x}{\tau}}{\uPsi}{\ue}{e}{\tau'}
%   }{\expandsUX{\aulam{\utau}{\ux}{\ue}}{\aelam{\tau}{x}{e}}{\aparr{\tau}{\tau'}}}
% \end{equation}
% \begin{equation}\label{rule:expandsU-ap}
%   \inferrule{
%     \expandsUX{\ue_1}{e_1}{\aparr{\tau}{\tau'}}\\
%     \expandsUX{\ue_2}{e_2}{\tau}
%   }{
%     \expandsUX{\auap{\ue_1}{\ue_2}}{\aeap{e_1}{e_2}}{\tau'}
%   }
% \end{equation}
% \begin{equation}\label{rule:expandsU-tlam}
%   \inferrule{
%     \expandsU{\uDelta, \uDhyp{\ut}{t}}{\uGamma}{\uPsi}{\ue}{e}{\tau}
%   }{
%     \expandsUX{\autlam{\ut}{\ue}}{\aetlam{t}{e}}{\aall{t}{\tau}}
%   }
% \end{equation}
% \begin{equation}\label{rule:expandsU-tap}
%   \inferrule{
%     \expandsUX{\ue}{e}{\aall{t}{\tau}}\\
%     \expandsTU{\uDelta}{\utau'}{\tau'}
%   }{
%     \expandsUX{\autap{\ue}{\utau'}}{\aetap{e}{\tau'}}{[\tau'/t]\tau}
%   }
% \end{equation}
% Observe that, in each of these rules, the unexpanded and expanded expression forms in the conclusion correspond, and the premises correspond to those of the typing rule for the expanded expression form, i.e. Rules (\ref{rule:hastypeU-var}) through (\ref{rule:hastypeU-tap}), respectively. In particular, each type expansion premise in each rule above corresponds to a  type formation premise in the corresponding typing rule, and each typed expression expansion premise in each rule above corresponds to a typing premise in the corresponding typing rule. The type assigned in the conclusion of each rule above is identical to the type assigned in the conclusion of the corresponding typing rule. The ueTLM context, $\uPsi$, passes opaquely through these rules (we will define ueTLM contexts below). Rules (\ref{rules:expandsTU}) were similarly generated by mechanically transforming Rules (\ref{rules:istypeU}).

% We can express this scheme more precisely with the following rule transformation. For each rule in Rules (\ref{rules:istypeU}) and Rules (\ref{rules:hastypeU}),
% \begin{mathpar}
% \refstepcounter{equation}
% % \label{rule:expandsU-tlam}
% % \refstepcounter{equation}
% % \label{rule:expandsU-tap}
% % \refstepcounter{equation}
% \label{rule:expandsU-fold}
% \refstepcounter{equation}
% \label{rule:expandsU-unfold}
% \refstepcounter{equation}
% \label{rule:expandsU-tpl}
% \refstepcounter{equation}
% \label{rule:expandsU-pr}
% \refstepcounter{equation}
% \label{rule:expandsU-in}
% \refstepcounter{equation}
% \label{rule:expandsU-case}
% \inferrule{J_1\\ \cdots \\ J_k}{J}
% \end{mathpar}
% the corresponding typed expansion rule is 
% \begin{mathpar}
% \inferrule{
%   \Uof{J_1} \\
%   \cdots\\
%   \Uof{J_k}
% }{
%   \Uof{J}
% }
% \end{mathpar}
% where
% \[\begin{split}
% \Uof{\istypeU{\Delta}{\tau}} & = \expandsTU{\Uof{\Delta}}{\Uof{\tau}}{\tau} \\
% \Uof{\hastypeU{\Gamma}{\Delta}{e}{\tau}} & = \expandsU{\Uof{\Gamma}}{\Uof{\Delta}}{\uPsi}{\Uof{e}}{e}{\tau}\\
% \Uof{\{J_i\}_{i \in \labelset}} & = \{\Uof{J_i}\}_{i \in \labelset}
% \end{split}\]
% and where:
% \begin{itemize}
% \item $\Uof{\tau}$ is defined as follows:
%   \begin{itemize}
%   \item When $\tau$ is of definite form, $\Uof{\tau}$ is defined as in Sec. \ref{sec:syntax-U}.
%   \item When $\tau$ is of indefinite form, $\Uof{\tau}$ is a uniquely corresponding metavariable of sort $\mathsf{UTyp}$ also of indefinite form. For example, in Rule (\ref{rule:istypeU-parr}), $\tau_1$ and $\tau_2$ are of indefinite form, i.e. they match arbitrary types. The rule transformation simply ``hats'' them, i.e. $\Uof{\tau_1}=\utau_1$ and $\Uof{\tau_2}=\utau_2$.
%   \end{itemize}
% \item $\Uof{e}$ is defined as follows
% \begin{itemize}
% \item When $e$ is of definite form, $\Uof{e}$ is defined as in Sec. \ref{sec:syntax-U}. 
% \item When $e$ is of indefinite form, $\Uof{e}$ is a uniquely corresponding metavariable of sort $\mathsf{UExp}$ also of indefinite form. For example, $\Uof{e_1}=\ue_1$ and $\Uof{e_2}=\ue_2$.
% \end{itemize}
% \item $\Uof{\Delta}$ is defined as follows:
%   \begin{itemize} 
%   \item When $\Delta$ is of definite form, $\Uof{\Delta}$ is defined as above.
%   \item When $\Delta$ is of indefinite form, $\Uof{\Delta}$ is a uniquely corresponding metavariable ranging over unexpanded type formation contexts. For example, $\Uof{\Delta} = \uDelta$.
%   \end{itemize}
% \item $\Uof{\Gamma}$ is defined as follows:
%   \begin{itemize}
%   \item When $\Gamma$ is of definite form, $\Uof{\Gamma}$ produces the corresponding unexpanded typing context as follows:
% \begin{align*}
% \Uof{\emptyset} & = \uGG{\emptyset}{\emptyset}\\
% \Uof{\Gamma, \Ghyp{x}{\tau}} & = \Uof{\Gamma}, \uGhyp{\sigilof{x}}{x}{\tau}
% \end{align*}
%   \item When $\Gamma$ is of indefinite form, $\Uof{\Gamma}$ is a uniquely corresponding metavariable ranging over unexpanded typing contexts. For example, $\Uof{\Gamma} = \uGamma$.
% \end{itemize}
% \end{itemize}

% It is instructive to use this rule transformation to generate Rules (\ref{rules:expandsTU}) and Rules (\ref{rule:expandsU-var}) through (\ref{rule:expandsU-tap}) above. We omit the remaining rules, i.e. Rules (\ref*{rule:expandsU-fold}) through (\ref*{rule:expandsU-case}). By instead defining these rules solely by the rule transformation just described, we avoid having to write down a number of rules that are of limited marginal interest. Moreover, this demonstrates the general technique for generating typed expansion rules for unexpanded types and expressions of common form, so our exposition is somewhat ``robust'' to changes to the inner core. 
\vspace{-5px}\begin{grayparbox}
\vspace{8px}
\noindent\fcolorbox{black}{light-gray}{\strut$\ruleExpands{\uDelta}{\uGamma}{\uPsi}{\uPhi}{\urv}{r}{\tau}{\tau'}$}~~$\urv$ has expansion $r$ taking values of type $\tau$ to values of type $\tau'$
\begin{equation}\label{rule:ruleExpands}
\graybox{\inferrule{
  \patExpands{\uAS{\uG'}{\pctx'}}{\uPhi}{\upv}{p}{\tau}\\
  \expandsUP{\uDelta}{\uGG{\uGcons{\uG}{\uG'}}{\Gcons{\Gamma}{\pctx'}}}{\uPsi}{\uPhi}{\ue}{e}{\tau'} 
}{
  \ruleExpands{\uDelta}{\uGG{\uG}{\Gamma}}{\uPsi}{\uPhi}{\aumatchrule{\upv}{\ue}}{\aematchrule{p}{e}}{\tau}{\tau'}
}}
\end{equation}

% Rule (\ref{rule:ruleExpands}) is defined mutually with Rules (\ref{rules:expandsU}).

\subsubsection{Typed Pattern Expansion}
% \vspace{8px}
\noindent\fcolorbox{black}{light-gray}{\strut$\patExpands{\uGamma}{\uPhi}{\upv}{p}{\tau}$}~~$\upv$ has expansion $p$ matching against $\tau$ generating hypotheses $\uGamma$
\begin{subequations}\label{rules:patExpands}
\begin{equation}\label{rule:patExpands-var}
\graybox{\inferrule{ }{
  \patExpands{\uGG{\vExpands{\ux}{x}}{\Ghyp{x}{\tau}}}{\uPhi}{\ux}{x}{\tau}
}}
\end{equation}
\begin{equation}\label{rule:patExpands-wild}
\graybox{\inferrule{ }{
  \patExpands{\uGG{\emptyset}{\emptyset}}{\uPhi}{\wildp}{\aewildp}{\tau}
}}
\end{equation}
\begin{equation}\label{rule:patExpands-fold}
\graybox{\inferrule{ 
  \patExpands{\upctx}{\uPhi}{\upv}{p}{[\arec{t}{\tau}/t]\tau}
}{
  \patExpands{\upctx}{\uPhi}{\foldp{\upv}}{\aefoldp{p}}{\arec{t}{\tau}}
}}
\end{equation}
\begin{equation}\label{rule:patExpands-tpl}
\graybox{
  \inferrule{
    \tau = \aprod{\labelset}{\mapschema{\tau}{i}{\labelset}}\\\\
    \{\patExpands{{\upctx_i}}{\uPhi}{\upv_i}{p_i}{\tau_i}\}_{i \in \labelset}
  }{
    % \left(\shortstack{
    %   $\Delta \vdash_{\uPhi} \tplp{\mapschema{\upv}{i}{\labelset}}$\\
    %   $\leadsto$\\
    %   $\aetplp{\labelset}{\mapschema{p}{i}{\labelset}} : \aprod{\labelset}{\mapschema{\tau}{i}{\labelset}}$\vspace{-1.2em}}\right)
    \patExpands{\GIconsi{i \in \labelset}{\upctx_i}}{\uPhi}{\tplp{\mapschema{\upv}{i}{\labelset}}}{\aetplp{\labelset}{\mapschema{p}{i}{\labelset}}}{\tau}
  }
}
% \graybox{\inferrule{
%   \{\patExpands{{\upctx_i}}{\uPhi}{\upv_i}{p_i}{\tau_i}\}_{i \in \labelset}\\
% }{
%   % \patExpands{\Gconsi{i \in \labelset}{\pctx_i}}{\Phi}{
%   %   \autplp{\labelset}{\mapschema{\upv}{i}{\labelset}}
%   % }{
%   %   \aetplp{\labelset}{\mapschema{p}{i}{\labelset}}
%   % }{
%   %   \aprod{\labelset}{\mapschema{\tau}{i}{\labelset}}
%   % } %{\autplp{\labelset}{\mapschema{\upv}{i}{\labelset}}}{\aetplp{\labelset}{\mapschema}{p}{i}{\labelset}}{...}
%   \left(\shortstack{$\Delta \vdash_{\uPhi} \autplp{\labelset}{\mapschema{\upv}{i}{\labelset}}$\\$\leadsto$\\$\aetplp{\labelset}{\mapschema{p}{i}{\labelset}} : \aprod{\labelset}{\mapschema{\tau}{i}{\labelset}} \dashV \Gconsi{i \in \labelset}{\upctx_i}$\vspace{-1.2em}}\right)
% }}
\end{equation}
\begin{equation}\label{rule:patExpands-in}
\graybox{\inferrule{
  \patExpands{\upctx}{\uPhi}{\upv}{p}{\tau}
}{
  \patExpands{\upctx}{\uPhi}{\injp{\ell}{\upv}}{\aeinjp{\ell}{p}}{\asum{\labelset, \ell}{\mapschema{\tau}{i}{\labelset}; \mapitem{\ell}{\tau}}}
}}
\end{equation}
\begin{equation}\label{rule:patExpands-apuptsm}
\graybox{\inferrule{
  \uPhi = \uPhi', \uPhyp{\tsmv}{\_}{\tau}{\eparse}\\\\
  \encodeBody{b}{\ebody}\\
  \evalU{\ap{\eparse}{\ebody}}{\aein{\mathtt{SuccessP}}{\ecand}}\\
  \decodeCEPat{\ecand}{\cpv}\\\\
    \segOK{\segof{\cpv}}{b}\\
  \cvalidP{\upctx}{\pscene{\uDelta}{\uPhi}{b}}{\cpv}{p}{\tau}
}{
  \patExpands{\upctx}{\uPhi}{\utsmap{\tsmv}{b}}{p}{\tau}
}}
\end{equation}

\end{subequations}

In Rule (\ref{rule:patExpands-tpl}), $\upctx_i$ is shorthand for $\uGG{\uG_i}{\pctx_i}$ and $\GIconsi{i \in \labelset}{\upctx_i}$ is shorthand for \[\uGG{\GIconsi{i \in \labelset}{\uG_i}}{\Gconsi{i \in \labelset}{\pctx_i}}\] 
\end{grayparbox}


% \end{subequations}
% \clearpage
\section{Proto-Expansion Validation}\label{appendix:proto-expansions-SES}
\subsection{Syntax of Proto-Expansions}
$\arraycolsep=2pt\begin{array}{lllllll}
\textbf{Sort} & & & \textbf{Operational Form} & \textbf{Stylized Form} & \textbf{Description}\\
\mathsf{PrTyp} & \ctau & ::= & t & t & \text{variable}\\
&&& \aceparr{\ctau}{\ctau} & \parr{\ctau}{\ctau} & \text{partial function}\\
&&& \aceall{t}{\ctau} & \forallt{t}{\ctau} & \text{polymorphic}\\
&&& \acerec{t}{\ctau} & \rect{t}{\ctau} & \text{recursive}\\
&&& \aceprod{\labelset}{\mapschema{\ctau}{i}{\labelset}} & \prodt{\mapschema{\ctau}{i}{\labelset}} & \text{labeled product}\\
&&& \acesum{\labelset}{\mapschema{\ctau}{i}{\labelset}} & \sumt{\mapschema{\ctau}{i}{\labelset}} & \text{labeled sum}\\
% \LCC &&& \color{light-gray} & \color{light-gray} & \color{light-gray}\\
&&& \acesplicedt{m}{n} & \splicedt{m}{n} & \text{spliced type ref.}\\%\ECC
\mathsf{PrExp} & \ce & ::= & x & x & \text{variable}\\
&&& \aceasc{\ctau}{\ce} & \asc{\ce}{\ctau} & \text{ascription}\\
&&& \aceletsyn{x}{\ce}{\ce} & \letsyn{x}{\ce}{\ce} & \text{value binding}\\
&&& \acelam{\ctau}{x}{\ce} & \lam{x}{\ctau}{\ce} & \text{abstraction}\\
&&& \aceap{\ce}{\ce} & \ap{\ce}{\ce} & \text{application}\\
&&& \acetlam{t}{\ce} & \Lam{t}{\ce} & \text{type abstraction}\\
&&& \acetap{\ce}{\ctau} & \App{\ce}{\ctau} & \text{type application}\\
&&& \acefold{\ce} & \fold{\ce} & \text{fold}\\
&&& \aceunfold{\ce} & \unfold{\ce} & \text{unfold}\\
&&& \acetpl{\labelset}{\mapschema{\ce}{i}{\labelset}} & \tpl{\mapschema{\ce}{i}{\labelset}} & \text{labeled tuple}\\
&&& \acepr{\ell}{\ce} & \prj{\ce}{\ell} & \text{projection}\\
&&& \acein{\ell}{\ce} & \inj{\ell}{\ce} & \text{injection}\\
&&& \acecase{\labelset}{\ce}{\mapschemab{x}{\ce}{i}{\labelset}} & \caseof{\ce}{\mapschemab{x}{\ce}{i}{\labelset}} & \text{case analysis}\\
&&& \acesplicede{m}{n}{\ctau} & \splicede{m}{n}{\ctau} & \text{spliced expr. ref.}\\
\LCC \color{light-gray} &\color{light-gray} & \color{light-gray}& \color{light-gray} & \color{light-gray} & \color{light-gray}\\
&&& \acematchwith{n}{\ce}{\seqschemaX{\crv}} & \matchwith{\ce}{\seqschemaX{\crv}} & \text{match}\\
\mathsf{PrRule} & \crv & ::= & \acematchrule{p}{\ce} & \matchrule{p}{\ce} & \text{rule}\\
\mathsf{PrPat} & \cpv & ::= & \acewildp & \wildp & \text{wildcard pattern}\\
&&& \acefoldp{\cpv} & \foldp{\cpv} & \text{fold pattern}\\
&&& \acetplp{\labelset}{\mapschema{\cpv}{i}{\labelset}} & \tplp{\mapschema{\cpv}{i}{\labelset}} & \text{labeled tuple pattern}\\
&&& \aceinjp{\ell}{\cpv} & \injp{\ell}{\cpv} & \text{injection pattern}\\
% \LCC &&& \color{Yellow} & \color{Yellow} & \color{Yellow}\\
&&& \acesplicedp{m}{n}{\ctau} & \splicedp{m}{n}{\ctau} & \text{spliced pattern ref.}\ECC
\end{array}$

\subsubsection{Common Proto-Expansion Terms} Each core language term\graytxtbox{, except variable patterns,} maps onto a proto-expansion term. We refer to these as the \emph{common proto-expansion terms}. In particular:
\begin{itemize}
  \item Each type, $\tau$, maps onto a proto-type, $\Cof{\tau}$, as follows:
  \[\arraycolsep=1pt\begin{array}{rl}
  \Cof{t} & = t\\
  \Cof{\aparr{\tau_1}{\tau_2}} & = \aceparr{\Cof{\tau_1}}{\Cof{\tau_2}}\\
  \Cof{\aall{t}{\tau}} & = \aceall{t}{\Cof{\tau}}\\
  \Cof{\arec{t}{\tau}} & = \acerec{t}{\Cof{\tau}}\\
  \Cof{\aprod{\labelset}{\mapschema{\tau}{i}{\labelset}}} & = \aceprod{\labelset}{\mapschemax{\Cofv}{\tau}{i}{\labelset}}\\
  \Cof{\asum{\labelset}{\mapschema{\tau}{i}{\labelset}}} & = \acesum{\labelset}{\mapschemax{\Cofv}{\tau}{i}{\labelset}}
  \end{array}\]
  \item Each core language expression, $e$, maps onto a proto-expression, $\Cof{e}$, as follows:
  \[\arraycolsep=1pt\begin{array}{rl}
  \Cof{x} & = x\\
  \Cof{\aelam{\tau}{x}{e}} & = \acelam{\Cof{\tau}}{x}{\Cof{e}}\\
  \Cof{\aeap{e_1}{e_2}} & = \aceap{\Cof{e_1}}{\Cof{e_2}}\\
  \Cof{\aetlam{t}{e}} & = \acetlam{t}{\Cof{e}}\\
  \Cof{\aetap{e}{\tau}} & = \acetap{\Cof{e}}{\Cof{\tau}}\\
  \Cof{\aefold{e}} & = \acefold{\Cof e}\\
  \Cof{\aeunfold{e}} & = \aceunfold{\Cof{e}}\\
  \Cof{\aetpl{\labelset}{\mapschema{e}{i}{\labelset}}} & = \acetpl{\labelset}{\mapschemax{\Cofv}{e}{i}{\labelset}}\\
  \Cof{\aein{\ell}{e}} &= \acein{\ell}{\Cof{e}}\\
  \LCC \color{light-gray} & \color{light-gray} \\
  \Cof{\aematchwith{n}{e}{\seqschemaX{r}}} & = \acematchwith{n}{\Cof{e}}{\seqschemaXx{\Cofv}{r}} \ECC
  \end{array}\]
  \end{itemize}
  \begin{grayparbox}
  \begin{itemize}
  \item Each core language rule, $r$, maps onto the proto-rule, $\Cof{r}$, as follows:
  \begin{align*}
  \Cof{\aematchrule{p}{e}} & = \acematchrule{p}{\Cof{e}}
  \end{align*}
  Notice that proto-rules bind expanded patterns, not proto-patterns. This is because proto-rules appear in proto-expressions, which are generated by seTLMs. It would not be sensible for an seTLM to splice a pattern out of a literal body.
  \item Each core language pattern, $p$, except for the variable patterns, maps onto a proto-pattern, $\Cof{p}$, as follows:
  \begin{align*}
  \Cof{\aewildp} & = \acewildp\\
  \Cof{\aefoldp{p}} & = \acefoldp{\Cof{p}}\\
  \Cof{\aetplp{\labelset}{\mapschema{p}{i}{\labelset}}} & = \acetplp{\labelset}{\mapschemax{\Cofv}{p}{i}{\labelset}}\\
  \Cof{\aeinjp{\ell}{p}} & = \aceinjp{\ell}{\Cof{p}}
  \end{align*}
\end{itemize}
\end{grayparbox}

\subsubsection{Proto-Expression Encoding and Decoding}
The type abbreviated $\tCEExp$ classifies encodings of \emph{proto-expressions}. The mapping from proto-expressions to values of type $\tCEExp$ is defined by the \emph{proto-expression encoding judgement}, $\encodeCondE{\ce}{e}$. An inverse mapping is defined by the \emph{proto-expression decoding judgement}, $\decodeCondE{e}{\ce}$.

\[\begin{array}{ll}
\textbf{Judgement Form} & \textbf{Description}\\
\encodeCondE{\ce}{e} & \text{$\ce$ has encoding $e$}\\
\decodeCondE{e}{\ce} & \text{$e$ has decoding $\ce$}
\end{array}\]

Rather than picking a particular definition of $\tCEExp$ and defining the judgements above inductively against it, we only state the following condition, which establishes an isomorphism between values of type $\tCEExp$ and proto-expressions.

\begin{condition}[Proto-Expression Isomorphism]\label{condition:proto-expression-isomorphism} ~
\begin{enumerate}
\item For every $\ce$, we have $\encodeCondE{\ce}{\ecand}$ for some $\ecand$ such that $\hastypeUC{\ecand}{\tCEExp}$ and $\isvalU{\ecand}$.
\item If $\hastypeUC{\ecand}{\tCEExp}$ and $\isvalU{\ecand}$ then $\decodeCondE{\ecand}{\ce}$ for some $\ce$.
\item If $\encodeCondE{\ce}{\ecand}$ then $\decodeCondE{\ecand}{\ce}$.
\item If $\hastypeUC{\ecand}{\tCEExp}$ and $\isvalU{\ecand}$ and $\decodeCondE{\ecand}{\ce}$ then $\encodeCondE{\ce}{\ecand}$.
\item If $\encodeCondE{\ce}{\ecand}$ and $\encodeCondE{\ce}{\ecand'}$ then $\ecand=\ecand'$.
\item If $\hastypeUC{\ecand}{\tCEExp}$ and $\isvalU{\ecand}$ and $\decodeCondE{\ecand}{\ce}$ and $\decodeCondE{\ecand}{\ce'}$ then $\ce=\ce'$.
\end{enumerate}
\end{condition}\vspace{10px}

\begin{grayparbox}\vspace{-16px}
\subsubsection{Proto-Pattern Encoding and Decoding}
The type abbreviated $\tCEPat$ classifies encodings of \emph{proto-patterns}. The mapping from proto-patterns to values of type $\tCEPat$ is defined by the \emph{proto-pattern encoding judgement}, $\encodeCEPat{\cpv}{p}$. An inverse mapping is defined by the \emph{proto-expression decoding judgement}, $\decodeCEPat{p}{\cpv}$.

\[\begin{array}{ll}
\textbf{Judgement Form} & \textbf{Description}\\
\encodeCEPat{\cpv}{p} & \text{$\cpv$ has encoding $p$}\\
\decodeCEPat{p}{\cpv} & \text{$p$ has decoding $\cpv$}
\end{array}\]

Again, rather than picking a particular definition of $\tCEPat$ and defining the judgements above inductively against it, we only state the following condition, which establishes an isomorphism between values of type $\tCEPat$ and proto-patterns.

\begin{condition}[Proto-Pattern Isomorphism]\label{condition:proto-pattern-isomorphism} ~
\begin{enumerate}
\item For every $\cpv$, we have $\encodeCEPat{\cpv}{\ecand}$ for some $\ecand$ such that $\hastypeUC{\ecand}{\tCEPat}$ and $\isvalU{\ecand}$.
\item If $\hastypeUC{\ecand}{\tCEPat}$ and $\isvalU{\ecand}$ then $\decodeCEPat{\ecand}{\cpv}$ for some $\cpv$.
\item If $\encodeCEPat{\cpv}{\ecand}$ then $\decodeCEPat{\ecand}{\cpv}$.
\item If $\hastypeUC{\ecand}{\tCEPat}$ and $\isvalU{\ecand}$ and $\decodeCEPat{\ecand}{\cpv}$ then $\encodeCEPat{\cpv}{\ecand}$.
\item If $\encodeCEPat{\cpv}{\ecand}$ and $\encodeCEPat{\cpv}{\ecand'}$ then $\ecand=\ecand'$.
\item If $\hastypeUC{\ecand}{\tCEPat}$ and $\isvalU{\ecand}$ and $\decodeCEPat{\ecand}{\cpv}$ and $\decodeCEPat{\ecand}{\cpv'}$ then $\cpv=\cpv'$.
\end{enumerate}
\end{condition}
\end{grayparbox}

\subsubsection{Segmentations}\label{appendix:segmentations-U}
The \emph{segmentation}, $\psi$, of a proto-type, $\segof{\ctau}$ or proto-expression, $\segof{\ce}$, is the finite set of references to spliced types and expressions that it mentions.
\[
\begin{array}{lll}
\segof{t} & = & \emptyset\\
\segof{\aceparr{\ctau_1}{\ctau_2}} & = & \segof{\ctau_1} \cup \segof{\ctau_2}\\
\segof{\aceall{t}{\ctau}} & = & \segof{\ctau}\\
\segof{\acerec{t}{\ctau}} & = & \segof{\ctau}\\
\segof{\aceprod{\labelset}{\mapschema{\ctau}{i}{\labelset}}} & = & \bigcup_{i \in \labelset} \segof{\ctau_i}\\
\segof{\acesum{\labelset}{\mapschema{\ctau}{i}{\labelset}}} & = & \bigcup_{i \in \labelset} \segof{\ctau_i}\\
\segof{\acesplicedt{m}{n}} & = & \{ \acesplicedt{m}{n} \}\\
~\\
\segof{x} & = & \emptyset\\
\segof{\aceasc{\ctau}{\ce}} & = & \segof{\ctau} \cup \segof{\ce}\\
\segof{\aceletsyn{x}{\ce_1}{\ce_2}} & = & \segof{\ce_1} \cup \segof{\ce_2}\\
\segof{\acelam{\ctau}{x}{\ce}} & = & \segof{\ctau} \cup \segof{\ce}\\
\segof{\aceap{\ce_1}{\ce_2}} & = & \segof{\ce_1} \cup \segof{\ce_2}\\
\segof{\acetlam{t}{\ce}} & = & \segof{\ce}\\
\segof{\acetap{\ce}{\ctau}} & = & \segof{\ce} \cup \segof{\ctau}\\
\segof{\acefold{\ce}} & = & \segof{\ce}\\
\segof{\aceunfold{\ce}} & = & \segof{\ce}\\
\segof{\acetpl{\labelset}{\mapschemab{x}{\ce}{i}{\labelset}}} & = & \bigcup_{i \in \labelset} \segof{\ce_i}\\
\segof{\acepr{\ell}{\ce}} & = & \segof{\ce}\\
\segof{\acein{\ell}{\ce}} & = & \segof{\ce}\\
\segof{\acecase{\labelset}{\ce}{\mapschemab{x}{\ce}{i}{\labelset}}} & = & \segof{\ce} \cup \bigcup_{i \in \labelset} \segof{\ce_i}\\
\segof{\acesplicede{m}{n}{\ctau}} & = & \{ \acesplicede{m}{n}{\ctau} \} \cup \segof{\ctau}\\
\LCC \color{light-gray} & \color{light-gray} & \color{light-gray}\\
\segof{\acematchwith{n}{\ce}{\seqschemaX{\crv}}} & = & \segof{\ce} \cup \bigcup_{1 \leq i \leq n} \segof{\crv_i}\\
~\\
\segof{\acematchrule{p}{\ce}} & = & \segof{\ce}\ECC
\end{array}
\]

\begin{grayparbox}
The splice summary of a proto-pattern, $\segof{\cpv}$, is the finite set of references to spliced types and patterns that it mentions.
\[
\begin{array}{lll}
\segof{\acewildp} & = & \emptyset\\
\segof{\acefoldp{\cpv}} & = & \segof{\cpv}\\
\segof{\acetplp{\labelset}{\mapschema{\cpv}{i}{\labelset}}} & = & \bigcup_{i \in \labelset} \segof{\cpv_i}\\
\segof{\aceinjp{\ell}{\cpv}} & = & \segof{\cpv}\\
\segof{\acesplicedp{m}{n}{\ctau}} & = & \{ \acesplicedp{m}{n}{\ctau} \} \cup \segof{\ctau}
\end{array}\]
\end{grayparbox}

The predicate $\segOK{\psi}{b}$ defined below checks that each segment in $\psi$, has positive extent and is within bounds of $b$, and that the segments in $\psi$ do not overlap or sit immediately adjacent to one another, and that spliced segments that are exactly overlapping have equal segment types.

\begin{definition}[Segmentation Validity] $\segOK{\psi}{b}$ iff 
\begin{enumerate}
  \item For each $\acesplicedt{m}{n} \in \psi$, all of the following hold:
    \begin{enumerate}
      \item $0 \leq m \leq n < \sizeof{b}$
      \item For each $\acesplicedt{m'}{n'} \in \psi$, either 
        \begin{enumerate}
          \item $m=m'$ and $n=n'$; or 
          \item $n' < m - 1$; or 
          \item $m' > n + 1$
        \end{enumerate}
      \item For each $\acesplicede{m'}{n'}{\ctau} \in \psi$, either 
        \begin{enumerate}
          \item $n' < m - 1$; or 
          \item $m' > n + 1$
        \end{enumerate}
    \end{enumerate}
    \begin{grayparbox}
    \begin{enumerate}
      \item[(d)] For each $\acesplicedp{m'}{n'}{\ctau} \in \psi$, either 
        \begin{enumerate}
          \item $n' < m - 1$; or
          \item $m' > n + 1$
        \end{enumerate}
    \end{enumerate}
    \end{grayparbox}
  \item For each $\acesplicede{m}{n}{\ctau} \in \psi$, all of the following hold:
    \begin{enumerate}
      \item $0 \leq m \leq n < \sizeof{b}$
      \item For each $\acesplicedt{m'}{n'} \in \psi$, either 
        \begin{enumerate}
          \item $n' < m - 1$; or 
          \item $m' > n + 1$
        \end{enumerate}
      \item For each $\acesplicede{m'}{n'}{\ctau'} \in \psi$, either 
        \begin{enumerate}
          \item $m=m'$ and $n=n'$ and $\ctau=\ctau'$; or 
          \item $n' < m - 1$; or 
          \item $m' > n + 1$
        \end{enumerate}
    \end{enumerate}
\end{enumerate}
\begin{grayparbox}
\begin{enumerate}
\item[3.] For each $\acesplicedp{m}{n}{\ctau} \in \psi$, all of the following hold:
    \begin{enumerate}
      \item $0 \leq m \leq n < \sizeof{b}$
      \item For each $\acesplicedt{m'}{n'} \in \psi$, either 
        \begin{enumerate}
          \item $n' < m - 1$; or 
          \item $m' > n + 1$
        \end{enumerate}
      \item For each $\acesplicede{m'}{n'}{\ctau'} \in \psi$, either 
        \begin{enumerate}
          \item $n' < m - 1$; or 
          \item $m' > n + 1$
        \end{enumerate}
      \item For each $\acesplicedp{m'}{n'}{\ctau'} \in \psi$, either 
        \begin{enumerate}
          \item $m=m'$ and $n=n'$ and $\ctau=\ctau'$; or 
          \item $n' < m - 1$; or 
          \item $m' > n + 1$
        \end{enumerate}
    \end{enumerate}
\end{enumerate}
\end{grayparbox}
\end{definition}

\subsection{Proto-Type Validation}\label{appendix:proto-type-validation-SES}
%Each of these rules is defined based on the corresponding type formation rule, i.e. Rules (\ref{rule:istypeU-var}) through (\ref{rule:istypeU-sum}), respectively. For example, the following candidate expansion type validation rules are based on type formation rules (\ref{rule:istypeU-var}), (\ref{rule:istypeU-parr}) and (\ref{rule:istypeU-all}), respectively: 
\emph{Type splicing scenes}, $\tscenev$, are of the form $\tsceneUP{\uDelta}{b}$.

\vspace{10px}\noindent\fbox{\strut$\cvalidT{\Delta}{\tscenev}{\ctau}{\tau}$}~~$\ctau$ has well-formed expansion $\tau$
\begin{subequations}\label{rules:cvalidT-U}
\begin{equation}\label{rule:cvalidT-U-tvar}
\inferrule{ }{
  \cvalidT{\Delta, \Dhyp{t}}{\tscenev}{t}{t}
}
\end{equation}
\begin{equation}\label{rule:cvalidT-U-parr}
  \inferrule{
    \cvalidT{\Delta}{\tscenev}{\ctau_1}{\tau_1}\\
    \cvalidT{\Delta}{\tscenev}{\ctau_2}{\tau_2}
  }{
    \cvalidT{\Delta}{\tscenev}{\aceparr{\ctau_1}{\ctau_2}}{\aparr{\tau_1}{\tau_2}}
  }
\end{equation}
\begin{equation}\label{rule:cvalidT-U-all}
  \inferrule {
    \cvalidT{\Delta, \Dhyp{t}}{\tscenev}{\ctau}{\tau}
  }{
    \cvalidT{\Delta}{\tscenev}{\aceall{t}{\ctau}}{\aall{t}{\tau}}
  }
\end{equation}
\begin{equation}\label{rule:cvalidT-U-rec}
  \inferrule{
    \cvalidT{\Delta, \Dhyp{t}}{\tscenev}{\ctau}{\tau}
  }{
    \cvalidT{\Delta}{\tscenev}{\acerec{t}{\ctau}}{\arec{t}{\tau}}
  }
\end{equation}
\begin{equation}\label{rule:cvalidT-U-prod}
  \inferrule{
    \{\cvalidT{\Delta}{\tscenev}{\ctau_i}{\tau_i}\}_{i \in \labelset}
  }{
    \cvalidT{\Delta}{\tscenev}{\aceprod{\labelset}{\mapschema{\ctau}{i}{\labelset}}}{\aprod{\labelset}{\mapschema{\tau}{i}{\labelset}}}
  }
\end{equation}
\begin{equation}\label{rule:cvalidT-U-sum}
  \inferrule{
    \{\cvalidT{\Delta}{\tscenev}{\ctau_i}{\tau_i}\}_{i \in \labelset}
  }{
    \cvalidT{\Delta}{\tscenev}{\acesum{\labelset}{\mapschema{\ctau}{i}{\labelset}}}{\asum{\labelset}{\mapschema{\tau}{i}{\labelset}}}
  }
\end{equation}
\begin{equation}\label{rule:cvalidT-U-splicedt}
  \inferrule{
    \parseUTyp{\bsubseq{b}{m}{n}}{\utau}\\
    \expandsTU{\uDD{\uD}{\Delta_\text{app}}}{\utau}{\tau}\\
    \Delta \cap \Delta_\text{app} = \emptyset
  }{
    \cvalidT{\Delta}{\tsceneU{\uDD{\uD}{\Delta_\text{app}}}{b}}{\acesplicedt{m}{n}}{\tau}
  }
\end{equation}
\end{subequations}


%Rule (\ref*{rule:cvalidT-U-splicedt}) governs this form:
%\chapter{Dependent Labeled Product Kinds}
% \begin{landscape}
% \begin{equation}\label{rule:iskind-dlprod}
% \inferrule{
% 	\{\iskind{\Omega}{\Delta \cup \{u_{i, j} :: \kappa_j\}_{1 \leq j < i}}{\Gamma}{\kappa_i}\}_{1 \leq i \leq n}
% }{
% 	\iskindX{\akdprodstd}
% }
% \end{equation}

% \begin{equation}\label{rule:kequal-dlprod}
% \inferrule{
% 	\{\kequal{\Omega}{\Delta \cup \{u_{i, j} :: \kappa_j\}_{1 \leq j < i}}{\Gamma}{\kappa_i}{\kappa'_i}\}_{1 \leq i \leq n}
% }{
% 	\kequalX{\akdprodstd}{\akdprod{n}{\seqschemaX{\ell}}{\seqschemaijb{u}{\kappa'}{i}{1}{n}{j}{1}{i}}}
% }
% \end{equation}
% \begin{equation}\label{rule:ksub-dlprod}
% \inferrule{
% 	\{\ksub{\Omega}{\Delta \cup \{u_{i,j} :: \kappa_j\}_{1 \leq j < i}}{\Gamma}{\kappa_i}{\kappa'_i}\}_{1 \leq i \leq n}
% }{
% 	\ksubX{\akdprodstd}{\akdprod{n}{\seqschemaX{\ell}}{\seqschemaijb{u}{\kappa'}{i}{1}{n}{j}{1}{i}}}
% }
% \end{equation}
% \begin{equation}\label{rule:haskind-dtpl}
% \inferrule{
% 	\{\haskind{\Omega}{\Delta \cup \{u_{i, j} :: \aksing{c_j}\}_{1 \leq j < i}}{\Gamma}{c_i}{\kappa_i}\}_{1 \leq i \leq n}
% }{
% 	\haskindX{\adtplX}{\akdprodstd}
% }
% \end{equation}
% \begin{equation}\label{rule:haskind-prj}
% \inferrule{
% 	\haskindX{c}{
% 		\akdprod{
% 			n' + 1 + n''
% 		}{
% 			\seqschema{\ell'}{i}{1}{n'}, \ell, \seqschema{\ell''}{i}{1}{n''}
% 		}{
% 			\seqschemaijb{u'}{\kappa'}{i}{1}{n'}{j}{1}{i};
% 			\{u_{j}\}_{1 \leq j \leq n'}.\kappa;
% 			\seqschemaijb{u''}{\kappa''}{i}{1}{n''}{j}{1}{i}
% 		}
% 	}
% }{
% 	\haskindX{\adprj{\ell}{c}}{[\{\adprj{\ell'_j}{c}/u_{j}\}_{1 \leq j \leq n'}]\kappa}
% }
% \end{equation}
% \begin{equation}\label{rule:cequal-dtpl}
% \inferrule{
% 	c=\adtplX\\
% 	c'=\adtpl{n}{\seqschemaX{\ell}}{\seqschemaijb{u}{c'}{i}{1}{n}{j}{1}{i}}\\\\
% 	\{\cequal{\Omega}{\Delta \cup \{u_{i, j} :: \aksing{\kappa_j}\}_{1 \leq j < i}}{\Gamma}{c_i}{c'_i}{\kappa_i}\}_{1 \leq i \leq n}
% }{
% 	\cequalX{c}{c'}{\akdprodstd}
% }
% \end{equation}
% \begin{equation}\label{rule:cequal-prj-1}
% \inferrule{
%   \cequalX{c}{c'}{
% 		\akdprod{
% 			n' + 1 + n''
% 		}{
% 			\seqschema{\ell'}{i}{1}{n'}, \ell, \seqschema{\ell''}{i}{1}{n''}
% 		}{
% 			\seqschemaijb{u'}{\kappa'}{i}{1}{n'}{j}{1}{i};
% 			\{u_{j}\}_{1 \leq j \leq n'}.\kappa;
% 			\seqschemaijb{u''}{\kappa''}{i}{1}{n''}{j}{1}{i}
% 		}
% 	}	
% }{
% 	\cequalX{\adprj{\ell}{c}}{\adprj{\ell}{c'}}{\kappa}
% }
% \end{equation}
% \begin{equation}\label{rule:cequal-prj-2}
% \inferrule{
% 	c = \adtpl{
% 				n' + 1 + n''
% 			}{
% 				\seqschema{\ell'}{i}{1}{n'}, \ell, \seqschema{\ell''}{i}{1}{n''}
% 			}{
% 				\seqschemaijb{u'}{c'}{i}{1}{n'}{j}{1}{i};
% 				\{u_j\}_{1 \leq j \leq n}.c_\ell; 
% 				\seqschemaijb{u''}{c''}{i}{1}{n''}{j}{1}{i}
% 			}\\
% 	\haskindX{c}{
% 		\akdprod{
% 			n' + 1 + n''
% 		}{
% 			\seqschema{\ell'}{i}{1}{n'}, \ell, \seqschema{\ell''}{i}{1}{n''}
% 		}{
% 			\seqschemaijb{u'}{\kappa'}{i}{1}{n'}{j}{1}{i};
% 			\{u_{j}\}_{1 \leq j \leq n'}.\kappa;
% 			\seqschemaijb{u''}{\kappa''}{i}{1}{n''}{j}{1}{i}
% 		}
% 	}
% }{
% 	\cequalX{
% 		\adprj{\ell}{
% 			c
% 		}
% 	}{[\{\adprj{\ell'_j}{c}/u_j\}_{1 \leq j \leq n'}]c_\ell}{[\{\adprj{\ell'_j}{c}/u_j\}_{1 \leq j \leq n'}]\kappa}
% }
% \end{equation}

% \end{landscape}

\subsection{Proto-Expression Validation}

\emph{Expression splicing scenes}, $\escenev$, are of the form $\esceneSGB{\uDelta}{\uGamma}{\uPsi}{\uPhi}{b}$. We write $\tsfrom{\escenev}$ for the type splicing scene constructed by dropping unnecessary contexts from $\escenev$:
\[\tsfrom{\esceneSGB{\uDelta}{\uGamma}{\uPsi}{\uPhi}{b}} = \tsceneUP{\uDelta}{b}\]

\vspace{10px}\noindent\fbox{\strut$\cvalidE{\Delta}{\Gamma}{\escenev}{\ce}{e}{\tau}$}~~$\ce$ has expansion $e$ of type $\tau$
\begin{subequations}\label{rules:cvalidE-U}
\begin{equation}\label{rule:cvalidE-U-var}
\inferrule{ }{
  \cvalidE{\Delta}{\Gamma, \Ghyp{x}{\tau}}{\escenev}{x}{x}{\tau}
}
\end{equation}
\begin{equation}\label{rule:cvalidE-U-asc}
\inferrule{
  \cvalidT{\Delta}{\tsfrom{\escenev}}{\ctau}{\tau}\\
  \cvalidE{\Delta}{\Gamma}{\escenev}{\ce}{e}{\tau}
}{
  \cvalidE{\Delta}{\Gamma}{\escenev}{\aceasc{\ctau}{\ce}}{e}{\tau}
}
\end{equation}
\begin{equation}\label{rule:cvalidE-U-letsyn}
  \inferrule{
    \cvalidE{\Delta}{\Gamma}{\escenev}{\ce_1}{e_1}{\tau_1}\\
    \cvalidE{\Delta}{\Gamma, x : \tau_1}{\ce_2}{e_2}{\tau_2}
  }{
    \cvalidE{\Delta}{\Gamma}{\escenev}{\aceletsyn{x}{\ce_1}{\ce_2}}{
      \aeap{\aelam{\tau_1}{x}{e_2}}{e_1}
    }{\tau_2}
  }
\end{equation}
\begin{equation}\label{rule:cvalidE-U-lam}
\inferrule{
  \cvalidT{\Delta}{\tsfrom{\escenev}}{\ctau}{\tau}\\
  \cvalidE{\Delta}{\Gamma, \Ghyp{x}{\tau}}{\escenev}{\ce}{e}{\tau'}
}{
  \cvalidE{\Delta}{\Gamma}{\escenev}{\acelam{\ctau}{x}{\ce}}{\aelam{\tau}{x}{e}}{\aparr{\tau}{\tau'}}
}
\end{equation}
\begin{equation}\label{rule:cvalidE-U-ap}
  \inferrule{
    \cvalidE{\Delta}{\Gamma}{\escenev}{\ce_1}{e_1}{\aparr{\tau}{\tau'}}\\
    \cvalidE{\Delta}{\Gamma}{\escenev}{\ce_2}{e_2}{\tau}
  }{
    \cvalidE{\Delta}{\Gamma}{\escenev}{\aceap{\ce_1}{\ce_2}}{\aeap{e_1}{e_2}}{\tau'}
  }
\end{equation}
\begin{equation}\label{rule:cvalidE-U-tlam}
  \inferrule{
    \cvalidE{\Delta, \Dhyp{t}}{\Gamma}{\escenev}{\ce}{e}{\tau}
  }{
    \cvalidE{\Delta}{\Gamma}{\escenev}{\acetlam{t}{\ce}}{\aetlam{t}{e}}{\aall{t}{\tau}}
  }
\end{equation}
\begin{equation}\label{rule:cvalidE-U-tap}
  \inferrule{
    \cvalidE{\Delta}{\Gamma}{\escenev}{\ce}{e}{\aall{t}{\tau}}\\
    \cvalidT{\Delta}{\tsfrom{\escenev}}{\ctau'}{\tau'}
  }{
    \cvalidE{\Delta}{\Gamma}{\escenev}{\acetap{\ce}{\ctau'}}{\aetap{e}{\tau'}}{[\tau'/t]\tau}
  }
\end{equation}
\begin{equation}\label{rule:cvalidE-U-fold}
  \inferrule{\
    % \cvalidT{\Delta, \Dhyp{t}}{\tsfrom{\escenev}}{\ctau}{\tau}\\
    \cvalidE{\Delta}{\Gamma}{\escenev}{\ce}{e}{[\arec{t}{\tau}/t]\tau}
  }{
    \cvalidE{\Delta}{\Gamma}{\escenev}{\acefold{\ce}}{\aefold{e}}{\arec{t}{\tau}}
  }
\end{equation}
\begin{equation}\label{rule:cvalidE-U-unfold}
  \inferrule{
    \cvalidE{\Delta}{\Gamma}{\escenev}{\ce}{e}{\arec{t}{\tau}}
  }{
    \cvalidE{\Delta}{\Gamma}{\escenev}{\aceunfold{\ce}}{\aeunfold{e}}{[\arec{t}{\tau}/t]\tau}
  }
\end{equation}
\begin{equation}\label{rule:cvalidE-U-tpl}
  \inferrule{
    \tau = \aprod{\labelset}{\mapschema{\tau}{i}{\labelset}}\\\\
    \{\cvalidE{\Delta}{\Gamma}{\escenev}{\ce_i}{e_i}{\tau_i}\}_{i \in \labelset}
  }{
  % \left(\shortstack{$\Delta~\Gamma \vdash^{\escenev} \acetpl{\labelset}{\mapschema{\ce}{i}{\labelset}}$\\$\leadsto$\\$\aetpl{\labelset}{\mapschema{e}{i}{\labelset}} : \aprod{\labelset}{\mapschema{\tau}{i}{\labelset}}$\vspace{-1.2em}}\right)
    \cvalidE{\Delta}{\Gamma}{\escenev}{\acetpl{\labelset}{\mapschema{\ce}{i}{\labelset}}}{\aetpl{\labelset}{\mapschema{e}{i}{\labelset}}}{\tau}
  }
\end{equation}
\begin{equation}\label{rule:cvalidE-U-pr}
  \inferrule{
    \cvalidE{\Delta}{\Gamma}{\escenev}{\ce}{e}{\aprod{\labelset, \ell}{\mapschema{\tau}{i}{\labelset}; \ell \hookrightarrow \tau}}
  }{
    \cvalidE{\Delta}{\Gamma}{\escenev}{\acepr{\ell}{\ce}}{\aepr{\ell}{e}}{\tau}
  }
\end{equation}
\begin{equation}\label{rule:cvalidE-U-in}
  \inferrule{
    % \{\cvalidT{\Delta}{\tsfrom{\escenev}}{\ctau_i}{\tau_i}\}_{i \in \labelset}\\
    % \cvalidT{\Delta}{\tsfrom{\escenev}}{\ctau}{\tau}\\
    \cvalidE{\Delta}{\Gamma}{\escenev}{\ce}{e}{\tau'}
  }{
    % \left(\shortstack{
    %   $\Delta~\Gamma \vdash^{\escenev} \acein{\ell}{\ce}$\\
    %   $\leadsto$\\
    %   $\aein{\ell}{e} : \asum{\labelset, \ell}{\mapschema{\tau}{i}{\labelset}; \ell \hookrightarrow \tau}$\vspace{-1.2em}
    % }\right)
    \cvalidE{\Delta}{\Gamma}{\escenev}{\acein{\ell}{\ce}}{\aein{\ell}{e}}{\asum{\labelset, \ell}{\mapschema{\tau}{i}{\labelset}; \ell \hookrightarrow \tau'}}
  }
\end{equation}
\begin{equation}\label{rule:cvalidE-U-case}
  \inferrule{
    \cvalidE{\Delta}{\Gamma}{\escenev}{\ce}{e}{\asum{\labelset}{\mapschema{\tau}{i}{\labelset}}}\\
    % \cvalidT{\Delta}{\tsfrom{\escenev}}{\ctau}{\tau}\\
    \{\cvalidE{\Delta}{\Gamma, x_i : \tau_i}{\escenev}{\ce_i}{e_i}{\tau}\}_{i \in \labelset}
  }{
    \cvalidE{\Delta}{\Gamma}{\escenev}{\acecase{\labelset}{\ce}{\mapschemab{x}{\ce}{i}{\labelset}}}{\aecase{\labelset}{e}{\mapschemab{x}{e}{i}{\labelset}}}{\tau}
  }
\end{equation}
\begin{equation}\label{rule:cvalidE-U-splicede}
\inferrule{
  \cvalidT{\emptyset}{\tsfrom{\escenev}}{\ctau}{\tau}\\
  \escenev=\esceneSGB{\uDD{\uD}{\Delta_\text{app}}}{\uGG{\uG}{\Gamma_\text{app}}}{\uPsi}{\uPhi}{b}\\
  \parseUExp{\bsubseq{b}{m}{n}}{\ue}\\
  \expandsSG{\uDD{\uD}{\Delta_\text{app}}}{\uGG{\uG}{\Gamma_\text{app}}}{\uPsi}{\uPhi}{\ue}{e}{\tau}\\\\
  \Delta \cap \Delta_\text{app} = \emptyset\\
  \domof{\Gamma} \cap \domof{\Gamma_\text{app}} = \emptyset
}{
  \cvalidE{\Delta}{\Gamma}{\escenev}{\acesplicede{m}{n}{\ctau}}{e}{\tau}
}
\end{equation}
\begin{grayparbox}
\begin{equation}\label{rule:cvalidE-U-match}
\graybox{\inferrule{
  % \escenev = \esceneUP{\uDD{\uD}{\Delta_\text{app}}}{\uGamma}{\uPsi}{\uPhi}{b}\\\\
  % \istypeU{\Delta \cup \Delta_\text{app}}{\tau'}\\
  \cvalidE{\Delta}{\Gamma}{\escenev}{\ce}{e}{\tau}\\
  % \cvalidT{\Delta}{\tsfrom{\escenev}}{\ctau'}{\tau'}\\\\
  \{\cvalidR{\Delta}{\Gamma}{\escenev}{\crv_i}{r_i}{\tau}{\tau'}\}_{1 \leq i \leq n}\\
}{\cvalidE{\Delta}{\Gamma}{\escenev}{\acematchwith{n}{\ce}{\seqschemaX{\crv}}}{\aematchwith{n}{e}{\seqschemaX{r}}}{\tau'}}}
\end{equation}
\end{grayparbox}
\end{subequations}
% \clearpage
\vspace{-5px}
\begin{grayparbox}
\vspace{15px}
\noindent\fbox{\strut$\cvalidR{\Delta}{\Gamma}{\escenev}{\crv}{r}{\tau}{\tau'}$}~~$\crv$ has expansion $r$ taking values of type $\tau$ to values of type $\tau'$
\begin{equation}\label{rule:cvalidR-UP}
\inferrule{
  % \escenev = \esceneUP{\uDD{\uD}{\Delta_\text{app}}}{\uGamma}{\uPsi}{\uPhi}{b}\\\\
  \patTypeD{\Delta \cup \Delta_\text{app}}{\pctx'}{p}{\tau}\\
  \cvalidE{\Delta}{\Gcons{\Gamma}{\pctx'}}{\escenev}{\ce}{e}{\tau'}
}{
  \cvalidR{\Delta}{\Gamma}{\escenev}{\acematchrule{p}{\ce}}{\aematchrule{p}{e}}{\tau}{\tau'}
}
\end{equation}
\end{grayparbox}
\vspace{-5px}\begin{grayparbox}
\subsection{Proto-Pattern Validation}\label{appendix:proto-pattern-validation-P}
\emph{Pattern splicing scenes}, $\pscenev$, are of the form $\pscene{\uDelta}{\uPhi}{b}$.

\vspace{10px}\noindent\fbox{\strut$\cvalidP{\upctx}{\pscenev}{\cpv}{p}{\tau}$}~~$\cpv$ has expansion $p$ matching against $\tau$ generating hypotheses $\upctx$
% \begin{grayparbox}
\begin{subequations}\label{rules:cvalidP-UP}
\begin{equation}\label{rule:cvalidP-UP-wild}
\inferrule{ }{
  \cvalidP{\uGG{\emptyset}{\emptyset}}{\pscenev}{\acewildp}{\aewildp}{\tau}
}
\end{equation}
\begin{equation}\label{rule:cvalidP-UP-fold}
\inferrule{
  \cvalidP{\upctx}{\pscenev}{\cpv}{p}{[\arec{t}{\tau}/t]\tau}
}{
  \cvalidP{\upctx}{\pscenev}{\acefoldp{\cpv}}{\aefoldp{p}}{\arec{t}{\tau}}
}
\end{equation}
\begin{equation}\label{rule:cvalidP-UP-tpl}
\inferrule{
  \tau = \aprod{\labelset}{\mapschema{\tau}{i}{\labelset}}\\\\
  \{\cvalidP{\upctx_i}{\pscenev}{\cpv_i}{p_i}{\tau_i}\}_{i \in \labelset}
}{
% \left(\shortstack{$\vdash^{\pscenev} \acetplp{\labelset}{\mapschema{\cpv}{i}{\labelset}}$\\$\leadsto$\\$\aetplp{\labelset}{\mapschema{p}{i}{\labelset}} : \aprod{\labelset}{\mapschema{\tau}{i}{\labelset}}~\dashVx^{\,\Gconsi{i \in \labelset}{\upctx_i}}$\vspace{-1.2em}}\right)
  \cvalidP{\GIconsi{i \in \labelset}{\upctx_i}}{\pscenev}{\acetplp{\labelset}{\mapschema{\cpv}{i}{\labelset}}}{\aetplp{\labelset}{\mapschema{p}{i}{\labelset}}}{\tau}
}
\end{equation}
\begin{equation}\label{rule:cvalidP-UP-in}
\inferrule{
  \cvalidP{\upctx}{\pscenev}{\cpv}{p}{\tau}
}{
  \cvalidP{\upctx}{\pscenev}{\aceinjp{\ell}{\cpv}}{\aeinjp{\ell}{p}}{\asum{\labelset, \ell}{\mapschema{\tau}{i}{\labelset}; \mapitem{\ell}{\tau}}}
}
\end{equation}
\begin{equation}\label{rule:cvalidP-UP-spliced}
\inferrule{
  \cvalidT{\emptyset}{\tsceneUP{\uDelta}{b}}{\ctau}{\tau}\\
  \parseUPat{\bsubseq{b}{m}{n}}{\upv}\\
  \patExpands{\upctx}{\uPhi}{\upv}{p}{\tau}
}{
  \cvalidP{\upctx}{\pscene{\uDelta}{\uPhi}{b}}{\acesplicedp{m}{n}{\ctau}}{p}{\tau}
}
\end{equation}
\end{subequations}
\end{grayparbox}
% Observe that, in each of these rules, the proto-expression form and the expanded expression form in the conclusion correspond, and the premises correspond to those of the corresponding typing rule, i.e. Rules (\ref{rule:hastypeU-var}) through (\ref{rule:hastypeU-ap}), respectively. The expression splicing scene, $\escenev$, passes opaquely through these rules.


% We can express this scheme more precisely with the following rule transformation. For each rule in Rules (\ref{rules:hastypeU}),
% \begin{mathpar}\refstepcounter{equation}
% \label{rule:cvalidE-U-tlam}
% \refstepcounter{equation}
% \label{rule:cvalidE-U-tap}
% \refstepcounter{equation}
% \label{rule:cvalidE-U-fold}
% \refstepcounter{equation}
% \label{rule:cvalidE-U-unfold}
% \refstepcounter{equation}
% \label{rule:cvalidE-U-tpl}
% \refstepcounter{equation}
% \label{rule:cvalidE-U-pr}
% \refstepcounter{equation}
% \label{rule:cvalidE-U-in}
% \refstepcounter{equation}
% \label{rule:cvalidE-U-case}
%   \inferrule{
%     J_1\\
%     \cdots\\
%     J_k
%   }{
%     J
%   }
% \end{mathpar}
% the corresponding proto-expression validation rule is 
% \begin{mathpar}
%   \inferrule{
%     \Cof{J_1}\\
%     \cdots\\
%     \Cof{J_k}
%   }{
%     \Cof{J}
%   }
% \end{mathpar}
% where 
% \[\begin{split}
%   \Cof{\istypeU{\Delta}{\tau}} & = \cvalidT{\Delta}{\tsfrom{\escenev}}{\Cof{\tau}}{\tau}\\
%   \Cof{\hastypeU{\Delta}{\Gamma}{e}{\tau}} & = \cvalidE{\Delta}{\Gamma}{\escenev}{\Cof{e}}{e}{\tau}\\
%   \Cof{\{J_i\}_{i \in \labelset}} & = \{\Cof{J_i}\}_{i \in \labelset}
% \end{split}\]
% and where:
% \begin{itemize}
% \item $\Cof{\tau}$ is defined as follows:
%   \begin{itemize}
%   \item When $\tau$ is of definite form, $\Cof{\tau}$ is defined as in Sec. \ref{sec:ce-syntax-U}.
%   \item When $\tau$ is of indefinite form, $\Cof{\tau}$ is a uniquely corresponding metavariable of sort $\mathsf{CETyp}$ also of indefinite form. For example, $\Cof{\tau_1}=\ctau_1$ and $\Cof{\tau_2}=\ctau_2$.
%   \end{itemize}
% \item $\Cof{e}$ is defined as follows
%   \begin{itemize}
%   \item When $e$ is of definite form, $\Cof{e}$ is defined as in Sec. \ref{sec:ce-syntax-U}. 
%   \item When $e$ is of indefinite form, $\Cof{e}$ is a uniquely corresponding metavariable of sort $\mathsf{CEExp}$ also of indefinite form. For example, $\Cof{e_1}=\ce_1$ and $\Cof{e_2}=\ce_2$.
%   \end{itemize}
% \end{itemize}

% It is instructive to use this rule transformation to generate Rules (\ref{rule:cvalidE-U-var}) through (\ref{rule:cvalidE-U-ap}) above. We omit the remaining rules for common forms, i.e. Rules (\ref*{rule:cvalidE-U-tlam}) through (\ref*{rule:cvalidE-U-case}).

\section{Metatheory}\label{appendix:metatheory-SES}
\subsection{Type Expansion}
% The Type Expansion Lemma establishes that the expansion of an unexpanded type is a well-formed type.

\begin{lemma}[Type Expansion]\label{lemma:type-expansion-U} If $\expandsTU{\uDD{\uD}{\Delta}}{\utau}{\tau}$ then $\istypeU{\Delta}{\tau}$.\end{lemma}
\begin{proof} By rule induction over Rules (\ref{rules:expandsTU}). In each case, we apply the IH to or over each premise, then apply the corresponding type formation rule in Rules (\ref{rules:istypeU}). \end{proof}

\begin{lemma}[Proto-Type Validation]\label{lemma:candidate-expansion-type-validation}
If $\cvalidT{\Delta}{\tsceneU{\uDD{\uD}{\Delta_\text{app}}}{b}}{\ctau}{\tau}$ and $\Delta \cap \Delta_\text{app}=\emptyset$ then $\istypeU{\Dcons{\Delta}{\Delta_\text{app}}}{\tau}$.
\end{lemma}
\begin{proof} By rule induction over Rules (\ref{rules:cvalidT-U}).
\begin{byCases}
\item[\text{(\ref{rule:cvalidT-U-tvar})}] ~
\begin{pfsteps*}
   \item $\Delta=\Delta', \Dhyp{t}$ \BY{assumption}
   \item $\ctau=t$ \BY{assumption}
   \item $\tau=t$ \BY{assumption}
   \item $\istypeU{\Delta', \Dhyp{t}}{t}$ \BY{Rule (\ref{rule:istypeU-var})} \pflabel{istype}
   \item $\istypeU{\Dcons{\Delta', \Dhyp{t}}{\Delta_\text{app}}}{t}$ \BY{Lemma \ref{lemma:weakening-U} over $\Delta_\text{app}$ to \pfref{istype}}
 \end{pfsteps*} 
\resetpfcounter

\item[\text{(\ref{rule:cvalidT-U-parr})}] ~
\begin{pfsteps*}
  \item $\ctau=\aceparr{\ctau_1}{\ctau_2}$ \BY{assumption}
  \item $\tau=\aparr{\tau_1}{\tau_2}$ \BY{assumption}
  \item $\cvalidT{\Delta}{\tsceneU{\uDD{\uD}{\Delta_\text{app}}}{b}}{\ctau_1}{\tau_1}$ \BY{assumption} \pflabel{cvalid-ctau1}
  \item $\cvalidT{\Delta}{\tsceneU{\uDD{\uD}{\Delta_\text{app}}}{b}}{\ctau_2}{\tau_2}$ \BY{assumption} \pflabel{cvalid-ctau2}
  \item $\istypeU{\Dcons{\Delta}{\Delta_\text{app}}}{\tau_1}$ \BY{IH on \pfref{cvalid-ctau1}} \pflabel{istype1}
  \item $\istypeU{\Dcons{\Delta}{\Delta_\text{app}}}{\tau_2}$ \BY{IH on \pfref{cvalid-ctau2}} \pflabel{istype2}
  \item $\istypeU{\Dcons{\Delta}{\Delta_\text{app}}}{\aparr{\tau_1}{\tau_2}}$ \BY{Rule (\ref{rule:istypeU-parr}) on \pfref{istype1} and \pfref{istype2}}
\end{pfsteps*}
\resetpfcounter

\item[\text{(\ref{rule:cvalidT-U-all})}] ~
\begin{pfsteps*}
  \item $\ctau=\aceall{t}{\ctau'}$ \BY{assumption}
  \item $\tau=\aall{t}{\tau'}$ \BY{assumption}
  \item $\cvalidT{\Delta, \Dhyp{t}}{\tsceneU{\uDD{\uD}{\Delta_\text{app}}}{b}}{\ctau'}{\tau'}$ \BY{assumption} \label{cvalidT}
  \item $\istypeU{\Dcons{\Delta, \Dhyp{t}}{\Delta_\text{app}}}{\tau'}$ \BY{IH on \pfref{cvalidT}} \pflabel{istypeU1}
  \item $\istypeU{\Dcons{\Delta}{\Delta_\text{app}}, \Dhyp{t}}{\tau'}$ \BY{exchange over $\Delta_\text{app}$ on \pfref{istypeU1}} \pflabel{istypeU2}
  \item $\istypeU{\Dcons{\Delta}{\Delta_\text{app}}}{\aall{t}{\tau'}}$ \BY{Rule (\ref{rule:istypeU-all}) on \pfref{istypeU2}}
\end{pfsteps*}
\resetpfcounter

% \item[{\text{(\ref{rule:cvalidT-U-rec})}}~\textbf{through}~{\text{(\ref{rule:cvalidT-U-sum})}}] These cases follow analagously, i.e. we apply the IH to or over all proto-type validation premises, apply exchange as needed, and then apply the corresponding type formation rule.
% \\
\item[\text{(\ref{rule:cvalidT-U-rec})}] ~
\begin{pfsteps*}
  \item $\ctau=\acerec{t}{\ctau'}$ \BY{assumption}
  \item $\tau=\arec{t}{\tau'}$ \BY{assumption}
  \item $\cvalidT{\Delta, \Dhyp{t}}{\tsceneU{\Delta_\text{app}}{b}}{\ctau'}{\tau'}$ \BY{assumption} \label{cvalidT}
  \item $\istypeU{\Dcons{\Delta, \Dhyp{t}}{\Delta_\text{app}}}{\tau'}$ \BY{IH on \pfref{cvalidT}} \pflabel{istypeU1}
  \item $\istypeU{\Dcons{\Delta}{\Delta_\text{app}}, \Dhyp{t}}{\tau'}$ \BY{exchange over $\Delta_\text{app}$ on \pfref{istypeU1}} \pflabel{istypeU2}
  \item $\istypeU{\Dcons{\Delta}{\Delta_\text{app}}}{\arec{t}{\tau'}}$ \BY{Rule (\ref{rule:istypeU-rec}) on \pfref{istypeU2}}
\end{pfsteps*}
\resetpfcounter

\item[\text{(\ref{rule:cvalidT-U-prod})}] ~
\begin{pfsteps*}
\item $\ctau=\aceprod{\labelset}{\mapschema{\ctau}{i}{\labelset}}$ \BY{assumption}  
\item $\tau=\aprod{\labelset}{\mapschema{\tau}{i}{\labelset}}$ \BY{assumption}
\item $\{\cvalidT{\Delta}{\tsceneU{\Delta_\text{app}}{b}}{\ctau_i}{\tau_i}\}_{i \in \labelset}$ \BY{assumption} \pflabel{cvalidT-ass}
\item $\{\istypeU{\Dcons{\Delta}{\Delta_\text{app}}}{\tau_i}\}_{i \in \labelset}$ \BY{IH over \pfref{cvalidT-ass}} \pflabel{istype}
\item $\istypeU{\Dcons{\Delta}{\Delta_\text{app}}}{\aprod{\labelset}{\mapschema{\tau}{i}{\labelset}}}$ \BY{Rule (\ref{rule:istypeU-prod}) on \pfref{istype}}
\end{pfsteps*}
\resetpfcounter 

\item[\text{(\ref{rule:cvalidT-U-sum})}] ~
\begin{pfsteps*}
\item $\ctau=\acesum{\labelset}{\mapschema{\ctau}{i}{\labelset}}$ \BY{assumption}  
\item $\tau=\asum{\labelset}{\mapschema{\tau}{i}{\labelset}}$ \BY{assumption}
\item $\{\cvalidT{\Delta}{\tsceneU{\Delta_\text{app}}{b}}{\ctau_i}{\tau_i}\}_{i \in \labelset}$ \BY{assumption} \pflabel{cvalidT-ass}
\item $\{\istypeU{\Dcons{\Delta}{\Delta_\text{app}}}{\tau_i}\}_{i \in \labelset}$ \BY{IH over \pfref{cvalidT-ass}} \pflabel{istype}
\item $\istypeU{\Dcons{\Delta}{\Delta_\text{app}}}{\asum{\labelset}{\mapschema{\tau}{i}{\labelset}}}$ \BY{Rule (\ref{rule:istypeU-sum}) on \pfref{istype}}
\end{pfsteps*}
\resetpfcounter

\item[\text{(\ref{rule:cvalidT-U-splicedt})}] ~
\begin{pfsteps*}
\item $\ctau=\acesplicedt{m}{n}$ \BY{assumption}
\item $\parseUTyp{\bsubseq{b}{m}{n}}{\utau}$ \BY{assumption}
\item $\expandsTU{\uDD{\uD}{\Delta_\text{app}}}{\utau}{\tau}$ \BY{assumption} \label{expandsTU}
\item $\Delta \cap \Delta_\text{app} = \emptyset$ \BY{assumption}
\item $\istypeU{\Delta_\text{app}}{\tau}$ \BY{Lemma \ref{lemma:type-expansion-U} on \pfref{expandsTU}}\pflabel{istype}
\item $\istypeU{\Dcons{\Delta}{\Delta_\text{app}}}{\tau}$ \BY{Lemma \ref{lemma:weakening-U} over $\Delta$ on \pfref{istype} and exchange over $\Delta$}
\end{pfsteps*}
\resetpfcounter
\end{byCases}
\end{proof}

\vspace{15px}
\begin{grayparbox}\vspace{-15px}
\subsection{Typed Pattern Expansion}\label{appendix:SES-typed-pattern-expansion}
\begin{theorem}[Typed Pattern Expansion]\label{thm:typed-pattern-expansion} ~
\begin{enumerate}
  \item If $\pExpandsSP{\uDD{\uD}{\Delta}}{\uAS{\uA}{\Phi}}{\upv}{p}{\tau}{\uGG{\uG}{\pctx}}$ then $\patType{\pctx}{p}{\tau}$.
  \item If $\cvalidP{\uGG{\uG}{\pctx}}{\pscene{\uDD{\uD}{\Delta}}{\uAP{\uA}{\Phi}}{b}}{\cpv}{p}{\tau}$ then $\patType{\pctx}{p}{\tau}$.
\end{enumerate}
\end{theorem}
\begin{proof}
  By mutual rule induction over Rules (\ref{rules:patExpands}) and Rules (\ref{rules:cvalidP-UP}).
  \begin{enumerate}
  \item We induct on the premise. In the following, let $\uDelta=\uDD{\uD}{\Delta}$ and $\upctx=\uGG{\uG}{\pctx}$ and $\uPhi=\uAP{\uA}{\Phi}$.
  \begin{byCases}
    \item[\text{(\ref{rule:patExpands-var})}] ~
      \begin{pfsteps*}
        \item $\upv=\ux$ \BY{assumption}
        \item $p=x$ \BY{assumption}
        \item $\pctx=\Ghyp{x}{\tau}$ \BY{assumption}
        \item $\patType{\Ghyp{x}{\tau}}{x}{\tau}$ \BY{Rule (\ref{rule:patType-var})}
      \end{pfsteps*}
      \resetpfcounter
    \item[\text{(\ref{rule:patExpands-wild})}] ~
      \begin{pfsteps*}
        \item $p=\aewildp$ \BY{assumption}
        \item $\pctx = \emptyset$ \BY{assumption}
        \item $\patType{\emptyset}{\aewildp}{\tau}$ \BY{Rule (\ref{rule:patType-wild})}
      \end{pfsteps*}
      \resetpfcounter
    \item[\text{(\ref{rule:patExpands-fold})}] ~
      \begin{pfsteps*}
        \item $\upv=\foldp{\upv'}$ \BY{assumption}
        \item $p=\aefoldp{p'}$ \BY{assumption}
        \item $\tau=\arec{t}{\tau'}$ \BY{assumption}
        %\item $\uptsmenv{\Delta}{\Phi}$ \BY{assumption} \pflabel{env}
        \item $\patExpands{\upctx}{\uPhi}{\upv'}{p'}{[\arec{t}{\tau'}/t]\tau'}$ \BY{assumption} \pflabel{patExpands}
        \item $\patType{\pctx}{p'}{[\arec{t}{\tau'}/t]\tau'}$ \BY{IH, part 1 on \pfref{patExpands}} \pflabel{patType}
        \item $\patType{\pctx}{\aefoldp{p'}}{\arec{t}{\tau'}}$ \BY{Rule (\ref{rule:patType-fold}) on \pfref{patType}}
      \end{pfsteps*}
      \resetpfcounter
    \item[\text{(\ref{rule:patExpands-tpl})}] ~
      \begin{pfsteps*}
        \item $\upv=\tplp{\mapschema{\upv}{i}{\labelset}}$ \BY{assumption}
        \item $p=\aetplp{\labelset}{\mapschema{p}{i}{\labelset}}$ \BY{assumption}
        \item $\tau=\aprod{\labelset}{\mapschema{\tau}{i}{\labelset}}$ \BY{assumption}
        \item $\{\patExpands{\uGG{\uG_i}{\pctx_i}}{\uPhi}{\upv_i}{p_i}{\tau_i}\}_{i \in \labelset}$ \BY{assumption} \pflabel{patExpands}
        \item $\pctx = \Gconsi{i \in \labelset}{\pctx_i}$ \BY{assumption}
        %\item $\uptsmenv{\Delta}{\Phi}$ \BY{assumption} \pflabel{env}
        \item $\{\patType{\pctx_i}{p_i}{\tau_i}\}_{i \in \labelset}$ \BY{IH, part 1 over \pfref{patExpands}}\pflabel{patType}
        \item $\patType{\Gconsi{i \in \labelset}{\pctx_i}}{\aetplp{\labelset}{\mapschema{p}{i}{\labelset}}}{\aprod{\labelset}{\mapschema{\tau}{i}{\labelset}}}$ \BY{Rule (\ref{rule:patType-tpl}) on \pfref{patType}}
      \end{pfsteps*}
      \resetpfcounter
    \item[\text{(\ref{rule:patExpands-in})}] ~
      \begin{pfsteps*}
        \item $\upv=\injp{\ell}{\upv'}$ \BY{assumption}
        \item $p=\aeinjp{\ell}{p'}$ \BY{assumption}
        \item $\tau=\asum{\labelset, \ell}{\mapschema{\tau}{i}{\labelset}; \mapitem{\ell}{\tau'}}$ \BY{assumption}
        \item $\patExpands{\upctx}{\uPhi}{\upv'}{p'}{\tau'}$ \BY{assumption} \pflabel{patExpands}
%        \item $\uptsmenv{\Delta}{\Phi}$ \BY{assumption} \pflabel{env}
        \item $\patType{\pctx}{p'}{\tau'}$ \BY{IH, part 1 on \pfref{patExpands}} \pflabel{patType}
        \item $\patType{\pctx}{\aeinjp{\ell}{p'}}{\asum{\labelset, \ell}{\mapschema{\tau}{i}{\labelset}; \mapitem{\ell}{\tau'}}}$ \BY{Rule (\ref{rule:patType-inj}) on \pfref{patType}}
      \end{pfsteps*}
      \resetpfcounter
    \item[\text{(\ref{rule:patExpands-apuptsm})}] ~
      \begin{pfsteps*}
        \item $\upv=\utsmap{\tsmv}{b}$ \BY{assumption}
        \item $\uA=\uA', \vExpands{\tsmv}{x}$ \BY{assumption}
        \item $\Phi=\Phi', \xuptsmbnd{a}{\tau}{\eparse}$ \BY{assumption}
        \item $\encodeBody{b}{\ebody}$ \BY{assumption}
        \item $\evalU{\eparse(\ebody)}{\aein{\mathtt{SuccessP}}{\ecand}}$ \BY{assumption}
        \item $\decodeCEPat{\ecand}{\cpv}$ \BY{assumption}
        \item $\cvalidP{\uGG{\uG}{\pctx}}{\pscene{\uDelta}{\uAP{\uA}{\Phi}}{b}}{\cpv}{p}{\tau}$ \BY{assumption} \pflabel{cvalidP}
%        \item $\uptsmenv{\Delta}{\Phi', \xuptsmbnd{a}{\tau}{\eparse}}$ \BY{assumption} \pflabel{env}
        \item $\patType{\pctx}{p}{\tau}$ \BY{IH, part 2 on \pfref{cvalidP}}
      \end{pfsteps*}
      \resetpfcounter
  \end{byCases}

  \item We induct on the premise. In the following, let $\upctx=\uGG{\uG}{\pctx}$ and $\uDelta = \uDD{\uD}{\Delta}$ and $\uPhi=\uAP{\uA}{\Phi}$.
  \begin{byCases}
    \item[\text{(\ref{rule:cvalidP-UP-wild})}] ~
      \begin{pfsteps*}
        \item $p=\aewildp$ \BY{assumption}
        \item $\pctx=\emptyset$ \BY{assumption}
        \item $\patType{\emptyset}{\aewildp}{\tau}$ \BY{Rule (\ref{rule:patType-wild})}
      \end{pfsteps*}
      \resetpfcounter
    \item[\text{(\ref{rule:cvalidP-UP-fold})}] ~
      \begin{pfsteps*}
        \item $\cpv=\acefoldp{\cpv'}$ \BY{assumption}
        \item $p=\aefoldp{p'}$ \BY{assumption}
        \item $\tau=\arec{t}{\tau'}$ \BY{assumption}
        % \item $\uptsmenv{\Delta}{\Phi}$ \BY{assumption} \pflabel{env}
        \item $\cvalidP{\upctx}{\pscene{\uDelta}{\uPhi}{b}}{\cpv'}{p'}{[\arec{t}{\tau'}/t]\tau'}$ \BY{assumption} \pflabel{cvalidP}
        \item $\patType{\pctx}{p'}{[\arec{t}{\tau'}/t]\tau'}$ \BY{IH, part 2 on \pfref{cvalidP}} \pflabel{patType}
        \item $\patType{\pctx}{\aefoldp{p'}}{\arec{t}{\tau'}}$ \BY{Rule (\ref{rule:patType-fold}) on \pfref{patType}}
      \end{pfsteps*}
      \resetpfcounter
    \item[\text{(\ref{rule:cvalidP-UP-tpl})}] ~
      \begin{pfsteps*}
        \item $\cpv=\acetplp{\labelset}{\mapschema{\cpv}{i}{\labelset}}$ \BY{assumption}
        \item $p=\aetplp{\labelset}{\mapschema{p}{i}{\labelset}}$ \BY{assumption}
        \item $\tau=\aprod{\labelset}{\mapschema{\tau}{i}{\labelset}}$ \BY{assumption}
        \item $\{\cvalidP{\uGG{\uG_i}{\pctx_i}}{\pscene{\uDelta}{\uPhi}{b}}{\cpv_i}{p_i}{\tau_i}\}_{i \in \labelset}$ \BY{assumption} \pflabel{cvalidP}
        \item $\pctx = \Gconsi{i \in \labelset}{\pctx_i}$ \BY{assumption}
        %\item $\uptsmenv{\Delta}{\Phi}$ \BY{assumption} \pflabel{env}
        \item $\{\patType{\pctx_i}{p_i}{\tau_i}\}_{i \in \labelset}$ \BY{IH, part 2 over \pfref{cvalidP}}\pflabel{patType}
        \item $\patType{\Gconsi{i \in \labelset}{\pctx_i}}{\aetplp{\labelset}{\mapschema{p}{i}{\labelset}}}{\aprod{\labelset}{\mapschema{\tau}{i}{\labelset}}}$ \BY{Rule (\ref{rule:patType-tpl}) on \pfref{patType}}
      \end{pfsteps*}
      \resetpfcounter
    \item[\text{(\ref{rule:cvalidP-UP-in})}] ~
      \begin{pfsteps*}
        \item $\cpv=\aceinjp{\ell}{\cpv'}$ \BY{assumption}
        \item $p=\aeinjp{\ell}{p'}$ \BY{assumption}
        \item $\tau=\asum{\labelset, \ell}{\mapschema{\tau}{i}{\labelset}; \mapitem{\ell}{\tau'}}$ \BY{assumption}
        \item $\cvalidP{\upctx}{\pscene{\uDelta}{\uPhi}{b}}{\cpv'}{p'}{\tau'}$ \BY{assumption} \pflabel{cvalidP}
%        \item $\uptsmenv{\Delta}{\Phi}$ \BY{assumption} \pflabel{env}
        \item $\patType{\pctx}{p'}{\tau'}$ \BY{IH, part 2 on \pfref{cvalidP}} \pflabel{patType}
        \item $\patType{\pctx}{\aeinjp{\ell}{p'}}{\asum{\labelset, \ell}{\mapschema{\tau}{i}{\labelset}; \mapitem{\ell}{\tau'}}}$ \BY{Rule (\ref{rule:patType-inj}) on \pfref{patType}}
      \end{pfsteps*}
      \resetpfcounter
    \item[\text{(\ref{rule:cvalidP-UP-spliced})}] ~
      \begin{pfsteps*}
        \item $\cpv=\acesplicedp{m}{n}{\ctau}$ \BY{assumption}
        \item $\cvalidT{\emptyset}{\tsceneUP{\uDelta}{b}}{\ctau}{\tau}$ \BY{assumption}
        \item $\parseUExp{\bsubseq{b}{m}{n}}{\upv}$ \BY{assumption}
        \item $\patExpands{\upctx}{\uPhi}{\upv}{p}{\tau}$ \BY{assumption} \pflabel{patExpands}
        \item $\patType{\pctx}{p}{\tau}$ \BY{IH, part 1 on \pfref{patExpands}}
      \end{pfsteps*}
      \resetpfcounter
  \end{byCases}
  \end{enumerate}
The mutual induction can be shown to be well-founded by showing that the following numeric metric on the judgements that we induct on is decreasing:
\begin{align*}
\sizeof{\patExpands{\upctx}{\uPhi}{\upv}{p}{\tau}} & = \sizeof{\upv}\\
\sizeof{{\cvalidP{\upctx}{\pscene{\uDelta}{\uPhi}{b}}{\cpv}{p}{\tau}}} & = \sizeof{b}
\end{align*}
where $\sizeof{b}$ is the length of $b$ and $\sizeof{\upv}$ is the sum of the lengths of the literal bodies in $\upv$, as defined in Sec. \ref{appendix:SES-syntax}.

The only case in the proof of part 1 that invokes part 2 is Case (\ref{rule:patExpands-apuptsm}). There, we have that the metric remains stable: \begin{align*}
 & \sizeof{\patExpands{\upctx}{\uPhi}{\utsmap{\tsmv}{b}}{p}{\tau}}\\
=& \sizeof{{\cvalidP{\upctx}{\pscene{\uDelta}{\uPhi}{b}}{\cpv}{p}{\tau}}}\\
=&\sizeof{b}\end{align*}

The only case in the proof of part 2 that invokes part 1 is Case (\ref{rule:cvalidP-UP-spliced}). There, we have that $\parseUPat{\bsubseq{b}{m}{n}}{\upv}$ and the IH is applied to the judgement $\patExpands{\upctx}{\uPhi}{\upv}{p}{\tau}$. Because the metric is stable when passing from part 1 to part 2, we must have that it is strictly decreasing in the other direction:
\[\sizeof{\patExpands{\upctx}{\uPhi}{\upv}{p}{\tau}} < \sizeof{{\cvalidP{\upctx}{\pscene{\uDelta}{\uPhi}{b}}{\acesplicedp{m}{n}{\ctau}}{p}{\tau}}}\]
i.e. by the definitions above, 
\[\sizeof{\upv} < \sizeof{b}\]

This is established by appeal to Condition \ref{condition:body-subsequences}, which states that subsequences of $b$ are no longer than $b$, and the Condition \ref{condition:pattern-parsing}, which states that an unexpanded pattern constructed by parsing a textual sequence $b$ is strictly smaller, as measured by the metric defined above, than the length of $b$, because some characters must necessarily be used to apply the pattern TLM and delimit each literal body. Combining Conditions \ref{condition:body-subsequences} and \ref{condition:pattern-parsing}, we have that $\sizeof{\upv} < \sizeof{b}$ as needed.
\end{proof}

\end{grayparbox}
\subsection{Typed Expression Expansion}\label{appendix:SES-typed-expression-expansion-metatheory}
\begin{theorem}[Typed Expansion (Strong)]\label{thm:typed-expansion-full-U} ~
\begin{enumerate}
  \item \begin{enumerate}
    \item If $\expandsSG{\uDD{\uD}{\Delta}}{\uGG{\uG}{\Gamma}}{\uPsi}{\uPhi}{\ue}{e}{\tau}$ then $\hastypeU{\Delta}{\Gamma}{e}{\tau}$.
    \item \graytxtbox{If $\ruleExpands{\uDD{\uD}{\Delta}}{\uGG{\uG}{\Gamma}}{\uPsi}{\uPhi}{\urv}{r}{\tau}{\tau'}$  then $\ruleType{\Delta}{\Gamma}{r}{\tau}{\tau'}$.}
  \end{enumerate}
  \item \begin{enumerate}
    \item If $\cvalidE{\Delta}{\Gamma}{\esceneSG{\uDD{\uD}{\Delta_\text{app}}}{\uGG{\uG}{\Gamma_\text{app}}}{\uPsi}{\uPhi}{b}}{\ce}{e}{\tau}$ and $\Delta \cap \Delta_\text{app}=\emptyset$ and $\domof{\Gamma} \cap \domof{\Gamma_\text{app}}=\emptyset$ then $\hastypeU{\Dcons{\Delta}{\Delta_\text{app}}}{\Gcons{\Gamma}{\Gamma_\text{app}}}{e}{\tau}$. 
    \item \begin{grayparbox}If $\cvalidR{\Delta}{\Gamma}{\esceneUP{\uDD{\uD}{\Delta_\text{app}}}{\uGG{\uG}{\Gamma_\text{app}}}{\uPsi}{\uPhi}{b}}{\crv}{r}{\tau}{\tau'}$ and $\Delta \cap \Delta_\text{app}=\emptyset$ and $\domof{\Gamma} \cap \domof{\Gamma_\text{app}}=\emptyset$ then $\ruleType{\Dcons{\Delta}{\Delta_\text{app}}}{\Gcons{\Gamma}{\Gamma_\text{app}}}{r}{\tau}{\tau'}$.\end{grayparbox}
  \end{enumerate}
\end{enumerate}
\end{theorem}
\begin{proof}
By mutual rule induction over Rules (\ref{rules:expandsU}), \graytxtbox{Rule (\ref{rule:ruleExpands}),} Rules (\ref{rules:cvalidE-U}) \graytxtbox{and Rule (\ref{rule:cvalidR-UP})}.

\begin{enumerate}
\item In the following, let $\uDelta=\uDD{\uD}{\Delta}$ and $\uGamma=\uGG{\uG}{\Gamma}$.
  \begin{enumerate}
  \item 
  \begin{byCases} \item[\text{(\ref{rule:expandsU-var})}] ~
\begin{pfsteps}
  \item \ue=\ux \BY{assumption}
  \item e=x \BY{assumption}
  \item \Gamma=\Gamma', \Ghyp{x}{\tau} \BY{assumption}
  \item \hastypeU{\Delta}{\Gamma', \Ghyp{x}{\tau}}{x}{\tau} \BY{Rule (\ref{rule:hastypeU-var})}
\end{pfsteps}
\resetpfcounter

\item[\text{(\ref{rule:expandsU-asc})}] ~
\begin{pfsteps}
  \item \ue=\asc{\ue'}{\utau} \BY{assumption}
  \item \expandsTU{\uDelta}{\utau}{\tau} \BY{assumption} \pflabel{expandsTU}
  \item \expandsSG{\uDelta}{\uGamma}{\uPsi}{\uPhi}{\ue'}{e}{\tau} \BY{assumption} \pflabel{expandsSG}
  \item \hastypeU{\Delta}{\Gamma}{e}{\tau} \BY{IH, part 1(a) on \pfref{expandsSG}}
\end{pfsteps}
\resetpfcounter

\item[\text{(\ref{rule:expandsU-letsyn})}] ~
\begin{pfsteps}
  \item \ue=\letsyn{\ux}{\ue_1}{\ue_2} \BY{assumption}
  \item e =   \aeap{\aelam{\tau_1}{x}{e_2}}{e_1} \BY{assumption}
  \item     \expandsSG{\uDelta}{\uGamma}{\uPsi}{\uPhi}{\ue_1}{e_1}{\tau_1} \BY{assumption} \pflabel{expandsSG1}
  \item     \expandsSG{\uDelta}{\uGamma, \uGhyp{\ux}{x}{\tau_1}}{\uPsi}{\uPhi}{\ue_2}{e_2}{\tau} \BY{assumption} \pflabel{expandsSG2}
  \item \hastypeU{\Delta}{\Gamma}{e_1}{\tau_1} \BY{IH, part 1(a) on \pfref{expandsSG1}} \pflabel{hastype1}
  \item \hastypeU{\Delta}{\Gamma, x : \tau}{e_2}{\tau} \BY{IH, part 1(a) on \pfref{expandsSG2}} \pflabel{hastype2}
  \item \hastypeU{\Delta}{\Gamma}{\aelam{\tau_1}{x}{e_2}}{\aparr{\tau_1}{\tau}} \BY{Rule (\ref{rule:hastypeU-lam}) on \pfref{hastype2}} \pflabel{hastype3}
  \item \hastypeU{\Delta}{\Gamma}{\aeap{\aelam{\tau_1}{x}{e_2}}{e_1}}{\tau} \BY{Rule (\ref{rule:hastypeU-ap}) on \pfref{hastype3} and \pfref{hastype1}}
\end{pfsteps}
\resetpfcounter

\item[\text{(\ref{rule:expandsU-lam})}] ~
\begin{pfsteps}
  \item \ue=\lam{\ux}{\utau_1}{\ue'} \BY{assumption}
  \item e=\aelam{\tau_1}{x}{e'} \BY{assumption}
  \item \tau=\aparr{\tau_1}{\tau_2} \BY{assumption}
  \item \expandsTU{\uDelta}{\utau_1}{\tau_1} \BY{assumption} \pflabel{istype}
  \item \expandsSG{\uDelta}{\uGamma, \uGhyp{\ux}{x}{\tau_1}}{\uPsi}{\uPhi}{\ue'}{e'}{\tau_2} \BY{assumption} \pflabel{expandsU}
%  \item \uetsmenv{\Delta}{\Psi} \BY{assumption} \pflabel{uetsmenv}
  \item \istypeU{\Delta}{\tau_1} \BY{Lemma \ref{lemma:type-expansion-U} on \pfref{istype}} \pflabel{istype2}
  \item \hastypeU{\Delta}{\Gamma, \Ghyp{x}{\tau_1}}{e'}{\tau_2} \BY{IH, part 1(a) on \pfref{expandsU}} \pflabel{hastypeU}
  \item \hastypeU{\Delta}{\Gamma}{\aelam{\tau_1}{x}{e'}}{\aparr{\tau_1}{\tau_2}} \BY{Rule (\ref{rule:hastypeU-lam}) on \pfref{istype2} and \pfref{hastypeU}}
\end{pfsteps}
\resetpfcounter

\item[\text{(\ref{rule:expandsU-ap})}] ~
\begin{pfsteps}
  \item \ue=\ap{\ue_1}{\ue_2} \BY{assumption}
  \item e=\aeap{e_1}{e_2} \BY{assumption}
  \item \expandsSG{\uDelta}{\uGamma}{\uPsi}{\uPhi}{\ue_1}{e_1}{\aparr{\tau_2}{\tau}} \BY{assumption}\pflabel{expandsU1}
  \item \expandsSG{\uDelta}{\uGamma}{\uPsi}{\uPhi}{\ue_2}{e_2}{\tau_2} \BY{assumption}\pflabel{expandsU2}
%  \item \uetsmenv{\Delta}{\Psi} \BY{assumption} \pflabel{uetsmenv}
  \item \hastypeU{\Delta}{\Gamma}{e_1}{\aparr{\tau_2}{\tau}} \BY{IH, part 1(a) on \pfref{expandsU1}}\pflabel{hastypeU1}
  \item \hastypeU{\Delta}{\Gamma}{e_2}{\tau_2} \BY{IH, part 1(a) on \pfref{expandsU2}}\pflabel{hastypeU2}
  \item \hastypeU{\Delta}{\Gamma}{\aeap{e_1}{e_2}}{\tau} \BY{Rule (\ref{rule:hastypeU-ap}) on \pfref{hastypeU1} and \pfref{hastypeU2}}
\end{pfsteps}
\resetpfcounter

\item[\text{(\ref{rule:expandsU-tlam})}~\textbf{through}~\text{(\ref{rule:expandsU-case})}] These cases follow analagously, i.e. we apply Lemma \ref{lemma:type-expansion-U} to or over the type expansion premises and the IH part 1(a) to or over the typed expression expansion premises and then apply the corresponding typing rule in Rules (\ref{rule:hastypeU-tlam}) through (\ref{rule:hastypeU-case}).
\\
\item[\text{(\ref{rule:expandsU-syntax})}] ~ 
\begin{pfsteps}
  \item \ue=\uesyntaxqr{\tsmv}{\utau'}{\eparse}{\ue'} \BY{assumption}
  \item \expandsTU{\uDelta}{\utau'}{\tau'} \BY{assumption} \pflabel{expandsTU}
  \item \expandsUP{\uDelta}{\uGamma}{\uPsi}{\uPhi}{\uedep}{\edep}{\taudep} \BY{assumption} \pflabel{expands}
 \item \hastypeU{\emptyset}{\emptyset}{\eparse}{\aparr{\tBody}{\tParseResultExp}} \BY{assumption}\pflabel{eparse}
  \item \expandsSG{\uDelta}{\uGG{\uG}{\Gamma, x : \taudep}}{\uPsi, \uShyp{\tsmv}{x}{\tau'}{\eparse}}{\uPhi}{\ue'}{e'}{\tau} \BY{assumption}\pflabel{expandsU}
%  \item \uetsmenv{\Delta}{\Psi} \BY{assumption}\pflabel{uetsmenv1}
 \item \istypeU{\Delta}{\tau'} \BY{Lemma \ref{lemma:type-expansion-U} to \pfref{expandsTU}} \pflabel{istype}
%  \item \uetsmenv{\Delta}{\Psi, \xuetsmbnd{\tsmv}{\tau'}{\eparse}} \BY{Definition \ref{def:seTLM-def-ctx-formation} on \pfref{uetsmenv1}, \pfref{istype} and \pfref{eparse}}\pflabel{uetsmenv3}
 \item \istypeU{\Delta}{\taudep} \BY{Lemma \ref{lemma:type-expansion-U} to \pfref{expands}} \pflabel{istype2}
     \item \hastypeU{\Delta}{\Gamma, x : \taudep}{e'}{\tau} \BY{IH, part 1(a) on \pfref{expandsU}} \label{hastype34}
     \item \hastypeU{\Delta}{\Gamma}{\edep}{\taudep} \BY{IH, part 1(a) on \pfref{expands}} \pflabel{deptype}
  \item   e = \aeap{\aelam{\taudep}{x}{e'}}{\edep} \BY{assumption}
  \item \hastypeU{\Delta}{\Gamma}{e}{\tau} \BY{Rule (\ref{rule:hastypeU-ap}) and Rule (\ref{rule:hastypeU-lam}) with \pfref{hastype34} and \pfref{istype2} and \pfref{deptype}}
\end{pfsteps}
\resetpfcounter 

\item[\text{(\ref{rule:expandsU-tsmap})}] ~ 
\begin{pfsteps}
  \item \ue=\utsmap{\tsmv}{b} \BY{assumption}
  \item \uA = \uA', \vExpands{\tsmv}{x} \BY{assumption}
  \item \Psi=\Psi', \xuetsmbnd{x}{\tau}{\eparse} \BY{assumption}
  \item \Gamma = \Gamma', x : \taudep \BY{assumption} \pflabel{gamma}
    \item e = \aeap{e'}{x} \BY{assumption}
  \item \encodeBody{b}{\ebody} \BY{assumption}
  \item \evalU{\eparse(\ebody)}{\aein{\mathtt{SuccessE}}{\ecand}} \BY{assumption}
  \item \decodeCondE{\ecand}{\ce} \BY{assumption}
  \item \cvalidE{\emptyset}{\emptyset}{\esceneSG{\uDelta}{\uGamma}{\uPsi}{\uPhi}{b}}{\ce}{e'}{\aparr{\taudep}{\tau}} \BY{assumption}\pflabel{cvalidE}
%  \item \uetsmenv{\Delta}{\Psi} \BY{assumption} \pflabel{uetsmenv}
  \item \emptyset \cap \Delta = \emptyset \BY{finite set intersection} \pflabel{delta-cap}
  \item {\emptyset} \cap \domof{\Gamma} = \emptyset \BY{finite set intersection} \pflabel{gamma-cap}
  \item \hastypeU{\emptyset \cup \Delta}{\emptyset \cup \Gamma}{e'}{\aparr{\taudep}{\tau}} \BY{IH, part 2(a) on \pfref{cvalidE}, \pfref{delta-cap}, and \pfref{gamma-cap}} \pflabel{penultimate}
  \item \hastypeU{\Delta}{\Gamma}{e'}{\aparr{\taudep}{\tau}} \BY{finite set and finite function identity over \pfref{penultimate}} \pflabel{cc1}
  \item \hastypeU{\Delta}{\Gamma}{x}{\taudep} \BY{Rule (\ref{rule:hastypeU-var})} \pflabel{cc2}
  \item \hastypeU{\Delta}{\Gamma}{e}{\tau} \BY{Rule (\ref{rule:hastypeU-ap}) on \pfref{cc1} and \pfref{cc2}}
\end{pfsteps}
\resetpfcounter
\end{byCases}
\end{enumerate}
\begin{grayparbox}
\begin{enumerate}
\item[\hphantom{(a)}] \begin{byCases}
    \item[\text{(\ref{rule:expandsU-match})}] ~
      \begin{pfsteps*}
        \item $\ue=\matchwith{\ue'}{\seqschemaX{\urv}}$ \BY{assumption}
        \item $e=\aematchwith{n}{e'}{\seqschemaX{r}}$ \BY{assumption}
        \item $\expandsUP{\uDelta}{\uGamma}{\uPsi}{\uPhi}{\ue'}{e'}{\tau'}$ \BY{assumption} \pflabel{expandsUP}
        % \item $\istypeU{\Delta}{\tau}$ \BY{assumption}\pflabel{istype}
        % \item $\expandsTU{\uDelta}{\utau}{\tau}$ \BY{assumption} \pflabel{expandsTU}
        \item $\{\ruleExpands{\uDelta}{\uGamma}{\uPsi}{\uPhi}{\urv_i}{r_i}{\tau'}{\tau}\}_{1 \leq i \leq n}$ \BY{assumption}\pflabel{ruleExpands}
        \item $\hastypeU{\Delta}{\Gamma}{e'}{\tau'}$ \BY{IH, part 1(a) on \pfref{expandsUP}}\pflabel{hasType}
        \item $\{\ruleType{\Delta}{\Gamma}{r_i}{\tau'}{\tau}\}_{1 \leq i \leq n}$ \BY{IH, part 1(b) over \pfref{ruleExpands}}\pflabel{ruleType}
        \item $\hastypeU{\Delta}{\Gamma}{\aematchwith{n}{e'}{\seqschemaX{r}}}{\tau}$ \BY{Rule (\ref{rule:hastypeUP-match}) on \pfref{hasType} and \pfref{ruleType}}
      \end{pfsteps*}
      \resetpfcounter

    \item[\text{(\ref{rule:expandsU-defuptsm})}] ~
      \begin{pfsteps}
          \item \ue=\usyntaxup{\tsmv}{\utau'}{\eparse}{\ue'} \BY{assumption}
          \item \expandsTU{\uDelta}{\utau'}{\tau'} \BY{assumption} \pflabel{expandsTU}
         \item \hastypeU{\emptyset}{\emptyset}{\eparse}{\aparr{\tBody}{\tParseResultExp}} \BY{assumption}\pflabel{eparse}
          \item \expandsUP{\uDelta}{\uGamma}{\uPsi}{\uPhi, \uPhyp{\tsmv}{x}{\tau'}{\eparse}}{\ue'}{e}{\tau} \BY{assumption}\pflabel{expandsU}
        %  \item \uetsmenv{\Delta}{\Psi} \BY{assumption}\pflabel{uetsmenv1}
         \item \istypeU{\Delta}{\tau'} \BY{Lemma \ref{lemma:type-expansion-U} to \pfref{expandsTU}} \pflabel{istype}
        %  \item \uetsmenv{\Delta}{\Psi, \xuetsmbnd{\tsmv}{\tau'}{\eparse}} \BY{Definition \ref{def:seTLM-def-ctx-formation} on \pfref{uetsmenv1}, \pfref{istype} and \pfref{eparse}}\pflabel{uetsmenv3}
          \item \hastypeU{\Delta}{\Gamma}{e}{\tau} \BY{IH, part 1(a) on \pfref{expandsU}}
        \end{pfsteps}
        \resetpfcounter 
  \end{byCases}
  \end{enumerate}
  \end{grayparbox}
  \vspace{-4px}\begin{grayparbox}\vspace{4px}
  \begin{enumerate}
  \item[(b)] \begin{byCases}
    \item[\text{(\ref{rule:ruleExpands})}] ~
      \begin{pfsteps*}
        \item $\urv=\matchrule{\upv}{\ue}$ \BY{assumption}
        \item $r=\aematchrule{p}{e}$ \BY{assumption}
        \item $\patExpands{\uGG{\uA'}{\pctx}}{\uPhi}{\upv}{p}{\tau}$ \BY{assumption} \pflabel{patExpands}
        \item $\expandsUP{\uDelta}{\uGG{{\uA}\uplus{\uA'}}{\Gcons{\Gamma}{\pctx}}}{\uPsi}{\uPhi}{\ue}{e}{\tau'}$ \BY{assumption} \pflabel{expandsUP}
        \item $\patType{\pctx}{p}{\tau}$ \BY{Theorem \ref{thm:typed-pattern-expansion}, part 1 on \pfref{patExpands}}\pflabel{patType}
        \item $\hastypeU{\Delta}{\Gcons{\Gamma}{\pctx}}{e}{\tau'}$ \BY{IH, part 1(a) on \pfref{expandsUP}} \pflabel{hasType}
        \item $\ruleType{\Delta}{\Gamma}{\aematchrule{p}{e}}{\tau}{\tau'}$ \BY{Rule (\ref{rule:ruleType}) on \pfref{patType} and \pfref{hasType}}
      \end{pfsteps*}
      \resetpfcounter
  \end{byCases}
  \end{enumerate}
  \end{grayparbox}

\item In the following, let $\uDelta=\uDD{\uD}{\Delta_\text{app}}$ and $\uGamma=\uGG{\uG}{\Gamma_\text{app}}$. \begin{enumerate}
  \item 
  \begin{byCases}
    \item[\text{(\ref{rule:cvalidE-U-var})}] ~
\begin{pfsteps*}
  \item $\ce=x$ \BY{assumption}
  \item $e=x$ \BY{assumption}
  \item $\Gamma=\Gamma', \Ghyp{x}{\tau}$ \BY{assumption}
  \item $\hastypeU{\Dcons{\Delta}{\Delta_\text{app}}}{\Gamma', \Ghyp{x}{\tau}}{x}{\tau}$ \BY{Rule (\ref{rule:hastypeU-var})} \pflabel{hastypeU}
  \item $\hastypeU{\Dcons{\Delta}{\Delta_\text{app}}}{\Gcons{\Gamma', \Ghyp{x}{\tau}}{\Gamma_\text{app}}}{x}{\tau}$ \BY{Lemma \ref{lemma:weakening-U} over $\Gamma_\text{app}$ to \pfref{hastypeU}}
\end{pfsteps*}
\resetpfcounter

\item[\text{(\ref{rule:cvalidE-U-lam})}] ~
\begin{pfsteps*}
  \item $\ce=\acelam{\ctau_1}{x}{\ce'}$ \BY{assumption}
  \item $e=\aelam{\tau_1}{x}{e'}$ \BY{assumption}
  \item $\tau=\aparr{\tau_1}{\tau_2}$ \BY{assumption}
  \item $\cvalidT{\Delta}{\tsceneU{\uDelta_\text{app}}{b}}{\ctau_1}{\tau_1}$ \BY{assumption} \pflabel{cvalidT}
  \item $\cvalidE{\Delta}{\Gamma, \Ghyp{x}{\tau_1}}{\esceneSG{\uDelta_\text{app}}{\uGamma_\text{app}}{\uPsi}{\uPhi}{b}}{\ce'}{e'}{\tau_2}$ \BY{assumption} \pflabel{cvalidE}
%  \item $\uetsmenv{\Delta_\text{app}}{\Psi}$ \BY{assumption} \pflabel{uetsmenv}
  \item $\Delta \cap \Delta_\text{app}=\emptyset$ \BY{assumption} \pflabel{delta-disjoint}
  \item $\domof{\Gamma} \cap \domof{\Gamma_\text{app}}=\emptyset$ \BY{assumption} \pflabel{gamma-disjoint}
  \item $x \notin \domof{\Gamma_\text{app}}$ \BY{identification convention} \pflabel{x-fresh}
  \item $\domof{\Gamma, x : \tau_1} \cap \domof{\Gamma_\text{app}}=\emptyset$ \BY{\pfref{gamma-disjoint} and \pfref{x-fresh}} \pflabel{gamma-disjoint2}
  \item $\istypeU{\Dcons{\Delta}{\Delta_\text{app}}}{\tau_1}$ \BY{Lemma \ref{lemma:candidate-expansion-type-validation} on \pfref{cvalidT} and \pfref{delta-disjoint}} \pflabel{istype}
  \item $\hastypeU{\Dcons{\Delta}{\Delta_\text{app}}}{\Gcons{\Gamma, \Ghyp{x}{\tau_1}}{\Gamma_\text{app}}}{e'}{\tau_2}$ \BY{IH, part 2(a) on \pfref{cvalidE}, \pfref{delta-disjoint} and \pfref{gamma-disjoint2}} \pflabel{hastype1}
  \item $\hastypeU{\Dcons{\Delta}{\Delta_\text{app}}}{\Gcons{\Gamma}{\Gamma_\text{app}}, \Ghyp{x}{\tau_1}}{e'}{\tau_2}$ \BY{exchange over $\Gamma_\text{app}$ on \pfref{hastype1}} \pflabel{hastype2}
  \item $\hastypeU{\Dcons{\Delta}{\Delta_\text{app}}}{\Gcons{\Gamma}{\Gamma_\text{app}}}{\aelam{\tau_1}{x}{e'}}{\aparr{\tau_1}{\tau_2}}$ \BY{Rule (\ref{rule:hastypeU-lam}) on \pfref{istype} and \pfref{hastype2}}
\end{pfsteps*}
\resetpfcounter

\item[\text{(\ref{rule:cvalidE-U-ap})}] ~
\begin{pfsteps*}
  \item $\ce=\aceap{\ce_1}{\ce_2}$ \BY{assumption}
  \item $e=\aeap{e_1}{e_2}$ \BY{assumption}
  \item $\cvalidE{\Delta}{\Gamma}{\esceneSG{\uDelta_\text{app}}{\uGamma_\text{app}}{\uPsi}{\uPhi}{b}}{\ce_1}{e_1}{\aparr{\tau_2}{\tau}}$ \BY{assumption} \pflabel{cvalidE1}
  \item $\cvalidE{\Delta}{\Gamma}{\esceneSG{\uDelta_\text{app}}{\uGamma_\text{app}}{\uPsi}{\uPhi}{b}}{\ce_2}{e_2}{\tau_2}$ \BY{assumption} \pflabel{cvalidE2}
%  \item $\uetsmenv{\Delta_\text{app}}{\Psi}$ \BY{assumption} \pflabel{uetsmenv}
  \item $\Delta \cap \Delta_\text{app}=\emptyset$ \BY{assumption} \pflabel{delta-disjoint}
  \item $\domof{\Gamma} \cap \domof{\Gamma_\text{app}}=\emptyset$ \BY{assumption} \pflabel{gamma-disjoint}
  \item $\hastypeU{\Dcons{\Delta}{\Delta_\text{app}}}{\Gcons{\Gamma}{\Gamma_\text{app}}}{e_1}{\aparr{\tau_2}{\tau}}$ \BY{IH, part 2(a) on \pfref{cvalidE1}, \pfref{delta-disjoint} and \pfref{gamma-disjoint}} \pflabel{hastypeU1}
  \item $\hastypeU{\Dcons{\Delta}{\Delta_\text{app}}}{\Gcons{\Gamma}{\Gamma_\text{app}}}{e_2}{\tau_2}$ \BY{IH, part 2(a) on \pfref{cvalidE2}, \pfref{delta-disjoint} and \pfref{gamma-disjoint}} \pflabel{hastypeU2}
  \item $\hastypeU{\Dcons{\Delta}{\Delta_\text{app}}}{\Gcons{\Gamma}{\Gamma_\text{app}}}{\aeap{e_1}{e_2}}{\tau}$ \BY{Rule (\ref{rule:hastypeU-ap}) on \pfref{hastypeU1} and \pfref{hastypeU2}}
\end{pfsteps*}
\resetpfcounter

\item[\text{(\ref{rule:cvalidE-U-tlam})}] ~
\begin{pfsteps}
  \item \ce=\acetlam{t}{\ce'} \BY{assumption}
  \item e = \aetlam{t}{e'} \BY{assumption}
  \item \tau = \aall{t}{\tau'}\BY{assumption}
  \item \cvalidE{\Delta, \Dhyp{t}}{\Gamma}{\esceneSG{\uDelta_\text{app}}{\uGamma_\text{app}}{\uPsi}{\uPhi}{b}}{\ce'}{e'}{\tau'} \BY{assumption} \pflabel{cvalidE}
%  \item \uetsmenv{\Delta_\text{app}}{\Psi} \BY{assumption} \pflabel{uetsmenv}
  \item \Delta \cap \Delta_\text{app}=\emptyset \BY{assumption} \pflabel{delta-disjoint}
  \item \domof{\Gamma} \cap \domof{\Gamma_\text{app}}=\emptyset \BY{assumption} \pflabel{gamma-disjoint}
  \item \Dhyp{t} \notin \Delta_\text{app} \BY{identification convention}\pflabel{t-fresh}
  \item \Delta, \Dhyp{t} \cap \Delta_\text{app} = \emptyset \BY{\pfref{delta-disjoint} and \pfref{t-fresh}}\pflabel{delta-disjoint2}
  \item \hastypeU{\Dcons{\Delta, \Dhyp{t}}{\Delta_\text{app}}}{\Gcons{\Gamma}{\Gamma_\text{app}}}{e'}{\tau'} \BY{IH, part 2(a) on \pfref{cvalidE}, \pfref{delta-disjoint2} and \pfref{gamma-disjoint}}\pflabel{hastype1}
  \item \hastypeU{\Dcons{\Delta}{\Delta_\text{app}, \Dhyp{t}}}{\Gcons{\Gamma}{\Gamma_\text{app}}}{e'}{\tau'} \BY{exchange over $\Delta_\text{app}$ on \pfref{hastype1}}\pflabel{hastype2}
  \item \hastypeU{\Dcons{\Delta}{\Delta_\text{app}}}{\Gcons{\Gamma}{\Gamma_\text{app}}}{\aetlam{t}{e'}}{\aall{t}{\tau'}} \BY{Rule (\ref{rule:hastypeU-tlam}) on \pfref{hastype2}}
\end{pfsteps}
\resetpfcounter

\item[{\text{(\ref{rule:cvalidE-U-tap})}}~\textbf{through}~{\text{(\ref{rule:cvalidE-U-case})}}] These cases follow analagously, i.e. we apply the IH, part 2(a) to all proto-expression validation judgements, Lemma \ref{lemma:candidate-expansion-type-validation} to all proto-type validation judgements, the identification convention to ensure that extended contexts remain disjoint, weakening and exchange as needed, and the corresponding typing rule in Rules (\ref{rule:hastypeU-tap}) through (\ref{rule:hastypeU-case}).
\\

\item[\text{(\ref{rule:cvalidE-U-splicede})}] ~
\begin{pfsteps*}
  \item $\ce=\acesplicede{m}{n}{\ctau}$ \BY{assumption}
  \item $\escenev=\esceneU{\uDD{\uD}{\Delta_\text{app}}}{\uGG{\uG}{\Gamma_\text{app}}}{\uPsi}{b}$ \BY{assumption}
  \item   $\cvalidT{\emptyset}{\tsfrom{\escenev}}{\ctau}{\tau}$ \BY{assumption}
  \item $\parseUExp{\bsubseq{b}{m}{n}}{\ue}$ \BY{assumption}
  \item $\expandsU{\uDelta_\text{app}}{\uGamma_\text{app}}{\uPsi}{\ue}{e}{\tau}$ \BY{assumption} \pflabel{expands}
%  \item $\uetsmenv{\Delta_\text{app}}{\Psi}$ \BY{assumption} \pflabel{uetsmenv}
  \item $\Delta \cap \Delta_\text{app}=\emptyset$ \BY{assumption} \pflabel{delta-disjoint}
  \item $\domof{\Gamma} \cap \domof{\Gamma_\text{app}}=\emptyset$ \BY{assumption} \pflabel{gamma-disjoint}
  \item $\hastypeU{\Delta_\text{app}}{\Gamma_\text{app}}{e}{\tau}$ \BY{IH, part 1 on \pfref{expands}} \pflabel{hastype}
  \item $\hastypeU{\Dcons{\Delta}{\Delta_\text{app}}}{\Gcons{\Gamma}{\Gamma_\text{app}}}{e}{\tau}$ \BY{Lemma \ref{lemma:weakening-U} over $\Delta$ and $\Gamma$ and exchange on \pfref{hastype}}
\end{pfsteps*}
\resetpfcounter
\end{byCases}
\end{enumerate}
\begin{grayparbox}
\begin{enumerate}
\item[\hphantom{(a)}] \begin{byCases}
    \item[\text{(\ref{rule:cvalidE-U-match})}] ~
      \begin{pfsteps*}
        \item $\ce=\acematchwith{n}{\ce'}{\seqschemaX{\crv}}$ \BY{assumption}
        \item $e=\aematchwith{n}{e'}{\seqschemaX{r}}$ \BY{assumption}
        \item $\cvalidE{\Delta}{\Gamma}{\esceneUP{\uDelta}{\uGamma}{\uPsi}{\uPhi}{b}}{\ce'}{e'}{\tau'}$ \BY{assumption} \pflabel{cvalidE}        
        % \item $\istypeU{\Delta \cup \Delta_\text{app}}{\tau}$ \BY{assumption} \pflabel{istype}
        % \item $\cvalidT{\Delta}{\tsceneUP{\uDelta}{b}}{\ctau}{\tau}$ \BY{assumption} \pflabel{cvalidT}
        \item $\{\cvalidR{\Delta}{\Gamma}{\esceneUP{\uDelta}{\uGamma}{\uPsi}{\uPhi}{b}}{\crv_i}{r_i}{\tau'}{\tau}\}_{1 \leq i \leq n}$ \BY{assumption} \pflabel{cvalidR}
        \item $\Delta \cap \Delta_\text{app} = \emptyset$ \BY{assumption} \pflabel{delta-disjoint}
        \item $\domof{\Gamma} \cap \domof{\Gamma_\text{app}} = \emptyset$ \BY{assumption} \pflabel{gamma-disjoint}
        \item $\hastypeU{\Delta \cup \Delta_\text{app}}{\Gamma \cup \Gamma_\text{app}}{e'}{\tau'}$ \BY{IH, part 2(a) on \pfref{cvalidE}, \pfref{delta-disjoint} and \pfref{gamma-disjoint}} \pflabel{hastype}
        \item $\ruleType{\Delta \cup \Delta_\text{app}}{\Gamma \cup \Gamma_\text{app}}{r}{\tau'}{\tau}$ \BY{IH, part 2(b) on \pfref{cvalidR}, \pfref{delta-disjoint} and \pfref{gamma-disjoint}} \pflabel{ruleType}
        \item $\hastypeU{\Delta \cup \Delta_\text{app}}{\Gamma \cup \Gamma_\text{app}}{\aematchwith{n}{e'}{\seqschemaX{r}}}{\tau}$ \BY{Rule (\ref{rule:hastypeUP-match}) on \pfref{hastype} and \pfref{ruleType}}
      \end{pfsteps*}
      \resetpfcounter
  \end{byCases}
    \end{enumerate}
  \end{grayparbox}\vspace{-3px}
  \begin{grayparbox}\vspace{3px}
  \begin{enumerate}
  \item[(b)] There is only one case. 
    \begin{byCases}
     \item[\text{(\ref{rule:cvalidR-UP})}] ~
      \begin{pfsteps*}
        \item $\crv=\acematchrule{p}{\ce}$ \BY{assumption}
        \item $r=\aematchrule{p}{e}$ \BY{assumption}
        \item $\patType{\pctx'}{p}{\tau}$ \BY{assumption} \pflabel{patType}
        \item $\cvalidE{\Delta}{\Gcons{\Gamma}{\pctx'}}{\esceneUP{\uDelta}{\uGamma}{\uPsi}{\uPhi}{b}}{\ce}{e}{\tau'}$ \BY{assumption} \pflabel{cvalidE}
        \item $\Delta \cap \Delta_\text{app} = \emptyset$ \BY{assumption}\pflabel{delta-disjoint}
        \item $\domof{\Gamma} \cap \domof{\pctx'} = \emptyset$ \BY{identification convention}\pflabel{gamma-disjoint1}
        \item $\domof{\Gamma_\text{app}} \cap \domof{\pctx'} = \emptyset$ \BY{identification convention}\pflabel{gamma-disjoint2}
        \item $\domof{\Gamma} \cap \domof{\Gamma_\text{app}} = \emptyset$ \BY{assumption}\pflabel{gamma-disjoint3}
        \item $\domof{\Gcons{\Gamma}{\pctx'}} \cap \domof{\Gamma_\text{app}} = \emptyset$ \BY{standard finite set definitions and identities on \pfref{gamma-disjoint1}, \pfref{gamma-disjoint2} and \pfref{gamma-disjoint3}}\pflabel{gamma-disjoint4}
        \item $\hastypeU{\Dcons{\Delta}{\Delta_\text{app}}}{\Gcons{\Gcons{\Gamma}{\pctx'}}{\Gamma_\text{app}}}{e}{\tau'}$ \BY{IH, part 2(a) on \pfref{cvalidE}, \pfref{delta-disjoint} and \pfref{gamma-disjoint4}}\pflabel{hastype}
        \item $\hastypeU{\Dcons{\Delta}{\Delta_\text{app}}}{\Gcons{\Gcons{\Gamma}{\Gamma_\text{app}}}{\pctx'}}{e}{\tau'}$ \BY{exchange of $\pctx'$ and $\Gamma_\text{app}$ on \pfref{hastype}}\pflabel{hastype2}
        \item $\ruleType{\Dcons{\Delta}{\Delta_\text{app}}}{\Gcons{\Gamma}{\Gamma_\text{app}}}{\aematchrule{p}{e}}{\tau}{\tau'}$ \BY{Rule (\ref{rule:ruleType}) on \pfref{patType} and \pfref{hastype2}}
      \end{pfsteps*}
      \resetpfcounter
   \end{byCases} 
\end{enumerate}
\end{grayparbox}
\end{enumerate}
\vspace{10px}

The mutual induction can be shown to be well-founded by showing that the following numeric metric on the judgements that we induct on is decreasing:
\begin{align*}
\sizeof{\expandsSG{\uDelta}{\uGamma}{\uPsi}{\uPhi}{\ue}{e}{\tau}} & = \sizeof{\ue}\\
\sizeof{\cvalidE{\Delta}{\Gamma}{\esceneSG{\uDelta}{\uGamma}{\uPsi}{\uPhi}{b}}{\ce}{e}{\tau}} & = \sizeof{b}
\end{align*}
where $\sizeof{b}$ is the length of $b$ and $\sizeof{\ue}$ is the sum of the lengths of the seTLM literal bodies in $\ue$, as defined in Sec. \ref{appendix:SES-syntax}.

The only case in the proof of part 1 that invokes part 2 is Case (\ref{rule:expandsU-tsmap}). There, we have that the metric remains stable: \begin{align*}
 & \sizeof{\expandsSG{\uDelta}{\uGamma}{\uPsi}{\uPhi}{\utsmap{\tsmv}{b}}{e}{\tau}}\\
=& \sizeof{\cvalidE{\emptyset}{\emptyset}{\esceneSG{\uDelta}{\uGamma}{\uPsi}{\uPhi}{b}}{\ce}{e}{\tau}}\\
=&\sizeof{b}\end{align*}

The only case in the proof of part 2 that invokes part 1 is Case (\ref{rule:cvalidE-U-splicede}). There, we have that $\parseUExp{\bsubseq{b}{m}{n}}{\ue}$ and the IH is applied to the judgement $\expandsSG{\uDelta}{\uGamma}{\uPsi}{\uPhi}{\ue}{e}{\tau}$. Because the metric is stable when passing from part 1 to part 2, we must have that it is strictly decreasing in the other direction:
\[\sizeof{\expandsSG{\uDelta}{\uGamma}{\uPsi}{\uPhi}{\ue}{e}{\tau}} < \sizeof{\cvalidE{\Delta}{\Gamma}{\esceneSG{\uDelta}{\uGamma}{\uPsi}{\uPhi}{b}}{\acesplicede{m}{n}{\ctau}}{e}{\tau}}\]
i.e. by the definitions above, 
\[\sizeof{\ue} < \sizeof{b}\]

This is established by appeal to Condition \ref{condition:body-subsequences}, which states that subsequences of $b$ are no longer than $b$, and Condition \ref{condition:body-parsing}, which states that an unexpanded expression constructed by parsing a textual sequence $b$ is strictly smaller, as measured by the metric defined above, than the length of $b$, because some characters must necessarily be used to apply a TLM and delimit each literal body. 
Combining these conditions, we have that $\sizeof{\ue} < \sizeof{b}$ as needed.
\end{proof}

\begin{theorem}[Typed Expression Expansion]\label{thm:typed-expansion-short-U} If $\expandsSG{\uDD{\uD}{\Delta}}{\uGG{\uG}{\Gamma}\hspace{-3px}}{\uPsi}{\uPhi}{\ue}{e}{\tau}$ then $\hastypeU{\Delta}{\Gamma}{e}{\tau}$.
\end{theorem}
\begin{proof} This theorem follows immediately from Theorem \ref{thm:typed-expansion-full-U}, part 1(a). \end{proof}

% % \subsection{Expressibility}
% The following lemma establishes that each type can be expressed as a well-formed proto-type, under the same type formation context and any type splicing scene.
% \begin{lemma}[Proto-Expansion Type Expressibility]\label{lemma:proto-type-expressibility-U} If $\istypeU{\Delta}{\tau}$ then $\cvalidT{\Delta}{\tscenev}{\Cof{\tau}}{\tau}$. \end{lemma}
% \begin{proof}
% By rule induction over Rules (\ref{rules:istypeU}). In each case, we apply the IH on or over each premise, then apply the corresponding proto-type validation rule in Rules (\ref{rules:cvalidT-U}).
% \end{proof}

% The Type Expressibility Lemma establishes that every well-formed type, $\tau$, can be expressed as a well-formed unexpanded type, $\Uof{\tau}$. This requires defining the metafunction $\Uof{\Delta}$ which maps $\Delta$ onto an unexpanded type formation context as follows:
% \begin{align*}
% \Uof{\emptyset} &= \uDD{\emptyset}{\emptyset}\\
% \Uof{\Delta, \Dhyp{t}} &= \Uof{\Delta}, \uDhyp{\sigilof{t}}{t}
% \end{align*}
% \begin{lemma}[Type Expressibility]\label{lemma:type-expressibility} If $\istypeU{\Delta}{\tau}$ then $\expandsTU{\Uof{\Delta}}{\Uof{\tau}}{\tau}$.\end{lemma}
% \begin{proof} By rule induction over Rules (\ref{rules:istypeU}) using the definitions of $\Uof{\tau}$ and $\Uof{\Delta}$ above. In each case, we apply the IH to or over each premise, then apply the corresponding type expansion rule in Rules (\ref{rules:expandsTU}).\end{proof}


% The following lemma establishes that each well-typed expanded expression, $e$, can be expressed as a valid proto-expression, $\Cof{e}$, that is assigned the same type under any expression splicing scene.
% \begin{theorem}[Proto-Expansion Expression Expressibility]\label{theorem:proto-expressions-expressibility-U} If $\hastypeU{\Delta}{\Gamma}{e}{\tau}$ then $\cvalidE{\Delta}{\Gamma}{\escenev}{\Cof{e}}{e}{\tau}$.\end{theorem}
% \begin{proof} By rule induction over Rules (\ref{rules:hastypeU}). The rule transformation above guarantees that this lemma holds by construction. In particular, in each case, we apply Lemma \ref{lemma:proto-type-expressibility-U} to or over each type formation premise, the IH to or over each typing premise, then apply the corresponding proto-expression validation rule in Rules (\ref{rule:cvalidE-U-var}) through (\ref{rule:cvalidE-U-case}).
% \end{proof}

% The following lemma establishes that each well-typed expanded expression, $e$, can be expressed as a valid ce-expression, $\Cof{e}$, that is assigned the same type under any expression splicing scene.
% \begin{theorem}[Candidate Expansion Expression Expressibility]\label{lemma:ce-expressions-expressibility-UP} Both of the following hold:
% \begin{enumerate}
% \item If $\hastypeU{\Delta}{\Gamma}{e}{\tau}$ then $\cvalidE{\Delta}{\Gamma}{\escenev}{\Cof{e}}{e}{\tau}$.
% \item If $\ruleType{\Delta}{\Gamma}{r}{\tau}{\tau'}$ then $\cvalidR{\Delta}{\Gamma}{\escenev}{\Cof{r}}{r}{\tau}{\tau'}$.
% \end{enumerate}
% \end{theorem}
% \begin{proof} By mutual rule induction over Rules (\ref{rules:hastypeUP}) and Rule (\ref{rule:ruleType}). 

% For part 1, we induct on the assumption. 
% \begin{byCases}
% \item[\text{(\ref{rule:hastypeUP-var}) through (\ref{rule:hastypeUP-in})}] In each of these cases, we apply Lemma \ref{lemma:ce-type-expressibility-U} to or over each type formation premise, the IH (part 1) to or over each typing premise, then apply the corresponding ce-expression validation rule in Rules (\ref{rule:cvalidE-UP-var}) through (\ref{rule:cvalidE-UP-in}).
% \item[\text{(\ref{rule:hastypeUP-match})}] ~
%   \begin{pfsteps}
%   \item e = \aematchwith{n}{e'}{\seqschemaX{r}} \BY{assumption}
%   \item \Cof{e} = \acematchwith{n}{\Cof{\tau}}{\Cof{e'}}{\seqschemaXx{\Cofv}{r}} \BY{definition of $\Cof{e}$}
%   \item \hastypeU{\Delta}{\Gamma}{e'}{\tau'} \BY{assumption} \pflabel{hasType}
%   \item \istypeU{\Delta}{\tau} \BY{assumption} \pflabel{isType}
%   \item \{\ruleType{\Delta}{\Gamma}{r_i}{\tau'}{\tau}\}_{1 \leq i \leq n} \BY{assumption} \pflabel{ruleType}
%   \item \cvalidE{\Delta}{\Gamma}{\escenev}{\Cof{e'}}{e'}{\tau'} \BY{IH, part 1 on \pfref{hasType}} \pflabel{cvalidE}
%   \item \cvalidT{\Delta}{\tsfrom{\escenev}}{\Cof{\tau}}{\tau} \BY{Lemma \ref{lemma:candidate-expansion-type-validation} on \pfref{isType}} \pflabel{cvalidT}
%   \item \{\cvalidR{\Delta}{\Gamma}{\escenev}{\Cof{r_i}}{r_i}{\tau'}{\tau}\}_{1 \leq i \leq n} \BY{IH, part 2 over \pfref{ruleType}} \pflabel{cvalidR}
%   \item \cvalidE{\Delta}{\Gamma}{\escenev}{\acematchwith{n}{\Cof{\tau}}{\Cof{e'}}{\seqschemaXx{\Cofv}{r}}}{\aematchwith{n}{e'}{\seqschemaX{r}}}{\tau} \BY{Rule (\ref{rule:cvalidE-UP-match}) on \pfref{cvalidE}, \pfref{cvalidT} and \pfref{cvalidR}}
%   \end{pfsteps}
% \end{byCases}
% \resetpfcounter

% For part 2, we induct on the assumption. There is only one case.
% \begin{byCases}
% \item[\text{(\ref{rule:ruleType})}] ~
%   \begin{pfsteps}
%     \item r = \aematchrule{p}{e} \BY{assumption}
%     \item \Cof{r} = \acematchrule{p}{\Cof{e}} \BY{definition of $\Cof{r}$}
%     \item \patType{\pctx}{p}{\tau} \BY{assumption} \pflabel{patType}
%     \item \hastypeU{\Delta}{\Gcons{\Gamma}{\pctx}}{e}{\tau'} \BY{assumption} \pflabel{hasType}
%     \item \cvalidE{\Delta}{\Gcons{\Gamma}{\pctx}}{\escenev}{\Cof{e}}{e}{\tau'} \BY{IH, part 1 on \pfref{hasType}} \pflabel{cvalidE}
%     \item \cvalidR{\Delta}{\Gamma}{\escenev}{\acematchrule{p}{\Cof{e}}}{\aematchrule{p}{e}}{\tau}{\tau'} \BY{Rule (\ref{rule:cvalidR-UP}) on \pfref{patType} and \pfref{cvalidE}}
%   \end{pfsteps}
%   \resetpfcounter
% \end{byCases}
% \end{proof}

% The following lemma establishes that every well-typed expanded pattern that generates no hypotheses can be expressed as a ce-pattern.
% \begin{lemma}[Candidate Expansion Pattern Expressibility]\label{lemma:ce-pattern-expressibility-U} If $\patType{\emptyset}{p}{\tau}$ then $\cvalidP{\uGG{\emptyset}{\emptyset}}{\pscene{\uDelta}{\uPhi}{b}}{\Cof{p}}{p}{\tau}$.\end{lemma}
% \begin{proof} By rule induction over Rules (\ref{rules:patType}).
% \begin{byCases}
% \item[\text{(\ref{rule:patType-var})}] This case does not apply.
% \item[\text{(\ref{rule:patType-wild})}] ~
%   \begin{pfsteps*}
%     \item $p=\aewildp$ \BY{assumption}
%     \item $\Cof{p}=\acewildp$ \BY{definition of $\Cof{p}$}
%     \item $\cvalidP{\uGG{\emptyset}{\emptyset}}{\pscene{\uDelta}{\uPhi}{b}}{\acewildp}{\aewildp}{\tau}$ \BY{Rule (\ref{rule:cvalidP-UP-wild})}
%   \end{pfsteps*}
%   \resetpfcounter
% \item[\text{(\ref{rule:patType-fold})}] ~
%   \begin{pfsteps*}
%     \item $p=\aefoldp{p'}$ \BY{assumption}
%     \item $\Cof{p}=\acefoldp{\Cof{p'}}$ \BY{definition of $\Cof{p}$}
%     \item $\tau=\arec{t}{\tau'}$ \BY{assumption}
%     \item $\patType{\emptyset}{p'}{[\arec{t}{\tau'}/t]\tau'}$ \BY{assumption} \pflabel{patType}
%     \item $\cvalidP{\uGG{\emptyset}{\emptyset}}{\pscene{\uDelta}{\uPhi}{b}}{\Cof{p'}}{p}{[\arec{t}{\tau'}/t]\tau'}$ \BY{IH on \pfref{patType}} \pflabel{cvalidP}
%     \item $\cvalidP{\uGG{\emptyset}{\emptyset}}{\pscene{\uDelta}{\uPhi}{b}}{\acefoldp{\Cof{p'}}}{\aefoldp{p'}}{\arec{t}{\tau'}}$ \BY{Rule (\ref{rule:cvalidP-UP-fold}) on \pfref{cvalidP}}
%   \end{pfsteps*}
%   \resetpfcounter
% \item[\text{(\ref{rule:patType-tpl})}] ~
%   \begin{pfsteps*}
%     \item $p=\aetplp{\labelset}{\mapschema{p}{i}{\labelset}}$ \BY{assumption}
%     \item $\Cof{p}=\acetpl{\labelset}{\mapschemax{\Cofv}{p}{i}{\labelset}}$ \BY{definition of $\Cof{p}$}
%     \item $\tau=\aprod{\labelset}{\mapschema{\tau}{i}{\labelset}}$ \BY{assumption}
%     \item $\{\patType{\emptyset}{p_i}{\tau_i}\}_{i \in \labelset}$ \BY{assumption} \pflabel{patType}
%     \item $\{\cvalidP{\uGG{\emptyset}{\emptyset}}{\pscene{\uDelta}{\uPhi}{b}}{\Cof{p_i}}{p_i}{\tau_i}\}_{i \in \labelset}$ \BY{IH over \pfref{patType}} \pflabel{cvalidP}
%     \item $\cvalidP{\uGG{\emptyset}{\emptyset}}{\pscene{\uDelta}{\uPhi}{b}}{\acetpl{\labelset}{\mapschemax{\Cofv}{p}{i}{\labelset}}}{\aetplp{\labelset}{\mapschema{p}{i}{\labelset}}}{\aprod{\labelset}{\mapschema{\tau}{i}{\labelset}}}$ \BY{Rule (\ref{rule:cvalidP-UP-tpl}) on \pfref{cvalidP}}
%   \end{pfsteps*}
%   \resetpfcounter
% \item[\text{(\ref{rule:patType-inj})}] ~
%   \begin{pfsteps*}
%     \item $p=\aeinjp{\ell}{p'}$ \BY{assumption}
%     \item $\Cof{p}=\aceinjp{\ell}{\Cof{p'}}$ \BY{definition of $\Cof{p}$}
%     \item $\tau=\asum{\labelset, \ell}{\mapschema{\tau}{i}{\labelset}; \mapitem{\ell}{\tau'}}$ \BY{assumption}
%     \item $\patType{\emptyset}{p'}{\tau'}$ \BY{assumption}\pflabel{patType}
%     \item $\cvalidP{\uGG{\emptyset}{\emptyset}}{\pscene{\uDelta}{\uPhi}{b}}{\Cof{p'}}{p'}{\tau'}$ \BY{IH on \pfref{patType}}\pflabel{cvalidP}
%     \item $\cvalidP{\uGG{\emptyset}{\emptyset}}{\pscene{\uDelta}{\uPhi}{b}}{\aceinjp{\ell}{\Cof{p'}}}{\aeinjp{\ell}{p'}}{\asum{\labelset, \ell}{\mapschema{\tau}{i}{\labelset}; \mapitem{\ell}{\tau'}}}$ \BY{Rule (\ref{rule:cvalidP-UP-in}) on \pfref{cvalidP}}
%   \end{pfsteps*}
%   \resetpfcounter
% \end{byCases}
% \end{proof}

% \subsubsection{Expressibility}
% The following lemma establishes that each well-typed expanded pattern can be expressed as an unexpanded pattern matching values of the same type and generating the same hypotheses and corresponding identifier updates. The metafunction $\Uof{\pctx}$ maps $\pctx$ to an unexpanded typing context as follows:
% \begin{align*}
% \Uof{\emptyset} & = \uGG{\emptyset}{\emptyset}\\
% \Uof{\pctx, x : \tau} & = \Uof{\pctx}, \uGhyp{\sigilof{x}}{x}{\tau}\\
% \Uof{\Gconsi{i \in \labelset}{\pctx_i}} & = \Gconsi{i \in \labelset}{\Uof{\pctx_i}}
% \end{align*}
% \begin{lemma}[Pattern Expressibility]\label{lemma:pattern-expressibility} If $\patType{\pctx}{p}{\tau}$ then $\patExpands{\Uof{\pctx}}{\uPhi}{\Uof{p}}{p}{\tau}$.\end{lemma}
% \begin{proof} By rule induction over Rules (\ref{rules:patType}), using the definitions of $\Uof{\pctx}$ and $\Uof{p}$ given above. In each case, we can apply the IH to or over each premise, then apply the corresponding rule in Rules (\ref{rules:patExpands}).\end{proof}

% We can now establish the Expressibility Theorem -- that each well-typed expanded expression, $e$, can be expressed as an unexpanded expression, $\ue$, and assigned the same type under the corresponding contexts.

% \begin{theorem}[Expressibility] Both of the following hold:
% \begin{enumerate}
% \item If $\hastypeU{\Delta}{\Gamma}{e}{\tau}$ then $\expandsUP{\Uof{\Delta}}{\Uof{\Gamma}}{\uPsi}{\uPhi}{\Uof{e}}{e}{\tau}$.
% \item If $\ruleType{\Delta}{\Gamma}{r}{\tau}{\tau'}$ then $\ruleExpands{\Uof{\Delta}}{\Uof{\Gamma}}{\uPsi}{\uPhi}{\Uof{r}}{r}{\tau}{\tau'}$.
% \end{enumerate}
% \end{theorem}
% \begin{proof} By mutual rule induction over Rules (\ref{rules:hastypeUP}) and Rule (\ref{rule:ruleType}). 

% For part 1, we induct on the assumption. The rule transformation defined above guarantees that this part holds by its construction. In particular, in each case, we can apply Lemma \ref{lemma:type-expressibility} to or over each type formation premise, the IH (part 1) to or over each typing premise, the IH (part 2) over each rule typing premise, then apply the corresponding rule in Rules (\ref{rules:expandsUP}).

% For part 2, we induct on the assumption. There is only one case:
% \begin{byCases}
% \item[(\ref{rule:ruleType})] ~
% \begin{pfsteps*}
% \item $r = \aematchrule{p}{e}$ \BY{assumption}
% \item $\patType{\pctx}{p}{\tau}$ \BY{assumption} \pflabel{patType}
% \item $\hastypeU{\Delta}{\Gamma \cup \pctx}{e}{\tau'}$ \BY{assumption} \pflabel{hasType}
% \item $\Uof{\Gamma}=\uGG{\uG}{\Gamma}$, for some $\uG$ \BY{definition of $\Uof{\Gamma}$}
% \item $\Uof{\pctx} =\uGG{\uG'}{\pctx}$, for some $\uG'$ \BY{definition of $\Uof{\pctx}$}
% \item $\Uof{\Gamma \cup \pctx} = \uGG{\uG \uplus \uG'}{\Gamma \cup \pctx}$ \BY{definition of $\Uof{\pctx}$}
% \item $\Uof{r} = \aumatchrule{\Uof{p}}{\Uof{e}}$ \BY{definition of $\Uof{r}$}
% \item $\patExpands{\uGG{\uG'}{\pctx}}{\uPhi}{\Uof{p}}{p}{\tau}$ \BY{Lemma \ref{lemma:pattern-expressibility} on \pfref{patType}} \pflabel{patExpands}
% \item $\expandsUP{\uDelta}{\uGG{\uGcons{\uG}{\uG'}}{\Gcons{\Gamma}{\pctx}}}{\uPsi}{\uPhi}{\Uof{e}}{e}{\tau'}$ \BY{IH, part 1 on \pfref{hasType}} \pflabel{expandsUP}
% \item $\ruleExpands{\Uof{\Delta}}{\uGG{\uG}{\Gamma}}{\uPsi}{\uPhi}{\aumatchrule{\Uof{p}}{\Uof{e}}}{\aematchrule{p}{e}}{\tau}{\tau'}$ \BY{Rule (\ref{rule:ruleExpands}) on \pfref{patExpands} and \pfref{expandsUP}}
% \end{pfsteps*}
% \resetpfcounter
% \end{byCases}
% \end{proof}

\subsection{Abstract Reasoning Principles}\label{appendix:SES-reasoning-principles}
\begin{lemma}[Proto-Type Expansion Decomposition] 
\label{thm:proto-type-expansion-decomposition-SES}
If $\cvalidT{\Delta}{\tsceneU{\uDD{\uD}{\Delta_\text{app}}}{b}}{\ctau}{\tau}$ where $\segof{\ctau} = \sseq{\acesplicedt{m_i}{n_i}}{n}$ then all of the following hold:
\begin{enumerate}
\item $\sseq{\expandsTU{\uDD{\uD}{\Delta_\text{app}}}{
  \parseUTypF{\bsubseq{b}{m_i}{n_i}}
}{\tau_i}}{n}$
% \item $\sseq{\istypeU{\Delta_\text{app}}{\tau_i}}{n}$
\item $\tau = [\sseq{\tau_i/t_i}{n}]\tau'$ for some $\tau'$ and fresh $\sseq{t_i}{n}$ (i.e.  $\sseq{t_i \notin \domof{\Delta}}{n}$ and $\sseq{t_i \notin \domof{\Delta_\text{app}}}{n}$)
\item $\mathsf{fv}(\tau') \subset \domof{\Delta} \cup \sseq{t_i}{n}$
\end{enumerate}
\end{lemma}
\begin{proof}
By rule induction over Rules (\ref{rules:cvalidT-U}). In the following, let $\uDelta = \uDD{\uD}{\Delta_\text{app}}$ and $\tscenev=\tsceneU{\uDelta}{b}$.
\begin{byCases}
  \item[\text{(\ref{rule:cvalidT-U-tvar})}] 
    \begin{pfsteps}
    \item \ctau = t \BY{assumption}
    \item \tau = t \BY{assumption}
    \item \Delta = \Delta', \Dhyp{t} \BY{assumption} \pflabel{Delta}
    \item \segof{\ctau} = \emptyset \BY{definition}
    \item \fvof{t} = \{ t \} \BY{definition} \pflabel{fv}
    \item \{ t \} \subset \domof{\Delta} \cup \emptyset \BY{definition} \pflabel{subset}
    \end{pfsteps}
    The conclusions hold as follows:
    \begin{enumerate}
    \item This conclusion holds trivially because $n=0$.
    % \item This conclusion holds trivially because $n=0$.
    \item Choose $\tau'=t$ and $\emptyset$.
    \item \pfref{subset}
    \end{enumerate}
    \resetpfcounter
  \item[\text{(\ref{rule:cvalidT-U-parr})}] 
    \begin{pfsteps}
    \item \ctau = \aceparr{\ctau_1}{\ctau_2} \BY{assumption}
    \item \tau = \aparr{\tau'_1}{\tau'_2} \BY{assumption}
    \item \cvalidT{\Delta}{\tscenev}{\ctau_1}{\tau'_1} \BY{assumption} \pflabel{cvalidT1}
    \item \cvalidT{\Delta}{\tscenev}{\ctau_2}{\tau'_2} \BY{assumption} \pflabel{cvalidT2}
    \item \segof{\ctau} = \segof{\ctau_1} \cup \segof{\ctau_2} \BY{definition} \pflabel{summaryOf}
    \item \segof{\ctau_1} = \sseqS{\acesplicedt{m_{i}}{n_{i}}}{0}{n'} \BY{definition} \pflabel{summaryOf1}
    \item \segof{\ctau_2} = \sseqS{\acesplicedt{m_{i}}{n_{i}}}{n'}{n} \BY{definition} \pflabel{summaryOf2}
    \item \sseq{\expandsTU{\uDD{\uD}{\Delta_\text{app}}}{
  \parseUTypF{\bsubseq{b}{m_i}{n_i}}
}{\tau_i}}{n'} 
    \BY{IH on \pfref{cvalidT1} and \pfref{summaryOf1}}
    \pflabel{expands1}
    % \item \sseq{\istypeU{\Delta_\text{app}}{\tau_i}}{n}
    \item \tau'_1 = [\sseq{\tau_i/t_i}{n'}]\tau''_1 \text{~for some $\tau''_1$ and fresh $\sseq{t_i}{n'}$} \BY{IH on \pfref{cvalidT1} and \pfref{summaryOf1}
        \pflabel{decompose1}}
    % \item \istypeU{\Delta \cup \sseq{\Dhyp{t_i}}{n'}}{\tau''_1}
    %     \BY{IH on \pfref{cvalidT1} and \pfref{summaryOf1}}
    %     \pflabel{istype1}
    \item \fvof{\tau''_1} \subset \domof{\Delta} \cup \sseq{t_i}{n'} 
        \BY{IH on \pfref{cvalidT1} and \pfref{summaryOf1}
        \pflabel{fv1}}
    \item \sseqS{\expandsTU{\uDD{\uD}{\Delta_\text{app}}}{
  \parseUTypF{\bsubseq{b}{m_i}{n_i}}
}{\tau_i}}{n'}{n} 
    \BY{IH on \pfref{cvalidT2} and \pfref{summaryOf2}}
    \pflabel{expands2}
    % \item \sseq{\istypeU{\Delta_\text{app}}{\tau_i}}{n}
    \item \tau'_2 = [\sseqS{\tau_i/t_i}{n'}{n}]\tau''_2 \text{~for some $\tau''_2$ and fresh $\sseqS{t_i}{n'}{n}$}
        \BY{IH on \pfref{cvalidT2} and \pfref{summaryOf2}}
        \pflabel{decompose2}
    % \item \istypeU{\Delta \cup \sseqS{\Dhyp{t_i}}{n'}{n}}{\tau''_2}
    %     \BY{IH on \pfref{cvalidT2} and \pfref{summaryOf2}}
    %     \pflabel{istype2}
    \item \fvof{\tau''_2} \subset \domof{\Delta} \cup \sseqS{t_i}{n'}{n}
        \BY{IH on \pfref{cvalidT2} and \pfref{summaryOf2}}
        \pflabel{fv2}
    \item \sseq{t_i}{n'} \cap \sseqS{t_i}{n'}{n} = \emptyset \BY{identification convention} \pflabel{fvid}
    \item \fvof{\tau''_1} \subset \domof{\Delta} \cup \sseq{t_i}{n} \BY{\pfref{fv1} and \pfref{fvid}} \pflabel{fvf1}
\item \fvof{\tau''_2} \subset \domof{\Delta} \cup \sseq{t_i}{n} \BY{\pfref{fv2} and \pfref{fvid}}     \pflabel{fvf2}
    \item \tau_1' = [\sseq{\tau_i/t_i}{n}]\tau_1'' \BY{substitution properties and \pfref{decompose1} and \pfref{fvid}} \pflabel{fdecompose1}
    \item \tau_2' = [\sseq{\tau_i/t_i}{n}]\tau_2'' \BY{substitution properties and \pfref{decompose2} and \pfref{fvid}} \pflabel{fdecompose2}
    \item \aparr{\tau_1'}{\tau_2'} = [\sseq{\tau_i/t_i}{n}]\aparr{\tau_1''}{\tau_2''} \BY{substitution and \pfref{fdecompose1} and \pfref{fdecompose2}}
    \pflabel{parr}
    % \item \istypeU{\Delta \cup \sseq{\Dhyp{t_i}}{n}}{\tau''_1} \BY{Weakening and \pfref{istype1}} \pflabel{istype3}
    % \item \istypeU{\Delta \cup \sseq{\Dhyp{t_i}}{n}}{\tau''_2} \BY{Weakening and \pfref{istype2}} \pflabel{istype4}
    % \item \istypeU{\Delta \cup \sseq{\Dhyp{t_i}}{n}}{\aparr{\tau''_1}{\tau''_2}} \BY{Rule (\ref{rule:istypeU-parr}) on \pfref{istype3} and \pfref{istype4}}
    % \pflabel{finalistype}
    \item \fvof{\aparr{\tau''_1}{\tau''_2}} = \fvof{\tau''_1} \cup \fvof{\tau''_2} \BY{definition} \pflabel{fvparr}
    \item \fvof{\aparr{\tau''_1}{\tau''_2}} \subset \domof{\Delta} \cup \sseq{t_i}{n} \BY{\pfref{fvparr} and \pfref{fvf1} and \pfref{fvf2}} \pflabel{finalfv}
    \end{pfsteps}
    The conclusions hold as follows:
    \begin{enumerate}
      \item \pfref{expands1} $\cup$ \pfref{expands2}
      \item Choosing $\sseq{t_i}{n}$ and $\aparr{\tau''_1}{\tau''_2}$, by \pfref{parr}
      \item \pfref{finalfv}
    \end{enumerate}
    \resetpfcounter
  \item[\text{(\ref{rule:cvalidT-U-all}) \textbf{through} (\ref{rule:cvalidT-U-sum})}] These cases follow by analagous inductive argument.
  \item[\text{(\ref{rule:cvalidT-U-splicedt})}] ~
  \begin{pfsteps}
  \item \ctau = \acesplicedt{m}{n} \BY{assumption}
  \item \segof{\acesplicedt{m}{n}} = \{ \acesplicedt{m}{n} \} \BY{definition}
  \item \parseUTyp{\bsubseq{b}{m}{n}}{\utau} \BY{assumption} \pflabel{parseUTyp}
  \item \expandsTU{\uDD{\uD}{\Delta_\text{app}}}{\utau}{\tau} \BY{assumption} \pflabel{expandsTU}
  \item t \notin \domof{\Delta} \BY{identification convention} \pflabel{notin}
  \item t \notin \domof{\Delta_\text{app}} \BY{identification} \pflabel{notin2}
  \item \tau = [\tau/t]\tau \BY{definition} \pflabel{subst}
  \item \fvof{t} \subset \Delta \cup \{ t \} \BY{definition} \pflabel{fv}
  \end{pfsteps}
  The conclusions hold as follows:
  \begin{enumerate}
    \item \pfref{parseUTyp} and \pfref{expandsTU}
    \item Choosing $\{t\}$ and $t$, by \pfref{notin}, \pfref{notin2} and \pfref{subst}
    \item \pfref{fv}
  \end{enumerate}
  \resetpfcounter
\end{byCases}
\end{proof}

\begin{lemma}[Proto-Expression \graytxtbox{and Proto-Rule} Expansion Decomposition] 
\label{thm:proto-expression-expansion-decomposition} ~
\begin{enumerate}
\item If $\cvalidE
  {\Delta}{\Gamma}
  {\esceneSG
    {\uDD{\uD}{\Delta_\text{app}}}
    {\uGG{\uG}{\Gamma_\text{app}}}
    {\uPsi}{\uPhi}{b}
  }{\ce}{e}{\tau}$ where $\segof{\ce} = \sseq{\acesplicedt{m'_i}{n'_i}}{\nty} \cup \sseq{\acesplicede{m_i}{n_i}{\ctau_i}}{\nexp}$ then all of the following hold:
  \begin{enumerate}
    \item $\sseq{
          \expandsTU{\uDD{\uD}{\Delta_\text{app}}}
          {
            \parseUTypF{\bsubseq{b}{m'_i}{n'_i}}
          }{\tau'_i}
        }{\nty}$
    \item $\sseq{
      \cvalidT{\emptyset}{
        \tsceneUP
          {\uDD
            {\uD}{\Delta_\text{app}}
          }{b}
      }{
        \ctau_i
      }{\tau_i}
    }{\nexp}$
    \item $\sseq{
      \expandsSG
        {\uDD{\uD}{\Delta_\text{app}}}
        {\uGG{\uG}{\Gamma_\text{app}}}
        {\uPsi}
        {\uPhi}
        {\parseUExpF{\bsubseq{b}{m_i}{n_i}}}
        {e_i}
        {\tau_i}
    }{\nexp}$
    \item $e = [\sseq{\tau'_i/t_i}{\nty}, \sseq{e_i/x_i}{\nexp}]e'$ for some $e'$ and $\sseq{t_i}{\nty}$ and $\sseq{x_i}{\nexp}$ such that $\sseq{t_i}{\nty}$ fresh and $\sseq{x_i}{\nexp}$ fresh 
    \item $\mathsf{fv}(e') \subset \domof{\Delta} \cup \domof{\Gamma} \cup \sseq{t_i}{\nty} \cup \sseq{x_i}{\nexp}$
    % \item $\hastypeU
    %   {\Delta \cup \sseq{\Dhyp{t_i}}{\nty}}
    %   {\Gamma \cup \sseq{x_i : \tau_i}{\nexp}}
    %   {e'}{\tau}$
  \end{enumerate}
\end{enumerate}

\begin{grayparbox}
\begin{enumerate}
\item[2.] If $\cvalidR{\Delta}{\Gamma}{\esceneUP{\uDD{\uD}{\Delta_\text{app}}}{\uGG{\uG}{\Gamma_\text{app}}}{\uPsi}{\uPhi}{b}}{\crv}{r}{\tau}{\tau'}$ and \[\segof{\crv} = \sseq{\acesplicedt{m'_i}{n'_i}}{\nty} \cup \sseq{\acesplicede{m_i}{n_i}{\ctau_i}}{\nexp}\] then all of the following hold:
  \begin{enumerate}
    \item $\sseq{
          \expandsTU{\uDD{\uD}{\Delta_\text{app}}}
          {
            \parseUTypF{\bsubseq{b}{m'_i}{n'_i}}
          }{\tau'_i}
        }{\nty}$
    \item $\sseq{
      \cvalidT{\emptyset}{
        \tsceneUP
          {\uDD
            {\uD}{\Delta_\text{app}}
          }{b}
      }{
        \ctau_i
      }{\tau_i}
    }{\nexp}$
    \item $\sseq{
      \expandsSG
        {\uDD{\uD}{\Delta_\text{app}}}
        {\uGG{\uG}{\Gamma_\text{app}}}
        {\uPsi}
        {\uPhi}
        {\parseUExpF{\bsubseq{b}{m_i}{n_i}}}
        {e_i}
        {\tau_i}
    }{\nexp}$
    \item $r = [\sseq{\tau'_i/t_i}{\nty}, \sseq{e_i/x_i}{\nexp}]r'$ for some $e'$ and fresh $\sseq{t_i}{\nty}$ and fresh $\sseq{x_i}{\nexp}$ 
    \item $\mathsf{fv}(r') \subset \domof{\Delta} \cup \domof{\Gamma} \cup \sseq{t_i}{\nty} \cup \sseq{x_i}{\nexp}$
    % \item $\ruleType
    %   {\Delta \cup \sseq{\Dhyp{t_i}}{\nty}}
    %   {\Gamma \cup \sseq{x_i : \tau_i}{\nexp}}
    %   {r'}{\tau}{\tau'}$
  \end{enumerate}
\end{enumerate}
\end{grayparbox}
\end{lemma}
\begin{proof} By rule induction over Rules (\ref{rules:cvalidE-U}) and Rule (\ref{rule:cvalidR-UP}). In the following, let $\uDelta=\uDD{\uD}{\Delta_\text{app}}$ and $\uGamma=\uGG{\uG}{\Gamma_\text{app}}$ and $\escenev=\esceneUP{\uDelta}{\uGamma}{\uPsi}{\uPhi}{b}$.
\begin{enumerate}
\item \begin{byCases}
  \item[\text{(\ref{rule:cvalidE-U-var})}] \begin{pfsteps}
    \item \ce=x \BY{assumption}
    \item e=x \BY{assumption}
    \item \Gamma = \Gamma', x : \tau \BY{assumption}
    \item \segof{x}=\{ \} \BY{definition}
    \item \fvof{x} = \{ x \} \BY{definition}
    \item \fvof{x} \subset \domof{\Gamma} \BY{definition} \pflabel{fv}
    \item \fvof{x} \subset \domof{\Gamma} \cup \domof{\Delta} \BY{\pfref{fv} and definition of subset} \pflabel{fv2}
  \end{pfsteps}
  The conclusions hold as follows:
  \begin{enumerate}
    \item This conclusion holds trivially because $\nty=0$.
    \item This conclusion holds trivially because $\nexp=0$.
    \item This conclusion holds trivially because $\nexp=0$.
    \item Choose $x$, $\emptyset$ and $\emptyset$. 
    \item \pfref{fv2}
  \end{enumerate}
  \resetpfcounter
  \item[\text{(\ref{rule:cvalidE-U-asc}) \textbf{through} (\ref{rule:cvalidE-U-case})}] These cases follow by straightforward inductive argument.
  \item[\text{(\ref{rule:cvalidE-U-splicede})}] \begin{pfsteps}
    \item \ce=\acesplicede{m}{n}{\ctau} \BY{assumption}
    \item \segof{\acesplicede{m}{n}{\ctau}} = \segof{\ctau} \cup \{ \acesplicede{m}{n}{\ctau} \} \BY{definition}
    \item \segof{\ctau} = \sseq{\acesplicedt{m'_i}{n'_i}}{\nty} \BY{definition} \pflabel{ctau}
    \item \cvalidT{\emptyset}{\tsfrom{\escenev}}{\ctau}{\tau} \BY{assumption} \pflabel{cvalidT}
    \item \parseUExp{\bsubseq{b}{m}{n}}{\ue} \BY{assumption} \pflabel{parseUExp}
    \item \expandsSG{\uDD{\uD}{\Delta_\text{app}}}{\uGG{\uG}{\Gamma_\text{app}}}{\uPsi}{\uPhi}{\ue}{e}{\tau}
    \BY{assumption} \pflabel{expandsSG}
    \item \sseq{\expandsTU{\uDD{\uD}{\Delta_\text{app}}}{
  \parseUTypF{\bsubseq{b}{m'_i}{n'_i}}
}{\tau'_i}}{\nty}
% \item $\sseq{\istypeU{\Delta_\text{app}}{\tau_i}}{n}$
   \BY{Lemma \ref{thm:proto-type-expansion-decomposition-SES} on \pfref{cvalidT} and \pfref{ctau}} \pflabel{parseUTyp}
   \item x \notin \domof{\Gamma} \BY{identification convention} \pflabel{notin1}
   \item x \notin \domof{\Gamma_\text{app}} \BY{identification convention} \pflabel{notin2}
   \item x \notin \domof{\Delta} \BY{identificaiton convention} \pflabel{notin3}
   \item x \notin \domof{\Delta_\text{app}} \BY{identification convention} \pflabel{notin4}
   \item e = [\sseq{\tau'_i/t_i}{\nty}, e/x]x \BY{definition} \pflabel{subst}
   \item \fvof{x} = \{ x \} \BY{definition}
   \item \fvof{x} \subset \domof{\Delta} \cup \domof{\Gamma} \cup \sseq{t_i}{\nty} \cup \{x\} \BY{definition} \pflabel{subset}
  \end{pfsteps}
  The conclusions hold as follows:
  \begin{enumerate}
    \item \pfref{parseUTyp}
    \item $\{$\pfref{cvalidT}$\}$
    \item $\{$\pfref{expandsSG}$\}$
    \item Choosing $x$, $\sseq{t_i}{\nty}$ and $\{x\}$, by \pfref{notin1}, \pfref{notin2}, \pfref{notin3}, \pfref{notin4} and \pfref{subst}.
    \item \pfref{subset}
  \end{enumerate}
  \resetpfcounter
\end{byCases}
\begin{grayparbox}
\begin{byCases}
  \item[\text{(\ref{rule:cvalidE-U-match})}] \begin{pfsteps*}
    \item $\ce = \acematchwith{\nrules}{\ce'}{\seqschemaX{\crv}}$ \BY{assumption}
    \item $e = \aematchwith{\nrules}{\tau}{e'}{\seqschemaX{r}}$ \BY{assumption}
    % \item $\istypeU{\Delta \cup \Delta_\text{app}}{\tau'}$ \BY{assumption} \pflabel{istype}
    \item $\cvalidE{\Delta}{\Gamma}{\escenev}{\ce}{e}{\tau'}$ \BY{assumption} \pflabel{cvalidE}
    \item $\{\cvalidR{\Delta}{\Gamma}{\escenev}{\crv_j}{r_j}{\tau'}{\tau}\}_{1 \leq j \leq \nrules}$
    \BY{assumption} \pflabel{cvalidR}
    \item $\segof{\acematchwith{\nrules}{\ce'}{\seqschemaX{\crv}}} = \segof{\ce} \cup \bigcup_{0 \leq i < \nrules} \segof{\crv_i}$ \BY{definition} 
    \item $\segof{\ce'} = \sseq{\acesplicedt{m'_i}{n'_i}}{\nty'} \cup \sseq{\acesplicede{m_i}{n_i}{\ctau_i}}{\nexp'}$ \BY{definition} \pflabel{summaryOfscrut}
    \item $\{\segof{\crv_j} = \sseq{\acesplicedt{m'_{i,j}}{n'_{i,j}}}{\ntyj} \cup \sseq{\acesplicede{m_{i,j}}{n_{i,j}}{\ctau_{i,j}}}{\nexpj}\}_{0 \leq j < \nrules}$ \BY{definition} \pflabel{summaryOfr}
    \item $\sseq{
              \expandsTU{\uDD{\uD}{\Delta_\text{app}}}
              {
                \parseUTypF{\bsubseq{b}{m'_i}{n'_i}}
              }{\tau'_i}
            }{\nty'}$
        \BY{IH, part 1 on \pfref{cvalidE} and \pfref{summaryOfscrut}} \pflabel{c1scrut}
    \item $\sseq{
          \cvalidT{\emptyset}{
            \tsceneUP
              {\uDD
                {\uD}{\Delta_\text{app}}
              }{b}
          }{
            \ctau_i
          }{\tau_i}
        }{\nexp'}$
            \BY{IH, part 1 on \pfref{cvalidE} and \pfref{summaryOfscrut}} \pflabel{c2scrut}
    \item $\sseq{
          \expandsSG
            {\uDD{\uD}{\Delta_\text{app}}}
            {\uGG{\uG}{\Gamma_\text{app}}}
            {\uPsi}
            {\uPhi}
            {\parseUExpF{\bsubseq{b}{m_i}{n_i}}}
            {e_i}
            {\tau_i}
        }{\nexp'}$
            \BY{IH, part 1 on \pfref{cvalidE} and \pfref{summaryOfscrut}} \pflabel{c3scrut}
    \item $e' = [\sseq{\tau'_i/t_i}{\nty'}, \sseq{e_i/x_i}{\nexp'}]e''$ for some $e''$ and fresh $\sseq{t_i}{\nty'}$ and fresh $\sseq{x_i}{\nexp'}$ 
        \BY{IH, part 1 on \pfref{cvalidE} and \pfref{summaryOfscrut}} \pflabel{c4scrut}
    \item $\mathsf{fv}(e'') \subset \domof{\Delta} \cup \domof{\Gamma} \cup \sseq{t_i}{\nty'} \cup \sseq{x_i}{\nexp'}$ \BY{IH, part 1 on \pfref{cvalidE} and \pfref{summaryOfscrut}} \pflabel{c5scrut}
    % \item $\hastypeU
    %       {\Delta \cup \sseq{\Dhyp{t_i}}{\nty'}}
    %       {\Gamma \cup \sseq{x_i : \tau_i}{\nexp'}}
    %       {e''}{\tau'}$
    %           \BY{IH, part 1 on \pfref{cvalidE} and \pfref{summaryOfscrut}} \pflabel{c5scrut}
    \item $\{\sseq{
              \expandsTU{\uDD{\uD}{\Delta_\text{app}}}
              {
                \parseUTypF{\bsubseq{b}{m'_{i,j}}{n'_{i,j}}}
              }{\tau'_{i,j}}
            }{\ntyj}\}_{0 \leq j < \nrules}$
        \BY{IH, part 2 over \pfref{cvalidR} and \pfref{summaryOfr}} \pflabel{c1r}
    \item $\{\sseq{
          \cvalidT{\emptyset}{
            \tsceneUP
              {\uDD
                {\uD}{\Delta_\text{app}}
              }{b}
          }{
            \ctau_{i,j}
          }{\tau_{i,j}}
        }{\nexpj}\}_{0 \leq j < \nrules}$
            \BY{IH, part 2 over \pfref{cvalidR} and \pfref{summaryOfr}} \pflabel{c2r}
    \item $\{\sseq{
          \expandsSG
            {\uDD{\uD}{\Delta_\text{app}}}
            {\uGG{\uG}{\Gamma_\text{app}}}
            {\uPsi}
            {\uPhi}
            {\parseUExpF{\bsubseq{b}{m_{i,j}}{n_{i,j}}}}
            {e_{i,j}}
            {\tau_{i,j}}
        }{\nexpj}\}_{0 \leq j < \nrules}$
            \BY{IH, part 2 over \pfref{cvalidR} and \pfref{summaryOfr}} \pflabel{c3r}
    \item $\{r_j = [\sseq{\tau'_{i,j}/t_{i,j}}{\ntyj}, \sseq{e_{i,j}/x_{i,j}}{\nexpj}]r_j'\}_{0 \leq j < \nrules}$ for some  $\{r_j'\}_{0 \leq j < \nrules}$  and fresh $\{\sseq{t_{i,j}}{\ntyj}\}_{0 \leq j < \nrules}$ and fresh $\{\sseq{x_{i,j}}{\nexpj}\}_{0 \leq j < \nrules}$ 
        \BY{IH, part 2 over \pfref{cvalidR} and \pfref{summaryOfr}} \pflabel{c4r}
        \item $\{\mathsf{fv}(r_j') \subset \domof{\Delta} \cup \domof{\Gamma} \cup \sseq{t_{i,j}}{\ntyj} \cup \sseq{x_{i,j}}{\nexpj}\}_{0 \leq j < \nrules}$ \BY{IH, part 2 over \pfref{cvalidR} and \pfref{summaryOfr}} \pflabel{c5r}
    % \item $\{\ruleType
    %       {\Delta \cup \sseq{\Dhyp{t_{i,j}}}{\ntyj}}
    %       {\Gamma \cup \sseq{x_{i,j} : \tau_{i,j}}{\nexpj}}
    %       {r_j'}{\tau'}{\tau}\}_{0 \leq j < \nrules}$
    %           \BY{IH, part 2 over \pfref{cvalidR} and \pfref{summaryOfr}} \pflabel{c5r}
    \item $(\cup_{0 \leq j < \nrules} \sseq{t_{i,j}}{\ntyj}) \cap \sseq{t_i}{\nty'} = \emptyset$ \BY{identification convention}\pflabel{tfresh}
    \item $(\cup_{0 \leq j < \nrules} \sseq{x_{i,j}}{\nexpj}) \cap \sseq{x_i}{\nexp'} = \emptyset$ \BY{identification convention} \pflabel{xfresh}
    \item $e' = [\sseq{\tau'_i/t_i}{\nty'} \cup_{0 \leq j < \nrules} \sseq{\tau_{i,j}/t_{i,j}}{\ntyj}, \sseq{e_i/x_i}{\nexp} \cup_{0 \leq j < \nrules} \sseq{\tau_{i,j}/t_{i,j}}{\ntyj} ]e''$ \BY{substitution properties and \pfref{c4scrut} and \pfref{c5scrut} and \pfref{tfresh} and \pfref{xfresh}} \pflabel{c4fscrut}
    \item $\{r_j = [\sseq{\tau'_i/t_i}{\nty'} \cup_{0 \leq j < \nrules} \sseq{\tau_{i,j}/t_{i,j}}{\ntyj}, \sseq{e_i/x_i}{\nexp} \cup_{0 \leq j < \nrules} \sseq{\tau_{i,j}/t_{i,j}}{\ntyj} ]r_j'\}_{0 \leq j < \nrules}$ \BY{substitution properties and \pfref{c4r} and \pfref{c5r} and \pfref{tfresh} and \pfref{xfresh}} \pflabel{c4fr}
    \item $e = [\sseq{\tau'_i/t_i}{\nty'} \cup_{0 \leq j < \nrules} \sseq{\tau_{i,j}/t_{i,j}}{\ntyj} , \sseq{e_i/x_i}{\nexp'} \cup_{0 \leq j < \nrules} \sseq{e_{i,j}/x_{i,j}}{\nexpj}]\aematchwith{\nrules}{e''}{\seqschemaX{r'}}$ \BY{\pfref{c4fscrut} and \pfref{c4fr} and definition of substitution} \pflabel{c4e}
    \item $\mathsf{fv}(e'') \subset \domof{\Delta} \cup \domof{\Gamma} \cup \sseq{t_i}{\nty'} \cup_{0 \leq j < \nrules} \sseq{t_{i,j}}{\ntyj} \cup \sseq{x_i}{\nexp'} \cup_{0 \leq j < \nrules} \sseq{x_{i,j}}{\nexpj}$ \BY{\pfref{c5scrut} and \pfref{tfresh} and \pfref{xfresh}} \pflabel{c5fscrut}
    \item $\{\mathsf{fv}(r'_j) \subset \domof{\Delta} \cup \domof{\Gamma} \cup \sseq{t_i}{\nty'} \cup_{0 \leq j < \nrules} \sseq{t_{i,j}}{\ntyj} \cup \sseq{x_i}{\nexp'} \cup_{0 \leq j < \nrules} \sseq{x_{i,j}}{\nexpj}\}_{0 \leq j < \nrules}$ \BY{\pfref{c5r} and \pfref{tfresh} and \pfref{xfresh}} \pflabel{c5fr}
    \item $\mathsf{fv}(\aematchwith{\nrules}{e''}{\seqschemaX{r'}}) \subset \domof{\Delta} \cup \domof{\Gamma} \cup \sseq{t_i}{\nty'} \cup_{0 \leq j < \nrules} \sseq{t_{i,j}}{\ntyj} \cup \sseq{x_i}{\nexp'} \cup_{0 \leq j < \nrules} \sseq{x_{i,j}}{\nexpj}$ \BY{\pfref{c5fscrut} and \pfref{c5fr}} \pflabel{c5ff}
    % \item $\hastypeU
    %       {\Delta \cup \sseq{\Dhyp{t_i}}{\nty'} \cup \{ \sseq{t_{i,j}}{\ntyj} \}_{0 \leq j < \nrules}}
    %       {\Gamma \cup \sseq{x_i : \tau_i}{\nexp'} \cup \{ \sseq{x_{i,j}}{\nexpj} \}_{0 \leq j < \nrules}}
    %       {e''}{\tau'}$ \BY{Weakening on \pfref{c5scrut}} \pflabel{c5scrutf}
    % \item $\{\ruleType
    %       {\Delta \cup \sseq{\Dhyp{t_i}}{\nty'} \cup \{ \sseq{t_{i,j}}{\ntyj} \}_{0 \leq j < \nrules}}
    %       {\Gamma \cup \sseq{x_{i,j} : \tau_{i,j}}{\nexpj} \cup \{ \sseq{x_{i,j}}{\nexpj} \}_{0 \leq j < \nrules}}
    %       {r_j'}{\tau'}{\tau}\}_{0 \leq j < \nrules}$ \BY{Weakening over \pfref{c5r}} \pflabel{c5rf}
    % \item $\hastypeU
    %           {\Delta \cup \sseq{\Dhyp{t_i}}{\nty'} \cup \{ \sseq{t_{i,j}}{\ntyj} \}_{0 \leq j < \nrules}}
    %           {\Gamma \cup \sseq{x_i : \tau_i}{\nexp'} \cup \{ \sseq{x_{i,j}}{\nexpj} \}_{0 \leq j < \nrules}}
    %           {\aematchwith{\nrules}{e''}{\seqschemaX{r'}}}{\tau}$ \BY{Rule (\ref{rule:hastypeUP-match}) on \pfref{c5scrutf} and \pfref{c5rf}} \pflabel{c5ff}
  \end{pfsteps*} 

  The conclusions hold as follows:
  \begin{enumerate}
    \item $\text{\pfref{c1scrut}} \cup \bigcup_{0 \leq j < \nrules} \text{\pfref{c1r}}_j$
    \item $\text{\pfref{c2scrut}} \cup \bigcup_{0 \leq j < \nrules} \text{\pfref{c2r}}_j$
    \item $\text{\pfref{c3scrut}} \cup \bigcup_{0 \leq j < \nrules} \text{\pfref{c3r}}_j$
    \item Choose:
      \begin{enumerate}
              \item $\aematchwith{\nrules}{e''}{\seqschemaX{r'}}$
      \item $\sseq{t_i}{\nty'} \cup \{ \sseq{t_{i,j}}{\ntyj} \}_{0 \leq j < \nrules}$; and
      \item $\sseq{x_i}{\nexp'} \cup \{ \sseq{x_{i,j}}{\nexpj} \}_{0 \leq j < \nrules}$; and 
      \end{enumerate}
      We have $e = [\sseq{\tau'_i/t_i}{\nty'} \cup \{ \sseq{\tau_{i,j}/t_{i,j}}{\ntyj} \}_{0 \leq j < \nrules}, \sseq{e_i/x_i}{\nexp'} \cup \{ \sseq{e_{i,j}/x_{i,j}}{\nexpj} \}_{0 \leq j < \nrules}]\aematchwith{\nrules}{e''}{\seqschemaX{r'}}$ by \pfref{c4e}.
    \item \pfref{c5ff}
  \end{enumerate}
  \resetpfcounter
\end{byCases}
\end{grayparbox}
\end{enumerate}
\vspace{-2px}\begin{grayparbox}\vspace{2px}
\begin{enumerate}
\item[2.] By rule induction over the rule typing assumption. There is only one case. In the following, let $\uDelta=\uDD{\uD}{\Delta_\text{app}}$ and $\uGamma=\uGG{\uG}{\Gamma_\text{app}}$ and $\escenev=\esceneUP{\uDelta}{\uGamma}{\uPsi}{\uPhi}{b}$.
  \begin{byCases}
    \item[\text{(\ref{rule:cvalidR-UP})}] ~
    \begin{pfsteps*}
      \item $\crv = \acematchrule{p}{\ce}$ \BY{assumption}
      \item $r=\aematchrule{p}{e}$ \BY{assumption}
      \item $\patType{\pctx'}{p}{\tau}$ \BY{assumption} \pflabel{patType}
      \item $\cvalidE{\Delta}{\Gcons{\Gamma}{\pctx'}}{\escenev}{\ce}{e}{\tau'}$ \BY{assumption} \pflabel{cvalid}
      \item $\segof{\crv} = \segof{\ce}$ \BY{definition}
      \item $\segof{\ce} = \sseq{\acesplicedt{m'_i}{n'_i}}{\nty} \cup \sseq{\acesplicede{m_i}{n_i}{\ctau_i}}{\nexp}$ \BY{definition} \pflabel{summaryOfce}
      \item $\sseq{
            \expandsTU{\uDD{\uD}{\Delta_\text{app}}}
            {
              \parseUTypF{\bsubseq{b}{m'_i}{n'_i}}
            }{\tau'_i}
          }{\nty}$
          \BY{IH, part 1 on \pfref{cvalid} and \pfref{summaryOfce}}
          \pflabel{c1e}
      \item $\sseq{
        \cvalidT{\emptyset}{
          \tsceneUP
            {\uDD
              {\uD}{\Delta_\text{app}}
            }{b}
        }{
          \ctau_i
        }{\tau_i}
      }{\nexp}$
                \BY{IH, part 1 on \pfref{cvalid} and \pfref{summaryOfce}}
          \pflabel{c2e}
      \item $\sseq{
        \expandsSG
          {\uDD{\uD}{\Delta_\text{app}}}
          {\uGG{\uG}{\Gamma_\text{app}}}
          {\uPsi}
          {\uPhi}
          {\parseUExpF{\bsubseq{b}{m_i}{n_i}}}
          {e_i}
          {\tau_i}
      }{\nexp}$
          \BY{IH, part 1 on \pfref{cvalid} and \pfref{summaryOfce}}
          \pflabel{c3e}      
      \item $e = [\sseq{\tau'_i/t_i}{\nty}, \sseq{e_i/x_i}{\nexp}]e'$ for some $e'$ and fresh $\sseq{t_i}{\nty}$ and fresh $\sseq{x_i}{\nexp}$
          \BY{IH, part 1 on \pfref{cvalid} and \pfref{summaryOfce}}
          \pflabel{c4e}
      % \item $\hastypeU
      %   {\Delta \cup \sseq{\Dhyp{t_i}}{\nty}}
      %   {\Gamma \cup \pctx' \cup \sseq{x_i : \tau_i}{\nexp}}
      %   {e'}{\tau}$
      %     \BY{IH, part 1 on \pfref{cvalid} and \pfref{summaryOfce}}
      \item  $\mathsf{fv}(e') \subset \domof{\Delta} \cup \domof{\Gamma} \cup \domof{\Gamma'} \cup \sseq{t_i}{\nty} \cup \sseq{x_i}{\nexp}$ \BY{IH, part 1 on \pfref{cvalid} and \pfref{summaryOfce}}
          \pflabel{c5e}
      \item $r=[\sseq{\tau'_i/t_i}{\nty}, \sseq{e_i/x_i}{\nexp}]\aematchrule{p}{e'}$ \BY{substitution properties and \pfref{c4e}} \pflabel{c4r}
      \item $\fvof{p} = \domof{\Gamma'}$ \BY{Lemma \ref{lemma:pattern-regularity-UP} on \pfref{patType}}\pflabel{fvofp}
         \item  $\mathsf{fv}(\aematchrule{p}{e'}) \subset \domof{\Delta} \cup \domof{\Gamma} \cup \sseq{t_i}{\nty} \cup \sseq{x_i}{\nexp}$ \BY{definition of $\fvof{r}$ and \pfref{c5e} and \pfref{fvofp}}
          \pflabel{c5r}
      % \item $\ruleType
      %         {\Delta \cup \sseq{\Dhyp{t_i}}{\nty}}
      %         {\Gamma \cup \sseq{x_i : \tau_i}{\nexp}}
      %         {\aematchrule{p}{e'}}
      %         {\tau}{\tau'}$ \BY{Rule (\ref{rule:ruleType}) on \pfref{patType} and \pfref{c5e}} \pflabel{c5r}
    \end{pfsteps*}
    The conclusions hold as follows:
    \begin{enumerate}
      \item \pfref{c1e}
      \item \pfref{c2e}
      \item \pfref{c3e}
      \item Choosing $\aematchrule{p}{e'}$ and $\sseq{t_i}{\nty}$ and $\sseq{x_i}{\nexp}$, by \pfref{c4r}
      \item \pfref{c5r}
    \end{enumerate}
    \resetpfcounter
  \end{byCases}
\end{enumerate}
\end{grayparbox}
\end{proof}
% \begin{equation}\label{rule:cvalidR-UP}
% \inferrule{
%   \patType{\pctx}{p}{\tau}\\
%   \cvalidE{\Delta}{\Gcons{\Gamma}{\pctx}}{\escenev}{\ce}{e}{\tau'}
% }{
%   \cvalidR{\Delta}{\Gamma}{\escenev}{\acematchrule{p}{\ce}}{\aematchrule{p}{e}}{\tau}{\tau'}
% }
% \end{equation}

\begin{theorem}[seTLM Abstract Reasoning Principles]
\label{thm:tsc-SES}
If $\expandsSG{\uDD{\uD}{\Delta}}{\uGG{\uG}{\Gamma}}{\uPsi}{\uPhi}{\utsmap{\tsmv}{b}}{e}{\tau}$ then:
\begin{enumerate}
\item (\textbf{Expansion Typing}) $\uPsi = \uPsi', \uShyp{\tsmv}{x}{\tau}{\eparse}$ and $\hastypeU{\Delta}{\Gamma}{e}{\tau}$
\item (\textbf{Responsibility}) $\encodeBody{b}{\ebody}$ and $\evalU{\ap{\eparse}{\ebody}}{\aein{\mathtt{SuccessE}}{\ecand}}$ and $\decodeCondE{\ecand}{\ce}$
\item (\textbf{Segmentation}) $\segOK{\segof{\ce}}{b}$
\item (\textbf{Segment Typing}) $\segof{\ce} = \sseq{\acesplicedt{m'_i}{n'_i}}{\nty} \cup \sseq{\acesplicede{m_i}{n_i}{\ctau_i}}{\nexp}$ and 
\begin{enumerate}
\item $\sseq{
      \expandsTU{\uDD{\uD}{\Delta}}
      {
        \parseUTypF{\bsubseq{b}{m'_i}{n'_i}}
      }{\tau'_i}
    }{\nty}$ and $\sseq{\istypeU{\Delta}{\tau'_i}}{\nty}$
\item $\sseq{
  \cvalidT{\emptyset}{
    \tsceneUP
      {\uDD
        {\uD}{\Delta}
      }{b}
  }{
    \ctau_i
  }{\tau_i}
}{\nexp}$ and $\sseq{\istypeU{\Delta}{\tau_i}}{\nexp}$
\item $\sseq{
  \expandsSG
    {\uDD{\uD}{\Delta}}
    {\uGG{\uG}{\Gamma}}
    {\uPsi}
    {\uPhi}
    {\parseUExpF{\bsubseq{b}{m_i}{n_i}}}
    {e_i}
    {\tau_i}
}{\nexp}$ and $\sseq{\hastypeU{\Delta}{\Gamma}{e_i}{\tau_i}}{\nexp}$
\end{enumerate}
\item (\textbf{Capture Avoidance}) $e = [\sseq{\tau'_i/t_i}{\nty}, \sseq{e_i/x_i}{\nexp}]e'$ for some $\sseq{t_i}{\nty}$ and $\sseq{x_i}{\nexp}$ and $e'$
\item (\textbf{Context Independence}) $\mathsf{fv}(e') \subset \sseq{t_i}{\nty} \cup \sseq{x_i}{\nexp}$
  % $\hastypeU
  % {\sseq{\Dhyp{t_i}}{\nty}}
  % {\sseq{x_i : \tau_i}{\nexp}}
  % {e'}{\tau}$
\end{enumerate}
\end{theorem}
\begin{proof} By rule induction over Rules (\ref{rules:expandsU}). There is only one rule that applies. In the following, let $\uDelta = \uDD{\uD}{\Delta}$ and $\uGamma = \uGG{\uG}{\Gamma}$.
\begin{byCases}
\item[\text{(\ref{rule:expandsU-tsmap})}] ~
\begin{pfsteps*}
  \item $\uPsi = \uPsi', \uShyp{\tsmv}{x}{\tau}{\eparse}$ \BY{assumption} \pflabel{uPsidef}
  \item $\Gamma = \Gamma', x  : \taudep$ \BY{assumption} \pflabel{depgamma}
  \item $e = \aeap{e_x}{x}$ \BY{assumption} \pflabel{concform}
  \item $\expandsSG{\uDD{\uD}{\Delta}}{\uGG{\uG}{\Gamma}}{\uPsi}{\uPhi}{\utsmap{\tsmv}{b}}{e}{\tau}$ \BY{assumption} \pflabel{expandsSG}
  \item $\hastypeU{\Delta}{\Gamma}{e}{\tau}$ \BY{Theorem \ref{thm:typed-expansion-short-U} on \pfref{expandsSG}} \pflabel{hastype}
  \item $\encodeBody{b}{\ebody}$ \BY{assumption} \pflabel{encodeBody}
  \item $\evalU{\eparse(\ebody)}{\aein{\mathtt{SuccessE}}{\ecand}}$ \BY{assumption} \pflabel{evalU}
  \item $\decodeCondE{\ecand}{\ce}$ \BY{assumption} \pflabel{decodeCondE}
  \item $\segOK{\segof{\ce}}{b}$ \BY{assumption} \pflabel{segOK}
  \item $\cvalidE{\emptyset}{\emptyset}{\esceneSG{\uDelta}{\uGamma}{\uPsi}{\uPhi}{b}}{\ce}{e_x}{\aparr{\taudep}{\tau}}$ \BY{assumption}\pflabel{cvalidE}
  \item $\segof{\ce} = \sseq{\acesplicedt{m'_i}{n'_i}}{\nty} \cup \sseq{\acesplicede{m_i}{n_i}{\ctau_i}}{\nexp}$ \BY{definition} \pflabel{summaryOf}
  \item $\sseq{
      \expandsTU{\uDD{\uD}{\Delta}}
      {
        \parseUTypF{\bsubseq{b}{m'_i}{n'_i}}
      }{\tau'_i}
    }{\nty}$ \BY{Lemma \ref{thm:proto-expression-expansion-decomposition} on \pfref{cvalidE} and \pfref{summaryOf}} \pflabel{c1}
\item $\sseq{\istypeU{\Delta}{\tau'_i}}{\nty}$ \BY{Lemma \ref{lemma:type-expansion-U}, part 1 over \pfref{c1}} \pflabel{t1}
\item $\sseq{
  \cvalidT{\emptyset}{
    \tsceneUP
      {\uDD
        {\uD}{\Delta}
      }{b}
  }{
    \ctau_i
  }{\tau_i}
}{\nexp}$ \BY{Lemma \ref{thm:proto-expression-expansion-decomposition} on \pfref{cvalidE} and \pfref{summaryOf}} \pflabel{c2}
\item $\emptyset \cap \Delta = \emptyset$ \BY{definition} \pflabel{emptyintersect}
\item $\sseq{\istypeU{\Delta}{\tau_i}}{\nexp}$ \BY{Lemma \ref{lemma:type-expansion-U}, part 2 over \pfref{c2} and \pfref{emptyintersect}} \pflabel{t2}
\item $\sseq{
  \expandsSG
    {\uDD{\uD}{\Delta}}
    {\uGG{\uG}{\Gamma}}
    {\uPsi}
    {\uPhi}
    {\parseUExpF{\bsubseq{b}{m_i}{n_i}}}
    {e_i}
    {\tau_i}
}{\nexp}$ \BY{Lemma \ref{thm:proto-expression-expansion-decomposition} on \pfref{cvalidE} and \pfref{summaryOf}} \pflabel{c3}
\item $\sseq{\hastypeU{\Delta}{\Gamma}{e_i}{\tau_i}}{\nexp}$ \BY{Theorem \ref{thm:typed-expansion-short-U} over \pfref{c3}} \pflabel{t3}
\item $e_x = [\sseq{\tau'_i/t_i}{\nty}, \sseq{e_i/x_i}{\nexp}]e'$ for some $e'$ and fresh $\sseq{t_i}{\nty}$ and fresh $\sseq{x_i}{\nexp}$ \BY{Lemma \ref{thm:proto-expression-expansion-decomposition} on \pfref{cvalidE} and \pfref{summaryOf}} \pflabel{c4}
\item $\fvof{e'} \subset \sseq{t_i}{\nty} \cup \sseq{x_i}{\nexp}$ \BY{Lemma \ref{thm:proto-expression-expansion-decomposition} on \pfref{cvalidE} and \pfref{summaryOf}} \pflabel{c5}
\item $e = [\sseq{\tau'_i/t_i}{\nty}, \sseq{e_i/x_i}{\nexp}]\aeap{e'}{x}$ \BY{definition of substitution on \pfref{c4}} \pflabel{c42}
% \item $\hastypeU
%   {\sseq{\Dhyp{t_i}}{\nty}}
%   {\sseq{x_i : \tau_i}{\nexp}}
%   {e'}{\tau}$ \BY{Lemma \ref{thm:proto-expression-expansion-decomposition} on \pfref{cvalidE} and \pfref{summaryOf}} \pflabel{c5}
\end{pfsteps*}
The conclusions hold as follows:
\begin{enumerate}
  \item \pfref{uPsidef} and \pfref{hastype}
  \item \pfref{encodeBody} and \pfref{evalU} and \pfref{decodeCondE}
  \item \pfref{segOK}
  \item \pfref{summaryOf} and 
  \begin{enumerate}
  \item \pfref{c1} and \pfref{t1}
  \item \pfref{c2} and \pfref{t2}
  \item \pfref{c3} and \pfref{t3}
  \end{enumerate}
  \item \pfref{c42}
  \item \pfref{c5}
\end{enumerate}
\resetpfcounter
\end{byCases}
\end{proof}

% The following theorem establishes a prohibition on \textbf{Shadowing} as discussed in Sec. \ref{sec:uetsms-validation}.

% \begin{theorem}[Shadowing Prohibition]
% \label{thm:shadowing-prohibition-SES} ~
% \begin{enumerate}
% \item If $\cvalidT{\Delta}{\tsceneU{\uDD{\uD}{\Delta_\text{app}}}{b}}{\acesplicedt{m}{n}}{\tau}$ then:\begin{enumerate}
% \item $\parseUTyp{\bsubseq{b}{m}{n}}{\utau}$
% \item $\expandsTU{\uDD{\uD}{\Delta_\text{app}}}{\utau}{\tau}$
% \item $\Delta \cap \Delta_\text{app} = \emptyset$
% \end{enumerate}
% \item If $\cvalidE{\Delta}{\Gamma}{\escenev}{\acesplicede{m}{n}{\ctau}}{e}{\tau}$ then:
% \begin{enumerate}
% \item $\cvalidT{\emptyset}{\tsfrom{\escenev}}{\ctau}{\tau}$
% \item $  \escenev=\esceneU{\uDD{\uD}{\Delta_\text{app}}}{\uGG{\uG}{\Gamma_\text{app}}}{\uPsi}{b}$
% \item $\parseUExp{\bsubseq{b}{m}{n}}{\ue}$
% \item $\expandsU{\uDD{\uD}{\Delta_\text{app}}}{\uGG{\uG}{\Gamma_\text{app}}}{\uPsi}{\ue}{e}{\tau}$
% \item $\Delta \cap \Delta_\text{app} = \emptyset$
% \item $\domof{\Gamma} \cap \domof{\Gamma_\text{app}} = \emptyset$
% \end{enumerate}
% \end{enumerate}
% \end{theorem}
% \begin{proof} ~
% \begin{enumerate}
% \item By rule induction over Rules (\ref{rules:cvalidT-U}). The only rule that applies is Rule (\ref{rule:cvalidT-U-splicedt}). The conclusions are the premises of this rule.
% \item By rule induction over Rules (\ref{rules:cvalidE-U}). The only rule that applies is Rule (\ref{rule:cvalidE-U-splicede}). The conclusions are the premises of this rule.
% \end{enumerate}
% \end{proof}

\begin{grayparbox}
\begin{lemma}[Proto-Pattern Expansion Decomposition]
\label{lemma:proto-pattern-expansion-decomposition-S}
If $\cvalidP{\upctx}{\pscene{\uDelta}{\uPhi}{b}}{\cpv}{p}{\tau}$ where  
\[ 
\segof{\cpv} = \sseq{\acesplicedt{m'_i}{n'_i}}{\nty} \cup \sseq{\acesplicedp{m_i}{n_i}{\ctau_i}}{\npat}
\]
then all of the following hold:
\begin{enumerate}
    \item $\sseq{
          \expandsTU{\uDelta}
          {
            \parseUTypF{\bsubseq{b}{m'_i}{n'_i}}
          }{\tau'_i}
        }{\nty}$
    \item $\sseq{
      \cvalidT{\emptyset}{
        \tsceneUP
          {\uDelta}{b}
      }{
        \ctau_i
      }{\tau_i}
    }{\npat}$
    \item $\sseq{
      \patExpands
        {\upctx_i}
        {\uPhi}
        {\parseUPatF{\bsubseq{b}{m_i}{n_i}}}
        {p_i}
        {\tau_i}
    }{\npat}$
  \item $\upctx = \biguplus_{0 \leq i < \npat} \upctx_i$
\end{enumerate}
\end{lemma}
\begin{proof} By rule induction over Rules (\ref{rules:cvalidP-UP}). In the following, let $\pscenev=\pscene{\uDelta}{\uPhi}{b}$.
\begin{byCases}
  \item[\text{(\ref{rule:cvalidP-UP-wild})}] ~
    \begin{pfsteps*}
      \item $\cpv=\acewildp$ \BY{assumption}
      \item $e = \aewildp$ \BY{assumption}
      \item $\upctx = \uGG{\emptyset}{\emptyset}$ \BY{assumption}
      \item $\segof{\acewildp} = \emptyset$ \BY{definition}
    \end{pfsteps*}
    The conclusions hold as follows:
    \begin{enumerate}
      \item This conclusion holds trivially because $\nty=0$.
      \item This conclusion holds trivially because $\npat=0$.
      \item This conclusion holds trivially because $\npat=0$.
      \item This conclusion holds trivially because $\upctx=\emptyset$ and $\npat=0$.
    \end{enumerate}
    \resetpfcounter
  \item[\text{(\ref{rule:cvalidP-UP-fold})}] ~
    \begin{pfsteps*}
      \item $\cpv=\acefoldp{\cpv'}$ \BY{assumption}
      \item $p=\aefoldp{p'}$ \BY{assumption}
      \item $\tau=\arec{t}{\tau'}$ \BY{assumption}
      \item $\cvalidP{\upctx}{\pscenev}{\cpv}{p}{[\arec{t}{\tau'}/t]\tau'}$ \BY{assumption} \pflabel{cvalidP}
      \item $\segof{\acefoldp{\cpv'}} = \segof{\cpv'}$ \BY{definition} \pflabel{summaryOf}
      \item $\segof{\cpv'} = \sseq{\acesplicedt{m'_i}{n'_i}}{\nty} \cup \sseq{\acesplicedp{m_i}{n_i}{\ctau_i}}{\npat}$ \BY{definition} \pflabel{summaryOf2}
      \item $\sseq{
            \expandsTU{\uDelta}
            {
              \parseUTypF{\bsubseq{b}{m'_i}{n'_i}}
            }{\tau'_i}
          }{\nty}$ \BY{IH on \pfref{cvalidP} and \pfref{summaryOf2}} \pflabel{c1}
      \item $\sseq{
        \cvalidT{\emptyset}{
          \tsceneUP
            {\uDelta}{b}
        }{
          \ctau_i
        }{\tau_i}
      }{\npat}$ \BY{IH on \pfref{cvalidP} and \pfref{summaryOf2}} \pflabel{c2}
      \item $\sseq{
        \patExpands
          {\upctx_i}
          {\uPhi}
          {\parseUPatF{\bsubseq{b}{m_i}{n_i}}}
          {p_i}
          {\tau_i}
      }{\npat}$ \BY{IH on \pfref{cvalidP} and \pfref{summaryOf2}} \pflabel{c3}
    \item $\upctx = \biguplus_{0 \leq i < \npat} \upctx_i$ \BY{IH on \pfref{cvalidP} and \pfref{summaryOf2}}\pflabel{c4}\end{pfsteps*}
    The conclusions hold as follows:
    \begin{enumerate}
    \item \pfref{c1}
    \item \pfref{c2}
    \item \pfref{c3}
    \item \pfref{c4}
    \end{enumerate}
    \resetpfcounter
  \item[\text{(\ref{rule:cvalidP-UP-tpl})}] ~
    \begin{pfsteps*}
      \item $\cpv=\acetplp{\labelset}{\mapschema{\cpv}{j}{\labelset}}$ \BY{assumption}
      \item $p=\aetplp{\labelset}{\mapschema{p}{j}{\labelset}}$ \BY{assumption}
      \item $\tau = \aprod{\labelset}{\mapschema{\tau}{j}{\labelset}}$ \BY{assumption}
      \item $\upctx=\biguplus_{j \in \labelset}{\upctx_j}$ \BY{assumption} \pflabel{Gconsi}
      \item $\{\cvalidP{\upctx_j}{\pscenev}{\cpv_j}{p_j}{\tau_j}\}_{j \in \labelset}$ \BY{assumption} \pflabel{cvalidP}
      \item $\segof{\acetplp{\labelset}{\mapschema{\cpv}{j}{\labelset}}} = \bigcup_{j \in \labelset} \segof{\cpv_j}$ \BY{definition}
      \item $\{ \segof{\cpv_j} = \sseq{\acesplicedt{m'_{i,j}}{n'_{i,j}}}{\ntyj} \cup \sseq{\acesplicedp{m_{i,j}}{n_{i,j}}{\ctau_{i,j}}}{\npatj} \}_{j \in \labelset}$ \BY{definition}\pflabel{summaryOf2}
      \item $\npat = \Sigma_{j \in \labelset} \npatj$ \BY{definition}
      \item $\{\sseq{
            \expandsTU{\uDelta}
            {
              \parseUTypF{\bsubseq{b}{m'_{i,j}}{n'_{i,j}}}
            }{\tau'_{i,j}}
          }{\ntyj}\}_{j \in \labelset}$ \BY{IH over \pfref{cvalidP} and \pfref{summaryOf2}} \pflabel{c1}
      \item $\{\sseq{
        \cvalidT{\emptyset}{
          \tsceneUP
            {\uDelta}{b}
        }{
          \ctau_{i,j}
        }{\tau_{i,j}}
      }{\npatj}\}_{j \in \labelset}$ \BY{IH over \pfref{cvalidP} and \pfref{summaryOf2}} \pflabel{c2}
      \item $\{\sseq{
        \patExpands
          {\upctx_{i,j}}
          {\uPhi}
          {\parseUPatF{\bsubseq{b}{m_{i,j}}{n_{i,j}}}}
          {p_{i,j}}
          {\tau_{i,j}}
      }{\npatj}\}_{j \in \labelset}$ \BY{IH over \pfref{cvalidP} and \pfref{summaryOf2}} \pflabel{c3}
    \item $\{\upctx_j = \biguplus_{0 \leq i < \npatj} \upctx_{i,j}\}_{j \in \labelset}$ \BY{IH over \pfref{cvalidP} and \pfref{summaryOf2}}\pflabel{c4}
    \item $\biguplus_{j \in \labelset} \upctx_{j} = \biguplus_{j \in \labelset} \biguplus_{i \in \npatj} \upctx_{i,j}$ \BY{definition and \pfref{c4}}\pflabel{c4x}
    \end{pfsteps*}
    The conclusions hold as follows:
    \begin{enumerate}
      \item $\bigcup_{j \in \labelset} \bigcup_{i \in \ntyj} \text{\pfref{c1}}_{i,j}$
      \item $\bigcup_{j \in \labelset} \bigcup_{i \in \npatj} \text{\pfref{c2}}_{i,j}$
      \item $\bigcup_{j \in \labelset} \bigcup_{i \in \npatj} \text{\pfref{c3}}_{i,j}$
      \item \pfref{c4x}
    \end{enumerate}
    \resetpfcounter
  \item[\text{(\ref{rule:cvalidP-UP-in})}]
    \begin{pfsteps*}
      \item $\cpv=\aceinjp{\ell}{\cpv'}$ \BY{assumption}
      \item $p=\aeinjp{\ell}{p'}$ \BY{assumption}
      \item $\tau=\asum{\labelset, \ell}{\mapschema{\tau}{i}{\labelset}; \mapitem{\ell}{\tau'}}$ \BY{assumption}
      \item $\cvalidP{\upctx}{\pscenev}{\cpv}{p}{\tau'}$ \BY{assumption} \pflabel{cvalidP}
      \item $\segof{\aceinjp{\ell}{\cpv'}} = \segof{\cpv'}$ \BY{definition} \pflabel{summaryOf}
      \item $\segof{\cpv'} = \sseq{\acesplicedt{m'_i}{n'_i}}{\nty} \cup \sseq{\acesplicedp{m_i}{n_i}{\ctau_i}}{\npat}$ \BY{definition} \pflabel{summaryOf2}
      \item $\sseq{
            \expandsTU{\uDelta}
            {
              \parseUTypF{\bsubseq{b}{m'_i}{n'_i}}
            }{\tau'_i}
          }{\nty}$ \BY{IH on \pfref{cvalidP} and \pfref{summaryOf2}} \pflabel{c1}
      \item $\sseq{
        \cvalidT{\emptyset}{
          \tsceneUP
            {\uDelta}{b}
        }{
          \ctau_i
        }{\tau_i}
      }{\npat}$ \BY{IH on \pfref{cvalidP} and \pfref{summaryOf2}} \pflabel{c2}
      \item $\sseq{
        \patExpands
          {\upctx_i}
          {\uPhi}
          {\parseUPatF{\bsubseq{b}{m_i}{n_i}}}
          {p_i}
          {\tau_i}
      }{\npat}$ \BY{IH on \pfref{cvalidP} and \pfref{summaryOf2}} \pflabel{c3}
    \item $\upctx = \biguplus_{0 \leq i < \npat} \upctx_i$ \BY{IH on \pfref{cvalidP} and \pfref{summaryOf2}}\pflabel{c4}\end{pfsteps*}
    The conclusions hold as follows:
    \begin{enumerate}
    \item \pfref{c1}
    \item \pfref{c2}
    \item \pfref{c3}
    \item \pfref{c4}
    \end{enumerate}
    \resetpfcounter
  \item[\text{(\ref{rule:cvalidP-UP-spliced})}] ~
    \begin{pfsteps*}
      \item $\cpv=\acesplicedp{m}{n}{\ctau}$ \BY{assumption}
      \item $\cvalidT{\emptyset}{\tsceneUP{\uDelta}{b}}{\ctau}{\tau}$ \BY{assumption} \pflabel{cvalidT}
      \item $\parseUPat{\bsubseq{b}{m}{n}}{\upv}$ \BY{assumption} \pflabel{parseUPat}
      \item $\patExpands{\upctx}{\uPhi}{\upv}{p}{\tau}$ \BY{assumption} \pflabel{patExpands}
      \item $\segof{\acesplicedp{m}{n}{\ctau}} = \segof{\ctau} \cup \{ \acesplicedp{m}{n}{\ctau} \}$ \BY{definition} \pflabel{summaryOf}
      \item $\segof{\ctau} = \sseq{\acesplicedt{m'_i}{n'_i}}{\nty}$ \BY{definition} \pflabel{summaryOf2}
      \item $\sseq{\expandsTU{\uDD{\uD}{\Delta_\text{app}}}{
  \parseUTypF{\bsubseq{b}{m_i}{n_i}}
}{\tau_i}}{n}$ \BY{Lemma \ref{thm:proto-type-expansion-decomposition-SES} on \pfref{cvalidT} and \pfref{summaryOf2}} \pflabel{expandsTU}
% \item $\sseq{\istypeU{\Delta_\text{app}}{\tau_i}}{n}$
    \end{pfsteps*}
    The conclusions hold as follows:
    \begin{enumerate}
      \item \pfref{expandsTU}
      \item \pfref{cvalidT}
      \item \pfref{parseUPat} and \pfref{patExpands}
      \item This conclusion holds by \pfref{patExpands} because $\npat=1$.
    \end{enumerate}
    \resetpfcounter
\end{byCases}
\end{proof}

\begin{theorem}[spTLM Abstract Reasoning Principles]
\label{thm:spTLM-Typing-Segmentation}
If $\patExpands{\upctx}{\uPhi}{\utsmap{\tsmv}{b}}{p}{\tau}$ where $\uDelta=\uDD{\uD}{\Delta}$ and $\uGamma=\uGG{\uG}{\Gamma}$ then all of the following hold:
\begin{enumerate}
        \item (\textbf{Expansion Typing}) $\uPhi=\uPhi', \uPhyp{\tsmv}{x}{\tau}{\eparse}$ and $\patType{\pctx}{p}{\tau}$
        \item (\textbf{Responsibility}) $\encodeBody{b}{\ebody}$ and $\evalU{\eparse(\ebody)}{\aein{\mathtt{SuccessP}}{\ecand}}$ and $\decodeCEPat{\ecand}{\cpv}$
        \item (\textbf{Segmentation}) $\segOK{\segof{\cpv}}{b}$
        \item (\textbf{Segment Typing}) $\segof{\cpv} = \sseq{\acesplicedt{n'_i}{m'_i}}{\nty} \cup \sseq{\acesplicedp{m_i}{n_i}{\ctau_i}}{\npat}$ and 
        \begin{enumerate}
        \item  $\sseq{
              \expandsTU{\uDelta}
              {
                \parseUTypF{\bsubseq{b}{m'_i}{n'_i}}
              }{\tau'_i}
            }{\nty}$ and $\sseq{\istypeU{\Delta}{\tau'_i}}{\nty}$
        \item $\sseq{
          \cvalidT{\emptyset}{
            \tsceneUP
              {\uDelta}{b}
          }{
            \ctau_i
          }{\tau_i}
        }{\npat}$ and $\sseq{\istypeU{\Delta}{\tau_i}}{\npat}$
        \item $\sseq{
          \patExpands
            {\uGG{\uG_i}{\pctx_i}}
            {\uPhi}
            {\parseUPatF{\bsubseq{b}{m_i}{n_i}}}
            {p_i}
            {\tau_i}
        }{\npat}$  and $\sseq{\patType{\pctx_i}{p_i}{\tau_i}}{\npat}$
      \end{enumerate}
      \item (\textbf{Visibility}) $\uG = \biguplus_{0 \leq i < \npat} \uG_i$ and $\Gamma = \bigcup_{0 \leq i < \npat} \pctx_i$
\end{enumerate}
\end{theorem}
\begin{proof} By rule induction over Rules (\ref{rules:patExpands}). There is only one rule that applies.
\begin{byCases}
  \item[\text{(\ref{rule:patExpands-apuptsm})}] 
    \begin{pfsteps*}
      \item $\patExpands{\upctx}{\uPhi}{\utsmap{\tsmv}{b}}{p}{\tau}$ \BY{assumption} \pflabel{patExpands}
      \item $\uPhi = \uPhi', \uPhyp{\tsmv}{x}{\tau}{\eparse}$ \BY{assumption} \pflabel{uPhidef}
      \item $\patType{\pctx}{p}{\tau}$ \BY{Theorem \ref{thm:typed-pattern-expansion} on \pfref{patExpands}} \pflabel{patType}
      \item $\encodeBody{b}{\ebody}$ \BY{assumption} \pflabel{encodeBody}
      \item $\evalU{\ap{\eparse}{\ebody}}{\aein{\mathtt{SuccessP}}{\ecand}}$ \BY{assumption} \pflabel{evalU}
      \item $\decodeCEPat{\ecand}{\cpv}$ \BY{assumption} \pflabel{decodeCEPat}
      \item $\segOK{\segof{\cpv}}{b}$ \BY{assumption} \pflabel{segOK}
      \item $\cvalidP{\upctx}{\pscene{\uDelta}{\uPhi}{b}}{\cpv}{p}{\tau}$ \BY{assumption} \pflabel{cvalidP}
      \item $\segof{\cpv} = \sseq{\acesplicedt{m'_i}{n'_i}}{\nty} \cup \sseq{\acesplicedp{m_i}{n_i}}{\npat}$ \BY{definition}\pflabel{summaryOf}
      \item $\sseq{
            \expandsTU{\uDelta}
            {
              \parseUTypF{\bsubseq{b}{m'_i}{n'_i}}
            }{\tau'_i}
          }{\nty}$ \BY{Lemma \ref{lemma:proto-pattern-expansion-decomposition-S} on \pfref{cvalidP} and \pfref{summaryOf}}\pflabel{c1}
      \item $\sseq{\istypeU{\Delta}{\tau'_i}}{\nty}$ \BY{Lemma \ref{lemma:type-expansion-U}, part 1 over \pfref{c1}}\pflabel{t1}
      \item $\sseq{
        \cvalidT{\emptyset}{
          \tsceneUP
            {\uDelta}{b}
        }{
          \ctau_i
        }{\tau_i}
      }{\npat}$  \BY{Lemma \ref{lemma:proto-pattern-expansion-decomposition-S} on \pfref{cvalidP} and \pfref{summaryOf}}\pflabel{c2}
      \item $\sseq{\istypeU{\Delta}{\tau_i}}{\npat}$ \BY{Lemma \ref{lemma:type-expansion-U}, part 2 over \pfref{c2}}\pflabel{t2}
      \item $\sseq{
        \patExpands
          {\upctx_i}
          {\uPhi}
          {\parseUPatF{\bsubseq{b}{m_i}{n_i}}}
          {p_i}
          {\tau_i}
      }{\npat}$ \BY{Lemma \ref{lemma:proto-pattern-expansion-decomposition-S} on \pfref{cvalidP} and \pfref{summaryOf}}\pflabel{c3}
    \item $\sseq{\patType{\pctx_i}{p_i}{\tau_i}}{\npat}$ \BY{Theorem \ref{thm:typed-pattern-expansion} over \pfref{c3}}\pflabel{t3}
    \item $\uG = \biguplus_{0 \leq i < \npat} \uG_i$ and $\Gamma = \bigcup_{0 \leq i < \npat} \pctx_i$ \BY{Lemma \ref{lemma:proto-pattern-expansion-decomposition-S} on \pfref{cvalidP} and \pfref{summaryOf}}\pflabel{c4}

    \end{pfsteps*}
    The conclusions hold as follows:
    \begin{enumerate}
      \item \pfref{uPhidef} and \pfref{patType}
      \item \pfref{encodeBody} and \pfref{evalU} and \pfref{decodeCEPat}
      \item \pfref{segOK} 
      \item \pfref{summaryOf} and 
            \begin{enumerate}
      \item \pfref{c1} and \pfref{t1}
      \item \pfref{c2} and \pfref{t2}
      \item \pfref{c3} and \pfref{t3} 
    \end{enumerate}
      \item \pfref{c4}
    \end{enumerate}
    \resetpfcounter
\end{byCases}
\end{proof}
\end{grayparbox}

% \inferrule{
%   \uPhi = \uPhi', \uPhyp{\tsmv}{x}{\tau}{\eparse}\\\\
%   \encodeBody{b}{\ebody}\\
%   \evalU{\ap{\eparse}{\ebody}}{{\lbltxt{SuccessP}}\cdot{\ecand}}\\
%   \decodeCEPat{\ecand}{\cpv}\\\\
%     \segOK{\segof{\cpv}}{b}\\
%   \cvalidP{\upctx}{\pscene{\uDelta}{\uPhi}{b}}{\cpv}{p}{\tau}
% }{
%   \patExpands{\upctx}{\uPhi}{\utsmap{\tsmv}{b}}{p}{\tau}
% }}

% \ificfp
% \chapter{\texorpdfstring{A Calculus of Parametric TLMs}{A Calculus of Parametric TLMs}}
% \else
% \chapter{\texorpdfstring{$\miniVerseParam$}{miniVerseP}}
% \fi
% \label{appendix:miniVerseParam}

% \clearpage
% \section{Expanded Language (XL)}
% \subsection{Syntax}
% \subsubsection{Signatures and Module Expressions}
% \[\begin{array}{lllllll}
% \textbf{Sort} & & & \textbf{Operational Form} 
% %& \textbf{Stylized Form} 
% & \textbf{Description}\\
% \mathsf{Sig} & \sigma & ::= & \asignature{\kappa}{u}{\tau} 
% %& \signature{u}{\kappa}{\tau} 
% & \text{signature}\\
% \mathsf{Mod} & M & ::= & X 
% %& X 
% & \text{module variable}\\
% &&& \astruct{c}{e} 
% %& \struct{c}{e} 
% & \text{structure}\\
% &&& \aseal{\sigma}{M} 
% %& \seal{M}{\sigma} 
% & \text{seal}\\
% &&& \amlet{\sigma}{M}{X}{M} %& \mlet{X}{M}{M}{\sigma} 
% & \text{definition}
% \end{array}\]

% \subsubsection{Kinds and Constructions}
% \[\begin{array}{lrlllll}
% \textbf{Sort} & & & \textbf{Operational Form} 
% %& \textbf{Stylized Form} 
% & \textbf{Description}\\
% \mathsf{Kind} & \kappa & ::= & k & \text{kind variable}\\
% &&& \akdarr{\kappa}{u}{\kappa} 
% %& \kdarr{u}{\kappa}{\kappa} 
% & \text{dependent function}\\
% &&& \akunit 
% %& \kunit 
% & \text{nullary product}\\
% &&& \akdbprod{\kappa}{u}{\kappa} 
% %& \kdbprod{u}{\kappa}{\kappa} 
% & \text{dependent product}\\
% %&&& \akdprodstd & \kdprodstd & \text{labeled dependent product}\\
% &&& \akty 
% %& \kty
% & \text{type}\\
% &&& \aksing{\tau} 
% %& \ksing{\tau} 
% & \text{singleton}\\
% \mathsf{Con} & c, \tau & ::= & u 
% %& u 
% & \text{construction variable}\\
% &&& t 
% %& t 
% & \text{type variable}
% \\
% &&& \acabs{u}{c} 
% %& \cabs{u}{c} 
% & \text{abstraction}\\
% &&& \acapp{c}{c} 
% %& \capp{c}{c} 
% & \text{application}\\
% &&& \actriv 
% %& \ctriv 
% & \text{trivial}\\
% &&& \acpair{c}{c}
% % & \cpair{c}{c} 
% & \text{pair}\\
% &&& \acprl{c} 
% %& \cprl{c} 
% & \text{left projection}\\
% &&& \acprr{c} 
% %& \cprr{c} 
% & \text{right projection}\\
% %&&& \adtplX & \dtplX & \text{labeled dependent tuple}\\
% %&&& \adprj{\ell}{c} & \prj{c}{\ell} & \text{projection}\\
% &&& \aparr{\tau}{\tau} 
% %& \parr{\tau}{\tau} 
% & \text{partial function}\\
% &&& \aallu{\kappa}{u}{\tau} 
% %& \forallu{u}{\kappa}{\tau} 
% & \text{polymorphic}\\
% &&& \arec{t}{\tau} 
% %& \rect{t}{\tau} 
% & \text{recursive}\\
% &&& \aprod{\labelset}{\mapschema{\tau}{i}{\labelset}} 
% %& \prodt{\mapschema{\tau}{i}{\labelset}} 
% & \text{labeled product}\\
% &&& \asum{\labelset}{\mapschema{\tau}{i}{\labelset}} 
% %& \sumt{\mapschema{\tau}{i}{\labelset}} 
% & \text{labeled sum}\\
% &&& \amcon{M} 
% %& \mcon{M} 
% & \text{construction component}
% \end{array}\]
% \clearpage

% \subsubsection{Expressions, Rules and Patterns}
% \[\begin{array}{lllllll}
% \textbf{Sort} & & & \textbf{Operational Form} 
% %& \textbf{Stylized Form} 
% & \textbf{Description}\\
% \mathsf{Exp} & e & ::= & x 
% %& x 
% & \text{variable}\\
% &&& \aelam{\tau}{x}{e} 
% %& \lam{x}{\tau}{e} 
% & \text{abstraction}\\
% &&& \aeap{e}{e} 
% %& \ap{e}{e} 
% & \text{application}\\
% &&& \aeclam{\kappa}{u}{e} %& \clam{u}{\kappa}{e} 
% & \text{construction abstraction}\\
% &&& \aecap{e}{\kappa} %& \cAp{e}{\kappa} 
% & \text{construction application}\\
% &&& \aefold{e} %& \fold{e} 
% & \text{fold}\\
% &&& \aeunfold{e} %& \unfold{e} 
% & \text{unfold}\\
% &&& \aetpl{\labelset}{\mapschema{e}{i}{\labelset}} 
% %& \tpl{\mapschema{e}{i}{\labelset}} 
% & \text{labeled tuple}\\
% &&& \aepr{\ell}{e} 
% %& \prj{e}{\ell} 
% & \text{projection}\\
% &&& \aein{\ell}{e} 
% %& \inj{\ell}{e} 
% & \text{injection}\\
% &&& \aematchwith{n}{e}{\seqschemaX{r}} 
% %& \matchwith{e}{\seqschemaX{r}} 
% & \text{match}\\
% &&& \amval{M} 
% %& \mval{M} 
% & \text{value component}\\
% \mathsf{Rule} & r & ::= & \aematchrule{p}{e} 
% %& \matchrule{p}{e} 
% & \text{rule}\\
% \mathsf{Pat} & p & ::= & x 
% %& x 
% & \text{variable pattern}\\
% &&& \aewildp 
% %& \wildp 
% & \text{wildcard pattern}\\
% &&& \aefoldp{p} 
% %& \foldp{p} 
% & \text{fold pattern}\\
% &&& \aetplp{\labelset}{\mapschema{p}{i}{\labelset}} 
% %& \tplp{\mapschema{p}{i}{\labelset}} 
% & \text{labeled tuple pattern}\\
% &&& \aeinjp{\ell}{p} 
% %& \injp{\ell}{p} 
% & \text{injection pattern}
% \end{array}\]

% \subsection{Statics}\label{appendix:P-statics}
% \subsubsection{Unified Contexts}
% A \emph{unified context}, $\Omega$, is an ordered finite function. 
% We write
% \begin{itemize}
% \item $\Omega, X : \sigma$ when $X \notin \domof{\Omega}$ for the extension of $\Omega$ with a mapping from $X$ to the hypothesis $X : \sigma$.
% \item $\Omega, x : \tau$ when $x \notin \domof{\Omega}$ for the extension of $\Omega$ with a mapping from $x$ to the hypothesis $x : \tau$.
% \item $\Omega, u :: \kappa$ when $u \notin \domof{\Omega}$ for the extension of $\Omega$ with a mapping from $u$ to the hypothesis $u :: \kappa$.
% \end{itemize}

% % \begin{definition}[Unified Context Formation] $\isctxU{\Omega}$ iff:
% % \begin{enumerate}
% % \item $\Omega = \emptyset$; or
% % \item $\Omega = \Omega', X : \sigma$ and $\issig{\Omega'}{\sigma}$; or 
% % \item $\Omega = \Omega', u :: \kappa$ and $\iskind{\Omega'}{\kappa}$; or 
% % \item $\Omega = \Omega', x : \tau$ and $\isTypeP{\Omega'}{\tau}$.
% % \end{enumerate}
% % \end{definition}

% \subsubsection{Signatures and Structures}
% \noindent\fbox{$\strut\issigX{\sigma}$}~~$\sigma$ is a signature
% \begin{equation}\label{rule:issig}
% \inferrule{
%   \iskindX{\kappa}\\
%   \haskind{\Omega, u :: \kappa}{\tau}{\akty}
% }{
%   \issigX{\asignature{\kappa}{u}{\tau}}
% }
% \end{equation}

% \noindent\fbox{$\strut\sigequalX{\sigma}{\sigma'}$}~~$\sigma$ and $\sigma'$ are definitionally equal
% \begin{equation}\label{rule:sigequal}
% \inferrule{
%   \kequalX{\kappa}{\kappa'}\\
%   \cequal{\Omega, u :: \kappa}{\tau}{\tau'}{\akty}
% }{
%   \sigequalX{\asignature{\kappa}{u}{\tau}}{\asignature{\kappa'}{u}{\tau'}}
% }
% \end{equation}

% \noindent\fbox{$\strut\sigsubX{\sigma}{\sigma'}$}~~$\sigma$ is a subsignature of $\sigma'$
% \begin{equation}\label{rule:sigsub}
% \inferrule{
%   \ksubX{\kappa}{\kappa'}\\
%   \issubtypeP{\Omega, u :: \kappa}{\tau}{\tau'}
% }{
%   \sigsubX{\asignature{\kappa}{u}{\tau}}{\asignature{\kappa'}{u}{\tau'}}
% }
% \end{equation}

% \noindent\fbox{$\strut\hassigX{M}{\sigma}$}~~$M$ matches $\sigma$
% \begin{subequations}\label{rules:hassig}
% \begin{equation}\label{rule:hassig-subsume}
% \inferrule{
%   \hassigX{M}{\sigma}\\
%   \sigsubX{\sigma}{\sigma'}
% }{
%   \hassigX{M}{\sigma'}
% }
% \end{equation}
% \begin{equation}\label{rule:hassig-var}
% \inferrule{ }{
%   \hassig{\Omega, X : \sigma}{X}{\sigma}
% }
% \end{equation}
% \begin{equation}\label{rule:hassig-struct}
% \inferrule{
%   \haskindX{c}{\kappa}\\
%   \hastypeP{\Omega}{e}{[c/u]\tau}
% }{
%   \hassigX{\astruct{c}{e}}{\asignature{\kappa}{u}{\tau}}
% }
% \end{equation}
% \begin{equation}\label{rule:hassig-seal}
% \inferrule{
%   \issigX{\sigma}\\
%   \hassigX{M}{\sigma}
% }{
%   \hassigX{\aseal{\sigma}{M}}{\sigma}
% }
% \end{equation}
% \begin{equation}\label{rule:hassig-let}
% \inferrule{
%   \hassigX{M}{\sigma}\\
%   \issigX{\sigma'}\\
%   \hassig{\Omega, X : \sigma}{M'}{\sigma'}  
% }{
%   \hassigX{\amlet{\sigma'}{M}{X}{M'}}{\sigma'}
% }
% \end{equation}
% \end{subequations}

% \noindent\fbox{$\strut\ismvalX{M}$}~~$M$ is, or stands for, a module value
% \begin{subequations}\label{rules:ismval}
% \begin{equation}\label{rule:ismval-struct}
% \inferrule{ }{
%   \ismvalX{\astruct{c}{e}}
% }
% \end{equation}
% \begin{equation}\label{rule:ismval-var}
% \inferrule{ }{
%   \ismval{\Omega, X : \sigma}{X}
% }
% \end{equation}
% \end{subequations}

% \subsubsection{Kinds and Constructions}
% \noindent\fbox{$\strut\iskindX{\kappa}$}~~$\kappa$ is a kind
% \begin{subequations}\label{rules:iskind}
% \begin{equation}\label{rule:iskind-darr}
% \inferrule{
%   \iskindX{\kappa_1}\\
%   \iskind{\Omega, u :: \kappa_1}{\kappa_2}
% }{
%   \iskindX{\akdarr{\kappa_1}{u}{\kappa_2}}
% }
% \end{equation}
% \begin{equation}\label{rule:iskind-unit}
% \inferrule{ }{
%   \iskindX{\akunit}
% }
% \end{equation}
% \begin{equation}\label{rule:iskind-dprod}
% \inferrule{
%   \iskindX{\kappa_1}\\
%   \iskind{\Omega, u :: \kappa_1}{\kappa_2}
% }{
%   \iskindX{\akdbprod{\kappa_1}{u}{\kappa_2}}
% }
% \end{equation}
% \begin{equation}\label{rule:iskind-ty}
% \inferrule{ }{
%   \iskindX{\akty}
% }
% \end{equation}
% \begin{equation}\label{rule:iskind-sing}
% \inferrule{
%   \haskindX{\tau}{\akty}
% }{
%   \iskindX{\aksing{\tau}}
% }
% \end{equation}
% \end{subequations}

% \noindent\fbox{$\strut\kequalX{\kappa}{\kappa'}$}~~$\kappa$ and $\kappa'$ are definitionally equal
% \begin{subequations}\label{rules:kequal}
% \begin{equation}\label{rule:kequal-refl}
% \inferrule{
%   \iskindX{\kappa}
% }{
%   \kequalX{\kappa}{\kappa}
% }
% \end{equation}
% \begin{equation}\label{rule:kequal-sym}
% \inferrule{
%   \kequalX{\kappa}{\kappa'}
% }{
%   \kequalX{\kappa'}{\kappa}
% }
% \end{equation}
% \begin{equation}\label{rule:kequal-trans}
% \inferrule{
%   \kequalX{\kappa}{\kappa'}\\
%   \kequalX{\kappa'}{\kappa''}
% }{
%   \kequalX{\kappa}{\kappa''}
% }
% \end{equation}
% \begin{equation}\label{rule:kequal-darr}
% \inferrule{
%   \kequalX{\kappa_1}{\kappa_1'}\\
%   \kequal{\Omega, u :: \kappa_1}{\kappa_2}{\kappa_2'}
% }{
%   \kequalX{\akdarr{\kappa_1}{u}{\kappa_2}}{\akdarr{\kappa_1'}{u}{\kappa_2'}}
% }
% \end{equation}
% \begin{equation}\label{rule:kequal-dprod}
% \inferrule{
%   \kequalX{\kappa_1}{\kappa'_1}\\
%   \kequal{\Omega, u :: \kappa_1}{\kappa_2}{\kappa'_2}
% }{
%   \kequalX{\akdbprod{\kappa_1}{u}{\kappa_2}}{\akdbprod{\kappa'_1}{u}{\kappa'_2}}  
% }
% \end{equation}
% \begin{equation}\label{rule:kequal-sing}
% \inferrule{
%   \cequalX{c}{c'}{\akty}
% }{
%   \kequalX{\aksing{c}}{\aksing{c'}}
% }
% \end{equation}
% \end{subequations}

% \noindent\fbox{$\strut\ksubX{\kappa}{\kappa'}$}~~$\kappa$ is a subkind of $\kappa'$
% \begin{subequations}\label{rules:ksub}
% \begin{equation}\label{rule:ksub-equal}
% \inferrule{
%   \kequalX{\kappa}{\kappa'}
% }{
%   \ksubX{\kappa}{\kappa'}
% }
% \end{equation}
% \begin{equation}\label{rule:ksub-trans}
% \inferrule{
%   \ksubX{\kappa}{\kappa'}\\
%   \ksubX{\kappa'}{\kappa''}
% }{
%   \ksubX{\kappa}{\kappa''}
% }
% \end{equation}
% \begin{equation}\label{rule:ksub-darr}
% \inferrule{
%   \ksubX{\kappa'_1}{\kappa_1}\\
%   \ksub{\Omega, u :: \kappa'_1}{\kappa_2}{\kappa'_2}  
% }{
%   \ksubX{\akdarr{\kappa_1}{u}{\kappa_2}}{\akdarr{\kappa'_1}{u}{\kappa'_2}}
% }
% \end{equation}
% \begin{equation}\label{rule:ksub-dprod}
% \inferrule{
%   \ksubX{\kappa_1}{\kappa'_1}\\
%   \ksub{\Omega, u :: \kappa_1}{\kappa_2}{\kappa'_2}
% }{
%   \ksubX{\akdbprod{\kappa_1}{u}{\kappa_2}}{\akdbprod{\kappa'_1}{u}{\kappa'_2}}
% }
% \end{equation}
% \begin{equation}\label{rule:ksub-sing}
% \inferrule{
%   \haskindX{\tau}{\akty}
% }{
%   \ksubX{\aksing{\tau}}{\akty}
% }
% \end{equation}
% \begin{equation}\label{rule:ksub-sing-2}
% \inferrule{
%   \issubtypePX{\tau}{\tau'}
% }{
%   \ksubX{\aksing{\tau}}{\aksing{\tau'}}
% }
% \end{equation}
% \end{subequations}

% \noindent\fbox{$\strut\haskindX{c}{\kappa}$}~~$c$ has kind $\kappa$
% \begin{subequations}\label{rules:haskind}
% \begin{equation}\label{rule:haskind-subsume}
% \inferrule{
%   \haskindX{c}{\kappa_1}\\
%   \ksubX{\kappa_1}{\kappa_2}
% }{
%   \haskindX{c}{\kappa_2}
% }
% \end{equation}
% \begin{equation}\label{rule:haskind-var}
% \inferrule{ }{\haskind{\Omega, u :: \kappa}{u}{\kappa}}
% \end{equation}
% \begin{equation}\label{rule:haskind-abs}
% \inferrule{
%   \haskind{\Omega, u :: \kappa_1}{c_2}{\kappa_2}
% }{
%   \haskindX{\acabs{u}{c_2}}{\akdarr{\kappa_1}{u}{\kappa_2}}
% }
% \end{equation}
% \begin{equation}\label{rule:haskind-app}
% \inferrule{
%   \haskindX{c_1}{\akdarr{\kappa_2}{u}{\kappa}}\\
%   \haskindX{c_2}{\kappa_2}
% }{
%   \haskindX{\acapp{c_1}{c_2}}{[c_1/u]\kappa}
% }
% \end{equation}
% \begin{equation}\label{rule:haskind-unit}
% \inferrule{ }{
%   \haskindX{\actriv}{\akunit}
% }
% \end{equation}
% \begin{equation}\label{rule:haskind-pair}
% \inferrule{
%   \haskindX{c_1}{\kappa_1}\\
%   \haskindX{c_2}{[c_1/u]\kappa_2}
% }{
%   \haskindX{\acpair{c_1}{c_2}}{\akdbprod{\kappa_1}{u}{\kappa_2}}
% }
% \end{equation}
% \begin{equation}\label{rule:haskind-prl}
% \inferrule{
%   \haskindX{c}{\akdbprod{\kappa_1}{u}{\kappa_2}}
% }{
%   \haskindX{\acprl{c}}{\kappa_1}
% }
% \end{equation}
% \begin{equation}\label{rule:haskind-prr}
% \inferrule{
%   \haskindX{c}{\akdbprod{\kappa_1}{u}{\kappa_2}}
% }{
%   \haskindX{\acprr{c}}{[\acprl{c}/u]\kappa_2}
% }
% \end{equation}
% \begin{equation}\label{rule:haskind-parr}
% \inferrule{
%   \haskindX{\tau_1}{\akty}\\
%   \haskindX{\tau_2}{\akty}
% }{
%   \haskindX{\aparr{\tau_1}{\tau_2}}{\akty}
% }
% \end{equation}
% \begin{equation}\label{rule:haskind-all}
% \inferrule{
%   \iskindX{\kappa}\\
%   \haskind{\Omega, u :: \kappa}{\tau}{\akty}
% }{
%   \haskindX{\aallu{\kappa}{u}{\tau}}{\akty}
% }
% \end{equation}
% \begin{equation}\label{rule:haskind-rec}
% \inferrule{
%   \haskind{\Omega, t :: \akty}{\tau}{\akty}
% }{
%   \haskindX{\arec{t}{\tau}}{\akty}
% }
% \end{equation}
% \begin{equation}\label{rule:haskind-prod}
% \inferrule{
%   \{\haskindX{\tau_i}{\akty}\}_{1 \leq i \leq n}
% }{
%   \haskindX{\aprod{\labelset}{\mapschema{\tau}{i}{\labelset}}}{\akty}
% }
% \end{equation}
% \begin{equation}\label{rule:haskind-sum}
% \inferrule{
%   \{\haskindX{\tau_i}{\akty}\}_{1 \leq i \leq n}
% }{
%   \haskindX{\asum{\labelset}{\mapschema{\tau}{i}{\labelset}}}{\akty}
% }
% \end{equation}
% \begin{equation}\label{rule:haskind-sing}
% \inferrule{
%   \haskindX{c}{\akty}
% }{
%   \haskindX{c}{\aksing{c}}
% }
% \end{equation}
% \begin{equation}\label{rule:haskind-stat}
% \inferrule{
%   \ismvalX{M}\\
%   \hassigX{M}{\asignature{\kappa}{u}{\tau}}
% }{
%   \haskindX{\amcon{M}}{\kappa}
% }
% \end{equation}
% \end{subequations}

% \noindent\fbox{$\strut\cequalX{c}{c'}{\kappa}$}~~$c$ and $c'$ are definitionally equal as constructions of kind $\kappa$
% \begin{subequations}\label{rules:cequal}
% \begin{equation}\label{rule:cequal-refl}
% \inferrule{
%   \haskindX{c}{\kappa}
% }{
%   \cequalX{c}{c}{\kappa}
% }
% \end{equation}
% \begin{equation}\label{rule:cequal-sym}
% \inferrule{
%   \cequalX{c}{c'}{\kappa}
% }{
%   \cequalX{c'}{c}{\kappa}
% }
% \end{equation}
% \begin{equation}\label{rule:cequal-trans}
% \inferrule{
%   \cequalX{c}{c'}{\kappa}\\
%   \cequalX{c'}{c''}{\kappa}
% }{
%   \cequalX{c}{c''}{\kappa}
% }
% \end{equation}
% \begin{equation}\label{rule:cequal-lam}
% \inferrule{
%   \cequal{\Omega, u :: \kappa_1}{c}{c'}{\kappa_2}
% }{
%   \cequalX{\acabs{u}{c}}{\acabs{u}{c'}}{\akdarr{\kappa_1}{u}{\kappa_2}}
% }
% \end{equation}
% \begin{equation}\label{rule:cequal-app-1}
% \inferrule{
%   \cequalX{c_1}{c_1'}{\akdarr{\kappa_2}{u}{\kappa}}\\
%   \cequalX{c_2}{c_2'}{\kappa_2}
% }{
%   \cequalX{\acapp{c_1}{c_2}}{\acapp{c'_1}{c'_2}}{\kappa}
% }
% \end{equation}
% \begin{equation}\label{rule:cequal-app-2}
% \inferrule{
%   \haskindX{\acabs{u}{c}}{\akdarr{\kappa_2}{u}{\kappa}}\\
%   \haskindX{c_2}{\kappa_2}
% }{
%   \cequalX{\acapp{\acabs{u}{c}}{c_2}}{[c_2/u]c}{[c_2/u]\kappa}
% }
% \end{equation}
% \begin{equation}\label{rule:cequal-pair}
% \inferrule{
%   \cequalX{c_1}{c'_1}{\kappa_1}\\
%   \cequalX{c_2}{c'_2}{[c_1/u]\kappa_2}
% }{
%   \cequalX{\acpair{c_1}{c_2}}{\acpair{c'_1}{c'_2}}{\akdbprod{\kappa_1}{u}{\kappa_2}}
% }
% \end{equation}
% \begin{equation}\label{rule:cequal-prl-1}
% \inferrule{
%   \cequalX{c}{c'}{\akdbprod{\kappa_1}{u}{\kappa_2}}
% }{
%   \cequalX{\acprl{c}}{\acprl{c'}}{\kappa_1}
% }
% \end{equation}
% \begin{equation}\label{rule:cequal-prl-2}
% \inferrule{
%   \haskindX{c_1}{\kappa_1}\\
%   \haskindX{c_2}{\kappa_2}
% }{
%   \cequalX{\acprl{\acpair{c_1}{c_2}}}{c_1}{\kappa_1}
% }
% \end{equation}
% \begin{equation}\label{rule:cequal-prr-1}
% \inferrule{
%   \cequalX{c}{c'}{\akdbprod{\kappa_1}{u}{\kappa_2}}
% }{
%   \cequalX{\acprr{c}}{\acprr{c'}}{[\acprl{c}/u]\kappa_2}
% }
% \end{equation}
% \begin{equation}\label{rule:cequal-prr-2}
% \inferrule{
%   \haskindX{c_1}{\kappa_1}\\
%   \haskindX{c_2}{\kappa_2}
% }{
%   \cequalX{\acprr{\acpair{c_1}{c_2}}}{c_2}{\kappa_2}
% }
% \end{equation}
% \begin{equation}\label{rule:cequal-parr}
% \inferrule{
%   \cequalX{\tau_1}{\tau'_1}{\akty}\\
%   \cequalX{\tau_2}{\tau'_2}{\akty}
% }{
%   \cequalX{\aparr{\tau_1}{\tau_2}}{\aparr{\tau'_1}{\tau'_2}}{\akty}
% }
% \end{equation}
% \begin{equation}\label{rule:cequal-all}
% \inferrule{
%   \kequalX{\kappa}{\kappa'}\\
%   \cequal{\Omega, u :: \kappa}{\tau}{\tau'}{\akty}
% }{
%   \cequalX{\aallu{\kappa}{u}{\tau}}{\aallu{\kappa'}{u}{\tau'}}{\akty}
% }
% \end{equation}
% \begin{equation}\label{rule:cequal-rec}
% \inferrule{
%   \cequal{\Omega, t :: \akty}{\tau}{\tau'}{\akty}
% }{
%   \cequalX{\arec{t}{\tau}}{\arec{t}{\tau'}}{\akty}
% }
% \end{equation}
% \begin{equation}\label{rule:cequal-prod}
% \inferrule{
%   \{\cequalX{\tau_i}{\tau'_i}{\akty}\}_{1 \leq i \leq n}
% }{
%   \cequalX{\aprod{\labelset}{\mapschema{\tau}{i}{\labelset}}}{\aprod{\labelset}{\mapschema{\tau'}{i}{\labelset}}}{\akty}
% }
% \end{equation}
% \begin{equation}\label{rule:cequal-sum}
% \inferrule{
%   \{\cequalX{\tau_i}{\tau'_i}{\akty}\}_{1 \leq i \leq n}
% }{
%   \cequalX{\asum{\labelset}{\mapschema{\tau}{i}{\labelset}}}{\asum{\labelset}{\mapschema{\tau'}{i}{\labelset}}}{\akty}
% }
% \end{equation}
% \begin{equation}\label{rule:cequal-sing}
% \inferrule{
%   \haskindX{c}{\aksing{c'}}
% }{
%   \cequalX{c}{c'}{\akty}
% }
% \end{equation}
% \begin{equation}\label{rule:cequal-stat}
% \inferrule{
%   % \ismvalX{\astruct{c}{e}}\\
%   \hassigX{\astruct{c}{e}}{\asignature{\kappa}{u}{\tau}}
% }{
%   \cequalX{\amcon{\astruct{c}{e}}}{c}{\kappa}
% }
% \end{equation}
% \end{subequations}
% \subsubsection{Expressions, Rules and Patterns}
% % \noindent\fbox{$\strut\istypeP{\Omega}{\tau}$}~~$\tau$ is a type

% % \vspace{6px}\noindent Types, $\tau$, classify expressions. The constructions of kind $\akty$ coincide with the types of $\miniVerseParam$.
% % \begin{equation}\label{rule:istypeP}
% % \inferrule{
% %   \haskindX{\tau}{\akty}
% % }{
% %   \istypeP{\Omega}{\tau}
% % }
% % \end{equation}

% % \noindent\fbox{$\strut\tequalPX{\tau}{\tau'}$}~~$\tau$ and $\tau'$ are definitionally equal types

% % \vspace{6px}\noindent Type equality then coincides with construction equality at kind $\akty$.
% % \begin{equation}\label{rule:tequalP}
% % \inferrule{
% %   \cequalX{\tau}{\tau}{\akty}
% % }{
% %   \tequalPX{\tau}{\tau'}
% % }
% % \end{equation}


% \noindent\fbox{$\strut\issubtypePX{\tau}{\tau'}$}~~$\tau$ is a subtype of $\tau'$

% \begin{subequations}\label{rules:issubtypeP}  
% \begin{equation}\label{rule:issubtypeP-equal}
% \inferrule{
%   \cequalX{\tau_1}{\tau_2}{\akty}
% }{
%   \issubtypePX{\tau_1}{\tau_2}
% }
% \end{equation}
% \begin{equation}\label{rule:issubtypeP-trans}
% \inferrule{
%   \issubtypePX{\tau}{\tau'}\\
%   \issubtypePX{\tau'}{\tau''}
% }{
%   \issubtypePX{\tau}{\tau''}
% }
% \end{equation}
% \begin{equation}\label{rule:issubtypeP-parr}
% \inferrule{
%   \issubtypePX{\tau_1'}{\tau_1}\\
%   \issubtypePX{\tau_2}{\tau_2'}
% }{
%   \issubtypePX{\aparr{\tau_1}{\tau_2}}{\aparr{\tau_1'}{\tau_2'}}
% }
% \end{equation}
% \begin{equation}\label{rule:issubtypeP-all}
% \inferrule{
%   \ksubX{\kappa'}{\kappa}\\
%   \issubtypeP{\Omega, u :: \kappa'}{\tau}{\tau'}
% }{
%   \issubtypePX{\aallu{\kappa}{u}{\tau}}{\aallu{\kappa'}{u}{\tau'}}
% }
% \end{equation}
% \begin{equation}\label{rule:issubtypeP-prod}
% \inferrule{
%   \{\issubtypePX{\tau_i}{\tau'_i}\}_{i \in \labelset}
% }{
%   \issubtypePX{\aprod{\labelset}{\mapschema{\tau}{i}{\labelset}}}{\aprod{\labelset}{\mapschema{\tau'}{i}{\labelset}}}
% }
% \end{equation}
% \begin{equation}\label{rule:issubtypeP-sum}
% \inferrule{
%   \{\issubtypePX{\tau_i}{\tau'_i}\}_{i \in \labelset}
% }{
%   \issubtypePX{\asum{\labelset}{\mapschema{\tau}{i}{\labelset}}}{\asum{\labelset}{\mapschema{\tau'}{i}{\labelset}}}
% }
% \end{equation}
% \end{subequations}

% \noindent\fbox{$\strut\hastypeP{\Omega}{e}{\tau}$}~~$e$ has type $\tau$
% \begin{subequations}\label{rules:hastypeP}
% \begin{equation}\label{rule:hastypeP-subsume}
% \inferrule{
%   \hastypeP{\Omega}{e}{\tau}\\
%   \issubtypePX{\tau}{\tau'}
% }{
%   \hastypeP{\Omega}{e}{\tau'}
% }
% \end{equation}
% \begin{equation}\label{rule:hastypeP-var}
%   \inferrule{ }{
%     \hastypeP{\Omega, \Ghyp{x}{\tau}}{x}{\tau}
%   }
% \end{equation}
% \begin{equation}\label{rule:hastypeP-lam}
%   \inferrule{
%     \haskind{\Omega}{\tau}{\akty}\\
%     \hastypeP{\Omega, \Ghyp{x}{\tau}}{e}{\tau'}
%   }{
%     \hastypeP{\Omega}{\aelam{\tau}{x}{e}}{\aparr{\tau}{\tau'}}
%   }
% \end{equation}
% \begin{equation}\label{rule:hastypeP-ap}
%   \inferrule{
%     \hastypeP{\Omega}{e_1}{\aparr{\tau}{\tau'}}\\
%     \hastypeP{\Omega}{e_2}{\tau}
%   }{
%     \hastypeP{\Omega}{\aeap{e_1}{e_2}}{\tau'}
%   }
% \end{equation}
% \begin{equation}\label{rule:hastypeP-clam}
%   \inferrule{
%     \iskindX{\kappa}\\
%     \hastypeP{\Omega, u :: \kappa}{e}{\tau}
%   }{
%     \hastypeP{\Omega}{\aeclam{\kappa}{u}{e}}{\aallu{\kappa}{u}{\tau}}
%   }
% \end{equation}
% \begin{equation}\label{rule:hastypeP-cap}
%   \inferrule{
%     \hastypeP{\Omega}{e}{\aallu{\kappa}{u}{\tau}}\\
%     \haskindX{c}{\kappa}
%   }{
%     \hastypeP{\Omega}{\aecap{e}{c}}{[c/u]\tau}
%   }
% \end{equation}
% \begin{equation}\label{rule:hastypeP-fold}
%   \inferrule{\
%     % \haskind{\Omega, t :: \akty}{\tau}{\akty}\\
%     \hastypeP{\Omega}{e}{[\arec{t}{\tau}/t]\tau}
%   }{
%     \hastypeP{\Omega}{\aefold{e}}{\arec{t}{\tau}}
%   }
% \end{equation}
% \begin{equation}\label{rule:hastypeP-unfold}
%   \inferrule{
%     \hastypeP{\Omega}{e}{\arec{t}{\tau}}
%   }{
%     \hastypeP{\Omega}{\aeunfold{e}}{[\arec{t}{\tau}/t]\tau}
%   }
% \end{equation}
% \begin{equation}\label{rule:hastypeP-tpl}
%   \inferrule{
%     \{\hastypeP{\Omega}{e_i}{\tau_i}\}_{i \in \labelset}
%   }{
%     \hastypeP{\Omega}{\aetpl{\labelset}{\mapschema{e}{i}{\labelset}}}{\aprod{\labelset}{\mapschema{\tau}{i}{\labelset}}}
%   }
% \end{equation}
% \begin{equation}\label{rule:hastypeP-pr}
%   \inferrule{
%     \hastypeP{\Omega}{e}{\aprod{\labelset, \ell}{\mapschema{\tau}{i}{\labelset}; \ell \hookrightarrow \tau}}
%   }{
%     \hastypeP{\Omega}{\aepr{\ell}{e}}{\tau}
%   }
% \end{equation}
% \begin{equation}\label{rule:hastypeP-in}
%   \inferrule{
%     % \{\haskind{\Omega}{\tau_i}{\akty}\}_{i \in \labelset}\\
%     % \haskind{\Omega}{\tau}{\akty}\\
%     \hastypeP{\Omega}{e}{\tau}
%   }{
%     \hastypeP{\Omega}{\aein{\ell}{e}}{\asum{\labelset, \ell}{\mapschema{\tau}{i}{\labelset}; \ell \hookrightarrow \tau}}
%   }
% \end{equation}
% \begin{equation}\label{rule:hastypeP-match}
% \inferrule{
%   \hastypeP{\Omega}{e}{\tau}\\
%   % \haskind{\Omega}{\tau'}{\akty}\\
%   \{\ruleTypeP{\Omega}{r_i}{\tau}{\tau'}\}_{1 \leq i \leq n}\\
% }{\hastypeP{\Omega}{\aematchwith{n}{e}{\seqschemaX{r}}}{\tau'}}
% \end{equation}
% \begin{equation}\label{rule:hastypeP-dyn}
% \inferrule{
%   \ismvalX{M}\\
%   \hassigX{M}{\asignature{\kappa}{u}{\tau}}
% }{
%   \hastypeP{\Omega}{\amval{M}}{[\amcon{M}/u]\tau}
% }
% \end{equation}
% \end{subequations}
% \noindent\fbox{$\strut\ruleTypeP{\Omega}{r}{\tau}{\tau'}$}~~$r$ takes values of type $\tau$ to values of type $\tau'$
% \begin{equation}\label{rule:ruleTypeP}
% \inferrule{
%   \patTypeP{\Omega'}{p}{\tau}\\
%   \hastypeP{\Gcons{\Omega}{\Omega'}}{e}{\tau'}
% }{
%   \ruleTypeP{\Omega}{\aematchrule{p}{e}}{\tau}{\tau'}
% }
% \end{equation}

% \noindent\fbox{$\strut\patTypeP{\Omega'}{p}{\tau}$}~~$p$ matches values of type $\tau$ generating hypotheses $\Omega'$

% \begin{subequations}\label{rules:patTypeP}
% \begin{equation}\label{rule:patTypeP-subsume}
% \inferrule{
%   \patTypeP{\Omega'}{p}{\tau}\\
%   \issubtypePX{\tau}{\tau'}
% }{
%   \patTypeP{\Omega'}{p}{\tau'}
% }
% \end{equation}
% \begin{equation}\label{rule:patTypeP-var}
% \inferrule{ }{\patTypeP{\Ghyp{x}{\tau}}{x}{\tau}}
% \end{equation}
% \begin{equation}\label{rule:patTypeP-wild}
% \inferrule{ }{\patTypeP{\emptyset}{\aewildp}{\tau}}
% \end{equation}
% \begin{equation}\label{rule:patTypeP-fold}
% \inferrule{
%   \patTypeP{\Omega'}{p}{[\arec{t}{\tau}/t]\tau}
% }{
%   \patTypeP{\Omega'}{\aefoldp{p}}{\arec{t}{\tau}}
% }
% \end{equation}
% \begin{equation}\label{rule:patTypeP-tpl}
% \inferrule{
%   \{\patTypeP{\Omega_i}{p_i}{\tau_i}\}_{i \in \labelset}
% }{
%   \patTypeP{\Gconsi{i \in \labelset}{\Omega_i}}{\aetplp{\labelset}{\mapschema{p}{i}{\labelset}}}{\aprod{\labelset}{\mapschema{\tau}{i}{\labelset}}}
% }
% \end{equation}
% \begin{equation}\label{rule:patTypeP-inj}
% \inferrule{
%   \patTypeP{\Omega'}{p}{\tau}
% }{
%   \patTypeP{\Omega'}{\aeinjp{\ell}{p}}{\asum{\labelset, \ell}{\mapschema{\tau}{i}{\labelset}; \mapitem{\ell}{\tau}}}
% }
% \end{equation}
% \end{subequations}

% \subsubsection{Metatheory}
% The rules above are syntax-directed, so we assume an inversion lemma for each rule as needed without stating it separately or proving it explicitly. The following standard lemmas also hold, for all basic judgements $J$ above.

% \begin{lemma}[Weakening]\label{lemma:weakening-P}  If $\Omega \vdash J$ then $\Omega \cup \Omega' \vdash J$.
% % \begin{enumerate}
% % \item \begin{enumerate}
% %   \item If $\issigX{\sigma}$ then $\issig{\Omega \cup \Omega'}{\sigma}$.
% %   \item If $\sigequal{\Omega}{\sigma}{\sigma'}$ then $\sigequal{\Omega \cup \Omega'}{\sigma}{\sigma'}$.
% %   \item If $\sigsub{\Omega}{\sigma}{\sigma'}$ then $\sigsub{\Omega \cup \Omega'}{\sigma}{\sigma'}$.
% %   \item If $\hassigX{M}{\sigma}$ then $\hassig{\Omega \cup \Omega'}{M}{\sigma}$.
% %   \item If $\ismvalX{M}$ then $\ismval{\Omega \cup \Omega'}{M}$.
% %   \end{enumerate}
% % \item \begin{enumerate}
% % \item If $\iskindX{\kappa}$ then $\iskind{\Omega \cup \Omega'}{\kappa}$.
% % \item If $\kequalX{\kappa}{\kappa'}$ then $\kequal{\Omega \cup \Omega'}{\kappa}{\kappa'}$.
% % \item If $\ksubX{\kappa}{\kappa'}$ then $\ksub{\Omega \cup \Omega'}{\kappa}{\kappa'}$.
% % \item If $\haskindX{c}{\kappa}$ then $\haskind{\Omega \cup \Omega'}{c}{\kappa}$.
% % \item If $\cequalX{c}{c'}{\kappa}$ then $\cequal{\Omega \cup \Omega'}{c}{c'}{\kappa}$.
% % \end{enumerate}
% % \item \begin{enumerate}
% % \item If $\istypeP{\Omega}{\tau}$ then $\istypeP{\Omega \cup \Omega'}{\tau}$.
% % \item If $\tequalPX{\tau}{\tau'}$ then $\tequalP{\Omega \cup \Omega'}{\tau}{\tau'}$.
% % \item If $\issubtypePX{\tau}{\tau'}$ then $\issubtypeP{\Omega \cup \Omega'}{\tau}{\tau'}$.
% % \item If $\hastypeP{\Omega}{e}{\tau}$ then $\hastypeP{\Omega \cup \Omega'}{e}{\tau}$.
% % \item If $\ruleTypeP{\Omega}{r}{\tau}{\tau'}$ then $\ruleTypeP{\Omega \cup \Omega'}{r}{\tau}{\tau'}$.
% % \item If $\patTypePC{\Omega}{\Omega''}{p}{\tau}$ then $\patTypePC{\Omega \cup \Omega'}{\Omega''}{p}{\tau}$.
% % \end{enumerate}
% % \end{enumerate}
% \end{lemma}
% \begin{proof-sketch} By straightforward mutual rule induction.
% \end{proof-sketch}

% \begin{definition} 
% \label{def:substitution-p}
% A \emph{substitution}, $\omega$, is a finite function that maps:
% \begin{itemize}
% \item each $X \in \domof{\omega}$ to a module expression subtitution, $M/X$; 
% \item each $u \in \domof{\omega}$ to a construction substitution, $c/u$; and 
% \item each $x \in \domof{\omega}$ to an expression substitution, $e/x$.
% \end{itemize}

% We write $\hastypeP{\Omega}{\omega}{\Omega'}$ iff $\domof{\omega}=\domof{\Omega'}$ and:
% \begin{itemize}
% \item for each $M/X \in \omega$, we have $X : \sigma \in \Omega'$ and $\hassigX{M}{\sigma}$ and $\ismvalX{M}$; and
% \item for each $c/u \in \omega$, we have $u :: \kappa \in \Omega'$ and $\haskindX{c}{\kappa}$; and 
% \item for each $e/x \in \omega$, we have $x : \tau \in \Omega'$ and $\hastypeP{\Omega}{e}{\tau}$.
% \end{itemize}

% We simultaneously apply a substitution by placing it in prefix position. For example, $[\omega]e$ applies the substitutions $\omega$ simultaneously to $e$.
% \end{definition}

% \begin{lemma}[Substitution]\label{lemma:substitution-P} If $\Omega \cup \Omega' \cup \Omega'' \vdash J$ and $\hastypeP{\Omega}{\omega}{\Omega'}$ then $\Omega \cup [\omega]\Omega'' \vdash [\omega]J$.
% \end{lemma}
% \begin{proof-sketch} By straightforward rule induction. 
% \end{proof-sketch}

% \begin{lemma}[Decomposition]\label{lemma:decomposition-P} 
% If $\Omega \cup [\omega]\Omega'' \vdash [\omega]J$ and $\hastypeP{\Omega}{\omega}{\Omega'}$ then $\Omega \cup \Omega' \cup \Omega'' \vdash J$.
% \end{lemma}
% \begin{proof-sketch} By straightforward rule induction.
% \end{proof-sketch}

% \begin{lemma}[Pattern Binding]\label{lemma:pattern-binding}
% If $\patTypeP{\Omega'}{p}{\tau}$ then $\domof{\Omega'} = \mathsf{patvars}(p)$.
% \end{lemma}
% \begin{proof-sketch} By straightforward rule induction over Rules (\ref{rules:patTypeP}). \end{proof-sketch}
% % \begin{lemma}[Regularity]\label{lemma:regularity-P} ~
% % \begin{enumerate}
% % \item ...
% % \end{enumerate}
% % \end{lemma}

% \subsection{Structural Dynamics}
% The structural dynamics of modules is defined as a transition system, and is organized around judgements of the following form:

% \vspace{10px}
% $\begin{array}{ll}
% \textbf{Judgement Form} & \textbf{Description}\\
% \stepsU{M}{M'} & \text{$M$ transitions to $M'$}\\
% \isvalP{M} & \text{$M$ is a module value}\\
% \matchfail{M} & \text{$M$ raises match failure}
% \end{array}$
% \vspace{10px}

% The structural dynamics of expressions is also defined as a transition system, and is organized around judgements of the following form:

% \vspace{10px}
% $\begin{array}{ll}
% \textbf{Judgement Form} & \textbf{Description}\\
% \stepsU{e}{e'} & \text{$e$ transitions to $e'$}\\
% \isvalP{e} & \text{$e$ is a value}\\
% \matchfail{e} & \text{$e$ raises match failure}
% \end{array}$
% \vspace{10px}

% We also define auxiliary judgements for \emph{iterated transition}, $\multistepU{e}{e'}$, and \emph{evaluation}, $\evalU{e}{e'}$ of expressions.

% \begin{definition}[Iterated Transition]\label{defn:iterated-transition-P} Iterated transition, $\multistepU{e}{e'}$, is the reflexive, transitive closure of the transition judgement, $\stepsU{e}{e'}$.\end{definition}

% \begin{definition}[Evaluation]\label{defn:evaluation-P} $\evalU{e}{e'}$ iff $\multistepU{e}{e'}$ and $\isvalU{e'}$. \end{definition}

% Similarly, we lift these definitions to the level of module expressions as well.

% \begin{definition}[Iterated Module Transition]\label{defn:iterated-transition-modules-P} Iterated transition, $\multistepU{M}{M'}$, is the reflexive, transitive closure of the transition judgement, $\stepsU{M}{M'}$.\end{definition}

% \begin{definition}[Module Evaluation]\label{defn:evaluation-modules-P} $\evalU{M}{M'}$ iff $\multistepU{M}{M'}$ and $\isvalU{M'}$. \end{definition}



% As in $\miniVersePat$, our subsequent developments do not make mention of particular rules in the dynamics, nor do they make mention of other judgements, not listed above, that are used only for defining the dynamics of the match operator, so we do not produce these details here. Instead, it suffices to state the following conditions.

% The Preservation condition ensures that evaluation preserves typing.
% \begin{condition}[Preservation]\label{condition:preservation-P} ~
% \begin{enumerate}
% \item If $\hassig{}{M}{\sigma}$ and $\stepsU{M}{M'}$ then $\hassig{}{M}{\sigma}$.
% \item If $\hastypeUC{e}{\tau}$ and $\stepsU{e}{e'}$ then $\hastypeUC{e'}{\tau}$.
% \end{enumerate}
% \end{condition}

% The Progress condition ensures that evaluation of a well-typed expanded expression cannot ``get stuck''. We must consider the possibility of match failure in this condition.
% \begin{condition}[Progress]\label{condition:progress-P} ~
% \begin{enumerate}
% \item If $\hassig{}{M}{\sigma}$ then either $\isvalU{M}$ or $\matchfail{M}$ or there exists an $M'$ such that $\stepsU{M}{M'}$.
% \item If $\hastypeUC{e}{\tau}$ then either $\isvalU{e}$ or $\matchfail{e}$ or there exists an $e'$ such that $\stepsU{e}{e'}$.
% \end{enumerate}
% \end{condition}

% \section{Unexpanded Language (UL)}
% \subsection{Syntax}
% \subsubsection{Stylized Syntax -- Unexpanded Signatures and Modules}
% \[\begin{array}{lllllll}
% \textbf{Sort} & & 
% %& \textbf{Operational Form} 
% & \textbf{Stylized Form} & \textbf{Description}\\
% \mathsf{USig} & \usigma & ::= 
% %& \ausignature{\ukappa}{\uu}{\utau} 
% & \signature{\uu}{\ukappa}{\utau} & \text{signature}\\
% \mathsf{UMod} & \uM & ::= 
% %& \uX 
% & \uX & \text{module identifier}\\
% &&
% %& \austruct{\uc}{\ue} 
% & \struct{\uc}{\ue} & \text{structure}\\
% &&
% %& \auseal{\usigma}{\uM} 
% & \seal{\uM}{\usigma} & \text{seal}\\
% &&
% %& \aumlet{\usigma}{\uM}{\uX}{\uM} 
% & \mlet{\uX}{\uM}{\uM}{\usigma} & \text{definition}\\
% % \LCC &&
% %& \color{light-gray} 
% % & \color{Yellow} & \color{Yellow}\\
% &&
% %& \aumdefpetsm{\urho}{e}{\tsmv}{\uM} 
% & \defpetsm{\tsmv}{\urho}{e}{\uM} & \text{peTLM definition}\\
% %&&&                                    & \texttt{expressions}~\{e\}~\texttt{in}~\uM\\
% &&
% %& \aumletpetsm{\uepsilon}{\tsmv}{\uM} 
% & \uletpetsm{\tsmv}{\uepsilon}{\uM} & \text{peTLM binding}\\
% % &&&                                  & \texttt{expressions}~\texttt{in}~\uM\\
% % &&& ... & ... & \text{peTLM designation}\\
% &&
% %& \audefpptsm{\urho}{e}{\tsmv}{\uM} 
% & \defpptsm{\tsmv}{\urho}{e}{\uM} & \text{ppTLM definition}\\
% % &&&                                    & \texttt{patterns}~\{e\}~\texttt{in}~\uM\\
% &&
% %& \auletpptsm{\uepsilon}{\tsmv}{\uM} 
% & \uletpptsm{\tsmv}{\uepsilon}{\uM} & \text{ppTLM binding}%\ECC%
% % &&& & \texttt{patterns}~\texttt{in}~\uM\\
% % &&& ... & ... & \text{ppTLM designation}\ECC
% \end{array}\]%\vspace{-15px}
% % \caption[Syntax of unexpanded module expressions and signatures in $\miniVerseParam$]{Syntax of unexpanded module expressions and signatures in $\miniVerseParam$.}\vspace{-5px}
% % \label{fig:P-unexpanded-modules-signatures}
% % \end{figure}
% % \begin{figure}[p] \vspace{-10px}

% \subsubsection{Stylized Syntax -- Unexpanded Kinds and Constructions}
% \[\begin{array}{lrlllll}
% \textbf{Sort} & & 
% %& \textbf{Operational Form} 
% & \textbf{Stylized Form} & \textbf{Description}\\
% \mathsf{UKind} & \ukappa & ::= 
% %& \aukdarr{\ukappa}{\uu}{\ukappa} 
% & \kdarr{\uu}{\ukappa}{\ukappa} & \text{dependent function}\\
% &&
% %& \aukunit 
% & \kunit & \text{nullary product}\\
% &&
% %& \aukdbprod{\ukappa}{\uu}{\ukappa} 
% & \kdbprod{\uu}{\ukappa}{\ukappa} & \text{dependent product}\\
% %&&& \akdprodstd & \kdprodstd & \text{labeled dependent product}\\
% &&
% %& \aukty 
% & \kty & \text{type}\\
% &&
% %& \auksing{\utau} 
% & \ksing{\utau} & \text{singleton}\\
% \mathsf{UCon} & \uc, \utau & ::= 
% %& \uu 
% & \uu & \text{construction identifier}\\
% &&
% %& \ut 
% & \ut & \\
% &&
% %& \aucasc{\ukappa}{\uc} 
% & \casc{\uc}{\ukappa} & \text{ascription}\\
% &&
% %& \aucabs{\uu}{\uc} 
% & \cabs{\uu}{\uc} & \text{abstraction}\\
% &&
% %& \aucapp{c}{c} 
% & \capp{c}{c} & \text{application}\\
% &&
% %& \auctriv 
% & \ctriv & \text{trivial}\\
% &&
% %& \aucpair{\uc}{\uc} 
% & \cpair{\uc}{\uc} & \text{pair}\\
% &&
% %& \aucprl{\uc} 
% & \cprl{\uc} & \text{left projection}\\
% &&
% %& \aucprr{\uc} 
% & \cprr{\uc} & \text{right projection}\\
% %&&& \adtplX & \dtplX & \text{labeled dependent tuple}\\
% %&&& \adprj{\ell}{c} & \prj{c}{\ell} & \text{projection}\\
% &&
% %& \auparr{\utau}{\utau} 
% & \parr{\utau}{\utau} & \text{partial function}\\
% &&
% %& \auallu{\ukappa}{\uu}{\utau} 
% & \forallu{\uu}{\ukappa}{\utau} & \text{polymorphic}\\
% &&
% %& \aurec{\ut}{\utau} 
% & \rect{\ut}{\utau} & \text{recursive}\\
% &&
% %& \auprod{\labelset}{\mapschema{\utau}{i}{\labelset}} 
% & \prodt{\mapschema{\utau}{i}{\labelset}} & \text{labeled product}\\
% &&
% %& \ausum{\labelset}{\mapschema{\utau}{i}{\labelset}} 
% & \sumt{\mapschema{\utau}{i}{\labelset}} & \text{labeled sum}\\
% &&
% %& \aumcon{\uX} 
% & \mcon{\uX} & \text{construction component}
% \end{array}\]%\vspace{-15px}
% % \caption[Syntax of unexpanded kinds and constructions in $\miniVerseParam$]{Syntax of unexpanded kinds and constructions in $\miniVerseParam$.}\vspace{-10px}
% % \label{fig:P-unexpanded-kinds-constructions}
% % \end{figure}
% \clearpage

% \subsubsection{Stylized Syntax -- Unexpanded Expressions, Rules and Patterns}
% % \clearpage
% % \begin{figure}[p]
% \[\begin{array}{lllllll}
% \textbf{Sort} & & 
% %& \textbf{Operational Form} 
% & \textbf{Stylized Form} & \textbf{Description}\\
% \mathsf{UExp} & \ue & ::= 
% %& \ux 
% & \ux & \text{identifier}\\
% &&
% % & \auasc{\utau}{\ue} 
% & \asc{\ue}{\utau} & \text{ascription}\\
% &&
% % & \auletsyn{\ux}{\ue}{\ue} 
% & \letsyn{\ux}{\ue}{\ue} & \text{value binding}\\
% % &&
% %& \auanalam{\ux}{\ue} 
% % & \analam{\ux}{\ue} & \text{abstraction (unannotated)}\\
% &&
% %& \aulam{\utau}{\ux}{\ue} 
% & \lam{\ux}{\utau}{\ue} & \text{abstraction}\\
% &&
% %& \auap{\ue}{\ue} 
% & \ap{\ue}{\ue} & \text{application}\\
% &&
% %& \auclam{\ukappa}{\uu}{\ue} 
% & \clam{\uu}{\ukappa}{\ue} & \text{construction abstraction}\\
% &&
% %& \aucap{\ue}{\uc} 
% & \cAp{\ue}{\uc} & \text{construction application}\\
% &&
% %& \auanafold{\ue} 
% & \fold{\ue} & \text{fold}\\
% &&
% %& \auunfold{\ue} 
% & \unfold{\ue} & \text{unfold}\\
% &&
% %& \autpl{\labelset}{\mapschema{\ue}{i}{\labelset}} 
% & \tpl{\mapschema{\ue}{i}{\labelset}} & \text{labeled tuple}\\
% &&
% %& \aupr{\ell}{\ue} 
% & \prj{\ue}{\ell} & \text{projection}\\
% &&
% %& \auanain{\ell}{\ue} 
% & \inj{\ell}{\ue} & \text{injection}\\
% &&
% %& \aumatchwithb{n}{\ue}{\seqschemaX{\urv}} 
% & \matchwith{\ue}{\seqschemaX{\urv}} & \text{match}\\
% &&
% %& \aumval{\uX} 
% & \mval{\uX} & \text{value component}\\
% % \LCC &&
% % %& \color{Yellow} 
% % & \color{Yellow} & \color{Yellow} \\
% % &&& \audefpetsm{\urho}{e}{\tsmv}{\ue} & \texttt{syntax}~\tsmv~\texttt{at}~\urho~\texttt{for} & \text{peTLM definition}\\
% % &&&                                    & \texttt{expressions}~\{e\}~\texttt{in}~\ue\\
% % &&& \auletpetsm{\uepsilon}{\tsmv}{\ue} & \texttt{let}~\texttt{syntax}~\tsmv=\uepsilon~\texttt{for} & \text{peTLM binding}\\
% % &&&                                  & \texttt{expressions}~\texttt{in}~\ue\\
% % &&& ... & ... & \text{peTLM designation}\\
% &&
% %& \auappetsm{b}{\uepsilon} 
% & \utsmap{\uepsilon}{b} & \text{peTLM application}\\%\ECC\\%\ECC
% % &&& \auelit{b} & {\lit{b}}  & \text{peTLM unadorned literal}\\
% % &&& \audefpptsm{\urho}{e}{\tsmv}{\ue} & \texttt{syntax}~\tsmv~\texttt{at}~\urho~\texttt{for} & \text{ppTLM definition}\\
% % &&&                                    & \texttt{patterns}~\{e\}~\texttt{in}~\ue\\
% % &&& \auletpptsm{\uepsilon}{\tsmv}{\ue} & \texttt{let}~\texttt{syntax}~\tsmv=\uepsilon~\texttt{for} & \text{ppTLM binding}\\
% % &&& & \texttt{patterns}~\texttt{in}~\ue\\
% % &&& ... & ... & \text{ppTLM designation}\\\ECC
% \mathsf{URule} & \urv & ::= 
% %& \aumatchrule{\upv}{\ue} 
% & \matchrule{\upv}{\ue} & \text{match rule}\\
% \mathsf{UPat} & \upv & ::= 
% %& \ux 
% & \ux & \text{identifier pattern}\\
% &&
% %& \auwildp 
% & \wildp & \text{wildcard pattern}\\
% &&
% %& \aufoldp{\upv} 
% & \foldp{\upv} & \text{fold pattern}\\
% &&
% %& \autplp{\labelset}{\mapschema{\upv}{i}{\labelset}} 
% & \tplp{\mapschema{\upv}{i}{\labelset}} & \text{labeled tuple pattern}\\
% &&
% %& \auinjp{\ell}{\upv} 
% & \injp{\ell}{\upv} 
% & \text{injection pattern}\\
% % \LCC &&
% %& \color{light-gray} 
% % & \color{Yellow} & \color{Yellow}\\
% &&
% %& \auappptsm{b}{\uepsilon} 
% & \utsmap{\uepsilon}{b} & \text{ppTLM application}%\ECC
% % &&& \auplit{b} & \lit{b} & \text{ppTLM unadorned literal}\ECC
% \end{array}\]
% % \caption[Syntax of unexpanded expressions, rules and patterns in $\miniVerseParam$]{Syntax of unexpanded expressions, rules and patterns in $\miniVerseParam$.}
% % \label{fig:P-unexpanded-terms}
% % \end{figure}

% \subsubsection{Stylized Syntax -- Unexpanded TLM Types and Expressions}
% % \begin{figure}[p]
% \[\begin{array}{lllllll}
% \textbf{Sort} & & 
% %& \textbf{Operational Form} 
% & \textbf{Stylized Form} 
% & \textbf{Description}\\
% % \LCC \color{Yellow}&\color{Yellow}& \color{Yellow}
% %& \color{light-gray} 
% % & \color{Yellow} & \color{Yellow}\\
% \mathsf{UMType} & \urho & ::= 
% %& \autype{\utau} 
% & \utau & \text{type annotation}\\
% % &&
% %& \aualltypes{\ut}{\urho} 
% % & \alltypes{\ut}{\urho} & \text{type parameterization}\\
% &&
% %& \auallmods{\usigma}{\uX}{\urho} 
% & \allmods{\uX}{\usigma}{\urho} & \text{module parameterization}\\
% \mathsf{UMExp} & \uepsilon & ::= 
% %& \abindref{\tsmv} 
% & \tsmv & \text{TLM identifier reference}\\
% % &&
% %& \auabstype{\ut}{\uepsilon} 
% % & \abstype{\ut}{\uepsilon} & \text{type abstraction}\\
% &&
% %& \auabsmod{\usigma}{\uX}{\uepsilon} 
% & \absmod{\uX}{\usigma}{\uepsilon} & \text{module abstraction}\\
% % &&
% %& \auaptype{\utau}{\uepsilon} 
% % & \aptype{\uepsilon}{\utau} & \text{type application}\\
% &&
% %& \auapmod{\uM}{\uepsilon} 
% & \apmod{\uepsilon}{\uX} & \text{module application}%\ECC
% \end{array}
% \]
% % \caption{Syntax of unexpanded TLM types and expressions.}
% % \label{fig:P-macro-expressions-types-u}
% % \end{figure}

% \subsubsection{Stylized Syntax -- TLM Types and Expressions}

% % \clearpage
% % \begin{figure}[p]
% \[\begin{array}{lllllll}
% \textbf{Sort} & & & \textbf{Operational Form} 
% %& \textbf{Stylized Form} 
% & \textbf{Description}\\
% % \LCC \color{Yellow}&\color{Yellow}& \color{Yellow}
% %& \color{light-gray} 
% % & \color{Yellow} & \color{Yellow}\\
% \mathsf{MType} & \rho & ::= & \aetype{\tau} 
% %& \tau 
% & \text{type annotation}\\
% % &&& \aealltypes{t}{\rho} 
% %& \alltypes{t}{\rho} 
% % & \text{type parameterization}\\
% &&& \aeallmods{\sigma}{X}{\rho} 
% %& \allmods{X}{\sigma}{\rho} 
% & \text{module parameterization}\\
% \mathsf{MExp} & \epsilon & ::= & \adefref{a} 
% %& a 
% & \text{TLM definition reference}\\
% % &&& \aeabstype{t}{\epsilon} 
% %& \abstype{t}{\epsilon} 
% % & \text{type abstraction}\\
% &&& \aeabsmod{\sigma}{X}{\epsilon} 
% %& \absmod{X}{\sigma}{\epsilon} 
% & \text{module abstraction}\\
% % &&& \aeaptype{\tau}{\epsilon} 
% %& \aptype{\epsilon}{\tau} 
% % & \text{type application}\\
% &&& \aeapmod{M}{\epsilon} 
% %& \aptype{\epsilon}{M} 
% & \text{module application}%\ECC
% \end{array}\]
% % \caption[Syntax of TLM types and expressions in $\miniVerseParam$]{Syntax of TLM types and expressions.}
% % \label{fig:P-macro-expressions-types}
% % \end{figure}

% \subsubsection{Body Lengths}
% We write $\sizeof{b}$ for the length of $b$. 
% The metafunction $\sizeof{\uM}$ computes the sum of the lengths of expression literal bodies in $\uM$:
% \[
% \begin{array}{ll}
% \sizeof{\uX} & = 0\\
% \sizeof{\struct{\uc}{\ue}} & = \sizeof{\ue}\\
% \sizeof{\seal{\uM}{\usigma}} & = \sizeof{\uM}\\
% \sizeof{\mlet{\uX}{\uM}{\uM'}{\usigma}} & = \sizeof{\uM} + \sizeof{\uM'}\\
% \sizeof{\defpetsm{\tsmv}{\urho}{e}{\uM}} & = \sizeof{\uM}\\
% \sizeof{\uletpetsm{\tsmv}{\uepsilon}{\uM}} & = \sizeof{\uM}\\
% \sizeof{\defpptsm{\tsmv}{\urho}{e}{\uM}} & = \sizeof{\uM}\\
% \sizeof{\uletpptsm{\tsmv}{\uepsilon}{\uM}} & = \sizeof{\uM}
% \end{array}
% \]
% and $\sizeof{\ue}$ computes the sum of the lengths of expression literal bodies in $\ue$:
% \[
% \begin{array}{ll}
% \sizeof{\ux} & = 0\\
% \sizeof{\lam{\ux}{\utau}{\ue}} &= \sizeof{\ue}\\
% \sizeof{\ap{\ue_1}{\ue_2}} & = \sizeof{\ue_1} + \sizeof{\ue_2}\\
% \sizeof{\clam{\uu}{\ukappa}{\ue}} & = \sizeof{\ue}\\
% \sizeof{\cAp{\ue}{\uc}} & = \sizeof{\ue}\\
% \sizeof{\fold{\ue}} & = \sizeof{\ue}\\
% \sizeof{\unfold{\ue}} & = \sizeof{\ue}\\
% %\end{align*}
% %\begin{align*}
% \sizeof{\tpl{\mapschema{\ue}{i}{\labelset}}} & = \sum_{i \in \labelset} \sizeof{\ue_i}\\
% \sizeof{\prj{\ell}{\ue}} & = \sizeof{\ue}\\
% \sizeof{\inj{\ell}{\ue}} & = \sizeof{\ue}\\
% \sizeof{\matchwith{\ue}{\seqschemaX{\urv}}} & = \sizeof{\ue} + \sum_{1 \leq i \leq n} \sizeof{r_i}\\
% \sizeof{\mval{\uX}} & = 0\\
% % \sizeof{\caseof{\ue}{\mapschemab{\ux}{\ue}{i}{\labelset}}} & = \sizeof{\ue} + \sum_{i \in \labelset} \sizeof{\ue_i}\\
% % \sizeof{\uesyntax{\tsmv}{\utau}{\eparse}{\ue}} & = \sizeof{\ue}\\
% \sizeof{\utsmap{\uepsilon}{b}} & = \sizeof{b}
% \end{array}
% \]
% and $\sizeof{\urv}$ computes the sum of the lengths of expression literal bodies in $\urv$:
% \begin{align*}
% \sizeof{\matchrule{\upv}{\ue}} & = \sizeof{\ue}
% \end{align*}
% % and $\sizeof{\uepsilon}$ computes the sum of the lengths of expression literal bodies in $\uepsilon$:
% % \begin{align*}
% % \sizeof{\tsmv} & = 0\\
% % \sizeof{\abstype{\ut}{\uepsilon}} & = \sizeof{\uepsilon}\\
% % \sizeof{\absmod{\uX}{\usigma}{\uepsilon}} & = \sizeof{\uepsilon}\\
% % \sizeof{\aptype{\uepsilon}{\utau}} & = 0\\
% % \sizeof{\apmod{\uepsilon}{\uM}} & = \sizeof{\uM}
% % \end{align*}

% Similarly, the metafunction $\sizeof{\upv}$ computes the sum of the lengths of the pattern literal bodies in $\upv$:
% \begin{align*}
% \sizeof{\ux} & = 0\\
% \sizeof{\foldp{\upv}} & = \sizeof{\upv}\\
% \sizeof{\tplp{\mapschema{\upv}{i}{\labelset}}} & = \sum_{i \in \labelset} \sizeof{\upv_i}\\
% \sizeof{\injp{\ell}{\upv}} & = \sizeof{\upv}\\
% \sizeof{\utsmap{\uepsilon}{b}} & = \sizeof{b}
% \end{align*}

% \subsubsection{Common Unexpanded Forms}\label{appendix:P-shared-forms}
% Each expanded form, with a few minor exceptions noted below, maps onto an unexpanded form. We refer to these as the \emph{common forms}. In particular:
% \begin{itemize}
% \item Each module variable, $X$, maps onto a unique module identifier, written $\sigilof{X}$.
% \item Each signature, $\sigma$, maps onto an unexpanded signature, $\Uof{\sigma}$, as follows:
% \begin{align*}
% \Uof{\asignature{\kappa}{u}{c}} & = \signature{\sigilof{u}}{\Uof{\kappa}}{\Uof{c}}
% \end{align*}
% \item Each module expression, $M$, maps onto an unexpanded module expression, $\uM$, as follows:
% \begin{align*}
% \Uof{X} & = \sigilof{X}\\
% \Uof{\astruct{\uc}{\ue}} & = \struct{\Uof{\uc}}{\Uof{\ue}}\\
% \Uof{\aseal{\sigma}{M}} & = \seal{\Uof{M}}{\Uof{\sigma}}\\
% \Uof{\amlet{\sigma}{M}{X}{M'}} & = \mlet{\sigilof{X}}{\Uof{M}}{\Uof{M'}}{\Uof{\sigma}}
% \end{align*}
% \item Each construction variable, $u$, maps onto a unique {type identifier}, written $\sigilof{u}$.
% \item Each kind, $\kappa$, maps onto an unexpanded kind, $\Uof{\kappa}$, as follows:
% \begin{align*}
% \Uof{\akdarr{\kappa}{u}{\kappa'}} & = \kdarr{\sigilof{u}}{\Uof{\kappa}}{\Uof{\kappa'}}\\
% \Uof{\akunit} & = \kunit\\
% \Uof{\akdbprod{\kappa}{u}{\kappa'}} & = \kdbprod{\sigilof{u}}{\Uof{\kappa}}{\Uof{\kappa'}}\\
% \Uof{\akty} & = \kty\\
% \Uof{\aksing{\tau}} & = \ksing{\Uof{\tau}}
% \end{align*}
% \item Each construction, $c$, except for constructions of the form $\amcon{M}$ where $M$ is not a module variable, maps onto an unexpanded type, $\Uof{c}$, as follows: 
%   \begin{align*}
%   \Uof{u} &= \sigilof{u}\\
%   \Uof{\acabs{u}{c}} & = \cabs{\sigilof{u}}{\Uof{c}}\\
%   \Uof{\acapp{c}{c'}} & = \capp{\Uof{c}}{\Uof{c'}}\\
%   \Uof{\actriv} & = \ctriv\\
%   \Uof{\acpair{c}{c'}} & = \cpair{\Uof{c}}{\Uof{c'}}\\
%   \Uof{\acprl{c}} & = \cprl{\Uof{c}}\\
%   \Uof{\acprr{c}} & = \cprr{\Uof{c}}\\
%   \Uof{\aparr{\tau_1}{\tau_2}} & = \parr{\Uof{\tau_1}}{\Uof{\tau_2}}\\
%   \Uof{\aallu{\kappa}{u}{\tau}} & = \forallu{\sigilof{u}}{\Uof{\kappa}}{\Uof{\tau}}\\
%   \Uof{\arec{t}{\tau}} & = \rect{\sigilof{t}}{\Uof{\tau}}\\
%   \Uof{\aprod{\labelset}{\mapschema{\tau}{i}{\labelset}}} & = \prodt{\mapschemax{\Uofv}{\tau}{i}{\labelset}}\\
%   \Uof{\asum{\labelset}{\mapschema{\tau}{i}{\labelset}}} & = \sumt{\mapschemax{\Uofv}{\tau}{i}{\labelset}}\\
%   \Uof{\amcon{X}} & = \mcon{\sigilof{X}}
%   \end{align*}
% \item Each expression variable, $x$, maps onto a unique expression identifier, written $\sigilof{x}$.
% \item Each expanded expression, $e$, except expressions of the form $\amval{M}$ where $M$ is not a module variable, maps onto an unexpanded expression, $\Uof{e}$, as follows:
% \begin{align*}
% \Uof{x} & = \sigilof{x}\\
% \Uof{\aelam{\tau}{x}{e}} & = \lam{\sigilof{x}}{\Uof{\tau}}{\Uof{e}}\\
% \Uof{\aeap{e_1}{e_2}} & = \ap{\Uof{e_1}}{\Uof{e_2}}\\
% \Uof{\aeclam{\kappa}{u}{e}} & = \clam{\sigilof{u}}{\Uof{\kappa}}{\Uof{e}}\\
% \Uof{\aecap{e}{c}} & = \cAp{\Uof{e}}{\Uof{c}}\\
% \Uof{\aefold{e}} & = \fold{\Uof e}\\
% \Uof{\aeunfold{e}} & = \unfold{\Uof{e}}\\
% \Uof{\aetpl{\labelset}{\mapschema{e}{i}{\labelset}}} & = \tpl{\mapschemax{\Uofv}{e}{i}{\labelset}}\\
% \Uof{\aepr{\ell}{e}} & = \prj{\Uof{e}}{\ell}\\
% \Uof{\aein{\ell}{e}} &= \inj{\ell}{\Uof{e}}\\
% \Uof{\aematchwith{n}{e}{\seqschemaX{r}}} & = \matchwith{\Uof{e}}{\seqschemaXx{\Uofv}{r}}\\
% \Uof{\amval{X}} & = \mval{\sigilof{X}}
% \end{align*}
% \end{itemize}
% \begin{itemize}
% \item Each expanded rule, $r$, maps onto an unexpanded rule, $\Uof{r}$, as follows:
% \begin{align*}
% \Uof{\aematchrule{p}{e}} & = \aumatchrule{\Uof{p}}{\Uof{e}}
% \end{align*}
% \item Each expanded pattern, $p$, maps onto an unexpanded pattern, $\Uof{p}$, as follows:
% \begin{align*}
% \Uof{x} & = \sigilof{x}\\
% \Uof{\aewildp} &= \auwildp\\
% \Uof{\aefoldp{p}} &= \aufoldp{\Uof{p}}\\
% \Uof{\aetplp{\labelset}{\mapschema{p}{i}{\labelset}}} & = \autplp{\labelset}{\mapschemax{\Uofv}{p}{i}{\labelset}}\\
% \Uof{\aeinjp{\ell}{p}} & = \auinjp{\ell}{\Uof{p}}
% \end{align*}
% \end{itemize}

% \subsubsection{Textual Syntax}
% There is also a context-free textual syntax for the UL. We need only posit the existence of partial metafunctions that satisfy the following condition. 
% \begin{condition}[Textual Representability]\label{condition:textual-representability-P} ~
% \begin{enumerate}
% % \item For each $\usigma$, there exists $b$ such that $\parseUSig{b}{\usigma}$.
% % \item For each $\uM$, there exists $b$ such that $\parseUMod{b}{\uM}$.
% \item For each $\ukappa$, there exists $b$ such that $\parseUKind{b}{\ukappa}$.
% \item For each $\uc$, there exists $b$ such that $\parseUCon{b}{\uc}$.
% \item For each $\ue$, there exists $b$ such that $\parseUExp{b}{\ue}$.
% \item For each $\upv$, there exists $b$ such that $\parseUPat{b}{\upv}$.
% \end{enumerate}
% \end{condition}

% \begin{condition}[Expression Parsing Monotonicity]\label{condition:body-parsing-P} If $\parseUExp{b}{\ue}$ then $\sizeof{\ue} < \sizeof{b}$.\end{condition}

% \begin{condition}[Pattern Parsing Monotonicity]\label{condition:pattern-parsing-P} If $\parseUPat{b}{\upv}$ then $\sizeof{\upv} < \sizeof{b}$.\end{condition}

% \subsection{Typed Expansion}\label{appendix:typed-expansion-P}
% \subsubsection{Unexpanded Unified Contexts}\label{appendix:u-unified-ctxs}
% A \emph{unexpanded unified context}, $\uOmega$, takes the form $\uOmegaEx{\uD}{\uG}{\uMctx}{\Omega}$, where $\uMctx$ is a \emph{module identifier expansion context}, $\uD$ is a \emph{construction identifier expansion context}, $\uG$ is an \emph{expression identifier expansion context}, and $\Omega$ is a unified context.

% % \subsubsection{Identifier Expansion Contexts}
% A module identifier expansion context, $\uMctx$, is a finite function that maps each module identifier $\uX \in \domof{\uMctx}$ to the module identifier expansion $\vExpands{\uX}{X}$. We write $\uOmega, \uMhyp{\uX}{X}{\sigma}$ when $\uOmega=\uOmegaEx{\uD}{\uG}{\uMctx}{\Omega}$ as an abbreviation of \[\uOmegaEx{\uD}{\uG}{\uMctx \uplus \vExpands{\uX}{X}}{\Omega, X : \sigma}\]

% A construction identifier expansion context, $\uD$, is a finite function that maps each construction identifier $\uu \in \domof{\uD}$ to the construction identifier expansion $\vExpands{\uu}{u}$. We write $\uOmega, \uKhyp{\uu}{u}{\kappa}$ when $\uOmega=\uOmegaEx{\uD}{\uG}{\uMctx}{\Omega}$ as an abbreviation of \[\uOmegaEx{\uD \uplus \vExpands{\uu}{u}}{\uG}{\uMctx}{\Omega, u :: \kappa}\]

% An expression identifier expansion context, $\uG$, is a finite function that maps each expression identifier $\ux \in \domof{\uG}$ to the expression identifier expansion $\vExpands{\ux}{x}$. We write $\uOmega, \uGhyp{\ux}{x}{\tau}$ when $\uOmega=\uOmegaEx{\uD}{\uG}{\uMctx}{\Omega}$ as an abbreviation of \[\uOmegaEx{\uD}{\uG \uplus \vExpands{\ux}{x}}{\uMctx}{\Omega, x : \tau}\]

% \subsubsection{Body Encoding and Decoding}
% An assumed type abbreviated $\tBody$ classifies encodings of literal bodies, $b$. The mapping from literal bodies to values of type $\tBody$ is defined by the \emph{body encoding judgement} $\encodeBody{b}{\ebody}$. An inverse mapping is defined   by the \emph{body decoding judgement} $\decodeBody{\ebody}{b}$.
% \[\begin{array}{ll}
% \textbf{Judgement Form} & \textbf{Description}\\
% \encodeBody{b}{e} & \text{$b$ has encoding $e$}\\
% \decodeBody{e}{b} & \text{$e$ has decoding $b$}
% \end{array}\]
% The following condition establishes an isomorphism between literal bodies and values of type $\tBody$ mediated by the judgements above.
% \begin{condition}[Body Isomorphism]\label{condition:body-isomorphism-P} ~
% \begin{enumerate}
% \item For every literal body $b$, we have that $\encodeBody{b}{\ebody}$ for some $\ebody$ such that $\hastypeUC{\ebody}{\tBody}$ and $\isvalU{\ebody}$.
% \item If $\hastypeUC{\ebody}{\tBody}$ and $\isvalU{\ebody}$ then $\decodeBody{\ebody}{b}$ for some $b$.
% \item If $\encodeBody{b}{\ebody}$ then $\decodeBody{\ebody}{b}$.
% \item If $\hastypeUC{\ebody}{\tBody}$ and $\isvalU{\ebody}$ and $\decodeBody{\ebody}{b}$ then $\encodeBody{b}{\ebody}$. 
% \item If $\encodeBody{b}{\ebody}$ and $\encodeBody{b}{\ebody'}$ then $\ebody = \ebody'$.
% \item If $\hastypeUC{\ebody}{\tBody}$ and $\isvalU{\ebody}$ and $\decodeBody{\ebody}{b}$ and $\decodeBody{\ebody}{b'}$ then $b=b'$.
% \end{enumerate}
% \end{condition}
% We also assume a partial metafunction, $\bsubseq{b}{m}{n}$, which extracts a subsequence of $b$ starting at position $m$ and ending at position $n$, inclusive, where $m$ and $n$ are natural numbers. The following condition is technically necessary.
% \begin{condition}[Body Subsequencing]\label{condition:body-subsequences-P} If $\bsubseq{b}{m}{n}=b'$ then $\sizeof{b'} \leq \sizeof{b}$. \end{condition}

% \subsubsection{Parse Results}
%  The type function abbreviated $\tParseResultF$, and auxiliary abbreviations used below, is defined as follows:
% \begin{align*}
% L_\mathtt{P} & \defeq \lbltxt{Error}, \lbltxt{Success}\\
% \tParseResultF & \defeq \acabs{t}{\asum{L_\mathtt{P}}{
%   \mapitem{\lbltxt{Error}}{\prodt{}}, 
%   \mapitem{\lbltxt{Success}}{t}
% }}\\
% \tParseResult{\tau} & \defeq \acapp{\tParseResultF}{\tau}\\
% % \lbltxt{SuccessE}\cdot e & \defeq \aein{L_\mathtt{P}}{\mathtt{Success}}{\mapitem{\mathtt{ParseError}}{\tpl{}}, \mapitem{\mathtt{Success}}{\tPProtoExpr}}{e}\\
% % \lbltxt{SuccessP}\cdot e & \defeq \aein{L_\mathtt{P}}{\mathtt{Success}}{\mapitem{\mathtt{ParseError}}{\tpl{}}, \mapitem{\mathtt{Success}}{\tCEPat}}{e}
% \end{align*} %[\mapitem{\lbltxt{Error}}{\prodt{}}, \mapitem{\lbltxt{SuccessE}}{\tCEExp}]

% \subsubsection{TLM Contexts}
% \emph{peTLM contexts}, $\uPsi$, are of the form $\uAS{\uA}{\Psi}$, where $\uA$ is a \emph{TLM identifier expansion context} and $\Psi$ is a \emph{peTLM definition context}.

% \emph{ppTLM contexts}, $\uPhi$, are of the form $\uAS{\uA}{\Phi}$, where $\uA$ is a TLM identifier expansion context and $\Phi$ is a \emph{ppTLM definition context}.

% A \emph{TLM identifier expansion context}, $\uA$, is a finite function mapping each TLM identifier $\tsmv \in \domof{\uA}$ to the \emph{TLM identifier expansion}, $\vExpands{\tsmv}{\epsilon}$, for some \emph{TLM expression}, $\epsilon$. We write $\ctxUpdate{\uA}{\tsmv}{\epsilon}$ for the TLM identifier expansion context that maps $\tsmv$ to $\vExpands{\tsmv}{\epsilon}$, and defers to $\uA$ for all other TLM identifiers (i.e. the previous mapping is \emph{updated}.)

% A \emph{peTLM definition context}, $\Psi$, is a finite function mapping each TLM name $x \in \domof{\Psi}$ to an \emph{expanded peTLM definition}, $\petsmdefn{a}{\rho}{\eparse}$, where $\rho$ is the peTLM's type annotation, and $\eparse$ is its parse function. We write $\Psi, \petsmdefn{a}{\rho}{\eparse}$ when $a \notin \domof{\Psi}$ for the extension of $\Psi$ that maps $x$ to $\petsmdefn{a}{\rho}{\eparse}$. We write $\petsmenv{\Omega}{\Psi}$  when all the TLM type annotations in $\Psi$ are well-formed assuming $\Omega$, and the parse functions in $\Psi$ are closed and of the appropriate type.


% \begin{definition}[peTLM Definition Context Formation]\label{def:peTLM-def-ctx-formation} $\petsmenv{\Omega}{\Psi}$ iff for each ${\petsmdefn{a}{\rho}{\eparse}} \in \Psi$, we have $\istsmty{\Omega}{\rho}$ and \[\hastypeP{\emptyset}{\eparse}{\aparr{\tBody}{\tParseResultPCEExp}}\]\end{definition}

% \begin{definition}[peTLM Context Formation] $\petsmctx{\Omega}{\uAS{\uA}{\Psi}}$ iff $\petsmenv{\Omega}{\Psi}$ and for each $\vExpands{\tsmv}{\epsilon} \in \uA$ we have $\hastsmtypeExp{\Omega}{\Psi}{\epsilon}{\rho}$ for some $\rho$.
% \end{definition}

% A \emph{ppTLM definition context}, $\Phi$, is a finite function mapping each TLM name $x \in \domof{\Phi}$ to an \emph{expanded ppTLM definition}, $\pptsmdefn{a}{\rho}{\eparse}$, where $\rho$ is the ppTLM's type annotation, and $\eparse$ is its parse function. We write $\Phi, \pptsmdefn{a}{\rho}{\eparse}$ when $a \notin \domof{\Phi}$ for the extension of $\Phi$ that maps $x$ to $\pptsmdefn{a}{\rho}{\eparse}$. We write $\pptsmenv{\Omega}{\Phi}$  when all the type annotations in $\Phi$ are well-formed assuming $\Omega$, and the parse functions in $\Phi$ are closed and of the appropriate type.

% \begin{definition}[ppTLM Definition Context Formation]\label{def:ppTLM-def-ctx-formation} $\pptsmenv{\Omega}{\Phi}$ iff for each $\pptsmdefn{\tsmv}{\rho}{\eparse} \in \Phi$, we have $\istsmty{\Omega}{\rho}$ and \[\hastypeP{\emptyset}{\eparse}{\aparr{\tBody}{\tParseResultCEPat}}\]\end{definition}

% \begin{definition}[ppTLM Context Formation] $\pptsmctx{\Omega}{\uAS{\uA}{\Phi}}$ iff $\pptsmenv{\Omega}{\Phi}$ and for each $\vExpands{\tsmv}{\epsilon} \in \uA$ we have $\hastsmtypePat{\Omega}{\Phi}{\epsilon}{\rho}$ for some $\rho$.
% \end{definition}

% \subsubsection{Signature and Module Expansion}
% \noindent\fbox{$\strut\sigExpandsPX{\usigma}{\sigma}$}~~$\usigma$ has well-formed expansion $\sigma$
% \begin{equation}\label{rule:sigExpandsP}
% \inferrule{
%   \kExpandsX{\ukappa}{\kappa}\\
%   \cExpands{\uOmega, \uKhyp{\uu}{u}{\kappa}}{\utau}{\tau}{\akty}
% }{
%   \sigExpandsPX{\signature{\uu}{\ukappa}{\utau}}{\asignature{\kappa}{u}{\tau}}
% }
% \end{equation}

% \noindent\fbox{$\strut\mExpandsPX{\uM}{M}{\sigma}$}~~$\uM$ has expansion $M$ matching $\sigma$
% \begin{subequations}\label{rules:mExpandsP}
% \begin{equation}\label{rule:mExpandsP-subsumes}
% \inferrule{
%   \mExpandsPX{\uM}{M}{\sigma}\\
%   \sigsub{\uOmega}{\sigma}{\sigma'}
% }{
%   \mExpandsPX{\uM}{M}{\sigma'}
% }
% \end{equation}
% \begin{equation}\label{rule:mExpandsP-var}
% \inferrule{ }{
%   \mExpandsP{\uOmega, \uMhyp{\uX}{X}{\sigma}}{\uPsi}{\uPhi}{\uX}{X}{\sigma}
% }
% \end{equation}
% \begin{equation}\label{rule:mExpandsP-struct}
% \inferrule{
%   \cExpandsX{\uc}{c}{\kappa}\\
%   \expandsPX{\ue}{e}{[c/u]\tau}
% }{
%   \mExpandsPX{\struct{\uc}{\ue}}{\astruct{c}{e}}{\asignature{\kappa}{u}{\tau}}
% }
% \end{equation}
% \begin{equation}\label{rule:mExpandsP-seal}
% \inferrule{
%   \sigExpandsPX{\usigma}{\sigma}\\
%   \mExpandsPX{\uM}{M}{\sigma}
% }{
%   \mExpandsPX{\seal{\uM}{\usigma}}{\aseal{\sigma}{M}}{\sigma} 
% }
% \end{equation}
% \begin{equation}\label{rule:mExpandsP-mlet}
% \inferrule{
%   \mExpandsPX{\uM}{M}{\sigma}\\
%   \sigExpandsPX{\usigma'}{\sigma'}\\\\
%   \mExpandsP{\uOmega, \uMhyp{\uX}{X}{\sigma}}{\uPsi}{\uPhi}{\uM'}{M'}{\sigma'}
% }{
%   \mExpandsPX{\mlet{\uX}{\uM}{\uM'}{\usigma'}}{\amlet{\sigma'}{M}{X}{M'}}{\sigma'}
% }
% \end{equation}
% \begin{equation}\label{rule:mExpandsP-syntaxpe}
% \inferrule{
%   \tsmtyExpands{\uOmega}{\urho}{\rho}\\
%   \hastypeP{\emptyset}{\eparse}{\aparr{\tBody}{\tParseResultPCEExp}}\\\\
%   \evalU{\eparse}{\eparse'}\\
%   \mExpandsP{\uOmega}{\uAS{\uA \uplus \mapitem{\tsmv}{\adefref{a}}}{\Psi, \petsmdefn{a}{\rho}{\eparse'}}}{\uPhi}{\uM}{M}{\sigma}
% }{
%   \mExpandsP{\uOmega}{\uAS{\uA}{\Psi}}{\uPhi}{\defpetsm{\tsmv}{\urho}{\eparse}{\uM}}{M}{\sigma}
% }
% \end{equation}
% \begin{equation}\label{rule:mExpandsP-letpetsm}
% \inferrule{
%   % \uOmega = \uOmegaEx{\uD}{\uG}{\uMctx}{\Omega}\\
%   \tsmexpExpandsExp{\uOmega}{\uAS{\uA}{\Psi}}{\uepsilon}{\epsilon}{\rho}\\
%   % \tsmexpEvalsExp{\Omega}{\Psi}{\epsilon}{\epsilon_\text{normal}}\\\\
%   \mExpandsP{\uOmega}{\uAS{\uA\uplus\mapitem{\tsmv}{\epsilon_\text{normal}}}{\Psi}}{\uPhi}{\uM}{M}{\sigma}
% }{
%   \mExpandsP{\uOmega}{\uAS{\uA}{\Psi}}{\uPhi}{\uletpetsm{\tsmv}{\uepsilon}{\uM}}{M}{\sigma}
% }
% \end{equation}
% \begin{equation}\label{rule:mExpandsP-syntaxpp}
% \inferrule{ 
%   \tsmtyExpands{\uOmega}{\urho}{\rho}\\
%   \hastypeP{\emptyset}{\eparse}{\aparr{\tBody}{\tParseResultCEPat }}\\\\
%   \evalU{\eparse}{\eparse'}\\
%   \mExpandsP{\uOmega}{\uPsi}{\uAS{\uA \uplus \mapitem{\tsmv}{\adefref{a}}}{\Phi, \pptsmdefn{a}{\rho}{\eparse'}}}{\uM}{M}{\sigma}
% }{
%   \mExpandsP{\uOmega}{\uPsi}{\uAS{\uA}{\Phi}}{\defpptsm{\tsmv}{\urho}{\eparse}{\uM}}{M}{\sigma}
% }
% \end{equation}
% \begin{equation}\label{rule:mExpandsP-letpptsm}
% \inferrule{
%   \tsmexpExpandsPat{\uOmega}{\uAS{\uA}{\Phi}}{\uepsilon}{\epsilon}{\rho}\\
%   \mExpandsP{\uOmega}{\uPsi}{\uAS{\uA\uplus\mapitem{\tsmv}{\epsilon}}{\Phi}}{\uM}{M}{\sigma}
% }{
%   \mExpandsP{\uOmega}{\uPsi}{\uAS{\uA}{\Phi}}{\uletpptsm{\tsmv}{\uepsilon}{\uM}}{M}{\sigma}
% }
% \end{equation}
% \end{subequations}

% \subsubsection{Kind and Construction Expansion}
% \noindent\fbox{$\strut\kExpandsX{\ukappa}{\kappa}$}~~$\ukappa$ has well-formed expansion $\kappa$
% \begin{subequations}\label{rules:kExpands}
% \begin{equation}\label{rule:kExpands-darr}
% \inferrule{
%   \kExpandsX{\ukappa_1}{\kappa_1}\\
%   \kExpands{\uOmega, \uKhyp{\uu}{u}{\kappa_1}}{\ukappa_2}{\kappa_2}
% }{
%   \kExpandsX{\kdarr{\uu}{\ukappa_1}{\ukappa_2}}{\akdarr{\kappa_1}{u}{\kappa_2}}
% }
% \end{equation}
% \begin{equation}\label{rule:kExpands-unit}
% \inferrule{ }{
%   \kExpandsX{\kunit}{\akunit}
% }
% \end{equation}
% \begin{equation}\label{rule:kExpands-dprod}
% \inferrule{
%   \kExpandsX{\ukappa_1}{\kappa_1}\\
%   \kExpands{\uOmega, \uKhyp{\uu}{u}{\kappa_1}}{\ukappa_2}{\kappa_2}
% }{
%   \kExpandsX{\kdbprod{\uu}{\ukappa_1}{\ukappa_2}}{\akdbprod{\kappa_1}{u}{\kappa_2}}
% }
% \end{equation}
% \begin{equation}\label{rule:kExpands-ty}
% \inferrule{ }{
%   \kExpandsX{\kty}{\akty}
% }
% \end{equation}
% \begin{equation}\label{rule:kExpands-sing}
% \inferrule{
%   \cExpandsX{\utau}{\tau}{\akty}
% }{
%   \kExpandsX{\ksing{\utau}}{\aksing{\tau}}
% }
% \end{equation}
% \end{subequations}

% \noindent\fbox{$\strut\cExpandsX{\uc}{c}{\kappa}$}~~$\uc$ has expansion $c$ of kind $\kappa$
% \begin{subequations}\label{rules:cExpands}
% \begin{equation}\label{rule:cExpands-subsume}
% \inferrule{
%   \cExpandsX{\uc}{c}{\kappa_1}\\
%   \ksubX{\kappa_1}{\kappa_2}
% }{
%   \cExpandsX{\uc}{c}{\kappa_2}
% }
% \end{equation}
% \begin{equation}\label{rule:cExpands-var}
% \inferrule{ }{\cExpands{\uOmega, \uKhyp{\uu}{u}{\kappa}}{\uu}{u}{\kappa}}
% \end{equation}
% \begin{equation}\label{rule:cExpands-abs}
% \inferrule{
%   \cExpands{\uOmega, \uKhyp{\uu}{u}{\kappa_1}}{\uc_2}{c_2}{\kappa_2}
% }{
%   \cExpandsX{\cabs{\uu}{\uc_2}}{\acabs{u}{c_2}}{\akdarr{\kappa_1}{u}{\kappa_2}}
% }
% \end{equation}
% \begin{equation}\label{rule:cExpands-app}
% \inferrule{
%   \cExpandsX{\uc_1}{c_1}{\akdarr{\kappa_2}{u}{\kappa}}\\
%   \cExpandsX{\uc_2}{c_2}{\kappa_2}
% }{
%   \cExpandsX{\capp{\uc_1}{\uc_2}}{\acapp{c_1}{c_2}}{[c_1/u]\kappa}
% }
% \end{equation}
% \begin{equation}\label{rule:cExpands-unit}
% \inferrule{ }{
%   \cExpandsX{\ctriv}{\actriv}{\akunit}
% }
% \end{equation}
% \begin{equation}\label{rule:cExpands-pair}
% \inferrule{
%   \cExpandsX{\uc_1}{c_1}{\kappa_1}\\
%   \cExpandsX{\uc_2}{c_2}{[c_1/u]\kappa_2}
% }{
%   \cExpandsX{\cpair{\uc_1}{\uc_2}}{\acpair{c_1}{c_2}}{\akdbprod{\kappa_1}{u}{\kappa_2}}
% }
% \end{equation}
% \begin{equation}\label{rule:cExpands-prl}
% \inferrule{
%   \cExpandsX{\uc}{c}{\akdbprod{\kappa_1}{u}{\kappa_2}}
% }{
%   \cExpandsX{\cprl{\uc}}{\acprl{c}}{\kappa_1}
% }
% \end{equation}
% \begin{equation}\label{rule:cExpands-prr}
% \inferrule{
%   \cExpandsX{\uc}{c}{\akdbprod{\kappa_1}{u}{\kappa_2}}
% }{
%   \cExpandsX{\cprr{\uc}}{\acprr{c}}{[\acprl{c}/u]\kappa_2}
% }
% \end{equation}
% \begin{equation}\label{rule:cExpands-parr}
% \inferrule{
%   \cExpandsX{\utau_1}{\tau_1}{\akty}\\
%   \cExpandsX{\utau_2}{\tau_2}{\akty}
% }{
%   \cExpandsX{\parr{\utau_1}{\utau_2}}{\aparr{\tau_1}{\tau_2}}{\akty}
% }
% \end{equation}
% \begin{equation}\label{rule:cExpands-all}
% \inferrule{
%   \kExpandsX{\ukappa}{\kappa}\\
%   \cExpands{\uOmega, \uKhyp{\uu}{u}{\kappa}}{\utau}{\tau}{\akty}
% }{
%   \cExpandsX{\forallu{\uu}{\ukappa}{\utau}}{\aallu{\kappa}{u}{\tau}}{\akty}
% }
% \end{equation}
% \begin{equation}\label{rule:cExpands-rec}
% \inferrule{
%   \cExpands{\uOmega, \uKhyp{\ut}{t}{\akty}}{\utau}{\tau}{\akty}
% }{
%   \cExpandsX{\rect{\ut}{\utau}}{\arec{t}{\tau}}{\akty}
% }
% \end{equation}
% \begin{equation}\label{rule:cExpands-prod}
% \inferrule{
%   \{\cExpandsX{\utau_i}{\tau_i}{\akty}\}_{1 \leq i \leq n}
% }{
%   \cExpandsX{\prodt{\mapschema{\utau}{i}{\labelset}}}{\aprod{\labelset}{\mapschema{\tau}{i}{\labelset}}}{\akty}
% }
% \end{equation}
% \begin{equation}\label{rule:cExpands-sum}
% \inferrule{
%   \{\cExpandsX{\utau_i}{\tau_i}{\akty}\}_{1 \leq i \leq n}
% }{
%   \cExpandsX{\sumt{\mapschema{\utau}{i}{\labelset}}}{\asum{\labelset}{\mapschema{\tau}{i}{\labelset}}}{\akty}
% }
% \end{equation}
% \begin{equation}\label{rule:cExpands-sing}
% \inferrule{
%   \cExpandsX{\uc}{c}{\akty}
% }{
%   \cExpandsX{\uc}{c}{\aksing{c}}
% }
% \end{equation}
% \begin{equation}\label{rule:cExpands-stat}
% \inferrule{ }{
%   \cExpands{\uOmega, \uMhyp{\uX}{X}{\asignature{\kappa}{u}{\tau}}}{\mcon{\uX}}{\amcon{X}}{\kappa}
% }
% \end{equation}
% \end{subequations}

% \subsubsection{Type, Expression, Rule and Pattern Expansion}
% % \noindent\fbox{$\strut\tExpandsPX{\utau}{\tau}$}~~$\utau$ has well-formed expansion $\tau$
% % \begin{equation}\label{rule:tExpandsP}
% % \inferrule{
% %   \cExpandsX{\utau}{\tau}{\akty}
% % }{
% %   \tExpandsPX{\utau}{\tau}
% % }
% % \end{equation}

% \noindent\fbox{$\strut\expandsPX{\ue}{e}{\tau}$}~~$\ue$ has expansion $e$ of type $\tau$
% \begin{subequations}\label{rules:expandsP}
% \begin{equation}\label{rule:expandsP-subsume}
%   \inferrule{
%     \expandsPX{\ue}{e}{\tau}\\
%     \issubtypePX{\tau}{\tau'}
%   }{
%     \expandsPX{\ue}{e}{\tau'}
%   }
% \end{equation}

% \begin{equation}\label{rule:expandsP-var}
%   \inferrule{ }{ 
%     \expandsP{\uOmega, \uGhyp{\ux}{x}{\tau}}{\uPsi}{\uPhi}{\ux}{x}{\tau}
%   }
% \end{equation}
% \begin{equation}\label{rule:expandsP-asc}
%   \inferrule{
%     \cExpands{\uOmega}{\utau}{\tau}{\akty}\\
%     \expandsP{\uOmega}{\uPsi}{\uPhi}{\ue}{e}{\tau}
%   }{
%     \expandsP{\uOmega}{\uPsi}{\uPhi}{\asc{\ue}{\utau}}{e}{\tau}
%   }
% \end{equation}
% \begin{equation}\label{rule:expandsP-letsyn}
%   \inferrule{
%     \expandsP{\uOmega}{\uPsi}{\uPhi}{\ue_1}{e_1}{\tau_1}\\
%     \expandsP{\uOmega, \uGhyp{\ux}{x}{\tau_1}}{\uPsi}{\uPhi}{\ue_2}{e_2}{\tau_2}
%   }{
%     \expandsP{\uOmega}{\uPsi}{\uPhi}{\letsyn{\ux}{\ue_1}{\ue_2}}{
%       \aeap{\aelam{\tau_1}{x}{e_2}}{e_1}
%     }{\tau_2}
%   }
% \end{equation}
% %Functions with an argument type annotation can appear in synthetic position.
% \begin{equation}\label{rule:expandsP-lam}
%   \inferrule{
%     \cExpands{\uOmega}{\utau_1}{\tau_1}{\akty}\\
%     \expandsP{\uOmega, \uGhyp{\ux}{x}{\tau_1}}{\uPsi}{\uPhi}{\ue}{e}{\tau_2}
%   }{
%     \expandsPX{\lam{\ux}{\utau_1}{\ue}}{\aelam{\tau_1}{x}{e}}{\aparr{\tau_1}{\tau_2}}
%   }
% \end{equation}

% %Function applications can appear in synthetic position. The argument is analyzed against the argument type synthesized by the function.
% \begin{equation}\label{rule:expandsP-ap}
%   \inferrule{
%     \expandsPX{\ue_1}{e_1}{\aparr{\tau_2}{\tau}}\\
%     \expandsPX{\ue_2}{e_2}{\tau_2}
%   }{
%     \expandsPX{\ap{\ue_1}{\ue_2}}{\aeap{e_1}{e_2}}{\tau}
%   }
% \end{equation}

% %Type lambdas and type applications can appear in synthetic position.
% \begin{equation}\label{rule:expandsP-tlam}
%   \inferrule{
%     \kExpandsX{\ukappa}{\kappa}\\
%     \expandsP{\uOmega, \uKhyp{\uu}{u}{\kappa}}{\uPsi}{\uPhi}{\ue}{e}{\tau}
%   }{
%     \expandsPX{\clam{\uu}{\ukappa}{\ue}}{\aeclam{\kappa}{u}{e}}{\aallu{\kappa}{u}{\tau}}
%   }
% \end{equation}
% \begin{equation}\label{rule:expandsP-tap}
%   \inferrule{
%     \expandsPX{\ue}{e}{\aallu{\kappa}{u}{\tau}}\\
%     \ksynX{\uc}{c}{\kappa}
%   }{
%     \expandsPX{\cAp{\ue}{\uc}}{\aecap{e}{c}}{[c/t]\tau}
%   }
% \end{equation}
% % Values of recursive types can be introduced only in analytic position.
% \begin{equation}\label{rule:expandsP-fold}
%   \inferrule{
%     \expandsPX{\ue}{e}{[\arec{t}{\tau}/t]\tau}
%   }{
%     \expandsPX{\fold{\ue}}{\aefold{e}}{\arec{t}{\tau}}
%   }
% \end{equation}

% %Unfoldings can appear in synthetic position.
% \begin{equation}\label{rule:expandsP-unfold}
%   \inferrule{
%     \expandsPX{\ue}{e}{\arec{t}{\tau}}
%   }{
%     \expandsPX{\unfold{\ue}}{\aeunfold{e}}{[\arec{t}{\tau}/t]\tau}
%   }
% \end{equation}

% %Labeled tuples can appear in synthetic position. Each of the field values are then in synthetic position. 
% \begin{equation}\label{rule:expandsP-tpl}
%   \inferrule{
%     \{\expandsPX{\ue_i}{e_i}{\tau_i}\}_{i \in \labelset}
%   }{
%     \expandsPX{\tpl{\mapschema{\ue}{i}{\labelset}}}{\aetpl{\labelset}{\mapschema{e}{i}{\labelset}}}{\aprod{\labelset}{\mapschema{\tau}{i}{\labelset}}}
%   }
% \end{equation}

% %Fields can be projected out of a labeled tuple in synthetic position.
% \begin{equation}\label{rule:expandsP-pr}
%   \inferrule{
%     \expandsPX{\ue}{e}{\aprod{\labelset, \ell}{\mapschema{\tau}{i}{\labelset}; \mapitem{\ell}{\tau}}}
%   }{
%     \expandsPX{\prj{\ue}{\ell}}{\aepr{\ell}{e}}{\tau}
%   }
% \end{equation}

% % Values of labeled sum type can appear only in analytic position.
% \begin{equation}\label{rule:expandsP-in}
%   \inferrule{
%     \expandsPX{\ue'}{e'}{\tau'}
%   }{
%     \expandsPX{\inj{\ell}{\ue}}{\aein{\ell}{e'}}{\asum{\labelset, \ell}{\mapschema{\tau}{i}{\labelset}; \mapitem{\ell}{\tau'}}}
%     % \uOmega \vdash_{\uPsi; \uPhi} \left(\shortstack{$\ue \leadsto $\\$\Leftarrow$\vspace{-1.2em}}\right)
%     %\expandsPX{\auanain{\ell}{\ue}}{\aein{\ell}}{\asum{\labelset, \ell}{\mapschema{\tau}{i}{\labelset}; \mapitem{\ell}{\tau}}}
%   }
% \end{equation}

% %Match expressions can appear in synthetic position.
% \begin{equation}\label{rule:expandsP-match}
%   \inferrule{
%     % \uOmega = \uOmegaEx{\uD}{\uG}{\uM}{\Omega}\\\\
%     \expandsPX{\ue}{e}{\tau}\\
%     % \haskind{\Omega}{\tau'}{\akty}\\
%     \{\rsynPX{\urv_i}{r_i}{\tau}{\tau'}\}_{1 \leq i \leq n}
%   }{
%     \expandsPX{\matchwith{\ue}{\seqschemaX{\urv}}}{\aematchwith{n}{e}{\seqschemaX{r}}}{\tau'}
%   }
% \end{equation}

% \begin{equation}\label{rule:expandsP-mval}
%   \inferrule{ }{
%     \expandsP{\uOmega, \uMhyp{\uX}{X}{\asignature{\kappa}{u}{\tau}}}{\uPsi}{\uPhi}{\mval{\uX}}{\amval{X}}{[\amcon{X}/u]\tau}
%   }
% \end{equation}

% % ueTLMs can be defined and applied in synthetic position.
% % \begin{equation}\label{rule:expandsP-defpetsm}
% % \inferrule{
% %   \tsmtyExpands{\uOmega}{\urho}{\rho}\\
% %   \hastypeP{\emptyset}{\eparse}{\aparr{\tBody}{\tParseResultPCEExp}}\\\\
% %   \expandsP{\uOmega}{\uASI{\uA \uplus \mapitem{\tsmv}{\adefref{a}}}{\Psi, \petsmdefn{a}{\rho}{\eparse}}{\uI}}{\uPhi}{\ue}{e}{\tau}
% % }{
% %   \expandsP{\uOmega}{\uASI{\uA}{\Psi}{\uI}}{\uPhi}{\usyntaxueP{\tsmv}{\urho}{\eparse}{\ue}}{e}{\tau}
% % }
% % \end{equation}

% % \begin{equation}\label{rule:expandsP-letpetsm}
% % \inferrule{
% %   \tsmexpExpandsExp{\uOmega}{\uASI{\uA}{\Psi}{\uI}}{\uepsilon}{\epsilon}{\rho}\\
% %   \expandsP{\uOmega}{\uASI{\uA\uplus\mapitem{\tsmv}{\epsilon}}{\Psi}{\uI}}{\uPhi}{\ue}{e}{\tau}
% % }{
% %   \expandsP{\uOmega}{\uASI{\uA}{\Psi}{\uI}}{\uPhi}{\uletpetsm{\tsmv}{\uepsilon}{\ue}}{e}{\tau}
% % }
% % \end{equation}

% \begin{equation}\label{rule:expandsP-apuetsm}
% \inferrule{
%   \uOmega = \uOmegaEx{\uD}{\uG}{\uMctx}{\Omega_\text{app}}\\
%   \uPsi=\uAS{\uA}{\Psi}\\\\
%   \tsmexpExpandsExp{\uOmega}{\uPsi}{\uepsilon}{\epsilon}{\aetype{\tau_\text{final}}}\\
%   \tsmexpEvalsExp{\Omega_\text{app}}{\Psi}{\epsilon}{\epsilon_\text{normal}}\\\\
%   \tsmdefof{\epsilon_\text{normal}}=a\\
%   \Psi = \Psi', \petsmdefn{a}{\rho}{\eparse}\\\\
%   \encodeBody{b}{\ebody}\\
%   \evalU{\ap{\eparse}{\ebody}}{\aein{\mathtt{SuccessE}}{e_\text{pproto}}}\\
%   \decodePCEExp{e_\text{pproto}}{\pce}\\\\
%   \prepce{\Omega_\text{app}}{\Psi}{\pce}{\ce}{\epsilon_\text{normal}}{\aetype{\tau_\text{proto}}}{\omega}{\Omega_\text{params}}\\\\
%      \segOKP{\OParams}{\csceneP{\omega : \OParams}{\uOmega}{b}}{\segof{\ce}}{b}\\
%   \cvalidEP{\Omega_\text{params}}{\esceneP{\omega : \OParams}{\uOmega}{\uPsi}{\uPhi}{b}}{\ce}{e}{\tau_\text{proto}}
% }{
%   \expandsP{\uOmega}{\uPsi}{\uPhi}{\utsmap{\uepsilon}{b}}{[\omega]e}{[\omega]\tau_\text{proto}}
% }
% \end{equation}

% % These rules are nearly identical to Rules (\ref{rule:expandsUP-syntax}) and (\ref{rule:expandsUP-tsmap}), differing only in that the typed expansion premises have been replaced by corresponding synthetic typed expansion premises. The premises of these rules can be understood as described in Sections \ref{sec:U-uetsm-definition} and \ref{sec:U-uetsm-application}. The body encoding judgement and candidate expansion expression decoding judgements were characterized in Sec. \ref{sec:typed-expansion-UP}. We discuss candidate expansion validation in Sec. \ref{sec:ce-validation-B} below.

% % To support ueTLM implicits, ueTLM contexts, $\uPsi$, are redefined to take the form $\uASI{\uA}{\Psi}{\uI}$. TLM naming contexts, $\uA$, and ueTLM definition contexts, $\Psi$, were defined in Sec. \ref{sec:typed-expansion-UP}. We write $\uPsi, \uShyp{\tsmv}{x}{\tau}{\eparse}$ when $\uPsi=\uASI{\uA}{\Psi}{\uI}$ as shorthand for \[\uASI{\ctxUpdate{\uA}{\tsmv}{x}}{\Psi, \xuetsmbnd{x}{\tau}{\eparse}}{\uI}\]

% % \emph{TLM designation contexts}, $\uI$, are finite functions that map each type $\tau \in \domof{\uI}$ to the \emph{TLM designation} $\designate{\tau}{a}$, for some symbol $x$. We write $\uI \uplus \designate{\tau}{a}$ for the TLM designation context that maps $\tau$ to $\designate{\tau}{a}$ and defers to $\uI$ for all other types (i.e. the previous designation, if any, is updated). 

% % The TLM designation context in the ueTLM context is updated by expressions of ueTLM designation form. Such expressions can appear in synthetic position, where they are governed by the following rule:% We write $\uIOK{\Delta}{\uI}$ when each type in $\uI$ is well-formed assuming $\Delta$.
% %\begin{definition}[TLM Designation Context Well-Formedness] $\uIOK{\Delta}{{\uI}$ iff for each $\designate{\tau}{a}$ we have $\istypeU{\Delta}{\tau}$.\end{definition}

% % \todo{peTLM implicit designation}
% % \begin{equation}\label{rule:expandsP-implicite}
% %   \inferrule{
% %     \esyn{\uDelta}{\uGamma}{\uASI{\uA \uplus \vExpands{\tsmv}{x}}{\Psi, \xuetsmbnd{x}{\tau}{\eparse}}{\uI \uplus \designate{\tau}{a}}}{\uPhi}{\ue}{e}{\tau'}
% %   }{
% %     \esyn{\uDelta}{\uGamma}{\uASI{\uA \uplus \vExpands{\tsmv}{x}}{\Psi, \xuetsmbnd{x}{\tau}{\eparse}}{\uI}}{\uPhi}{\implicite{\tsmv}{\ue}}{e}{\tau'}
% %   }
% % \end{equation}

% % % Like ueTLMs, upTLMs can be defined in synthetic position.
% % \begin{equation}\label{rule:expandsP-syntaxup}
% % \inferrule{
% %   \tsmtyExpands{\uOmega}{\urho}{\rho}\\
% %   \hastypeP{\emptyset}{\eparse}{\aparr{\tBody}{\tParseResultCEPat}}\\\\
% %   \expandsP{\uOmega}{\uPsi}{\uASI{\uA \uplus \mapitem{\tsmv}{\adefref{a}}}{\Phi, \pptsmdefn{a}{\rho}{\eparse}}{\uI}}{\ue}{e}{\tau}
% % }{
% %   \expandsP{\uOmega}{\uPsi}{\uASI{\uA}{\Phi}{\uI}}{\usyntaxup{\tsmv}{\urho}{\eparse}{\ue}}{e}{\tau}
% % }
% % \end{equation}


% % \begin{equation}\label{rule:expandsP-letpptsm}
% % \inferrule{
% %   \tsmexpExpandsPat{\uOmega}{\uASI{\uA}{\Phi}{\uI}}{\uepsilon}{\epsilon}{\rho}\\
% %   \expandsP{\uOmega}{\uPsi}{\uASI{\uA\uplus\mapitem{\tsmv}{\epsilon}}{\Phi}{\uI}}{\ue}{e}{\tau}
% % }{
% %   \expandsP{\uOmega}{\uPsi}{\uASI{\uA}{\Phi}{\uI}}{\uletpptsm{\tsmv}{\uepsilon}{\ue}}{e}{\tau}
% % }
% % \end{equation}

% % % This rule is nearly identical to Rule (\ref{rule:expandsUP-defuptsm}), differing only in that the typed expansion premise has been replaced by the corresponding synthetic typed expansion premise. The premises can be understood as described in Section \ref{sec:uptsm-definition}.

% % % To support upTLM implicits, upTLM contexts, $\uPhi$, are redefined to take the form $\uASI{\uA}{\Phi}{\uI}$. upTLM definition contexts, $\Phi$, were defined in Sec. \ref{sec:uptsm-definition}. We write $\uPhi, \uPhyp{\tsmv}{x}{\tau}{\eparse}$ when $\uPhi=\uASI{\uA}{\Phi}{\uI}$ as shorthand for \[\uASI{\ctxUpdate{\uA}{\tsmv}{x}}{\Phi, \xuptsmbnd{a}{\tau}{\eparse}}{\uI}\]

% % % The TLM designation context in the upTLM context is updated by expressions of upTLM designation form. Such expressions can appear in synthetic position, where they are governed by the following rule:% We write $\uIOK{\Delta}{\uI}$ when each type in $\uI$ is well-formed assuming $\Delta$.
% % %\begin{definition}[TLM Designation Context Well-Formedness] $\uIOK{\Delta}{{\uI}$ iff for each $\designate{\tau}{a}$ we have $\istypeU{\Delta}{\tau}$.\end{definition}
% % \todo{ppTLM implicit designation}
% % \begin{equation}\label{rule:expandsP-implicitp}
% %   \inferrule{
% %     \esyn{\uDelta}{\uGamma}{\uPsi}{\uASI{\uA\uplus\vExpands{\tsmv}{x}}{\Phi, \xuptsmbnd{a}{\tau}{\eparse}}{\uI \uplus \designate{\tau}{a}}}{\ue}{e}{\tau'}
% %   }{
% %     \esyn{\uDelta}{\uGamma}{\uPsi}{\uASI{\uA\uplus\vExpands{\tsmv}{x}}{\Phi, \xuetsmbnd{x}{\tau}{\eparse}}{\uI}}{\implicitp{\tsmv}{\ue}}{e}{\tau'}
% %   }
% % \end{equation}
% \end{subequations}

% % \begin{subequations}\label{rules:expandsP}
% % Type analysis subsumes type synthesis, in that when a type can be synthesized for an unexpanded expression, that unexpanded expression can also be analyzed against that type, producing the same expansion. This is expressed by the following \emph{subsumption rule} for unexpanded expressions.

% % Additional rules are needed for certain forms in order to propagate types for analysis into subexpressions, and for forms that can appear only in analytic position.



% % Rule (\ref{rule:esyn-tpl}) governed labeled tuples in synthetic position. The following rule governs labeled tuples in analytic position.


% % Rule (\ref{rule:esyn-match}) governed match expressions in synthetic position. The following rule governs match expressions in analytic position.

% % Rule (\ref{rule:esyn-defuetsm}) governed ueTLM definitions in synthetic position. The following rule governs ueTLM definitions in analytic position.
% % \begin{equation}\label{rule:expandsP-defpetsm}
% % \inferrule{
% %   \tsmtyExpands{\uOmega}{\urho}{\rho}\\
% %   \hastypeP{\emptyset}{\eparse}{\aparr{\tBody}{\tParseResultPCEExp}}\\\\
% %   \expandsP{\uOmega}{\uASI{\uA \uplus \mapitem{\tsmv}{\adefref{a}}}{\Psi, \petsmdefn{a}{\rho}{\eparse}}{\uI}}{\uPhi}{\ue}{e}{\tau}
% % }{
% %   \expandsP{\uOmega}{\uASI{\uA}{\Psi}{\uI}}{\uPhi}{\usyntaxueP{\tsmv}{\urho}{\eparse}{\ue}}{e}{\tau}
% % }
% % \end{equation}

% % \begin{equation}\label{rule:expandsP-letpetsm}
% % \inferrule{
% %   \tsmexpExpandsExp{\uOmega}{\uASI{\uA}{\Psi}{\uI}}{\uepsilon}{\epsilon}{\rho}\\
% %   \expandsP{\uOmega}{\uASI{\uA\uplus\mapitem{\tsmv}{\epsilon}}{\Psi}{\uI}}{\uPhi}{\ue}{e}{\tau}
% % }{
% %   \expandsP{\uOmega}{\uASI{\uA}{\Psi}{\uI}}{\uPhi}{\uletpetsm{\tsmv}{\uepsilon}{\ue}}{e}{\tau}
% % }
% % \end{equation}

% % \todo{peTLM implicit designation}
% % Rule (\ref{rule:esyn-implicite}) governed ueTLM designations in synthetic position. The following rule governs ueTLM designations in analytic position.
% % \begin{equation}\label{rule:expandsP-implicite}
% %   \inferrule{
% %     \eana{\uDelta}{\uGamma}{\uASI{\uA \uplus \vExpands{\tsmv}{x}}{\Psi, \xuetsmbnd{x}{\tau}{\eparse}}{\uI \uplus \designate{\tau}{a}}}{\uPhi}{\ue}{e}{\tau'}
% %   }{
% %     \eana{\uDelta}{\uGamma}{\uASI{\uA \uplus \vExpands{\tsmv}{x}}{\Psi, \xuetsmbnd{x}{\tau}{\eparse}}{\uI}}{\uPhi}{\implicite{\tsmv}{\ue}}{e}{\tau'}
% %   }
% % \end{equation}

% % \todo{peTLM implicit application}
% % % An expression of unadorned literal form can appear only in analytic position. The following rule extracts the TLM designated at the type that the expression is being analyzed against from the TLM designation context in the ueTLM context and applies it implicitly, i.e. the premises correspond to those of Rule (\ref{rule:esyn-apuetsm}).


% % Rule (\ref{rule:esyn-defuptsm}) governed upTLM definitions in synthetic position. The following rule governs upTLM definitions in analytic position.
% % \begin{equation}\label{rule:expandsP-syntaxup}
% % \inferrule{
% %   \tsmtyExpands{\uOmega}{\urho}{\rho}\\
% %   \hastypeP{\emptyset}{\eparse}{\aparr{\tBody}{\tParseResultCEPat}}\\\\
% %   \expandsP{\uOmega}{\uPsi}{\uASI{\uA \uplus \mapitem{\tsmv}{\adefref{a}}}{\Phi, \pptsmdefn{a}{\rho}{\eparse}}{\uI}}{\ue}{e}{\tau}
% % }{
% %   \expandsP{\uOmega}{\uPsi}{\uASI{\uA}{\Phi}{\uI}}{\usyntaxup{\tsmv}{\urho}{\eparse}{\ue}}{e}{\tau}
% % }
% % \end{equation}


% % \begin{equation}\label{rule:expandsP-letpptsm}
% % \inferrule{
% %   \tsmexpExpandsPat{\uOmega}{\uASI{\uA}{\Phi}{\uI}}{\uepsilon}{\epsilon}{\rho}\\
% %   \expandsP{\uOmega}{\uPsi}{\uASI{\uA\uplus\mapitem{\tsmv}{\epsilon}}{\Phi}{\uI}}{\ue}{e}{\tau}
% % }{
% %   \expandsP{\uOmega}{\uPsi}{\uASI{\uA}{\Phi}{\uI}}{\uletpptsm{\tsmv}{\uepsilon}{\ue}}{e}{\tau}
% % }
% % \end{equation}


% % \todo{ppTLM implicit designation}
% % % Rule (\ref{rule:esyn-implicitp}) governed upTLM designations in synthetic position. The following rule governs upTLM designations in analytic position.
% % \begin{equation}\label{rule:expandsP-implicitp}
% %   \inferrule{
% %     \eana{\uDelta}{\uGamma}{\uPsi}{\uASI{\uA\uplus\vExpands{\tsmv}{x}}{\Phi, \xuptsmbnd{a}{\tau}{\eparse}}{\uI \uplus \designate{\tau}{a}}}{\ue}{e}{\tau'}
% %   }{
% %     \eana{\uDelta}{\uGamma}{\uPsi}{\uASI{\uA\uplus\vExpands{\tsmv}{x}}{\Phi, \xuetsmbnd{x}{\tau}{\eparse}}{\uI}}{\implicitp{\tsmv}{\ue}}{e}{\tau'}
% %   }
% % \end{equation}

% % \end{subequations}

% \noindent\fbox{$\strut\rExpandsSP{\uOmega}{\uPsi}{\uPhi}{\urv}{r}{\tau}{\tau'}$}~~$\urv$ has expansion $r$ taking values of type $\tau$ to values of type $\tau'$
% \begin{equation}\label{rule:rExpandsP}
%   \inferrule{
%     \uOmega=\uOmegaEx{\uD}{\uG}{\uMctx}{\Omega}\\
%     \patExpandsP{\uOmegaEx{\emptyset}{\uG'}{\emptyset}{\Omega'}}{\uPhi}{\upv}{p}{\tau}\\
%     \expandsP{\uOmegaEx{\uD}{\uG \uplus \uG'}{\uMctx}{\Omega \cup \Omega'}}{\uPsi}{\uPhi}{\ue}{e}{\tau'}
%   }{
%     \rExpandsSP{\uOmega}{\uPsi}{\uPhi}{\matchrule{\upv}{\ue}}{\aematchrule{p}{e}}{\tau}{\tau'}
%   }
% \end{equation}


% \noindent\fbox{$\strut\patExpandsP{\uOmega'}{\uPhi}{\upv}{p}{\tau}$}~~$\upv$ has expansion $p$ matching against $\tau$ generating hypotheses $\uOmega'$
% % The typed pattern expansion judgement is inductively defined by Rules (\ref{rules:patExpandsP}) as follows. %As in $\miniVersePat$, \emph{unexpanded pattern typing contexts}, $\upctx$, are defined identically to unexpanded typing contexts (i.e. we only use a distinct metavariable to emphasize their distinct roles in the judgements above). 

% % The following rules are written identically to the typed pattern expansion rules for shared pattern forms in $\miniVersePat$, i.e. Rules (\ref{rule:patExpands-var}) through (\ref{rule:patExpands-in}).
% \begin{subequations}\label{rules:patExpandsP}
% \begin{equation}\label{rule:patExpandsP-subsume}
% \inferrule{
%   \uOmega=\uOmegaEx{\uD}{\uG}{\uMctx}{\Omega}\\\\
%   \patExpandsP{\uOmega'}{\uPhi}{\upv}{p}{\tau}\\
%   \issubtypeP{\Omega}{\tau}{\tau'}
% }{
%   \patExpandsP{\uOmega'}{\uPhi}{\upv}{p}{\tau'}
% }
% \end{equation}
% \begin{equation}\label{rule:patExpandsP-var}
% \inferrule{ }{
%   \patExpandsP{\uOmegaEx{\emptyset}{\vExpands{\ux}{x}}{\emptyset}{\Ghyp{x}{\tau}}}{\uPhi}{\ux}{x}{\tau}
% }
% \end{equation}
% \begin{equation}\label{rule:patExpandsP-wild}
% \inferrule{ }{
%   \patExpandsP{\uOmegaEx{\emptyset}{\emptyset}{\emptyset}{\emptyset}}{\uPhi}{\wildp}{\aewildp}{\tau}
% }
% \end{equation}
% \begin{equation}\label{rule:patExpandsP-fold}
% \inferrule{ 
%   \patExpandsP{\uOmega'}{\uPhi}{\upv}{p}{[\arec{t}{\tau}/t]\tau}
% }{
%   \patExpandsP{\uOmega'}{\uPhi}{\foldp{\upv}}{\aefoldp{p}}{\arec{t}{\tau}}
% }
% \end{equation}
% \begin{equation}\label{rule:patExpandsP-tpl}
% \inferrule{
%   \tau=\aprod{\labelset}{\mapschema{\tau}{i}{\labelset}}\\\\
%   \{\patExpandsP{{\uOmega_i}}{\uPhi}{\upv_i}{p_i}{\tau_i}\}_{i \in \labelset}
% }{
%   %\patExpandsP{\Gconsi{i \in \labelset}{\upctx_i}}{A}{B}{C}
%   \patExpandsP{\Gconsi{i \in \labelset}{\uOmega_i}}{\uPhi}{\tplp{\mapschema{\upv}{i}{\labelset}}}{\aetplp{\labelset}{\mapschema{p}{i}{\labelset}}}{\tau}
%   % \patExpands{\Gconsi{i \in \labelset}{\pctx_i}}{\Phi}{
%   %   \autplp{\labelset}{\mapschema{\upv}{i}{\labelset}}
%   % }{
%   %   \aetplp{\labelset}{\mapschema{p}{i}{\labelset}}
%   % }{
%   %   \aprod{\labelset}{\mapschema{\tau}{i}{\labelset}}
%   % } %{\autplp{\labelset}{\mapschema{\upv}{i}{\labelset}}}{\aetplp{\labelset}{\mapschema}{p}{i}{\labelset}}{...}
%   %\left(\shortstack{$\Delta \vdash_{\uPhi} \autplp{\labelset}{\mapschema{\upv}{i}{\labelset}}$\\$\leadsto$\\$\aetplp{\labelset}{\mapschema{p}{i}{\labelset}} : \aprod{\labelset}{\mapschema{\tau}{i}{\labelset}} \dashV \Gconsi{i \in \labelset}{\upctx_i}$\vspace{-1.2em}}\right)
% }
% \end{equation}
% \begin{equation}\label{rule:patExpandsP-in}
% \inferrule{
%   \patExpandsP{\uOmega'}{\uPhi}{\upv}{p}{\tau}
% }{
%   \patExpandsP{\uOmega'}{\uPhi}{\injp{\ell}{\upv}}{\aeinjp{\ell}{p}}{\asum{\labelset, \ell}{\mapschema{\tau}{i}{\labelset}; \mapitem{\ell}{\tau}}}
% }
% \end{equation}

% \begin{equation}\label{rule:patExpandsP-apuptsm}
% \inferrule{
%   \uOmega=\uOmegaEx{\uD}{\uG}{\uMctx}{\Omega_\text{app}}\\
%   \uPhi=\uAS{\uA}{\Phi}\\\\
%   \tsmexpExpandsPat{\uOmega}{\uPhi}{\uepsilon}{\epsilon}{\aetype{\tau_\text{final}}}\\
%   \tsmexpEvalsPat{\Omega_\text{app}}{\Phi}{\epsilon}{\epsilon_\text{normal}}\\\\
%   \tsmdefof{\epsilon_\text{normal}}=a\\
%   \Phi = \Phi', \pptsmdefn{a}{\rho}{\eparse}\\\\
%   \encodeBody{b}{\ebody}\\
%   \evalU{\ap{\eparse}{\ebody}}{\aein{\mathtt{SuccessP}}{e_\text{pproto}}}\\
%   \decodePCEPat{e_\text{pproto}}{\pcp}\\\\
%   \prepcp{\Omega_\text{app}}{\Phi}{\pcp}{\cpv}{\epsilon_\text{normal}}{\aetype{\tau_\text{proto}}}{\omega}{\Omega_\text{params}}\\\\
%      \segOKP{\OParams}{\csceneP{\omega : \OParams}{\uOmega}{b}}{\segof{\cpv}}{b}\\
%   \cvalidPP{\uOmega'}{\psceneP{\omega : \Omega_\text{params}}{\uOmega}{\uPhi}{b}}{\cpv}{p}{\tau_\text{proto}}
% }{
%   \patExpandsP{\uOmega'}{\uPhi}{\utsmap{\uepsilon}{b}}{p}{[\omega]\tau_\text{proto}}
% }
% \end{equation}
% \end{subequations}

% \subsubsection{TLM Type and Expression Expansion}
% \noindent\fbox{$\strut\tsmtyExpands{\uOmega}{\urho}{\rho}$}~~$\urho$ has well-formed expansion $\rho$
% \begin{subequations}\label{rules:tsmtyExpands}
% \begin{equation}\label{rule:tsmtyExpands-type}
% \inferrule{
%   \cExpandsX{\utau}{\tau}{\akty}
% }{
%   \tsmtyExpands{\uOmega}{{\utau}}{\aetype{\tau}}
% }
% \end{equation}
% % \begin{equation}\label{rule:tsmtyExpands-alltypes}
% % \inferrule{
% %   \tsmtyExpands{\uOmega, \uKhyp{\ut}{t}{\akty}}{\urho}{\rho}
% % }{
% %   \tsmtyExpands{\uOmega}{\alltypes{\ut}{\urho}}{\aealltypes{t}{\rho}}
% % }
% % \end{equation}
% \begin{equation}\label{rule:tsmtyExpands-allmods}
% \inferrule{
%   \sigExpandsPX{\usigma}{\sigma}\\
%   \tsmtyExpands{\uOmega, \uMhyp{\uX}{X}{\sigma}}{\urho}{\rho}
% }{
%   \tsmtyExpands{\uOmega}{\allmods{\uX}{\usigma}{\urho}}{\aeallmods{\sigma}{X}{\rho}}
% }
% \end{equation}
% \end{subequations}

% \noindent\fbox{$\strut\tsmexpExpandsExp{\uOmega}{\uPsi}{\uepsilon}{\epsilon}{\rho}$}~~$\uepsilon$ has peTLM expression expansion $\epsilon$ at $\rho$
% \begin{subequations}\label{rules:tsmexpExpandsExp}
% \begin{equation}\label{rule:tsmexpExpandsExp-bindref}
% \inferrule{
%   \hastsmtypeExp{\Omega}{\Psi}{\epsilon}{\rho}  
% }{
%   \tsmexpExpandsExp{\uOmegaEx{\uD}{\uG}{\uMctx}{\Omega}}{\uAS{\uA, \mapitem{\tsmv}{\epsilon}}{\Psi}}{{\tsmv}}{\epsilon}{\rho}
% }
% \end{equation}
% % \begin{equation}\label{rule:tsmexpExpandsExp-abstype}
% % \inferrule{
% %   \tsmexpExpandsExp{\uOmega, \uKhyp{\ut}{t}{\akty}}{\uPsi}{\uepsilon}{\epsilon}{\rho}
% % }{
% %   \tsmexpExpandsExp{\uOmega}{\uPsi}{\abstype{\ut}{\uepsilon}}{\aeabstype{t}{\epsilon}}{\aealltypes{t}{\rho}}
% % }
% % \end{equation}
% \begin{equation}\label{rule:tsmexpExpandsExp-absmod}
% \inferrule{
%   \sigExpandsPX{\usigma}{\sigma}\\
%   \tsmexpExpandsExp{\uOmega, \uMhyp{\uX}{X}{\sigma}}{\uPsi}{\uepsilon}{\epsilon}{\rho}
% }{
%   \tsmexpExpandsExp{\uOmega}{\uPsi}{\absmod{\uX}{\usigma}{\uepsilon}}{\aeabsmod{\sigma}{X}{\epsilon}}{\aeallmods{\sigma}{X}{\rho}}
% }
% \end{equation}
% % \begin{equation}\label{rule:tsmexpExpandsExp-aptype}
% % \inferrule{
% %   \tsmexpExpandsExp{\uOmega}{\uPsi}{\uepsilon}{\epsilon}{\aealltypes{t}{\rho}}\\
% %   \cExpandsX{\utau}{\tau}{\akty}
% % }{
% %   \tsmexpExpandsExp{\uOmega}{\uPsi}{\aptype{\uepsilon}{\utau}}{\aeaptype{\tau}{\epsilon}}{[\tau/t]\rho} 
% % }
% % \end{equation}
% \begin{equation}\label{rule:tsmexpExpandsExp-apmod}
% \inferrule{
%   \tsmexpExpandsExp{\uOmega}{\uPsi}{\uepsilon}{\epsilon}{\aeallmods{\sigma}{X'}{\rho}}\\
%   \mExpandsPX{\uX}{X}{\sigma}
% }{
%   \tsmexpExpandsExp{\uOmega}{\uPsi}{\apmod{\uepsilon}{\uX}}{\aeapmod{X}{\epsilon}}{[X/X']\rho}
% }
% \end{equation}
% \end{subequations}

% \noindent\fbox{$\strut\tsmexpExpandsPat{\uOmega}{\uPsi}{\uepsilon}{\epsilon}{\rho}$}~~$\uepsilon$ has ppTLM expression expansion $\epsilon$ at $\rho$
% \begin{subequations}\label{rules:tsmexpExpandsPat}
% \begin{equation}\label{rule:tsmexpExpandsPat-bindref}
% \inferrule{
%   \hastsmtypePat{\Omega}{\Phi}{\epsilon}{\rho}  
% }{
%   \tsmexpExpandsPat{\uOmegaEx{\uD}{\uG}{\uMctx}{\Omega}}{\uAS{\uA, \mapitem{\tsmv}{\epsilon}}{\Phi}}{{\tsmv}}{\epsilon}{\rho}
% }
% \end{equation}
% % \begin{equation}\label{rule:tsmexpExpandsPat-abstype}
% % \inferrule{
% %   \tsmexpExpandsPat{\uOmega, \uKhyp{\ut}{t}{\akty}}{\uPhi}{\uepsilon}{\epsilon}{\rho}
% % }{
% %   \tsmexpExpandsPat{\uOmega}{\uPhi}{\abstype{\ut}{\uepsilon}}{\aeabstype{t}{\epsilon}}{\aealltypes{t}{\rho}}
% % }
% % \end{equation}
% \begin{equation}\label{rule:tsmexpExpandsPat-absmod}
% \inferrule{
%   \sigExpandsPX{\usigma}{\sigma}\\
%   \tsmexpExpandsPat{\uOmega, \uMhyp{\uX}{X}{\sigma}}{\uPhi}{\uepsilon}{\epsilon}{\rho}
% }{
%   \tsmexpExpandsPat{\uOmega}{\uPhi}{\absmod{\uX}{\usigma}{\uepsilon}}{\aeabsmod{\sigma}{X}{\epsilon}}{\aeallmods{\sigma}{X}{\rho}}
% }
% \end{equation}
% % \begin{equation}\label{rule:tsmexpExpandsPat-aptype}
% % \inferrule{
% %   \tsmexpExpandsPat{\uOmega}{\uPhi}{\uepsilon}{\epsilon}{\aealltypes{t}{\rho}}\\
% %   \cExpandsX{\utau}{\tau}{\akty}
% % }{
% %   \tsmexpExpandsPat{\uOmega}{\uPhi}{\aptype{\uepsilon}{\utau}}{\aeaptype{\tau}{\epsilon}}{[\tau/t]\rho} 
% % }
% % \end{equation}
% \begin{equation}\label{rule:tsmexpExpandsPat-apmod}
% \inferrule{
%   \tsmexpExpandsPat{\uOmega}{\uPhi}{\uepsilon}{\epsilon}{\aeallmods{\sigma}{X'}{\rho}}\\
%   \mExpandsPX{\uX}{X}{\sigma}
% }{
%   \tsmexpExpandsPat{\uOmega}{\uPhi}{\apmod{\uepsilon}{\uX}}{\aeapmod{X}{\epsilon}}{[X/X']\rho}
% }
% \end{equation}
% \end{subequations}

% \subsubsection{Statics of the TLM Language}
% \noindent\fbox{$\strut\istsmty{\Omega}{\rho}$}~~$\rho$ is a TLM type
% \begin{subequations}\label{rules:istsmty}
% \begin{equation}\label{rule:istsmty-type}
% \inferrule{
%   \haskindX{\tau}{\akty}
% }{
%   \istsmty{\Omega}{\aetype{\tau}}
% }
% \end{equation}
% % \begin{equation}\label{rule:istsmty-alltypes}
% % \inferrule{
% %   \istsmty{\Omega, t :: \akty}{\rho}
% % }{
% %   \istsmty{\Omega}{\aealltypes{t}{\rho}}
% % }
% % \end{equation}
% \begin{equation}\label{rule:istsmty-allmods}
% \inferrule{
%   \issig{\Omega}{\sigma}\\
%   \istsmty{\Omega, X : \sigma}{\rho}
% }{
%   \istsmty{\Omega}{\aeallmods{\sigma}{X}{\rho}}
% }
% \end{equation}
% \end{subequations}

% \noindent\fbox{$\strut\hastsmtypeExp{\Omega}{\Psi}{\epsilon}{\rho}$}~~$\epsilon$ is a peTLM expression at $\rho$
% \begin{subequations}\label{rules:hastsmtypeExp}
% \begin{equation}\label{rule:hastsmtypeExp-defref}
% \inferrule{
%   \istsmty{\Omega}{\rho}
% }{
%   \hastsmtypeExp{\Omega}{\Psi, \petsmdefn{a}{\rho}{\eparse}}{\adefref{a}}{\rho}
% }
% \end{equation}
% % \begin{equation}\label{rule:hastsmtypeExp-abstype}
% % \inferrule{
% %   \hastsmtypeExp{\Omega, t :: \akty}{\Psi}{\epsilon}{\rho}
% % }{
% %   \hastsmtypeExp{\Omega}{\Psi}{\aeabstype{t}{\epsilon}}{\aealltypes{t}{\rho}}
% % }
% % \end{equation}
% \begin{equation}\label{rule:hastsmtypeExp-absmod}
% \inferrule{
%   \issigX{\sigma}\\
%   \hastsmtypeExp{\Omega, X : \sigma}{\Psi}{\epsilon}{\rho}
% }{
%   \hastsmtypeExp{\Omega}{\Psi}{\aeabsmod{\sigma}{X}{\epsilon}}{\aeallmods{\sigma}{X}{\rho}}
% }
% \end{equation}
% % \begin{equation}\label{rule:hastsmtypeExp-aptype}
% % \inferrule{
% %   \hastsmtypeExp{\Omega}{\Psi}{\epsilon}{\aealltypes{t}{\rho}}\\
% %   \haskindX{\tau}{\akty}
% % }{
% %   \hastsmtypeExp{\Omega}{\Psi}{\aeaptype{\tau}{\epsilon}}{[\tau/t]\rho}
% % }
% % \end{equation}
% \begin{equation}\label{rule:hastsmtypeExp-apmod}
% \inferrule{
%   \hastsmtypeExp{\Omega}{\Psi}{\epsilon}{\aeallmods{\sigma}{X'}{\rho}}\\
%   \hassig{\Omega}{X}{\sigma}
% }{
%   \hastsmtypeExp{\Omega}{\Psi}{\aeapmod{X}{\epsilon}}{[X/X']\rho}
% }
% \end{equation}
% \end{subequations}

% \noindent\fbox{$\strut\hastsmtypePat{\Omega}{\Phi}{\epsilon}{\rho}$}~~$\epsilon$ is a ppTLM expression at $\rho$
% \begin{subequations}\label{rules:hastsmtypePat}
% \begin{equation}\label{rule:hastsmtypePat-defref}
% \inferrule{
%   \istsmty{\Omega}{\rho}
% }{
%   \hastsmtypePat{\Omega}{\Phi, \pptsmdefn{a}{\rho}{\eparse}}{\adefref{a}}{\rho}
% }
% \end{equation}
% % \begin{equation}\label{rule:hastsmtypePat-abstype}
% % \inferrule{
% %   \hastsmtypePat{\Omega, t :: \akty}{\Phi}{\epsilon}{\rho}
% % }{
% %   \hastsmtypePat{\Omega}{\Phi}{\aeabstype{t}{\epsilon}}{\aealltypes{t}{\rho}}
% % }
% % \end{equation}
% \begin{equation}\label{rule:hastsmtypePat-absmod}
% \inferrule{
%   \issigX{\sigma}\\
%   \hastsmtypePat{\Omega, X : \sigma}{\Phi}{\epsilon}{\rho}
% }{
%   \hastsmtypePat{\Omega}{\Phi}{\aeabsmod{\sigma}{X}{\epsilon}}{\aeallmods{\sigma}{X}{\rho}}
% }
% \end{equation}
% % \begin{equation}\label{rule:hastsmtypePat-aptype}
% % \inferrule{
% %   \hastsmtypePat{\Omega}{\Phi}{\epsilon}{\aealltypes{t}{\rho}}\\
% %   \haskindX{\tau}{\akty}
% % }{
% %   \hastsmtypePat{\Omega}{\Phi}{\aeaptype{\tau}{\epsilon}}{[\tau/t]\rho}
% % }
% % \end{equation}
% \begin{equation}\label{rule:hastsmtypePat-apmod}
% \inferrule{
%   \hastsmtypePat{\Omega}{\Phi}{\epsilon}{\aeallmods{\sigma}{X'}{\rho}}\\
%   \hassig{\Omega}{X}{\sigma}
% }{
%   \hastsmtypePat{\Omega}{\Phi}{\aeapmod{X}{\epsilon}}{[X/X']\rho}
% }
% \end{equation}

% \end{subequations}

% The following metafunction extracts the TLM name from a TLM expression.
% \begin{subequations}
% \begin{align}
% \tsmdefof{\adefref{a}} & = a \label{eqn:tsmdefof-adefref}\\
% % \tsmdefof{\aeabstype{t}{\epsilon}} & = \tsmdefof{\epsilon} \label{eqn:tsmdefof-abstype}\\
% \tsmdefof{\aeabsmod{\sigma}{X}{\epsilon}} & = \tsmdefof{\epsilon} \label{eqn:tsmdefof-absmod}\\
% % \tsmdefof{\aeaptype{\tau}{\epsilon}} & = \tsmdefof{\epsilon} \label{eqn:tsmdefof-aptype}\\
% \tsmdefof{\aeapmod{X}{\epsilon}} & = \tsmdefof{\epsilon} \label{eqn:tsmdefof-apmod}
% \end{align}
% \end{subequations}

% \subsubsection{Dynamics of the TLM Language}

% \noindent\fbox{$\strut\tsmexpStepsExp{\Omega}{\Psi}{\epsilon}{\epsilon'}$}~~peTLM expression $\epsilon$ transitions to $\epsilon'$
% \begin{subequations}\label{rules:tsmexpStepsExp}
% % \begin{equation}\label{rule:tsmexpStepsExp-aptype-1}
% % \inferrule{
% %   \tsmexpStepsExp{\Omega}{\Psi}{\epsilon}{\epsilon'}
% % }{
% %   \tsmexpStepsExp{\Omega}{\Psi}{\aeaptype{\tau}{\epsilon}}{\aeaptype{\tau}{\epsilon'}}
% % }
% % \end{equation}
% % \begin{equation}\label{rule:tsmexpStepsExp-aptype-2}
% % \inferrule{ }{
% %   \tsmexpStepsExp{\Omega}{\Psi}{\aeaptype{\tau}{\aeabstype{t}{\epsilon}}}{[\tau/t]\epsilon}
% % }
% % \end{equation}
% \begin{equation}\label{rule:tsmexpStepsExp-apmod-1}
% \inferrule{
%   \tsmexpStepsExp{\Omega}{\Psi}{\epsilon}{\epsilon'}
% }{
%   \tsmexpStepsExp{\Omega}{\Psi}{\aeapmod{X}{\epsilon}}{\aeapmod{X}{\epsilon'}}
% }
% \end{equation}
% \begin{equation}\label{rule:tsmexpStepsExp-apmod-2}
% \inferrule{ }{
%   \tsmexpStepsExp{\Omega}{\Psi}{\aeapmod{X}{\aeabsmod{\sigma}{X'}{\epsilon}}}{[X/X']\epsilon}
% }
% \end{equation}
% \end{subequations}

% \noindent\fbox{$\strut\tsmexpStepsPat{\Omega}{\Psi}{\epsilon}{\epsilon'}$}~~ppTLM expression $\epsilon$ transitions to $\epsilon'$
% \begin{subequations}\label{rules:tsmexpStepsPat}
% % \begin{equation}\label{rule:tsmexpStepsPat-aptype-1}
% % \inferrule{
% %   \tsmexpStepsPat{\Omega}{\Psi}{\epsilon}{\epsilon'}
% % }{
% %   \tsmexpStepsPat{\Omega}{\Psi}{\aeaptype{\tau}{\epsilon}}{\aeaptype{\tau}{\epsilon'}}
% % }
% % \end{equation}
% % \begin{equation}\label{rule:tsmexpStepsPat-aptype-2}
% % \inferrule{ }{
% %   \tsmexpStepsPat{\Omega}{\Psi}{\aeaptype{\tau}{\aeabstype{t}{\epsilon}}}{[\tau/t]\epsilon}
% % }
% % \end{equation}
% \begin{equation}\label{rule:tsmexpStepsPat-apmod-1}
% \inferrule{
%   \tsmexpStepsPat{\Omega}{\Psi}{\epsilon}{\epsilon'}
% }{
%   \tsmexpStepsPat{\Omega}{\Psi}{\aeapmod{X}{\epsilon}}{\aeapmod{X}{\epsilon'}}
% }
% \end{equation}
% \begin{equation}\label{rule:tsmexpStepsPat-apmod-2}
% \inferrule{ }{
%   \tsmexpStepsPat{\Omega}{\Psi}{\aeapmod{X}{\aeabsmod{\sigma}{X'}{\epsilon}}}{[X/X']\epsilon}
% }
% \end{equation}
% \end{subequations}

% \noindent\fbox{$\strut\tsmexpMultistepsExp{\Omega}{\Psi}{\epsilon}{\epsilon'}$}~~peTLM expression $\epsilon$ transitions in multiple steps to $\epsilon'$
% \begin{subequations}\label{rules:tsmexpMultistepsExp}
% \begin{equation}\label{rule:tsmexpMultistepsExp-refl}
% \inferrule{ }{
%   \tsmexpMultistepsExp{\Omega}{\Psi}{\epsilon}{\epsilon}
% }
% \end{equation}
% \begin{equation}\label{rule:tsmexpMultistepsExp-steps}
% \inferrule{
%   \tsmexpStepsExp{\Omega}{\Psi}{\epsilon}{\epsilon'}
% }{
%   \tsmexpMultistepsExp{\Omega}{\Psi}{\epsilon}{\epsilon'}
% }
% \end{equation}
% \begin{equation}\label{rule:tsmexpMultistepsExp-trans}
% \inferrule{
%   \tsmexpMultistepsExp{\Omega}{\Psi}{\epsilon}{\epsilon'}\\
%   \tsmexpMultistepsExp{\Omega}{\Psi}{\epsilon'}{\epsilon''}
% }{
%   \tsmexpMultistepsExp{\Omega}{\Psi}{\epsilon}{\epsilon''}
% }
% \end{equation}
% \end{subequations}

% \noindent\fbox{$\strut\tsmexpMultistepsPat{\Omega}{\Psi}{\epsilon}{\epsilon'}$}~~ppTLM expression $\epsilon$ transitions in multiple steps to $\epsilon'$
% \begin{subequations}\label{rules:tsmexpMultistepsPat}
% \begin{equation}\label{rule:tsmexpMultistepsPat-refl}
% \inferrule{ }{
%   \tsmexpMultistepsPat{\Omega}{\Psi}{\epsilon}{\epsilon}
% }
% \end{equation}
% \begin{equation}\label{rule:tsmexpMultistepsPat-steps}
% \inferrule{
%   \tsmexpStepsExp{\Omega}{\Psi}{\epsilon}{\epsilon'}
% }{
%   \tsmexpMultistepsPat{\Omega}{\Psi}{\epsilon}{\epsilon'}
% }
% \end{equation}
% \begin{equation}\label{rule:tsmexpMultistepsPat-trans}
% \inferrule{
%   \tsmexpMultistepsPat{\Omega}{\Psi}{\epsilon}{\epsilon'}\\
%   \tsmexpMultistepsPat{\Omega}{\Psi}{\epsilon'}{\epsilon''}
% }{
%   \tsmexpMultistepsPat{\Omega}{\Psi}{\epsilon}{\epsilon''}
% }
% \end{equation}
% \end{subequations}

% \noindent\fbox{$\strut\tsmexpEvalsExp{\Omega}{\Psi}{\epsilon}{\epsilon'}$}~~peTLM expression $\epsilon$ normalizes to $\epsilon'$
% \begin{equation}\label{rule:tsmexpEvalsExp}
% \inferrule{
%   \tsmexpMultistepsExp{\Omega}{\Psi}{\epsilon}{\epsilon'}\\
%   \tsmexpNormalExp{\Omega}{\Psi}{\epsilon'}
% }{
%   \tsmexpEvalsExp{\Omega}{\Psi}{\epsilon}{\epsilon'}
% }
% \end{equation}


% \noindent\fbox{$\strut\tsmexpEvalsPat{\Omega}{\Psi}{\epsilon}{\epsilon'}$}~~ppTLM expression $\epsilon$ normalizes to $\epsilon'$
% \begin{equation}\label{rule:tsmexpEvalsPat}
% \inferrule{
%   \tsmexpMultistepsExp{\Omega}{\Psi}{\epsilon}{\epsilon'}\\
%   \tsmexpNormalExp{\Omega}{\Psi}{\epsilon'}
% }{
%   \tsmexpEvalsPat{\Omega}{\Psi}{\epsilon}{\epsilon'}
% }
% \end{equation}


% \noindent\fbox{$\tsmexpNormalExp{\Omega}{\Psi}{\epsilon}$}~~$\epsilon$ is a normal peTLM expression
% \begin{subequations}\label{rules:tsmexpNormalExp}
% \begin{equation}\label{rule:tsmexpNormalExp-defref}
% \inferrule{ }{
%   \tsmexpNormalExp{\Omega}{\Psi, \petsmdefn{a}{\rho}{\eparse}}{\adefref{a}}
% }
% \end{equation}
% % \begin{equation}\label{rule:tsmexpNormalExp-abstype}
% % \inferrule{ }{
% %   \tsmexpNormalExp{\Omega}{\Psi}{\aeabstype{t}{\epsilon}}
% % }
% % \end{equation}
% \begin{equation}\label{rule:tsmexpNormalExp-absmod}
% \inferrule{ }{
%   \tsmexpNormalExp{\Omega}{\Psi}{\aeabsmod{\sigma}{X}{\epsilon}}
% }
% \end{equation}
% % \begin{equation}\label{rule:tsmexpNormalExp-aptype}
% % \inferrule{
% %   \epsilon \neq \aeabstype{t}{\epsilon'}\\
% %   \tsmexpNormalExp{\Omega}{\Psi}{\epsilon}
% % }{
% %   \tsmexpNormalExp{\Omega}{\Psi}{\aeaptype{\tau}{\epsilon}}
% % }
% % \end{equation}
% \begin{equation}\label{rule:tsmexpNormalExp-apmod}
% \inferrule{
%   \epsilon \neq \aeabsmod{\sigma}{X'}{\epsilon'}\\
%   \tsmexpNormalExp{\Omega}{\Psi}{\epsilon}
% }{
%   \tsmexpNormalExp{\Omega}{\Psi}{\aeapmod{X}{\epsilon}}
% }
% \end{equation}
% \end{subequations}

% \noindent\fbox{$\tsmexpNormalPat{\Omega}{\Psi}{\epsilon}$}~~$\epsilon$ is a normal ppTLM expression
% \begin{subequations}\label{rules:tsmexpNormalPat}
% \begin{equation}\label{rule:tsmexpNormalPat-defref}
% \inferrule{ }{
%   \tsmexpNormalPat{\Omega}{\Psi, \petsmdefn{a}{\rho}{\eparse}}{\adefref{a}}
% }
% \end{equation}
% % \begin{equation}\label{rule:tsmexpNormalPat-abstype}
% % \inferrule{ }{
% %   \tsmexpNormalPat{\Omega}{\Psi}{\aeabstype{t}{\epsilon}}
% % }
% % \end{equation}
% \begin{equation}\label{rule:tsmexpNormalPat-absmod}
% \inferrule{ }{
%   \tsmexpNormalPat{\Omega}{\Psi}{\aeabsmod{\sigma}{X}{\epsilon}}
% }
% \end{equation}
% % \begin{equation}\label{rule:tsmexpNormalPat-aptype}
% % \inferrule{
% %   \epsilon \neq \aeabstype{t}{\epsilon'}\\
% %   \tsmexpNormalPat{\Omega}{\Psi}{\epsilon}
% % }{
% %   \tsmexpNormalPat{\Omega}{\Psi}{\aeaptype{\tau}{\epsilon}}
% % }
% % \end{equation}
% \begin{equation}\label{rule:tsmexpNormalPat-apmod}
% \inferrule{
%   \epsilon \neq \aeabsmod{\sigma}{X'}{\epsilon'}\\
%   \tsmexpNormalPat{\Omega}{\Psi}{\epsilon}
% }{
%   \tsmexpNormalPat{\Omega}{\Psi}{\aeapmod{X}{\epsilon}}
% }
% \end{equation}
% \end{subequations}


% \section{Proto-Expansion Validation}\label{appendix:P-proto-expansion-validation}
% \subsection{Syntax of Proto-Expansions}
% \subsubsection{Syntax -- Parameterized Proto-Expressions}
% \[\begin{array}{lllllll}
% \textbf{Sort} & & & \textbf{Operational Form} & \textbf{Stylized Form} & \textbf{Description}\\
% % \LCC \color{Yellow}&\color{Yellow}&\color{Yellow}& \color{Yellow} & \color{Yellow} & \color{Yellow}\\
% \mathsf{PPrExpr} & \pce & ::= & \apceexp{\ce} & \pceexp{\ce} & \text{proto-expression}\\
% % &&& \apcebindtype{t}{\pce} & \pcebindtype{t}{\pce} & \text{type binding}\\
% &&& \apcebindmod{X}{\pce} & \pcebindmod{X}{\pce} & \text{module binding}%\ECC
% \end{array}\]

% \subsubsection{Syntax -- Parameterized Proto-Patterns}
% \[\begin{array}{lllllll}
% \textbf{Sort} & & & \textbf{Operational Form} & \textbf{Stylized Form} & \textbf{Description}\\
% % \LCC \color{Yellow}&\color{Yellow}&\color{Yellow}& \color{Yellow} & \color{Yellow} & \color{Yellow}\\
% \mathsf{PPrPat} & \pcp & ::= & \apcepat{\cpv} & {\cpv} & \text{proto-pattern}\\
% % &&& \apcebindtype{t}{\pcp} & \pcebindtype{t}{\pcp} & \text{type binding}\\
% &&& \apcebindmod{X}{\pcp} & \pcebindmod{X}{\pcp} & \text{module binding}%\ECC
% \end{array}\]

% \subsubsection{Syntax -- Proto-Kinds and Proto-Constructions}
% \[\begin{array}{lrlllll}
% \textbf{Sort} & & & \textbf{Operational Form} & \textbf{Stylized Form} & \textbf{Description}\\
% \mathsf{PrKind} & \cekappa & ::= & \acekdarr{\cekappa}{u}{\cekappa} & \kdarr{u}{\cekappa}{\cekappa} & \text{dependent function}\\
% &&& \acekunit & \kunit & \text{nullary product}\\
% &&& \acekdbprod{\cekappa}{u}{\cekappa} & \kdbprod{u}{\cekappa}{\cekappa} & \text{dependent product}\\
% %&&& \akdprodstd & \kdprodstd & \text{labeled dependent product}\\
% &&& \acekty & \kty & \text{type}\\
% &&& \aceksing{\ctau} & \ksing{\ctau} & \text{singleton}\\
% % \LCC &&& \color{Yellow} & \color{Yellow} & \color{Yellow}\\
% &&& \acesplicedk{m}{n} & \splicedk{m}{n} & \text{spliced kind}\\%\ECC\\
% \mathsf{PrCon} & \cec, \ctau & ::= & u & u & \text{construction variable}\\
% &&& t & t & \text{type variable}\\
% % &&& \acecasc{\cekappa}{\cec} & \casc{\cec}{\cekappa} & \text{ascription}\\
% &&& \acecabs{u}{\cec} & \cabs{u}{\cec} & \text{abstraction}\\
% &&& \acecapp{\cec}{\cec} & \capp{\cec}{\cec} & \text{application}\\
% &&& \acectriv & \ctriv & \text{trivial}\\
% &&& \acecpair{\cec}{\cec} & \cpair{\cec}{\cec} & \text{pair}\\
% &&& \acecprl{\cec} & \cprl{\cec} & \text{left projection}\\
% &&& \acecprr{\cec} & \cprr{\cec} & \text{right projection}\\
% %&&& \adtplX & \dtplX & \text{labeled dependent tuple}\\
% %&&& \adprj{\ell}{c} & \prj{c}{\ell} & \text{projection}\\
% &&& \aceparr{\ctau}{\ctau} & \parr{\ctau}{\ctau} & \text{partial function}\\
% &&& \aceallu{\cekappa}{u}{\ctau} & \forallu{u}{\cekappa}{\ctau} & \text{polymorphic}\\
% &&& \acerec{t}{\ctau} & \rect{t}{\ctau} & \text{recursive}\\
% &&& \aceprod{\labelset}{\mapschema{\ctau}{i}{\labelset}} & \prodt{\mapschema{\ctau}{i}{\labelset}} & \text{labeled product}\\
% &&& \acesum{\labelset}{\mapschema{\ctau}{i}{\labelset}} & \sumt{\mapschema{\ctau}{i}{\labelset}} & \text{labeled sum}\\
% &&& \acemcon{X} & \mcon{X} & \text{construction component}\\
% % \LCC &&& \color{Yellow} & \color{Yellow} & \color{Yellow}\\
% &&& \acesplicedc{m}{n}{\cekappa} & \splicedc{m}{n}{\cekappa} & \text{spliced construction}%\ECC
% \end{array}\]

% \subsubsection{Syntax -- Proto-Expressions and Proto-Rules}
% \[\arraycolsep=4pt\begin{array}{lllllll}
% \textbf{Sort} & & & \textbf{Operational Form} & \textbf{Stylized Form} & \textbf{Description}\\
% \mathsf{PrExp} & \ce & ::= & x & x & \text{variable}\\
% &&& \aceasc{\ctau}{\ce} & \asc{\ce}{\ctau} & \text{ascription}\\
% &&& \aceletsyn{x}{\ce}{\ce} & \letsyn{x}{\ce}{\ce} & \text{value binding}\\
% % &&& \aceanalam{x}{\ce} & \analam{x}{\ce} & \text{abstraction (unannotated)}\\
% &&& \acelam{\ctau}{x}{\ce} & \lam{x}{\ctau}{\ce} & \text{abstraction}\\
% &&& \aceap{\ce}{\ce} & \ap{\ce}{\ce} & \text{application}\\
% &&& \aceclam{\cekappa}{u}{\ce} & \clam{u}{\cekappa}{\ce} & \text{construction abstraction}\\
% &&& \acecap{\ce}{\cec} & \cAp{\ce}{\cec} & \text{construction application}\\
% &&& \acefold{\ce} & \fold{\ce} & \text{fold}\\
% &&& \aceunfold{\ce} & \unfold{\ce} & \text{unfold}\\
% &&& \acetpl{\labelset}{\mapschema{\ce}{i}{\labelset}} & \tpl{\mapschema{\ce}{i}{\labelset}} & \text{labeled tuple}\\
% &&& \acepr{\ell}{\ce} & \prj{\ce}{\ell} & \text{projection}\\
% &&& \aceanain{\ell}{\ce} & \inj{\ell}{\ce} & \text{injection}\\
% &&& \acematchwith{n}{\ce}{\seqschemaX{\crv}} & \matchwith{\ce}{\seqschemaX{\crv}} & \text{match}\\
% &&& \acemval{X} & \mval{X} & \text{value component}\\
% % \LCC &&& \color{Yellow} & \color{Yellow} & \color{Yellow}\\
% &&& \acesplicede{m}{n}{\ctau} & \splicede{m}{n}{\ctau} & \text{spliced expression}\\%\ECC\\
% \mathsf{PrRule} & \crv & ::= & \acematchrule{p}{\ce} & \matchrule{p}{\ce} & \text{rule}\end{array}\]

% \subsubsection{Syntax -- Proto-Patterns}
% \[\begin{array}{lllllll}
% \mathsf{PrPat} & \cpv & ::= & \acewildp & \wildp & \text{wildcard pattern}\\
% &&& \acefoldp{\cpv} & \foldp{\cpv} & \text{fold pattern}\\
% &&& \acetplp{\labelset}{\mapschema{\cpv}{i}{\labelset}} & \tplp{\mapschema{\cpv}{i}{\labelset}} & \text{labeled tuple pattern}\\
% &&& \aceinjp{\ell}{\cpv} & \injp{\ell}{\cpv} & \text{injection pattern}\\
% % \LCC &&& \color{Yellow} & \color{Yellow} & \color{Yellow}\\
% &&& \acesplicedp{m}{n}{\ctau} & \splicedp{m}{n}{\ctau} & \text{spliced pattern} %\ECC
% \end{array}\]

% \subsubsection{Common Proto-Expansion Terms}
% Each expanded term, with a few exceptions noted below, maps onto a proto-expansion term. We refer to these as the \emph{common proto-expansion terms}. In particular:
% \begin{itemize}
%   \item Each kind, $\kappa$, maps onto a proto-kind, $\Cof{\kappa}$, as follows:
%   \[\arraycolsep=1pt\begin{array}{rl}
%   \Cof{\akdarr{\kappa_1}{u}{\kappa_2}} & = \acekdarr{\Cof{\kappa_1}}{u}{\Cof{\kappa_2}}\\
%   \Cof{\akunit} & = \acekunit\\
%   \Cof{\akdbprod{\kappa_1}{u}{\kappa_2}} & = \acekdbprod{\Cof{\kappa_1}}{u}{\Cof{\kappa_2}}\\
%   \Cof{\akty} & = \acekty\\
%   \Cof{\aksing{\tau}} & = \aceksing{\Cof{\tau}}
%   \end{array}\]
%   \item Each construction, $c$, maps onto a proto-construction, $\Cof{c}$, as follows:
%   \[\arraycolsep=1pt\begin{array}{rl}
%   \Cof{u} & = u\\
%   \Cof{\acabs{u}{c}} & = \acecabs{u}{\Cof{c}}\\
%   \Cof{\acapp{c_1}{c_2}} & = \acecapp{\Cof{c_1}}{\Cof{c_2}}\\
%   \Cof{\actriv} & = \acectriv\\
%   \Cof{\acpair{c_1}{c_2}} & = \acecpair{\Cof{c_1}}{\Cof{c_2}}\\
%   \Cof{\acprl{c}} & = \acecprl{\Cof{c}}\\
%   \Cof{\acprr{c}} & = \acecprr{\Cof{c}}\\
%   \Cof{\aparr{\tau_1}{\tau_2}} & = \aceparr{\Cof{\tau_1}}{\Cof{\tau_2}}\\
%   \Cof{\aall{t}{\tau}} & = \aceall{t}{\Cof{\tau}}\\
%   \Cof{\arec{t}{\tau}} & = \acerec{t}{\Cof{\tau}}\\
%   \Cof{\aprod{\labelset}{\mapschema{\tau}{i}{\labelset}}} & = \aceprod{\labelset}{\mapschemax{\Cofv}{\tau}{i}{\labelset}}\\
%   \Cof{\asum{\labelset}{\mapschema{\tau}{i}{\labelset}}} & = \acesum{\labelset}{\mapschemax{\Cofv}{\tau}{i}{\labelset}}\\
%   \Cof{\amcon{X}} & = \acemcon{X}
%   \end{array}\]
%   \item Each expanded expression, $e$, except for the value projection of a module expression that is not of module variable form, maps onto a proto-expression, $\Cof{e}$, as follows:
%   \[\arraycolsep=1pt\begin{array}{rl}
%   \Cof{x} & = x\\
%   \Cof{\aelam{\tau}{x}{e}} & = \acelam{\Cof{\tau}}{x}{\Cof{e}}\\
%   \Cof{\aeap{e_1}{e_2}} & = \aceap{\Cof{e_1}}{\Cof{e_2}}\\
%   \Cof{\aeclam{\kappa}{u}{e}} & = \aceclam{\Cof{\kappa}}{u}{\Cof{e}}\\
%   \Cof{\aecap{e}{c}} & = \acecap{\Cof{e}}{\Cof{c}}\\
%   \Cof{\aefold{e}} & = \acefold{\Cof e}\\
%   \Cof{\aeunfold{e}} & = \aceunfold{\Cof{e}}\\
%   \Cof{\aetpl{\labelset}{\mapschema{e}{i}{\labelset}}} & = \acetpl{\labelset}{\mapschemax{\Cofv}{e}{i}{\labelset}}\\
%   \Cof{\aein{\ell}{e}} &= \acein{\ell}{\Cof{e}}\\
%   \Cof{\aematchwith{n}{e}{\seqschemaX{r}}} & = \acematchwith{n}{\Cof{e}}{\seqschemaXx{\Cofv}{r}}\\
%   \Cof{\amval{X}} & = \acemval{X}
%   \end{array}\]
%   \item Each expanded rule, $r$, maps onto the proto-rule, $\Cof{r}$, as follows:
%   \begin{align*}
%   \Cof{\aematchrule{p}{e}} & = \acematchrule{p}{\Cof{e}}
%   \end{align*}
%   Notice that proto-rules bind expanded patterns, not proto-patterns. This is because proto-rules appear in proto-expressions, which are generated by peTLMs. It would not be sensible for an peTLM to splice a pattern out of a literal body.
%   \item Each expanded pattern, $p$, except for the variable patterns, maps onto a proto-pattern, $\Cof{p}$, as follows:
%   \begin{align*}
%   \Cof{\aewildp} & = \acewildp\\
%   \Cof{\aefoldp{p}} & = \acefoldp{\Cof{p}}\\
%   \Cof{\aetplp{\labelset}{\mapschema{p}{i}{\labelset}}} & = \acetplp{\labelset}{\mapschemax{\Cofv}{p}{i}{\labelset}}\\
%   \Cof{\aeinjp{\ell}{p}} & = \aceinjp{\ell}{\Cof{p}}
%   \end{align*}
% \end{itemize}

% \subsubsection{Parameterized Proto-Expression Encoding and Decoding}
% The type abbreviated $\tPProtoExpr$ classifies encodings of \emph{parameterized proto-expressions}. The mapping from parameterized proto-expressions to values of type $\tPProtoExpr$ is defined by the \emph{parameterized proto-expression encoding judgement}, $\encodePCEExp{\pce}{e}$. An inverse mapping is defined by the \emph{parameterized proto-expression decoding judgement}, $\decodePCEExp{e}{\pce}$.

% \[\begin{array}{ll}
% \textbf{Judgement Form} & \textbf{Description}\\
% \encodePCEExp{\pce}{e} & \text{$\pce$ has encoding $e$}\\
% \decodePCEExp{e}{\pce} & \text{$e$ has decoding $\pce$}
% \end{array}\]

% Rather than picking a particular definition of $\tPProtoExpr$ and defining the judgements above inductively against it, we only state the following condition, which establishes an isomorphism between values of type $\tPProtoExpr$ and parameterized proto-expressions.

% \begin{condition}[Parameterized Proto-Expression Isomorphism]\label{condition:parameterized-proto-expression-isomorphism} ~
% \begin{enumerate}
% \item For every $\pce$, we have $\encodePCEExp{\pce}{\ecand}$ for some $\ecand$ such that $\hastypeUC{\ecand}{\tPProtoExpr}$ and $\isvalU{\ecand}$.
% \item If $\hastypeUC{\ecand}{\tPProtoExpr}$ and $\isvalU{\ecand}$ then $\decodePCEExp{\ecand}{\pce}$ for some $\pce$.
% \item If $\encodePCEExp{\pce}{\ecand}$ then $\decodePCEExp{\ecand}{\pce}$.
% \item If $\hastypeUC{\ecand}{\tPProtoExpr}$ and $\isvalU{\ecand}$ and $\decodePCEExp{\ecand}{\pce}$ then $\encodePCEExp{\pce}{\ecand}$.
% \item If $\encodePCEExp{\pce}{\ecand}$ and $\encodePCEExp{\pce}{\ecand'}$ then $\ecand=\ecand'$.
% \item If $\hastypeUC{\ecand}{\tPProtoExpr}$ and $\isvalU{\ecand}$ and $\decodePCEExp{\ecand}{\pce}$ and $\decodePCEExp{\ecand}{\pce'}$ then $\pce=\pce'$.
% \end{enumerate}
% \end{condition}

% \subsubsection{Parameterized Proto-Pattern Encoding and Decoding}
% The type abbreviated $\tPCEPat$ classifies encodings of \emph{parameterized proto-patterns}. The mapping from parameterized proto-patterns to values of type $\tPCEPat$ is defined by the \emph{parameterized proto-pattern encoding judgement}, $\encodePCEPat{\pcp}{p}$. An inverse mapping is defined by the \emph{parameterized proto-expression decoding judgement}, $\decodePCEPat{p}{\pcp}$.

% \[\begin{array}{ll}
% \textbf{Judgement Form} & \textbf{Description}\\
% \encodePCEPat{\pcp}{p} & \text{$\pcp$ has encoding $p$}\\
% \decodePCEPat{p}{\pcp} & \text{$p$ has decoding $\pcp$}
% \end{array}\]

% Again, rather than picking a particular definition of $\tPCEPat$ and defining the judgements above inductively against it, we only state the following condition, which establishes an isomorphism between values of type $\tPCEPat$ and parameterized proto-patterns.

% \begin{condition}[Parameterized Proto-Pattern Isomorphism]\label{condition:proto-pattern-isomorphism-P} ~
% \begin{enumerate}
% \item For every $\pcp$, we have $\encodePCEPat{\pcp}{\ecand}$ for some $\ecand$ such that $\hastypeUC{\ecand}{\tPCEPat}$ and $\isvalU{\ecand}$.
% \item If $\hastypeUC{\ecand}{\tPCEPat}$ and $\isvalU{\ecand}$ then $\decodePCEPat{\ecand}{\pcp}$ for some $\pcp$.
% \item If $\encodePCEPat{\pcp}{\ecand}$ then $\decodePCEPat{\ecand}{\pcp}$.
% \item If $\hastypeUC{\ecand}{\tPCEPat}$ and $\isvalU{\ecand}$ and $\decodePCEPat{\ecand}{\pcp}$ then $\encodePCEPat{\pcp}{\ecand}$.
% \item If $\encodePCEPat{\pcp}{\ecand}$ and $\encodePCEPat{\pcp}{\ecand'}$ then $\ecand=\ecand'$.
% \item If $\hastypeUC{\ecand}{\tPCEPat}$ and $\isvalU{\ecand}$ and $\decodePCEPat{\ecand}{\pcp}$ and $\decodePCEPat{\ecand}{\pcp'}$ then $\pcp=\pcp'$.
% \end{enumerate}
% \end{condition}

% \subsubsection{Segmentations}\label{appendix:segmentations-P}
% The \emph{segmentation}, $\psi$, of a proto-kind, $\segof{\cekappa}$, proto-construction, $\segof{\cec}$, proto-expression, $\segof{\ce}$, or proto-rule, $\segof{\crv}$, is the finite set of references to spliced kinds, constructions and expressions that it mentions.
% \[
% \begin{array}{lll}
% \segof{\acekdarr{\cekappa_1}{u}{\cekappa_2}} & = & \segof{\cekappa_1} \cup \segof{\cekappa_2} \\
% \segof{\acekunit} & = & \emptyset\\
% \segof{\acekdbprod{\cekappa_1}{u}{\cekappa_2}} & = & \segof{\cekappa_1} \cup \cekappa_2\\
% \segof{\acekty} & = & \emptyset\\
% \segof{\aceksing{\ctau}} & = & \segof{\ctau}\\
% \segof{\acesplicedk{m}{n}} & = & \{ \acesplicedk{m}{n} \}\\
% ~\\
% \segof{u} & = & \emptyset\\
% \segof{\acecabs{u}{\cec}} & = & \segof{\cec}\\
% \segof{\acecapp{\cec_1}{\cec_2}} & = & \segof{\cec_1} \cup \segof{\cec_2}\\
% \segof{\acectriv} & = & \emptyset\\
% \segof{\acecpair{\cec_1}{\cec_2}} & = & \segof{\cec_1} \cup \segof{\cec_2}\\
% \segof{\acecprl{\cec}} & = & \segof{\cec}\\
% \segof{\acecprr{\cec}} & = & \segof{\cec}\\
% \segof{\aceparr{\ctau_1}{\ctau_2}} & = & \segof{\ctau_1} \cup \segof{\ctau_2}\\
% \segof{\aceallu{\cekappa}{u}{\ctau}} & = & \segof{\cekappa} \cup \segof{\ctau}\\
% \segof{\acerec{t}{\ctau}} & = & \segof{\ctau}\\
% \segof{\aceprod{\labelset}{\mapschema{\ctau}{i}{\labelset}}} & = & \bigcup_{i \in \labelset} \segof{\ctau_i}\\
% \segof{\acesum{\labelset}{\mapschema{\ctau}{i}{\labelset}}} & = & \bigcup_{i \in \labelset} \segof{\ctau_i}\\
% \segof{\acemcon{X}} & = & \emptyset\\
% \segof{\acesplicedc{m}{n}{\cekappa}} & = & \{ \acesplicedc{m}{n}{\cekappa} \} \cup \segof{\cekappa}\\
% ~\\
% \segof{x} & = & \emptyset\\
% \segof{\aceasc{\ctau}{\ce}} & = & \segof{\ctau} \cup \segof{\ce}\\
% \segof{\aceletsyn{x}{\ce_1}{\ce_2}} & = & \segof{\ce_1} \cup \segof{\ce_2}\\
% \segof{\acelam{\ctau}{x}{\ce}} & = & \segof{\ctau} \cup \segof{\ce}\\
% \segof{\aceap{\ce_1}{\ce_2}} & = & \segof{\ce_1} \cup \segof{\ce_2}\\
% \segof{\aceclam{\cekappa}{u}{\ce}} & = & \segof{\cekappa} \cup \segof{\ce}\\
% \segof{\acecap{\ce}{\cec}} & = & \segof{\ce} \cup \segof{\cec}\\
% \segof{\acefold{\ce}} & = & \segof{\ce}\\
% \segof{\aceunfold{\ce}} & = & \segof{\ce}\\
% \segof{\acetpl{\labelset}{\mapschema{\ce}{i}{\labelset}}} & = & \bigcup_{i \in \labelset} \segof{\ce_i}\\
% \segof{\acepr{\ell}{\ce}} & = & \segof{\ce}\\
% \segof{\aceanain{\ell}{\ce}} & = & \segof{\ce}\\
% \segof{\acematchwith{n}{\ce}{\seqschemaX{\crv}}} & = & \segof{\ce} \cup \bigcup_{1 \leq i \leq n} \segof{\crv_i}\\
% \segof{\acemval{X}} & = & \emptyset\\
% \segof{\acesplicede{m}{n}{\ctau}} & = & \{\acesplicede{m}{n}{\ctau}\} \cup \segof{\ctau}\\
% ~\\
% \segof{\acematchrule{p}{\ce}} & = & \segof{\ce}
% \end{array}
% \]

% The segmentation of a proto-pattern, $\segof{\cpv}$, is the finite set of references to spliced patterns and types that it mentions.
% \[
% \begin{array}{lll}
% \segof{\acewildp} & = & \emptyset\\
% \segof{\acefoldp{\cpv}} & = & \segof{\cpv}\\
% \segof{\acetplp{\labelset}{\mapschema{\cpv}{i}{\labelset}}} & = & \bigcup_{i \in \labelset} \segof{\cpv_i}\\
% \segof{\aceinjp{\ell}{\cpv}} & = & \segof{\cpv}\\
% \segof{\acesplicedp{m}{n}{\ctau}} & = & \{ \acesplicedp{m}{n}{\ctau} \} \cup \segof{\ctau}
% \end{array}
% \]

% The predicate $\segOKP{\Omega}{\cscenev}{\psi}{b}$ checks that each segment in $\psi$, has non-negative length and is within bounds of $b$, and that the segments in $\psi$ do not overlap and operate at consistent sorts, kinds and types. The contexts are needed because kind and type equivalence are contextual.

% \begin{definition}[Segmentation Validity] $\segOKP{\Omega}{\cscenev}{\psi}{b}$ where \[\cscenev=\csceneP{\omega : \OParams}{\uOmegaEx{\uD}{\uG}{\uMctx}{\Omega_\text{app}}}{b}\] iff
% \begin{enumerate}
%   \item For each $\acesplicedk{m}{n} \in \psi$, all of the following hold:
%     \begin{enumerate}
%       \item $0 \leq m < n \leq \sizeof{b}$
%       \item For each $\acesplicedk{m'}{n'} \in \psi$, either 
%         \begin{enumerate}
%           \item $m=m'$ and $n=n'$; or
%           \item $n' < m$; or 
%           \item $m' > n$
%         \end{enumerate}
%       \item For each $\acesplicedc{m'}{n'}{\cekappa} \in \psi$, either
%         \begin{enumerate}
%           \item $n' < m$; or 
%           \item $m' > n$
%         \end{enumerate}        
%       \item For each $\acesplicede{m'}{n'}{\ctau} \in \psi$, either 
%         \begin{enumerate}
%           \item $n' < m$; or 
%           \item $m' > n$
%         \end{enumerate}
%       \item For each $\acesplicedp{m'}{n'}{\ctau} \in \psi$, either 
%         \begin{enumerate}
%           \item $n' < m$; or 
%           \item $m' > n$
%         \end{enumerate}
%     \end{enumerate}
%   \item For each $\acesplicedc{m}{n}{\cekappa} \in \psi$, all of the following hold:
%     \begin{enumerate}
%       \item $0 \leq m < n \leq \sizeof{b}$
%       \item For each $\acesplicedk{m'}{n'} \in \psi$, either 
%         \begin{enumerate}
%           \item $n' < m$; or 
%           \item $m' > n$
%         \end{enumerate}
%       \item For each $\acesplicedc{m'}{n'}{\cekappa'} \in \psi$, either
%         \begin{enumerate}
%           \item $m=m'$ and $n=n'$ and $\cvalidKX{\cekappa}{\kappa}$ and $\cvalidKX{\cekappa'}{\kappa'}$ and $\kequal{\Omega \cup \Omega_\text{app}}{\kappa}{\kappa'}$; or
%           \item $n' < m$; or 
%           \item $m' > n$
%         \end{enumerate}        
%       \item For each $\acesplicede{m'}{n'}{\ctau} \in \psi$, either 
%         \begin{enumerate}
%           \item $n' < m$; or 
%           \item $m' > n$
%         \end{enumerate}
%       \item For each $\acesplicedp{m'}{n'}{\ctau} \in \psi$, either 
%         \begin{enumerate}
%           \item $n' < m$; or 
%           \item $m' > n$
%         \end{enumerate}
%     \end{enumerate}
%   \item For each $\acesplicede{m}{n}{\ctau} \in \psi$, all of the following hold:
%     \begin{enumerate}
%       \item $0 \leq m < n \leq \sizeof{b}$
%       \item For each $\acesplicedk{m'}{n'} \in \psi$, either 
%         \begin{enumerate}
%           \item $n' < m$; or 
%           \item $m' > n$
%         \end{enumerate}
%       \item For each $\acesplicedc{m'}{n'}{\cekappa'} \in \psi$, either
%         \begin{enumerate}
%           \item $n' < m$; or 
%           \item $m' > n$
%         \end{enumerate}        
%       \item For each $\acesplicede{m'}{n'}{\ctau'} \in \psi$, either 
%         \begin{enumerate}
%           \item $m=m'$ and $n=n'$ and $\cvalidCX{\ctau}{\tau}{\akty}$ and $\cvalidCX{\ctau'}{\tau'}{\akty}$ and $\cequal{\Omega \cup \Omega_\text{app}}{\tau}{\tau'}{\akty}$; or
%           \item $n' < m$; or 
%           \item $m' > n$
%         \end{enumerate}
%       \item For each $\acesplicedp{m'}{n'}{\ctau} \in \psi$, either 
%         \begin{enumerate}
%           \item $n' < m$; or 
%           \item $m' > n$
%         \end{enumerate}
%     \end{enumerate}
%   \item For each $\acesplicedp{m}{n}{\ctau} \in \psi$, all of the following hold:
%     \begin{enumerate}
%       \item $0 \leq m < n \leq \sizeof{b}$
%       \item For each $\acesplicedk{m'}{n'} \in \psi$, either 
%         \begin{enumerate}
%           \item $n' < m$; or 
%           \item $m' > n$
%         \end{enumerate}
%       \item For each $\acesplicedc{m'}{n'}{\cekappa'} \in \psi$, either
%         \begin{enumerate}
%           \item $n' < m$; or 
%           \item $m' > n$
%         \end{enumerate}        
%       \item For each $\acesplicede{m'}{n'}{\ctau'} \in \psi$, either 
%         \begin{enumerate}
%           \item $n' < m$; or 
%           \item $m' > n$
%         \end{enumerate}
%       \item For each $\acesplicedp{m'}{n'}{\ctau} \in \psi$, either 
%         \begin{enumerate}
%           \item $m=m'$ and $n=n'$ and $\cvalidCX{\ctau}{\tau}{\akty}$ and $\cvalidCX{\ctau'}{\tau'}{\akty}$ and $\cequal{\Omega \cup \Omega_\text{app}}{\tau}{\tau'}{\akty}$; or
%           \item $n' < m$; or 
%           \item $m' > n$
%         \end{enumerate}
%     \end{enumerate}
% \end{enumerate}
% \end{definition}
% % \subsubsection{Segmentations}
% % A \emph{segmentation}, $\psi$, is a finite set of \emph{segments}. Segments consist of two natural numbers and a sort, i.e. segments are of the form $\segKind{m}{n}$ or $\segCon{m}{n}$ or $\segExp{m}{n}$ or $\segPat{m}{n}$.

% % The metafunction $\segof{\ce}$ determines the segmentation of $\ce$ by generating one segment for each reference to a spliced term. More specifically:
% % \begin{itemize}
% % \item We define $\segof{\cekappa}$ as follows:
% % \[\arraycolsep=1pt\begin{array}{rl}
% %   \segof{\acekdarr{\cekappa_1}{u}{\cekappa_2}} & = \segof{\cekappa_1} \cup \segof{\cekappa_2}\\
% %   \segof{\acekunit} & = \emptyset\\
% %   \segof{\acekdbprod{\cekappa_1}{u}{\cekappa_2}} & = \segof{\cekappa_1} \cup \segof{\cekappa_2}\\
% %   \segof{\acekty} & = \emptyset\\
% %   \segof{\aceksing{\ctau}} & = \segof{\ctau}\\
% %   \segof{\acesplicedk{m}{n}} & = \{ \segKind{m}{n} \}
% % \end{array}\]
% % \item We define $\segof{\cec}$ as follows:
% % \[\arraycolsep=1pt\begin{array}{rl}
% %   \segof{u} & = \emptyset\\
% %   \segof{\acecabs{u}{\cec}} & = \segof{\cec}\\
% %   \segof{\acecapp{\cec_1}{\cec_2}} & = \segof{\cec_1} \cup \segof{\cec_2}\\
% %   \segof{\acectriv} & = \emptyset\\
% %   \segof{\acecpair{\cec_1}{\cec_2}} & = \segof{\cec_1} \cup \segof{\cec_2}\\
% %   \segof{\acecprl{\cec}} & = \segof{\cec}\\
% %   \segof{\acecprr{\cec}} & = \segof{\cec}\\
% %   \segof{\aceparr{\ctau_1}{\ctau_2}} & = \segof{\ctau_1} \cup \segof{\ctau_2}\\
% %   \segof{\aceallu{u}{\cekappa}{\ctau}} &= \segof{\cekappa} \cup \segof{\ctau}\\
% %   \segof{\acerec{t}{\ctau}} & = \segof{\ctau}\\
% %   \segof{\aceprod{\labelset}{\mapschema{\ctau}{i}{\labelset}}} & = \cup_{i \in \labelset} \segof{\ctau_i}\\
% %   \segof{\acesum{\labelset}{\mapschema{\ctau}{i}{\labelset}}} & = \cup_{i \in \labelset} \segof{\ctau_i}\\
% %   \segof{\acemcon{X}} & = \emptyset\\
% %   \segof{\acesplicedc{m}{n}} & = \{ \segCon{m}{n} \}
% %   \end{array}\]
% % \item We define $\segof{\ce}$ as follows:
% % \[\arraycolsep=1pt\begin{array}{rl} 

% % \segof{x} & = \emptyset\\
% % \segof{\acelam{\ctau}{x}{\ce}} & = \segof{\ctau} \cup \segof{\ce} \\
% % \segof{\aceclam{\cekappa}{u}{\ce}} & = \segof{\cekappa} \cup \segof{\ce}\\
% % \segof{\acecap{\ce}{\cec}} & = \segof{\cec} \cup \segof{\ce}\\
% % \segof{\acefold{\ce}} & = \segof{\ce}\\
% % \segof{\aceunfold{\ce}} & = \segof{\ce}\\
% % \segof{\acetpl{\labelset}{\mapschema{\ce}{i}{\labelset}}} & = \cup_{i \in \labelset} \segof{\ce_i}\\
% % \segof{\acepr{\ell}{\ce}} & = \segof{\ce}\\
% % \segof{\acein{\ell}{\ce}} & = \segof{\ce}\\
% % \segof{\acematchwith{n}{\ce}{\seqschemaX{\crv}}} & = \segof{\ce} \cup_{1 \leq i \leq n} \segof{\crv_i}\\
% % \segof{\acemval{X}} & = \emptyset\\
% % \segof{\acesplicede{m}{n}{\ctau}} & = \{ \segExp{m}{n} \} \cup \segof{\ctau}\\
% % \end{array}\]
% % \item We define $\segof{\crv}$ as follows:
% % \[\arraycolsep=1pt\begin{array}{rl} 

% % \segof{\acematchrule{p}{\ce}} & = \segof{\ce}
% % \end{array}\]
% % \end{itemize}

% % The metafunction $\segof{\cpv}$ determines the segmentation of $\cpv$ by generating one segment for each reference to a spliced type or pattern:
% % \[
% % \arraycolsep=1pt\begin{array}{rl}

% % \segof{\acewildp} & = \emptyset\\
% % \segof{\acefoldp{\cpv}} & = \segof{\cpv}\\
% % \segof{\acetplp{\labelset}{\mapschema{\cpv}{i}{\labelset}}} & = \cup_{i \in \labelset} \segof{\cpv_i}\\
% % \segof{\aceinjp{\ell}{\cpv}} & = \segof{\cpv}\\
% % \segof{\acesplicedp{m}{n}{\ctau}} & = \{ \segPat{m}{n} \} \cup \segof{\ctau}
% % \end{array}
% % \]

% % The predicate $\segOK{\psi}{b}$ checks that each segment in $\psi$, has non-negative length and is within bounds of $b$, and that the segments in $\psi$ do not overlap.

% \subsection{Deparameterization}
% \begin{minipage}{0.42\textwidth}
% \noindent\fbox{$\strut\prepce{\Omega_\text{app}}{\Psi}{\pce}{\ce}{\epsilon}{\rho}{\omega}{\Omega_\text{params}}$}\end{minipage}
% \begin{minipage}{0.58\textwidth}
% When applying peTLM $\epsilon$, $\pce$ has deparameterization $\ce$ leaving $\rho$ with parameter substitution $\omega$\end{minipage}
% \begin{subequations}\label{rules:prepce}
% \begin{equation}\label{rule:prepce-ceexp}
% \inferrule{
%   \istsmty{\Omega_\text{app}}{\rho}
% }{
%   \prepce{\Omega_\text{app}}{\Psi, \petsmdefn{a}{\rho}{\eparse}}{\apceexp{\ce}}{\ce}{\adefref{a}}{\rho}{\emptyset}{\emptyset}
% }
% \end{equation}
% % \begin{equation}\label{rule:prepce-alltypes}
% % \inferrule{
% %   \prepce{\Omega_\text{app}}{\Psi}{\pce}{\ce}{\epsilon}{\aealltypes{t}{\rho}}{\omega}{\Omega}\\
% %   t \notin \domof{\Omega_\text{app}}
% % }{
% %   \prepce{\Omega_\text{app}}{\Psi}{\apcebindtype{t}{\pce}}{\ce}{\aeaptype{\tau}{\epsilon}}{\rho}{\omega, \tau/t}{\Omega, t :: \akty}
% % }
% % \end{equation}
% \begin{equation}\label{rule:prepce-allmods}
% \inferrule{
%   \prepce{\Omega_\text{app}}{\Psi}{\pce}{\ce}{\epsilon}{\aeallmods{\sigma}{X}{\rho}}{\omega}{\Omega}\\\\
%   \hassig{\Omega_\text{app}}{X'}{\sigma}\\
%   X \notin \domof{\Omega_\text{app}}
% }{
%   \prepce{\Omega_\text{app}}{\Psi}{\apcebindmod{X}{\pce}}{\ce}{\aeapmod{X'}{\epsilon}}{\rho}{(\omega, X'/X)}{(\Omega, X : \sigma)}
% }
% \end{equation}
% \end{subequations}

% \noindent\begin{minipage}{0.42\textwidth}
% \fbox{$\strut\prepcp{\Omega_\text{app}}{\Phi}{\pcp}{\cpv}{\epsilon}{\rho}{\omega}{\Omega_\text{params}}$}\end{minipage}
% \begin{minipage}{0.58\textwidth}
% When applying ppTLM $\epsilon$, $\pcp$ has deparameterization $\cpv$ leaving $\rho$ with parameter substitution $\omega$\end{minipage}
% \begin{subequations}\label{rules:prepcp}
% \begin{equation}\label{rule:prepcp-cepat}
% \inferrule{
%   \istsmty{\Omega_\text{app}}{\rho}
% }{
%   \prepcp{\Omega_\text{app}}{\Phi, \pptsmdefn{a}{\rho}{\eparse}}{\apcepat{\cpv}}{\cpv}{\adefref{a}}{\rho}{\emptyset}{\emptyset}
% }
% \end{equation}
% % \begin{equation}\label{rule:prepcp-alltypes}
% % \inferrule{
% %   \prepcp{\Omega_\text{app}}{\Phi}{\pcp}{\cpv}{\epsilon}{\aealltypes{t}{\rho}}{\omega}{\Omega}\\
% %   t \notin \domof{\Omega_\text{app}}
% % }{
% %   \prepcp{\Omega_\text{app}}{\Phi}{\apcebindtype{t}{\pcp}}{\cpv}{\aeaptype{\tau}{\epsilon}}{\rho}{\omega, \tau/t}{\Omega, t :: \akty}
% % }
% % \end{equation}
% \begin{equation}\label{rule:prepcp-allmods}
% \inferrule{
%   \prepcp{\Omega_\text{app}}{\Phi}{\pcp}{\cpv}{\epsilon}{\aeallmods{\sigma}{X}{\rho}}{\omega}{\Omega}\\\\
%   \hassig{\Omega_\text{app}}{X'}{\sigma}\\
%   X \notin \domof{\Omega_\text{app}}
% }{
%   \prepcp{\Omega_\text{app}}{\Phi}{\apcebindmod{X}{\pcp}}{\cpv}{\aeapmod{X'}{\epsilon}}{\rho}{(\omega, X'/X)}{(\Omega, X : \sigma)}
% }
% \end{equation}
% % \begin{equation}\label{rule:prepcp-ceexp}
% % \inferrule{ }{
% %   \prepcp{\Omega_\text{app}}{\Phi, \pptsmdefn{a}{\rho}{\eparse}}{\adefref{a}}{\rho}{\emptyset}{\emptyset}
% % }
% % \end{equation}
% % \begin{equation}\label{rule:prepcp-alltypes}
% % \inferrule{
% %   \prepcp{\Omega_\text{app}}{\Phi}{\epsilon}{\aealltypes{t}{\rho}}{\omega}{\Omega}\\
% %   t \notin \domof{\Omega_\text{app}}
% % }{
% %   \prepcp{\Omega_\text{app}}{\Phi}{\aeaptype{\tau}{\epsilon}}{\rho}{\omega, \tau/t}{\Omega, t :: \akty}
% % }
% % \end{equation}
% % \begin{equation}\label{rule:prepcp-allmods}
% % \inferrule{
% %   \prepcp{\Omega_\text{app}}{\Phi}{\epsilon}{\aeallmods{\sigma}{X}{\rho}}{\omega}{\Omega}\\
% %   X \notin \domof{\Omega_\text{app}}
% % }{
% %   \prepcp{\Omega_\text{app}}{\Phi}{\aeapmod{X'}{\epsilon}}{\rho}{\omega, X'/X}{\Omega, X : \sigma}
% % }
% % \end{equation}
% \end{subequations}

% \subsection{Proto-Expansion Validation}
% \subsubsection{Splicing Scenes}
% \emph{Expression splicing scenes}, $\escenev$, are of the form $\esceneP{\omega : \Omega_\text{params}}{\uOmega}{\uPsi}{\uPhi}{b}$, \emph{construction splicing scenes}, $\cscenev$, are of the form $\csceneP{\omega : \Omega_\text{params}}{\uOmega}{b}$, and \emph{pattern splicing scenes}, $\pscenev$, are of the form $\psceneP{\omega : \Omega_\text{params}}{\uOmega}{\uPhi}{b}$. We write $\csfrom{\escenev}$ for the construction splicing scene constructed by dropping the TLM contexts from $\escenev$:
% \[\csfrom{\esceneP{\omega : \OParams}{\uOmega}{\uPsi}{\uPhi}{b}} = \csceneP{\omega : \OParams}{\uOmega}{b}\]

% \subsubsection{Proto-Kind and Proto-Construction Validation}
% \noindent\fbox{$\strut\cvalidKX{\cekappa}{\kappa}$}~~$\cekappa$ has well-formed expansion $\kappa$
% \begin{subequations}\label{rules:cvalidK}
% \begin{equation}\label{rule:cvalidK-darr}
% \inferrule{
%   \cvalidKX{\cekappa_1}{\kappa_1}\\
%   \cvalidK{\Omega, u :: \kappa_1}{\cscenev}{\cekappa_2}{\kappa_2}
% }{
%   \cvalidKX{\acekdarr{\cekappa_1}{u}{\cekappa_2}}{\akdarr{\kappa_1}{u}{\kappa_2}}
% }
% \end{equation}
% \begin{equation}\label{rule:cvalidK-unit}
% \inferrule{ }{
%   \cvalidKX{\acekunit}{\akunit}
% }
% \end{equation}
% \begin{equation}\label{rule:cvalidK-dprod}
% \inferrule{
%   \cvalidKX{\cekappa_1}{\kappa_1}\\
%   \cvalidK{\Omega, u :: \kappa_1}{\cscenev}{\cekappa_2}{\kappa_2}
% }{
%   \cvalidKX{\acekdbprod{\cekappa_1}{u}{\cekappa_2}}{\akdbprod{\kappa_1}{u}{\kappa_2}}
% }
% \end{equation}
% \begin{equation}\label{rule:cvalidK-ty}
% \inferrule{ }{
%   \cvalidKX{\acekty}{\akty}
% }
% \end{equation}
% \begin{equation}\label{rule:cvalidK-sing}
% \inferrule{
%   \cvalidCX{\ctau}{\tau}{\akty}
% }{
%   \cvalidKX{\aceksing{\ctau}}{\aksing{\tau}}
% }
% \end{equation}
% \begin{equation}\label{rule:cvalidK-spliced}
% \inferrule{
%   \parseUKind{\bsubseq{b}{m}{n}}{\ukappa}\\
%   \kExpands{\uOmega}{\ukappa}{\kappa}\\\\
%   \uOmega=\uOmegaEx{\uD}{\uG}{\uMctx}{\Omega_\text{app}}\\
%   \domof{\Omega} \cap \domof{\Omega_\text{app}} = \emptyset
% }{
%   \cvalidK{\Omega}{\csceneP{\omega : \OParams}{\uOmega}{b}}{\acesplicedk{m}{n}}{\kappa}
% }
% \end{equation}
% \end{subequations}

% \noindent\fbox{$\strut\cvalidCX{\cec}{c}{\kappa}$}~~$\cec$ has expansion $c$ of kind $\kappa$
% \begin{subequations}\label{rules:cvalidC}
% \begin{equation}\label{rule:cvalidC-subsume}
% \inferrule{
%   \cvalidCX{\cec}{c}{\kappa_1}\\
%   \ksubX{\kappa_1}{\kappa_2}
% }{
%   \cvalidCX{\cec}{c}{\kappa_2}
% }
% \end{equation}
% \begin{equation}\label{rule:cvalidC-var}
% \inferrule{ }{\cvalidC{\Omega, {u} :: {\kappa}}{\cscenev}{u}{u}{\kappa}}
% \end{equation}
% \begin{equation}\label{rule:cvalidC-abs}
% \inferrule{
%   \cvalidC{\Omega, u :: \kappa_1}{\cscenev}{\cec_2}{c_2}{\kappa_2}
% }{
%   \cvalidCX{\acecabs{u}{\cec_2}}{\acabs{u}{c_2}}{\akdarr{\kappa_1}{u}{\kappa_2}}
% }
% \end{equation}
% \begin{equation}\label{rule:cvalidC-app}
% \inferrule{
%   \cvalidCX{\cec_1}{c_1}{\akdarr{\kappa_2}{u}{\kappa}}\\
%   \cvalidCX{\cec_2}{c_2}{\kappa_2}
% }{
%   \cvalidCX{\acecapp{\cec_1}{\cec_2}}{\acapp{c_1}{c_2}}{[c_1/u]\kappa}
% }
% \end{equation}
% \begin{equation}\label{rule:cvalidC-unit}
% \inferrule{ }{
%   \cvalidCX{\acectriv}{\actriv}{\akunit}
% }
% \end{equation}
% \begin{equation}\label{rule:cvalidC-pair}
% \inferrule{
%   \cvalidCX{\cec_1}{c_1}{\kappa_1}\\
%   \cvalidCX{\cec_2}{c_2}{[c_1/u]\kappa_2}
% }{
%   \cvalidCX{\acecpair{\cec_1}{\cec_2}}{\acpair{c_1}{c_2}}{\akdbprod{\kappa_1}{u}{\kappa_2}}
% }
% \end{equation}
% \begin{equation}\label{rule:cvalidC-prl}
% \inferrule{
%   \cvalidCX{\cec}{c}{\akdbprod{\kappa_1}{u}{\kappa_2}}
% }{
%   \cvalidCX{\acecprl{\cec}}{\acprl{c}}{\kappa_1}
% }
% \end{equation}
% \begin{equation}\label{rule:cvalidC-prr}
% \inferrule{
%   \cvalidCX{\cec}{c}{\akdbprod{\kappa_1}{u}{\kappa_2}}
% }{
%   \cvalidCX{\acecprr{\cec}}{\acprr{c}}{[\acprl{c}/u]\kappa_2}
% }
% \end{equation}
% \begin{equation}\label{rule:cvalidC-parr}
% \inferrule{
%   \cvalidCX{\ctau_1}{\tau_1}{\akty}\\
%   \cvalidCX{\ctau_2}{\tau_2}{\akty}
% }{
%   \cvalidCX{\aceparr{\ctau_1}{\ctau_2}}{\aparr{\tau_1}{\tau_2}}{\akty}
% }
% \end{equation}
% \begin{equation}\label{rule:cvalidC-all}
% \inferrule{
%   \cvalidKX{\cekappa}{\kappa}\\
%   \cvalidC{\Omega, u :: \kappa}{\cscenev}{\ctau}{\tau}{\akty}
% }{
%   \cvalidCX{\aceallu{\cekappa}{u}{\ctau}}{\aallu{\kappa}{u}{\tau}}{\akty}
% }
% \end{equation}
% \begin{equation}\label{rule:cvalidC-rec}
% \inferrule{
%   \cvalidC{\Omega, t :: \akty}{\cscenev}{\ctau}{\tau}{\akty}
% }{
%   \cvalidCX{\acerec{t}{\ctau}}{\arec{t}{\tau}}{\akty}
% }
% \end{equation}
% \begin{equation}\label{rule:cvalidC-prod}
% \inferrule{
%   \{\cvalidCX{\ctau_i}{\tau_i}{\akty}\}_{1 \leq i \leq n}
% }{
%   \cvalidCX{\aceprod{\labelset}{\mapschema{\ctau}{i}{\labelset}}}{\aprod{\labelset}{\mapschema{\tau}{i}{\labelset}}}{\akty}
% }
% \end{equation}
% \begin{equation}\label{rule:cvalidC-sum}
% \inferrule{
%   \{\cvalidCX{\ctau_i}{\tau_i}{\akty}\}_{1 \leq i \leq n}
% }{
%   \cvalidCX{\acesum{\labelset}{\mapschema{\ctau}{i}{\labelset}}}{\asum{\labelset}{\mapschema{\tau}{i}{\labelset}}}{\akty}
% }
% \end{equation}
% \begin{equation}\label{rule:cvalidC-sing}
% \inferrule{
%   \cvalidCX{\cec}{c}{\akty}
% }{
%   \cvalidCX{\cec}{c}{\aksing{c}}
% }
% \end{equation}
% \begin{equation}\label{rule:cvalidC-stat}
% \inferrule{ }{
%   \cvalidC{\Omega, X : {\asignature{\kappa}{u}{\tau}}}{\cscenev}{\acemcon{X}}{\amcon{X}}{\kappa}
% }
% \end{equation}
% \begin{equation}\label{rule:cvalidC-spliced}
% \inferrule{
%   \cscenev=\csceneP{\omega : \OParams}{\uOmega}{b}\\
%   \cvalidK{\OParams}{\cscenev}{\cekappa}{\kappa}\\\\
%   \parseUCon{\bsubseq{b}{m}{n}}{\uc}\\
%   \cExpands{\uOmega}{\uc}{c}{[\omega]\kappa}\\\\
%   \uOmega=\uOmegaEx{\uD}{\uG}{\uMctx}{\Omega_\text{app}}\\
%   \domof{\Omega} \cap \domof{\Omega_\text{app}} = \emptyset
% }{
%   \cvalidC{\Omega}{\cscenev}{\acesplicedc{m}{n}{\cekappa}}{c}{\kappa}
% }
% \end{equation}
% \end{subequations}
% \subsubsection{Proto-Expression and Proto-Rule Validation}
% % \begin{equation}
% % \inferrule{
% %   \ccanaX{\ctau}{\tau}{\akty}
% % }{
% %   \cvalidTP{\Omega}{\cscenev}{\ctau}{\tau}
% % }
% % \end{equation}
% \noindent\fbox{$\strut\cvalidEPX{\ce}{e}{\tau}$}~~$\ce$ has expansion $e$ of type $\tau$
% \begin{subequations}\label{rules:cvalidE-P}
% \begin{equation}\label{rule:cvalidE-P-subsume}
%   \inferrule{
%     \cvalidEPX{\ce}{e}{\tau}\\
%     \issubtypePX{\tau}{\tau'}
%   }{
%     \cvalidEPX{\ce}{e}{\tau'}
%   }
% \end{equation}
% \begin{equation}\label{rule:cvalidE-P-var}
%   \inferrule{ }{ 
%     \cvalidEP{\Omega, \Ghyp{x}{\tau}}{\escenev}{x}{x}{\tau}
%   }
% \end{equation}
% \begin{equation}\label{rule:cvalidE-P-asc}
% \inferrule{
%   \cvalidC{\Omega}{\csfrom{\escenev}}{\ctau}{\tau}{\akty}\\
%   \cvalidEP{\Omega}{\escenev}{\ce}{e}{\tau}
% }{
%   \cvalidEP{\Omega}{\escenev}{\aceasc{\ctau}{\ce}}{e}{\tau}
% }
% \end{equation}
% \begin{equation}\label{rule:cvalidE-P-letsyn}
%   \inferrule{
%     \cvalidEP{\Omega}{\escenev}{\ce_1}{e_1}{\tau_1}\\
%     \cvalidEP{\Omega, x : \tau_1}{\ce_2}{e_2}{\tau_2}
%   }{
%     \cvalidEP{\Omega}{\escenev}{\aceletsyn{x}{\ce_1}{\ce_2}}{
%       \aeap{\aelam{\tau_1}{x}{e_2}}{e_1}
%     }{\tau_2}
%   }
% \end{equation}
% \begin{equation}\label{rule:cvalidE-P-lam}
%   \inferrule{
%     \cvalidC{\Omega}{\csfrom{\escenev}}{\ctau_1}{\tau_1}{\akty}\\
%     \cvalidEP{\Omega, \Ghyp{x}{\tau_1}}{\escenev}{\ce}{e}{\tau_2}
%   }{
%     \cvalidEPX{\acelam{\ctau_1}{x}{\ce}}{\aelam{\tau_1}{x}{e}}{\aparr{\tau_1}{\tau_2}}
%   }
% \end{equation}
% \begin{equation}\label{rule:cvalidE-P-ap}
%   \inferrule{
%     \cvalidEPX{\ce_1}{e_1}{\aparr{\tau_2}{\tau}}\\
%     \cvalidEPX{\ce_2}{e_2}{\tau_2}
%   }{
%     \cvalidEPX{\aceap{\ce_1}{\ce_2}}{\aeap{e_1}{e_2}}{\tau}
%   }
% \end{equation}
% \begin{equation}\label{rule:cvalidE-P-clam}
%   \inferrule{
%     \cvalidK{\Omega}{\csfrom{\escenev}}{\cekappa}{\kappa}\\
%     \cvalidEP{\Omega, u :: \kappa}{\escenev}{\ce}{e}{\tau}
%   }{
%     \csynX{\aceclam{\cekappa}{u}{\ce}}{\aeclam{\kappa}{u}{e}}{\aallu{\kappa}{u}{\tau}}
%   }
% \end{equation}
% \begin{equation}\label{rule:cvalidE-P-cap}
%   \inferrule{
%     \cvalidEPX{\ce}{e}{\aallu{\kappa}{u}{\tau}}\\
%     \cvalidC{\Omega}{\csfrom{\escenev}}{\cec}{c}{\kappa}
%   }{
%     \cvalidEPX{\acecap{\ce}{\cec}}{\aecap{e}{c}}{[c/u]\tau}
%   }
% \end{equation}
% \begin{equation}\label{rule:cvalidE-P-fold}
%   \inferrule{
%     \cvalidEPX{\ce}{e}{[\arec{t}{\tau}/t]\tau}
%   }{
%     \cvalidEPX{\aceanafold{\ce}}{\aefold{e}}{\arec{t}{\tau}}
%   }
% \end{equation}
% \begin{equation}\label{rule:cvalidE-P-unfold}
%   \inferrule{
%     \cvalidEPX{\ce}{e}{\arec{t}{\tau}}
%   }{
%     \cvalidEPX{\aceunfold{\ce}}{\aeunfold{e}}{[\arec{t}{\tau}/t]\tau}
%   }
% \end{equation}
% \begin{equation}\label{rule:cvalidE-P-tpl}
%   \inferrule{
%     \tau=\aprod{\labelset}{\mapschema{\tau}{i}{\labelset}}\\\\    
%     \{\cvalidEPX{\ce_i}{e_i}{\tau_i}\}_{i \in \labelset}
%   }{
%     \cvalidEPX{\acetpl{\labelset}{\mapschema{\ce}{i}{\labelset}}}{\aetpl{\labelset}{\mapschema{e}{i}{\labelset}}}{\tau}
%   }
% \end{equation}
% \begin{equation}\label{rule:cvalidE-P-pr}
%   \inferrule{
%     \cvalidEPX{\ce}{e}{\aprod{\labelset, \ell}{\mapschema{\tau}{i}{\labelset}; \mapitem{\ell}{\tau}}}
%   }{
%     \cvalidEPX{\acepr{\ell}{\ce}}{\aepr{\ell}{e}}{\tau}
%   }
% \end{equation}
% \begin{equation}\label{rule:cvalidE-P-in}
%   \inferrule{
%     \asum{\labelset, \ell}{\mapschema{\tau}{i}{\labelset}; \mapitem{\ell}{\tau'}}\\\\
%     \cvalidEPX{\ce'}{e'}{\tau'}
%   }{
%     \cvalidEPX{\aceanain{\ell}{\ce'}}{\aein{\ell}{e'}}{\tau}
%     %\left(\shortstack{$\Delta~\Gamma \vdash^{\escenev} $\\$\leadsto$\\$ \Leftarrow $\vspace{-1.2em}}\right)
%     %\eanaX{\auanain{\ell}{\ue}}{\aein{\ell}}{\asum{\labelset, \ell}{\mapschema{\tau}{i}{\labelset}; \mapitem{\ell}{\tau}}}
%   }
% \end{equation}
% \begin{equation}\label{rule:cvalidE-P-match}
%   \inferrule{
%     % n > 0\\
%     \cvalidEPX{\ce}{e}{\tau}\\
%     \{\cvalidRP{\Omega}{\escenev}{\crv_i}{r_i}{\tau}{\tau'}\}_{1 \leq i \leq n}
%   }{
%     \cvalidEPX{\acematchwithb{n}{\ce}{\seqschemaX{\crv}}}{\aematchwith{n}{e}{\seqschemaX{r}}}{\tau'}
%   }
% \end{equation}
% \begin{equation}\label{rule:cvalidE-P-mval}
% \inferrule{ }{
%   \cvalidEP{\Omega, X : \asignature{\kappa}{u}{\tau}}{\escenev}{\acemval{X}}{\amval{X}}{[\amcon{X}/u]\tau}
% }
% \end{equation}
% \begin{equation}\label{rule:cvalidE-P-splicede}
% \inferrule{
%   \escenev = \esceneP{\omega : \OParams}{\uOmega}{\uPsi}{\uPhi}{b}\\
%   \cvalidC{\OParams}{\csfrom{\escenev}}{\ctau}{\tau}{\akty}\\\\
%   \parseUExp{\bsubseq{b}{m}{n}}{\ue}\\
%   \expandsP{\uOmega}{\uPsi}{\uPhi}{\ue}{e}{[\omega]\tau}\\\\
%   \uOmega=\uOmegaEx{\uD}{\uG}{\uMctx}{\Omega_\text{app}}\\
%   \domof{\Omega} \cap \domof{\Omega_\text{app}} = \emptyset
% }{
%   \cvalidEP{\Omega}{\escenev}{\acesplicede{m}{n}{\ctau}}{e}{\tau}
% }
% \end{equation}
% \end{subequations}

% \noindent\fbox{$\strut\cvalidRP{\Omega}{\escenev}{\crv}{r}{\tau}{\tau'}$}~~$\crv$ has expansion $r$ taking values of type $\tau$ to values of type $\tau'$
% \begin{equation}\label{rule:cvalidR-P}
% \inferrule{
%   \patTypeP{\Omega'}{p}{\tau}\\
%   \cvalidEP{\Gcons{\Omega}{\Omega'}}{\escenev}{\ce}{e}{\tau'}
% }{
%   \cvalidRP{\Omega}{\escenev}{\acematchrule{p}{\ce}}{\aematchrule{p}{e}}{\tau}{\tau'}
% }
% \end{equation}

% \subsubsection{Proto-Pattern Validation}
% \noindent\fbox{$\strut\cvalidPPE{\uOmega}{\pscenev}{\cpv}{p}{\tau}$}~~$\cpv$ has expansion $p$ matching against $\tau$ generating hypotheses $\uOmega$
% \begin{subequations}\label{rules:cvalidPP}
% \begin{equation}\label{rule:cvalidPP-wild}
% \inferrule{ }{
%   \cvalidPP{\uOmegaEx{\emptyset}{\emptyset}{\emptyset}{\emptyset}}{\pscenev}{\acewildp}{\aewildp}{\tau}
% }
% \end{equation}
% \begin{equation}\label{rule:cvalidPP-fold}
% \inferrule{
%   \cvalidPP{\uOmega}{\pscenev}{\cpv}{p}{[\arec{t}{\tau}/t]\tau}
% }{
%   \cvalidPP{\uOmega}{\pscenev}{\acefoldp{\cpv}}{\aefoldp{p}}{\arec{t}{\tau}}
% }
% \end{equation}
% \begin{equation}\label{rule:cvalidPP-tpl}
% \inferrule{
%   \cpv=\acetplp{\labelset}{\mapschema{\cpv}{i}{\labelset}}\\
%   p=\aetplp{\labelset}{\mapschema{p}{i}{\labelset}}\\\\
%   \{\cvalidPP{\upctx_i}{\pscenev}{\cpv_i}{p_i}{\tau_i}\}_{i \in \labelset}
% }{
%   \cvalidPP{\GIconsi{i \in \labelset}{\uOmega_i}}{\pscenev}{\cpv}{p}{\aprod{\labelset}{\mapschema{\tau}{i}{\labelset}}}
%   %\cvalidPP{}{\cpv}{p}{}
% %\left(\shortstack{$\vdash^{\pscenev} $\\$\leadsto$\\$ :~\dashVx^{\,\Gconsi{i \in \labelset}{\upctx_i}}$\vspace{-1.2em}}\right)
% }
% \end{equation}
% \begin{equation}\label{rule:cvalidPP-in}
% \inferrule{
%   \cvalidPP{\uOmega}{\pscenev}{\cpv}{p}{\tau}
% }{
%   \cvalidPP{\uOmega}{\pscenev}{\aceinjp{\ell}{\cpv}}{\aeinjp{\ell}{p}}{\asum{\labelset, \ell}{\mapschema{\tau}{i}{\labelset}; \mapitem{\ell}{\tau}}}
% }
% \end{equation}
% \begin{equation}\label{rule:cvalidPP-spliced}
% \inferrule{
%   \cvalidC{\OParams}{\csceneP{\omega : \OParams}{\uOmega}{b}}{\ctau}{\tau}{\akty}\\
%   \parseUPat{\bsubseq{b}{m}{n}}{\upv}\\
%   \patExpandsP{\uOmega'}{\uPhi}{\upv}{p}{[\omega]\tau}
% }{
%   \cvalidPP{\uOmega'}{\psceneP{\omega : \Omega_\text{params}}{\uOmega}{\uPhi}{b}}{\acesplicedp{m}{n}{\ctau}}{p}{\tau}
% }
% \end{equation}
% \end{subequations}


% \section{Metatheory}\label{appendix:metatheory-P}
% \subsection{TLM Expressions}
% \begin{lemma}[peTLM Regularity]
% \label{lemma:peTLM-regularity}
% If $\hastsmtypeExp{\Omega}{\Psi}{\epsilon}{\rho}$ then $\istsmty{\Omega}{\rho}$.
% \end{lemma}
% \begin{proof}
%   By rule induction over Rules (\ref{rules:hastsmtypeExp}). 
%   \begin{byCases}
%     \item[\text{(\ref{rule:hastsmtypeExp-defref})}] ~
%       \begin{pfsteps*}
%         \item $\istsmty{\Omega}{\rho}$ \BY{assumption}
%       \end{pfsteps*}
%       \resetpfcounter
%     \item[\text{(\ref{rule:hastsmtypeExp-absmod})}] ~
%       \begin{pfsteps*}
%         \item $\epsilon=\aeabsmod{\sigma}{X}{\epsilon'}$ \BY{assumption}
%         \item $\rho=\aeallmods{\sigma}{X}{\rho'}$ \BY{assumption}
%         \item $\hastsmtypeExp{\Omega, X : \sigma}{\Psi}{\epsilon'}{\rho'}$ \BY{assumption} \pflabel{hastsmtype}
%         \item $\issigX{\sigma}$ \BY{assumption} \pflabel{issig}
%         \item $\istsmty{\Omega, X : \sigma}{\rho'}$ \BY{IH on \pfref{hastsmtype}} \pflabel{istsmty}
%         \item $\istsmty{\Omega}{\aeallmods{\sigma}{X}{\rho'}}$ \BY{Rule (\ref{rule:istsmty-allmods}) on \pfref{issig} and \pfref{istsmty}}
%       \end{pfsteps*}
%       \resetpfcounter
%     \item[\text{(\ref{rule:hastsmtypeExp-apmod})}] ~
%       \begin{pfsteps*}
%         \item $\epsilon=\aeapmod{X}{\epsilon'}$ \BY{assumption}
%         \item $\rho=[X/X']\rho'$ \BY{assumption}
%         \item $\hastsmtypeExp{\Omega}{\Psi}{\epsilon}{\aeallmods{\sigma}{X'}{\rho'}}$ \BY{assumption} \pflabel{hastsmtype}
%         \item $\hassig{\Omega}{X}{\sigma}$ \BY{assumption} \pflabel{hassig}
%         \item $\istsmty{\Omega}{\aeallmods{\sigma}{X'}{\rho'}}$ \BY{IH on \pfref{hastsmtype}} \pflabel{istsmty}
%         \item $\istsmty{\Omega, X' : \sigma}{\rho'}$ \BY{Inversion of Rule (\ref{rule:istsmty-allmods}) on \pfref{istsmty}} \pflabel{istsmtyprime}
%         \item $\istsmty{\Omega}{[X'/X]\rho'}$ \BY{Substitution Lemma \ref{lemma:substitution-P} on \pfref{hassig} and \pfref{istsmtyprime}}
%       \end{pfsteps*}
%       \resetpfcounter
%   \end{byCases}
% \end{proof}

% \begin{lemma}[ppTLM Regularity]
% \label{lemma:ppTLM-regularity}
% If $\hastsmtypePat{\Omega}{\Phi}{\epsilon}{\rho}$ then $\istsmty{\Omega}{\rho}$.
% \end{lemma}
% \begin{proof}
% By rule induction over Rules (\ref{rules:hastsmtypePat}). The proof is nearly identical to the proof of Lemma \ref{lemma:peTLM-regularity}, differing only in that ppTLM contexts and the corresponding judgements are mentioned.
% \end{proof}

% \begin{lemma}[peTLM Unicity]
% \label{lemma:peTLM-unicity}
% If $\hastsmtypeExp{\Omega}{\Psi}{\epsilon}{\rho}$ and $\hastsmtypeExp{\Omega}{\Psi}{\epsilon}{\rho'}$ then $\rho=\rho'$.
% \end{lemma}
% \begin{proof}
%   By rule induction over Rules (\ref{rules:hastsmtypeExp}). The rules are syntax-directed, so the proof is by straightforward observations of syntactic contradictions.
% \end{proof}

% \begin{lemma}[ppTLM Unicity]
% \label{lemma:ppTLM-unicity}
% If $\hastsmtypePat{\Omega}{\Phi}{\epsilon}{\rho}$ and $\hastsmtypePat{\Omega}{\Phi}{\epsilon}{\rho'}$ then $\rho=\rho'$.
% \end{lemma}
% \begin{proof}
%   By rule induction over Rules (\ref{rules:hastsmtypePat}). The rules are syntax-directed, so the proof is by straightforward observations of syntactic contradictions.
% \end{proof}

% \begin{theorem}[peTLM Preservation]
% \label{thm:peTLM-preservation}
% If $\hastsmtypeExp{\Omega}{\Psi}{\epsilon}{\rho}$ and $\tsmexpStepsExp{\Omega}{\Psi}{\epsilon}{\epsilon'}$ then $\hastsmtypeExp{\Omega}{\Psi}{\epsilon'}{\rho}$.
% \end{theorem}
% \begin{proof}
% By rule induction over Rules (\ref{rules:tsmexpStepsExp}).

% \begin{byCases}
% \item[\text{(\ref{rule:tsmexpStepsExp-apmod-1})}] By rule induction over Rules (\ref{rules:hastsmtypeExp}). There is only one rule that applies.
%   \begin{byCases}
%   \item[\text{(\ref{rule:hastsmtypeExp-apmod})}] ~
%     \begin{pfsteps*}
%       \item $\epsilon=\aeapmod{X}{\epsilon''}$ \BY{assumption}
%       \item $\epsilon'=\aeapmod{X}{\epsilon'''}$ \BY{assumption}
%       \item $\rho=[X/X']\rho'$ \BY{assumption}
%       \item $\tsmexpStepsExp{\Omega}{\Psi}{\epsilon''}{\epsilon'''}$ \BY{assumption} \pflabel{steps}
%       \item $\hastsmtypeExp{\Omega}{\Psi}{\epsilon''}{\aeallmods{\sigma}{X'}{\rho'}}$ \BY{assumption} \pflabel{tsmtype}
%       \item $\hassig{\Omega}{X}{\sigma}$ \BY{assumption} \pflabel{hassig}
%       \item $\hastsmtypeExp{\Omega}{\Psi}{\epsilon'''}{\aeallmods{\sigma}{X'}{\rho'}}$ \BY{IH on \pfref{tsmtype} and \pfref{steps}}\pflabel{IH}
%       \item $\hastsmtypeExp{\Omega}{\Psi}{\aeapmod{X}{\epsilon'''}}{[X/X']\rho}$ \BY{Rule (\ref{rule:hastsmtypeExp-apmod}) on \pfref{IH} and \pfref{hassig}}
%     \end{pfsteps*}
%     \resetpfcounter
%   \end{byCases}
% %   \begin{equation}\label{rule:tsmexpStepsExp-apmod-2}
% % \inferrule{ }{
% %   \tsmexpStepsExp{\Omega}{\Psi}{\aeapmod{X}{\aeabsmod{\sigma}{X'}{\epsilon}}}{[X/X']\epsilon}
% % }
% % \end{equation}
% \item[\text{(\ref{rule:tsmexpStepsExp-apmod-2})}] By rule induction over Rules (\ref{rules:hastsmtypeExp}). There is only one rule that applies.
%   \begin{byCases}
%     \item[\text{(\ref{rule:hastsmtypeExp-apmod})}] ~
%       \begin{pfsteps*}
%         \item $\epsilon=\aeapmod{X}{\aeabsmod{\sigma}{X'}{\epsilon''}}$ \BY{assumption}
%         \item $\epsilon'=[X/X']\epsilon''$ \BY{assumption}
%         \item $\rho=[X/X']\rho'$ \BY{assumption}
%         \item $\hastsmtypeExp{\Omega}{\Psi}{\aeabsmod{\sigma}{X'}{\epsilon''}}{\aeallmods{\sigma}{X'}{\rho'}}$ \BY{assumption} \pflabel{tsmtype}
%         \item $\hassig{\Omega}{X}{\sigma}$ \BY{assumption} \pflabel{hassig}
%         \item $\hastsmtypeExp{\Omega, X' : \sigma}{\Psi}{\epsilon''}{\rho'}$ \BY{Inversion Lemma for Rule (\ref{rule:hastsmtypeExp-absmod}) on \pfref{tsmtype}} \pflabel{tsmtype2}
%         \item $\hastsmtypeExp{\Omega}{\Psi}{[X/X']\epsilon''}{[X/X']\rho'}$ \BY{Substitution Lemma \ref{lemma:substitution-P} on \pfref{tsmtype2}}
%       \end{pfsteps*}
%       \resetpfcounter
%   \end{byCases}
% \end{byCases}
% \end{proof}

% \begin{corollary}[peTLM Preservation (Multistep)]
% \label{thm:peTLM-preservation-multistep}
% If $\hastsmtypeExp{\Omega}{\Psi}{\epsilon}{\rho}$ and $\tsmexpMultistepsExp{\Omega}{\Psi}{\epsilon}{\epsilon'}$ then $\hastsmtypeExp{\Omega}{\Psi}{\epsilon'}{\rho}$.
% \end{corollary}
% \begin{proof} The multistep relation is the reflexive, transitive closure of the single step relation, so the proof follows by applying Theorem \ref{thm:peTLM-preservation} over each step. \end{proof}

% \begin{corollary}[peTLM Preservation (Evaluation)]
% \label{thm:peTLM-preservation-evaluation}
% If $\hastsmtypeExp{\Omega}{\Psi}{\epsilon}{\rho}$ and $\tsmexpEvalsExp{\Omega}{\Psi}{\epsilon}{\epsilon'}$ then $\hastsmtypeExp{\Omega}{\Psi}{\epsilon'}{\rho}$.
% \end{corollary}
% \begin{proof} The evaluation relation is the multistep relation with an additional requirement, so the proof follows directly from Corollary \ref{thm:peTLM-preservation-multistep}. \end{proof}

% \begin{theorem}[ppTLM Preservation]
% \label{thm:ppTLM-preservation}
% If $\hastsmtypePat{\Omega}{\Phi}{\epsilon}{\rho}$ and $\tsmexpStepsPat{\Omega}{\Phi}{\epsilon}{\epsilon'}$ then $\hastsmtypePat{\Omega}{\Phi}{\epsilon'}{\rho}$.
% \end{theorem}
% \begin{proof} The proof is nearly identical to the proof of Theorem \ref{thm:peTLM-preservation}, differing only in that ppTLM contexts and the corresponding judgements are mentioned. \end{proof}

% \begin{corollary}[ppTLM Preservation (Multistep)]
% \label{thm:ppTLM-preservation-multistep}
% If $\hastsmtypePat{\Omega}{\Phi}{\epsilon}{\rho}$ and $\tsmexpMultistepsPat{\Omega}{\Phi}{\epsilon}{\epsilon'}$ then $\hastsmtypePat{\Omega}{\Phi}{\epsilon'}{\rho}$.
% \end{corollary}
% \begin{proof} The multistep relation is the reflexive, transitive closure of the single step relation, so the proof follows by applying Theorem \ref{thm:ppTLM-preservation} over each step. \end{proof}

% \begin{corollary}[ppTLM Preservation (Evaluation)]
% \label{thm:ppTLM-preservation-evaluation}
% If $\hastsmtypePat{\Omega}{\Phi}{\epsilon}{\rho}$ and $\tsmexpEvalsPat{\Omega}{\Phi}{\epsilon}{\epsilon'}$ then $\hastsmtypePat{\Omega}{\Phi}{\epsilon'}{\rho}$.
% \end{corollary}
% \begin{proof} The evaluation relation is the multistep relation with an additional requirement, so the proof follows directly from Corollary \ref{thm:ppTLM-preservation-multistep}. \end{proof}

% \begin{theorem}[peTLM Progress]
% \label{thm:peTLM-progress}
% If $\hastsmtypeExp{\Omega}{\Psi}{\epsilon}{\rho}$ then either $\tsmexpStepsExp{\Omega}{\Psi}{\epsilon}{\epsilon'}$ for some $\epsilon'$ or $\tsmexpNormalExp{\Omega}{\Psi}{\epsilon}$.
% \end{theorem}
% \begin{proof}
% By rule induction over Rules (\ref{rules:hastsmtypeExp}).
% \begin{byCases}
%   \item[\text{(\ref{rule:hastsmtypeExp-defref})}] ~
%     \begin{pfsteps*}
% %     \begin{equation}\label{rule:hastsmtypeExp-defref}
% % \inferrule{ }{
% %   \hastsmtypeExp{\Omega}{\Psi, \petsmdefn{a}{\rho}{\eparse}}{\adefref{a}}{\rho}
% % }
% % \end{equation}
%       \item $\epsilon=\adefref{a}$ \BY{assumption}
%       \item $\Psi=\Psi', \petsmdefn{a}{\rho}{\eparse}$ \BY{assumption}
%       \item $\tsmexpNormalExp{\Omega}{\Psi', \petsmdefn{a}{\rho}{\eparse}}{\adefref{a}}$ \BY{Rule (\ref{rule:tsmexpNormalExp-defref})}
%     \end{pfsteps*}
%     \resetpfcounter
%   \item[\text{(\ref{rule:hastsmtypeExp-absmod})}] ~
%     \begin{pfsteps*}
% %       \begin{equation}\label{rule:hastsmtypeExp-absmod}
% % \inferrule{
% %   \issigX{\sigma}\\
% %   \hastsmtypeExp{\Omega, X : \sigma}{\Psi}{\epsilon}{\rho}
% % }{
% %   \hastsmtypeExp{\Omega}{\Psi}{\aeabsmod{\sigma}{X}{\epsilon}}{\aeallmods{\sigma}{X}{\rho}}
% % }
% % \end{equation}
%       \item $\epsilon=\aeabsmod{\sigma}{X}{\epsilon'}$ \BY{assumption}
%       \item $\tsmexpNormalExp{\Omega}{\Psi}{\aeabsmod{\sigma}{X}{\epsilon'}}$ \BY{Rule (\ref{rule:tsmexpNormalExp-absmod})}
%     \end{pfsteps*}
%     \resetpfcounter
%   \item[\text{(\ref{rule:hastsmtypeExp-apmod})}] ~
%     \begin{pfsteps*}
% %      \begin{equation}\label{rule:hastsmtypeExp-apmod}
% % \inferrule{
% %   \hastsmtypeExp{\Omega}{\Psi}{\epsilon}{\aeallmods{\sigma}{X'}{\rho}}\\
% %   \hassig{\Omega}{X}{\sigma}
% % }{
% %   \hastsmtypeExp{\Omega}{\Psi}{\aeapmod{X}{\epsilon}}{[X/X']\rho}
% % }
% % \end{equation} 
%       \item $\epsilon=\aeapmod{X}{\epsilon'}$ \BY{assumption}
%       \item $\rho=[X/X']\rho'$ \BY{assumption}
%       \item $\hastsmtypeExp{\Omega}{\Psi}{\epsilon'}{\aeallmods{\sigma}{X'}{\rho'}}$ \BY{assumption} \pflabel{tsmtype}
%       \item $\tsmexpStepsExp{\Omega}{\Psi}{\epsilon'}{\epsilon''}$ for some $\epsilon''$ or $\tsmexpNormalExp{\Omega}{\Psi}{\epsilon'}$ \BY{IH on \pfref{tsmtype}} \pflabel{ih}
%     \end{pfsteps*}
%     Proceed by cases on \pfref{ih}.
%     \begin{byCases}
%       \item[\tsmexpStepsExp{\Omega}{\Psi}{\epsilon'}{\epsilon''}] 
%         \begin{pfsteps*}
%           \item $\tsmexpStepsExp{\Omega}{\Psi}{\epsilon'}{\epsilon''}$ \BY{assumption} \pflabel{steps}
%           \item $\tsmexpStepsExp{\Omega}{\Psi}{\aeapmod{X}{\epsilon'}}{\aeapmod{X}{\epsilon''}}$ \BY{Rule (\ref{rule:tsmexpStepsExp-apmod-1}) on \pfref{steps}}
%         \end{pfsteps*}
%       \item[\tsmexpNormalExp{\Omega}{\Psi}{\epsilon'}] Proceed by rule induction over Rules (\ref{rules:tsmexpNormalExp}).
%         \begin{byCases}
%           \item[\text{(\ref{rule:tsmexpNormalExp-defref})}] ~
%             \begin{pfsteps*}
%               \item $\epsilon' = \adefref{a}$ \BY{assumption}
%               \item $\tsmexpNormalExp{\Omega}{\Psi}{\epsilon'}$ \BY{assumption} \pflabel{normal}
%               \item $\tsmexpNormalExp{\Omega}{\Psi}{\aeapmod{X}{\epsilon'}}$ \BY{Rule (\ref{rule:tsmexpNormalExp-apmod}) on \pfref{normal}}
%             \end{pfsteps*}
%           \item[\text{(\ref{rule:tsmexpNormalExp-absmod})}] ~
%             \begin{pfsteps*}
%               \item $\epsilon' = \aeabsmod{\sigma}{X}{\epsilon''}$ \BY{assumption}
%               \item $\tsmexpStepsExp{\Omega}{\Psi}{\aeapmod{X}{\aeabsmod{\sigma}{X'}{\epsilon''}}}{[X/X']\epsilon''}$ \BY{Rule (\ref{rule:tsmexpStepsExp-apmod-2})}
%             \end{pfsteps*}
%           \item[\text{(\ref{rule:tsmexpNormalExp-apmod})}] ~
%             \begin{pfsteps*}
%               \item $\epsilon' = \aeapmod{X''}{\epsilon''}$ \BY{assumption}
%               \item $\tsmexpNormalExp{\Omega}{\Psi}{\epsilon'}$ \BY{assumption} \pflabel{normal2}              
%               \item $\tsmexpNormalExp{\Omega}{\Psi}{\aeapmod{X}{\epsilon'}}$ \BY{Rule (\ref{rule:tsmexpNormalExp-apmod}) on \pfref{normal}}
%             \end{pfsteps*}
%         \end{byCases}
%     \end{byCases}
%     \resetpfcounter
% \end{byCases}
% \end{proof}

% \begin{theorem}[ppTLM Progress]
% \label{thm:ppTLM-progress}
% If $\hastsmtypePat{\Omega}{\Phi}{\epsilon}{\rho}$ then either $\tsmexpStepsPat{\Omega}{\Phi}{\epsilon}{\epsilon'}$ for some $\epsilon'$ or $\tsmexpNormalPat{\Omega}{\Phi}{\epsilon}$.
% \end{theorem}
% \begin{proof} The proof is nearly identical to the proof of Theorem \ref{thm:peTLM-progress}, differing only in that ppTLM contexts and the corresponding judgements are mentioned. \end{proof}

% \subsection{Typed Expansion}
% \subsubsection{Kinds, Constructions and Signatures}
% \begin{theorem}[Kind and Construction Expansion]
% \label{thm:kind-and-constructor-expansion-P}
% ~
% \begin{enumerate}
% \item If $\kExpands{\uOmegaEx{\uD}{\uG}{\uMctx}{\Omega}}{\ukappa}{\kappa}$ then $\iskind{\Omega}{\kappa}$.
% \item If $\cExpands{\uOmegaEx{\uD}{\uG}{\uMctx}{\Omega}}{\uc}{c}{\kappa}$ then $\haskind{\Omega}{c}{\kappa}$.
% \end{enumerate}
% \end{theorem}
% \begin{proof}
% By mutual rule induction over Rules (\ref{rules:kExpands}) and Rules (\ref{rules:cExpands}). In each case, we apply the IH to each premise and then apply the corresponding kind formation rule from Rules (\ref{rules:iskind}) or kinding rule from Rules (\ref{rules:haskind}).
% \end{proof}

% \begin{theorem}[Signature Expansion]
% \label{thm:signature-expansion-P}
% If $\sigExpandsP{\uOmegaEx{\uD}{\uG}{\uMctx}{\Omega}}{\usigma}{\sigma}$ then $\issig{\Omega}{\sigma}$.
% \end{theorem}
% \begin{proof} By rule induction over Rule (\ref{rule:sigExpandsP}). Apply Theorem \ref{thm:kind-and-constructor-expansion-P} to each premise, then apply Rule (\ref{rule:issig}). \end{proof}


% \subsubsection{TLM Types and Expressions}
% \begin{theorem}[TLM Type Expansion]
% \label{thm:tsm-type-expansion-P}
% If $\tsmtyExpands{\uOmegaEx{\uD}{\uG}{\uMctx}{\Omega}}{\urho}{\rho}$ then $\istsmty{\Omega}{\rho}$.
% \end{theorem}
% \begin{proof} By rule induction over Rules (\ref{rules:tsmtyExpands}).
% \begin{byCases}
%   \item[\text{(\ref{rule:tsmtyExpands-type})}] ~
%     \begin{pfsteps*}
%       \item $\urho = \utau$ \BY{assumption}
%       \item $\rho = \aetype{\tau}$ \BY{assumption}
%       \item $\cExpandsX{\utau}{\tau}{\akty}$ \BY{assumption} \pflabel{cexpands}
%       \item $\haskindX{\tau}{\akty}$ \BY{Theorem \ref{thm:kind-and-constructor-expansion-P} on \pfref{cexpands}} \pflabel{haskind}
%       \item $\istsmty{\Omega}{\aetype{\tau}}$ \BY{Rule (\ref{rule:istsmty-type}) on \pfref{haskind}}
%     \end{pfsteps*}
%     \resetpfcounter
%   \item[\text{(\ref{rule:tsmtyExpands-allmods})}] ~
%     \begin{pfsteps*}
%       \item $\urho=\allmods{\uX}{\usigma}{\urho'}$ \BY{assumption}
%       \item $\rho=\aeallmods{\sigma}{X}{\rho'}$ \BY{assumption}
%       \item $\sigExpandsPX{\usigma}{\sigma}$ \BY{assumption} \pflabel{sigexpands}
%       \item $\tsmtyExpands{\uOmega, \uMhyp{\uX}{X}{\sigma}}{\urho'}{\rho'}$ \BY{assumption} \pflabel{tsmtyexpands}
%       \item $\issig{\Omega}{\sigma}$ \BY{Theorem \ref{thm:signature-expansion-P} on \pfref{sigexpands}} \pflabel{issig}
%       \item $\istsmty{\Omega, X : \sigma}{\rho'}$ \BY{IH on \pfref{tsmtyexpands}} \pflabel{istsmty} 
%       \item $\istsmty{\Omega}{\aeallmods{\sigma}{X}{\rho'}}$ \BY{Rule (\ref{rule:istsmty-allmods}) on \pfref{issig} and \pfref{istsmty}}
%     \end{pfsteps*}
%     \resetpfcounter
% \end{byCases}

% \end{proof}

% \begin{theorem}[peTLM Expression Expansion]
% \label{thm:peTLM-expression-expansion}
% If $\tsmexpExpandsExp{\uOmegaEx{\uD}{\uG}{\uMctx}{\Omega}}{\uAS{\uA}{\Psi}}{\uepsilon}{\epsilon}{\rho}$ then $\hastsmtypeExp{\Omega}{\Psi}{\epsilon}{\rho}$.
% \end{theorem}
% \begin{proof}
% By rule induction over Rules (\ref{rules:tsmexpExpandsExp}). In the following, let $\uOmega=\uOmegaEx{\uD}{\uG}{\uMctx}{\Omega}$ and $\uPsi=\uAS{\uA}{\Psi}$.
% \begin{byCases}
%   \item[\text{(\ref{rule:tsmexpExpandsExp-bindref})}] ~
%     \begin{pfsteps*}
%       \item $\uepsilon = \tsmv$ \BY{assumption}
%       \item $\uA = \uA', \mapitem{\tsmv}{\epsilon}$ \BY{assumption}
%       \item $\hastsmtypeExp{\Omega}{\Psi}{\epsilon}{\rho}$ \BY{assumption} 
%     \end{pfsteps*}
%     \resetpfcounter
%   \item[\text{(\ref{rule:tsmexpExpandsExp-absmod})}] ~
%     \begin{pfsteps*}
%       \item $\uepsilon=\absmod{\uX}{\usigma}{\uepsilon'}$ \BY{assumption}
%       \item $\epsilon=\aeabsmod{\sigma}{X}{\epsilon'}$ \BY{assumption}
%       \item $\rho=\aeallmods{\sigma}{X}{\rho'}$ \BY{assumption}
%       \item $\sigExpandsPX{\usigma}{\sigma}$ \BY{assumption} \pflabel{sigexpands}
%       \item $\tsmexpExpandsExp{\uOmega, \uMhyp{\uX}{X}{\sigma}}{\uPsi}{\uepsilon'}{\epsilon'}{\rho'}$ \BY{assumption} \pflabel{tsmexpExpands}
%       \item $\hastsmtypeExp{\Omega, X : \sigma}{\Psi}{\epsilon'}{\rho'}$ \BY{IH on \pfref{sigexpands} and \pfref{tsmexpExpands}} \pflabel{hastsmtype}
%       \item $\issigX{\sigma}$ \BY{Theorem \ref{thm:signature-expansion-P} on \pfref{sigexpands}} \pflabel{issig}
%       \item $\hastsmtypeExp{\Omega}{\Psi}{\aeabsmod{\sigma}{X}{\epsilon'}}{\aeallmods{\sigma}{X}{\rho'}}$ \BY{Rule (\ref{rule:hastsmtypeExp-absmod}) on \pfref{issig} and \pfref{hastsmtype}}
%     \end{pfsteps*}
%     \resetpfcounter
%   \item[\text{(\ref{rule:tsmexpExpandsExp-apmod})}] ~
%     \begin{pfsteps*}
%       \item $\uepsilon=\apmod{\uepsilon'}{\uX}$ \BY{assumption}
%       \item $\epsilon=\aeapmod{X}{\epsilon'}$ \BY{assumption}
%       \item $\rho=[X/X']\rho'$ \BY{assumption}
%       \item $\tsmexpExpandsExp{\uOmega}{\uPsi}{\uepsilon'}{\epsilon'}{\aeallmods{\sigma}{X'}{\rho'}}$ \BY{assumption} \pflabel{tsmexpExpands}
%       \item $\mExpandsPX{\uX}{X}{\sigma}$ \BY{assumption} \pflabel{mExpands}
%       \item $\hastsmtypeExp{\Omega}{\Psi}{\epsilon'}{\aeallmods{\sigma}{X'}{\rho'}}$ \BY{IH on \pfref{tsmexpExpands}} \pflabel{hastsmtype}
%       \item $\hassig{\Omega}{X}{\sigma}$ \BY{Theorem \ref{thm:module-expansion-P} on \pfref{mExpands}} \pflabel{hassig}
%       \item $\hastsmtypeExp{\Omega}{\Psi}{\aeapmod{X}{\epsilon'}}{[X/X']\rho'}$ \BY{Rule (\ref{rule:hastsmtypeExp-apmod}) on \pfref{hastsmtype} and \pfref{hassig}}
%     \end{pfsteps*}
%     \resetpfcounter
% \end{byCases}

% \end{proof}

% \begin{theorem}[ppTLM Expression Expansion]
% \label{thm:ppTLM-expression-expansion}
% If $\tsmexpExpandsPat{\uOmegaEx{\uD}{\uG}{\uMctx}{\Omega}}{\uAS{\uA}{\Phi}}{\uepsilon}{\epsilon}{\rho}$ then $\hastsmtypePat{\Omega}{\Phi}{\epsilon}{\rho}$.
% \end{theorem}
% \begin{proof} The proof is nearly identical to the proof of Theorem \ref{thm:peTLM-expression-expansion}, differing only in that ppTLM contexts and the corresponding judgements are mentioned. \end{proof}

% \subsubsection{Patterns}
% \begin{lemma}[Proto-Pattern Deparameterization]
% \label{lemma:pattern-deparameterization-P}
% If $\prepcp{\Omega_\text{app}}{\Phi}{\pcp}{\cpv}{\epsilon}{\rho}{\omega}{\Omega_\text{params}}$ then $\domof{\Omega_\text{app}} \cap \domof{\Omega_\text{params}} = \emptyset$ and $\hastypeP{\Omega_\text{app}}{\omega}{\Omega_\text{params}}$ and $\hastsmtypePat{\Omega_\text{app}}{\Phi}{\epsilon}{[\omega]\rho}$. 
% \end{lemma}
% \begin{proof} By rule induction over Rules (\ref{rules:prepcp}).
% \begin{byCases}
%   \item[\text{(\ref{rule:prepcp-cepat})}] We have:
%     \begin{pfsteps*}
%       \item $\epsilon=\adefref{a}$ \BY{assumption}
%       \item $\omega=\emptyset$ \BY{assumption}
%       \item $\Phi=\Phi', \pptsmdefn{a}{\rho}{\eparse}$ \BY{assumption}
%       \item $\Omega_\text{params}=\emptyset$ \BY{assumption}
%       \item $\istsmty{\Omega_\text{app}}{\rho}$ \BY{assumption} \pflabel{istsmty}
%       \item $\domof{\Omega_\text{app}} \cap \domof{\emptyset} = \emptyset$ \BY{definition}
%       \item $\hastypeP{\Omega_\text{app}}{\emptyset}{\emptyset}$ \BY{definition}
%       \item $[\emptyset]\rho=\rho$ \BY{definition}
%       \item $\hastsmtypePat{\Omega_\text{app}}{\Phi', \pptsmdefn{a}{\rho}{\eparse}}{a}{\rho}$ \BY{Rule (\ref{rule:hastsmtypePat-defref}) on \pfref{istsmty}}
%     \end{pfsteps*}
%     \resetpfcounter
%   \item[\text{(\ref{rule:prepcp-allmods})}] We have:
%     \begin{pfsteps*}
%       \item $\epsilon=\aeapmod{X}{\epsilon'}$ \BY{assumption}
%       \item $\prepcp{\Omega_\text{app}}{\Phi}{\pcp}{\cpv}{\epsilon'}{\aeallmods{\sigma}{X'}{\rho}}{\omega'}{\Omega'}$ \BY{assumption} \pflabel{prepcp}
%       \item $\hassig{\Omega_\text{app}}{X}{\sigma}$ \BY{assumption} \pflabel{hassig}
%       \item $X' \notin \domof{\Omega_\text{app}}$ \BY{assumption} \pflabel{notin}
%       \item $\omega=\omega', X/X'$ \BY{assumption}
%       \item $\Omega_\text{params} = \Omega', X' : \sigma$ \BY{assumption}
%       \item $\domof{\Omega_\text{app}} \cap \domof{\Omega'} = \emptyset$ \BY{IH on \pfref{prepcp}} \pflabel{IH1}
%       \item $\hastypeP{\Omega_\text{app}}{\omega'}{\Omega'}$ \BY{IH on \pfref{prepcp}} \pflabel{IH2}
%       \item $\hastsmtypePat{\Omega_\text{app}}{\Phi}{\epsilon'}{[\omega']\aeallmods{\sigma}{X'}{\rho}}$ \BY{IH on \pfref{prepcp}} \pflabel{IH3}
%       \item $\domof{\Omega_\text{app}} \cap \domof{\Omega', X' : \sigma}$ \BY{\pfref{notin} and \pfref{IH1} and definition of finite set intersection}
%       \item $\hastypeP{\Omega_\text{app}}{\omega', X/X'}{\Omega', X' : \sigma}$ \BY{Definition \ref{def:substitution-p} on \pfref{hassig}}
%       \item $\hastsmtypePat{\Omega_\text{app}}{\Phi}{\aeapmod{X}{\epsilon'}}{(\omega', X/X')\rho}$ \BY{Rule (\ref{rule:hastsmtypePat-apmod}) on \pfref{IH3} and \pfref{hassig}}
%     \end{pfsteps*}
%     \resetpfcounter
% \end{byCases}
% \end{proof}

% \begin{theorem}[Typed Pattern Expansion]\label{thm:typed-pattern-expansion-P} ~
% \begin{enumerate}
%   \item If $\pExpandsSP{\uOmegaEx{\uD}{\uG}{\uMctx}{\Omega_\text{app}}}{\uPhi}{\upv}{p}{\tau}{\uOmegaEx{\uD'}{\uG'}{\uMctx'}{\Omega'}}$ then $\uMctx' = \emptyset$ and $\uD' = \emptyset$ and $\patTypePC{\Omega_\text{app}}{\Omega'}{p}{\tau}$.
%   \item If $\cvalidPP{\uOmegaEx{\uD'}{\uG'}{\uMctx'}{\Omega'}}{\psceneP{\omega : \Omega_\text{params}}{\uOmegaEx{\uD}{\uG}{\uMctx}{\Omega_\text{app}}}{\uPhi}{b}}{\cpv}{p}{\tau}$ and $\domof{\Omega_\text{params}} \cap \domof{\Omega_\text{app}} = \emptyset$ then $\uMctx' = \emptyset$ and $\uD' = \emptyset$ and $\patTypePC{\Omega_\text{params} \cup \Omega_\text{app}}{\Omega'}{p}{\tau}$.
% \end{enumerate}
% \end{theorem}
% \begin{proof} My mutual rule induction over Rules (\ref{rules:patExpandsP}) and Rules (\ref{rules:cvalidPP}).
% \begin{enumerate}
% \item In the following, let $\uOmega = \uOmegaEx{\uD}{\uG}{\uMctx}{\Omega_\text{app}}$ and $\uOmega' = \uOmegaEx{\uD'}{\uG'}{\uMctx'}{\Omega'}$.
%   \begin{byCases}
%     \item[\text{(\ref{rule:patExpandsP-subsume}) \textbf{through} (\ref{rule:patExpandsP-in})}] These cases follow by applying the IH, part 1 and applying the corresponding pattern typing rule in Rules (\ref{rules:patTypeP}).

%     \item[\text{(\ref{rule:patExpandsP-apuptsm})}] We have:
%     \begin{pfsteps*}
%     \item $\upv=\utsmap{\uepsilon}{b}$ \BY{assumption}
%     \item $\uPhi=\uAS{\uA}{\Phi}$ \BY{assumption}
%     \item $\tsmexpExpandsPat{\uOmega}{\uPhi}{\uepsilon}{\epsilon}{\aetype{\tau_\text{final}}}$ \BY{assumption}
%     \item $\tsmexpEvalsPat{\Omega_\text{app}}{\Phi}{\epsilon}{\epsilon_\text{normal}}$ \BY{assumption}
%     \item $\tsmdefof{\epsilon_\text{normal}}=a$ \BY{assumption}
%     \item $\Phi = \Phi', \pptsmdefn{a}{\rho}{\eparse}$ \BY{assumption}
%     \item $\encodeBody{b}{\ebody}$ \BY{assumption}
%     \item $\evalU{\ap{\eparse}{\ebody}}{\aein{\mathtt{SuccessP}}{\ecand}}$ \BY{assumption}
%     \item $\decodePCEPat{\ecand}{\pcp}$ \BY{assumption}
%     \item $\prepcp{\Omega_\text{app}}{\Phi}{\pcp}{\cpv}{\epsilon_\text{normal}}{\aetype{\tau_\text{proto}}}{\omega}{\Omega_\text{params}}$ \BY{assumption} \pflabel{prepcp}
%     \item $\cvalidPP{\uOmega'}{\psceneP{\omega : \Omega_\text{params}}{\uOmega}{\uPhi}{b}}{\cpv}{p}{\tau_\text{proto}}$ \BY{assumption} \pflabel{cvalidPP}
%     \item $\tau = [\omega]\tau_\text{proto}$ \BY{assumption}
%     \item $\domof{\Omega_\text{params}} \cap \domof{\Omega_\text{app}} = \emptyset$ \BY{Lemma \ref{lemma:pattern-deparameterization-P} on \pfref{prepcp}} \pflabel{disjoint}
%     \item $\hastypeP{\Omega_\text{app}}{\omega}{\Omega_\text{params}}$ \BY{Lemma \ref{lemma:pattern-deparameterization-P} on \pfref{prepcp}} \pflabel{hastypeP}
%     \item $\uMctx' = \emptyset$ and $\uD' = \emptyset$ \BY{IH, part 2 on \pfref{cvalidPP} and \pfref{disjoint}}
%     \item $\patTypePC{\Omega_\text{params} \cup \Omega_\text{app}}{\Omega'}{p}{\tau_\text{proto}}$ \BY{IH, part 2 on \pfref{cvalidPP} and \pfref{disjoint}} \pflabel{patType}
%     \item $\patTypePC{\Omega_\text{app}}{\Omega'}{p}{[\omega]\tau_\text{proto}}$ \BY{Substitution Lemma \ref{lemma:substitution-P} on \pfref{hastypeP} and \pfref{patType}}
%     \end{pfsteps*}
%     \resetpfcounter
%     % \item[\text{(\ref{rule:patExpandsP-subsume})}] We have:
%     %   \begin{pfsteps*}
%     %   \item $\patExpandsP{\uOmega'}{\uPhi}{\upv}{p}{\tau'}$ \BY{assumption} \pflabel{patExpandsP}
%     %   \item $\issubtypeP{\Omega}{\tau'}{\tau}$ \BY{assumption} \pflabel{issubtypeP}
%     %   \item $\uMctx' = \emptyset$  \BY{IH, part 1 on \pfref{patExpandsP}}
%     %   \item $\uD' = \emptyset$   \BY{IH, part 1 on \pfref{patExpandsP}}
%     %   \item $\patTypeP{\Omega'}{p}{\tau'}$  \BY{IH, part 1 on \pfref{patExpandsP}}
%     %   \item $\patTypeP{\Omega'}{p}{\tau}$ \BY{Rule (\ref{rule:patTypeP-subsume}) on \pfref{patExpandsP} and \pfref{issubtypeP}}
%     %   \end{pfsteps*}
%     %   \resetpfcounter
%     % \item[\text{(\ref{rule:patExpandsP-var})}] We have:
%     %   \begin{pfsteps*}
%     %   \item $\upv=x$ \BY{assumption}
%     %   \item $p=x$ \BY{assumption}
%     %   \item $\uMctx' = \emptyset$ \BY{assumption}
%     %   \item $\uD' = \emptyset$ \BY{assumption}
%     %   \item $\Omega' = x : \tau$ \BY{assumption}
%     %   \item $\patTypeP{\Omega'}{x}{\tau}$ \BY{Rule (\ref{rule:patTypeP-var})}
%     %   \end{pfsteps*}
%     %    \resetpfcounter
%     % \item[\text{(\ref{rule:patExpandsP-wild})}] We have:
%     %   \begin{pfsteps*}
%     %   \item $\upv=\wildp$ \BY{assumption}
%     %   \item $p = \aewildp$ \BY{assumption}
%     %   \item $\uMctx' = \emptyset$ \BY{assumption}
%     %   \item $\uD' = \emptyset$ \BY{assumption}
%     %   \item $\Omega' = \emptyset$ \BY{assumption}
%     %   \item $\patTypeP{\Omega'}{\aewildp}{\tau}$ \BY{Rule (\ref{rule:patTypeP-wild})}
%     %   \end{pfsteps*}
%     %   \resetpfcounter
%     % \item[\text{(\ref{rule:patExpandsP-fold})}] We have:
%     %   \begin{pfsteps*}
%     %   \item $\upv = \foldp{\upv'}$ \BY{assumption}
%     %   \item $p = \aefoldp{p'}$ \BY{assumption}
%     %   \item $\tau=\arec{t}{\tau'}$ \BY{assumption}
%     %   \item $\patExpandsP{\uOmega'}{\uPhi}{\upv'}{p'}{[\arec{t}{\tau'}/t]\tau'}$ \BY{assumption}\pflabel{patExpandsP}
%     %   \item $\uMctx' = \emptyset$ \BY{IH, part 1 on \pfref{patExpandsP}}
%     %   \item $\uD' = \emptyset$ \BY{IH, part 1 on \pfref{patExpandsP}}
%     %   \item $\patTypeP{\Omega'}{p'}{[\arec{t}{\tau'}/t]\tau'}$ \BY{IH, part 1 on \pfref{patExpandsP}} \pflabel{patTypeP}
%     %   \item $\patTypeP{\Omega'}{\aefoldp{p'}}{\arec{t}{\tau'}}$ \BY{Rule (\ref{rule:patTypeP-fold}) on \pfref{patTypeP}}
%     %   \end{pfsteps*}
%     %   \resetpfcounter
%   \end{byCases}
% \item We induct on the premise. In the following, let $\uOmega = \uOmegaEx{\uD}{\uG}{\uMctx}{\Omega_\text{app}}$ and $\uOmega' = \uOmegaEx{\uD'}{\uG'}{\uMctx'}{\Omega'}$.
%   \begin{byCases}
%     \item[\text{(\ref{rule:cvalidPP-wild}) \textbf{through} (\ref{rule:cvalidPP-in})}] These cases follow by applying the IH, part 2 and then applying the corresponding pattern rule in Rules (\ref{rules:patTypeP}).
%     \item[\text{(\ref{rule:cvalidPP-spliced})}] ~
%       \begin{pfsteps*}
%         \item $\cpv=\acesplicedp{m}{n}{\ctau}$ \BY{assumption}
%         \item $\tau=[\omega]\tau'$ \BY{assumption}
%         \item $\cvalidC{\OParams}{\csceneP{\omega : \OParams}{\uOmega}{b}}{\ctau}{\tau'}{\akty}$ \BY{assumption} \pflabel{cvalidC}
%         \item $\parseUPat{\bsubseq{b}{m}{n}}{\upv}$ \BY{assumption} \pflabel{parseUPat}
%         \item $\patExpandsP{\uOmega'}{\uPhi}{\upv}{p}{[\omega]\tau'}$ \BY{assumption} \pflabel{patExpands}
%         \item  $\uMctx' = \emptyset$ and $\uD' = \emptyset$ \BY{IH, part 1 on \pfref{patExpands}}
%         \item $\patTypePC{\Omega_\text{app}}{\Omega'}{p}{[\omega]\tau'}$ \BY{IH, part 1 on \pfref{patExpands}} \pflabel{pattype}
%         \item $\patTypePC{\OParams \cup \Omega_\text{app}}{\Omega'}{p}{[\omega]\tau'}$ \BY{Weakening on \pfref{pattype}}
%       \end{pfsteps*}
%       \resetpfcounter
%     \end{byCases}
% \end{enumerate}
% The mutual induction can be shown to be well-founded by an argument analagous to that in the proof of Theorem \ref{thm:typed-pattern-expansion}, appealing to Condition  \ref{condition:pattern-parsing-P} and Condition \ref{condition:body-subsequences-P}.

% % \begin{align*}
% % \sizeof{\patExpands{\upctx}{\uPhi}{\upv}{p}{\tau}} & = \sizeof{\upv}\\
% % \sizeof{{\cvalidP{\upctx}{\pscene{\uDelta}{\uPhi}{b}}{\cpv}{p}{\tau}}} & = \sizeof{b}
% % \end{align*}
% % where $\sizeof{b}$ is the length of $b$ and $\sizeof{\upv}$ is the sum of the lengths of the literal bodies in $\upv$,
% % \begin{align*}
% % \sizeof{\ux} & = 0\\
% % \sizeof{\aufoldp{\upv}} & = \sizeof{\upv}\\
% % \sizeof{\autplp{\labelset}{\mapschema{\upv}{i}{\labelset}}} & = \sum_{i \in \labelset} \sizeof{\upv_i}\\
% % \sizeof{\auinjp{\ell}{\upv}} & = \sizeof{\upv}\\
% % \sizeof{\auapuptsm{b}{\tsmv}} & = \sizeof{b}\\
% % \sizeof{\auplit{b}} & = \sizeof{b}
% % \end{align*}

% % The only case in the proof of part 1 that invokes part 2 are Case (\ref{rule:patExpandsP-apuptsm}) and (\ref{rule:patExpandsP-lit}). There, we have that the metric remains stable: \begin{align*}
% %  & \sizeof{\patExpands{\upctx}{\uPhi, \uShyp{\tsmv}{x}{\tau}{\eparse}}{\auapuptsm{b}{\tsmv}}{p}{\tau}}\\
% % =& \sizeof{\patExpands{\upctx}{\uASI{\uA}{\Phi', \xuptsmbnd{a}{\tau}{\eparse}}{\uI', \designate{\tau}{a}}}{\auplit{b}}{p}{\tau}}\\
% % =& \sizeof{{\cvalidP{\upctx}{\pscene{\uDelta}{\uPhi, \uShyp{\tsmv}{x}{\tau}{\eparse}}{b}}{\cpv}{p}{\tau}}}\\
% % =&\sizeof{b}\end{align*}

% % The only case in the proof of part 2 that invokes part 1 is Case (\ref{rule:cvalidP-B-spliced}). There, we have that $\parseUPat{\bsubseq{b}{m}{n}}{\upv}$ and the IH is applied to the judgement $\patExpands{\upctx}{\uPhi}{\upv}{p}{\tau}$. Because the metric is stable when passing from part 1 to part 2, we must have that it is strictly decreasing in the other direction:
% % \[\sizeof{\patExpands{\upctx}{\uPhi}{\upv}{p}{\tau}} < \sizeof{{\cvalidP{\upctx}{\pscene{\uDelta}{\uPhi}{b}}{\acesplicedp{m}{n}{\ctau}}{p}{\tau}}}\]
% % i.e. by the definitions above, 
% % \[\sizeof{\upv} < \sizeof{b}\]

% % This is established by appeal to Condition \ref{condition:body-subsequences-P}, which states that subsequences of $b$ are no longer than $b$, and the following condition, which states that an unexpanded pattern constructed by parsing a textual sequence $b$ is strictly smaller, as measured by the metric defined above, than the length of $b$, because some characters must necessarily be used to delimit each literal body.
% % % \begin{condition}[Pattern Parsing Monotonicity]\label{condition:pattern-parsing-B} If $\parseUPat{b}{\upv}$ then $\sizeof{\upv} < \sizeof{b}$.\end{condition}

% % Combining Conditions \ref{condition:body-subsequences-P} and \ref{condition:pattern-parsing-P}, we have that $\sizeof{\ue} < \sizeof{b}$ as needed.
% \end{proof}
% \subsubsection{Expressions and Rules}
% \begin{lemma}[Proto-Expression Deparameterization]
% \label{lemma:expression-deparameterization-P}
% If $\prepce{\Omega_\text{app}}{\Psi}{\pce}{\ce}{\epsilon}{\rho}{\omega}{\Omega_\text{params}}$ then $\domof{\Omega_\text{app}} \cap \domof{\Omega_\text{params}} = \emptyset$ and $\hastypeP{\Omega_\text{app}}{\omega}{\Omega_\text{params}}$ and $\hastsmtypeExp{\Omega_\text{app}}{\Psi}{\epsilon}{[\omega]\rho}$. 
% \end{lemma}
% \begin{proof} By rule induction over Rules (\ref{rules:prepce}).
% \begin{byCases}
%   \item[\text{(\ref{rule:prepce-ceexp})}] We have:
%     \begin{pfsteps*}
%       \item $\epsilon=\adefref{a}$ \BY{assumption}
%       \item $\omega=\emptyset$ \BY{assumption}
%       \item $\Psi=\Psi', \petsmdefn{a}{\rho}{\eparse}$ \BY{assumption}
%       \item $\Omega_\text{params}=\emptyset$ \BY{assumption}
%       \item $\istsmty{\Omega_\text{app}}{\rho}$ \BY{assumption} \pflabel{istsmty}
%       \item $\domof{\Omega_\text{app}} \cap \domof{\emptyset} = \emptyset$ \BY{definition}
%       \item $\hastypeP{\Omega_\text{app}}{\emptyset}{\emptyset}$ \BY{definition}
%       \item $[\emptyset]\rho=\rho$ \BY{definition}
%       \item $\hastsmtypeExp{\Omega_\text{app}}{\Psi', \petsmdefn{a}{\rho}{\eparse}}{a}{\rho}$ \BY{Rule (\ref{rule:hastsmtypeExp-defref}) on \pfref{istsmty}}
%     \end{pfsteps*}
%     \resetpfcounter
%   \item[\text{(\ref{rule:prepce-allmods})}] We have:
%     \begin{pfsteps*}
%       \item $\epsilon=\aeapmod{X}{\epsilon'}$ \BY{assumption}
%       \item $\prepce{\Omega_\text{app}}{\Psi}{\pce}{\ce}{\epsilon'}{\aeallmods{\sigma}{X'}{\rho}}{\omega'}{\Omega'}$ \BY{assumption} \pflabel{prepcp}
%       \item $\hassig{\Omega_\text{app}}{X}{\sigma}$ \BY{assumption} \pflabel{hassig}
%       \item $X' \notin \domof{\Omega_\text{app}}$ \BY{assumption} \pflabel{notin}
%       \item $\omega=\omega', X/X'$ \BY{assumption}
%       \item $\Omega_\text{params} = \Omega', X' : \sigma$ \BY{assumption}
%       \item $\domof{\Omega_\text{app}} \cap \domof{\Omega'} = \emptyset$ \BY{IH on \pfref{prepcp}} \pflabel{IH1}
%       \item $\hastypeP{\Omega_\text{app}}{\omega'}{\Omega'}$ \BY{IH on \pfref{prepcp}} \pflabel{IH2}
%       \item $\hastsmtypeExp{\Omega_\text{app}}{\Psi}{\epsilon'}{[\omega']\aeallmods{\sigma}{X'}{\rho}}$ \BY{IH on \pfref{prepcp}} \pflabel{IH3}
%       \item $\domof{\Omega_\text{app}} \cap \domof{\Omega', X' : \sigma}$ \BY{\pfref{notin} and \pfref{IH1} and definition of finite set intersection}
%       \item $\hastypeP{\Omega_\text{app}}{\omega', X/X'}{\Omega', X' : \sigma}$ \BY{Definition \ref{def:substitution-p} on \pfref{hassig}}
%       \item $\hastsmtypeExp{\Omega_\text{app}}{\Psi}{\aeapmod{X}{\epsilon'}}{(\omega', X/X')\rho}$ \BY{Rule (\ref{rule:hastsmtypeExp-apmod}) on \pfref{IH3} and \pfref{hassig}}
%     \end{pfsteps*}
%     \resetpfcounter
% \end{byCases}
% \end{proof}

% \begin{theorem}[Typed Expression and Rule Expansion]
% \label{thm:typed-expression-expansion-P}
% ~
% \begin{enumerate}
% \item \begin{enumerate}
%   \item If $\expandsP{\uOmegaEx{\uD}{\uG}{\uMctx}{\Omega}}{\uPsi}{\uPhi}{\ue}{e}{\tau}$ then $\hastypeP{\Omega}{e}{\tau}$.
%   \item If $\rExpandsSP{\uOmegaEx{\uD}{\uG}{\uMctx}{\Omega}}{\uPsi}{\uPhi}{\urv}{r}{\tau}{\tau'}$ then $\ruleTypeP{\Omega}{r}{\tau}{\tau'}$.
%   \end{enumerate}
% \item \begin{enumerate}
%   \item If $\cvalidEP{\Omega}{\esceneP{\omega : \OParams}{\uOmegaEx{\uD}{\uG}{\uMctx}{\Omega_\text{app}}}{\uPsi}{\uPhi}{b}}{\ce}{e}{\tau}$ and $\domof{\Omega} \cap \domof{\Omega_\text{app}} = \emptyset$ then $\hastypeP{\Omega \cup \Omega_\text{app}}{e}{\tau}$.
%   \item If $\cvalidRP{\Omega}{\esceneP{\omega : \OParams}{\uOmegaEx{\uD}{\uG}{\uMctx}{\Omega_\text{app}}}{\uPsi}{\uPhi}{b}}{\crv}{r}{\tau}{\tau'}$ and $\domof{\Omega} \cap \domof{\Omega_\text{app}} = \emptyset$ then $\ruleTypeP{\Omega \cup \Omega_\text{app}}{r}{\tau}{\tau'}$.
%   \end{enumerate}
% \end{enumerate}
% \end{theorem}
% \begin{proof} 
% By mutual rule induction over Rules (\ref{rules:expandsP}), Rule (\ref{rule:rExpandsP}), Rules (\ref{rules:cvalidE-P}) and Rule (\ref{rule:cvalidR-P}).
% \begin{enumerate}
%   \item \begin{enumerate}
%     \item In the following, let $\uOmega = \uOmegaEx{\uD}{\uG}{\uMctx}{\Omega}$. 
%     \begin{byCases}
%       \item[\text{(\ref{rule:expandsP-subsume})}] ~
%         \begin{pfsteps*}
%           \item $\expandsPX{\ue}{e}{\tau'}$ \BY{assumption} \pflabel{expandsP}
%           \item $\issubtypePX{\tau}{\tau'}$ \BY{assumption} \pflabel{issubtype}
%           \item $\hastypeP{\Omega}{e}{\tau'}$ \BY{IH, part 1(a) on \pfref{expandsP}} \pflabel{hastype}
%           \item $\hastypeP{\Omega}{e}{\tau}$ \BY{Rule (\ref{rule:hastypeP-subsume}) on \pfref{hastype} and \pfref{issubtype}}
%         \end{pfsteps*}
%         \resetpfcounter
%       \item[\text{(\ref{rule:expandsP-var}) \textbf{through} (\ref{rule:expandsP-mval})}] In each of these cases, we apply the IH, part 1(a) or 1(b), over the premises and then apply the corresponding typing rule in Rules (\ref{rules:hastypeP}) and weakening as needed.
%       \item[\text{(\ref{rule:expandsP-apuetsm})}] ~
%         \begin{pfsteps*}
%           \item $\ue=\utsmap{\uepsilon}{b}$ \BY{assumption}
%           \item $e=[\omega]e'$ \BY{assumption}
%           \item $\tau=[\omega]\tau_\text{proto}$ \BY{assumption}
%           \item $\uPsi=\uAS{\uA}{\Psi}$ \BY{assumption}
%           \item $\tsmexpExpandsExp{\uOmega}{\uPsi}{\uepsilon}{\epsilon}{\aetype{\tau_\text{final}}}$ \BY{assumption} \pflabel{tsmexpExpands}
%           \item $\tsmexpEvalsExp{\Omega}{\Psi}{\epsilon}{\epsilon_\text{normal}}$ \BY{assumption} \pflabel{tsmexpEvals}
%           \item $\tsmdefof{\epsilon_\text{normal}}=a$ \BY{assumption} \pflabel{tsmdefof}
%           \item $\Psi = \Psi', \petsmdefn{a}{\rho}{\eparse}$ \BY{assumption} 
%           \item $\encodeBody{b}{\ebody}$ \BY{assumption}
%           \item $\evalU{\ap{\eparse}{\ebody}}{\aein{\mathtt{SuccessE}}{e_\text{pproto}}}$ \BY{assumption}
%           \item $\decodePCEExp{e_\text{pproto}}{\pce}$ \BY{assumption}
%           \item $\prepce{\Omega}{\Psi}{\pce}{\ce}{\epsilon_\text{normal}}{\aetype{\tau_\text{proto}}}{\omega}{\Omega_\text{params}}$ \BY{assumption} \pflabel{prepce}
%           \item $\cvalidEP{\Omega_\text{params}}{\esceneP{\omega : \OParams}{\uOmega}{\uPsi}{\uPhi}{b}}{\ce}{e'}{\tau_\text{proto}}$ \BY{assumption} \pflabel{cvalidEP}
%           \item $\hastypeP{\Omega}{\omega}{\Omega_\text{params}}$ \BY{Lemma \ref{lemma:expression-deparameterization-P} on \pfref{prepce}} \pflabel{hastypeP}
%           \item $\hastypeP{\Omega \cup \OParams}{e'}{\tau_\text{proto}}$ \BY{IH, part 2(a) on \pfref{cvalidEP}} \pflabel{hastype}
%           \item $\hastypeP{\Omega}{[\omega]e'}{[\omega]\tau_\text{proto}}$ \BY{Substitution Lemma \ref{lemma:substitution-P} on \pfref{hastypeP} and \pfref{hastype}}
%         \end{pfsteps*}
%         \resetpfcounter
%     \end{byCases}
%     \item In the following, let  $\uOmega = \uOmegaEx{\uD}{\uG}{\uMctx}{\Omega}$. 
%     \begin{byCases}
%       \item[\text{(\ref{rule:rExpandsP})}] ~
%         \begin{pfsteps*}
%           \item $\urv=\matchrule{\upv}{\ue}$ \BY{assumption}
%           \item $r=\aematchrule{p}{e}$ \BY{assumption}
%           \item $\patExpandsP{\uOmegaEx{\emptyset}{\uG'}{\emptyset}{\Omega'}}{\uPhi}{\upv}{p}{\tau}$ \BY{assumption} \pflabel{patexpands}
%           \item $\expandsP{\uOmegaEx{\uD}{\uG \uplus \uG'}{\uMctx}{\Omega \cup \Omega'}}{\uPsi}{\uPhi}{\ue}{e}{\tau'}$ \BY{assumption} \pflabel{expandsP}
%           \item $\patTypePC{\Omega}{\Omega'}{p}{\tau}$ \BY{Theorem \ref{thm:typed-pattern-expansion-P} on \pfref{patexpands}} \pflabel{pattype}
%           \item $\hastypeP{\Omega \cup \Omega'}{e}{\tau'}$ \BY{IH, part 1(a) on \pfref{expandsP}} \pflabel{hastype}
%           \item $\ruleTypeP{\Omega}{r}{\tau}{\tau'}$ \BY{Rule (\ref{rule:ruleTypeP}) on \pfref{pattype} and \pfref{hastype}}
%         \end{pfsteps*}
%         \resetpfcounter
%     \end{byCases}
%   \end{enumerate}
%   \item \begin{enumerate}
%     \item In the following, let $\escenev=\esceneP{\omega : \OParams}{\uOmegaEx{\uD}{\uG}{\uMctx}{\Omega_\text{app}}}{\uPsi}{\uPhi}{b}$.
%      \begin{byCases}
%       \item[\text{(\ref{rule:cvalidE-P-subsume})}] ~
%         \begin{pfsteps*}
%           \item $\cvalidEP{\Omega}{\escenev}{\ce}{e}{\tau'}$ \BY{assumption} \pflabel{cvalid}
%           \item $\issubtypeP{\Omega}{\tau'}{\tau}$ \BY{assumption} \pflabel{issubtype}
%           \item $\hastypeP{\Omega \cup \Omega_\text{app}}{e}{\tau'}$ \BY{IH, part 2(a) on \pfref{cvalid}} \pflabel{hastype}
%           \item $\hastypeP{\Omega \cup \Omega_\text{app}}{e}{\tau}$ \BY{Rule (\ref{rule:hastypeP-subsume}) on \pfref{hastype} and \pfref{issubtype}}
%         \end{pfsteps*}
%         \resetpfcounter
%       \item[\text{(\ref{rule:cvalidE-P-var}) \textbf{through} (\ref{rule:cvalidE-P-mval})}] In each of these cases, we apply the IH, part 2(a) or 2(b), over the premises and then apply the corresponding typing rule in Rules (\ref{rules:hastypeP}) and weakening as needed.
%       \item[\text{(\ref{rule:cvalidE-P-splicede})}] ~
%         \begin{pfsteps*}
%           \item $\ce=\acesplicede{m}{n}{\ctau}$ \BY{assumption}
%           \item $\tau=[\omega]\tau'$ \BY{assumption}
%           \item $\cvalidC{\OParams}{\csfrom{\escenev}}{\ctau}{\tau'}{\akty}$ \BY{assumption} \pflabel{cvalid}
%           \item $\parseUExp{\bsubseq{b}{m}{n}}{\ue}$ \BY{assumption} \pflabel{parseUExp}
%           \item $\expandsP{\uOmegaEx{\uD}{\uG}{\uMctx}{\Omega_\text{app}}}{\uPsi}{\uPhi}{\ue}{e}{[\omega]\tau'}$ \BY{assumption} \pflabel{expandsP}
%           % \item $\domof{\Omega} \cap \domof{\Omega_\text{app}} = \emptyset$ \BY{assumption} \pflabel{disjoint}
%           \item $\hastypeP{\Omega_\text{app}}{e}{[\omega]\tau'}$ \BY{IH, part 1(a) on \pfref{expandsP}} \pflabel{hastype}
%           \item $\hastypeP{\Omega_\text{app} \cup \Omega}{e}{[\omega]\tau'}$ \BY{Weakening on \pfref{hastype}}
%         \end{pfsteps*}
%         \resetpfcounter
%     \end{byCases}
%     \item \begin{byCases}
%       \item[\text{(\ref{rule:cvalidR-P})}] ~
%       \begin{pfsteps*}
%         \item $\crv=\acematchrule{p}{\ce}$ \BY{assumption}
%         \item $r=\aematchrule{p}{e}$ \BY{assumption}
%         \item $\patTypePC{\Omega}{\Omega'}{p}{\tau}$ \BY{assumption} \pflabel{pattype}
%         \item $\cvalidEP{\Gcons{\Omega}{\Omega'}}{\esceneP{\omega : \OParams}{\uOmegaEx{\uD}{\uG}{\uMctx}{\Omega_\text{app}}}{\uPsi}{\uPhi}{b}}{\ce}{e}{\tau'}$ \BY{assumption} \pflabel{cvalidEP}
%         \item $(\Omega \cup \Omega') \cap \Omega_\text{app} = \emptyset$ \BY{identification convention} \pflabel{disjoint}
%         \item $\hastypeP{\Omega \cup \Omega' \cup \Omega_\text{app}}{e}{\tau'}$ \BY{IH, part 2(a) on \pfref{cvalidEP} and \pfref{disjoint}} \pflabel{hastype}
%         \item $\patTypePC{\Omega \cup \Omega_\text{app}}{\Omega'}{p}{\tau}$ \BY{Weakening on \pfref{pattype}} \pflabel{pattype2}
%         \item $\ruleTypeP{\Omega \cup \Omega_\text{app}}{r}{\tau}{\tau'}$ \BY{Rule (\ref{rule:ruleTypeP}) on \pfref{pattype2} and \pfref{hastype}}
%       \end{pfsteps*}
%       \resetpfcounter
%     \end{byCases}
%   \end{enumerate}
% \end{enumerate}

% The mutual induction can be shown to be well-founded by an argument analagous to that in the proof of Theorem \ref{thm:typed-expansion-full-U}, appealing to Condition  \ref{condition:body-parsing-P} and Condition \ref{condition:body-subsequences-P}.
% \end{proof}

% \subsubsection{Modules}

% \begin{theorem}[Module Expansion]
% \label{thm:module-expansion-P}
% If $\mExpandsP{\uOmegaEx{\uD}{\uG}{\uMctx}{\Omega}}{\uPsi}{\uPhi}{\uM}{M}{\sigma}$ then $\hassig{\Omega}{M}{\sigma}$.
% \end{theorem}
% \begin{proof} 
% By rule induction over Rules (\ref{rules:mExpandsP}). In the following, let $\uOmega=\uOmegaEx{\uD}{\uG}{\uMctx}{\Omega}$.

% \begin{byCases}
%   \item[\text{(\ref{rule:mExpandsP-subsumes})}] ~
%     \begin{pfsteps*}
%       \item $\mExpandsPX{\uM}{M}{\sigma'}$ \BY{assumption} \pflabel{mexpands}
%       \item $\sigsub{\uOmega}{\sigma'}{\sigma}$ \BY{assumption} \pflabel{sigsub}
%       \item $\hassig{\Omega}{M}{\sigma'}$ \BY{IH on \pfref{mexpands}} \pflabel{hassig}
%       \item $\hassig{\Omega}{M}{\sigma}$ \BY{Rule (\ref{rule:hassig-subsume}) on \pfref{hassig} and \pfref{sigsub}}
%     \end{pfsteps*}
%     \resetpfcounter
%   \item[\text{(\ref{rule:mExpandsP-var}) \textbf{through} (\ref{rule:mExpandsP-mlet})}] In each of these cases, we apply the IH over each module expansion premise, Theorem \ref{thm:typed-expression-expansion-P} over each expression expansion premise and Theorem \ref{thm:kind-and-constructor-expansion-P} over each construction expansion premise, then apply the corresponding signature matching rule in Rules (\ref{rules:hassig}) and weakening as needed.
%   \item[\text{(\ref{rule:mExpandsP-syntaxpe}) \textbf{through} (\ref{rule:mExpandsP-letpptsm})}] In each of these cases, we apply the IH to the module expansion premise.
% \end{byCases}
% \end{proof}

% \subsection{Abstract Reasoning Principles}
% \begin{lemma}[Proto-Construction and Proto-Kind Decomposition]\label{lemma:proto-con-decomp}
% ~
% \begin{enumerate}
%   \item If $\cvalidC{\Omega}{\csceneP{\omega : \OParams}{\uOmega}{b}}{\cec}{c}{\kappa}$
%     where $\segof{\cec} = \sseq{\acesplicedk{m_i}{n_i}}{\nkind} \cup \sseq{\acesplicedc{m'_i}{n'_i}{\cekappa'_i}}{\ncon}$ then 
%     \begin{enumerate}
%       \item $\sseq{\kExpands{\uOmega}{\parseUKindF{\bsubseq{b}{m_i}{n_i}}}{\kappa_i}}{\nkind}$
%       \item $\sseq{\cvalidK{\OParams}{\csceneP{\omega : \OParams}{\uOmega}{b}}{\cekappa'_i}{\kappa'_i}}{\ncon}$
%       \item $\sseq{\cExpands{\uOmega}{\parseUConF{\bsubseq{b}{m'_i}{n'_i}}}{c_i}{[\omega]\kappa'_i}}{\ncon}$
%       \item $c = [\sseq{\kappa_i/k_i}{\nkind}, \sseq{c_i/u_i}{\ncon}, \omega]c'$ for some $c'$ and fresh $\sseq{k_i}{\nkind}$ and fresh $\sseq{u_i}{\ncon}$
%       \item $\mathsf{fv}(c') \subset \sseq{k_i}{\nkind} \cup \sseq{u_i}{\ncon} \cup \domof{\Omega}$
%     \end{enumerate}
%   \item If $\cvalidK{\Omega}{\csceneP{\omega : \OParams}{\uOmega}{b}}{\cekappa}{\kappa}$
%     where $\segof{\cekappa} = \sseq{\acesplicedk{m_i}{n_i}}{\nkind} \cup \sseq{\acesplicedc{m'_i}{n'_i}{\cekappa'_i}}{\ncon}$ then 
%     \begin{enumerate}
%       \item $\sseq{\kExpands{\uOmega}{\parseUKindF{\bsubseq{b}{m_i}{n_i}}}{\kappa_i}}{\nkind}$
%       \item $\sseq{\cvalidK{\OParams}{\csceneP{\omega : \OParams}{\uOmega}{b}}{\cekappa'_i}{\kappa'_i}}{\ncon}$
%       \item $\sseq{\cExpands{\uOmega}{\parseUConF{\bsubseq{b}{m'_i}{n'_i}}}{c_i}{[\omega]\kappa'_i}}{\ncon}$
%       \item $\kappa = [\sseq{\kappa_i/k_i}{\nkind}, \sseq{c_i/u_i}{\ncon}, \omega]\kappa'$ for some $\kappa'$ and fresh $\sseq{k_i}{\nkind}$ and fresh $\sseq{u_i}{\ncon}$
%       \item $\mathsf{fv}(\kappa') \subset \sseq{k_i}{\nkind} \cup \sseq{u_i}{\ncon} \cup \domof{\Omega}$
%     \end{enumerate}
% \end{enumerate}
% \end{lemma}
% \begin{proof} % TODO this one is sloppy
%   By mutual rule induction over Rules (\ref{rules:cvalidC}) and Rules (\ref{rules:cvalidK}).
%   \begin{enumerate}
%     \item \begin{byCases}
%         \item[\text{(\ref{rule:cvalidC-subsume})}] This case follows by applying the IH.
%         \item[\text{(\ref{rule:cvalidC-var}) \textbf{through} (\ref{rule:cvalidC-stat})}] These cases follow by applying the IH, gathering together the sets of conclusions and invoking the identification convention as necessary.
%         \item[\text{(\ref{rule:cvalidC-spliced})}] Letting $\cscenev=\csceneP{\omega : \OParams}{\uOmega}{b}$, 
%           \begin{pfsteps*}
%           \item $\cec=\acesplicedc{m}{n}{\cekappa}$ \BY{assumption}
%   \item $\cvalidK{\OParams}{\cscenev}{\cekappa}{\kappa}$ \BY{assumption} \pflabel{cvalidK}
%   \item $\parseUCon{\bsubseq{b}{m}{n}}{\uc}$ \BY{assumption} \pflabel{parseUCon}
%   \item $\cExpands{\uOmega}{\uc}{c}{[\omega]\kappa}$ \BY{assumption} \pflabel{cExpands}
%   \item $\uOmega=\uOmegaEx{\uD}{\uG}{\uMctx}{\Omega_\text{app}}$ \BY{assumption} \pflabel{uOmega}
%   \item $\domof{\Omega} \cap \domof{\Omega_\text{app}} = \emptyset$ \BY{assumption} \pflabel{domof}
%   \item $\segof{\acesplicedc{m}{n}{\cekappa}} = \segof{\cekappa} \cup \{ \acesplicedc{m}{n}{\cekappa} \}$ \BY{definition} \pflabel{summaryOf1}
%   \item $\segof{\cekappa} = \sseq{\acesplicedk{m_i}{n_i}}{\nkind} \cup \sseq{\acesplicedc{m'_i}{n'_i}{\cekappa'_i}}{\ncon - 1}$ \BY{definition} \pflabel{summaryOf2}
%       \item $\sseq{\kExpands{\uOmega}{\parseUKindF{\bsubseq{b}{m_i}{n_i}}}{\kappa_i}}{\nkind}$ \BY{IH, part 2 on \pfref{cvalidK} and \pfref{summaryOf2}} \pflabel{puk}
%       \item $\sseq{\cvalidK{\OParams}{\csceneP{\omega : \OParams}{\uOmega}{b}}{\cekappa'_i}{\kappa'_i}}{{\ncon - 1}}$ \BY{IH, part 2 on \pfref{cvalidK} and \pfref{summaryOf2}} \pflabel{cvalidK2}
%       \item $\sseq{\cExpands{\uOmega}{\parseUConF{\bsubseq{b}{m'_i}{n'_i}}}{c_i}{[\omega]\kappa'_i}}{{\ncon - 1}}$ \BY{IH, part 2 on \pfref{cvalidK} and \pfref{summaryOf2}} \pflabel{cExpands2}
%       % \item $\kappa = [\sseq{\kappa_i/k_i}{\nkind}, \sseq{c_i/u_i}{{\ncon - 1}}, \omega]\kappa'$ for some $\kappa'$ and fresh $\sseq{k_i}{\nkind}$ and fresh $\sseq{u_i}{\ncon}$ \BY{IH, part 2 on \pfref{cvalidK} and \pfref{summaryOf2}}
%       % \item $\mathsf{fv}(\kappa') \subset \sseq{k_i}{\nkind} \cup \sseq{u_i}{{\ncon - 1}} \cup \domof{\Omega}$ \BY{IH, part 2 on \pfref{cvalidK} and \pfref{summaryOf2}}
%           \end{pfsteps*}
%           The conclusions hold as follows:
%           \begin{enumerate}
%             \item \pfref{puk}
%             \item \pfref{cvalidK} and \pfref{cvalidK2}
%             \item \pfref{parseUCon}, \pfref{cExpands} and \pfref{cExpands2}
%             \item Choose $c'=u$ for fresh $u$. Then $c=[\omega', c/u]u$ for any $\omega'$.
%             \item $\mathsf{fv}({u})=u$ and $u \subset \{u\} \cup \sseq{k_i}{\nkind} \cup \sseq{u_i}{{\ncon - 1}} \cup \domof{\Omega}$ by definition.
%           \end{enumerate}
%           \resetpfcounter
%       \end{byCases}
%     \item \begin{byCases}
%         \item[\text{(\ref{rule:cvalidK-darr}) \textbf{through} (\ref{rule:cvalidK-sing})}] These cases follow by applying the IH and gathering together the sets of conclusions, invoking the identification convention as necessary.
%         \item[\text{(\ref{rule:cvalidK-spliced})}] Letting $\cscenev=\csceneP{\omega : \OParams}{\uOmega}{b}$ 
%           \begin{pfsteps*}
%             \item $\cekappa = \acesplicedk{m}{n}$ \BY{assumption}
%             \item $\parseUKind{\bsubseq{b}{m}{n}}{\ukappa}$ \BY{assumption} \pflabel{parse}
%             \item $\kExpands{\uOmega}{\ukappa}{\kappa}$ \BY{assumption} \pflabel{ex}
%             \item $\uOmega=\uOmegaEx{\uD}{\uG}{\uMctx}{\Omega_\text{app}}$ \BY{assumption}
%             \item $\domof{\Omega} \cap \domof{\Omega_\text{app}} = \emptyset$ \BY{assumption}
%             \item $\ncon=0$ \BY{definition of summary} \pflabel{ncon}
%           \end{pfsteps*}
%           The conclusions hold as follows:
%           \begin{enumerate}
%             \item \pfref{parse} and \pfref{ex}
%             \item \pfref{ncon}
%             \item \pfref{ncon}
%             \item Choose $\kappa'=k$ for fresh $k$. Then $\kappa=[\omega', \kappa/k]k$ for any $\omega'$.
%             \item $\mathsf{fv}(k)=k$ and $k \subset \{k\} \cup \domof{\Omega}$ by definition.
%           \end{enumerate}
%           \resetpfcounter
%         \end{byCases}
%   \end{enumerate}
% \end{proof}

% \begin{lemma}[Proto-Expression and Proto-Rule Decomposition]\label{lemma:proto-expr-decomp}
% ~
% \begin{enumerate}
%   \item If $\cvalidEP{\Omega}{\esceneP{\omega : \OParams}{\uOmega}{\uPsi}{\uPhi}{b}}{\ce}{e}{\tau_\text{proto}}$
%     where $\segof{\ce} = \sseq{\acesplicedk{m_i}{n_i}}{\nkind} \cup \sseq{\acesplicedc{m'_i}{n'_i}{\cekappa'_i}}{\ncon} \cup \sseq{\acesplicede{m''_i}{n''_i}{\ctau_i}}{\nexp}$ then 
%     \begin{enumerate}
%       \item $\sseq{\kExpands{\uOmega}{\parseUKindF{\bsubseq{b}{m_i}{n_i}}}{\kappa_i}}{\nkind}$
%       \item $\sseq{\cvalidK{\OParams}{\csceneP{\omega : \OParams}{\uOmega}{b}}{\cekappa'_i}{\kappa'_i}}{\ncon}$
%       \item $\sseq{\cExpands{\uOmega}{\parseUConF{\bsubseq{b}{m'_i}{n'_i}}}{c_i}{[\omega]\kappa'_i}}{\ncon}$
%       \item $\sseq{\cvalidC{\OParams}{\csceneP{\omega : \OParams}{\uOmega}{b}}{\ctau_i}{\tau_i}{\akty}}{\nexp}$
%       \item $\sseq{\expandsP{\uOmega}{\uPsi}{\uPhi}{\parseUExpF{\bsubseq{b}{m''_i}{n''_i}}}{e_i}{[\omega]\tau_i}}{\nexp}$
%       \item $e = [\sseq{\kappa_i/k_i}{\nkind}, \sseq{c_i/u_i}{\ncon}, \sseq{e_i/x_i}{\nexp}, \omega]e''$ for some $e''$ and fresh $\sseq{k_i}{\nkind}$ and fresh $\sseq{u_i}{\ncon}$ and fresh $\sseq{x_i}{\nexp}$
%       \item $\mathsf{fv}(e'') \subset \sseq{k_i}{\nkind} \cup \sseq{u_i}{\ncon} \cup \sseq{x_i}{\nexp} \cup \domof{\Omega}$
%     \end{enumerate}
%   \item If $\cvalidRP{\Omega}{\esceneP{\omega : \OParams}{\uOmega}{\uPsi}{\uPhi}{b}}{\crv}{r}{\tau}{\tau'}$
%     where $\segof{\crv} = \sseq{\acesplicedk{m_i}{n_i}}{\nkind} \cup \sseq{\acesplicedc{m'_i}{n'_i}{\cekappa'_i}}{\ncon} \cup \sseq{\acesplicede{m''_i}{n''_i}{\ctau_i}}{\nexp}$ then 
%     \begin{enumerate}
%       \item $\sseq{\kExpands{\uOmega}{\parseUKindF{\bsubseq{b}{m_i}{n_i}}}{\kappa_i}}{\nkind}$
%       \item $\sseq{\cvalidK{\OParams}{\csceneP{\omega : \OParams}{\uOmega}{b}}{\cekappa'_i}{\kappa'_i}}{\ncon}$
%       \item $\sseq{\cExpands{\uOmega}{\parseUConF{\bsubseq{b}{m'_i}{n'_i}}}{c_i}{[\omega]\kappa'_i}}{\ncon}$
%       \item $\sseq{\cvalidC{\OParams}{\csceneP{\omega : \OParams}{\uOmega}{b}}{\ctau_i}{\tau_i}{\akty}}{\nexp}$
%       \item $\sseq{\expandsP{\uOmega}{\uPsi}{\uPhi}{\parseUExpF{\bsubseq{b}{m''_i}{n''_i}}}{e_i}{[\omega]\tau_i}}{\nexp}$
%       \item $r = [\sseq{\kappa_i/k_i}{\nkind}, \sseq{c_i/u_i}{\ncon}, \sseq{e_i/x_i}{\nexp}, \omega]r'$ for some $r'$ and fresh $\sseq{k_i}{\nkind}$ and fresh $\sseq{u_i}{\ncon}$ and fresh $\sseq{x_i}{\nexp}$
%       \item $\mathsf{fv}(r') \subset \sseq{k_i}{\nkind} \cup \sseq{u_i}{\ncon} \cup \sseq{x_i}{\nexp} \cup \domof{\Omega}$
%     \end{enumerate}
% \end{enumerate}
% \end{lemma} % TODO this is sloppy too
% \begin{proof} By mutual rule induction over Rules (\ref{rules:cvalidE-P}) and Rule (\ref{rule:cvalidR-P}).
%   \begin{enumerate}
%     \item \begin{byCases}
%       \item[\text{(\ref{rule:cvalidE-P-subsume})}] This case follows by applying the IH.
%       \item[\text{(\ref{rule:cvalidE-P-var}) \textbf{through} (\ref{rule:cvalidE-P-mval})}] These cases follow by applying the IH or Lemma \ref{lemma:proto-con-decomp}, gathering together the sets of conclusions and invoking the identification convention as necessary. 
%       \item[\text{(\ref{rule:cvalidE-P-splicede})}] Letting $\escenev = \esceneP{\omega : \OParams}{\uOmega}{\uPsi}{\uPhi}{b}$,    
%       \begin{pfsteps*}
%         \item $\ce = \acesplicede{m}{n}{\ctau}$ \BY{assumption}
%         \item $\cvalidC{\OParams}{\csfrom{\escenev}}{\ctau}{\tau}{\akty}$ \BY{assumption} \pflabel{cvalidC}
%         \item $\parseUExp{\bsubseq{b}{m}{n}}{\ue}$ \BY{assumption} \pflabel{parse}
%         \item $\expandsP{\uOmega}{\uPsi}{\uPhi}{\ue}{e}{[\omega]\tau}$ \BY{assumption} \pflabel{expandsP}
%         \item $\uOmega=\uOmegaEx{\uD}{\uG}{\uMctx}{\Omega_\text{app}}$ \BY{assumption}
%         \item $\domof{\Omega} \cap \domof{\Omega_\text{app}} = \emptyset$ \BY{assumption}
%         \item $\segof{\ce} = \segof{\ctau} \cup \{ \acesplicede{m}{n}{\ctau} \}$ \BY{assumption}
%         \item $\segof{\ctau} = \sseq{\acesplicedk{m_i}{n_i}}{\nkind} \cup \sseq{\acesplicedc{m'_i}{n'_i}{\cekappa'_i}}{\ncon}$  \BY{definition} \pflabel{sumC}
%         \item $\sseq{\kExpands{\uOmega}{\parseUKindF{\bsubseq{b}{m_i}{n_i}}}{\kappa_i}}{\nkind}$ \BY{Lemma \ref{lemma:proto-con-decomp} on \pfref{cvalidC} and \pfref{sumC}} \pflabel{A}
%         \item $\sseq{\cvalidK{\OParams}{\csceneP{\omega : \OParams}{\uOmega}{b}}{\cekappa'_i}{\kappa'_i}}{\ncon}$ \BY{Lemma \ref{lemma:proto-con-decomp} on \pfref{cvalidC} and \pfref{sumC}} \pflabel{B}
%         \item $\sseq{\cExpands{\uOmega}{\parseUConF{\bsubseq{b}{m'_i}{n'_i}}}{c_i}{[\omega]\kappa'_i}}{\ncon}$ \BY{Lemma \ref{lemma:proto-con-decomp} on \pfref{cvalidC} and \pfref{sumC}} \pflabel{C}
%       \end{pfsteps*}
%       The conclusions hold as follows:
%       \begin{enumerate}
%         \item \pfref{A}
%         \item \pfref{B}
%         \item \pfref{C}
%         \item \pfref{cvalidC}
%         \item \pfref{parse} and \pfref{expandsP}
%             \item Choose $e''=x$ for fresh $x$. Then $e=[\omega', e/x]x$ for any $\omega'$.
%             \item $\mathsf{fv}({x})=x$ and $x \subset \{x\} \cup \sseq{k_i}{\nkind} \cup \sseq{u_i}{{\ncon}} \cup \domof{\Omega}$ by definition.
%       \end{enumerate}
%       \resetpfcounter 
%     \end{byCases}
%     \item There is only one case.
%     \begin{byCases}
%       \item[\text{(\ref{rule:cvalidR-P})}] ~
%         \begin{pfsteps*}
%           \item $\crv=\acematchrule{p}{e}$ \BY{assumption}
%           \item $r=\matchrule{p}{e}$ \BY{assumption}
%           \item $\patTypeP{\Omega'}{p}{\tau}$ \BY{assumption} \pflabel{pattype}
%           \item $\cvalidEP{\Gcons{\Omega}{\Omega'}}{\escenev}{\ce}{e}{\tau'}$ \BY{assumption} \pflabel{cvalidEP}
%           \item $\segof{\crv} = \segof{e}$ \BY{definition} \pflabel{summ}
%           \item $\sseq{\kExpands{\uOmega}{\parseUKindF{\bsubseq{b}{m_i}{n_i}}}{\kappa_i}}{\nkind}$ \BY{IH, part 1 on \pfref{cvalidEP} and \pfref{summ}} \pflabel{c1}
%           \item $\sseq{\cvalidK{\OParams}{\csceneP{\omega : \OParams}{\uOmega}{b}}{\cekappa'_i}{\kappa'_i}}{\ncon}$ \BY{IH, part 1 on \pfref{cvalidEP} and \pfref{summ}} \pflabel{c2}
%           \item $\sseq{\cExpands{\uOmega}{\parseUConF{\bsubseq{b}{m'_i}{n'_i}}}{c_i}{[\omega]\kappa'_i}}{\ncon}$ \BY{IH, part 1 on \pfref{cvalidEP} and \pfref{summ}} \pflabel{c3}
%           \item $\sseq{\cvalidC{\OParams}{\csceneP{\omega : \OParams}{\uOmega}{b}}{\ctau_i}{\tau_i}{\akty}}{\nexp}$ \BY{IH, part 1 on \pfref{cvalidEP} and \pfref{summ}} \pflabel{c4}
%           \item $\sseq{\expandsP{\uOmega}{\uPsi}{\uPhi}{\parseUExpF{\bsubseq{b}{m''_i}{n''_i}}}{e_i}{[\omega]\tau_i}}{\nexp}$ \BY{IH, part 1 on \pfref{cvalidEP} and \pfref{summ}} \pflabel{c5}
%           \item $e = [\sseq{\kappa_i/k_i}{\nkind}, \sseq{c_i/u_i}{\ncon}, \sseq{e_i/x_i}{\nexp}, \omega]e''$ for some $e''$ and fresh $\sseq{k_i}{\nkind}$ and fresh $\sseq{u_i}{\ncon}$ and fresh $\sseq{x_i}{\nexp}$ \BY{IH, part 1 on \pfref{cvalidEP} and \pfref{summ}} \pflabel{c6}
%           \item $\mathsf{fv}(e'') \subset \sseq{k_i}{\nkind} \cup \sseq{u_i}{\ncon} \cup \sseq{x_i}{\nexp} \cup \domof{\Omega \cup \Omega'}$ \BY{IH, part 1 on \pfref{cvalidEP} and \pfref{summ}} \pflabel{c7}
%           \item $\mathsf{fv}(e'') \subset \sseq{k_i}{\nkind} \cup \sseq{u_i}{\ncon} \cup \sseq{x_i}{\nexp} \cup \domof{\Omega} \cup \domof{\Omega'}$ \BY{distributivity of union} \pflabel{c7d}
%           \item $r=[\sseq{\kappa_i/k_i}{\nkind}, \sseq{c_i/u_i}{\ncon}, \sseq{e_i/x_i}{\nexp}, \omega]\matchrule{p}{e''}$ \BY{definition of substitution} \pflabel{req}
%           \item $\domof{\Omega'} = \mathsf{patvars}(p)$ \BY{Lemma \ref{lemma:pattern-binding} on \pfref{pattype}} \pflabel{domeq}
%           \item $\mathsf{fv}(\matchrule{p}{e''}) \subset \sseq{k_i}{\nkind} \cup \sseq{u_i}{\ncon} \cup \sseq{x_i}{\nexp} \cup \domof{\Omega}$ \BY{Definition of $\mathsf{fv}(r)$ and \pfref{domeq} and \pfref{c7d}} \pflabel{fvfinal}
%         \end{pfsteps*}
%     \end{byCases}
%     The conclusions hold as follows:
%     \begin{enumerate}
%       \item \pfref{c1}
%       \item \pfref{c2}
%       \item \pfref{c3}
%       \item \pfref{c4}
%       \item \pfref{c5}
%       \item Choose $r'=\matchrule{p}{e''}$, then by \pfref{req}
%       \item \pfref{fvfinal}
%     \end{enumerate}
%     \resetpfcounter
%   \end{enumerate}
% \end{proof}

% \begin{theorem}[peTLM Abstract Reasoning Principles]
% \label{thm:petsm-abstract-reasoning-principles}
% If $\expandsP{\uOmega}{\uPsi}{\uPhi}{\utsmap{\uepsilon}{b}}{e}{\tau}$ then:
% \begin{enumerate}
%   \item $\uOmega=\uOmegaEx{\uD}{\uG}{\uMctx}{\Omega_\text{app}}$
%   \item $\uPsi=\uAS{\uA}{\Psi}$
%   \item (\textbf{Typing 1}) $\tsmexpExpandsExp{\uOmega}{\uPsi}{\uepsilon}{\epsilon}{\aetype{\tau'}}$ and $\hastypeP{\Omega_\text{app}}{e}{\tau'}$ for $\tau'$ such that $\issubtypeP{\Omega_\text{app}}{\tau'}{\tau}$.
%   \item $\tsmexpEvalsExp{\Omega_\text{app}}{\Psi}{\epsilon}{\epsilon_\text{normal}}$
%   \item $\tsmdefof{\epsilon_\text{normal}}=a$
%   \item $\Psi = \Psi', \petsmdefn{a}{\rho}{\eparse}$
%   \item $\encodeBody{b}{\ebody}$
%     \item $\evalU{\ap{\eparse}{\ebody}}{\aein{\mathtt{SuccessE}}{e_\text{pproto}}}$
%   \item $\decodePCEExp{e_\text{pproto}}{\pce}$
%   \item $\prepce{\Omega_\text{app}}{\Psi}{\pce}{\ce}{\epsilon_\text{normal}}{\aetype{\tau_\text{proto}}}{\omega}{\Omega_\text{params}}$
%   \item (\textbf{Segmentation}) $\segOKP{\OParams}{\csceneP{\omega : \OParams}{\uOmega}{b}}{\segof{\ce}}{b}$
%   \item $\cvalidEP{\Omega_\text{params}}{\esceneP{\omega : \OParams}{\uOmega}{\uPsi}{\uPhi}{b}}{\ce}{e'}{\tau_\text{proto}}$
%   \item $e = [\omega]e'$
%   \item $\tau' = [\omega]\tau_\text{proto}$
%   \item $
%     \segof{\ce} = \sseq{\acesplicedk{m_i}{n_i}}{\nkind} \cup \sseq{\acesplicedc{m'_i}{n'_i}{\cekappa'_i}}{\ncon} \cup\\
%                \sseq{\acesplicede{m''_i}{n''_i}{\ctau_i}}{\nexp}
%     $
%   \item (\textbf{Kinding 1}) $\sseq{\kExpands{\uOmega}{\parseUKindF{\bsubseq{b}{m_i}{n_i}}}{\kappa_i}}{\nkind}$ and $\sseq{\iskind{\Omega_\text{app}}{\kappa_i}}{\nkind}$
%   \item (\textbf{Kinding 2}) $\sseq{\cvalidK{\OParams}{\csceneP{\omega : \OParams}{\uOmega}{b}}{\cekappa'_i}{\kappa'_i}}{\ncon}$ and $\sseq{\iskind{\Omega_\text{app}}{[\omega]\kappa'_i}}{\ncon}$
%   \item (\textbf{Kinding 3}) $\sseq{\cExpands{\uOmega}{\parseUConF{\bsubseq{b}{m'_i}{n'_i}}}{c_i}{[\omega]\kappa'_i}}{\ncon}$ and $\sseq{\haskind{\Omega_\text{app}}{c_i}{[\omega]\kappa'_i}}{\ncon}$
%   \item (\textbf{Kinding 4}) $\sseq{\cvalidC{\OParams}{\csceneP{\omega : \OParams}{\uOmega}{b}}{\ctau_i}{\tau_i}{\akty}}{\nexp}$ and $\sseq{\haskind{\Omega_\text{app}}{[\omega]\tau_i}{\akty}}{\nexp}$
%   \item (\textbf{Typing 2}) $\sseq{\expandsP{\uOmega}{\uPsi}{\uPhi}{\parseUExpF{\bsubseq{b}{m''_i}{n''_i}}}{e_i}{[\omega]\tau_i}}{\nexp}$ and $\sseq{\hastypeP{\Omega_\text{app}}{e_i}{[\omega]\tau_i}}{\nexp}$
%   \item (\textbf{Capture Avoidance}) $e = [\sseq{\kappa_i/k_i}{\nkind}, \sseq{c_i/u_i}{\ncon}, \sseq{e_i/x_i}{\nexp}, \omega]e''$ for some $e''$ and fresh $\sseq{k_i}{\nkind}$ and fresh $\sseq{u_i}{\ncon}$ and fresh $\sseq{x_i}{\nexp}$
%   \item (\textbf{Context Independence}) \[\mathsf{fv}(e'') \subset \sseq{k_i}{\nkind} \cup \sseq{u_i}{\ncon} \cup \sseq{x_i}{\nexp} \cup \domof{\OParams}\]
%   % $\hastypeP{\sseq{\Khyp{k_i}}{\nkind} \cup \sseq{u_i :: [\omega]\kappa'_i}{\ncon} \cup \sseq{x_i : [\omega]\tau_i}{\nexp}}{[\omega]e''}{\tau}$\todo{maybe restate this in terms of free variables of e'' here and elsewhere, because context isn't technically well-formed here?}
% \end{enumerate}
% \end{theorem}
% \begin{proof} By rule induction over Rules (\ref{rules:expandsP}). There are two rules that apply.
% \begin{byCases}
%   \item[\text{(\ref{rule:expandsP-subsume})}] ~
%     \begin{pfsteps*}
%       \item $\expandsPX{\utsmap{\uepsilon}{b}}{e}{\tau'}$ \BY{assumption} \pflabel{expands}
%       \item $\issubtypePX{\tau'}{\tau}$ \BY{assumption} \pflabel{issubtype1}
%       \item $\uOmega=\uOmegaEx{\uD}{\uG}{\uMctx}{\Omega_\text{app}}$ \BY{IH on \pfref{expands}} \pflabel{A}
%       \item $\uPsi=\uAS{\uA}{\Psi}$ \BY{IH on \pfref{expands}} \pflabel{B}
%       \item $\tsmexpExpandsExp{\uOmega}{\uPsi}{\uepsilon}{\epsilon}{\aetype{\tau''}}$ and $\hastypeP{\Omega_\text{app}}{e}{\tau''}$ for $\tau''$ such that $\issubtypeP{\Omega_\text{app}}{\tau''}{\tau'}$. \BY{IH on \pfref{expands}} \pflabel{C}
%       \item $\tsmexpEvalsExp{\Omega_\text{app}}{\Psi}{\epsilon}{\epsilon_\text{normal}}$ \BY{IH on \pfref{expands}} \pflabel{D}
%       \item $\tsmdefof{\epsilon_\text{normal}}=a$ \BY{IH on \pfref{expands}} \pflabel{E}
%       \item $\Psi = \Psi', \petsmdefn{a}{\rho}{\eparse}$ \BY{IH on \pfref{expands}} \pflabel{F}
%       \item $\encodeBody{b}{\ebody}$ \BY{IH on \pfref{expands}} \pflabel{G}
%         \item $\evalU{\ap{\eparse}{\ebody}}{\aein{\mathtt{SuccessE}}{e_\text{pproto}}}$ \BY{IH on \pfref{expands}} \pflabel{H}
%       \item $\decodePCEExp{e_\text{pproto}}{\pce}$ \BY{IH on \pfref{expands}} \pflabel{I}
%       \item $\prepce{\Omega_\text{app}}{\Psi}{\pce}{\ce}{\epsilon_\text{normal}}{\aetype{\tau_\text{proto}}}{\omega}{\Omega_\text{params}}$ \BY{IH on \pfref{expands}} \pflabel{J}
%       \item $\segOKP{\OParams}{\csceneP{\omega : \OParams}{\uOmega}{b}}{\segof{\ce}}{b}$ \BY{IH on \pfref{expands}} \pflabel{K}
%       \item $\cvalidEP{\Omega_\text{params}}{\esceneP{\omega : \OParams}{\uOmega}{\uPsi}{\uPhi}{b}}{\ce}{e'}{\tau_\text{proto}}$ \BY{IH on \pfref{expands}} \pflabel{L}
%       \item $e = [\omega]e'$ \BY{IH on \pfref{expands}} \pflabel{M}
%       \item $\tau' = [\omega]\tau_\text{proto}$ \BY{IH on \pfref{expands}} \pflabel{N}
%       \item $
%         \segof{\ce} = \sseq{\acesplicedk{m_i}{n_i}}{\nkind} \cup \sseq{\acesplicedc{m'_i}{n'_i}{\cekappa'_i}}{\ncon} \cup 
%                    \sseq{\acesplicede{m''_i}{n''_i}{\ctau_i}}{\nexp}
%         $ \BY{IH on \pfref{expands}} \pflabel{O}
%       \item $\sseq{\kExpands{\uOmega}{\parseUKindF{\bsubseq{b}{m_i}{n_i}}}{\kappa_i}}{\nkind}$ and $\sseq{\iskind{\Omega_\text{app}}{\kappa_i}}{\nkind}$ \BY{IH on \pfref{expands}} \pflabel{P}
%       \item $\sseq{\cvalidK{\OParams}{\csceneP{\omega : \OParams}{\uOmega}{b}}{\cekappa'_i}{\kappa'_i}}{\ncon}$ and $\sseq{\iskind{\Omega_\text{app}}{[\omega]\kappa'_i}}{\ncon}$ \BY{IH on \pfref{expands}} \pflabel{Q}
%       \item $\sseq{\cExpands{\uOmega}{\parseUConF{\bsubseq{b}{m'_i}{n'_i}}}{c_i}{[\omega]\kappa'_i}}{\ncon}$ and $\sseq{\haskind{\Omega_\text{app}}{c_i}{[\omega]\kappa'_i}}{\ncon}$ \BY{IH on \pfref{expands}} \pflabel{R}
%       \item $\sseq{\cvalidC{\OParams}{\csceneP{\omega : \OParams}{\uOmega}{b}}{\ctau_i}{\tau_i}{\akty}}{\nexp}$ and $\sseq{\haskind{\Omega_\text{app}}{[\omega]\tau_i}{\akty}}{\nexp}$ \BY{IH on \pfref{expands}} \pflabel{S}
%       \item $\sseq{\expandsP{\uOmega}{\uPsi}{\uPhi}{\parseUExpF{\bsubseq{b}{m''_i}{n''_i}}}{e_i}{[\omega]\tau_i}}{\nexp}$ and $\sseq{\hastypeP{\Omega_\text{app}}{e_i}{[\omega]\tau_i}}{\nexp}$ \BY{IH on \pfref{expands}} \pflabel{T}
%       \item $e = [\sseq{\kappa_i/k_i}{\nkind}, \sseq{c_i/u_i}{\ncon}, \sseq{e_i/x_i}{\nexp}, \omega]e''$ for some $e''$ and fresh $\sseq{k_i}{\nkind}$ and fresh $\sseq{u_i}{\ncon}$ and fresh $\sseq{x_i}{\nexp}$ \BY{IH on \pfref{expands}} \pflabel{U}
%       \item $\mathsf{fv}(e'') \subset \sseq{k_i}{\nkind} \cup \sseq{u_i}{\ncon} \cup \sseq{x_i}{\nexp} \cup \domof{\OParams}$ \BY{IH on \pfref{expands}} \pflabel{V}
%       \item $\issubtypeP{\Omega_\text{app}}{\tau''}{\tau}$ \BY{Rule (\ref{rule:issubtypeP-trans}) on \pfref{issubtype1} and \pfref{C}} \pflabel{issubtype2}
%     \end{pfsteps*}
%     The conclusions hold as follows:
%     \begin{enumerate}
%       \item \pfref{A}
%       \item \pfref{B}
%       \item Choosing $\tau''$, by \pfref{C} and \pfref{issubtype2}
%       \item \pfref{D}
%       \item \pfref{E}
%       \item \pfref{F}
%       \item \pfref{G}
%       \item \pfref{H}
%       \item \pfref{I}
%       \item \pfref{J}
%       \item \pfref{K}
%       \item \pfref{L}
%       \item \pfref{M}
%       \item \pfref{N}
%       \item \pfref{O}
%       \item \pfref{P}
%       \item \pfref{Q}
%       \item \pfref{R}
%       \item \pfref{S}
%       \item \pfref{T}
%       \item \pfref{U}
%       \item \pfref{V}
%     \end{enumerate}
%     \resetpfcounter
%   \item[\text{(\ref{rule:expandsP-apuetsm})}] ~
%     \begin{pfsteps*}
%       \item $e=[\omega]e'$ \BY{assumption} \pflabel{e}
%       \item $\tau=[\omega]\tau_\text{proto}$ \BY{assumption} \pflabel{tau}
%       \item $\uOmega = \uOmegaEx{\uD}{\uG}{\uMctx}{\Omega_\text{app}}$ \BY{assumption} \pflabel{uomega}
%       \item $\uPsi=\uAS{\uA}{\Psi}$ \BY{assumption} \pflabel{upsi}
%       \item $\tsmexpExpandsExp{\uOmega}{\uPsi}{\uepsilon}{\epsilon}{\aetype{\tau_\text{final}}}$ \BY{assumption} \pflabel{tsmexpExpands}
%       \item $\tsmexpEvalsExp{\Omega_\text{app}}{\Psi}{\epsilon}{\epsilon_\text{normal}}$ \BY{assumption} \pflabel{tsmexpEvals}
%       \item $\tsmdefof{\epsilon_\text{normal}}=a$ \BY{assumption} \pflabel{tsmdefof}
%       \item $\Psi = \Psi', \petsmdefn{a}{\rho}{\eparse}$ \BY{assumption} \pflabel{Psi}
%       \item $\encodeBody{b}{\ebody}$ \BY{assumption} \pflabel{encodeBody}
%       \item $\evalU{\ap{\eparse}{\ebody}}{\aein{\mathtt{SuccessE}}{e_\text{pproto}}}$ \BY{assumption} \pflabel{evalU}
%       \item $\decodePCEExp{e_\text{pproto}}{\pce}$ \BY{assumption} \pflabel{decode}
%       \item $\prepce{\Omega_\text{app}}{\Psi}{\pce}{\ce}{\epsilon_\text{normal}}{\aetype{\tau_\text{proto}}}{\omega}{\Omega_\text{params}}$ \BY{assumption} \pflabel{prepce}
%       \item $\segOKP{\OParams}{\csceneP{\omega : \OParams}{\uOmega}{b}}{\segof{\ce}}{b}$ \BY{assumption} \pflabel{segOK}
%       \item $\cvalidEP{\Omega_\text{params}}{\esceneP{\omega : \OParams}{\uOmega}{\uPsi}{\uPhi}{b}}{\ce}{e'}{\tau_\text{proto}}$ \BY{assumption} \pflabel{cvalid}
%       \item $\hastsmtypeExp{\Omega_\text{app}}{\Psi}{\epsilon}{\aetype{\tau_\text{final}}}$ \BY{Theorem \ref{thm:peTLM-expression-expansion} on \pfref{tsmexpExpands}} \pflabel{hastsmtype}
%       \item $\hastsmtypeExp{\Omega_\text{app}}{\Psi}{\epsilon_\text{normal}}{\aetype{\tau_\text{final}}}$ \BY{Corollary \ref{thm:peTLM-preservation-evaluation} on \pfref{hastsmtype} and \pfref{tsmexpEvals}} \pflabel{hastsmtypeNormal}
%       \item $\hastsmtypeExp{\Omega_\text{app}}{\Psi}{\epsilon_\text{normal}}{[\omega]\aetype{\tau_\text{proto}}}$ \BY{Lemma \ref{lemma:expression-deparameterization-P} on \pfref{prepce}} \pflabel{hastsmtypeNormal2}
%       \item $\aetype{\tau_\text{final}} = [\omega]\aetype{\tau_\text{proto}}$ \BY{Theorem \ref{lemma:peTLM-unicity} on \pfref{hastsmtypeNormal} and \pfref{hastsmtypeNormal2}} \pflabel{tsmtypeseq}
%       \item $\tau_\text{final}=[\omega]\tau_\text{proto}=\tau$ \BY{definition of substitution and \pfref{tsmtypeseq} and \pfref{tau}} \pflabel{feqp}
%       \item $\istsmty{\Omega_\text{app}}{\aetype{\tau_\text{final}}}$ \BY{Lemma \ref{lemma:peTLM-regularity} on \pfref{hastsmtype}} \pflabel{istsmty}
%       \item $\haskind{\Omega_\text{app}}{\tau}{\akty}$ \BY{Inversion of Rule (\ref{rule:istsmty-type}) on \pfref{istsmty}} \pflabel{istype}
%       \item $\cequal{\Omega_\text{app}}{\tau}{\tau}{\akty}$ \BY{Rule (\ref{rule:cequal-refl}) on \pfref{istype}} \pflabel{refl}
%       \item $\issubtypeP{\Omega_\text{app}}{\tau}{\tau}$ \BY{Rule (\ref{rule:issubtypeP-equal}) on \pfref{refl}} \pflabel{subty}
%       \item  $\expandsP{\uOmega}{\uPsi}{\uPhi}{\utsmap{\uepsilon}{b}}{e}{\tau}$ \BY{assumption} \pflabel{expands}
%       \item $\hastypeP{\Omega_\text{app}}{e}{\tau}$ \BY{Theorem \ref{thm:typed-expression-expansion-P} on \pfref{expands}} \pflabel{hastype}
%       \item $\hastypeP{\Omega_\text{app}}{\omega}{\Omega_\text{params}}$ \BY{Lemma \ref{lemma:expression-deparameterization-P} on \pfref{prepce}} \pflabel{omegaType}
%       \item $
%     \segof{\ce} = \sseq{\acesplicedk{m_i}{n_i}}{\nkind} \cup \sseq{\acesplicedc{m'_i}{n'_i}{\cekappa'_i}}{\ncon} \cup
%                \sseq{\acesplicede{m''_i}{n''_i}{\ctau_i}}{\nexp}
%     $ \BY{definition} \pflabel{summaryOf}
%        \item $\sseq{\kExpands{\uOmega}{\parseUKindF{\bsubseq{b}{m_i}{n_i}}}{\kappa_i}}{\nkind}$ \BY{Lemma \ref{lemma:proto-expr-decomp} on \pfref{cvalid} and \pfref{summaryOf}} \pflabel{A}
%        \item $\sseq{\iskind{\Omega_\text{app}}{\kappa_i}}{\nkind}$ \BY{Theorem \ref{thm:kind-and-constructor-expansion-P} over \pfref{A}} \pflabel{A2}
%       \item $\sseq{\cvalidK{\OParams}{\csceneP{\omega : \OParams}{\uOmega}{b}}{\cekappa'_i}{\kappa'_i}}{\ncon}$ \BY{Lemma \ref{lemma:proto-expr-decomp} on \pfref{cvalid} and \pfref{summaryOf}} \pflabel{B}
%       \item $\sseq{\iskind{\Omega_\text{app} \cup \OParams}{\kappa'_i}}{\ncon}$ \BY{Theorem \ref{thm:kind-and-constructor-expansion-P} over \pfref{B}} \pflabel{B2}
%       \item $\sseq{\iskind{\Omega_\text{app}}{[\omega]\kappa'_i}}{\ncon}$ \BY{Lemma \ref{lemma:substitution-P} over \pfref{B2} and \pfref{omegaType}} \pflabel{B3}
%       \item $\sseq{\cExpands{\uOmega}{\parseUConF{\bsubseq{b}{m'_i}{n'_i}}}{c_i}{[\omega]\kappa'_i}}{\ncon}$ \BY{Lemma \ref{lemma:proto-expr-decomp} on \pfref{cvalid} and \pfref{summaryOf}} \pflabel{C}
%       \item $\sseq{\haskind{\Omega_\text{app}}{c_i}{[\omega]\kappa'_i}}{\ncon}$ \BY{Theorem \ref{thm:kind-and-constructor-expansion-P} over \pfref{C}} \pflabel{C2}
%       \item $\sseq{\cvalidC{\OParams}{\csceneP{\omega : \OParams}{\uOmega}{b}}{\ctau_i}{\tau_i}{\akty}}{\nexp}$ \BY{Lemma \ref{lemma:proto-expr-decomp} on \pfref{cvalid} and \pfref{summaryOf}} \pflabel{D}
%       \item $\sseq{\haskind{\Omega_\text{app} \cup \OParams}{\tau_i}{\akty}}{\nexp}$ \BY{Theorem \ref{thm:kind-and-constructor-expansion-P} over \pfref{D}} \pflabel{D2}
%       \item $\sseq{\haskind{\Omega_\text{app}}{[\omega]\tau_i}{\akty}}{\nexp}$ \BY{Lemma \ref{lemma:substitution-P} over \pfref{D2} and \pfref{omegaType}} \pflabel{D3}
%       \item $\sseq{\expandsP{\uOmega}{\uPsi}{\uPhi}{\parseUExpF{\bsubseq{b}{m''_i}{n''_i}}}{e_i}{[\omega]\tau_i}}{\nexp}$ \BY{Lemma \ref{lemma:proto-expr-decomp} on \pfref{cvalid} and \pfref{summaryOf}} \pflabel{E}
%       \item $\sseq{\hastypeP{\Omega_\text{app}}{e_i}{[\omega]\tau_i}}{\nexp}$ \BY{Theorem \ref{thm:typed-expression-expansion-P} over \pfref{E}} \pflabel{E2}
%       \item $e = [\sseq{\kappa_i/k_i}{\nkind}, \sseq{c_i/u_i}{\ncon}, \sseq{e_i/x_i}{\nexp}, \omega]e''$ for some $e''$ and fresh $\sseq{k_i}{\nkind}$ and fresh $\sseq{u_i}{\ncon}$ and fresh $\sseq{x_i}{\nexp}$ \BY{Lemma \ref{lemma:proto-expr-decomp} on \pfref{cvalid} and \pfref{summaryOf}} \pflabel{F}
%       \item $\mathsf{fv}(e'') \subset \sseq{k_i}{\nkind} \cup \sseq{u_i}{\ncon} \cup \sseq{x_i}{\nexp} \cup \domof{\OParams}$ \BY{Lemma \ref{lemma:proto-expr-decomp} on \pfref{cvalid} and \pfref{summaryOf}} \pflabel{G}
%     \end{pfsteps*}
%     The conclusions hold as follows:
%     \begin{enumerate}
%      \item \pfref{uomega}
%      \item \pfref{upsi}
%      \item Choosing $\tau'$, by \pfref{tsmexpExpands}, \pfref{feqp}, \pfref{hastype} and \pfref{subty}
%      \item \pfref{tsmexpEvals}
%      \item \pfref{tsmdefof}
%      \item \pfref{Psi}
%      \item \pfref{encodeBody}
%      \item \pfref{evalU}
%      \item \pfref{decode}
%      \item \pfref{prepce}
%      \item \pfref{segOK}
%      \item \pfref{cvalid}
%      \item \pfref{e}
%      \item \pfref{tau}
%      \item \pfref{summaryOf}
%      \item \pfref{A} and \pfref{A2}
%      \item \pfref{B} and \pfref{B3}
%      \item \pfref{C} and \pfref{C2}
%      \item \pfref{D} and \pfref{D3}
%      \item \pfref{E} and \pfref{E2}
%      \item \pfref{F}
%      \item \pfref{G}
%     \end{enumerate}
%     \resetpfcounter
% \end{byCases}

%  \end{proof}

% \begin{lemma}[Proto-Pattern Decomposition]
% \label{lemma:proto-pattern-decomposition-P}
% If $\cvalidPP{\uOmega'}{\psceneP{\omega : \Omega_\text{params}}{\uOmega}{\uPhi}{b}}{\cpv}{p}{\tau}$ where \[
%   \begin{array}{lll}
%   \segof{\cpv} & = &\sseq{\acesplicedk{m_i}{n_i}}{\nkind} \\
%   & \cup & \sseq{\acesplicedc{m'_i}{n'_i}{\cekappa'_i}}{\ncon} \\
%   & \cup & \sseq{\acesplicedp{m''_i}{n''_i}{\ctau_i}}{\npat}
%   \end{array}
%   \] then 
%   \begin{enumerate}
%       \item $\sseq{\kExpands{\uOmega}{\parseUKindF{\bsubseq{b}{m_i}{n_i}}}{\kappa_i}}{\nkind}$
%       \item $\sseq{\cvalidK{\OParams}{\csceneP{\omega : \OParams}{\uOmega}{b}}{\cekappa'_i}{\kappa'_i}}{\ncon}$
%       \item $\sseq{\cExpands{\uOmega}{\parseUConF{\bsubseq{b}{m'_i}{n'_i}}}{c_i}{[\omega]\kappa'_i}}{\ncon}$
%       \item $\sseq{\cvalidC{\OParams}{\csceneP{\omega : \OParams}{\uOmega}{b}}{\ctau_i}{\tau_i}{\akty}}{\npat}$
%       \item $\sseq{\patExpandsP{\uOmegaEx{\emptyset}{\uG_i}{\emptyset}{\Omega_i}}{\uPhi}{\parseUPatF{\bsubseq{b}{m''_i}{n''_i}}}{p_i}{[\omega]\tau_i}}{\npat}$
%       \item $\uOmega'=\uOmegaEx{\emptyset}{\biguplus_{0 \leq i < \npat} \uG_i}{\emptyset}{\bigcup_{0 \leq i < \npat} \Omega_i}$
%   \end{enumerate}
% \end{lemma}
% \begin{proof} By rule induction over Rules (\ref{rules:cvalidPP}). 
% \begin{byCases}
%   \item[\text{(\ref{rule:cvalidPP-wild}) \textbf{through} (\ref{rule:cvalidPP-in})}] In each of these cases, we apply the IH to or over each premise and then gather the sets of conclusions, applying the identification convention as necessary. 
%   \item[\text{(\ref{rule:cvalidPP-spliced})}] ~
%     \begin{pfsteps*}
%       \item $\cpv = \acesplicedp{m}{n}{\ctau}$ \BY{assumption} 
%       \item $\cvalidC{\OParams}{\csceneP{\omega : \OParams}{\uOmega}{b}}{\ctau}{\tau}{\akty}$ \BY{assumption} \pflabel{cvalidC}
%       \item $\parseUPat{\bsubseq{b}{m}{n}}{\upv}$ \BY{assumption} \pflabel{parseUPat}
%       \item $\patExpandsP{\uOmega'}{\uPhi}{\upv}{p}{[\omega]\tau}$ \BY{assumption} \pflabel{patExpandsP}
%       \item $\segof{\cpv} = \segof{\ctau} \cup \{ \acesplicedp{m}{n}{\ctau} \}$ \BY{definition} \pflabel{summ}
%       \item $\segof{\ctau} = \sseq{\acesplicedk{m_i}{n_i}}{\nkind} \cup \sseq{\acesplicedc{m'_i}{n'_i}{\cekappa'_i}}{\ncon}$ \BY{definition} \pflabel{ctausumm}
%       \item $\sseq{\kExpands{\uOmega}{\parseUKindF{\bsubseq{b}{m_i}{n_i}}}{\kappa_i}}{\nkind}$ \BY{Lemma \ref{lemma:proto-con-decomp} on \pfref{cvalidC} and \pfref{ctausumm}} \pflabel{A}
%       \item $\sseq{\cvalidK{\OParams}{\csceneP{\omega : \OParams}{\uOmega}{b}}{\cekappa'_i}{\kappa'_i}}{\ncon}$ \BY{Lemma \ref{lemma:proto-con-decomp} on \pfref{cvalidC} and \pfref{ctausumm}} \pflabel{B}
%       \item $\sseq{\cExpands{\uOmega}{\parseUConF{\bsubseq{b}{m'_i}{n'_i}}}{c_i}{[\omega]\kappa'_i}}{\ncon}$ \BY{Lemma \ref{lemma:proto-con-decomp} on \pfref{cvalidC} and \pfref{ctausumm}} \pflabel{C}
%     \end{pfsteps*}
%     The conclusions hold as follows:
%     \begin{enumerate}
%       \item \pfref{A}
%       \item \pfref{B}
%       \item \pfref{C}
%       \item \pfref{cvalidC}
%       \item \pfref{parseUPat} and \pfref{patExpandsP}
%       \item \pfref{patExpandsP} because $\npat = 1$
%     \end{enumerate}
%     \resetpfcounter
% \end{byCases}

%  \end{proof}

% \begin{theorem}[ppTLM Abstract Reasoning Principles]
% \label{thm:pptsm-abstract-reasoning-principles}
% If $\patExpandsP{\uOmega'}{\uPhi}{\utsmap{\uepsilon}{b}}{p}{\tau}$ then:
% \begin{enumerate}
%   \item $\uOmega=\uOmegaEx{\uD}{\uG}{\uMctx}{\Omega_\text{app}}$
%   \item $\uPhi=\uAS{\uA}{\Phi}$
%   \item (\textbf{Typing 1}) $\tsmexpExpandsPat{\uOmega}{\uPhi}{\uepsilon}{\epsilon}{\aetype{\tau'}}$ and $\patTypePC{\Omega_\text{app}}{\uOmega'}{p}{\tau'}$ for $\tau'$ such that $\issubtypeP{\Omega_\text{app}}{\tau'}{\tau}$
%   \item $\tsmexpEvalsPat{\Omega_\text{app}}{\Phi}{\epsilon}{\epsilon_\text{normal}}$
%   \item $\tsmdefof{\epsilon_\text{normal}}=a$
%   \item $\Phi = \Phi', \pptsmdefn{a}{\rho}{\eparse}$
%   \item $\encodeBody{b}{\ebody}$
%   \item $\evalU{\ap{\eparse}{\ebody}}{\aein{\mathtt{SuccessP}}{e_\text{pproto}}}$
%   \item $\decodePCEPat{e_\text{pproto}}{\pcp}$
%   \item $\prepcp{\Omega_\text{app}}{\Phi}{\pcp}{\cpv}{\epsilon_\text{normal}}{\aetype{\tau_\text{proto}}}{\omega}{\Omega_\text{params}}$
%   \item (\textbf{Segmentation}) $\segOKP{\OParams}{\csceneP{\omega : \OParams}{\uOmega}{b}}{\segof{\cpv}}{b}$
%   \item $\cvalidPP{\uOmega'}{\psceneP{\omega : \Omega_\text{params}}{\uOmega}{\uPhi}{b}}{\cpv}{p}{\tau_\text{proto}}$
%   \item $\tau'=[\omega]\tau_\text{proto}$
%   \item $
%   \segof{\cpv} = \sseq{\acesplicedk{m_i}{n_i}}{\nkind} \cup \sseq{\acesplicedc{m'_i}{n'_i}{\cekappa'_i}}{\ncon} \cup\\
%              \sseq{\acesplicedp{m''_i}{n''_i}{\ctau_i}}{\npat}
%   $
%   \item (\textbf{Kinding 1}) $\sseq{\kExpands{\uOmega}{\parseUKindF{\bsubseq{b}{m_i}{n_i}}}{\kappa_i}}{\nkind}$ and $\sseq{\iskind{\Omega_\text{app}}{\kappa_i}}{\nkind}$
%   \item (\textbf{Kinding 2}) $\sseq{\cvalidK{\OParams}{\csceneP{\omega : \OParams}{\uOmega}{b}}{\cekappa'_i}{\kappa'_i}}{\ncon}$ and $\sseq{\iskind{\Omega_\text{app}}{[\omega]\kappa'_i}}{\ncon}$
%   \item (\textbf{Kinding 3}) $\sseq{\cExpands{\uOmega}{\parseUConF{\bsubseq{b}{m'_i}{n'_i}}}{c_i}{[\omega]\kappa'_i}}{\ncon}$ and $\sseq{\haskind{\Omega_\text{app}}{c_i}{[\omega]\kappa'_i}}{\ncon}$
%   \item (\textbf{Kinding 4}) $\sseq{\cvalidC{\OParams}{\csceneP{\omega : \OParams}{\uOmega}{b}}{\ctau_i}{\tau_i}{\akty}}{\npat}$ and $\sseq{\haskind{\Omega_\text{app}}{[\omega]\tau_i}{\akty}}{\npat}$
%   \item (\textbf{Typing 2}) $\sseq{\patExpandsP{\uOmegaEx{\emptyset}{\uG_i}{\emptyset}{\Omega_i}}{\uPhi}{\parseUPatF{\bsubseq{b}{m''_i}{n''_i}}}{p_i}{[\omega]\tau_i}}{\npat}$ and $\sseq{\patTypePC{\Omega_\text{app}}{\Omega_i}{p_i}{[\omega]\tau_i}}{\npat}$
%       \item (\textbf{Visibility}) $\uOmega'=\uOmegaEx{\emptyset}{\biguplus_{0 \leq i < \npat} \uG_i}{\emptyset}{\bigcup_{0 \leq i < \npat} \Omega_i}$

%   % \item $\cvalidPP{\uOmega'}{\psceneP{\omega : \Omega_\text{params}}{\uOmega}{\uPhi}{b}}{\cpv}{p}{\tau_\text{proto}}$
%   % \item $\tau = [\omega]\tau_\text{proto}$
%   % \item (\textbf{Typing}) $\tau_\text{final} = [\omega]\tau_\text{proto}$
% \end{enumerate}
% \end{theorem}
% \begin{proof} 
% By rule induction over Rules (\ref{rules:patExpandsP}). There are two rules that apply.
% \begin{byCases}
%   \item[\text{(\ref{rule:patExpandsP-subsume})}] ~
%     \begin{pfsteps*}
%       \item $\uOmega=\uOmegaEx{\uD}{\uG}{\uMctx}{\Omega}$ \BY{assumption}
%       \item $\patExpandsP{\uOmega'}{\uPhi}{\utsmap{\uepsilon}{b}}{p}{\tau'}$ \BY{assumption} \pflabel{patExpands}
%       \item $\issubtypeP{\Omega}{\tau'}{\tau}$ \BY{assumption} \pflabel{issubtypeP}
%       \item $\uOmega=\uOmegaEx{\uD}{\uG}{\uMctx}{\Omega_\text{app}}$ \BY{IH on \pfref{patExpands}} \pflabel{A}
%       \item $\uPhi=\uAS{\uA}{\Phi}$ \BY{IH on \pfref{patExpands}} \pflabel{B}
%       \item $\tsmexpExpandsPat{\uOmega}{\uPhi}{\uepsilon}{\epsilon}{\aetype{\tau''}}$ and $\patTypePC{\Omega_\text{app}}{\uOmega'}{p}{\tau''}$ for $\tau''$ such that $\issubtypeP{\Omega_\text{app}}{\tau''}{\tau'}$ \BY{IH on \pfref{patExpands}} \pflabel{C}
%       \item $\tsmexpEvalsPat{\Omega_\text{app}}{\Phi}{\epsilon}{\epsilon_\text{normal}}$ \BY{IH on \pfref{patExpands}} \pflabel{D}
%       \item $\tsmdefof{\epsilon_\text{normal}}=a$ \BY{IH on \pfref{patExpands}} \pflabel{E}
%       \item $\Phi = \Phi', \pptsmdefn{a}{\rho}{\eparse}$\BY{IH on \pfref{patExpands}} \pflabel{F}
%       \item $\encodeBody{b}{\ebody}$\BY{IH on \pfref{patExpands}} \pflabel{G}
%       \item $\evalU{\ap{\eparse}{\ebody}}{\aein{\mathtt{SuccessP}}{e_\text{pproto}}}$\BY{IH on \pfref{patExpands}} \pflabel{H}
%       \item $\decodePCEPat{e_\text{pproto}}{\pcp}$\BY{IH on \pfref{patExpands}} \pflabel{I}
%       \item $\prepcp{\Omega_\text{app}}{\Phi}{\pcp}{\cpv}{\epsilon_\text{normal}}{\aetype{\tau_\text{proto}}}{\omega}{\Omega_\text{params}}$\BY{IH on \pfref{patExpands}} \pflabel{J}
%       \item $\segOKP{\OParams}{\csceneP{\omega : \OParams}{\uOmega}{b}}{\segof{\cpv}}{b}$ \BY{IH on \pfref{patExpands}} \pflabel{K}
%       \item $\cvalidPP{\uOmega'}{\psceneP{\omega : \Omega_\text{params}}{\uOmega}{\uPhi}{b}}{\cpv}{p}{\tau_\text{proto}}$\BY{IH on \pfref{patExpands}} \pflabel{L}
%       \item $\tau'=[\omega]\tau_\text{proto}$ \BY{IH on \pfref{patExpands}}\pflabel{L2}
%       \item $
%       \segof{\cpv} = \sseq{\acesplicedk{m_i}{n_i}}{\nkind} \cup \sseq{\acesplicedc{m'_i}{n'_i}{\cekappa'_i}}{\ncon} \cup
%                  \sseq{\acesplicedp{m''_i}{n''_i}{\ctau_i}}{\npat}
%       $\BY{IH on \pfref{patExpands}} \pflabel{M}
%       \item $\sseq{\kExpands{\uOmega}{\parseUKindF{\bsubseq{b}{m_i}{n_i}}}{\kappa_i}}{\nkind}$ and $\sseq{\iskind{\Omega_\text{app}}{\kappa_i}}{\nkind}$\BY{IH on \pfref{patExpands}} \pflabel{N}
%       \item $\sseq{\cvalidK{\OParams}{\csceneP{\omega : \OParams}{\uOmega}{b}}{\cekappa'_i}{\kappa'_i}}{\ncon}$ and $\sseq{\iskind{\Omega_\text{app}}{[\omega]\kappa'_i}}{\ncon}$\BY{IH on \pfref{patExpands}} \pflabel{O}
%       \item $\sseq{\cExpands{\uOmega}{\parseUConF{\bsubseq{b}{m'_i}{n'_i}}}{c_i}{[\omega]\kappa'_i}}{\ncon}$ and $\sseq{\haskind{\Omega_\text{app}}{c_i}{[\omega]\kappa'_i}}{\ncon}$\BY{IH on \pfref{patExpands}} \pflabel{P}
%       \item $\sseq{\cvalidC{\OParams}{\csceneP{\omega : \OParams}{\uOmega}{b}}{\ctau_i}{\tau_i}{\akty}}{\npat}$ and $\sseq{\haskind{\Omega_\text{app}}{[\omega]\tau_i}{\akty}}{\npat}$\BY{IH on \pfref{patExpands}} \pflabel{Q}
%       \item $\sseq{\patExpandsP{\uOmegaEx{\emptyset}{\uG_i}{\emptyset}{\Omega_i}}{\uPhi}{\parseUPatF{\bsubseq{b}{m''_i}{n''_i}}}{p_i}{[\omega]\tau_i}}{\npat}$ and $\sseq{\patTypePC{\Omega_\text{app}}{\Omega_i}{p_i}{[\omega]\tau_i}}{\npat}$\BY{IH on \pfref{patExpands}} \pflabel{R}
%       \item $\uOmega'=\uOmegaEx{\emptyset}{\biguplus_{0 \leq i < \npat} \uG_i}{\emptyset}{\bigcup_{0 \leq i < \npat} \Omega_i}$\BY{IH on \pfref{patExpands}} \pflabel{S}
%       \item $\issubtypeP{\Omega_\text{app}}{\tau''}{\tau}$ \BY{Rule (\ref{rule:issubtypeP-trans}) on \pfref{C} and \pfref{issubtypeP}} \pflabel{issubtypeF}
%     \end{pfsteps*}
%     The conclusions hold as follows:
%     \begin{enumerate}
%       \item \pfref{A}
%       \item \pfref{B}
%       \item Choosing $\tau''$, by \pfref{C} and \pfref{issubtypeF}
%       \item \pfref{D}
%       \item \pfref{E}
%       \item \pfref{F}
%       \item \pfref{G}
%       \item \pfref{H}
%       \item \pfref{I}
%       \item \pfref{J}
%       \item \pfref{K}
%       \item \pfref{L}
%       \item \pfref{L2}
%       \item \pfref{M}
%       \item \pfref{N}
%       \item \pfref{O}
%       \item \pfref{P}
%       \item \pfref{Q}
%       \item \pfref{R}
%       \item \pfref{S}
%     \end{enumerate}
%     \resetpfcounter
%   \item[\text{(\ref{rule:patExpandsP-apuptsm})}] ~
%     \begin{pfsteps*}
%       \item $\uOmega=\uOmegaEx{\uD}{\uG}{\uMctx}{\Omega_\text{app}}$ \BY{assumption} \pflabel{uOmega}
%       \item $\uPhi=\uAS{\uA}{\Phi}$ \BY{assumption} \pflabel{uPhi}
%       \item $\tsmexpExpandsPat{\uOmega}{\uPhi}{\uepsilon}{\epsilon}{\aetype{\tau_\text{final}}}$ \BY{assumption} \pflabel{tsmexpExpands}
%       \item $\tsmexpEvalsPat{\Omega_\text{app}}{\Phi}{\epsilon}{\epsilon_\text{normal}}$ \BY{assumption} \pflabel{tsmexpEvals}
%       \item $\tsmdefof{\epsilon_\text{normal}}=a$ \BY{assumption} \pflabel{tsmdefof}
%       \item $\Phi = \Phi', \pptsmdefn{a}{\rho}{\eparse}$ \BY{assumption} \pflabel{Phi}
%       \item $\encodeBody{b}{\ebody}$ \BY{assumption} \pflabel{encodeBody}
%       \item $\evalU{\ap{\eparse}{\ebody}}{\aein{\mathtt{SuccessP}}{e_\text{pproto}}}$ \BY{assumption} \pflabel{eval}
%       \item $\decodePCEPat{e_\text{pproto}}{\pcp}$ \BY{assumption} \pflabel{decode}
%       \item $\prepcp{\Omega_\text{app}}{\Phi}{\pcp}{\cpv}{\epsilon_\text{normal}}{\aetype{\tau_\text{proto}}}{\omega}{\Omega_\text{params}}$ \BY{assumption} \pflabel{prep}
%       \item $\segOKP{\OParams}{\csceneP{\omega : \OParams}{\uOmega}{b}}{\segof{\cpv}}{b}$ \BY{assumption} \pflabel{segOK}
%       \item $\cvalidPP{\uOmega'}{\psceneP{\omega : \Omega_\text{params}}{\uOmega}{\uPhi}{b}}{\cpv}{p}{\tau_\text{proto}}$ \BY{assumption} \pflabel{cvalidPP}
%         \item $\tau=[\omega]\tau_\text{proto}$ \BY{assumption} \pflabel{tau}
%       \item $\hastsmtypePat{\Omega_\text{app}}{\Psi}{\epsilon}{\aetype{\tau_\text{final}}}$ \BY{Theorem \ref{thm:ppTLM-expression-expansion} on \pfref{tsmexpExpands}} \pflabel{hastsmtype}
%       \item $\hastsmtypePat{\Omega_\text{app}}{\Psi}{\epsilon_\text{normal}}{\aetype{\tau_\text{final}}}$ \BY{Corollary \ref{thm:ppTLM-preservation-evaluation} on \pfref{hastsmtype} and \pfref{tsmexpEvals}} \pflabel{hastsmtypeNormal}
%       \item $\hastsmtypePat{\Omega_\text{app}}{\Psi}{\epsilon_\text{normal}}{[\omega]\aetype{\tau_\text{proto}}}$ \BY{Lemma \ref{lemma:pattern-deparameterization-P} on \pfref{prep}} \pflabel{hastsmtypeNormal2}
%       \item $\aetype{\tau_\text{final}} = [\omega]\aetype{\tau_\text{proto}}$ \BY{Theorem \ref{lemma:ppTLM-unicity} on \pfref{hastsmtypeNormal} and \pfref{hastsmtypeNormal2}} \pflabel{tsmtypeseq}
%       \item $\tau_\text{final}=[\omega]\tau_\text{proto}=\tau$ \BY{definition of substitution and \pfref{tsmtypeseq} and \pfref{tau}} \pflabel{feqp}
%       \item $\istsmty{\Omega_\text{app}}{\aetype{\tau_\text{final}}}$ \BY{Lemma \ref{lemma:ppTLM-regularity} on \pfref{hastsmtype}} \pflabel{istsmty}
%       \item $\haskind{\Omega_\text{app}}{\tau}{\akty}$ \BY{Inversion of Rule (\ref{rule:istsmty-type}) on \pfref{istsmty}} \pflabel{istype}
%       \item $\cequal{\Omega_\text{app}}{\tau}{\tau}{\akty}$ \BY{Rule (\ref{rule:cequal-refl}) on \pfref{istype}} \pflabel{refl}
%       \item $\issubtypeP{\Omega_\text{app}}{\tau}{\tau}$ \BY{Rule (\ref{rule:issubtypeP-equal}) on \pfref{refl}} \pflabel{subty}
%       \item $\patExpandsP{\uOmega'}{\uPhi}{\utsmap{\uepsilon}{b}}{p}{\tau}$ \BY{assumption} \pflabel{patExpandsA}
%       \item $\patTypePC{\Omega_\text{app}}{\uOmega'}{p}{\tau}$ \BY{Theorem \ref{thm:typed-pattern-expansion-P} on \pfref{patExpandsA}} \pflabel{patType}
%       \item $\hastypeP{\Omega_\text{app}}{\omega}{\Omega_\text{params}}$ \BY{Lemma \ref{lemma:pattern-deparameterization-P} on \pfref{prep}} \pflabel{omegaType}
%       \item $\segof{\cpv} = \sseq{\acesplicedk{m_i}{n_i}}{\nkind} \cup \sseq{\acesplicedc{m'_i}{n'_i}{\cekappa'_i}}{\ncon} \cup
%              \sseq{\acesplicedp{m''_i}{n''_i}{\ctau_i}}{\npat}
%   $ \BY{definition} \pflabel{summ}
%       \item $\sseq{\kExpands{\uOmega}{\parseUKindF{\bsubseq{b}{m_i}{n_i}}}{\kappa_i}}{\nkind}$ \BY{Lemma \ref{lemma:proto-pattern-decomposition-P} on \pfref{cvalidPP} and \pfref{summ}} \pflabel{A}
%       \item $\sseq{\iskind{\Omega_\text{app}}{\kappa_i}}{\nkind}$ \BY{Lemma \ref{thm:kind-and-constructor-expansion-P} over \pfref{A}} \pflabel{A2}
%       \item $\sseq{\cvalidK{\OParams}{\csceneP{\omega : \OParams}{\uOmega}{b}}{\cekappa'_i}{\kappa'_i}}{\ncon}$ \BY{Lemma \ref{lemma:proto-pattern-decomposition-P} on \pfref{cvalidPP} and \pfref{summ}} \pflabel{B}
%       \item $\sseq{\iskind{\Omega_\text{app} \cup \Omega_\text{params}}{\kappa'_i}}{\ncon}$ \BY{Theorem \ref{thm:kind-and-constructor-expansion-P} over \pfref{B}} \pflabel{B2}
%       \item $\sseq{\iskind{\Omega_\text{app}}{[\omega]\kappa'_i}}{\ncon}$ \BY{Lemma \ref{lemma:substitution-P} over \pfref{B2} and \pfref{omegaType}} \pflabel{B3}
%       \item $\sseq{\cExpands{\uOmega}{\parseUConF{\bsubseq{b}{m'_i}{n'_i}}}{c_i}{[\omega]\kappa'_i}}{\ncon}$ \BY{Lemma \ref{lemma:proto-pattern-decomposition-P} on \pfref{cvalidPP} and \pfref{summ}} \pflabel{C}
%       \item $\sseq{\haskind{\Omega_\text{app}}{[\omega]\tau_i}{\akty}}{\npat}$ \BY{Theorem \ref{thm:kind-and-constructor-expansion-P} over \pfref{C}} \pflabel{C2}
%       \item $\sseq{\cvalidC{\OParams}{\csceneP{\omega : \OParams}{\uOmega}{b}}{\ctau_i}{\tau_i}{\akty}}{\npat}$ \BY{Lemma \ref{lemma:proto-pattern-decomposition-P} on \pfref{cvalidPP} and \pfref{summ}} \pflabel{D}
%       \item $\sseq{\haskind{\Omega_\text{app} \cup \OParams}{\tau_i}{\akty}}{\npat}$ \BY{Theorem \ref{thm:kind-and-constructor-expansion-P} over \pfref{D}} \pflabel{D2}
%       \item $\sseq{\haskind{\Omega_\text{app}}{[\omega]\tau_i}{\akty}}{\npat}$ \BY{Lemma \ref{lemma:substitution-P} over \pfref{D2} and \pfref{omegaType}} \pflabel{D3}
%       \item $\sseq{\patExpandsP{\uOmegaEx{\emptyset}{\uG_i}{\emptyset}{\Omega_i}}{\uPhi}{\parseUPatF{\bsubseq{b}{m''_i}{n''_i}}}{p_i}{[\omega]\tau_i}}{\npat}$ \BY{Lemma \ref{lemma:proto-pattern-decomposition-P} on \pfref{cvalidPP} and \pfref{summ}} \pflabel{E}
%       \item $\sseq{\patTypePC{\Omega_\text{app}}{\Omega_i}{p_i}{[\omega]\tau_i}}{\npat}$ \BY{Theorem \ref{thm:typed-pattern-expansion-P} over \pfref{E}} \pflabel{E2}
%       \item $\uOmega'=\uOmegaEx{\emptyset}{\biguplus_{0 \leq i < \npat} \uG_i}{\emptyset}{\bigcup_{0 \leq i < \npat} \Omega_i}$ \BY{Lemma \ref{lemma:proto-pattern-decomposition-P} on \pfref{cvalidPP} and \pfref{summ}} \pflabel{F}
%     \end{pfsteps*}
%     The conclusions hold as follows:
%     \begin{enumerate}
%       \item \pfref{uOmega}
%       \item \pfref{uPhi}
%       \item Choosing $\tau$, by \pfref{tsmexpExpands} and \pfref{patType} and \pfref{subty}
%       \item \pfref{tsmexpEvals}
%       \item \pfref{tsmdefof}
%       \item \pfref{Phi}
%       \item \pfref{encodeBody}
%       \item \pfref{eval}
%       \item \pfref{decode}
%       \item \pfref{prep}
%       \item \pfref{segOK}
%       \item \pfref{cvalidPP}
%       \item \pfref{tau}
%       \item \pfref{summ}
%       \item \pfref{A} and \pfref{A2}
%       \item \pfref{B} and \pfref{B3}
%       \item \pfref{C} and \pfref{C2}
%       \item \pfref{D} and \pfref{D3}
%       \item \pfref{E} and \pfref{E2}
%       \item \pfref{F}
%     \end{enumerate}
%     \resetpfcounter
% \end{byCases}
% \end{proof}

% % \begin{byCases}
% %   \item[\text{(\ref{rule:patExpandsP-subsume})}] \todo{subsumption}
% %   \item[\text{(\ref{rule:patExpandsP-var}) \textbf{through} (\ref{rule:patExpandsP-in})}] \todo{inductive cases}
% %   \item[\text{(\ref{rule:patExpandsP-apuptsm})}] ~
% %     \begin{pfsteps*}
% %       \item 
% %     \end{pfsteps*}
% %     \resetpfcounter
% %     The conclusions hold as follows:
% %     \begin{enumerate}
% %      \item \todo{1}
% %      \item \todo{2}
% %      \item \todo{3}
% %      \item \todo{4}
% %      \item \todo{5}
% %      \item \todo{6}
% %      \item \todo{7}
% %      \item \todo{8}
% %      \item \todo{9}
% %      \item \todo{10}
% %      \item \todo{11}
% %      \item \todo{12}
% %      \item \todo{13}
% %      \item \todo{14}
% %      \item \todo{15}
% %      \item \todo{16}
% %      \item \todo{17}
% %      \item \todo{18}
% %     \end{enumerate}
% % \end{byCases}

% % \end{proof}

\ificfp 
\else

\chapter{\texorpdfstring{Bidirectional $\miniVersePat$}{Bidirectional miniVerseS}}\label{appendix:simple-implicits}

\section{Expanded Language (XL)}
The Bidirectional $\miniVersePat$ expanded language (XL) is the same as the  $\miniVersePat$ XL, which was detailed in Appendix \ref{appendix:SES-XL}. %It consists of types, $\tau$, expanded expressions, $e$, expanded rules, $r$, and expanded patterns, $p$.

\section{Unexpanded Language (UL)}
\subsection{Syntax}
\subsubsection{Stylized Syntax}
The stylized syntax extends the stylized syntax of the $\miniVersePat$ UL given in Sec. \ref{appendix:SES-syntax}.

\[\begin{array}{lllllll}
\textbf{Sort} & & 
%& \textbf{Operational Form} 
& \textbf{Stylized Form} & \textbf{Description}\\
\mathsf{UTyp} & \utau & ::= 
%& \cdots 
& \cdots & \text{(as in $\miniVersePat$)}\\
\mathsf{UExp} & \ue & ::= 
%& \cdots 
& \cdots & \text{(as in $\miniVersePat$)}\\
% &&
%& \auasc{\utau}{\ue} 
% & \asc{\ue}{\utau} & \text{ascription}\\
% &&
%& \auletsyn{\ux}{\ue}{\ue} 
% & \letsyn{\ux}{\ue}{\ue} & \text{value binding}\\
&&& \implicite{\tsmv}{\ue} & \text{seTLM designation}\\
&&& \implicitp{\tsmv}{\ue} & \text{spTLM designation}\\
&&& \lit{b} & \text{seTLM unadorned literal}\\
% &&& \auanalam{\ux}{\ue} & \analam{\ux}{\ue} & \text{abstraction (unannotated)}\\
% &&& \aulam{\utau}{\ux}{\ue} & \lam{\ux}{\utau}{\ue} & \text{abstraction (annotated)}\\
% &&& \auap{\ue}{\ue} & \ap{\ue}{\ue} & \text{application}\\
% &&& \autlam{\ut}{\ue} & \Lam{\ut}{\ue} & \text{type abstraction}\\
% &&& \autap{\ue}{\utau} & \App{\ue}{\utau} & \text{type application}\\
% &&& \auanafold{\ue} & \fold{\ue} & \text{fold}\\
% &&& \auunfold{\ue} & \unfold{\ue} & \text{unfold}\\
% &&& \autpl{\labelset}{\mapschema{\ue}{i}{\labelset}} & \tpl{\mapschema{\ue}{i}{\labelset}} & \text{labeled tuple}\\
% &&& \aupr{\ell}{\ue} & \prj{\ue}{\ell} & \text{projection}\\
% &&& \auanain{\ell}{\ue} & \inj{\ell}{\ue} & \text{injection}\\
% &&& \aumatchwithb{n}{\ue}{\seqschemaX{\urv}} & \matchwith{\ue}{\seqschemaX{\urv}} & \text{match}\\
% &&& \audefuetsm{\utau}{e}{\tsmv}{\ue} & \texttt{syntax}~\tsmv~\texttt{at}~\utau~\texttt{for} & \text{ueTLM definition}\\
% &&&                                    & \texttt{expressions}~\{e\}~\texttt{in}~\ue\\
% \LCC &&& \color{light-gray} & \color{light-gray} & \color{light-gray} \\
% &&& \auimplicite{\tsmv}{\ue} & \texttt{implicit\,syntax}~\tsmv~\texttt{for} & \text{ueTLM designation}\\
% &&&                          & \texttt{expressions\,in}~\ue\\ \ECC
% &&& \autsmap{b}{\tsmv} & \utsmap{\tsmv}{b} & \text{ueTLM application}\\%\ECC
% \LCC &&& \color{light-gray} & \color{light-gray} & \color{light-gray} \\
% &&& \auelit{b} & {\lit{b}}  & \text{ueTLM unadorned literal}\\\ECC
% &&& \audefuptsm{\utau}{e}{\tsmv}{\ue} & \texttt{syntax}~\tsmv~\texttt{at}~\utau~\texttt{for} & \text{upTLM definition}\\
% &&&                                    & \texttt{patterns}~\{e\}~\texttt{in}~\ue\\
% \LCC &&& \color{light-gray} & \color{light-gray} & \color{light-gray} \\
% &&& \auimplicitp{\tsmv}{\ue} & \texttt{implicit\,syntax}~\tsmv~\texttt{for} & \text{upTLM designation}\\
% &&&                          & \texttt{patterns\,in}~\ue\\ \ECC
\mathsf{URule} & \urv & ::= 
%& \aumatchrule{\upv}{\ue} 
& \cdots & \text{(as in $\miniVersePat$)}\\
\mathsf{UPat} & \upv & ::= 
%& \ux 
& \cdots & \text{(as in $\miniVersePat$)}\\
&&& \lit{b} & \text{spTLM unadorned literal}
% &&& \auwildp & \wildp & \text{wildcard pattern}\\
% &&& \aufoldp{\upv} & \foldp{\upv} & \text{fold pattern}\\
% &&& \autplp{\labelset}{\mapschema{\upv}{i}{\labelset}} & \tplp{\mapschema{\upv}{i}{\labelset}} & \text{labeled tuple pattern}\\
% &&& \auinjp{\ell}{\upv} & \injp{\ell}{\upv} & \text{injection pattern}\\
% &&& \auapuptsm{b}{\tsmv} & \utsmap{\tsmv}{b} & \text{upTLM application}\\
% \LCC &&& \color{light-gray} & \color{light-gray} & \color{light-gray}\\
% &&& \auplit{b} & \lit{b} & \text{upTLM unadorned literal}\ECC
\end{array}\]
\subsubsection{Body Lengths}
We write $\sizeof{b}$ for the length of $b$. The metafunction $\sizeof{\ue}$ computes the sum of the lengths of expression literal bodies in $\ue$. It is defined by extending the definition given in Sec. \ref{appendix:SES-syntax} with the following additional cases:
\[
\begin{array}{ll}
% \sizeof{\asc{\ue}{\utau}} & = \sizeof{\ue}\\
% \sizeof{\letsyn{\ux}{\ue_1}{\ue_2}} & = \sizeof{\ue_1} + \sizeof{\ue_2}\\
\sizeof{\implicite{\tsmv}{\ue}} & = \sizeof{\ue}\\
\sizeof{\implicitp{\tsmv}{\ue}} & = \sizeof{\ue}\\
\sizeof{\lit{b}} & = \sizeof{b}
\end{array}
\]

Similarly, the metafunction $\sizeof{\upv}$ computes the sum of the lengths of the pattern literal bodies in $\upv$. It is defined by extending the definition given in Sec. \ref{appendix:SES-syntax} with the following additional case:
\begin{align*}
\sizeof{\lit{b}} & = \sizeof{b}
\end{align*}

\subsubsection{Textual Syntax}
In addition to the stylized syntax, there is also a context-free textual syntax for the UL. We need only posit the existence of the following partial metafunctions.

\begin{condition}[Textual Representability]\label{condition:textual-representability-BS} ~
\begin{enumerate}
\item For each $\utau$, there exists $b$ such that $\parseUTyp{b}{\utau}$. 
\item For each $\ue$, there exists $b$ such that $\parseUExp{b}{\ue}$.
% \item For each $\urv$, there exists $b$ such that $\parseURule{b}{\urv}$.
\item {For each $\upv$, there exists $b$ such that $\parseUPat{b}{\upv}$.}
\end{enumerate}
\end{condition}

We also impose the following technical conditions.

\begin{condition}[Expression Parsing Monotonicity]\label{condition:body-parsing-BS} If $\parseUExp{b}{\ue}$ then $\sizeof{\ue} < \sizeof{b}$.\end{condition}

\begin{condition}[Pattern Parsing Monotonicity]\label{condition:pattern-parsing-BS} If $\parseUPat{b}{\upv}$ then $\sizeof{\upv} < \sizeof{b}$.\end{condition}

\subsection{Bidirectionally Typed Expansion}
\subsubsection{Contexts}
\emph{Unexpanded type formation contexts}, $\uDelta$, and \emph{unexpanded typing contexts}, $\uGamma$, were defined in Sec. \ref{appendix:typed-expression-expansion-S}.

\subsubsection{Body Encoding and Decoding}
The type $\tBody$ and the judgements $\encodeBody{b}{e}$ and $\decodeBody{e}{b}$ are characterized in Sec. \ref{appendix:typed-expression-expansion-S}.

\subsubsection{Parse Results}
The types $\tParseResultExp$ and $\tParseResultPat$ are defined as in Sec. \ref{appendix:typed-expression-expansion-S}.

\subsubsection{TLM Contexts}

\emph{seTLM contexts}, $\uPsi$, are of the form $\uASI{\uA}{\Psi}{\uI}$, where $\uA$ is a \emph{TLM identifier expansion context}, $\Psi$ is a \emph{seTLM definition context} and $\uI$ is a \emph{TLM implicit designation context}.

\emph{spTLM contexts}, $\uPhi$, are of the form $\uASI{\uA}{\Phi}{\uI}$, where $\uA$ is a {TLM identifier expansion context}, defined above, and $\Phi$ is a \emph{spTLM definition context}. 

A \emph{TLM identifier expansion context}, $\uA$, is a finite function mapping each TLM identifier $\tsmv \in \domof{\uA}$ to the \emph{TLM identifier expansion}, $\vExpands{\tsmv}{x}$, for some variable, $x$. We write $\ctxUpdate{\uA}{\tsmv}{x}$ for the TLM identifier expansion context that maps $\tsmv$ to $\vExpands{\tsmv}{x}$, and defers to $\uA$ for all other TLM identifiers (i.e. the previous mapping is \emph{updated}.)

An \emph{seTLM definition context}, $\Psi$, is a finite function mapping each TLM name $x \in \domof{\Psi}$ to an \emph{expanded seTLM definition}, $\xuetsmbnd{x}{\tau}{\eparse}$, where $\tau$ is the seTLM's type annotation, and $\eparse$ is its parse function. We write $\Psi, \xuetsmbnd{x}{\tau}{\eparse}$ when $a \notin \domof{\Psi}$ for the extension of $\Psi$ that maps $x$ to $\xuetsmbnd{x}{\tau}{\eparse}$. We write $\uetsmenv{\Delta}{\Psi}$  when all the type annotations in $\Psi$ are well-formed assuming $\Delta$, and the parse functions in $\Psi$ are closed and of the appropriate type.

\begin{definition}[seTLM Definition Context Formation]\label{def:seTLM-def-ctx-formation-B} $\uetsmenv{\Delta}{\Psi}$ iff for each $\xuetsmbnd{x}{\tau}{\eparse} \in \Psi$, we have $\istypeU{\Delta}{\tau}$ and $\hastypeU{\emptyset}{\emptyset}{\eparse}{\aparr{\tBody}{\tParseResultExp}}$.\end{definition}

An \emph{spTLM definition context}, $\Phi$, is a finite function mapping each TLM name $x \in \domof{\Phi}$ to an \emph{expanded seTLM definition}, $\xuptsmbnd{a}{\tau}{\eparse}$, where $\tau$ is the spTLM's type annotation, and $\eparse$ is its parse function. We write $\Phi, \xuptsmbnd{a}{\tau}{\eparse}$ when $a \notin \domof{\Phi}$ for the extension of $\Phi$ that maps $x$ to $\xuptsmbnd{a}{\tau}{\eparse}$. We write $\uptsmenv{\Delta}{\Phi}$  when all the type annotations in $\Phi$ are well-formed assuming $\Delta$, and the parse functions in $\Phi$ are closed and of the appropriate type.

\begin{definition}[spTLM Definition Context Formation]\label{def:spTLM-def-ctx-formation-B} $\uptsmenv{\Delta}{\Phi}$ iff for each $\xuptsmbnd{a}{\tau}{\eparse} \in \Phi$, we have $\istypeU{\Delta}{\tau}$ and $\hastypeU{\emptyset}{\emptyset}{\eparse}{\aparr{\tBody}{\tParseResultPat}}$.\end{definition}

A \emph{TLM implicit designation context}, $\uI$, is a finite function that maps each type $\tau \in \domof{\uI}$ to the \emph{TLM designation} $\designate{\tau}{a}$, for some TLM name $x$. We write $\uI \uplus \designate{\tau}{a}$ for the TLM implicit designation context that maps $\tau$ to $\designate{\tau}{a}$ and defers to $\uI$ for all other types (i.e. the previous designation, if any, is updated.)

\begin{definition}[TLM Implicit Designation Context Formation]\label{def:implicit-designation-ctx-formation-S} $\uIOK{\Delta}{\uI}$ iff for each $\designate{\tau}{a} \in \uI$, we have $\istypeU{\Delta}{\tau}$.
\end{definition}

\begin{definition}[seTLM Context Formation] $\uetsmctx{\Delta}{\uASI{\uA}{\Psi}{\uI}}$ iff
\begin{enumerate}
\item $\uetsmenv{\Delta}{\Psi}$; and
\item for each $\vExpands{\tsmv}{x} \in \uA$ we have $x \in \domof{\Psi}$; and
\item $\uIOK{\Delta}{\uI}$; and
\item for each $\designate{\tau}{a} \in \uI$, we have $x \in \domof{\Psi}$.
\end{enumerate}
\end{definition}

\begin{definition}[spTLM Context Formation] $\uptsmctx{\Delta}{\uASI{\uA}{\Phi}{\uI}}$ iff 
\begin{enumerate}
\item $\uptsmenv{\Delta}{\Phi}$; and
\item for each $\vExpands{\tsmv}{x} \in \uA$ we have $x \in \domof{\Phi}$; and
\item $\uIOK{\Delta}{\uI}$; and
\item for each $\designate{\tau}{a} \in \uI$ we have $x \in \domof{\Phi}$.
\end{enumerate}
\end{definition}

We define $\uPsi, \uShyp{\tsmv}{x}{\tau}{\eparse}$, when $\uPsi=\uASI{\uA}{\Phi}{\uI}$, as an abbreviation of \[\uASI{\ctxUpdate{\uA}{\tsmv}{x}}{\Psi, \xuetsmbnd{x}{\tau}{\eparse}}{\uI}\]
%\vspace{10px}

We define $\uPhi, \uPhyp{\tsmv}{x}{\tau}{\eparse}$, when $\uPhi=\uASI{\uA}{\Phi}{\uI}$, as an abbreviation of \[\uASI{\ctxUpdate{\uA}{\tsmv}{x}}{\Phi, \xuptsmbnd{a}{\tau}{\eparse}}{\uI}\]

\subsubsection{Type Expansion}
The \emph{type expansion judgement}, $\expandsTU{\uDelta}{\utau}{\tau}$, is inductively defined as in $\miniVersePat$ by Rules (\ref{rules:expandsTU}).

\subsubsection{Bidirectionally Typed Expression and Rule Expansion}
\noindent\fbox{$\esyn{\uDelta}{\uGamma}{\uPsi}{\uPhi}{\ue}{e}{\tau}$}~~$\ue$ has expansion $e$ synthesizing type $\tau$

\begin{subequations}\label{rules:esyn-S}
\begin{equation}\label{rule:esyn-S-var}
  \inferrule{ }{ 
    \esyn{\uDelta}{\uGamma, \uGhyp{\ux}{x}{\tau}}{\uPsi}{\uPhi}{\ux}{x}{\tau}
  }
\end{equation}
\begin{equation}\label{rule:esyn-S-asc}
  \inferrule{
    \expandsTU{\uDelta}{\utau}{\tau}\\
    \eanaX{\ue}{e}{\tau}
  }{
    \esynX{\asc{\ue}{\utau}}{e}{\tau}
  }
\end{equation}
\begin{equation}\label{rule:esyn-S-let}
  \inferrule{
    \esynX{\ue}{e}{\tau}\\
    \esyn{\uDelta}{\uGamma, \uGhyp{\ux}{x}{\tau}}{\uPsi}{\uPhi}{\ue'}{e'}{\tau'}
  }{
    \esynX{\letsyn{\ux}{\ue}{\ue'}}{\aeap{\aelam{\tau}{x}{e'}}{e}}{\tau'}
  }
\end{equation}
\begin{equation}\label{rule:esyn-S-lam}
  \inferrule{
    \expandsTU{\uDelta}{\utau_1}{\tau_1}\\
    \esyn{\uDelta}{\uGamma, \uGhyp{\ux}{x}{\tau_1}}{\uPsi}{\uPhi}{\ue}{e}{\tau_2}
  }{
    \esynX{\lam{\ux}{\utau_1}{\ue}}{\aelam{\tau_1}{x}{e}}{\aparr{\tau_1}{\tau_2}}
  }
\end{equation}
\begin{equation}\label{rule:esyn-S-ap}
  \inferrule{
    \esynX{\ue_1}{e_1}{\aparr{\tau_2}{\tau}}\\
    \eanaX{\ue_2}{e_2}{\tau_2}
  }{
    \esynX{\ap{\ue_1}{\ue_2}}{\aeap{e_1}{e_2}}{\tau}
  }
\end{equation}
\begin{equation}\label{rule:esyn-S-tlam}
  \inferrule{
    \esyn{\uDelta, \uDhyp{\ut}{t}}{\uGamma}{\uPsi}{\uPhi}{\ue}{e}{\tau}
  }{
    \esynX{\Lam{\ut}{\ue}}{\aetlam{t}{e}}{\aall{t}{\tau}}
  }
\end{equation}
\begin{equation}\label{rule:esyn-S-tap}
  \inferrule{
    \esynX{\ue}{e}{\aall{t}{\tau}}\\
    \expandsTU{\uDelta}{\utau'}{\tau'}
  }{
    \esynX{\App{\ue}{\utau'}}{\aetap{e}{\tau'}}{[\tau'/t]\tau}
  }
\end{equation}
\begin{equation}\label{rule:esyn-S-unfold}
  \inferrule{
    \esynX{\ue}{e}{\arec{t}{\tau}}
  }{
    \esynX{\unfold{\ue}}{\aeunfold{e}}{[\arec{t}{\tau}/t]\tau}
  }
\end{equation}
\begin{equation}\label{rule:esyn-S-tpl}
  \inferrule{
    \{\esynX{\ue_i}{e_i}{\tau_i}\}_{i \in \labelset}
  }{
    \esynX{\tpl{\mapschema{\ue}{i}{\labelset}}}{\aetpl{\labelset}{\mapschema{e}{i}{\labelset}}}{\aprod{\labelset}{\mapschema{\tau}{i}{\labelset}}}
  }
\end{equation}
\begin{equation}\label{rule:esyn-S-pr}
  \inferrule{
    \esynX{\ue}{e}{\aprod{\labelset, \ell}{\mapschema{\tau}{i}{\labelset}; \mapitem{\ell}{\tau}}}
  }{
    \esynX{\prj{\ue}{\ell}}{\aepr{\ell}{e}}{\tau}
  }
\end{equation}
\begin{equation}\label{rule:esyn-defuetsm}
\inferrule{
  \expandsTU{\uDelta}{\utau}{\tau}\\
  \hastypeU{\emptyset}{\emptyset}{\eparse}{\aparr{\tBody}{\tParseResultExp}}\\\\
  \evalU{\eparse}{\eparse'}\\
  \esyn{\uDelta}{\uGamma}{\uPsi, \uShyp{\tsmv}{x}{\tau}{\eparse'}}{\uPhi}{\ue}{e}{\tau'}
}{
  \esynX{\usyntaxueP{\tsmv}{\utau}{\eparse}{\ue}}{e}{\tau'}
}
\end{equation}
\begin{equation}\label{rule:esyn-S-apuetsm}
\inferrule{
  \uPsi = \uPsi', \uShyp{\tsmv}{x}{\tau}{\eparse}\\\\
  \encodeBody{b}{\ebody}\\
  \evalU{\ap{\eparse}{\ebody}}{\aein{\mathtt{SuccessE}}{\ecand}}\\
  \decodeCondE{\ecand}{\ce}\\\\
    \segOK{\segof{\ce}}{b}\\
  \cana{\emptyset}{\emptyset}{\esceneUP{\uDelta}{\uGamma}{\uPsi}{\uPhi}{b}}{\ce}{e}{\tau}
}{
  \esyn{\uDelta}{\uGamma}{\uPsi}{\uPhi}{\utsmap{\tsmv}{b}}{e}{\tau}
}
\end{equation}
\begin{equation}\label{rule:esyn-S-implicite}
  \inferrule{
    \uPsi = \uASI{\uA \uplus \vExpands{\tsmv}{x}}{\Psi, \xuetsmbnd{x}{\tau}{\eparse}}{\uI}\\\\
    \esyn{\uDelta}{\uGamma}{\uASI{\uA \uplus \vExpands{\tsmv}{x}}{\Psi, \xuetsmbnd{x}{\tau}{\eparse}}{\uI \uplus \designate{\tau}{a}}}{\uPhi}{\ue}{e}{\tau'}
  }{
    \esyn{\uDelta}{\uGamma}{\uPsi}{\uPhi}{\implicite{\tsmv}{\ue}}{e}{\tau'}
  }
\end{equation}
\begin{equation}\label{rule:esyn-S-defuptsm}
\inferrule{
  \expandsTU{\uDelta}{\utau}{\tau}\\
  \hastypeU{\emptyset}{\emptyset}{\eparse}{\aparr{\tBody}{\tParseResultPat}}\\\\
  \evalU{\eparse}{\eparse'}\\
  \esyn{\uDelta}{\uGamma}{\uPsi}{\uPhi, \uPhyp{\tsmv}{x}{\tau}{\eparse'}}{\ue}{e}{\tau'}
}{
  \esynX{\usyntaxup{\tsmv}{\utau}{\eparse}{\ue}}{e}{\tau'}
}
\end{equation}
\begin{equation}\label{rule:esyn-S-implicitp}
  \inferrule{
    \uPhi = \uASI{\uA\uplus\vExpands{\tsmv}{x}}{\Phi, \xuptsmbnd{a}{\tau}{\eparse}}{\uI}\\\\
    \esyn{\uDelta}{\uGamma}{\uPsi}{\uASI{\uA\uplus\vExpands{\tsmv}{x}}{\Phi, \xuptsmbnd{a}{\tau}{\eparse}}{\uI \uplus \designate{\tau}{a}}}{\ue}{e}{\tau'}
  }{
    \esyn{\uDelta}{\uGamma}{\uPsi}{\uPhi}{\implicitp{\tsmv}{\ue}}{e}{\tau'}
  }
\end{equation}
\end{subequations}

\noindent\fbox{$\eana{\uDelta}{\uGamma}{\uPsi}{\uPhi}{\ue}{e}{\tau}$}~~$\ue$ has expansion $e$ when analyzed against type $\tau$

\begin{subequations}\label{rules:eana-S}
\begin{equation}\label{rule:eana-S-subsume}
  \inferrule{
    \esynX{\ue}{e}{\tau}
  }{
    \eanaX{\ue}{e}{\tau}
  }
\end{equation}
\begin{equation}\label{rule:eana-S-let}
  \inferrule{
    \esynX{\ue}{e}{\tau}\\
    \eana{\uDelta}{\uGamma, \uGhyp{\ux}{x}{\tau}}{\uPsi}{\uPhi}{\ue'}{e'}{\tau'}
  }{
    \eanaX{\letsyn{\ux}{\ue}{\ue'}}{\aeap{\aelam{\tau}{x}{e'}}{e}}{\tau'}
  }
\end{equation}
\begin{equation}\label{rule:eana-S-tlam}
  \inferrule{
    \eana{\uDelta, \uDhyp{\ut}{t}}{\uGamma}{\uPsi}{\uPhi}{\ue}{e}{\tau}
  }{
    \eanaX{\Lam{\ut}{\ue}}{\aetlam{t}{e}}{\aall{t}{\tau}}
  }
\end{equation}
\begin{equation}\label{rule:eana-S-fold}
  \inferrule{
    \eanaX{\ue}{e}{[\arec{t}{\tau}/t]\tau}
  }{
    \eanaX{\fold{\ue}}{\aefold{e}}{\arec{t}{\tau}}
  }
\end{equation}
\begin{equation}\label{rule:eana-S-tpl}
  \inferrule{
    \{\eanaX{\ue_i}{e_i}{\tau_i}\}_{i \in \labelset}
  }{
    \eanaX{\tpl{\mapschema{\ue}{i}{\labelset}}}{\aetpl{\labelset}{\mapschema{e}{i}{\labelset}}}{\aprod{\labelset}{\mapschema{\tau}{i}{\labelset}}}
  }
\end{equation}
\begin{equation}\label{rule:eana-S-in}
  \inferrule{
    \eanaX{\ue}{e}{\tau'}
  }{
    \eanaX{\inj{\ell}{\ue}}{\aein{\ell}{e}}{\asum{\labelset, \ell}{\mapschema{\tau}{i}{\labelset}; \mapitem{\ell}{\tau'}}}
  }
\end{equation}
\begin{equation}\label{rule:eana-S-match}
  \inferrule{
    \esynX{\ue}{e}{\tau}\\
    \{\ranaX{\urv_i}{r_i}{\tau}{\tau'}\}_{1 \leq i \leq n}
  }{
    \eanaX{\matchwith{\ue}{\seqschemaX{\urv}}}{\aematchwith{n}{e}{\seqschemaX{r}}}{\tau'}
  }
\end{equation}
\begin{equation}\label{rule:eana-S-defuetsm}
\inferrule{
  \expandsTU{\uDelta}{\utau}{\tau}\\
  \hastypeU{\emptyset}{\emptyset}{\eparse}{\aparr{\tBody}{\tParseResultExp}}\\\\
  \evalU{\eparse}{\eparse'}\\
  \eana{\uDelta}{\uGamma}{\uPsi, \uShyp{\tsmv}{x}{\tau}{\eparse'}}{\uPhi}{\ue}{e}{\tau'}
}{
  \eanaX{\usyntaxueP{\tsmv}{\utau}{\eparse}{\ue}}{e}{\tau'}
}
\end{equation}
\begin{equation}\label{rule:eana-S-implicite}
  \inferrule{
    \uPsi = \uASI{\uA \uplus \vExpands{\tsmv}{x}}{\Psi, \xuetsmbnd{x}{\tau}{\eparse}}{\uI}\\\\
    \eana{\uDelta}{\uGamma}{\uASI{\uA \uplus \vExpands{\tsmv}{x}}{\Psi, \xuetsmbnd{x}{\tau}{\eparse}}{\uI \uplus \designate{\tau}{a}}}{\uPhi}{\ue}{e}{\tau'}
  }{
    \eana{\uDelta}{\uGamma}{\uPsi}{\uPhi}{\implicite{\tsmv}{\ue}}{e}{\tau'}
  }
\end{equation}
\begin{equation}\label{rule:eana-S-lit}
  \inferrule{
    \uPsi = \uASI{\uA}{\Psi, \xuetsmbnd{x}{\tau}{\eparse}}{\uI \uplus \designate{\tau}{a}}\\\\
  \encodeBody{b}{\ebody}\\
  \evalU{\ap{\eparse}{\ebody}}{\aein{\mathtt{SuccessE}}{\ecand}}\\
  \decodeCondE{\ecand}{\ce}\\\\
    \segOK{\segof{\ce}}{b}\\
  \cana{\emptyset}{\emptyset}{\esceneUP{\uDelta}{\uGamma}{\uPsi}{\uPhi}{b}}{\ce}{e}{\tau}
  }{
    \eana{\uDelta}{\uGamma}{\uPsi}{\uPhi}{\lit{b}}{e}{\tau}
  }
\end{equation}
\begin{equation}\label{rule:eana-S-defuptsm}
\inferrule{
  \expandsTU{\uDelta}{\utau}{\tau}\\
  \hastypeU{\emptyset}{\emptyset}{\eparse}{\aparr{\tBody}{\tParseResultPat}}\\\\
  \evalU{\eparse}{\eparse'}\\
  \eana{\uDelta}{\uGamma}{\uPsi}{\uPhi, \uPhyp{\tsmv}{x}{\tau}{\eparse'}}{\ue}{e}{\tau'}
}{
  \eanaX{\usyntaxup{\tsmv}{\utau}{\eparse}{\ue}}{e}{\tau'}
}
\end{equation}
\begin{equation}\label{rule:eana-S-implicitp}
  \inferrule{
    \uPhi = \uASI{\uA\uplus\vExpands{\tsmv}{x}}{\Phi, \xuptsmbnd{a}{\tau}{\eparse}}{\uI}\\\\
    \eana{\uDelta}{\uGamma}{\uPsi}{\uASI{\uA\uplus\vExpands{\tsmv}{x}}{\Phi, \xuptsmbnd{a}{\tau}{\eparse}}{\uI \uplus \designate{\tau}{a}}}{\ue}{e}{\tau'}
  }{
    \eana{\uDelta}{\uGamma}{\uPsi}{\uPhi}{\implicitp{\tsmv}{\ue}}{e}{\tau'}
  }
\end{equation}
\end{subequations}

\noindent\fbox{$\rana{\uDelta}{\uGamma}{\uPsi}{\uPhi}{\urv}{r}{\tau}{\tau'}$}~~$\urv$ has expansion $r$ taking values of type $\tau$ to values of type $\tau'$

\begin{equation}\label{rule:rana-S}
  \inferrule{
    \patExpands{\uGG{\uG'}{\Gamma'}}{\uPhi}{\upv}{p}{\tau}\\
    \eana{\uDD{\uD}{\Delta}}{\uGG{\uG \uplus \uG'}{\Gamma \cup \Gamma'}}{\uPsi}{\uPhi}{\ue}{e}{\tau'}
  }{
    \rana{\uDD{\uD}{\Delta}}{\uGG{\uG}{\Gamma}}{\uPsi}{\uPhi}{\matchrule{\upv}{\ue}}{\aematchrule{p}{e}}{\tau}{\tau'}
  }
\end{equation}

\subsubsection{Pattern Expansion}
\noindent\fbox{$\patExpands{\upctx}{\uPhi}{\upv}{p}{\tau}$}~~$\upv$ has expansion $p$ matching against $\tau$ generating hypotheses $\upctx$

\begin{subequations}\label{rules:patExpands-B}
\begin{equation}\label{rule:patExpands-B-var}
\inferrule{ }{
  \patExpands{\uGG{\vExpands{\ux}{x}}{\Ghyp{x}{\tau}}}{\uPhi}{\ux}{x}{\tau}
}
\end{equation}
\begin{equation}\label{rule:patExpands-B-wild}
\inferrule{ }{
  \patExpands{\uGG{\emptyset}{\emptyset}}{\uPhi}{\wildp}{\aewildp}{\tau}
}
\end{equation}
\begin{equation}\label{rule:patExpands-B-fold}
\inferrule{ 
  \patExpands{\upctx}{\uPhi}{\upv}{p}{[\arec{t}{\tau}/t]\tau}
}{
  \patExpands{\upctx}{\uPhi}{\foldp{\upv}}{\aefoldp{p}}{\arec{t}{\tau}}
}
\end{equation}
\begin{equation}\label{rule:patExpands-B-tpl}
\inferrule{
    \tau = \aprod{\labelset}{\mapschema{\tau}{i}{\labelset}}\\\\
  \{\patExpands{{\upctx_i}}{\uPhi}{\upv_i}{p_i}{\tau_i}\}_{i \in \labelset}\\
}{
    \patExpands{\GIconsi{i \in \labelset}{\upctx_i}}{\uPhi}{\tplp{\mapschema{\upv}{i}{\labelset}}}{\aetplp{\labelset}{\mapschema{p}{i}{\labelset}}}{\tau}
}
\end{equation}
\begin{equation}\label{rule:patExpands-B-in}
\inferrule{
  \patExpands{\upctx}{\uPhi}{\upv}{p}{\tau}
}{
  \patExpands{\upctx}{\uPhi}{\injp{\ell}{\upv}}{\aeinjp{\ell}{p}}{\asum{\labelset, \ell}{\mapschema{\tau}{i}{\labelset}; \mapitem{\ell}{\tau}}}
}
\end{equation}
\begin{equation}\label{rule:patExpands-B-apuptsm}
\inferrule{
  \uPhi = \uPhi', \uPhyp{\tsmv}{x}{\tau}{\eparse}\\\\
  \encodeBody{b}{\ebody}\\
  \evalU{\ap{\eparse}{\ebody}}{\aein{\mathtt{SuccessP}}{\ecand}}\\
  \decodeCEPat{\ecand}{\cpv}\\\\
    \segOK{\segof{\cpv}}{b}\\
  \cvalidP{\upctx}{\pscene{\uDelta}{\uPhi}{b}}{\cpv}{p}{\tau}
}{
  \patExpands{\upctx}{\uPhi}{\utsmap{\tsmv}{b}}{p}{\tau}
}
\end{equation}
\begin{equation}\label{rule:patExpands-B-lit}
\inferrule{
  \uPhi = \uASI{\uA}{\Phi, \xuptsmbnd{a}{\tau}{\eparse}}{\uI, \designate{\tau}{a}}\\\\
  \encodeBody{b}{\ebody}\\
  \evalU{\ap{\eparse}{\ebody}}{\aein{\mathtt{SuccessP}}{\ecand}}\\
  \decodeCEPat{\ecand}{\cpv}\\\\
    \segOK{\segof{\cpv}}{b}\\
  \cvalidP{\upctx}{\pscene{\uDelta}{\uPhi}{b}}{\cpv}{p}{\tau}
}{
  \patExpands{\upctx}{\uPhi}{\lit{b}}{p}{\tau}
}
\end{equation}
\end{subequations}

\section{Proto-Expansion Validation}
\subsection{Syntax of Proto-Expansions}
The syntax of proto-expansions was defined in Sec. \ref{appendix:proto-expansions-SES}.

% \[\begin{array}{lllllll}
% \textbf{Sort} & & & \textbf{Operational Form} & \textbf{Stylized Form} & \textbf{Description}\\
% \mathsf{PrTyp} & \ctau & ::= & \cdots & \cdots & \text{(as in $\miniVersePat$)}\\
% % &&& \aceparr{\ctau}{\ctau} & \parr{\ctau}{\ctau} & \text{partial function}\\
% % &&& \aceall{t}{\ctau} & \forallt{t}{\ctau} & \text{polymorphic}\\
% % &&& \acerec{t}{\ctau} & \rect{t}{\ctau} & \text{recursive}\\
% % &&& \aceprod{\labelset}{\mapschema{\ctau}{i}{\labelset}} & \prodt{\mapschema{\ctau}{i}{\labelset}} & \text{labeled product}\\
% % &&& \acesum{\labelset}{\mapschema{\ctau}{i}{\labelset}} & \sumt{\mapschema{\ctau}{i}{\labelset}} & \text{labeled sum}\\
% %\LCC &&& \gray & \gray & \gray\\
% % &&& \acesplicedt{m}{n} & \splicedt{m}{n} & \text{spliced}\\%\ECC
% \mathsf{PrExp} & \ce & ::= & \cdots & \cdots & \text{(as in $\miniVersePat$)}\\
% &&& \aceasc{\ctau}{\ce} & \asc{\ce}{\ctau} & \text{ascription}\\
% &&& \aceletsyn{x}{\ce}{\ce} & \letsyn{x}{\ce}{\ce} & \text{value binding}\\
% % &&& \aceanalam{x}{\ce} & \analam{x}{\ce} & \text{abstraction (unannotated)}\\
% % &&& \acelam{\ctau}{x}{\ce} & \lam{x}{\ctau}{\ce} & \text{abstraction (annotated)}\\
% % &&& \aceap{\ce}{\ce} & \ap{\ce}{\ce} & \text{application}\\
% % &&& \acetlam{t}{\ce} & \Lam{t}{\ce} & \text{type abstraction}\\
% % &&& \acetap{\ce}{\ctau} & \App{\ce}{\ctau} & \text{type application}\\
% % &&& \aceanafold{\ce} & \fold{\ce} & \text{fold}\\
% % &&& \aceunfold{\ce} & \unfold{\ce} & \text{unfold}\\
% % &&& \acetpl{\labelset}{\mapschema{\ce}{i}{\labelset}} & \tpl{\mapschema{\ce}{i}{\labelset}} & \text{labeled tuple}\\
% % &&& \acepr{\ell}{\ce} & \prj{\ce}{\ell} & \text{projection}\\
% % &&& \aceanain{\ell}{\ce} & \inj{\ell}{\ce} & \text{injection}\\
% % &&& \acematchwithb{n}{\ce}{\seqschemaX{\urv}} & \matchwith{\ce}{\seqschemaX{\crv}} & \text{match}\\%\LCC &&& \gray & \gray & \gray\\
% % &&& \acesplicede{m}{n} & \splicede{m}{n} & \text{spliced}\\%\ECC
% % &&& \acesplicedet{m}{n}{\ctau} & \splicedet{m}{n}{\ctau} & \text{spliced (analytic)}\\
% \mathsf{PrRule} & \crv & ::= & \cdots & \cdots & \text{(as in $\miniVersePat$)}\\
% \mathsf{PrPat} & \cpv & ::= & \cdots & \cdots & \text{(as in $\miniVersePat$)}\\
% % &&& \acefoldp{p} & \foldp{p} & \text{fold pattern}\\
% % &&& \acetplp{\labelset}{\mapschema{\cpv}{i}{\labelset}} & \tplp{\mapschema{\cpv}{i}{\labelset}} & \text{labeled tuple pattern}\\
% % &&& \aceinjp{\ell}{\cpv} & \injp{\ell}{\cpv} & \text{injection pattern}\\
% % &&& \acesplicedp{m}{n} & \splicedp{m}{n} & \text{spliced}
% \end{array}\]

\subsubsection{Common Proto-Expansion Terms}
Each expanded term, except variable patterns, maps onto a proto-expansion term. We refer to these as the \emph{common proto-expansion terms}. In particular:
\begin{itemize}
  \item Each type, $\tau$, maps onto a proto-type, $\Cof{\tau}$, as follows:
  \[\arraycolsep=1pt\begin{array}{rl}
  \Cof{t} & = t\\
  \Cof{\aparr{\tau_1}{\tau_2}} & = \aceparr{\Cof{\tau_1}}{\Cof{\tau_2}}\\
  \Cof{\aall{t}{\tau}} & = \aceall{t}{\Cof{\tau}}\\
  \Cof{\arec{t}{\tau}} & = \acerec{t}{\Cof{\tau}}\\
  \Cof{\aprod{\labelset}{\mapschema{\tau}{i}{\labelset}}} & = \aceprod{\labelset}{\mapschemax{\Cofv}{\tau}{i}{\labelset}}\\
  \Cof{\asum{\labelset}{\mapschema{\tau}{i}{\labelset}}} & = \acesum{\labelset}{\mapschemax{\Cofv}{\tau}{i}{\labelset}}
  \end{array}\]
  \item Each expanded expression, $e$, maps onto a proto-expression, $\Cof{e}$, as follows:
  \[\arraycolsep=1pt\hspace{-15px}\begin{array}{rl}
  \Cof{x} & = x\\
  \Cof{\aelam{\tau}{x}{e}} & = \acelam{\Cof{\tau}}{x}{\Cof{e}}\\
  \Cof{\aeap{e_1}{e_2}} & = \aceap{\Cof{e_1}}{\Cof{e_2}}\\
  \Cof{\aetlam{t}{e}} & = \acetlam{t}{\Cof{e}}\\
  \Cof{\aetap{e}{\tau}} & = \acetap{\Cof{e}}{\Cof{\tau}}\\
  \Cof{\aefold{e}} & = \aceasc{\acerec{t}{\Cof{\tau}}}{\acefold{\Cof e}}\\
  \Cof{\aeunfold{e}} & = \aceunfold{\Cof{e}}\\
  \Cof{\aetpl{\labelset}{\mapschema{e}{i}{\labelset}}} & = \acetpl{\labelset}{\mapschemax{\Cofv}{e}{i}{\labelset}}\\
  \Cof{\aein{\ell}{e}} &= \aceasc
    {
      \acesum{\labelset}{\mapschemax{\Cofv}{\tau}{i}{\labelset}}
    }{\acein{\ell}{\Cof{e}}}\\
  \Cof{\aematchwith{n}{e}{\seqschemaX{r}}} & = \aceasc{\Cof{\tau}}{\acematchwith{n}{\Cof{e}}{\seqschemaXx{\Cofv}{r}}}
  \end{array}\]
  \item Each expanded rule, $r$, maps onto the proto-rule, $\Cof{r}$, as follows:
  \begin{align*}
  \Cof{\aematchrule{p}{e}} & = \acematchrule{p}{\Cof{e}}
  \end{align*}
  \item Each expanded pattern, $p$, except for the variable patterns, maps onto a proto-pattern, $\Cof{p}$, as follows:
  \begin{align*}
  \Cof{\aewildp} & = \acewildp\\
  \Cof{\aefoldp{p}} & = \acefoldp{\Cof{p}}\\
  \Cof{\aetplp{\labelset}{\mapschema{p}{i}{\labelset}}} & = \acetplp{\labelset}{\mapschemax{\Cofv}{p}{i}{\labelset}}\\
  \Cof{\aeinjp{\ell}{p}} & = \aceinjp{\ell}{\Cof{p}}
  \end{align*}
\end{itemize}

These definitions differ from those given in Sec. \ref{appendix:proto-expansions-SES} in that they include the type information necessary for bidirectional typechecking.

\subsubsection{Proto-Expression Encoding and Decoding}
The type $\tCEExp$ and the judgements $\encodeCondE{\ce}{e}$ and $\decodeCondE{e}{\ce}$ are characterized as described in Sec. \ref{appendix:proto-expansions-SES}.

\subsubsection{Proto-Pattern Encoding and Decoding}
The type $\tCEPat$ and the judgements $\encodeCEPat{\cpv}{e}$ and $\decodeCEPat{e}{\cpv}$ are characterized as described in Sec. \ref{appendix:proto-expansions-SES}.

\subsubsection{Splice Summaries}
The \emph{splice summary} of a proto-expression, $\segof{\ce}$, or proto-pattern, $\segof{\cpv}$, is the finite set of references to spliced types, expressions {and patterns} that it mentions.

\subsubsection{Segmentations}
A \emph{segment set}, $\psi$, is a finite set of pairs of natural numbers indicating the locations of spliced terms. The \emph{segmentation} of a proto-expression, $\segof{\ce}$, or proto-pattern, $\segof{\cpv}$, is the segment set implied by its splice summary.

\subsection{Proto-Expansion Validation}\label{appendix:proto-expansion-validation-BS}
\subsubsection{Proto-Type Validation}
The \emph{proto-type validation judgement}, $\cvalidT{\Delta}{\tscenev}{\ctau}{\tau}$, is inductively defined by Rules (\ref{rules:cvalidT-U}), which were defined in Sec. \ref{appendix:proto-type-validation-SES}.

\subsubsection{Bidirectional Proto-Expression and Proto-Rule Validation}
\emph{Expression splicing scenes}, $\escenev$, are of the form $\esceneUP{\uDelta}{\uGamma}{\uPsi}{\uPhi}{b}$. We write $\tsfrom{\escenev}$ for the type splicing scene constructed by dropping unnecessary contexts from $\escenev$:
\[\tsfrom{\esceneUP{\uDelta}{\uGamma}{\uPsi}{\uPhi}{b}} = \tsceneUP{\uDelta}{b}\]

\noindent\fbox{$\strut\csynX{\ce}{e}{\tau}$}~~$\ce$ has expansion $e$ synthesizing type $\tau$
\begin{subequations}\label{rules:csyn}
\begin{equation}\label{rule:csyn-var}
  \inferrule{ }{ 
    \csyn{\Delta}{\Gamma, \Ghyp{x}{\tau}}{\escenev}{x}{x}{\tau}
  }
\end{equation}
\begin{equation}\label{rule:csyn-asc}
  \inferrule{
    \cvalidT{\Delta}{\tsfrom{\escenev}}{\ctau}{\tau}\\
    \canaX{\ce}{e}{\tau}
  }{
    \csynX{\aceasc{\ctau}{\ce}}{e}{\tau}
  }
\end{equation}
\begin{equation}\label{rule:csyn-let}
  \inferrule{
    \csynX{\ce}{e}{\tau}\\
    \csyn{\Delta}{\Gamma, \Ghyp{x}{\tau}}{\escenev}{\ce'}{e'}{\tau'}
  }{
    \csynX{\aceletsyn{x}{\ce}{\ce'}}{\aeap{\aelam{\tau}{x}{e'}}{e}}{\tau'}
  }
\end{equation}
\begin{equation}\label{rule:csyn-lam}
  \inferrule{
    \cvalidT{\Delta}{\tsfrom{\escenev}}{\ctau_1}{\tau_1}\\
    \csyn{\Delta}{\Gamma, \Ghyp{x}{\tau_1}}{\escenev}{\ce}{e}{\tau_2}
  }{
    \csynX{\acelam{\ctau_1}{x}{\ce}}{\aelam{\tau_1}{x}{e}}{\aparr{\tau_1}{\tau_2}}
  }
\end{equation}
\begin{equation}\label{rule:csyn-ap}
  \inferrule{
    \csynX{\ce_1}{e_1}{\aparr{\tau_2}{\tau}}\\
    \canaX{\ce_2}{e_2}{\tau_2}
  }{
    \csynX{\aceap{\ce_1}{\ce_2}}{\aeap{e_1}{e_2}}{\tau}
  }
\end{equation}
\begin{equation}\label{rule:csyn-tlam}
  \inferrule{
    \csyn{\Delta, \Dhyp{t}}{\Gamma}{\escenev}{\ce}{e}{\tau}
  }{
    \csynX{\acetlam{t}{\ce}}{\aetlam{t}{e}}{\aall{t}{\tau}}
  }
\end{equation}
\begin{equation}\label{rule:csyn-tap}
  \inferrule{
    \csynX{\ce}{e}{\aall{t}{\tau}}\\
    \cvalidT{\Delta}{\tsfrom{\escenev}}{\ctau'}{\tau'}
  }{
    \csynX{\acetap{\ce}{\ctau'}}{\aetap{e}{\tau'}}{[\tau'/t]\tau}
  }
\end{equation}
\begin{equation}\label{rule:csyn-unfold}
  \inferrule{
    \csynX{\ce}{e}{\arec{t}{\tau}}
  }{
    \csynX{\aceunfold{\ce}}{\aeunfold{e}}{[\arec{t}{\tau}/t]\tau}
  }
\end{equation}
\begin{equation}\label{rule:csyn-tpl}
  \inferrule{
    \tau = \aprod{\labelset}{\mapschema{\tau}{i}{\labelset}}\\\\
    \{\csynX{\ce_i}{e_i}{\tau_i}\}_{i \in \labelset}
  }{
    \csynX{\acetpl{\labelset}{\mapschema{\ce}{i}{\labelset}}}{\aetpl{\labelset}{\mapschema{e}{i}{\labelset}}}{\tau}
  }
\end{equation}
\begin{equation}\label{rule:csyn-pr}
  \inferrule{
    \csynX{\ce}{e}{\aprod{\labelset, \ell}{\mapschema{\tau}{i}{\labelset}; \mapitem{\ell}{\tau}}}
  }{
    \csynX{\acepr{\ell}{\ce}}{\aepr{\ell}{e}}{\tau}
  }
\end{equation}
% \begin{equation}\label{rule:csyn-match}
%   \inferrule{
%     n > 0\\
%     \csynX{\ce}{e}{\tau}\\
%     \{\crsynX{\crv_i}{r_i}{\tau}{\tau'}\}_{1 \leq i \leq n}
%   }{
%     \csynX{\acematchwithb{n}{\ce}{\seqschemaX{\crv}}}{\aematchwith{n}{e}{\seqschemaX{r}}}{\tau'}
%   }
% \end{equation}
\begin{equation}\label{rule:csyn-splicede}
\inferrule{
  \cvalidT{\emptyset}{\tsfrom{\escenev}}{\ctau}{\tau}\\
  \escenev=\esceneUP{\uDD{\uD}{\Delta_\text{app}}}{\uGG{\uG}{\Gamma_\text{app}}}{\uPsi}{\uPhi}{b}\\
  \parseUExp{\bsubseq{b}{m}{n}}{\ue}\\
  \eana{\uDD{\uD}{\Delta_\text{app}}}{\uGG{\uG}{\Gamma_\text{app}}}{\uPsi}{\uPhi}{\ue}{e}{\tau}\\\\
  \Delta \cap \Delta_\text{app} = \emptyset\\
  \domof{\Gamma} \cap \domof{\Gamma_\text{app}} = \emptyset
}{
  \csyn{\Delta}{\Gamma}{\escenev}{\acesplicede{m}{n}{\ctau}}{e}{\tau}
}
\end{equation}
\end{subequations}

\noindent\fbox{$\strut\canaX{\ce}{e}{\tau}$}~~$\ce$ has expansion $e$ when analyzed against type $\tau$
\begin{subequations}\label{rules:cana}
\begin{equation}\label{rule:cana-subsume}
  \inferrule{
    \csynX{\ce}{e}{\tau}
  }{
    \canaX{\ce}{e}{\tau}
  }
\end{equation}
\begin{equation}\label{rule:cana-let}
  \inferrule{
    \csynX{\ce}{e}{\tau}\\
    \cana{\Delta}{\Gamma, \Ghyp{x}{\tau}}{\escenev}{\ce'}{e'}{\tau'}
  }{
    \canaX{\aceletsyn{x}{\ce}{\ce'}}{\aeap{\aelam{\tau}{x}{e'}}{e}}{\tau'}
  }
\end{equation}
% \begin{equation}\label{rule:cana-analam}
%   \inferrule{
%     \cana{\Delta}{\Gamma, \Ghyp{x}{\tau_1}}{\escenev}{\ce}{e}{\tau_2}
%   }{
%     \canaX{\aceanalam{x}{\ue}}{\aelam{\tau_1}{x}{e}}{\aparr{\tau_1}{\tau_2}}
%   }
% \end{equation}
\begin{equation}\label{rule:cana-tlam}
  \inferrule{
    \cana{\Delta, \Dhyp{t}}{\Gamma}{\escenev}{\ce}{e}{\tau}
  }{
    \canaX{\acetlam{t}{\ce}}{\aetlam{t}{e}}{\aall{t}{\tau}}
  }
\end{equation}
\begin{equation}\label{rule:cana-fold}
  \inferrule{
    \canaX{\ce}{e}{[\arec{t}{\tau}/t]\tau}
  }{
    \canaX{\aceanafold{\ce}}{\aefold{e}}{\arec{t}{\tau}}
  }
\end{equation}
\begin{equation}\label{rule:cana-tpl}
  \inferrule{
    \tau = \aprod{\labelset}{\mapschema{\tau}{i}{\labelset}}\\\\
    \{\canaX{\ce_i}{e_i}{\tau_i}\}_{i \in \labelset}
  }{
    \canaX{\acetpl{\labelset}{\mapschema{\ce}{i}{\labelset}}}{\aetpl{\labelset}{\mapschema{e}{i}{\labelset}}}{\tau}
  }
\end{equation}
\begin{equation}\label{rule:cana-in}
  \inferrule{
    \canaX{\ce}{e}{\tau'}
  }{
    % \left(\shortstack{$\Delta~\Gamma \vdash^{\escenev} \aceanain{\ell}{\ce}$\\$\leadsto$\\$\aein{\ell} \Leftarrow \asum{\labelset, \ell}{\mapschema{\tau}{i}{\labelset}; \mapitem{\ell}{\tau}}$\vspace{-1.2em}}\right)
    \canaX{\acein{\ell}{\ce}}{\aein{\ell}{e}}{\asum{\labelset, \ell}{\mapschema{\tau}{i}{\labelset}; \mapitem{\ell}{\tau'}}}
  }
\end{equation}
\begin{equation}\label{rule:cana-match}
  \inferrule{
    \csynX{\ce}{e}{\tau}\\
    \{\cranaX{\crv_i}{r_i}{\tau}{\tau'}\}_{1 \leq i \leq n}
  }{
    \canaX{\acematchwithb{n}{\ce}{\seqschemaX{\crv}}}{\aematchwith{n}{e}{\seqschemaX{r}}}{\tau'}
  }
\end{equation}
\end{subequations}

\noindent\fbox{$\strut\crana{\Delta}{\Gamma}{\escenev}{\crv}{r}{\tau}{\tau'}$}~~$\crv$ has expansion $r$ taking values of type $\tau$ to values of type $\tau'$
\begin{equation}\label{rule:crana}
\inferrule{
  \patType{\pctx}{p}{\tau}\\
  \cana{\Delta}{\Gcons{\Gamma}{\pctx}}{\escenev}{\ce}{e}{\tau'}
}{
  \crana{\Delta}{\Gamma}{\escenev}{\acematchrule{p}{\ce}}{\aematchrule{p}{e}}{\tau}{\tau'}
}
\end{equation}

\subsubsection{Proto-Pattern Validation}
\emph{Pattern splicing scenes}, $\pscenev$, are of the form $\pscene{\uDelta}{\uPhi}{b}$.

\vspace{10px}\noindent\fbox{\strut$\cvalidP{\upctx}{\pscenev}{\cpv}{p}{\tau}$}~~$\cpv$ has expansion $p$ matching against $\tau$ generating hypotheses $\upctx$
\begin{subequations}\label{rules:cvalidP-B}
\begin{equation}\label{rule:cvalidP-B-wild}
\inferrule{ }{
  \cvalidP{\uGG{\emptyset}{\emptyset}}{\pscenev}{\acewildp}{\aewildp}{\tau}
}
\end{equation}
\begin{equation}\label{rule:cvalidP-B-fold}
\inferrule{
  \cvalidP{\upctx}{\pscenev}{\cpv}{p}{[\arec{t}{\tau}/t]\tau}
}{
  \cvalidP{\upctx}{\pscenev}{\acefoldp{\cpv}}{\aefoldp{p}}{\arec{t}{\tau}}
}
\end{equation}
\begin{equation}\label{rule:cvalidP-B-tpl}
\inferrule{
  \tau = \aprod{\labelset}{\mapschema{\tau}{i}{\labelset}}\\\\
  \{\cvalidP{\upctx_i}{\pscenev}{\cpv_i}{p_i}{\tau_i}\}_{i \in \labelset}
}{
% \left(\shortstack{$\vdash^{\pscenev} \acetplp{\labelset}{\mapschema{\cpv}{i}{\labelset}}$\\$\leadsto$\\$\aetplp{\labelset}{\mapschema{p}{i}{\labelset}} : \aprod{\labelset}{\mapschema{\tau}{i}{\labelset}}~\dashVx^{\,\Gconsi{i \in \labelset}{\upctx_i}}$\vspace{-1.2em}}\right)
  \cvalidP{\GIconsi{i \in \labelset}{\upctx_i}}{\pscenev}{\acetplp{\labelset}{\mapschema{\cpv}{i}{\labelset}}}{\aetplp{\labelset}{\mapschema{p}{i}{\labelset}}}{\tau}
}
\end{equation}
\begin{equation}\label{rule:cvalidP-B-in}
\inferrule{
  \cvalidP{\upctx}{\pscenev}{\cpv}{p}{\tau}
}{
  \cvalidP{\upctx}{\pscenev}{\aceinjp{\ell}{\cpv}}{\aeinjp{\ell}{p}}{\asum{\labelset, \ell}{\mapschema{\tau}{i}{\labelset}; \mapitem{\ell}{\tau}}}
}
\end{equation}
\begin{equation}\label{rule:cvalidP-B-spliced}
\inferrule{
  \cvalidT{\emptyset}{\tsceneUP{\uDelta}{b}}{\ctau}{\tau}\\
  \parseUPat{\bsubseq{b}{m}{n}}{\upv}\\
  \patExpands{\upctx}{\uPhi}{\upv}{p}{\tau}
}{
  \cvalidP{\upctx}{\pscene{\uDelta}{\uPhi}{b}}{\acesplicedp{m}{n}{\ctau}}{p}{\tau}
}
\end{equation}
\end{subequations}
\section{Metatheory}\label{appendix:B-metatheory}
\subsection{Typed Pattern Expansion}\label{appendix:B-typed-pattern-expansion}
\begin{theorem}[Typed Pattern Expansion]\label{thm:typed-pattern-expansion-B} ~
\begin{enumerate}
  \item If $\pExpandsSP{\uDD{\uD}{\Delta}}{\uASI{\uA}{\Phi}{\uI}}{\upv}{p}{\tau}{\uGG{\uG}{\pctx}}$ then $\patType{\pctx}{p}{\tau}$.
  \item If $\cvalidP{\uGG{\uG}{\pctx}}{\pscene{\uDD{\uD}{\Delta}}{\uAP{\uA}{\Phi}}{b}}{\cpv}{p}{\tau}$ then $\patType{\pctx}{p}{\tau}$.
\end{enumerate}
\end{theorem}
\begin{proof}
  By mutual rule induction over Rules (\ref{rules:patExpands-B}) and Rules (\ref{rules:cvalidP-B}
  \begin{enumerate}
  \item We induct on the premise. In the following, let $\uDelta=\uDD{\uD}{\Delta}$ and $\upctx=\uGG{\uG}{\pctx}$ and $\uPhi=\uASI{\uA}{\Phi}{\uI}$.
  \begin{byCases}
    \item[\text{(\ref{rule:patExpands-B-var}) \textbf{through} (\ref{rule:patExpands-B-apuptsm})}] These cases follow by identical argument to the corresponding cases of Theorem \ref{thm:typed-pattern-expansion}.

    \item[\text{(\ref{rule:patExpands-B-lit})}]~
    \resetpfcounter
    \begin{pfsteps}
      \item \upv = \lit{b} \BY{assumption}
      \item \Phi = \Phi', \xuptsmbnd{a}{\tau}{\eparse} \BY{assumption}
      \item \uI = \uI', \designate{\tau}{a} \BY{assumption}
      \item   \encodeBody{b}{\ebody} \BY{assumption}
      \item   \evalU{\ap{\eparse}{\ebody}}{\aein{\mathtt{SuccessP}}{\ecand}} \BY{assumption}
      \item  \decodeCEPat{\ecand}{\cpv} \BY{assumption}
      \item  \cvalidP{\upctx}{\pscene{\uDelta}{\uPhi}{b}}{\cpv}{p}{\tau} \BY{assumption} \pflabel{cvalidP}
      \item \patType{\pctx}{p}{\tau} \BY{IH, part 2 on \pfref{cvalidP}}
    \end{pfsteps}
    \resetpfcounter
%     \item[\text{(\ref{rule:patExpands-var})}] ~
%       \begin{pfsteps*}
%         \item $\upv=\ux$ \BY{assumption}
%         \item $p=x$ \BY{assumption}
%         \item $\pctx=\Ghyp{x}{\tau}$ \BY{assumption}
%         \item $\patType{\Ghyp{x}{\tau}}{x}{\tau}$ \BY{Rule (\ref{rule:patType-var})}
%       \end{pfsteps*}
%       \resetpfcounter
%     \item[\text{(\ref{rule:patExpands-wild})}] ~
%       \begin{pfsteps*}
%         \item $p=\aewildp$ \BY{assumption}
%         \item $\pctx = \emptyset$ \BY{assumption}
%         \item $\patType{\emptyset}{\aewildp}{\tau}$ \BY{Rule (\ref{rule:patType-wild})}
%       \end{pfsteps*}
%       \resetpfcounter
%     \item[\text{(\ref{rule:patExpands-fold})}] ~
%       \begin{pfsteps*}
%         \item $\upv=\foldp{\upv'}$ \BY{assumption}
%         \item $p=\aefoldp{p'}$ \BY{assumption}
%         \item $\tau=\arec{t}{\tau'}$ \BY{assumption}
%         %\item $\uptsmenv{\Delta}{\Phi}$ \BY{assumption} \pflabel{env}
%         \item $\patExpands{\upctx}{\uPhi}{\upv'}{p'}{[\arec{t}{\tau'}/t]\tau'}$ \BY{assumption} \pflabel{patExpands}
%         \item $\patType{\pctx}{p'}{[\arec{t}{\tau'}/t]\tau'}$ \BY{IH, part 1 on \pfref{patExpands}} \pflabel{patType}
%         \item $\patType{\pctx}{\aefoldp{p'}}{\arec{t}{\tau'}}$ \BY{Rule (\ref{rule:patType-fold}) on \pfref{patType}}
%       \end{pfsteps*}
%       \resetpfcounter
%     \item[\text{(\ref{rule:patExpands-tpl})}] ~
%       \begin{pfsteps*}
%         \item $\upv=\tplp{\mapschema{\upv}{i}{\labelset}}$ \BY{assumption}
%         \item $p=\aetplp{\labelset}{\mapschema{p}{i}{\labelset}}$ \BY{assumption}
%         \item $\tau=\aprod{\labelset}{\mapschema{\tau}{i}{\labelset}}$ \BY{assumption}
%         \item $\{\patExpands{\uGG{\uG_i}{\pctx_i}}{\uPhi}{\upv_i}{p_i}{\tau_i}\}_{i \in \labelset}$ \BY{assumption} \pflabel{patExpands}
%         \item $\pctx = \Gconsi{i \in \labelset}{\pctx_i}$ \BY{assumption}
%         %\item $\uptsmenv{\Delta}{\Phi}$ \BY{assumption} \pflabel{env}
%         \item $\{\patType{\pctx_i}{p_i}{\tau_i}\}_{i \in \labelset}$ \BY{IH, part 1 over \pfref{patExpands}}\pflabel{patType}
%         \item $\patType{\Gconsi{i \in \labelset}{\pctx_i}}{\aetplp{\labelset}{\mapschema{p}{i}{\labelset}}}{\aprod{\labelset}{\mapschema{\tau}{i}{\labelset}}}$ \BY{Rule (\ref{rule:patType-tpl}) on \pfref{patType}}
%       \end{pfsteps*}
%       \resetpfcounter
%     \item[\text{(\ref{rule:patExpands-in})}] ~
%       \begin{pfsteps*}
%         \item $\upv=\injp{\ell}{\upv'}$ \BY{assumption}
%         \item $p=\aeinjp{\ell}{p'}$ \BY{assumption}
%         \item $\tau=\asum{\labelset, \ell}{\mapschema{\tau}{i}{\labelset}; \mapitem{\ell}{\tau'}}$ \BY{assumption}
%         \item $\patExpands{\upctx}{\uPhi}{\upv'}{p'}{\tau'}$ \BY{assumption} \pflabel{patExpands}
% %        \item $\uptsmenv{\Delta}{\Phi}$ \BY{assumption} \pflabel{env}
%         \item $\patType{\pctx}{p'}{\tau'}$ \BY{IH, part 1 on \pfref{patExpands}} \pflabel{patType}
%         \item $\patType{\pctx}{\aeinjp{\ell}{p'}}{\asum{\labelset, \ell}{\mapschema{\tau}{i}{\labelset}; \mapitem{\ell}{\tau'}}}$ \BY{Rule (\ref{rule:patType-inj}) on \pfref{patType}}
%       \end{pfsteps*}
%       \resetpfcounter
%     \item[\text{(\ref{rule:patExpands-apuptsm})}] ~
%       \begin{pfsteps*}
%         \item $\upv=\utsmap{\tsmv}{b}$ \BY{assumption}
%         \item $\uA=\uA', \vExpands{\tsmv}{x}$ \BY{assumption}
%         \item $\Phi=\Phi', \xuptsmbnd{a}{\tau}{\eparse}$ \BY{assumption}
%         \item $\encodeBody{b}{\ebody}$ \BY{assumption}
%         \item $\evalU{\eparse(\ebody)}{{\lbltxt{SuccessP}}\cdot{\ecand}}$ \BY{assumption}
%         \item $\decodeCEPat{\ecand}{\cpv}$ \BY{assumption}
%         \item $\cvalidP{\uGG{\uG}{\pctx}}{\pscene{\uDelta}{\uAP{\uA}{\Phi}}{b}}{\cpv}{p}{\tau}$ \BY{assumption} \pflabel{cvalidP}
% %        \item $\uptsmenv{\Delta}{\Phi', \xuptsmbnd{a}{\tau}{\eparse}}$ \BY{assumption} \pflabel{env}
%         \item $\patType{\pctx}{p}{\tau}$ \BY{IH, part 2 on \pfref{cvalidP}}
%       \end{pfsteps*}
%       \resetpfcounter
  \end{byCases}

  \item We induct on the premise. All cases follow by identical argument to the corresponding cases of Theorem \ref{thm:typed-pattern-expansion}.
%   \begin{byCases}
%     \item[\text{(\ref{rule:cvalidP-B-wild})}] ~
%       \begin{pfsteps*}
%         \item $p=\aewildp$ \BY{assumption}
%         \item $\pctx=\emptyset$ \BY{assumption}
%         \item $\patType{\emptyset}{\aewildp}{\tau}$ \BY{Rule (\ref{rule:patType-wild})}
%       \end{pfsteps*}
%       \resetpfcounter
%     \item[\text{(\ref{rule:cvalidP-UP-fold})}] ~
%       \begin{pfsteps*}
%         \item $\cpv=\acefoldp{\cpv'}$ \BY{assumption}
%         \item $p=\aefoldp{p'}$ \BY{assumption}
%         \item $\tau=\arec{t}{\tau'}$ \BY{assumption}
%         % \item $\uptsmenv{\Delta}{\Phi}$ \BY{assumption} \pflabel{env}
%         \item $\cvalidP{\upctx}{\pscene{\uDelta}{\uPhi}{b}}{\cpv'}{p'}{[\arec{t}{\tau'}/t]\tau'}$ \BY{assumption} \pflabel{cvalidP}
%         \item $\patType{\pctx}{p'}{[\arec{t}{\tau'}/t]\tau'}$ \BY{IH, part 2 on \pfref{cvalidP}} \pflabel{patType}
%         \item $\patType{\pctx}{\aefoldp{p'}}{\arec{t}{\tau'}}$ \BY{Rule (\ref{rule:patType-fold}) on \pfref{patType}}
%       \end{pfsteps*}
%       \resetpfcounter
%     \item[\text{(\ref{rule:cvalidP-UP-tpl})}] ~
%       \begin{pfsteps*}
%         \item $\cpv=\acetplp{\labelset}{\mapschema{\cpv}{i}{\labelset}}$ \BY{assumption}
%         \item $p=\aetplp{\labelset}{\mapschema{p}{i}{\labelset}}$ \BY{assumption}
%         \item $\tau=\aprod{\labelset}{\mapschema{\tau}{i}{\labelset}}$ \BY{assumption}
%         \item $\{\cvalidP{\uGG{\uG_i}{\pctx_i}}{\pscene{\uDelta}{\uPhi}{b}}{\cpv_i}{p_i}{\tau_i}\}_{i \in \labelset}$ \BY{assumption} \pflabel{cvalidP}
%         \item $\pctx = \Gconsi{i \in \labelset}{\pctx_i}$ \BY{assumption}
%         %\item $\uptsmenv{\Delta}{\Phi}$ \BY{assumption} \pflabel{env}
%         \item $\{\patType{\pctx_i}{p_i}{\tau_i}\}_{i \in \labelset}$ \BY{IH, part 2 over \pfref{cvalidP}}\pflabel{patType}
%         \item $\patType{\Gconsi{i \in \labelset}{\pctx_i}}{\aetplp{\labelset}{\mapschema{p}{i}{\labelset}}}{\aprod{\labelset}{\mapschema{\tau}{i}{\labelset}}}$ \BY{Rule (\ref{rule:patType-tpl}) on \pfref{patType}}
%       \end{pfsteps*}
%       \resetpfcounter
%     \item[\text{(\ref{rule:cvalidP-UP-in})}] ~
%       \begin{pfsteps*}
%         \item $\cpv=\aceinjp{\ell}{\cpv'}$ \BY{assumption}
%         \item $p=\aeinjp{\ell}{p'}$ \BY{assumption}
%         \item $\tau=\asum{\labelset, \ell}{\mapschema{\tau}{i}{\labelset}; \mapitem{\ell}{\tau'}}$ \BY{assumption}
%         \item $\cvalidP{\upctx}{\pscene{\uDelta}{\uPhi}{b}}{\cpv'}{p'}{\tau'}$ \BY{assumption} \pflabel{cvalidP}
% %        \item $\uptsmenv{\Delta}{\Phi}$ \BY{assumption} \pflabel{env}
%         \item $\patType{\pctx}{p'}{\tau'}$ \BY{IH, part 2 on \pfref{cvalidP}} \pflabel{patType}
%         \item $\patType{\pctx}{\aeinjp{\ell}{p'}}{\asum{\labelset, \ell}{\mapschema{\tau}{i}{\labelset}; \mapitem{\ell}{\tau'}}}$ \BY{Rule (\ref{rule:patType-inj}) on \pfref{patType}}
%       \end{pfsteps*}
%       \resetpfcounter
%     \item[\text{(\ref{rule:cvalidP-UP-spliced})}] ~
%       \begin{pfsteps*}
%         \item $\cpv=\acesplicedp{m}{n}{\ctau}$ \BY{assumption}
%         \item $\cvalidT{\emptyset}{\tsceneUP{\uDelta}{b}}{\ctau}{\tau}$ \BY{assumption}
%         \item $\parseUExp{\bsubseq{b}{m}{n}}{\upv}$ \BY{assumption}
%         \item $\patExpands{\upctx}{\uPhi}{\upv}{p}{\tau}$ \BY{assumption} \pflabel{patExpands}
%         \item $\patType{\pctx}{p}{\tau}$ \BY{IH, part 1 on \pfref{patExpands}}
%       \end{pfsteps*}
%       \resetpfcounter
%   \end{byCases}
  \end{enumerate}
The mutual induction can be shown to be well-founded by an argument nearly identical to that that given in the proof of Theorem \ref{thm:typed-pattern-expansion}, differing only in that the appeal to Condition \ref{condition:pattern-parsing} is replaced by an appeal to the analagous Condition \ref{condition:pattern-parsing-BS}.

% showing that the following numeric metric on the judgements that we induct on is decreasing:
% \begin{align*}
% \sizeof{\patExpands{\upctx}{\uPhi}{\upv}{p}{\tau}} & = \sizeof{\upv}\\
% \sizeof{{\cvalidP{\upctx}{\pscene{\uDelta}{\uPhi}{b}}{\cpv}{p}{\tau}}} & = \sizeof{b}
% \end{align*}
% where $\sizeof{b}$ is the length of $b$ and $\sizeof{\upv}$ is the sum of the lengths of the literal bodies in $\upv$, as defined in Sec. \ref{appendix:SES-syntax}.

% The only case in the proof of part 1 that invokes part 2 is Case (\ref{rule:patExpands-apuptsm}). There, we have that the metric remains stable: \begin{align*}
%  & \sizeof{\patExpands{\upctx}{\uPhi}{\utsmap{\tsmv}{b}}{p}{\tau}}\\
% =& \sizeof{{\cvalidP{\upctx}{\pscene{\uDelta}{\uPhi}{b}}{\cpv}{p}{\tau}}}\\
% =&\sizeof{b}\end{align*}

% The only case in the proof of part 2 that invokes part 1 is Case (\ref{rule:cvalidP-UP-spliced}). There, we have that $\parseUPat{\bsubseq{b}{m}{n}}{\upv}$ and the IH is applied to the judgement $\patExpands{\upctx}{\uPhi}{\upv}{p}{\tau}$. Because the metric is stable when passing from part 1 to part 2, we must have that it is strictly decreasing in the other direction:
% \[\sizeof{\patExpands{\upctx}{\uPhi}{\upv}{p}{\tau}} < \sizeof{{\cvalidP{\upctx}{\pscene{\uDelta}{\uPhi}{b}}{\acesplicedp{m}{n}{\ctau}}{p}{\tau}}}\]
% i.e. by the definitions above, 
% \[\sizeof{\upv} < \sizeof{b}\]

% This is established by appeal to Condition \ref{condition:body-subsequences}, which states that subsequences of $b$ are no longer than $b$, and the Condition \ref{condition:pattern-parsing}, which states that an unexpanded pattern constructed by parsing a textual sequence $b$ is strictly smaller, as measured by the metric defined above, than the length of $b$, because some characters must necessarily be used to apply the pattern TLM and delimit each literal body. Combining Conditions \ref{condition:body-subsequences} and \ref{condition:pattern-parsing}, we have that $\sizeof{\upv} < \sizeof{b}$ as needed.
\end{proof}
\subsection{Typed Expression and Rule Expansion}\label{appendix:P-typed-expression-expansion}
\begin{theorem}[Typed Expression and Rule Expansion]\label{thm:typed-expansion-full-B} ~
\begin{enumerate}
  \item \begin{enumerate}
    \item If $\esyn{\uDD{\uD}{\Delta}}{\uGG{\uG}{\Gamma}}{\uPsi}{\uPhi}{\ue}{e}{\tau}$ then $\hastypeU{\Delta}{\Gamma}{e}{\tau}$.
    \item If $\eana{\uDD{\uD}{\Delta}}{\uGG{\uG}{\Gamma}}{\uPsi}{\uPhi}{\ue}{e}{\tau}$ and $\istypeU{\Delta}{\tau}$ then $\hastypeU{\Delta}{\Gamma}{e}{\tau}$.
    \item If $\rana{\uDD{\uD}{\Delta}}{\uGG{\uG}{\Gamma}}{\uPsi}{\uPhi}{\urv}{r}{\tau}{\tau'}$ and $\istypeU{\Delta}{\tau'}$ then $\ruleType{\Delta}{\Gamma}{r}{\tau}{\tau'}$.
  \end{enumerate}
  \item \begin{enumerate}
    \item If $\csyn{\Delta}{\Gamma}{\esceneUP{\uDD{\uD}{\Delta_\text{app}}}{\uGG{\uG}{\Gamma_\text{app}}}{\uPsi}{\uPhi}{b}}{\ce}{e}{\tau}$ and $\Delta \cap \Delta_\text{app}=\emptyset$ and $\domof{\Gamma} \cap \domof{\Gamma_\text{app}}=\emptyset$ then $\hastypeU{\Dcons{\Delta}{\Delta_\text{app}}}{\Gcons{\Gamma}{\Gamma_\text{app}}}{e}{\tau}$. 
    \item If $\cana{\Delta}{\Gamma}{\esceneUP{\uDD{\uD}{\Delta_\text{app}}}{\uGG{\uG}{\Gamma_\text{app}}}{\uPsi}{\uPhi}{b}}{\ce}{e}{\tau}$ and $\istypeU{\Delta}{\tau}$ and $\Delta \cap \Delta_\text{app}=\emptyset$ and $\domof{\Gamma} \cap \domof{\Gamma_\text{app}}=\emptyset$ then $\hastypeU{\Dcons{\Delta}{\Delta_\text{app}}}{\Gcons{\Gamma}{\Gamma_\text{app}}}{e}{\tau}$. 
    \item If $\crana{\Delta}{\Gamma}{\esceneUP{\uDD{\uD}{\Delta_\text{app}}}{\uGG{\uG}{\Gamma_\text{app}}}{\uPsi}{\uPhi}{b}}{\crv}{r}{\tau}{\tau'}$ and $\istypeU{\Delta}{\tau'}$ and $\Delta \cap \Delta_\text{app}=\emptyset$ and $\domof{\Gamma} \cap \domof{\Gamma_\text{app}}=\emptyset$ then $\ruleType{\Dcons{\Delta}{\Delta_\text{app}}}{\Gcons{\Gamma}{\Gamma_\text{app}}}{r}{\tau}{\tau'}$.
  \end{enumerate}
\end{enumerate}
\end{theorem}
\begin{proof}
By mutual rule induction over Rules (\ref{rules:esyn-S}), Rules (\ref{rules:eana-S}), Rule (\ref{rule:rana-S}), Rules (\ref{rules:csyn}), Rules (\ref{rules:cana}) and Rule (\ref{rule:crana}). The proof follows the proof of Theorem \ref{thm:typed-expansion-full-U}.

% \begin{enumerate}
% \item In the following, let $\uDelta=\uDD{\uD}{\Delta}$ and $\uGamma=\uGG{\uG}{\Gamma}$.
%   \begin{enumerate}
%     \item \todo{1a}
%     \item \todo{1b}
%     \item \todo{1c}
%   \end{enumerate}
% \item \todo{2}
%   \begin{enumerate}
%     \item \todo{2a}
%     \item \todo{2b}
%     \item \todo{2c}
%   \end{enumerate}
% \end{enumerate}
\end{proof}

% % \subsection{Expressibility}
% The following lemma establishes that each type can be expressed as a well-formed proto-type, under the same type formation context and any type splicing scene.
% \begin{lemma}[Proto-Expansion Type Expressibility]\label{lemma:proto-type-expressibility-U} If $\istypeU{\Delta}{\tau}$ then $\cvalidT{\Delta}{\tscenev}{\Cof{\tau}}{\tau}$. \end{lemma}
% \begin{proof}
% By rule induction over Rules (\ref{rules:istypeU}). In each case, we apply the IH on or over each premise, then apply the corresponding proto-type validation rule in Rules (\ref{rules:cvalidT-U}).
% \end{proof}

% The Type Expressibility Lemma establishes that every well-formed type, $\tau$, can be expressed as a well-formed unexpanded type, $\Uof{\tau}$. This requires defining the metafunction $\Uof{\Delta}$ which maps $\Delta$ onto an unexpanded type formation context as follows:
% \begin{align*}
% \Uof{\emptyset} &= \uDD{\emptyset}{\emptyset}\\
% \Uof{\Delta, \Dhyp{t}} &= \Uof{\Delta}, \uDhyp{\sigilof{t}}{t}
% \end{align*}
% \begin{lemma}[Type Expressibility]\label{lemma:type-expressibility} If $\istypeU{\Delta}{\tau}$ then $\expandsTU{\Uof{\Delta}}{\Uof{\tau}}{\tau}$.\end{lemma}
% \begin{proof} By rule induction over Rules (\ref{rules:istypeU}) using the definitions of $\Uof{\tau}$ and $\Uof{\Delta}$ above. In each case, we apply the IH to or over each premise, then apply the corresponding type expansion rule in Rules (\ref{rules:expandsTU}).\end{proof}


% The following lemma establishes that each well-typed expanded expression, $e$, can be expressed as a valid proto-expression, $\Cof{e}$, that is assigned the same type under any expression splicing scene.
% \begin{theorem}[Proto-Expansion Expression Expressibility]\label{theorem:proto-expressions-expressibility-U} If $\hastypeU{\Delta}{\Gamma}{e}{\tau}$ then $\cvalidE{\Delta}{\Gamma}{\escenev}{\Cof{e}}{e}{\tau}$.\end{theorem}
% \begin{proof} By rule induction over Rules (\ref{rules:hastypeU}). The rule transformation above guarantees that this lemma holds by construction. In particular, in each case, we apply Lemma \ref{lemma:proto-type-expressibility-U} to or over each type formation premise, the IH to or over each typing premise, then apply the corresponding proto-expression validation rule in Rules (\ref{rule:cvalidE-U-var}) through (\ref{rule:cvalidE-U-case}).
% \end{proof}

% The following lemma establishes that each well-typed expanded expression, $e$, can be expressed as a valid ce-expression, $\Cof{e}$, that is assigned the same type under any expression splicing scene.
% \begin{theorem}[Candidate Expansion Expression Expressibility]\label{lemma:ce-expressions-expressibility-UP} Both of the following hold:
% \begin{enumerate}
% \item If $\hastypeU{\Delta}{\Gamma}{e}{\tau}$ then $\cvalidE{\Delta}{\Gamma}{\escenev}{\Cof{e}}{e}{\tau}$.
% \item If $\ruleType{\Delta}{\Gamma}{r}{\tau}{\tau'}$ then $\cvalidR{\Delta}{\Gamma}{\escenev}{\Cof{r}}{r}{\tau}{\tau'}$.
% \end{enumerate}
% \end{theorem}
% \begin{proof} By mutual rule induction over Rules (\ref{rules:hastypeUP}) and Rule (\ref{rule:ruleType}). 

% For part 1, we induct on the assumption. 
% \begin{byCases}
% \item[\text{(\ref{rule:hastypeUP-var}) through (\ref{rule:hastypeUP-in})}] In each of these cases, we apply Lemma \ref{lemma:ce-type-expressibility-U} to or over each type formation premise, the IH (part 1) to or over each typing premise, then apply the corresponding ce-expression validation rule in Rules (\ref{rule:cvalidE-UP-var}) through (\ref{rule:cvalidE-UP-in}).
% \item[\text{(\ref{rule:hastypeUP-match})}] ~
%   \begin{pfsteps}
%   \item e = \aematchwith{n}{e'}{\seqschemaX{r}} \BY{assumption}
%   \item \Cof{e} = \acematchwith{n}{\Cof{\tau}}{\Cof{e'}}{\seqschemaXx{\Cofv}{r}} \BY{definition of $\Cof{e}$}
%   \item \hastypeU{\Delta}{\Gamma}{e'}{\tau'} \BY{assumption} \pflabel{hasType}
%   \item \istypeU{\Delta}{\tau} \BY{assumption} \pflabel{isType}
%   \item \{\ruleType{\Delta}{\Gamma}{r_i}{\tau'}{\tau}\}_{1 \leq i \leq n} \BY{assumption} \pflabel{ruleType}
%   \item \cvalidE{\Delta}{\Gamma}{\escenev}{\Cof{e'}}{e'}{\tau'} \BY{IH, part 1 on \pfref{hasType}} \pflabel{cvalidE}
%   \item \cvalidT{\Delta}{\tsfrom{\escenev}}{\Cof{\tau}}{\tau} \BY{Lemma \ref{lemma:candidate-expansion-type-validation} on \pfref{isType}} \pflabel{cvalidT}
%   \item \{\cvalidR{\Delta}{\Gamma}{\escenev}{\Cof{r_i}}{r_i}{\tau'}{\tau}\}_{1 \leq i \leq n} \BY{IH, part 2 over \pfref{ruleType}} \pflabel{cvalidR}
%   \item \cvalidE{\Delta}{\Gamma}{\escenev}{\acematchwith{n}{\Cof{\tau}}{\Cof{e'}}{\seqschemaXx{\Cofv}{r}}}{\aematchwith{n}{e'}{\seqschemaX{r}}}{\tau} \BY{Rule (\ref{rule:cvalidE-UP-match}) on \pfref{cvalidE}, \pfref{cvalidT} and \pfref{cvalidR}}
%   \end{pfsteps}
% \end{byCases}
% \resetpfcounter

% For part 2, we induct on the assumption. There is only one case.
% \begin{byCases}
% \item[\text{(\ref{rule:ruleType})}] ~
%   \begin{pfsteps}
%     \item r = \aematchrule{p}{e} \BY{assumption}
%     \item \Cof{r} = \acematchrule{p}{\Cof{e}} \BY{definition of $\Cof{r}$}
%     \item \patType{\pctx}{p}{\tau} \BY{assumption} \pflabel{patType}
%     \item \hastypeU{\Delta}{\Gcons{\Gamma}{\pctx}}{e}{\tau'} \BY{assumption} \pflabel{hasType}
%     \item \cvalidE{\Delta}{\Gcons{\Gamma}{\pctx}}{\escenev}{\Cof{e}}{e}{\tau'} \BY{IH, part 1 on \pfref{hasType}} \pflabel{cvalidE}
%     \item \cvalidR{\Delta}{\Gamma}{\escenev}{\acematchrule{p}{\Cof{e}}}{\aematchrule{p}{e}}{\tau}{\tau'} \BY{Rule (\ref{rule:cvalidR-UP}) on \pfref{patType} and \pfref{cvalidE}}
%   \end{pfsteps}
%   \resetpfcounter
% \end{byCases}
% \end{proof}

% The following lemma establishes that every well-typed expanded pattern that generates no hypotheses can be expressed as a ce-pattern.
% \begin{lemma}[Candidate Expansion Pattern Expressibility]\label{lemma:ce-pattern-expressibility-U} If $\patType{\emptyset}{p}{\tau}$ then $\cvalidP{\uGG{\emptyset}{\emptyset}}{\pscene{\uDelta}{\uPhi}{b}}{\Cof{p}}{p}{\tau}$.\end{lemma}
% \begin{proof} By rule induction over Rules (\ref{rules:patType}).
% \begin{byCases}
% \item[\text{(\ref{rule:patType-var})}] This case does not apply.
% \item[\text{(\ref{rule:patType-wild})}] ~
%   \begin{pfsteps*}
%     \item $p=\aewildp$ \BY{assumption}
%     \item $\Cof{p}=\acewildp$ \BY{definition of $\Cof{p}$}
%     \item $\cvalidP{\uGG{\emptyset}{\emptyset}}{\pscene{\uDelta}{\uPhi}{b}}{\acewildp}{\aewildp}{\tau}$ \BY{Rule (\ref{rule:cvalidP-UP-wild})}
%   \end{pfsteps*}
%   \resetpfcounter
% \item[\text{(\ref{rule:patType-fold})}] ~
%   \begin{pfsteps*}
%     \item $p=\aefoldp{p'}$ \BY{assumption}
%     \item $\Cof{p}=\acefoldp{\Cof{p'}}$ \BY{definition of $\Cof{p}$}
%     \item $\tau=\arec{t}{\tau'}$ \BY{assumption}
%     \item $\patType{\emptyset}{p'}{[\arec{t}{\tau'}/t]\tau'}$ \BY{assumption} \pflabel{patType}
%     \item $\cvalidP{\uGG{\emptyset}{\emptyset}}{\pscene{\uDelta}{\uPhi}{b}}{\Cof{p'}}{p}{[\arec{t}{\tau'}/t]\tau'}$ \BY{IH on \pfref{patType}} \pflabel{cvalidP}
%     \item $\cvalidP{\uGG{\emptyset}{\emptyset}}{\pscene{\uDelta}{\uPhi}{b}}{\acefoldp{\Cof{p'}}}{\aefoldp{p'}}{\arec{t}{\tau'}}$ \BY{Rule (\ref{rule:cvalidP-UP-fold}) on \pfref{cvalidP}}
%   \end{pfsteps*}
%   \resetpfcounter
% \item[\text{(\ref{rule:patType-tpl})}] ~
%   \begin{pfsteps*}
%     \item $p=\aetplp{\labelset}{\mapschema{p}{i}{\labelset}}$ \BY{assumption}
%     \item $\Cof{p}=\acetpl{\labelset}{\mapschemax{\Cofv}{p}{i}{\labelset}}$ \BY{definition of $\Cof{p}$}
%     \item $\tau=\aprod{\labelset}{\mapschema{\tau}{i}{\labelset}}$ \BY{assumption}
%     \item $\{\patType{\emptyset}{p_i}{\tau_i}\}_{i \in \labelset}$ \BY{assumption} \pflabel{patType}
%     \item $\{\cvalidP{\uGG{\emptyset}{\emptyset}}{\pscene{\uDelta}{\uPhi}{b}}{\Cof{p_i}}{p_i}{\tau_i}\}_{i \in \labelset}$ \BY{IH over \pfref{patType}} \pflabel{cvalidP}
%     \item $\cvalidP{\uGG{\emptyset}{\emptyset}}{\pscene{\uDelta}{\uPhi}{b}}{\acetpl{\labelset}{\mapschemax{\Cofv}{p}{i}{\labelset}}}{\aetplp{\labelset}{\mapschema{p}{i}{\labelset}}}{\aprod{\labelset}{\mapschema{\tau}{i}{\labelset}}}$ \BY{Rule (\ref{rule:cvalidP-UP-tpl}) on \pfref{cvalidP}}
%   \end{pfsteps*}
%   \resetpfcounter
% \item[\text{(\ref{rule:patType-inj})}] ~
%   \begin{pfsteps*}
%     \item $p=\aeinjp{\ell}{p'}$ \BY{assumption}
%     \item $\Cof{p}=\aceinjp{\ell}{\Cof{p'}}$ \BY{definition of $\Cof{p}$}
%     \item $\tau=\asum{\labelset, \ell}{\mapschema{\tau}{i}{\labelset}; \mapitem{\ell}{\tau'}}$ \BY{assumption}
%     \item $\patType{\emptyset}{p'}{\tau'}$ \BY{assumption}\pflabel{patType}
%     \item $\cvalidP{\uGG{\emptyset}{\emptyset}}{\pscene{\uDelta}{\uPhi}{b}}{\Cof{p'}}{p'}{\tau'}$ \BY{IH on \pfref{patType}}\pflabel{cvalidP}
%     \item $\cvalidP{\uGG{\emptyset}{\emptyset}}{\pscene{\uDelta}{\uPhi}{b}}{\aceinjp{\ell}{\Cof{p'}}}{\aeinjp{\ell}{p'}}{\asum{\labelset, \ell}{\mapschema{\tau}{i}{\labelset}; \mapitem{\ell}{\tau'}}}$ \BY{Rule (\ref{rule:cvalidP-UP-in}) on \pfref{cvalidP}}
%   \end{pfsteps*}
%   \resetpfcounter
% \end{byCases}
% \end{proof}

% \subsubsection{Expressibility}
% The following lemma establishes that each well-typed expanded pattern can be expressed as an unexpanded pattern matching values of the same type and generating the same hypotheses and corresponding identifier updates. The metafunction $\Uof{\pctx}$ maps $\pctx$ to an unexpanded typing context as follows:
% \begin{align*}
% \Uof{\emptyset} & = \uGG{\emptyset}{\emptyset}\\
% \Uof{\pctx, x : \tau} & = \Uof{\pctx}, \uGhyp{\sigilof{x}}{x}{\tau}\\
% \Uof{\Gconsi{i \in \labelset}{\pctx_i}} & = \Gconsi{i \in \labelset}{\Uof{\pctx_i}}
% \end{align*}
% \begin{lemma}[Pattern Expressibility]\label{lemma:pattern-expressibility} If $\patType{\pctx}{p}{\tau}$ then $\patExpands{\Uof{\pctx}}{\uPhi}{\Uof{p}}{p}{\tau}$.\end{lemma}
% \begin{proof} By rule induction over Rules (\ref{rules:patType}), using the definitions of $\Uof{\pctx}$ and $\Uof{p}$ given above. In each case, we can apply the IH to or over each premise, then apply the corresponding rule in Rules (\ref{rules:patExpands}).\end{proof}

% We can now establish the Expressibility Theorem -- that each well-typed expanded expression, $e$, can be expressed as an unexpanded expression, $\ue$, and assigned the same type under the corresponding contexts.

% \begin{theorem}[Expressibility] Both of the following hold:
% \begin{enumerate}
% \item If $\hastypeU{\Delta}{\Gamma}{e}{\tau}$ then $\expandsUP{\Uof{\Delta}}{\Uof{\Gamma}}{\uPsi}{\uPhi}{\Uof{e}}{e}{\tau}$.
% \item If $\ruleType{\Delta}{\Gamma}{r}{\tau}{\tau'}$ then $\ruleExpands{\Uof{\Delta}}{\Uof{\Gamma}}{\uPsi}{\uPhi}{\Uof{r}}{r}{\tau}{\tau'}$.
% \end{enumerate}
% \end{theorem}
% \begin{proof} By mutual rule induction over Rules (\ref{rules:hastypeUP}) and Rule (\ref{rule:ruleType}). 

% For part 1, we induct on the assumption. The rule transformation defined above guarantees that this part holds by its construction. In particular, in each case, we can apply Lemma \ref{lemma:type-expressibility} to or over each type formation premise, the IH (part 1) to or over each typing premise, the IH (part 2) over each rule typing premise, then apply the corresponding rule in Rules (\ref{rules:expandsUP}).

% For part 2, we induct on the assumption. There is only one case:
% \begin{byCases}
% \item[(\ref{rule:ruleType})] ~
% \begin{pfsteps*}
% \item $r = \aematchrule{p}{e}$ \BY{assumption}
% \item $\patType{\pctx}{p}{\tau}$ \BY{assumption} \pflabel{patType}
% \item $\hastypeU{\Delta}{\Gamma \cup \pctx}{e}{\tau'}$ \BY{assumption} \pflabel{hasType}
% \item $\Uof{\Gamma}=\uGG{\uG}{\Gamma}$, for some $\uG$ \BY{definition of $\Uof{\Gamma}$}
% \item $\Uof{\pctx} =\uGG{\uG'}{\pctx}$, for some $\uG'$ \BY{definition of $\Uof{\pctx}$}
% \item $\Uof{\Gamma \cup \pctx} = \uGG{\uG \uplus \uG'}{\Gamma \cup \pctx}$ \BY{definition of $\Uof{\pctx}$}
% \item $\Uof{r} = \aumatchrule{\Uof{p}}{\Uof{e}}$ \BY{definition of $\Uof{r}$}
% \item $\patExpands{\uGG{\uG'}{\pctx}}{\uPhi}{\Uof{p}}{p}{\tau}$ \BY{Lemma \ref{lemma:pattern-expressibility} on \pfref{patType}} \pflabel{patExpands}
% \item $\expandsUP{\uDelta}{\uGG{\uGcons{\uG}{\uG'}}{\Gcons{\Gamma}{\pctx}}}{\uPsi}{\uPhi}{\Uof{e}}{e}{\tau'}$ \BY{IH, part 1 on \pfref{hasType}} \pflabel{expandsUP}
% \item $\ruleExpands{\Uof{\Delta}}{\uGG{\uG}{\Gamma}}{\uPsi}{\uPhi}{\aumatchrule{\Uof{p}}{\Uof{e}}}{\aematchrule{p}{e}}{\tau}{\tau'}$ \BY{Rule (\ref{rule:ruleExpands}) on \pfref{patExpands} and \pfref{expandsUP}}
% \end{pfsteps*}
% \resetpfcounter
% \end{byCases}
% \end{proof}

\subsection{Abstract Reasoning Principles}
\begin{lemma}[Proto-Expression and Proto-Rule Expansion Decomposition] 
\label{thm:proto-expression-expansion-decomposition-B} ~
\begin{enumerate}
\item If ($\cana
  {\Delta}{\Gamma}
  {\esceneUP
    {\uDD{\uD}{\Delta_\text{app}}}
    {\uGG{\uG}{\Gamma_\text{app}}}
    {\uPsi}{\uPhi}{b}
  }{\ce}{e}{\tau}$ or $\csyn
  {\Delta}{\Gamma}
  {\esceneUP
    {\uDD{\uD}{\Delta_\text{app}}}
    {\uGG{\uG}{\Gamma_\text{app}}}
    {\uPsi}{\uPhi}{b}
  }{\ce}{e}{\tau}$) where $\segof{\ce} = \sseq{\acesplicedt{m'_i}{n'_i}}{\nty} \cup \sseq{\acesplicede{m_i}{n_i}{\ctau_i}}{\nexp}$ then all of the following hold:
  \begin{enumerate}
    \item $\sseq{
          \expandsTU{\uDD{\uD}{\Delta_\text{app}}}
          {
            \parseUTypF{\bsubseq{b}{m'_i}{n'_i}}
          }{\tau'_i}
        }{\nty}$
    \item $\sseq{
      \cvalidT{\emptyset}{
        \tsceneUP
          {\uDD
            {\uD}{\Delta_\text{app}}
          }{b}
      }{
        \ctau_i
      }{\tau_i}
    }{\nexp}$
    \item $\sseq{
      \eana
        {\uDD{\uD}{\Delta_\text{app}}}
        {\uGG{\uG}{\Gamma_\text{app}}}
        {\uPsi}
        {\uPhi}
        {\parseUExpF{\bsubseq{b}{m_i}{n_i}}}
        {e_i}
        {\tau_i}
    }{\nexp}$
    \item $e = [\sseq{\tau'_i/t_i}{\nty}, \sseq{e_i/x_i}{\nexp}]e'$ for some $e'$ and $\sseq{t_i}{\nty}$ and $\sseq{x_i}{\nexp}$ such that $\sseq{t_i}{\nty}$ fresh (i.e.  $\sseq{t_i \notin \domof{\Delta}}{\nty}$ and $\sseq{t_i \notin \domof{\Delta_\text{app}}}{\nty}$) and $\sseq{x_i}{\nexp}$ fresh (i.e.  $\sseq{x_i \notin \domof{\Gamma}}{\nexp}$ and $\sseq{x_i \notin \domof{\Gamma_\text{app}}}{\nty}$)
    \item $\mathsf{fv}(e') \subset \domof{\Delta} \cup \domof{\Gamma} \cup \sseq{t_i}{\nty} \cup \sseq{x_i}{\nexp}$
    % \item $\hastypeU
    %   {\Delta \cup \sseq{\Dhyp{t_i}}{\nty}}
    %   {\Gamma \cup \sseq{x_i : \tau_i}{\nexp}}
    %   {e'}{\tau}$
  \end{enumerate}
\item If $\cvalidR{\Delta}{\Gamma}{\esceneUP{\uDD{\uD}{\Delta_\text{app}}}{\uGG{\uG}{\Gamma_\text{app}}}{\uPsi}{\uPhi}{b}}{\crv}{r}{\tau}{\tau'}$ where \[\segof{\crv} = \sseq{\acesplicedt{m'_i}{n'_i}}{\nty} \cup \sseq{\acesplicede{m_i}{n_i}{\ctau_i}}{\nexp}\] then all of the following hold:
  \begin{enumerate}
    \item $\sseq{
          \expandsTU{\uDD{\uD}{\Delta_\text{app}}}
          {
            \parseUTypF{\bsubseq{b}{m'_i}{n'_i}}
          }{\tau'_i}
        }{\nty}$
    \item $\sseq{
      \cvalidT{\emptyset}{
        \tsceneUP
          {\uDD
            {\uD}{\Delta_\text{app}}
          }{b}
      }{
        \ctau_i
      }{\tau_i}
    }{\nexp}$
    \item $\sseq{
      \eana
        {\uDD{\uD}{\Delta_\text{app}}}
        {\uGG{\uG}{\Gamma_\text{app}}}
        {\uPsi}
        {\uPhi}
        {\parseUExpF{\bsubseq{b}{m_i}{n_i}}}
        {e_i}
        {\tau_i}
    }{\nexp}$
    \item $r = [\sseq{\tau'_i/t_i}{\nty}, \sseq{e_i/x_i}{\nexp}]r'$ for some $e'$ and fresh $\sseq{t_i}{\nty}$ and fresh $\sseq{x_i}{\nexp}$ 
    \item $\mathsf{fv}(r') \subset \domof{\Delta} \cup \domof{\Gamma} \cup \sseq{t_i}{\nty} \cup \sseq{x_i}{\nexp}$
    % \item $\ruleType
    %   {\Delta \cup \sseq{\Dhyp{t_i}}{\nty}}
    %   {\Gamma \cup \sseq{x_i : \tau_i}{\nexp}}
    %   {r'}{\tau}{\tau'}$
  \end{enumerate}
\end{enumerate}
\end{lemma}
\begin{proof} By rule induction over Rules (\ref{rules:cana}) and Rule (\ref{rule:crana}). The proof follows the proof of Theorem \ref{thm:proto-expression-expansion-decomposition}. 
\end{proof}
% \begin{equation}\label{rule:cvalidR-UP}
% \inferrule{
%   \patType{\pctx}{p}{\tau}\\
%   \cvalidE{\Delta}{\Gcons{\Gamma}{\pctx}}{\escenev}{\ce}{e}{\tau'}
% }{
%   \cvalidR{\Delta}{\Gamma}{\escenev}{\acematchrule{p}{\ce}}{\aematchrule{p}{e}}{\tau}{\tau'}
% }
% \end{equation}

\begin{theorem}[seTLM Abstract Reasoning Principles - Explicit Application]
\label{thm:tsc-implicit-B}
If \[\esyn{\uDD{\uD}{\Delta}}{\uGG{\uG}{\Gamma}}{\uPsi}{\uPhi}{\utsmap{\tsmv}{b}}{e}{\tau}\] then:
\begin{enumerate}
\item (\textbf{Typing 1}) $\uPsi = \uPsi', \uShyp{\tsmv}{x}{\tau}{\eparse}$ and $\hastypeU{\Delta}{\Gamma}{e}{\tau}$
\item $\encodeBody{b}{\ebody}$
\item $\evalU{\ap{\eparse}{\ebody}}{\aein{\mathtt{SuccessE}}{\ecand}}$
\item $\decodeCondE{\ecand}{\ce}$
\item (\textbf{Segmentation}) $\segOK{\segof{\ce}}{b}$
\item $\segof{\ce} = \sseq{\acesplicedt{m'_i}{n'_i}}{\nty} \cup \sseq{\acesplicede{m_i}{n_i}{\ctau_i}}{\nexp}$
\item \textbf{(Typing 2)} $\sseq{
      \expandsTU{\uDD{\uD}{\Delta}}
      {
        \parseUTypF{\bsubseq{b}{m'_i}{n'_i}}
      }{\tau'_i}
    }{\nty}$ and $\sseq{\istypeU{\Delta}{\tau'_i}}{\nty}$
\item \textbf{(Typing 3)} $\sseq{
  \cvalidT{\emptyset}{
    \tsceneUP
      {\uDD
        {\uD}{\Delta}
      }{b}
  }{
    \ctau_i
  }{\tau_i}
}{\nexp}$ and $\sseq{\istypeU{\Delta}{\tau_i}}{\nexp}$
\item \textbf{(Typing 4)} $\sseq{
  \eana
    {\uDD{\uD}{\Delta}}
    {\uGG{\uG}{\Gamma}}
    {\uPsi}
    {\uPhi}
    {\parseUExpF{\bsubseq{b}{m_i}{n_i}}}
    {e_i}
    {\tau_i}
}{\nexp}$ and $\sseq{\hastypeU{\Delta}{\Gamma}{e_i}{\tau_i}}{\nexp}$
\item (\textbf{Capture Avoidance}) $e = [\sseq{\tau'_i/t_i}{\nty}, \sseq{e_i/x_i}{\nexp}]e'$ for some $\sseq{t_i}{\nty}$ and $\sseq{x_i}{\nexp}$ and $e'$
\item (\textbf{Context Independence}) $\mathsf{fv}(e') \subset \sseq{t_i}{\nty} \cup \sseq{x_i}{\nexp}$
  % $\hastypeU
  % {\sseq{\Dhyp{t_i}}{\nty}}
  % {\sseq{x_i : \tau_i}{\nexp}}
  % {e'}{\tau}$
\end{enumerate}
\end{theorem}
\begin{proof} By rule induction over Rules (\ref{rules:esyn-S}). The proof follows the proof of Theorem \ref{thm:tsc-SES}.
\end{proof}

\begin{theorem}[seTLM Abstract Reasoning Principles - Implicit Application]
\label{thm:tsc-B}
If \[\eana{\uDD{\uD}{\Delta}}{\uGG{\uG}{\Gamma}}{\uPsi}{\uPhi}{\lit{b}}{e}{\tau}\] then:
\begin{enumerate}
\item (\textbf{Typing 1}) $\uPsi = \uASI{\uA}{\Psi, \xuetsmbnd{x}{\tau}{\eparse}}{\uI \uplus \designate{\tau}{a}}$ and $\hastypeU{\Delta}{\Gamma}{e}{\tau}$
\item $\encodeBody{b}{\ebody}$
\item $\evalU{\ap{\eparse}{\ebody}}{\aein{\mathtt{SuccessE}}{\ecand}}$
\item $\decodeCondE{\ecand}{\ce}$
\item (\textbf{Segmentation}) $\segOK{\segof{\ce}}{b}$
\item $\segof{\ce} = \sseq{\acesplicedt{m'_i}{n'_i}}{\nty} \cup \sseq{\acesplicede{m_i}{n_i}{\ctau_i}}{\nexp}$
\item \textbf{(Typing 2)} $\sseq{
      \expandsTU{\uDD{\uD}{\Delta}}
      {
        \parseUTypF{\bsubseq{b}{m'_i}{n'_i}}
      }{\tau'_i}
    }{\nty}$ and $\sseq{\istypeU{\Delta}{\tau'_i}}{\nty}$
\item \textbf{(Typing 3)} $\sseq{
  \cvalidT{\emptyset}{
    \tsceneUP
      {\uDD
        {\uD}{\Delta}
      }{b}
  }{
    \ctau_i
  }{\tau_i}
}{\nexp}$ and $\sseq{\istypeU{\Delta}{\tau_i}}{\nexp}$
\item \textbf{(Typing 4)} $\sseq{
  \eana
    {\uDD{\uD}{\Delta}}
    {\uGG{\uG}{\Gamma}}
    {\uPsi}
    {\uPhi}
    {\parseUExpF{\bsubseq{b}{m_i}{n_i}}}
    {e_i}
    {\tau_i}
}{\nexp}$ and $\sseq{\hastypeU{\Delta}{\Gamma}{e_i}{\tau_i}}{\nexp}$
\item (\textbf{Capture Avoidance}) $e = [\sseq{\tau'_i/t_i}{\nty}, \sseq{e_i/x_i}{\nexp}]e'$ for some $\sseq{t_i}{\nty}$ and $\sseq{x_i}{\nexp}$ and $e'$
\item (\textbf{Context Independence}) $\mathsf{fv}(e') \subset \sseq{t_i}{\nty} \cup \sseq{x_i}{\nexp}$
  % $\hastypeU
  % {\sseq{\Dhyp{t_i}}{\nty}}
  % {\sseq{x_i : \tau_i}{\nexp}}
  % {e'}{\tau}$
\end{enumerate}
\end{theorem}
\begin{proof} By rule induction over Rules (\ref{rules:eana-S}). The proof follows the proof of Theorem \ref{thm:tsc-SES}, differing only in how the TLM definition is looked up.
\end{proof}


% The following theorem establishes a prohibition on \textbf{Shadowing} as discussed in Sec. \ref{sec:uetsms-validation}.

% \begin{theorem}[Shadowing Prohibition]
% \label{thm:shadowing-prohibition-SES} ~
% \begin{enumerate}
% \item If $\cvalidT{\Delta}{\tsceneU{\uDD{\uD}{\Delta_\text{app}}}{b}}{\acesplicedt{m}{n}}{\tau}$ then:\begin{enumerate}
% \item $\parseUTyp{\bsubseq{b}{m}{n}}{\utau}$
% \item $\expandsTU{\uDD{\uD}{\Delta_\text{app}}}{\utau}{\tau}$
% \item $\Delta \cap \Delta_\text{app} = \emptyset$
% \end{enumerate}
% \item If $\cvalidE{\Delta}{\Gamma}{\escenev}{\acesplicede{m}{n}{\ctau}}{e}{\tau}$ then:
% \begin{enumerate}
% \item $\cvalidT{\emptyset}{\tsfrom{\escenev}}{\ctau}{\tau}$
% \item $  \escenev=\esceneU{\uDD{\uD}{\Delta_\text{app}}}{\uGG{\uG}{\Gamma_\text{app}}}{\uPsi}{b}$
% \item $\parseUExp{\bsubseq{b}{m}{n}}{\ue}$
% \item $\expandsU{\uDD{\uD}{\Delta_\text{app}}}{\uGG{\uG}{\Gamma_\text{app}}}{\uPsi}{\ue}{e}{\tau}$
% \item $\Delta \cap \Delta_\text{app} = \emptyset$
% \item $\domof{\Gamma} \cap \domof{\Gamma_\text{app}} = \emptyset$
% \end{enumerate}
% \end{enumerate}
% \end{theorem}
% \begin{proof} ~
% \begin{enumerate}
% \item By rule induction over Rules (\ref{rules:cvalidT-U}). The only rule that applies is Rule (\ref{rule:cvalidT-U-splicedt}). The conclusions are the premises of this rule.
% \item By rule induction over Rules (\ref{rules:cvalidE-U}). The only rule that applies is Rule (\ref{rule:cvalidE-U-splicede}). The conclusions are the premises of this rule.
% \end{enumerate}
% \end{proof}

\begin{lemma}[Proto-Pattern Expansion Decomposition]
\label{lemma:proto-pattern-expansion-decomposition-B}
If $\cvalidP{\upctx}{\pscene{\uDelta}{\uPhi}{b}}{\cpv}{p}{\tau}$ where  
\[ 
\segof{\cpv} = \sseq{\acesplicedt{m'_i}{n'_i}}{\nty} \cup \sseq{\acesplicedp{m_i}{n_i}{\ctau_i}}{\npat}
\]
then all of the following hold:
\begin{enumerate}
    \item $\sseq{
          \expandsTU{\uDelta}
          {
            \parseUTypF{\bsubseq{b}{m'_i}{n'_i}}
          }{\tau'_i}
        }{\nty}$
    \item $\sseq{
      \cvalidT{\emptyset}{
        \tsceneUP
          {\uDelta}{b}
      }{
        \ctau_i
      }{\tau_i}
    }{\npat}$
    \item $\sseq{
      \patExpands
        {\upctx_i}
        {\uPhi}
        {\parseUPatF{\bsubseq{b}{m_i}{n_i}}}
        {p_i}
        {\tau_i}
    }{\npat}$
  \item $\upctx = \biguplus_{0 \leq i < \npat} \upctx_i$
\end{enumerate}
\end{lemma}
\begin{proof} By rule induction over Rules (\ref{rules:cvalidP-B}). The proof follows the proof of Theorem \ref{lemma:proto-pattern-expansion-decomposition-S}.
\end{proof}

\begin{theorem}[spTLM Abstract Reasoning Principles - Explicit Application]
\label{thm:spTLM-Typing-Segmentation-B}
If \[\patExpands{\upctx}{\uPhi}{\utsmap{\tsmv}{b}}{p}{\tau}\] where $\uDelta=\uDD{\uD}{\Delta}$ and $\uGamma=\uGG{\uG}{\Gamma}$ then all of the following hold:
\begin{enumerate}
        \item (\textbf{Typing 1}) $\uPhi=\uPhi', \uPhyp{\tsmv}{x}{\tau}{\eparse}$ and $\patType{\pctx}{p}{\tau}$
        \item $\encodeBody{b}{\ebody}$
        \item $\evalU{\eparse(\ebody)}{\aein{\mathtt{SuccessP}}{\ecand}}$
        \item $\decodeCEPat{\ecand}{\cpv}$
        \item (\textbf{Segmentation}) $\segOK{\segof{\cpv}}{b}$
        \item $\segof{\cpv} = \sseq{\acesplicedt{n'_i}{m'_i}}{\nty} \cup \sseq{\acesplicedp{m_i}{n_i}{\ctau_i}}{\npat}$
        \item (\textbf{Typing 2}) $\sseq{
              \expandsTU{\uDelta}
              {
                \parseUTypF{\bsubseq{b}{m'_i}{n'_i}}
              }{\tau'_i}
            }{\nty}$ and $\sseq{\istypeU{\Delta}{\tau'_i}}{\nty}$
        \item (\textbf{Typing 3}) $\sseq{
          \cvalidT{\emptyset}{
            \tsceneUP
              {\uDelta}{b}
          }{
            \ctau_i
          }{\tau_i}
        }{\npat}$ and $\sseq{\istypeU{\Delta}{\tau_i}}{\npat}$
        \item (\textbf{Typing 4}) $\sseq{
          \patExpands
            {\upctx_i}
            {\uPhi}
            {\parseUPatF{\bsubseq{b}{m_i}{n_i}}}
            {p_i}
            {\tau_i}
        }{\npat}$ 
      \item (\textbf{Visibility}) $\upctx = \biguplus_{0 \leq i < \npat} \upctx_i$
\end{enumerate}
\end{theorem}
\begin{proof} By rule induction over Rules (\ref{rules:patExpands-B}). The proof follows the proof of Theorem \ref{thm:spTLM-Typing-Segmentation}.
\end{proof}

\begin{theorem}[spTLM Abstract Reasoning Principles - Implicit Application]
\label{thm:spTLM-Typing-Segmentation-Implicit-B}
If \[\patExpands{\upctx}{\uPhi}{\lit{b}}{p}{\tau}\] where $\uDelta=\uDD{\uD}{\Delta}$ and $\uGamma=\uGG{\uG}{\Gamma}$ then all of the following hold:
\begin{enumerate}
        \item (\textbf{Typing 1}) $\uPhi = \uASI{\uA}{\Phi, \xuptsmbnd{a}{\tau}{\eparse}}{\uI, \designate{\tau}{a}}$ and $\patType{\pctx}{p}{\tau}$
        \item $\encodeBody{b}{\ebody}$
        \item $\evalU{\eparse(\ebody)}{\aein{\mathtt{SuccessP}}{\ecand}}$
        \item $\decodeCEPat{\ecand}{\cpv}$
        \item (\textbf{Segmentation}) $\segOK{\segof{\cpv}}{b}$
        \item $\segof{\cpv} = \sseq{\acesplicedt{n'_i}{m'_i}}{\nty} \cup \sseq{\acesplicedp{m_i}{n_i}{\ctau_i}}{\npat}$
        \item (\textbf{Typing 2}) $\sseq{
              \expandsTU{\uDelta}
              {
                \parseUTypF{\bsubseq{b}{m'_i}{n'_i}}
              }{\tau'_i}
            }{\nty}$ and $\sseq{\istypeU{\Delta}{\tau'_i}}{\nty}$
        \item (\textbf{Typing 3}) $\sseq{
          \cvalidT{\emptyset}{
            \tsceneUP
              {\uDelta}{b}
          }{
            \ctau_i
          }{\tau_i}
        }{\npat}$ and $\sseq{\istypeU{\Delta}{\tau_i}}{\npat}$
        \item (\textbf{Typing 4}) $\sseq{
          \patExpands
            {\upctx_i}
            {\uPhi}
            {\parseUPatF{\bsubseq{b}{m_i}{n_i}}}
            {p_i}
            {\tau_i}
        }{\npat}$ 
      \item (\textbf{Visibility}) $\upctx = \biguplus_{0 \leq i < \npat} \upctx_i$
\end{enumerate}
\end{theorem}
\begin{proof} By rule induction over Rules (\ref{rules:patExpands-B}). The proof follows the proof of Theorem \ref{thm:spTLM-Typing-Segmentation}, differing only in how the parse function is looked up.
\end{proof}

\fi
% \subsubsection{Candidate Expansion Expressibility}
% The following lemma establishes that each well-typed expanded expression, $e$, can be expressed as a valid ce-expression, $\Cof{e}$, that synthesizes the same type under the same contexts and any expression splicing scene.
% \begin{theorem}[Candidate Expansion Expression Expressibility]\label{lemma:ce-expressions-expressibility-B} Both of the following hold:
% \begin{enumerate}
% \item If $\hastypeU{\Delta}{\Gamma}{e}{\tau}$ then $\csyn{\Delta}{\Gamma}{\escenev}{\Cof{e}}{e}{\tau}$.
% \item If $\ruleType{\Delta}{\Gamma}{r}{\tau}{\tau'}$ then $\crsyn{\Delta}{\Gamma}{\escenev}{\Cof{r}}{r}{\tau}{\tau'}$.
% \end{enumerate}
% \end{theorem}
% \begin{proof} By mutual rule induction over Rules (\ref{rules:hastypeUP}) and Rule (\ref{rule:ruleType}). In each case, we apply the IH, part 1 to or over each typing premise, the IH, part 2 over each rule typing premise, Lemma \ref{lemma:ce-type-expressibility-U} to or over each type formation premise and then derive the conclusion by applying Rules (\ref{rules:csyn}) and Rule (\ref{rule:crsyn}) as needed.
% \end{proof}

% The following lemma establishes that every well-typed expanded pattern that generates no hypotheses can be expressed as a ce-pattern.
% \begin{lemma}[Candidate Expansion Pattern Expressibility]\label{lemma:ce-pattern-expressibility-B} If $\patType{\emptyset}{p}{\tau}$ then $\cvalidP{\uGG{\emptyset}{\emptyset}}{\pscene{\Delta}{\uPhi}{b}}{\Cof{p}}{p}{\tau}$.\end{lemma}
% \begin{proof} The proof is nearly identical to the proof of Lemma \ref{lemma:ce-pattern-expressibility-U}, differing only in that each mention of a rule in Rules (\ref{rules:cvalidP-UP}) is replaced by a mention of the corresponding rule in Rules (\ref{rules:cvalidP-B}).
% \end{proof}

% \subsubsection{Outer Surface Expressibility}
% The following lemma establishes that each well-typed expanded pattern can be expressed as an unexpanded pattern matching values of the same type and generating the same hypotheses and corresponding sigil updates. The metafunction $\Uof{\pctx}$ was defined in \ref{sec:typed-expansion-UP}.
% \begin{lemma}[Pattern Expressibility]\label{lemma:pattern-expressibility-B} If $\patType{\pctx}{p}{\tau}$ then $\patExpands{\Uof{\pctx}}{\uPhi}{\Uof{p}}{p}{\tau}$.\end{lemma}
% \begin{proof} By rule induction over Rules (\ref{rules:patType}), using the definitions of $\Uof{\pctx}$ and $\Uof{p}$. In each case, we can apply the IH to or over each premise, then apply the corresponding rule in Rules (\ref{rules:patExpands-B}).\end{proof}

% We can now establish the Expressibility Theorem -- that each well-typed expanded expression, $e$, can be expressed as an unexpanded expression, $\ue$, which synthesizes the same type under the corresponding contexts.

% \begin{theorem}[Expressibility] Both of the following hold:
% \begin{enumerate}
% \item If $\hastypeU{\Delta}{\Gamma}{e}{\tau}$ then $\esyn{\Uof{\Delta}}{\Uof{\Gamma}}{\uPsi}{\uPhi}{\Uof{e}}{e}{\tau}$.
% \item If $\ruleType{\Delta}{\Gamma}{r}{\tau}{\tau'}$ then $\rsyn{\Uof{\Delta}}{\Uof{\Gamma}}{\uPsi}{\uPhi}{\Uof{r}}{r}{\tau}{\tau'}$.
% \end{enumerate}
% \end{theorem}
% \begin{proof} By mutual rule induction over Rules (\ref{rules:hastypeUP}) and Rule (\ref{rule:ruleType}) using the definitions of $\Uof{\Delta}$, $\Uof{\Gamma}$, $\Uof{e}$ and $\Uof{r}$. In each case, we apply the IH, part 1 to or over each typing premise, the IH, part 2 over each rule typing premise, Lemma \ref{lemma:type-expressibility} to or over each type formation premise, Lemma \ref{lemma:pattern-expressibility-B} to each pattern typing premise, then derive the conclusion by applying Rules (\ref{rules:esyn}) and Rule (\ref{rule:rsyn}).  
% \end{proof} 

% \chapter{Parametric Implicits}
% ...

% \subsubsection{Kinds and Constructors}
% Kind expansion

% \begin{subequations}\label{rules:kExpands}
% \begin{equation}\label{rule:kExpands-darr}
% \inferrule{
%   \kExpandsX{\ukappa_1}{\kappa_1}\\
%   \kExpands{\uOmega, \uKhyp{\uu}{u}{\kappa_1}}{\ukappa_2}{\kappa_2}
% }{
%   \kExpandsX{\kdarr{\uu}{\ukappa_1}{\ukappa_2}}{\akdarr{\kappa_1}{u}{\kappa_2}}
% }
% \end{equation}
% \begin{equation}\label{rule:kExpands-unit}
% \inferrule{ }{
%   \kExpandsX{\kunit}{\akunit}
% }
% \end{equation}
% \begin{equation}\label{rule:kExpands-dprod}
% \inferrule{
%   \kExpandsX{\ukappa_1}{\kappa_1}\\
%   \kExpands{\uOmega, \uKhyp{\uu}{u}{\kappa_1}}{\ukappa_2}{\kappa_2}
% }{
%   \kExpandsX{\kdbprod{\uu}{\ukappa_1}{\ukappa_2}}{\akdbprod{\kappa_1}{u}{\kappa_2}}
% }
% \end{equation}
% \begin{equation}\label{rule:kExpands-ty}
% \inferrule{ }{
%   \kExpandsX{\kty}{\akty}
% }
% \end{equation}
% \begin{equation}\label{rule:kExpands-sing}
% \inferrule{
%   \kanaX{\utau}{\tau}{\akty}
% }{
%   \kExpandsX{\ksing{\utau}}{\aksing{\tau}}
% }
% \end{equation}
% \end{subequations}

% Synthetic constructor expansion
% \begin{subequations}\label{rules:ksyn}
% \begin{equation}\label{rule:ksyn-var}
% \inferrule{ }{\ksyn{\uOmega, \uKhyp{\uu}{u}{\kappa}}{\uu}{u}{\kappa}}
% \end{equation}
% \begin{equation}\label{rule:ksyn-asc}
% \inferrule{
%   \kExpandsX{\ukappa}{\kappa}\\
%   \kanaX{\uc}{c}{\kappa}
% }{
%   \ksynX{\casc{\uc}{\ukappa}}{c}{\kappa}
% }
% \end{equation}
% \begin{equation}\label{rule:ksyn-app}
% \inferrule{
%   \ksynX{\uc_1}{c_1}{\akdarr{\kappa_2}{u}{\kappa}}\\
%   \kanaX{\uc_2}{c_2}{\kappa_2}
% }{
%   \ksynX{\capp{\uc_1}{\uc_2}}{\acapp{c_1}{c_2}}{[c_1/u]\kappa}
% }
% \end{equation}
% \begin{equation}\label{rule:ksyn-unit}
% \inferrule{ }{
%   \ksynX{\ctriv}{\actriv}{\akunit}
% }
% \end{equation}
% \begin{equation}\label{rule:ksyn-prl}
% \inferrule{
%   \ksynX{\uc}{c}{\akdbprod{\kappa_1}{u}{\kappa_2}}
% }{
%   \ksynX{\cprl{\uc}}{\acprl{c}}{\kappa_1}
% }
% \end{equation}
% \begin{equation}\label{rule:ksyn-prr}
% \inferrule{
%   \ksynX{\uc}{c}{\akdbprod{\kappa_1}{u}{\kappa_2}}
% }{
%   \ksynX{\cprr{\uc}}{\acprr{c}}{[\acprl{c}/u]\kappa_2}
% }
% \end{equation}
% \begin{equation}\label{rule:ksyn-parr}
% \inferrule{
%   \kanaX{\utau_1}{\tau_1}{\akty}\\
%   \kanaX{\utau_2}{\tau_2}{\akty}
% }{
%   \ksynX{\parr{\utau_1}{\utau_2}}{\aparr{\tau_1}{\tau_2}}{\akty}
% }
% \end{equation}
% \begin{equation}\label{rule:ksyn-all}
% \inferrule{
%   \kExpandsX{\ukappa}{\kappa}\\
%   \kana{\uOmega, \uKhyp{\uu}{u}{\kappa}}{\utau}{\tau}{\akty}
% }{
%   \ksynX{\forallu{\uu}{\ukappa}{\utau}}{\aallu{\kappa}{u}{\tau}}{\akty}
% }
% \end{equation}
% \begin{equation}\label{rule:ksyn-rec}
% \inferrule{
%   \kana{\uOmega, \uKhyp{\ut}{t}{\akty}}{\utau}{\tau}{\akty}
% }{
%   \ksynX{\rect{\ut}{\utau}}{\arec{t}{\tau}}{\akty}
% }
% \end{equation}
% \begin{equation}\label{rule:ksyn-prod}
% \inferrule{
%   \{\kanaX{\utau_i}{\tau_i}{\akty}\}_{1 \leq i \leq n}
% }{
%   \ksynX{\prodt{\mapschema{\utau}{i}{\labelset}}}{\aprod{\labelset}{\mapschema{\tau}{i}{\labelset}}}{\akty}
% }
% \end{equation}
% \begin{equation}\label{rule:ksyn-sum}
% \inferrule{
%   \{\kanaX{\utau_i}{\tau_i}{\akty}\}_{1 \leq i \leq n}
% }{
%   \ksynX{\sumt{\labelset}{\mapschema{\utau}{i}{\labelset}}}{\asum{\labelset}{\mapschema{\tau}{i}{\labelset}}}{\akty}
% }
% \end{equation}
% \begin{equation}\label{rule:ksyn-stat}
% \inferrule{ }{
%   \ksyn{\uOmega, \uMhyp{\uX}{X}{\asignature{\kappa}{u}{\tau}}}{\mcon{\uX}}{\amcon{X}}{\kappa}
% }
% \end{equation}
% \end{subequations}

% Analytic constructor expansion
% \begin{subequations}\label{rules:kana}
% \begin{equation}\label{rule:kana-subsume}
% \inferrule{
%   \ksynX{\uc}{c}{\kappa_1}\\
%   \ksubX{\kappa_1}{\kappa_2}
% }{
%   \kanaX{\uc}{c}{\kappa_2}
% }
% \end{equation}
% \begin{equation}\label{rule:kana-sing}
% \inferrule{
%   \kanaX{\uc}{c}{\akty}
% }{
%   \kanaX{\uc}{c}{\aksing{c}}
% }
% \end{equation}
% \begin{equation}\label{rule:kana-abs}
% \inferrule{
%   \kana{\uOmega, \uKhyp{\uu}{u}{\kappa_1}}{\uc_2}{c_2}{\kappa_2}
% }{
%   \kanaX{\cabs{\uu}{\uc_2}}{\acabs{u}{c_2}}{\akdarr{\kappa_1}{u}{\kappa_2}}
% }
% \end{equation}
% \begin{equation}\label{rule:kana-pair}
% \inferrule{
%   \kanaX{\uc_1}{c_1}{\kappa_1}\\
%   \kanaX{\uc_2}{c_2}{[c_1/u]\kappa_2}
% }{
%   \kanaX{\cpair{\uc_1}{\uc_2}}{\acpair{c_1}{c_2}}{\akdbprod{\kappa_1}{u}{\kappa_2}}
% }
% \end{equation}
% \end{subequations}


% \subsubsection{Types, Expressions, Rules and Patterns}
% Type expansion
% \begin{equation}\label{rule:tExpandsP-B}
% \inferrule{
%   \kanaX{\utau}{\tau}{\akty}
% }{
%   \cExpandsX{\utau}{\tau}{\akty}
% }
% \end{equation}

% Synthetic typed expression expansion
% \begin{subequations}\label{rules:esynP}
% \begin{equation}\label{rule:esynP-var}
%   \inferrule{ }{ 
%     \esynP{\uOmega, \uGhyp{\ux}{x}{\tau}}{\uPsi}{\uPhi}{\ux}{x}{\tau}
%   }
% \end{equation}

% %A \emph{type ascription} can be placed on an unexpanded expression to specify the type that it should be analyzed against. The ascribed type is synthesized if type analysis succeeds.
% \begin{equation}\label{rule:esynP-asc}
%   \inferrule{
%     \cExpandsX{\utau}{\tau}{\akty}\\
%     \eanaPX{\ue}{e}{\tau}
%   }{
%     \esynPX{\asc{\ue}{\utau}}{e}{\tau}
%   }
% \end{equation}

% %We define let-binding of a value in synthetic position primitively in $\miniVerseUB$. The following rule governs such bindings in synthetic position.
% \begin{equation}\label{rule:esynP-let}
%   \inferrule{
%     \esynPX{\ue}{e}{\tau}\\
%     \esynP{\uOmega, \uGhyp{\ux}{x}{\tau}}{\uPsi}{\uPhi}{\ue'}{e'}{\tau'}
%   }{
%     \esynPX{\letsyn{\ux}{\ue}{\ue'}}{\aeap{\aelam{\tau}{x}{e'}}{e}}{\tau'}
%   }
% \end{equation}

% %Functions with an argument type annotation can appear in synthetic position.
% \begin{equation}\label{rule:esynP-lam}
%   \inferrule{
%     \cExpandsX{\utau_1}{\tau_1}{\akty}\\
%     \esynP{\uOmega, \uGhyp{\ux}{x}{\tau_1}}{\uPsi}{\uPhi}{\ue}{e}{\tau_2}
%   }{
%     \esynPX{\lam{\ux}{\utau_1}{\ue}}{\aelam{\tau_1}{x}{e}}{\aparr{\tau_1}{\tau_2}}
%   }
% \end{equation}

% %Function applications can appear in synthetic position. The argument is analyzed against the argument type synthesized by the function.
% \begin{equation}\label{rule:esynP-ap}
%   \inferrule{
%     \esynPX{\ue_1}{e_1}{\aparr{\tau_2}{\tau}}\\
%     \eanaPX{\ue_2}{e_2}{\tau_2}
%   }{
%     \esynPX{\ap{\ue_1}{\ue_2}}{\aeap{e_1}{e_2}}{\tau}
%   }
% \end{equation}

% %Type lambdas and type applications can appear in synthetic position.
% \begin{equation}\label{rule:esynP-tlam}
%   \inferrule{
%     \kExpandsX{\ukappa}{\kappa}\\
%     \esynP{\uOmega, \uKhyp{\uu}{u}{\kappa}}{\uPsi}{\uPhi}{\ue}{e}{\tau}
%   }{
%     \esynPX{\clam{\uu}{\ukappa}{\ue}}{\aeclam{\kappa}{u}{e}}{\aallu{\kappa}{u}{\tau}}
%   }
% \end{equation}
% \begin{equation}\label{rule:esynP-tap}
%   \inferrule{
%     \esynPX{\ue}{e}{\aallu{\kappa}{u}{\tau}}\\
%     \ksynX{\uc}{c}{\kappa}
%   }{
%     \esynPX{\cAp{\ue}{\uc}}{\aecap{e}{c}}{[c/t]\tau}
%   }
% \end{equation}

% %Unfoldings can appear in synthetic position.
% \begin{equation}\label{rule:esynP-unfold}
%   \inferrule{
%     \esynPX{\ue}{e}{\arec{t}{\tau}}
%   }{
%     \esynPX{\unfold{\ue}}{\aeunfold{e}}{[\arec{t}{\tau}/t]\tau}
%   }
% \end{equation}

% %Labeled tuples can appear in synthetic position. Each of the field values are then in synthetic position. 
% \begin{equation}\label{rule:esynP-tpl}
%   \inferrule{
%     \{\esynPX{\ue_i}{e_i}{\tau_i}\}_{i \in \labelset}
%   }{
%     \esynPX{\tpl{\mapschema{\ue}{i}{\labelset}}}{\aetpl{\labelset}{\mapschema{e}{i}{\labelset}}}{\aprod{\labelset}{\mapschema{\tau}{i}{\labelset}}}
%   }
% \end{equation}

% %Fields can be projected out of a labeled tuple in synthetic position.
% \begin{equation}\label{rule:esynP-pr}
%   \inferrule{
%     \esynPX{\ue}{e}{\aprod{\labelset, \ell}{\mapschema{\tau}{i}{\labelset}; \mapitem{\ell}{\tau}}}
%   }{
%     \esynPX{\prj{\ue}{\ell}}{\aepr{\ell}{e}}{\tau}
%   }
% \end{equation}

% %Match expressions can appear in synthetic position.
% \begin{equation}\label{rule:esynP-match}
%   \inferrule{
%     n > 0\\
%     \esynPX{\ue}{e}{\tau}\\
%     \{\rsynPX{\urv_i}{r_i}{\tau}{\tau'}\}_{1 \leq i \leq n}
%   }{
%     \esynPX{\matchwith{\ue}{\seqschemaX{\urv}}}{\aematchwith{n}{e}{\seqschemaX{r}}}{\tau'}
%   }
% \end{equation}

% \begin{equation}\label{rule:esynP-mval}
%   \inferrule{ }{
%     \esynP{\uOmega, \uMhyp{\uX}{X}{\asignature{\kappa}{u}{\tau}}}{\uPsi}{\uPhi}{\mval{\uX}}{\amval{X}}{[\amcon{X}/u]\tau}
%   }
% \end{equation}

% % ueTLMs can be defined and applied in synthetic position.
% % \begin{equation}\label{rule:esynP-defpetsm}
% % \inferrule{
% %   \tsmtyExpands{\uOmega}{\urho}{\rho}\\
% %   \hastypeP{\emptyset}{\eparse}{\aparr{\tBody}{\tParseResultPCEExp}}\\\\
% %   \esynP{\uOmega}{\uASI{\uA \uplus \mapitem{\tsmv}{\adefref{a}}}{\Psi, \petsmdefn{a}{\rho}{\eparse}}{\uI}}{\uPhi}{\ue}{e}{\tau}
% % }{
% %   \esynP{\uOmega}{\uASI{\uA}{\Psi}{\uI}}{\uPhi}{\usyntaxueP{\tsmv}{\urho}{\eparse}{\ue}}{e}{\tau}
% % }
% % \end{equation}

% % \begin{equation}\label{rule:esynP-letpetsm}
% % \inferrule{
% %   \tsmexpExpandsExp{\uOmega}{\uASI{\uA}{\Psi}{\uI}}{\uepsilon}{\epsilon}{\rho}\\
% %   \esynP{\uOmega}{\uASI{\uA\uplus\mapitem{\tsmv}{\epsilon}}{\Psi}{\uI}}{\uPhi}{\ue}{e}{\tau}
% % }{
% %   \esynP{\uOmega}{\uASI{\uA}{\Psi}{\uI}}{\uPhi}{\uletpetsm{\tsmv}{\uepsilon}{\ue}}{e}{\tau}
% % }
% % \end{equation}

% \begin{equation}\label{rule:esynP-apuetsm}
% \inferrule{
%   \uOmega = \uOmegaEx{\uD}{\uG}{\uMctx}{\Omega_\text{app}}\\
%   \uPsi=\uAS{\uA}{\Psi, \petsmdefn{a}{\rho}{\eparse}}{\uI}\\\\
%   \tsmexpExpandsExp{\uOmega}{\uPsi}{\uepsilon}{\epsilon}{\aetype{\tau_\text{final}}}\\
%   \tsmexpEvalsExp{\Omega_\text{app}}{\Psi}{\epsilon}{\epsilon_\text{normal}}\\\\
%   \tsmdefof{\epsilon_\text{normal}}=a\\
%   \encodeBody{b}{\ebody}\\
%   \evalU{\ap{\eparse}{\ebody}}{{\lbltxt{SuccessE}}\cdot{e_\text{pproto}}}\\\\
%   \decodePCEExp{e_\text{pproto}}{\pce}\\\\
%   \prepce{\Omega_\text{app}}{\Psi, \petsmdefn{a}{\rho}{\eparse}}{\pce}{\ce}{\epsilon_\text{normal}}{\aetype{\tau_\text{proto}}}{\omega}{\Omega_\text{params}}\\\\
%   \segOK{\segof{\ce}}{b}\\
%   \canaP{\Omega_\text{params}}{\esceneP{\omega : \OParams}{\uOmega}{\uPsi}{\uPhi}{b}}{\ce}{e}{\tau_\text{proto}}
% }{
%   \esynP{\uOmega}{\uPsi}{\uPhi}{\utsmap{\uepsilon}{b}}{[\omega]e}{[\omega]\tau_\text{proto}}
% }
% \end{equation}

% % These rules are nearly identical to Rules (\ref{rule:expandsUP-syntax}) and (\ref{rule:expandsUP-tsmap}), differing only in that the typed expansion premises have been replaced by corresponding synthetic typed expansion premises. The premises of these rules can be understood as described in Sections \ref{sec:U-uetsm-definition} and \ref{sec:U-uetsm-application}. The body encoding judgement and candidate expansion expression decoding judgements were characterized in Sec. \ref{sec:typed-expansion-UP}. We discuss candidate expansion validation in Sec. \ref{sec:ce-validation-B} below.

% % To support ueTLM implicits, ueTLM contexts, $\uPsi$, are redefined to take the form $\uASI{\uA}{\Psi}{\uI}$. TLM naming contexts, $\uA$, and ueTLM definition contexts, $\Psi$, were defined in Sec. \ref{sec:typed-expansion-UP}. We write $\uPsi, \uShyp{\tsmv}{x}{\tau}{\eparse}$ when $\uPsi=\uASI{\uA}{\Psi}{\uI}$ as shorthand for \[\uASI{\ctxUpdate{\uA}{\tsmv}{x}}{\Psi, \xuetsmbnd{x}{\tau}{\eparse}}{\uI}\]

% % \emph{TLM designation contexts}, $\uI$, are finite functions that map each type $\tau \in \domof{\uI}$ to the \emph{TLM designation} $\designate{\tau}{a}$, for some symbol $x$. We write $\uI \uplus \designate{\tau}{a}$ for the TLM designation context that maps $\tau$ to $\designate{\tau}{a}$ and defers to $\uI$ for all other types (i.e. the previous designation, if any, is updated). 

% % The TLM designation context in the ueTLM context is updated by expressions of ueTLM designation form. Such expressions can appear in synthetic position, where they are governed by the following rule:% We write $\uIOK{\Delta}{\uI}$ when each type in $\uI$ is well-formed assuming $\Delta$.
% %\begin{definition}[TLM Designation Context Well-Formedness] $\uIOK{\Delta}{{\uI}$ iff for each $\designate{\tau}{a}$ we have $\istypeU{\Delta}{\tau}$.\end{definition}

% % \todo{peTLM implicit designation}
% % \begin{equation}\label{rule:esynP-implicite}
% %   \inferrule{
% %     \esyn{\uDelta}{\uGamma}{\uASI{\uA \uplus \vExpands{\tsmv}{x}}{\Psi, \xuetsmbnd{x}{\tau}{\eparse}}{\uI \uplus \designate{\tau}{a}}}{\uPhi}{\ue}{e}{\tau'}
% %   }{
% %     \esyn{\uDelta}{\uGamma}{\uASI{\uA \uplus \vExpands{\tsmv}{x}}{\Psi, \xuetsmbnd{x}{\tau}{\eparse}}{\uI}}{\uPhi}{\implicite{\tsmv}{\ue}}{e}{\tau'}
% %   }
% % \end{equation}

% % % Like ueTLMs, upTLMs can be defined in synthetic position.
% % \begin{equation}\label{rule:esynP-syntaxup}
% % \inferrule{
% %   \tsmtyExpands{\uOmega}{\urho}{\rho}\\
% %   \hastypeP{\emptyset}{\eparse}{\aparr{\tBody}{\tParseResultCEPat}}\\\\
% %   \esynP{\uOmega}{\uPsi}{\uASI{\uA \uplus \mapitem{\tsmv}{\adefref{a}}}{\Phi, \pptsmdefn{a}{\rho}{\eparse}}{\uI}}{\ue}{e}{\tau}
% % }{
% %   \esynP{\uOmega}{\uPsi}{\uASI{\uA}{\Phi}{\uI}}{\usyntaxup{\tsmv}{\urho}{\eparse}{\ue}}{e}{\tau}
% % }
% % \end{equation}


% % \begin{equation}\label{rule:esynP-letpptsm}
% % \inferrule{
% %   \tsmexpExpandsPat{\uOmega}{\uASI{\uA}{\Phi}{\uI}}{\uepsilon}{\epsilon}{\rho}\\
% %   \esynP{\uOmega}{\uPsi}{\uASI{\uA\uplus\mapitem{\tsmv}{\epsilon}}{\Phi}{\uI}}{\ue}{e}{\tau}
% % }{
% %   \esynP{\uOmega}{\uPsi}{\uASI{\uA}{\Phi}{\uI}}{\uletpptsm{\tsmv}{\uepsilon}{\ue}}{e}{\tau}
% % }
% % \end{equation}

% % % This rule is nearly identical to Rule (\ref{rule:expandsUP-defuptsm}), differing only in that the typed expansion premise has been replaced by the corresponding synthetic typed expansion premise. The premises can be understood as described in Section \ref{sec:uptsm-definition}.

% % % To support upTLM implicits, upTLM contexts, $\uPhi$, are redefined to take the form $\uASI{\uA}{\Phi}{\uI}$. upTLM definition contexts, $\Phi$, were defined in Sec. \ref{sec:uptsm-definition}. We write $\uPhi, \uPhyp{\tsmv}{x}{\tau}{\eparse}$ when $\uPhi=\uASI{\uA}{\Phi}{\uI}$ as shorthand for \[\uASI{\ctxUpdate{\uA}{\tsmv}{x}}{\Phi, \xuptsmbnd{a}{\tau}{\eparse}}{\uI}\]

% % % The TLM designation context in the upTLM context is updated by expressions of upTLM designation form. Such expressions can appear in synthetic position, where they are governed by the following rule:% We write $\uIOK{\Delta}{\uI}$ when each type in $\uI$ is well-formed assuming $\Delta$.
% % %\begin{definition}[TLM Designation Context Well-Formedness] $\uIOK{\Delta}{{\uI}$ iff for each $\designate{\tau}{a}$ we have $\istypeU{\Delta}{\tau}$.\end{definition}
% % \todo{ppTLM implicit designation}
% % \begin{equation}\label{rule:esynP-implicitp}
% %   \inferrule{
% %     \esyn{\uDelta}{\uGamma}{\uPsi}{\uASI{\uA\uplus\vExpands{\tsmv}{x}}{\Phi, \xuptsmbnd{a}{\tau}{\eparse}}{\uI \uplus \designate{\tau}{a}}}{\ue}{e}{\tau'}
% %   }{
% %     \esyn{\uDelta}{\uGamma}{\uPsi}{\uASI{\uA\uplus\vExpands{\tsmv}{x}}{\Phi, \xuetsmbnd{x}{\tau}{\eparse}}{\uI}}{\implicitp{\tsmv}{\ue}}{e}{\tau'}
% %   }
% % \end{equation}
% \end{subequations}


% Analytic typed expression expansion
% \begin{subequations}\label{rules:eanaP}
% % Type analysis subsumes type synthesis, in that when a type can be synthesized for an unexpanded expression, that unexpanded expression can also be analyzed against that type, producing the same expansion. This is expressed by the following \emph{subsumption rule} for unexpanded expressions.
% \begin{equation}\label{rule:eanaP-subsume}
%   \inferrule{
%     \esynPX{\ue}{e}{\tau}\\
%     \issubtypePX{\tau}{\tau'}
%   }{
%     \eanaPX{\ue}{e}{\tau'}
%   }
% \end{equation}

% % Additional rules are needed for certain forms in order to propagate types for analysis into subexpressions, and for forms that can appear only in analytic position.

% % Rule (\ref{rule:esyn-let}) governed value bindings in synthetic position. The following rule governs value bindings in analytic position.
% \begin{equation}\label{rule:eanaP-let}
%   \inferrule{
%     \esynPX{\ue}{e}{\tau}\\
%     \eanaP{\uOmega, \uGhyp{\ux}{x}{\tau}}{\uPsi}{\uPhi}{\ue'}{e'}{\tau'}
%   }{
%     \eanaPX{\letsyn{\ux}{\ue}{\ue'}}{\aeap{\aelam{\tau}{x}{e'}}{e}}{\tau'}
%   }
% \end{equation}

% % An unannotated function can appear only in analytic position. The argument type is determined from the type that the unannotated function is being analyzed against. 
% \begin{equation}\label{rule:eanaP-analam}
%   \inferrule{
%     \eanaP{\uOmega, \uGhyp{\ux}{x}{\tau_1}}{\uPsi}{\uPhi}{\ue}{e}{\tau_2}
%   }{
%     \eanaPX{\analam{\ux}{\ue}}{\aelam{\tau_1}{x}{e}}{\aparr{\tau_1}{\tau_2}}
%   }
% \end{equation}

% % Rule (\ref{rule:esyn-tlam}) governed type lambdas in synthetic position. The following rule governs type lambdas in analytic position.
% % \begin{equation}\label{rule:eanaP-tlam}
% %   \inferrule{
% %     \eana{\uDelta, \uDhyp{\ut}{t}}{\uGamma}{\uPsi}{\uPhi}{\ue}{e}{\tau}
% %   }{
% %     \eanaPX{\clam{\uu}{\ue}}{\aetlam{t}{e}}{\aall{t}{\tau}}
% %   }
% % \end{equation}

% % Values of recursive types can be introduced only in analytic position.
% \begin{equation}\label{rule:eanaP-fold}
%   \inferrule{
%     \eanaPX{\ue}{e}{[\arec{t}{\tau}/t]\tau}
%   }{
%     \eanaPX{\fold{\ue}}{\aefold{e}}{\arec{t}{\tau}}
%   }
% \end{equation}

% % Rule (\ref{rule:esyn-tpl}) governed labeled tuples in synthetic position. The following rule governs labeled tuples in analytic position.
% \begin{equation}\label{rule:eanaP-tpl}
%   \inferrule{
%     \{\eanaPX{\ue_i}{e_i}{\tau_i}\}_{i \in \labelset}
%   }{
%     \eanaPX{\tpl{\mapschema{\ue}{i}{\labelset}}}{\aetpl{\labelset}{\mapschema{e}{i}{\labelset}}}{\aprod{\labelset}{\mapschema{\tau}{i}{\labelset}}}
%   }
% \end{equation}

% % Values of labeled sum type can appear only in analytic position.
% \begin{equation}\label{rule:eanaP-in}
%   \inferrule{
%     \tau = \asum{\labelset, \ell}{\mapschema{\tau}{i}{\labelset}; \mapitem{\ell}{\tau'}}\\\\
%     \eanaPX{\ue'}{e'}{\tau'}
%   }{
%     \eanaPX{\inj{\ell}{\ue}}{\aein{\ell}{e'}}{\tau}
%     % \uOmega \vdash_{\uPsi; \uPhi} \left(\shortstack{$\ue \leadsto $\\$\Leftarrow$\vspace{-1.2em}}\right)
%     %\eanaPX{\auanain{\ell}{\ue}}{\aein{\ell}}{\asum{\labelset, \ell}{\mapschema{\tau}{i}{\labelset}; \mapitem{\ell}{\tau}}}
%   }
% \end{equation}

% % Rule (\ref{rule:esyn-match}) governed match expressions in synthetic position. The following rule governs match expressions in analytic position.
% \begin{equation}\label{rule:eanaP-match}
%   \inferrule{
%     \esynPX{\ue}{e}{\tau}\\
%     \{\ranaPX{\urv_i}{r_i}{\tau}{\tau'}\}_{1 \leq i \leq n}
%   }{
%     \eanaPX{\matchwith{\ue}{\seqschemaX{\urv}}}{\aematchwith{n}{e}{\seqschemaX{r}}}{\tau'}
%   }
% \end{equation}

% % Rule (\ref{rule:esyn-defuetsm}) governed ueTLM definitions in synthetic position. The following rule governs ueTLM definitions in analytic position.
% % \begin{equation}\label{rule:eanaP-defpetsm}
% % \inferrule{
% %   \tsmtyExpands{\uOmega}{\urho}{\rho}\\
% %   \hastypeP{\emptyset}{\eparse}{\aparr{\tBody}{\tParseResultPCEExp}}\\\\
% %   \eanaP{\uOmega}{\uASI{\uA \uplus \mapitem{\tsmv}{\adefref{a}}}{\Psi, \petsmdefn{a}{\rho}{\eparse}}{\uI}}{\uPhi}{\ue}{e}{\tau}
% % }{
% %   \eanaP{\uOmega}{\uASI{\uA}{\Psi}{\uI}}{\uPhi}{\usyntaxueP{\tsmv}{\urho}{\eparse}{\ue}}{e}{\tau}
% % }
% % \end{equation}

% % \begin{equation}\label{rule:eanaP-letpetsm}
% % \inferrule{
% %   \tsmexpExpandsExp{\uOmega}{\uASI{\uA}{\Psi}{\uI}}{\uepsilon}{\epsilon}{\rho}\\
% %   \eanaP{\uOmega}{\uASI{\uA\uplus\mapitem{\tsmv}{\epsilon}}{\Psi}{\uI}}{\uPhi}{\ue}{e}{\tau}
% % }{
% %   \eanaP{\uOmega}{\uASI{\uA}{\Psi}{\uI}}{\uPhi}{\uletpetsm{\tsmv}{\uepsilon}{\ue}}{e}{\tau}
% % }
% % \end{equation}

% % \todo{peTLM implicit designation}
% % Rule (\ref{rule:esyn-implicite}) governed ueTLM designations in synthetic position. The following rule governs ueTLM designations in analytic position.
% % \begin{equation}\label{rule:eanaP-implicite}
% %   \inferrule{
% %     \eana{\uDelta}{\uGamma}{\uASI{\uA \uplus \vExpands{\tsmv}{x}}{\Psi, \xuetsmbnd{x}{\tau}{\eparse}}{\uI \uplus \designate{\tau}{a}}}{\uPhi}{\ue}{e}{\tau'}
% %   }{
% %     \eana{\uDelta}{\uGamma}{\uASI{\uA \uplus \vExpands{\tsmv}{x}}{\Psi, \xuetsmbnd{x}{\tau}{\eparse}}{\uI}}{\uPhi}{\implicite{\tsmv}{\ue}}{e}{\tau'}
% %   }
% % \end{equation}

% % \todo{peTLM implicit application}
% % % An expression of unadorned literal form can appear only in analytic position. The following rule extracts the TLM designated at the type that the expression is being analyzed against from the TLM designation context in the ueTLM context and applies it implicitly, i.e. the premises correspond to those of Rule (\ref{rule:esyn-apuetsm}).
% \begin{equation}\label{rule:eanaP-lit}
%   \inferrule{
%     \encodeBody{b}{\ebody}\\
%     \evalU{\ap{\eparse}{\ebody}}{\inj{\lbltxt{Success}}{\ecand}}\\
%     \decodeCondE{\ecand}{\ce}\\\\
%     \cana{\emptyset}{\emptyset}{\esceneUP{\uDelta}{\uGamma}{\uASI{\uA}{\Psi, \xuetsmbnd{x}{\tau}{\eparse}}{\uI \uplus \designate{\tau}{a}}}{\uPhi}{b}}{\ce}{e}{\tau}
%   }{
%     \eana{\uDelta}{\uGamma}{\uASI{\uA}{\Psi, \xuetsmbnd{x}{\tau}{\eparse}}{\uI \uplus \designate{\tau}{a}}}{\uPhi}{\auelit{b}}{e}{\tau}
%   }
% \end{equation}

% % Rule (\ref{rule:esyn-defuptsm}) governed upTLM definitions in synthetic position. The following rule governs upTLM definitions in analytic position.
% % \begin{equation}\label{rule:eanaP-syntaxup}
% % \inferrule{
% %   \tsmtyExpands{\uOmega}{\urho}{\rho}\\
% %   \hastypeP{\emptyset}{\eparse}{\aparr{\tBody}{\tParseResultCEPat}}\\\\
% %   \eanaP{\uOmega}{\uPsi}{\uASI{\uA \uplus \mapitem{\tsmv}{\adefref{a}}}{\Phi, \pptsmdefn{a}{\rho}{\eparse}}{\uI}}{\ue}{e}{\tau}
% % }{
% %   \eanaP{\uOmega}{\uPsi}{\uASI{\uA}{\Phi}{\uI}}{\usyntaxup{\tsmv}{\urho}{\eparse}{\ue}}{e}{\tau}
% % }
% % \end{equation}


% % \begin{equation}\label{rule:eanaP-letpptsm}
% % \inferrule{
% %   \tsmexpExpandsPat{\uOmega}{\uASI{\uA}{\Phi}{\uI}}{\uepsilon}{\epsilon}{\rho}\\
% %   \eanaP{\uOmega}{\uPsi}{\uASI{\uA\uplus\mapitem{\tsmv}{\epsilon}}{\Phi}{\uI}}{\ue}{e}{\tau}
% % }{
% %   \eanaP{\uOmega}{\uPsi}{\uASI{\uA}{\Phi}{\uI}}{\uletpptsm{\tsmv}{\uepsilon}{\ue}}{e}{\tau}
% % }
% % \end{equation}


% % \todo{ppTLM implicit designation}
% % % Rule (\ref{rule:esyn-implicitp}) governed upTLM designations in synthetic position. The following rule governs upTLM designations in analytic position.
% % \begin{equation}\label{rule:eanaP-implicitp}
% %   \inferrule{
% %     \eana{\uDelta}{\uGamma}{\uPsi}{\uASI{\uA\uplus\vExpands{\tsmv}{x}}{\Phi, \xuptsmbnd{a}{\tau}{\eparse}}{\uI \uplus \designate{\tau}{a}}}{\ue}{e}{\tau'}
% %   }{
% %     \eana{\uDelta}{\uGamma}{\uPsi}{\uASI{\uA\uplus\vExpands{\tsmv}{x}}{\Phi, \xuetsmbnd{x}{\tau}{\eparse}}{\uI}}{\implicitp{\tsmv}{\ue}}{e}{\tau'}
% %   }
% % \end{equation}

% \end{subequations}

% Synthetic rule expansion
% %The synthetic typed rule expansion judgement is invoked iteratively by Rule (\ref{rule:esyn-match}) to synthesize a type, $\tau'$, from the branch expressions in the rule sequence. This judgement is defined mutually inductively with Rules (\ref{rules:esyn}) and Rules (\ref{rules:eana}) by the following rule. 
% \begin{equation}\label{rule:rsynP}
%   \inferrule{
%     \uOmega=\uOmegaEx{\uD}{\uG}{\uMctx}{\Omega}\\
%     \patExpandsP{\uOmegaEx{\emptyset}{\uG'}{\emptyset}{\Omega'}}{\uPhi}{\upv}{p}{\tau}\\
%     \esynP{\uOmegaEx{\uD}{\uG \uplus \uG'}{\uMctx}{\Omega \cup \Omega'}}{\uPsi}{\uPhi}{\ue}{e}{\tau'}
%   }{
%     \rsynP{\uOmega}{\uPsi}{\uPhi}{\matchrule{\upv}{\ue}}{\aematchrule{p}{e}}{\tau}{\tau'}
%   }
% \end{equation}

% Analytic rule expansion
% %The analytic typed rule expansion judgement is invoked iteratively by Rule (\ref{rule:eana-match}). This judgement is defined mutually inductively with Rules (\ref{rules:esyn}), Rules (\ref{rules:eana}), and Rule (\ref{rule:rsyn}) by the following rule, which is the analytic analag of Rule (\ref{rule:rsyn}).
% \begin{equation}\label{rule:ranaP}
%   \inferrule{
%     \uOmega=\uOmegaEx{\uD}{\uG}{\uMctx}{\Omega}\\
%     \patExpandsP{\uOmegaEx{\emptyset}{\uG'}{\emptyset}{\Omega'}}{\uPhi}{\upv}{p}{\tau}\\
%     \eanaP{\uOmegaEx{\uD}{\uG \uplus \uG'}{\uMctx}{\Omega \cup \Omega'}}{\uPsi}{\uPhi}{\ue}{e}{\tau'}
%   }{
%     \ranaP{\uOmega}{\uPsi}{\uPhi}{\matchrule{\upv}{\ue}}{\aematchrule{p}{e}}{\tau}{\tau'}
%   }
% \end{equation}

% %The premises of these rules can be understood as described in Sec. \ref{sec:typed-expansion-UP}.% We will define typed pattern expansion below.

% Typed pattern expansion
% % The typed pattern expansion judgement is inductively defined by Rules (\ref{rules:patExpandsP}) as follows. %As in $\miniVersePat$, \emph{unexpanded pattern typing contexts}, $\upctx$, are defined identically to unexpanded typing contexts (i.e. we only use a distinct metavariable to emphasize their distinct roles in the judgements above). 

% % The following rules are written identically to the typed pattern expansion rules for shared pattern forms in $\miniVersePat$, i.e. Rules (\ref{rule:patExpands-var}) through (\ref{rule:patExpands-in}).
% \begin{subequations}\label{rules:patExpandsP-B}
% \begin{equation}\label{rule:patExpandsP-B-subsume}
% \inferrule{
%   \uOmega=\uOmegaEx{\uD}{\uG}{\uMctx}{\Omega}\\\\
%   \patExpandsP{\uOmega'}{\uPhi}{\upv}{p}{\tau}\\
%   \issubtypeP{\Omega}{\tau}{\tau'}
% }{
%   \patExpandsP{\uOmega'}{\uPhi}{\upv}{p}{\tau'}
% }
% \end{equation}
% \begin{equation}\label{rule:patExpandsP-B-var}
% \inferrule{ }{
%   \patExpandsP{\uOmegaEx{\emptyset}{\vExpands{\ux}{x}}{\emptyset}{\Ghyp{x}{\tau}}}{\uPhi}{\ux}{x}{\tau}
% }
% \end{equation}
% \begin{equation}\label{rule:patExpandsP-B-wild}
% \inferrule{ }{
%   \patExpandsP{\uOmegaEx{\emptyset}{\emptyset}{\emptyset}{\emptyset}}{\uPhi}{\wildp}{\aewildp}{\tau}
% }
% \end{equation}
% \begin{equation}\label{rule:patExpandsP-B-fold}
% \inferrule{ 
%   \patExpandsP{\uOmega'}{\uPhi}{\upv}{p}{[\arec{t}{\tau}/t]\tau}
% }{
%   \patExpandsP{\uOmega'}{\uPhi}{\foldp{\upv}}{\aefoldp{p}}{\arec{t}{\tau}}
% }
% \end{equation}
% \begin{equation}\label{rule:patExpandsP-B-tpl}
% \inferrule{
%   \tau=\aprod{\labelset}{\mapschema{\tau}{i}{\labelset}}\\\\
%   \{\patExpandsP{{\uOmega_i}}{\uPhi}{\upv_i}{p_i}{\tau_i}\}_{i \in \labelset}
% }{
%   %\patExpandsP{\Gconsi{i \in \labelset}{\upctx_i}}{A}{B}{C}
%   \patExpandsP{\Gconsi{i \in \labelset}{\uOmega_i}}{\uPhi}{\tplp{\mapschema{\upv}{i}{\labelset}}}{\aetplp{\labelset}{\mapschema{p}{i}{\labelset}}}{\tau}
%   % \patExpands{\Gconsi{i \in \labelset}{\pctx_i}}{\Phi}{
%   %   \autplp{\labelset}{\mapschema{\upv}{i}{\labelset}}
%   % }{
%   %   \aetplp{\labelset}{\mapschema{p}{i}{\labelset}}
%   % }{
%   %   \aprod{\labelset}{\mapschema{\tau}{i}{\labelset}}
%   % } %{\autplp{\labelset}{\mapschema{\upv}{i}{\labelset}}}{\aetplp{\labelset}{\mapschema}{p}{i}{\labelset}}{...}
%   %\left(\shortstack{$\Delta \vdash_{\uPhi} \autplp{\labelset}{\mapschema{\upv}{i}{\labelset}}$\\$\leadsto$\\$\aetplp{\labelset}{\mapschema{p}{i}{\labelset}} : \aprod{\labelset}{\mapschema{\tau}{i}{\labelset}} \dashV \Gconsi{i \in \labelset}{\upctx_i}$\vspace{-1.2em}}\right)
% }
% \end{equation}
% \begin{equation}\label{rule:patExpandsP-B-in}
% \inferrule{
%   \patExpandsP{\uOmega'}{\uPhi}{\upv}{p}{\tau}
% }{
%   \patExpandsP{\uOmega'}{\uPhi}{\injp{\ell}{\upv}}{\aeinjp{\ell}{p}}{\asum{\labelset, \ell}{\mapschema{\tau}{i}{\labelset}; \mapitem{\ell}{\tau}}}
% }
% \end{equation}

% \begin{equation}\label{rule:patExpandsP-B-apuptsm}
% \inferrule{
%   \uOmega=\uOmegaEx{\uD}{\uG}{\uMctx}{\Omega_\text{app}}\\
%   \uPhi=\uASI{\uA}{\Phi, \pptsmdefn{a}{\rho}{\eparse}}{\uI}\\\\
%   \tsmexpExpandsPat{\uOmega}{\uPhi}{\uepsilon}{\epsilon}{\aetype{\tau_\text{final}}}\\
%   \tsmdefof{\epsilon}=a\\\\
%   \encodeBody{b}{\ebody}\\
%   \evalU{\ap{\eparse}{\ebody}}{{\lbltxt{Success}}\cdot{\ecand}}\\
%   \decodeCEPat{\ecand}{\cpv}\\\\
%   \prepcp{\Omega_\text{app}}{\Phi, \pptsmdefn{a}{\rho}{\eparse}}{\pcp}{\cpv}{\epsilon}{\aetype{\tau_\text{cand}}}{\omega}{\Omega_\text{params}}\\\\
%   \cvalidPP{\uOmega'}{\psceneP{\uOmega}{\uPhi}{b}}{\cpv}{p}{\tau_\text{cand}}
% }{
%   \patExpandsP{\uOmega'}{\uPhi}{\utsmap{\uepsilon}{b}}{p}{[\omega]\tau_\text{cand}}
% }
% \end{equation}

% \todo{ppTLM implicit application}
% % Unexpanded patterns of unadorned literal form are governed by the following rule, which extracts the designated upTLM from the upTLM context and applies it implicitly, i.e. the premises correspond to those of Rule (\ref{rule:patExpandsP-apuptsm}).
% \begin{equation}\label{rule:patExpandsP-B-lit}
% \inferrule{
%   \encodeBody{b}{\ebody}\\
%   \evalU{\ap{\eparse}{\ebody}}{\inj{\lbltxt{Success}}{\ecand}}\\
%   \decodeCEPat{\ecand}{\cpv}\\\\
%   \cvalidPP{\uOmega}{\pscene{\uDelta}{\uASI{\uA}{\Phi, \xuptsmbnd{a}{\tau}{\eparse}}{\uI, \designate{\tau}{a}}}{b}}{\cpv}{p}{\tau}
% }{
%   \patExpands{\uOmega}{\uASI{\uA}{\Phi, \xuptsmbnd{a}{\tau}{\eparse}}{\uI, \designate{\tau}{a}}}{\lit{b}}{p}{\tau}
% }
% \end{equation}

% \end{subequations}


% \subsubsection{Unexpanded Signatures and Module Expressions}
% Signature expansion
% \begin{equation}\label{rule:sigExpandsP-B}
% \inferrule{
%   \kExpandsX{\ukappa}{\kappa}\\
%   \cExpands{\uOmega, \uKhyp{\uu}{u}{\kappa}}{\utau}{\tau}{\akty}
% }{
%   \sigExpandsPX{\signature{\uu}{\ukappa}{\utau}}{\asignature{\kappa}{u}{\tau}}
% }
% \end{equation}

% Synthetic module expression expansion
% \begin{subequations}\label{rules:msyn}
% \begin{equation}\label{rule:msyn-var}
% \inferrule{ }{
%   \msyn{\uOmega, \uMhyp{\uX}{X}{\sigma}}{\uPsi}{\uPhi}{\uX}{X}{\sigma}
% }
% \end{equation}
% \begin{equation}\label{rule:msyn-seal}
% \inferrule{
%   \sigExpandsPX{\usigma}{\sigma}\\
%   \manaX{\uM}{M}{\sigma}
% }{
%   \msynX{\seal{\uM}{\usigma}}{\aseal{\sigma}{M}}{\sigma} 
% }
% \end{equation}
% \begin{equation}\label{rule:msyn-mlet}
% \inferrule{
%   \msynX{\uM}{M}{\sigma}\\
%   \sigExpandsPX{\usigma'}{\sigma'}\\\\
%   \mana{\uOmega, \uMhyp{\uX}{X}{\sigma}}{\uPsi}{\uPhi}{\uM'}{M'}{\sigma'}
% }{
%   \msynX{\mlet{\uX}{\uM}{\uM'}{\usigma'}}{\amlet{\sigma'}{M}{X}{M'}}{\sigma'}
% }
% \end{equation}
% \begin{equation}\label{rule:msyn-syntaxpe}
% \inferrule{
%   \tsmtyExpands{\uOmega}{\urho}{\rho}\\
%   \hastypeP{\emptyset}{\eparse}{\aparr{\tBody}{\tParseResultPCEExp}}\\\\
%   \msyn{\uOmega}{\uASI{\uA \uplus \mapitem{\tsmv}{\adefref{a}}}{\Psi, \petsmdefn{a}{\rho}{\eparse}}{\uI}}{\uPhi}{\uM}{M}{\sigma}
% }{
%   \msyn{\uOmega}{\uASI{\uA}{\Psi}{\uI}}{\uPhi}{\usyntaxueP{\tsmv}{\urho}{\eparse}{\uM}}{M}{\sigma}
% }
% \end{equation}
% \begin{equation}\label{rule:msyn-letpetsm}
% \inferrule{
%   \tsmexpExpandsExp{\uOmega}{\uASI{\uA}{\Psi}{\uI}}{\uepsilon}{\epsilon}{\rho}\\
%   \msyn{\uOmega}{\uASI{\uA\uplus\mapitem{\tsmv}{\epsilon}}{\Psi}{\uI}}{\uPhi}{\uM}{M}{\sigma}
% }{
%   \msyn{\uOmega}{\uASI{\uA}{\Psi}{\uI}}{\uPhi}{\uletpetsm{\tsmv}{\uepsilon}{\uM}}{M}{\sigma}
% }
% \end{equation}
% \todo{peTLM implicit designation at module level}
% \begin{equation}\label{rule:msyn-implicitpe}
% \inferrule{
%   ...
% }{
%   ...
% }
% \end{equation}
% \begin{equation}\label{rule:msyn-syntaxpp}
% \inferrule{
%   \tsmtyExpands{\uOmega}{\urho}{\rho}\\
%   \hastypeP{\emptyset}{\eparse}{\aparr{\tBody}{\tParseResultCEPat}}\\\\
%   \msyn{\uOmega}{\uPsi}{\uASI{\uA \uplus \mapitem{\tsmv}{\adefref{a}}}{\Phi, \pptsmdefn{a}{\rho}{\eparse}}{\uI}}{\uM}{M}{\sigma}
% }{
%   \msyn{\uOmega}{\uPsi}{\uASI{\uA}{\Phi}{\uI}}{\usyntaxup{\tsmv}{\urho}{\eparse}{\uM}}{M}{\sigma}
% }
% \end{equation}
% \begin{equation}\label{rule:msyn-letpptsm}
% \inferrule{
%   \tsmexpExpandsPat{\uOmega}{\uASI{\uA}{\Phi}{\uI}}{\uepsilon}{\epsilon}{\rho}\\
%   \msyn{\uOmega}{\uPsi}{\uASI{\uA\uplus\mapitem{\tsmv}{\epsilon}}{\Phi}{\uI}}{\uM}{M}{\sigma}
% }{
%   \msyn{\uOmega}{\uPsi}{\uASI{\uA}{\Phi}{\uI}}{\uletpptsm{\tsmv}{\uepsilon}{\uM}}{M}{\sigma}
% }
% \end{equation}
% \todo{ppTLM implicit designation at module level}
% \begin{equation}\label{rule:msyn-implicitpp}
% \inferrule{
%   ...
% }{
%   ...
% }
% \end{equation}
% \end{subequations}

% Analytic module expression expansion
% \begin{subequations}\label{rules:mana}
% \begin{equation}\label{rule:mana-subsumes}
% \inferrule{
%   \msynX{\uM}{M}{\sigma}\\
%   \sigsub{\uOmega}{\sigma}{\sigma'}
% }{
%   \manaX{\uM}{M}{\sigma'}
% }
% \end{equation}
% \begin{equation}\label{rule:mana-struct}
% \inferrule{
%   \kanaX{\uc}{c}{\kappa}\\
%   \eanaPX{\ue}{e}{[c/u]\tau}
% }{
%   \manaX{\struct{\uc}{\ue}}{\astruct{c}{e}}{\asignature{\kappa}{u}{\tau}}
% }
% \end{equation}
% \begin{equation}\label{rule:mana-syntaxpe}
% \inferrule{
%   \tsmtyExpands{\uOmega}{\urho}{\rho}\\
%   \hastypeP{\emptyset}{\eparse}{\aparr{\tBody}{\tParseResultPCEExp}}\\\\
%   \mana{\uOmega}{\uASI{\uA \uplus \mapitem{\tsmv}{\adefref{a}}}{\Psi, \petsmdefn{a}{\rho}{\eparse}}{\uI}}{\uPhi}{\uM}{M}{\sigma}
% }{
%   \mana{\uOmega}{\uASI{\uA}{\Psi}{\uI}}{\uPhi}{\usyntaxueP{\tsmv}{\urho}{\eparse}{\uM}}{M}{\sigma}
% }
% \end{equation}
% \begin{equation}\label{rule:mana-letpetsm}
% \inferrule{
%   \tsmexpExpandsExp{\uOmega}{\uASI{\uA}{\Psi}{\uI}}{\uepsilon}{\epsilon}{\rho}\\
%   \mana{\uOmega}{\uASI{\uA\uplus\mapitem{\tsmv}{\epsilon}}{\Psi}{\uI}}{\uPhi}{\uM}{M}{\sigma}
% }{
%   \mana{\uOmega}{\uASI{\uA}{\Psi}{\uI}}{\uPhi}{\uletpetsm{\tsmv}{\uepsilon}{\uM}}{M}{\sigma}
% }
% \end{equation}
% \todo{peTLM implicit designation at module level}
% \begin{equation}\label{rule:mana-implicitpe}
% \inferrule{
%   ...
% }{
%   ...
% }
% \end{equation}
% \begin{equation}\label{rule:mana-syntaxpp}
% \inferrule{
%   \tsmtyExpands{\uOmega}{\urho}{\rho}\\
%   \hastypeP{\emptyset}{\eparse}{\aparr{\tBody}{\tParseResultCEPat}}\\\\
%   \mana{\uOmega}{\uPsi}{\uASI{\uA \uplus \mapitem{\tsmv}{\adefref{a}}}{\Phi, \pptsmdefn{a}{\rho}{\eparse}}{\uI}}{\uM}{M}{\sigma}
% }{
%   \mana{\uOmega}{\uPsi}{\uASI{\uA}{\Phi}{\uI}}{\usyntaxup{\tsmv}{\urho}{\eparse}{\uM}}{M}{\sigma}
% }
% \end{equation}
% \begin{equation}\label{rule:mana-letpptsm}
% \inferrule{
%   \tsmexpExpandsPat{\uOmega}{\uASI{\uA}{\Phi}{\uI}}{\uepsilon}{\epsilon}{\rho}\\
%   \mana{\uOmega}{\uPsi}{\uASI{\uA\uplus\mapitem{\tsmv}{\epsilon}}{\Phi}{\uI}}{\uM}{M}{\sigma}
% }{
%   \mana{\uOmega}{\uPsi}{\uASI{\uA}{\Phi}{\uI}}{\uletpptsm{\tsmv}{\uepsilon}{\uM}}{M}{\sigma}
% }
% \end{equation}
% \todo{ppTLM implicit designation at module level}
% \begin{equation}\label{rule:mana-implicitpp}
% \inferrule{
%   ...
% }{
%   ...
% }
% \end{equation}
% \end{subequations}

% \subsubsection{TLM Types and Expressions}
% TLM Expression Typing

% \vspace{10px}
% $\begin{array}{ll}
% \textbf{Judgement Form} & \textbf{Description}\\
% \istsmty{\Omega}{\rho} & \text{$\rho$ is a well-formed TLM type}\\
% \hastsmtypeExp{\Omega}{\Psi}{\epsilon}{\rho} & \text{peTLM expression $\epsilon$ has TLM type $\rho$}\\
% \hastsmtypePat{\Omega}{\Phi}{\epsilon}{\rho} & \text{ppTLM expression $\epsilon$ has TLM type $\rho$}
% \end{array}$
% \vspace{10px}

% peTLM Expression Evaluation

% \vspace{10px}
% $\begin{array}{ll}
% \textbf{Judgement Form} & \textbf{Description}\\
% \tsmexpNormalExp{\Omega}{\Psi}{\epsilon} & \text{peTLM expression $\epsilon$ is in normal form}\\
% \tsmexpStepsExp{\Omega}{\Psi}{\epsilon}{\epsilon'} & \text{peTLM expression $\epsilon$ transitions to $\epsilon'$}\\
% \end{array}$
% \vspace{10px}

% + auxiliary judgements for multi-step transitions and evaluation

% unexpanded TLM types and expressions

% \vspace{10px}
% $\begin{array}{ll}
% \textbf{Judgement Form} & \textbf{Description}\\
% \tsmtyExpands{\uOmega}{\urho}{\rho} & \text{$\urho$ has expansion $\rho$}\\
% \tsmexpExpandsExp{\uOmega}{\uPsi}{\uepsilon}{\epsilon}{\rho} & \text{unexpanded peTLM expression $\uepsilon$ has expansion $\epsilon$ and type $\rho$}\\
% \tsmexpExpandsPat{\uOmega}{\uPhi}{\uepsilon}{\epsilon}{\rho} & \text{unexpanded ppTLM expression $\uepsilon$ has expansion $\epsilon$ and type $\rho$}
% \end{array}$
% \vspace{10px}

% TLM type formation
% \begin{subequations}\label{rules:istsmty-B}
% \begin{equation}\label{rule:istsmty-B-type}
% \inferrule{
%   \haskindX{\tau}{\akty}
% }{
%   \istsmty{\Omega}{\aetype{\tau}}
% }
% \end{equation}
% \begin{equation}\label{rule:istsmty-B-alltypes}
% \inferrule{
%   \istsmty{\Omega, t :: \akty}{\rho}
% }{
%   \istsmty{\Omega}{\aealltypes{t}{\rho}}
% }
% \end{equation}
% \begin{equation}\label{rule:istsmty-B-allmods}
% \inferrule{
%   \issig{\Omega}{\sigma}\\
%   \istsmty{\Omega, X : \sigma}{\rho}
% }{
%   \istsmty{\Omega}{\aeallmods{\sigma}{X}{\rho}}
% }
% \end{equation}
% \end{subequations}

% Unexpanded TLM type expansion
% \begin{subequations}\label{rules:tsmtyExpands-B}
% \begin{equation}\label{rule:tsmtyExpands-B-type}
% \inferrule{
%   \cExpandsX{\utau}{\tau}{\akty}
% }{
%   \tsmtyExpands{\uOmega}{{\utau}}{\aetype{\tau}}
% }
% \end{equation}
% \begin{equation}\label{rule:tsmtyExpands-B-alltypes}
% \inferrule{
%   \tsmtyExpands{\uOmega, \uKhyp{\ut}{t}{\akty}}{\urho}{\rho}
% }{
%   \tsmtyExpands{\uOmega}{\alltypes{\ut}{\urho}}{\aealltypes{t}{\rho}}
% }
% \end{equation}
% \begin{equation}\label{rule:tsmtyExpands-B-allmods}
% \inferrule{
%   \sigExpandsPX{\usigma}{\sigma}\\
%   \tsmtyExpands{\uOmega, \uMhyp{\uX}{X}{\sigma}}{\urho}{\rho}
% }{
%   \tsmtyExpands{\uOmega}{\allmods{\uX}{\usigma}{\urho}}{\aeallmods{\sigma}{X}{\rho}}
% }
% \end{equation}
% \end{subequations}
% peTLM Expression Typing
% \begin{subequations}\label{rules:hastsmtypeExp-B}
% \begin{equation}\label{rule:hastsmtypeExp-B-defref}
% \inferrule{ }{
%   \hastsmtypeExp{\Omega}{\Psi, \petsmdefn{a}{\rho}{\eparse}}{\adefref{a}}{\rho}
% }
% \end{equation}
% \begin{equation}\label{rule:hastsmtypeExp-B-abstype}
% \inferrule{
%   \hastsmtypeExp{\Omega, t :: \akty}{\Psi}{\epsilon}{\rho}
% }{
%   \hastsmtypeExp{\Omega}{\Psi}{\aeabstype{t}{\epsilon}}{\aealltypes{t}{\rho}}
% }
% \end{equation}
% \begin{equation}\label{rule:hastsmtypeExp-B-absmod}
% \inferrule{
%   \issigX{\sigma}\\
%   \hastsmtypeExp{\Omega, X : \sigma}{\Psi}{\epsilon}{\rho}
% }{
%   \hastsmtypeExp{\Omega}{\Psi}{\aeabsmod{\sigma}{X}{\epsilon}}{\aeallmods{\sigma}{X}{\rho}}
% }
% \end{equation}
% \begin{equation}\label{rule:hastsmtypeExp-B-aptype}
% \inferrule{
%   \hastsmtypeExp{\Omega}{\Psi}{\epsilon}{\aealltypes{t}{\rho}}\\
%   \haskindX{\tau}{\akty}
% }{
%   \hastsmtypeExp{\Omega}{\Psi}{\aeaptype{\tau}{\epsilon}}{[\tau/t]\rho}
% }
% \end{equation}
% \begin{equation}\label{rule:hastsmtypeExp-B-apmod}
% \inferrule{
%   \hastsmtypeExp{\Omega}{\Psi}{\epsilon}{\aeallmods{\sigma}{X'}{\rho}}\\
%   \hassig{\Omega}{X}{\sigma}
% }{
%   \hastsmtypeExp{\Omega}{\Psi}{\aeapmod{X}{\epsilon}}{[X/X']\rho}
% }
% \end{equation}
% \end{subequations}

% ppTLM Expression Typing
% \begin{subequations}\label{rules:hastsmtypePat-B}
% \begin{equation}\label{rule:hastsmtypePat-B-defref}
% \inferrule{ }{
%   \hastsmtypePat{\Omega}{\Phi, \pptsmdefn{a}{\rho}{\eparse}}{\adefref{a}}{\rho}
% }
% \end{equation}
% \begin{equation}\label{rule:hastsmtypePat-B-abstype}
% \inferrule{
%   \hastsmtypePat{\Omega, t :: \akty}{\Phi}{\epsilon}{\rho}
% }{
%   \hastsmtypePat{\Omega}{\Phi}{\aeabstype{t}{\epsilon}}{\aealltypes{t}{\rho}}
% }
% \end{equation}
% \begin{equation}\label{rule:hastsmtypePat-B-absmod}
% \inferrule{
%   \issigX{\sigma}\\
%   \hastsmtypePat{\Omega, X : \sigma}{\Phi}{\epsilon}{\rho}
% }{
%   \hastsmtypePat{\Omega}{\Phi}{\aeabsmod{\sigma}{X}{\epsilon}}{\aeallmods{\sigma}{X}{\rho}}
% }
% \end{equation}
% \begin{equation}\label{rule:hastsmtypePat-B-aptype}
% \inferrule{
%   \hastsmtypePat{\Omega}{\Phi}{\epsilon}{\aealltypes{t}{\rho}}\\
%   \haskindX{\tau}{\akty}
% }{
%   \hastsmtypePat{\Omega}{\Phi}{\aeaptype{\tau}{\epsilon}}{[\tau/t]\rho}
% }
% \end{equation}
% \begin{equation}\label{rule:hastsmtypePat-B-apmod}
% \inferrule{
%   \hastsmtypePat{\Omega}{\Phi}{\epsilon}{\aeallmods{\sigma}{X'}{\rho}}\\
%   \hassig{\Omega}{X}{\sigma}
% }{
%   \hastsmtypePat{\Omega}{\Phi}{\aeapmod{X}{\epsilon}}{[X/X']\rho}
% }
% \end{equation}

% \end{subequations}

% peTLM Expression Expansion
% \begin{subequations}\label{rules:tsmexpExpandsExp-B}
% \begin{equation}\label{rule:tsmexpExpandsExp-B-bindref}
% \inferrule{
%   \hastsmtypeExp{\Omega}{\Psi}{\epsilon}{\rho}  
% }{
%   \tsmexpExpandsExp{\uOmegaEx{\uD}{\uG}{\uMctx}{\Omega}}{\uASI{\uA, \mapitem{\tsmv}{\epsilon}}{\Psi}{\uI}}{{\tsmv}}{\epsilon}{\rho}
% }
% \end{equation}
% \begin{equation}\label{rule:tsmexpExpandsExp-B-abstype}
% \inferrule{
%   \tsmexpExpandsExp{\uOmega, \uKhyp{\ut}{t}{\akty}}{\uPsi}{\uepsilon}{\epsilon}{\rho}
% }{
%   \tsmexpExpandsExp{\uOmega}{\uPsi}{\abstype{\ut}{\uepsilon}}{\aeabstype{t}{\epsilon}}{\aealltypes{t}{\rho}}
% }
% \end{equation}
% \begin{equation}\label{rule:tsmexpExpandsExp-B-absmod}
% \inferrule{
%   \sigExpandsPX{\usigma}{\sigma}\\
%   \tsmexpExpandsExp{\uOmega, \uMhyp{\uX}{X}{\sigma}}{\uPsi}{\uepsilon}{\epsilon}{\rho}
% }{
%   \tsmexpExpandsExp{\uOmega}{\uPsi}{\absmod{\uX}{\usigma}{\uepsilon}}{\aeabsmod{\sigma}{X}{\epsilon}}{\aeallmods{\sigma}{X}{\rho}}
% }
% \end{equation}
% \begin{equation}\label{rule:tsmexpExpandsExp-B-aptype}
% \inferrule{
%   \tsmexpExpandsExp{\uOmega}{\uPsi}{\uepsilon}{\epsilon}{\aealltypes{t}{\rho}}\\
%   \cExpandsX{\utau}{\tau}{\akty}
% }{
%   \tsmexpExpandsExp{\uOmega}{\uPsi}{\aptype{\uepsilon}{\utau}}{\aeaptype{\tau}{\epsilon}}{[\tau/t]\rho} 
% }
% \end{equation}
% \begin{equation}\label{rule:tsmexpExpandsExp-B-apmod}
% \inferrule{
%   \tsmexpExpandsExp{\uOmega}{\uPsi}{\uepsilon}{\epsilon}{\aeallmods{\sigma}{X'}{\rho}}\\
%   \manaX{\uX}{X}{\sigma}
% }{
%   \tsmexpExpandsExp{\uOmega}{\uPsi}{\apmod{\uepsilon}{\uX}}{\aeapmod{X}{\epsilon}}{[X/X']\rho}
% }
% \end{equation}
% \end{subequations}

% ppTLM expression expansion
% \begin{subequations}\label{rules:tsmexpExpandsPat-B}
% \begin{equation}\label{rule:tsmexpExpandsPat-B-bindref}
% \inferrule{
%   \hastsmtypePat{\Omega}{\Phi}{\epsilon}{\rho}  
% }{
%   \tsmexpExpandsPat{\uOmegaEx{\uD}{\uG}{\uMctx}{\Omega}}{\uASI{\uA, \mapitem{\tsmv}{\epsilon}}{\Phi}{\uI}}{{\tsmv}}{\epsilon}{\rho}
% }
% \end{equation}
% \begin{equation}\label{rule:tsmexpExpandsPat-B-abstype}
% \inferrule{
%   \tsmexpExpandsPat{\uOmega, \uKhyp{\ut}{t}{\akty}}{\uPhi}{\uepsilon}{\epsilon}{\rho}
% }{
%   \tsmexpExpandsPat{\uOmega}{\uPhi}{\abstype{\ut}{\uepsilon}}{\aeabstype{t}{\epsilon}}{\aealltypes{t}{\rho}}
% }
% \end{equation}
% \begin{equation}\label{rule:tsmexpExpandsPat-B-absmod}
% \inferrule{
%   \sigExpandsPX{\usigma}{\sigma}\\
%   \tsmexpExpandsPat{\uOmega, \uMhyp{\uX}{X}{\sigma}}{\uPhi}{\uepsilon}{\epsilon}{\rho}
% }{
%   \tsmexpExpandsPat{\uOmega}{\uPhi}{\absmod{\uX}{\usigma}{\uepsilon}}{\aeabsmod{\sigma}{X}{\epsilon}}{\aeallmods{\sigma}{X}{\rho}}
% }
% \end{equation}
% \begin{equation}\label{rule:tsmexpExpandsPat-B-aptype}
% \inferrule{
%   \tsmexpExpandsPat{\uOmega}{\uPhi}{\uepsilon}{\epsilon}{\aealltypes{t}{\rho}}\\
%   \cExpandsX{\utau}{\tau}{\akty}
% }{
%   \tsmexpExpandsPat{\uOmega}{\uPhi}{\aptype{\uepsilon}{\utau}}{\aeaptype{\tau}{\epsilon}}{[\tau/t]\rho} 
% }
% \end{equation}
% \begin{equation}\label{rule:tsmexpExpandsPat-B-apmod}
% \inferrule{
%   \tsmexpExpandsPat{\uOmega}{\uPhi}{\uepsilon}{\epsilon}{\aeallmods{\sigma}{X'}{\rho}}\\
%   \manaX{\uX}{X}{\sigma}
% }{
%   \tsmexpExpandsPat{\uOmega}{\uPhi}{\apmod{\uepsilon}{\uX}}{\aeapmod{X}{\epsilon}}{[X/X']\rho}
% }
% \end{equation}
% \end{subequations}

% peTLM expression normal forms
% \begin{subequations}\label{rules:tsmexpNormalExp-B}
% \begin{equation}\label{rule:tsmexpNormalExp-B-defref}
% \inferrule{ }{
%   \tsmexpNormalExp{\Omega}{\Psi, \petsmdefn{a}{\rho}{\eparse}}{\adefref{a}}
% }
% \end{equation}
% \begin{equation}\label{rule:tsmexpNormalExp-B-abstype}
% \inferrule{ }{
%   \tsmexpNormalExp{\Omega}{\Psi}{\aeabstype{t}{\epsilon}}
% }
% \end{equation}
% \begin{equation}\label{rule:tsmexpNormalExp-B-absmod}
% \inferrule{ }{
%   \tsmexpNormalExp{\Omega}{\Psi}{\aeabsmod{\sigma}{X}{\epsilon}}
% }
% \end{equation}
% \begin{equation}\label{rule:tsmexpNormalExp-B-aptype}
% \inferrule{
%   \epsilon \neq \aeabstype{t}{\epsilon'}\\
%   \tsmexpNormalExp{\Omega}{\Psi}{\epsilon}
% }{
%   \tsmexpNormalExp{\Omega}{\Psi}{\aeaptype{\tau}{\epsilon}}
% }
% \end{equation}
% \begin{equation}\label{rule:tsmexpNormalExp-B-apmod}
% \inferrule{
%   \epsilon \neq \aeabsmod{\sigma}{X'}{\epsilon'}\\
%   \tsmexpNormalExp{\Omega}{\Psi}{\epsilon}
% }{
%   \tsmexpNormalExp{\Omega}{\Psi}{\aeapmod{X}{\epsilon}}
% }
% \end{equation}
% \end{subequations}

% peTLM transitions
% \begin{subequations}\label{rules:tsmexpStepsExp-B}
% \begin{equation}\label{rule:tsmexpStepsExp-B-aptype-1}
% \inferrule{
%   \tsmexpStepsExp{\Omega}{\Psi}{\epsilon}{\epsilon'}
% }{
%   \tsmexpStepsExp{\Omega}{\Psi}{\aeaptype{\tau}{\epsilon}}{\aeaptype{\tau}{\epsilon'}}
% }
% \end{equation}
% \begin{equation}\label{rule:tsmexpStepsExp-B-aptype-2}
% \inferrule{ }{
%   \tsmexpStepsExp{\Omega}{\Psi}{\aeaptype{\tau}{\aeabstype{t}{\epsilon}}}{[\tau/t]\epsilon}
% }
% \end{equation}
% \begin{equation}\label{rule:tsmexpStepsExp-B-apmod-1}
% \inferrule{
%   \tsmexpStepsExp{\Omega}{\Psi}{\epsilon}{\epsilon'}
% }{
%   \tsmexpStepsExp{\Omega}{\Psi}{\aeapmod{X}{\epsilon}}{\aeapmod{X}{\epsilon'}}
% }
% \end{equation}
% \begin{equation}\label{rule:tsmexpStepsExp-B-apmod-2}
% \inferrule{ }{
%   \tsmexpStepsExp{\Omega}{\Psi}{\aeapmod{X}{\aeabsmod{\sigma}{X'}{\epsilon}}}{[X/X']\epsilon}
% }
% \end{equation}
% \end{subequations}

% peTLM reflexive, transitive transitions
% \begin{subequations}\label{rules:tsmexpMultistepsExp-B}
% \begin{equation}\label{rule:tsmexpMultistepsExp-B-refl}
% \inferrule{ }{
%   \tsmexpMultistepsExp{\Omega}{\Psi}{\epsilon}{\epsilon}
% }
% \end{equation}
% \begin{equation}\label{rule:tsmexpMultistepsExp-B-steps}
% \inferrule{
%   \tsmexpStepsExp{\Omega}{\Psi}{\epsilon}{\epsilon'}
% }{
%   \tsmexpMultistepsExp{\Omega}{\Psi}{\epsilon}{\epsilon'}
% }
% \end{equation}
% \begin{equation}\label{rule:tsmexpMultistepsExp-B-trans}
% \inferrule{
%   \tsmexpMultistepsExp{\Omega}{\Psi}{\epsilon}{\epsilon'}\\
%   \tsmexpMultistepsExp{\Omega}{\Psi}{\epsilon'}{\epsilon''}
% }{
%   \tsmexpMultistepsExp{\Omega}{\Psi}{\epsilon}{\epsilon''}
% }
% \end{equation}
% \end{subequations}

% peTLM normalization
% \begin{equation}\label{rule:tsmexpEvalsExp-B}
% \inferrule{
%   \tsmexpMultistepsExp{\Omega}{\Psi}{\epsilon}{\epsilon'}\\
%   \tsmexpNormalExp{\Omega}{\Psi}{\epsilon'}
% }{
%   \tsmexpEvalsExp{\Omega}{\Psi}{\epsilon}{\epsilon'}
% }
% \end{equation}

% TLM expression definition extraction

% \begin{subequations}
% \begin{align}
% \tsmdefof{\adefref{a}} & = a\\
% \tsmdefof{\aeabstype{t}{\epsilon}} & = \tsmdefof{\epsilon}\\
% \tsmdefof{\aeabsmod{\sigma}{X}{\epsilon}} & = \tsmdefof{\epsilon}\\
% \tsmdefof{\aeaptype{\tau}{\epsilon}} & = \tsmdefof{\epsilon}\\
% \tsmdefof{\aeapmod{X}{\epsilon}} & = \tsmdefof{\epsilon}
% \end{align}
% \end{subequations}

% \subsubsection{Candidate Expansion Kind and Constructor Validation}
% %The \emph{ce-type validation judgement}, $\cvalidT{\Delta}{\tscenev}{\ctau}{\tau}$, is inductively defined by Rules (\ref{rules:cvalidT-U}), which were defined in Sec. \ref{sec:ce-validation-U}.

% ce-kind validation
% \begin{subequations}\label{rules:cvalidK-B}
% \begin{equation}\label{rule:cvalidK-B-darr}
% \inferrule{
%   \cvalidKX{\cekappa_1}{\kappa_1}\\
%   \cvalidK{\Omega, u :: \kappa_1}{\cscenev}{\cekappa_2}{\kappa_2}
% }{
%   \cvalidKX{\acekdarr{\cekappa_1}{u}{\cekappa_2}}{\akdarr{\kappa_1}{u}{\kappa_2}}
% }
% \end{equation}
% \begin{equation}\label{rule:cvalidK-B-unit}
% \inferrule{ }{
%   \cvalidKX{\acekunit}{\akunit}
% }
% \end{equation}
% \begin{equation}\label{rule:cvalidK-B-dprod}
% \inferrule{
%   \cvalidKX{\cekappa_1}{\kappa_1}\\
%   \cvalidK{\Omega, u :: \kappa_1}{\cscenev}{\cekappa_2}{\kappa_2}
% }{
%   \cvalidKX{\acekdbprod{\cekappa_1}{u}{\cekappa_2}}{\akdbprod{\kappa_1}{u}{\kappa_2}}
% }
% \end{equation}
% \begin{equation}\label{rule:cvalidK-B-ty}
% \inferrule{ }{
%   \cvalidKX{\acekty}{\akty}
% }
% \end{equation}
% \begin{equation}\label{rule:cvalidK-B-sing}
% \inferrule{
%   \ccanaX{\ctau}{\tau}{\akty}
% }{
%   \cvalidKX{\aceksing{\ctau}}{\aksing{\tau}}
% }
% \end{equation}
% \begin{equation}\label{rule:cvalidK-B-spliced}
% \inferrule{
%   \parseUKind{\bsubseq{b}{m}{n}}{\ukappa}\\
%   \kExpands{\uOmega}{\ukappa}{\kappa}\\\\
%   \uOmega=\uOmegaEx{\uD}{\uG}{\uMctx}{\Omega_\text{app}}\\
%   \domof{\Omega} \cap \domof{\Omega_\text{app}} = \emptyset
% }{
%   \cvalidK{\Omega}{\tsceneP{\uOmega}{b}}{\acesplicedk{m}{n}}{\kappa}
% }
% \end{equation}
% \end{subequations}

% Synthetic ce-constructor validation
% \begin{subequations}\label{rules:ccsyn}
% \begin{equation}\label{rule:ccsyn-var}
% \inferrule{ }{\ccsyn{\Omega, {u} :: {\kappa}}{\cscenev}{u}{u}{\kappa}}
% \end{equation}
% \begin{equation}\label{rule:ccsyn-asc}
% \inferrule{
%   \cvalidKX{\cekappa}{\kappa}\\
%   \ccanaX{\cec}{c}{\kappa}
% }{
%   \ccsynX{\acecasc{\cekappa}{\cec}}{c}{\kappa}
% }
% \end{equation}
% \begin{equation}\label{rule:ccsyn-app}
% \inferrule{
%   \ccsynX{\cec_1}{c_1}{\akdarr{\kappa_2}{u}{\kappa}}\\
%   \ccsynX{\cec_2}{c_2}{\kappa_2}
% }{
%   \ccsynX{\acecapp{\cec_1}{\cec_2}}{\acapp{c_1}{c_2}}{[c_1/u]\kappa}
% }
% \end{equation}
% \begin{equation}\label{rule:ccsyn-unit}
% \inferrule{ }{
%   \ccsynX{\acectriv}{\actriv}{\akunit}
% }
% \end{equation}
% \begin{equation}\label{rule:ccsyn-prl}
% \inferrule{
%   \ccsynX{\cec}{c}{\akdbprod{\kappa_1}{u}{\kappa_2}}
% }{
%   \ccsynX{\acecprl{\cec}}{\acprl{c}}{\kappa_1}
% }
% \end{equation}
% \begin{equation}\label{rule:ccsyn-prr}
% \inferrule{
%   \ccsynX{\cec}{c}{\akdbprod{\kappa_1}{u}{\kappa_2}}
% }{
%   \ccsynX{\acecprr{\cec}}{\acprr{c}}{[\acprl{c}/u]\kappa_2}
% }
% \end{equation}
% \begin{equation}\label{rule:ccsyn-parr}
% \inferrule{
%   \ccanaX{\ctau_1}{\tau_1}{\akty}\\
%   \ccanaX{\ctau_2}{\tau_2}{\akty}
% }{
%   \ccsynX{\aceparr{\ctau_1}{\ctau_2}}{\aparr{\tau_1}{\tau_2}}{\akty}
% }
% \end{equation}
% \begin{equation}\label{rule:ccsyn-all}
% \inferrule{
%   \cvalidKX{\cekappa}{\kappa}\\
%   \ccana{\Omega, u :: \kappa}{\cscenev}{\ctau}{\tau}{\akty}
% }{
%   \ccsynX{\aceallu{\cekappa}{u}{\ctau}}{\aallu{\kappa}{u}{\tau}}{\akty}
% }
% \end{equation}
% \begin{equation}\label{rule:ccsyn-rec}
% \inferrule{
%   \ccana{\Omega, t :: \akty}{\cscenev}{\ctau}{\tau}{\akty}
% }{
%   \ccsynX{\acerec{t}{\ctau}}{\arec{t}{\tau}}{\akty}
% }
% \end{equation}
% \begin{equation}\label{rule:ccsyn-prod}
% \inferrule{
%   \{\ccanaX{\ctau_i}{\tau_i}{\akty}\}_{1 \leq i \leq n}
% }{
%   \ccsynX{\aceprod{\labelset}{\mapschema{\ctau}{i}{\labelset}}}{\aprod{\labelset}{\mapschema{\tau}{i}{\labelset}}}{\akty}
% }
% \end{equation}
% \begin{equation}\label{rule:ccsyn-sum}
% \inferrule{
%   \{\ccanaX{\ctau_i}{\tau_i}{\akty}\}_{1 \leq i \leq n}
% }{
%   \ccsynX{\acesum{\labelset}{\mapschema{\ctau}{i}{\labelset}}}{\asum{\labelset}{\mapschema{\tau}{i}{\labelset}}}{\akty}
% }
% \end{equation}
% \begin{equation}\label{rule:ccsyn-stat}
% \inferrule{ }{
%   \ccsyn{\Omega, X : {\asignature{\kappa}{u}{\tau}}}{\cscenev}{\acemcon{X}}{\amcon{X}}{\kappa}
% }
% \end{equation}
% \begin{equation}\label{rule:ccsyn-spliced}
% \inferrule{
%   \parseUCon{\bsubseq{b}{m}{n}}{\uc}\\
%   \ksyn{\uOmega}{\uc}{c}{\kappa}\\\\
%   \uOmega=\uOmegaEx{\uD}{\uG}{\uMctx}{\Omega_\text{app}}\\
%   \domof{\Omega} \cap \domof{\Omega_\text{app}} = \emptyset
% }{
%   \ccsyn{\Omega}{\tsceneP{\uOmega}{b}}{\acesplicedc{m}{n}{\cekappa}}{c}{\kappa}
% }
% \end{equation}
% \end{subequations}

% Analytic constructor expansion
% \begin{subequations}\label{rules:ccana}
% \begin{equation}\label{rule:ccana-subsume}
% \inferrule{
%   \ccsynX{\cec}{c}{\kappa_1}\\
%   \ksubX{\kappa_1}{\kappa_2}
% }{
%   \ccanaX{\cec}{c}{\kappa_2}
% }
% \end{equation}
% \begin{equation}\label{rule:ccana-sing}
% \inferrule{
%   \kanaX{\cec}{c}{\akty}
% }{
%   \kanaX{\cec}{c}{\aksing{c}}
% }
% \end{equation}
% \begin{equation}\label{rule:ccana-abs}
% \inferrule{
%   \ccana{\Omega, u :: \kappa_1}{\cscenev}{\cec_2}{c_2}{\kappa_2}
% }{
%   \ccanaX{\acecabs{u}{\cec_2}}{\acabs{u}{c_2}}{\akdarr{\kappa_1}{u}{\kappa_2}}
% }
% \end{equation}
% \begin{equation}\label{rule:ccana-pair}
% \inferrule{
%   \ccanaX{\cec_1}{c_1}{\kappa_1}\\
%   \ccanaX{\cec_2}{c_2}{[c_1/u]\kappa_2}
% }{
%   \ccanaX{\acecpair{\cec_1}{\cec_2}}{\acpair{c_1}{c_2}}{\akdbprod{\kappa_1}{u}{\kappa_2}}
% }
% \end{equation}
% \begin{equation}\label{rule:ccana-spliced}
% \inferrule{
%   \parseUCon{\bsubseq{b}{m}{n}}{\uc}\\
%   \kana{\uOmega}{\uc}{c}{\kappa}\\\\
%   \uOmega=\uOmegaEx{\uD}{\uG}{\uMctx}{\Omega_\text{app}}\\
%   \domof{\Omega} \cap \domof{\Omega_\text{app}} = \emptyset
% }{
%   \ccana{\Omega}{\tsceneP{\uOmega}{b}}{\acesplicedc{m}{n}{\cekappa}}{c}{\kappa}
% }
% \end{equation}
% \end{subequations}

% \subsubsection{Bidirectional Candidate Expansion Expression Validation}
% Like the bidirectionally typed expression expansion judgements, the bidirectional ce-expression validation judgements distinguish type synthesis from type analysis. The \emph{synthetic ce-expression validation judgement}, $\csynX{\ce}{e}{\tau}$, and the \emph{analytic ce-expression validation judgement}, $\canaX{\ce}{e}{\tau}$, are defined mutually inductively with Rules (\ref{rules:esyn}) and Rules (\ref{rules:eana}) by Rules (\ref{rules:csyn}) and Rules (\ref{rules:cana}), respectively, as follows.


% \begin{equation}
% \inferrule{
%   \ccanaX{\ctau}{\tau}{\akty}
% }{
%   \cvalidTP{\Omega}{\cscenev}{\ctau}{\tau}
% }
% \end{equation}

% \paragraph{Type Synthesis} \begin{subequations}\label{rules:csynP}
% Synthetic ce-expression validation is governed by the following rules.
% \begin{equation}\label{rule:csynP-var}
%   \inferrule{ }{ 
%     \csynP{\Omega, \Ghyp{x}{\tau}}{\escenev}{x}{x}{\tau}
%   }
% \end{equation}
% \begin{equation}\label{rule:csynP-asc}
%   \inferrule{
%     \cvalidTP{\Omega}{\csfrom{\escenev}}{\ctau}{\tau}\\
%     \canaPX{\ce}{e}{\tau}
%   }{
%     \csynPX{\aceasc{\ctau}{\ce}}{e}{\tau}
%   }
% \end{equation}
% \begin{equation}\label{rule:csynP-let}
%   \inferrule{
%     \csynPX{\ce}{e}{\tau}\\
%     \csynP{\Omega, \Ghyp{x}{\tau}}{\escenev}{\ce'}{e'}{\tau'}
%   }{
%     \csynPX{\aceletsyn{x}{\ce}{\ce'}}{\aeap{\aelam{\tau}{x}{e'}}{e}}{\tau'}
%   }
% \end{equation}
% \begin{equation}\label{rule:csynP-lam}
%   \inferrule{
%     \cvalidTP{\Omega}{\csfrom{\escenev}}{\ctau_1}{\tau_1}\\
%     \csynP{\Omega, \Ghyp{x}{\tau_1}}{\escenev}{\ce}{e}{\tau_2}
%   }{
%     \csynPX{\acelam{\ctau_1}{x}{\ce}}{\aelam{\tau_1}{x}{e}}{\aparr{\tau_1}{\tau_2}}
%   }
% \end{equation}
% \begin{equation}\label{rule:csynP-ap}
%   \inferrule{
%     \csynPX{\ce_1}{e_1}{\aparr{\tau_2}{\tau}}\\
%     \canaPX{\ce_2}{e_2}{\tau_2}
%   }{
%     \csynPX{\aceap{\ce_1}{\ce_2}}{\aeap{e_1}{e_2}}{\tau}
%   }
% \end{equation}
% \begin{equation}\label{rule:csynP-clam}
%   \inferrule{
%     \cvalidK{\Omega}{\csfrom{\escenev}}{\cekappa}{\kappa}\\
%     \csynP{\Omega, u :: \kappa}{\escenev}{\ce}{e}{\tau}
%   }{
%     \csynX{\aceclam{\cekappa}{u}{\ce}}{\aeclam{\kappa}{u}{e}}{\aallu{\kappa}{u}{\tau}}
%   }
% \end{equation}
% \begin{equation}\label{rule:csynP-cap}
%   \inferrule{
%     \csynPX{\ce}{e}{\aallu{\kappa}{u}{\tau}}\\
%     \ccana{\Omega}{\csfrom{\escenev}}{\cec}{c}{\kappa}
%   }{
%     \csynPX{\acecap{\ce}{\cec}}{\aecap{e}{c}}{[c/u]\tau}
%   }
% \end{equation}
% \begin{equation}\label{rule:csynP-unfold}
%   \inferrule{
%     \csynPX{\ce}{e}{\arec{t}{\tau}}
%   }{
%     \csynPX{\aceunfold{\ce}}{\aeunfold{e}}{[\arec{t}{\tau}/t]\tau}
%   }
% \end{equation}
% \begin{equation}\label{rule:csynP-tpl}
%   \inferrule{
%     \ce=\acetpl{\labelset}{\mapschema{\ce}{i}{\labelset}}\\
%     e=\aetpl{\labelset}{\mapschema{e}{i}{\labelset}}\\\\
%     \{\csynPX{\ce_i}{e_i}{\tau_i}\}_{i \in \labelset}
%   }{
%     \csynPX{\ce}{e}{\aprod{\labelset}{\mapschema{\tau}{i}{\labelset}}}
%   }
% \end{equation}
% \begin{equation}\label{rule:csynP-pr}
%   \inferrule{
%     \csynPX{\ce}{e}{\aprod{\labelset, \ell}{\mapschema{\tau}{i}{\labelset}; \mapitem{\ell}{\tau}}}
%   }{
%     \csynPX{\acepr{\ell}{\ce}}{\aepr{\ell}{e}}{\tau}
%   }
% \end{equation}
% \begin{equation}\label{rule:csynP-match}
%   \inferrule{
%     n > 0\\
%     \csynPX{\ce}{e}{\tau}\\
%     \{\crsynPX{\crv_i}{r_i}{\tau}{\tau'}\}_{1 \leq i \leq n}
%   }{
%     \csynPX{\acematchwithb{n}{\ce}{\seqschemaX{\crv}}}{\aematchwith{n}{e}{\seqschemaX{r}}}{\tau'}
%   }
% \end{equation}
% \begin{equation}\label{rule:csynP-mval}
% \inferrule{ }{
%   \csynP{\Omega, X : \asignature{\kappa}{u}{\tau}}{\escenev}{\acemval{X}}{\amval{X}}{[\amcon{X}/u]\tau}
% }
% \end{equation}
% \begin{equation}\label{rule:csynP-splicede}
% \inferrule{
%   \parseUExp{\bsubseq{b}{m}{n}}{\ue}\\
%   \esynP{\uOmega}{\uPsi}{\uPhi}{\ue}{e}{\tau}\\\\
%   \uOmega=\uOmegaEx{\uD}{\uG}{\uMctx}{\Omega_\text{app}}\\
%   \domof{\Omega} \cap \domof{\Omega_\text{app}} = \emptyset
% }{
%   \csynP{\Omega}{\esceneP{\OParams}{\uOmega}{\uPsi}{\uPhi}{b}}{\acesplicede{m}{n}{\ctau}}{e}{\tau}
% }
% \end{equation}
% \end{subequations}

% Rules (\ref{rule:csyn-var}) through (\ref{rule:csyn-match}) are analagous to Rules (\ref{rule:esyn-var}) through (\ref{rule:esyn-match}). Rule (\ref{rule:csyn-splicede}) governs references to spliced unexpanded expressions in synthetic position, and can be understood as described in Sec. \ref{sec:ce-validation-U}.


% \paragraph{Type Analysis} \begin{subequations}\label{rules:canaP}
% Analytic ce-expression validation is governed by the following rules.
% \begin{equation}\label{rule:canaP-subsume}
%   \inferrule{
%     \csynPX{\ce}{e}{\tau}\\
%     \issubtypePX{\tau}{\tau'}
%   }{
%     \canaPX{\ce}{e}{\tau'}
%   }
% \end{equation}
% \begin{equation}\label{rule:canaP-let}
%   \inferrule{
%     \csynPX{\ce}{e}{\tau}\\
%     \canaP{\Omega, \Ghyp{x}{\tau}}{\escenev}{\ce'}{e'}{\tau'}
%   }{
%     \canaPX{\aceletsyn{x}{\ce}{\ce'}}{\aeap{\aelam{\tau}{x}{e'}}{e}}{\tau'}
%   }
% \end{equation}
% \begin{equation}\label{rule:canaP-analam}
%   \inferrule{
%     \canaP{\Gamma, \Ghyp{x}{\tau_1}}{\escenev}{\ce}{e}{\tau_2}
%   }{
%     \canaPX{\aceanalam{x}{\ue}}{\aelam{\tau_1}{x}{e}}{\aparr{\tau_1}{\tau_2}}
%   }
% \end{equation}
% \begin{equation}\label{rule:canaP-clam}
%   \inferrule{
%     \cvalidKX{\cekappa}{\kappa}\\
%     \canaP{\Omega, u :: \kappa}{\escenev}{\ce}{e}{\tau}
%   }{
%     \canaPX{\aceclam{\cekappa}{u}{\ce}}{\aeclam{\kappa}{u}{e}}{\aallu{\kappa}{u}{\tau}}
%   }
% \end{equation}
% \begin{equation}\label{rule:canaP-fold}
%   \inferrule{
%     \canaPX{\ce}{e}{[\arec{t}{\tau}/t]\tau}
%   }{
%     \canaPX{\aceanafold{\ce}}{\aefold{e}}{\arec{t}{\tau}}
%   }
% \end{equation}
% \begin{equation}\label{rule:canaP-tpl}
%   \inferrule{
%     \ce=\acetpl{\labelset}{\mapschema{\ce}{i}{\labelset}}\\
%     e=\aetpl{\labelset}{\mapschema{e}{i}{\labelset}}\\\\
%     \{\canaPX{\ce_i}{e_i}{\tau_i}\}_{i \in \labelset}
%   }{
%     \canaPX{\ce}{e}{\aprod{\labelset}{\mapschema{\tau}{i}{\labelset}}}
%   }
% \end{equation}
% \begin{equation}\label{rule:canaP-in}
%   \inferrule{
%     \ce=\aceanain{\ell}{\ce'}\\
%     e=\aein{\labelset, \ell}{\ell}{\mapschema{\tau}{i}{\labelset}; \mapitem{\ell}{\tau}}{e'}\\\\
%     \canaX{\ce'}{e'}{\tau}
%   }{
%     \canaPX{\ce}{e}{\asum{\labelset, \ell}{\mapschema{\tau}{i}{\labelset}; \mapitem{\ell}{\tau}}}
%     %\left(\shortstack{$\Delta~\Gamma \vdash^{\escenev} $\\$\leadsto$\\$ \Leftarrow $\vspace{-1.2em}}\right)
%     %\eanaX{\auanain{\ell}{\ue}}{\aein{\ell}}{\asum{\labelset, \ell}{\mapschema{\tau}{i}{\labelset}; \mapitem{\ell}{\tau}}}
%   }
% \end{equation}
% \begin{equation}\label{rule:canaP-match}
%   \inferrule{
%     \csynPX{\ce}{e}{\tau}\\
%     \{\cranaPX{\crv_i}{r_i}{\tau}{\tau'}\}_{1 \leq i \leq n}
%   }{
%     \canaPX{\acematchwithb{n}{\ce}{\seqschemaX{\crv}}}{\aematchwith{n}{e}{\seqschemaX{r}}}{\tau'}
%   }
% \end{equation}
% \begin{equation}\label{rule:canaP-splicede}
% \inferrule{
%   \parseUExp{\bsubseq{b}{m}{n}}{\ue}\\
%   \eanaP{\uOmega}{\uPsi}{\uPhi}{\ue}{e}{\tau}\\\\
%   \uOmega=\uOmegaEx{\uD}{\uG}{\uMctx}{\Omega_\text{app}}\\
%   \domof{\Omega} \cap \domof{\Omega_\text{app}} = \emptyset
% }{
%   \canaP{\Omega}{\esceneP{\OParams}{\uOmega}{\uPsi}{\uPhi}{b}}{\acesplicede{m}{n}{\ctau}}{e}{\tau}
% }
% \end{equation}
% \end{subequations}

% Rules (\ref{rule:cana-subsume}) through (\ref{rule:cana-match}) are analagous to Rules (\ref{rule:eana-subsume}) through (\ref{rule:eana-match}). Rule (\ref{rule:cana-splicede}) governs references to spliced unexpanded expressions in analytic position. 

% \subsubsection{Bidirectional Candidate Expansion Rule Validation}
% The \emph{synthetic ce-rule validation judgement} is defined mutually inductively with Rules (\ref{rules:esyn}) by the following rule.
% \begin{equation}\label{rule:crsynP}
% \inferrule{
%   \patTypeP{\Omega'}{p}{\tau}\\
%   \csynP{\Gcons{\Omega}{\Omega'}}{\escenev}{\ce}{e}{\tau'}
% }{
%   \crsynPX{\acematchrule{p}{\ce}}{\aematchrule{p}{e}}{\tau}{\tau'}
% }
% \end{equation}

% The \emph{analytic ce-rule validation judgement} is defined mutually inductively with Rules (\ref{rules:eana}) by the following rule.
% \begin{equation}\label{rule:cranaP}
% \inferrule{
%   \patType{\Omega'}{p}{\tau}\\
%   \canaP{\Gcons{\Omega}{\Omega'}}{\escenev}{\ce}{e}{\tau'}
% }{
%   \cranaPX{\acematchrule{p}{\ce}}{\aematchrule{p}{e}}{\tau}{\tau'}
% }
% \end{equation}

% \subsubsection{Candidate Expansion Pattern Validation}
% The \emph{ce-pattern validation judgement} is inductively defined by the following rules, which are written identically to Rules (\ref{rules:cvalidP-UP}).
% \begin{subequations}\label{rules:cvalidP-P}
% \begin{equation}\label{rule:cvalidP-P-wild}
% \inferrule{ }{
%   \cvalidPP{\uOmegaEx{\emptyset}{\emptyset}{\emptyset}{\emptyset}}{\pscenev}{\acewildp}{\aewildp}{\tau}
% }
% \end{equation}
% \begin{equation}\label{rule:cvalidP-P-fold}
% \inferrule{
%   \cvalidPP{\uOmega}{\pscenev}{\cpv}{p}{[\arec{t}{\tau}/t]\tau}
% }{
%   \cvalidPP{\uOmega}{\pscenev}{\acefoldp{\cpv}}{\aefoldp{p}}{\arec{t}{\tau}}
% }
% \end{equation}
% \begin{equation}\label{rule:cvalidP-P-tpl}
% \inferrule{
%   \cpv=\acetplp{\labelset}{\mapschema{\cpv}{i}{\labelset}}\\
%   p=\aetplp{\labelset}{\mapschema{p}{i}{\labelset}}\\\\
%   \{\cvalidPP{\upctx_i}{\pscenev}{\cpv_i}{p_i}{\tau_i}\}_{i \in \labelset}
% }{
%   \cvalidPP{\Gconsi{i \in \labelset}{\uOmega_i}}{\pscenev}{\cpv}{p}{\aprod{\labelset}{\mapschema{\tau}{i}{\labelset}}}
%   %\cvalidPP{}{\cpv}{p}{}
% %\left(\shortstack{$\vdash^{\pscenev} $\\$\leadsto$\\$ :~\dashVx^{\,\Gconsi{i \in \labelset}{\upctx_i}}$\vspace{-1.2em}}\right)
% }
% \end{equation}
% \begin{equation}\label{rule:cvalidP-P-in}
% \inferrule{
%   \cvalidPP{\uOmega}{\pscenev}{\cpv}{p}{\tau}
% }{
%   \cvalidPP{\uOmega}{\pscenev}{\aceinjp{\ell}{\cpv}}{\aeinjp{\ell}{p}}{\asum{\labelset, \ell}{\mapschema{\tau}{i}{\labelset}; \mapitem{\ell}{\tau}}}
% }
% \end{equation}
% \begin{equation}\label{rule:cvalidP-P-spliced}
% \inferrule{
%   \parseUPat{\bsubseq{b}{m}{n}}{\upv}\\
%   \patExpandsP{\uOmega'}{\uPhi}{\upv}{p}{\tau}
% }{
%   \cvalidPP{\uOmega'}{\pscene{\uOmega}{\uPhi}{b}}{\acesplicedp{m}{n}{\ctau}}{p}{\tau}
% }
% \end{equation}
% \end{subequations}

% Finally, the following theorem establishes that bidirectionally typed expression and rule expansion produces expanded expressions and rules of the appropriate type under the appropriate contexts. These statements must be stated mutually with the corresponding statements about birectional ce-expression and ce-rule validation because the judgements are mutually defined. 

% \begin{theorem}[Typed Expansion] Letting $\uPsi=\uASI{\uA}{\Psi}{\uI}$, if $\uetsmenv{\Delta}{\Psi}$ then all of the following hold:
% \begin{enumerate}
%   \item \begin{enumerate}
%     \item \begin{enumerate}
%       \item If $\esyn{\uDD{\uD}{\Delta}}{\uGG{\uG}{\Gamma}}{\uPsi}{\uPhi}{\ue}{e}{\tau}$ then $\hastypeU{\Delta}{\Gamma}{e}{\tau}$.
%       \item If $\rsyn{\uDD{\uD}{\Delta}}{\uGG{\uG}{\Gamma}}{\uPsi}{\uPhi}{\urv}{r}{\tau}{\tau'}$  then $\ruleType{\Delta}{\Gamma}{r}{\tau}{\tau'}$.
%     \end{enumerate}
%     \item \begin{enumerate}
%       \item If $\eana{\uDD{\uD}{\Delta}}{\uGG{\uG}{\Gamma}}{\uPsi}{\uPhi}{\ue}{e}{\tau}$ and $\istypeU{\Delta}{\tau}$ then $\hastypeU{\Delta}{\Gamma}{e}{\tau}$.
%       \item If $\rana{\uDD{\uD}{\Delta}}{\uGG{\uG}{\Gamma}}{\uPsi}{\uPhi}{\urv}{r}{\tau}{\tau'}$ and $\istypeU{\Delta}{\tau'}$ then $\ruleType{\Delta}{\Gamma}{r}{\tau}{\tau'}$.
%     \end{enumerate}
%   \end{enumerate}
%   \item \begin{enumerate}
%     \item \begin{enumerate}
%       \item If $\csyn{\Delta}{\Gamma}{\esceneUP{\uDD{\uD}{\Delta_\text{app}}}{\uGG{\uG}{\Gamma_\text{app}}}{\uPsi}{\uPhi}{b}}{\ce}{e}{\tau}$ and $\Delta \cap \Delta_\text{app}=\emptyset$ and $\domof{\Gamma} \cap \domof{\Gamma_\text{app}}=\emptyset$ then $\hastypeU{\Dcons{\Delta}{\Delta_\text{app}}}{\Gcons{\Gamma}{\Gamma_\text{app}}}{e}{\tau}$. 
%       \item If $\crsyn{\Delta}{\Gamma}{\esceneUP{\uDD{\uD}{\Delta_\text{app}}}{\uGG{\uG}{\Gamma_\text{app}}}{\uPsi}{\uPhi}{b}}{\crv}{r}{\tau}{\tau'}$ and $\Delta \cap \Delta_\text{app}=\emptyset$ and $\domof{\Gamma} \cap \domof{\Gamma_\text{app}}=\emptyset$ then $\ruleType{\Dcons{\Delta}{\Delta_\text{app}}}{\Gcons{\Gamma}{\Gamma_\text{app}}}{r}{\tau}{\tau'}$.
%     \end{enumerate}
%     \item \begin{enumerate}
%       \item If $\cana{\Delta}{\Gamma}{\esceneUP{\uDD{\uD}{\Delta_\text{app}}}{\uGG{\uG}{\Gamma_\text{app}}}{\uPsi}{\uPhi}{b}}{\ce}{e}{\tau}$ and $\Delta \cap \Delta_\text{app}=\emptyset$ and $\domof{\Gamma} \cap \domof{\Gamma_\text{app}}=\emptyset$ and $\istypeU{\Dcons{\Delta}{\Delta_\text{app}}}{\tau}$ then $\hastypeU{\Dcons{\Delta}{\Delta_\text{app}}}{\Gcons{\Gamma}{\Gamma_\text{app}}}{e}{\tau}$. 
%       \item If $\crana{\Delta}{\Gamma}{\esceneUP{\uDD{\uD}{\Delta_\text{app}}}{\uGG{\uG}{\Gamma_\text{app}}}{\uPsi}{\uPhi}{b}}{\crv}{r}{\tau}{\tau'}$ and $\Delta \cap \Delta_\text{app}=\emptyset$ and $\domof{\Gamma} \cap \domof{\Gamma_\text{app}}=\emptyset$ and $\istypeU{\Dcons{\Delta}{\Delta_\text{app}}}{\tau'}$ then $\ruleType{\Dcons{\Delta}{\Delta_\text{app}}}{\Gcons{\Gamma}{\Gamma_\text{app}}}{r}{\tau}{\tau'}$.
%     \end{enumerate}
%   \end{enumerate}
% \end{enumerate}
% \end{theorem}
% \begin{proof} By mutual rule induction over Rules (\ref{rules:esyn}), Rules (\ref{rules:eana}), Rule (\ref{rule:rsyn}), Rule (\ref{rule:rana}), Rules (\ref{rules:csyn}), Rules (\ref{rules:cana}), Rule (\ref{rule:crsyn}) and Rule (\ref{rule:crana}). In the following, we refer to the induction hypothesis applied to the assumption $\uetsmenv{\Delta}{\Psi}$ as simply the ``IH''. When we apply the induction hypothesis to a different argument, we refer to it as the ``Outer IH''.

% \begin{enumerate}
%   \item In the following, let $\uDelta=\uDD{\uD}{\Delta}$ and $\uGamma=\uGG{\uG}{\Gamma}$. We have:
%   \begin{enumerate}
%     \item \begin{enumerate}
%       \item We induct on the assumption.
%         \begin{byCases}
%           \item[\text{(\ref{rule:esyn-var})}] We have:
%             \begin{pfsteps*}
%               \item $e=x$ \BY{assumption}
%               \item $\Gamma=\Gamma', \Ghyp{x}{\tau}$ \BY{assumption}
%               \item $\hastypeU{\Delta}{\Gamma', \Ghyp{x}{\tau}}{x}{\tau}$ \BY{Rule (\ref{rule:hastypeUP-var})}
%             \end{pfsteps*}
%             \resetpfcounter
%           \item[\text{(\ref{rule:esyn-asc})}] We have:
%             \begin{pfsteps*}
%                \item $\ue=\auasc{\utau}{\ue'}$ \BY{assumption}
%                \item $\expandsTU{\uDelta}{\utau}{\tau}$ \BY{assumption}\pflabel{expandsTU}
%                \item $\eanaX{\ue'}{e}{\tau}$ \BY{assumption}\pflabel{eanaX}
%                \item $\istypeU{\Delta}{\tau}$ \BY{Lemma \ref{lemma:type-expansion-U} on \pfref{expandsTU}}\pflabel{istype}
%                \item $\hastypeU{\Delta}{\Gamma}{e}{\tau}$ \BY{IH, part 1(b)(i) to \pfref{eanaX} and \pfref{istype}}
%              \end{pfsteps*}
%              \resetpfcounter
%           \item[\text{(\ref{rule:esyn-let}) through (\ref{rule:esyn-match})}] In each of these cases, we apply:
%             \begin{itemize}
%               \item Lemma \ref{lemma:type-expansion-U} to or over all type expansion premises.
%               \item The IH, part 1(a)(i) to or over all synthetic typed expression expansion premises.
%               \item The IH, part 1(a)(ii) to or over all synthetic rule expansion premises.
%               \item The IH, part 1(b)(i) to or over all analytic typed expression expansion premises.
%             \end{itemize}
%             We then derive the conclusion by applying Rules (\ref{rules:hastypeUP}) and Rule (\ref{rule:ruleType}) as needed.
%           \item[\text{(\ref{rule:esyn-defuetsm})}] We have:
%             \begin{pfsteps*}
%               \item $\ue=\audefuetsm{\utau'}{\eparse}{\tsmv}{\ue'}$ \BY{assumption}
%               \item $\expandsTU{\uDelta}{\utau'}{\tau'}$ \BY{assumption} \pflabel{expandsTU}
%               \item $\hastypeU{\emptyset}{\emptyset}{\eparse}{\aparr{\tBody}{\tParseResultExp}}$ \BY{assumption}\pflabel{eparse}
%               \item $\esyn{\uDelta}{\uGamma}{\uASI{\ctxUpdate{\uA}{\tsmv}{x}}{\Psi, \xuetsmbnd{x}{\tau'}{\eparse}}{\uI}}{\uPhi}{\ue'}{e}{\tau}$ \BY{assumption}\pflabel{expandsU}
%               \item $\uetsmenv{\Delta}{\Psi}$ \BY{assumption}\pflabel{uetsmenv1}
%               \item $\istypeU{\Delta}{\tau'}$ \BY{Lemma \ref{lemma:type-expansion-U} to \pfref{expandsTU}} \pflabel{istype}
%               \item $\uetsmenv{\Delta}{\Psi, \xuetsmbnd{\tsmv}{\tau'}{\eparse}}$ \BY{Definition \ref{def:ueTLM-def-ctx-formation-UP} on \pfref{uetsmenv1}, \pfref{istype} and \pfref{eparse}}\pflabel{uetsmenv3}
%               \item $\hastypeU{\Delta}{\Gamma}{e}{\tau}$ \BY{Outer IH, part 1(a)(i) on \pfref{uetsmenv3} and \pfref{expandsU}}
%             \end{pfsteps*}
%             \resetpfcounter
%           \item[\text{(\ref{rule:esyn-apuetsm})}] We have:
%             \begin{pfsteps*}
%               \item $\ue=\autsmap{b}{\tsmv}$ \BY{assumption}
%               \item $\uPsi = \uASI{\ctxUpdate{\uA'}{\tsmv}{x}}{\Psi', \xuetsmbnd{x}{\tau}{\eparse}}{\uI}$ \BY{assumption}
%               \item $\encodeBody{b}{\ebody}$ \BY{assumption}
%               \item $\evalU{\eparse(\ebody)}{\inj{\lbltxt{Success}}{\ecand}}$ \BY{assumption}
%               \item $\decodeCondE{\ecand}{\ce}$ \BY{assumption}
%               \item $\cana{\emptyset}{\emptyset}{\esceneUP{\uDelta}{\uGamma}{\uPsi}{\uPhi}{b}}{\ce}{e}{\tau}$ \BY{assumption}\pflabel{cvalidE}
%               \item $\uetsmenv{\Delta}{\Psi}$ \BY{assumption} \pflabel{uetsmenv}
%               \item $\istypeU{\Delta}{\tau}$ \BY{Definition \ref{def:ueTLM-def-ctx-formation-UP} on \pfref{uetsmenv}} \pflabel{istype}
%               \item $\emptyset \cap \Delta = \emptyset$ \BY{finite set intersection identity} \pflabel{delta-cap}
%               \item ${\emptyset} \cap \domof{\Gamma} = \emptyset$ \BY{finite set intersection identity} \pflabel{gamma-cap}
%               \item $\hastypeU{\emptyset \cup \Delta}{\emptyset \cup \Gamma}{e}{\tau}$ \BY{IH, part 2(a)(i) on \pfref{cvalidE}, \pfref{delta-cap}, \pfref{gamma-cap} and \pfref{istype}} \pflabel{penultimate}
%               \item $\hastypeU{\Delta}{\Gamma}{e}{\tau}$ \BY{definition of finite set and finite function union over \pfref{penultimate}}               
%              \end{pfsteps*} 
%              \resetpfcounter
%           \item[\text{(\ref{rule:esyn-implicite})}] We have:
%             \begin{pfsteps*}
%               \item $\ue=\auimplicite{\tsmv}{\ue}$ \BY{assumption}
%               \item $\uPsi=\uASI{\uA' \uplus \vExpands{\tsmv}{x}}{\Psi', \xuetsmbnd{x}{\tau'}{\eparse}}{\uI}$ \BY{assumption}
%               \item $\esyn{\uDelta}{\uGamma}{\uASI{\uA' \uplus \vExpands{\tsmv}{x}}{\Psi', \xuetsmbnd{x}{\tau'}{\eparse}}{\uI \uplus \designate{\tau}{a}}}{\uPhi}{\ue}{e}{\tau}$ \BY{assumption} \pflabel{esyn}
%               \item $\hastypeU{\Delta}{\Gamma}{e}{\tau}$ \BY{IH, part 1(a)(i) on \pfref{esyn}}
%             \end{pfsteps*}
%             \resetpfcounter
%           \item[\text{(\ref{rule:esyn-defuptsm})}] We have:
%             \begin{pfsteps*}
%               \item $\ue=\audefuptsm{\utau'}{\eparse}{\tsmv}{\ue'}$ \BY{assumption}
%               \item $\expandsTU{\uDelta}{\utau'}{\tau'}$ \BY{assumption} \pflabel{expandsTU}
%             %  \item \hastypeU{\emptyset}{\emptyset}{\eparse}{\aparr{\tBody}{\tParseResultExp}} \BY{assumption}\pflabel{eparse}
%               \item $\esyn{\uDelta}{\uGamma}{\uPsi}{\uPhi, \uPhyp{\tsmv}{x}{\tau'}{\eparse}}{\ue'}{e}{\tau}$ \BY{assumption}\pflabel{expandsU}
%             %  \item \uetsmenv{\Delta}{\Psi} \BY{assumption}\pflabel{uetsmenv1}
%             %  \item \istypeU{\Delta}{\tau'} \BY{Lemma \ref{lemma:type-expansion-U} to \pfref{expandsTU}} \pflabel{istype}
%             %  \item \uetsmenv{\Delta}{\Psi, \xuetsmbnd{\tsmv}{\tau'}{\eparse}} \BY{Definition \ref{def:ueTLM-def-ctx-formation} on \pfref{uetsmenv1}, \pfref{istype} and \pfref{eparse}}\pflabel{uetsmenv3}
%               \item $\hastypeU{\Delta}{\Gamma}{e}{\tau}$ \BY{IH, part 1(a)(i) on \pfref{expandsU}}
%             \end{pfsteps*}
%             \resetpfcounter
%           \item[\text{(\ref{rule:esyn-implicitp})}] We have:
%             \begin{pfsteps*}
%               \item $\ue=\auimplicitp{\tsmv}{\ue}$ \BY{assumption}
%               \item $\uPhi=\uASI{\uA \uplus \vExpands{\tsmv}{x}}{\Phi, \xuptsmbnd{a}{\tau'}{\eparse}}{\uI}$ \BY{assumption}
%               \item $\esyn{\uDelta}{\uGamma}{\uPsi}{\uASI{\uA \uplus \vExpands{\tsmv}{x}}{\Phi, \xuptsmbnd{a}{\tau'}{\eparse}}{\uI \uplus \designate{\tau}{a}}}{\ue}{e}{\tau}$ \BY{assumption} \pflabel{esyn}
%               \item $\hastypeU{\Delta}{\Gamma}{e}{\tau}$ \BY{IH, part 1(a)(i) on \pfref{esyn}}
%             \end{pfsteps*}
%             \resetpfcounter
%         \end{byCases}
%       \item We induct on the assumption. There is one case.
%         \begin{byCases}
%           \item[\text{(\ref{rule:rsyn})}] We have:
%             \begin{pfsteps*}
%               \item $\urv=\aumatchrule{\upv}{\ue}$ \BY{assumption}
%               \item $r=\aematchrule{p}{e}$ \BY{assumption}
%               \item $\patExpands{\uGG{\uA'}{\pctx}}{\uPhi}{\upv}{p}{\tau}$ \BY{assumption} \pflabel{patExpands}
%               \item $\esyn{\uDelta}{\uGG{{\uA}\uplus{\uA'}}{\Gcons{\Gamma}{\pctx}}}{\uPsi}{\uPhi}{\ue}{e}{\tau'}$ \BY{assumption} \pflabel{expandsUP}
%               \item $\patType{\pctx}{p}{\tau}$ \BY{Theorem \ref{thm:typed-pattern-expansion-B}, part 1 on \pfref{patExpands}}\pflabel{patType}
%               \item $\hastypeU{\Delta}{\Gcons{\Gamma}{\pctx}}{e}{\tau'}$ \BY{IH, part 1(a)(i) on \pfref{expandsUP}} \pflabel{hasType}
%               \item $\ruleType{\Delta}{\Gamma}{\aematchrule{p}{e}}{\tau}{\tau'}$ \BY{Rule (\ref{rule:ruleType}) on \pfref{patType} and \pfref{hasType}}
%             \end{pfsteps*}
%             \resetpfcounter
%         \end{byCases}
%     \end{enumerate}
%     \item \begin{enumerate}
%       \item We induct on the assumption.
%         \begin{byCases}
%           \item[\text{(\ref{rule:eana-subsume})}] We have:
%             \begin{pfsteps*}
%               \item $\esynX{\ue}{e}{\tau}$ \BY{assumption} \pflabel{esyn}
%               \item $\hastypeU{\Delta}{\Gamma}{e}{\tau}$ \BY{IH, part 1(a)(i) on \pfref{esyn}}
%             \end{pfsteps*}
%           \item[\text{(\ref{rule:eana-let}) through (\ref{rule:eana-match})}] In each of these cases, we apply:
%             \begin{itemize}
%               \item Lemma \ref{lemma:type-expansion-U} to or over all type expansion premises.
%               \item The IH, part 1(a)(i) to or over all synthetic typed expression expansion premises.
%               \item The IH, part 1(a)(ii) to or over all synthetic rule expansion premises.
%               \item The IH, part 1(b)(i) to or over all analytic typed expression expansion premises.
%             \end{itemize}
%             We then derive the conclusion by applying Rules (\ref{rules:hastypeUP}) and Rule (\ref{rule:ruleType}) as needed. 
%           \item[\text{(\ref{rule:eana-defuetsm})}] We have:
%             \begin{pfsteps*}
%               \item $\ue=\audefuetsm{\utau'}{\eparse}{\tsmv}{\ue'}$ \BY{assumption}
%               \item $\expandsTU{\uDelta}{\utau'}{\tau'}$ \BY{assumption} \pflabel{expandsTU}
%               \item $,$ \BY{assumption}\pflabel{eparse}
%               \item $\eana{\uDelta}{\uGamma}{\uPsi, \uShyp{\tsmv}{x}{\tau'}{\eparse}}{\uPhi}{\ue'}{e}{\tau}$ \BY{assumption}\pflabel{expandsU}
%               \item $\uetsmenv{\Delta}{\Psi}$ \BY{assumption}\pflabel{uetsmenv1}
%               \item $\istypeU{\Delta}{\tau'}$ \BY{Lemma \ref{lemma:type-expansion-U} to \pfref{expandsTU}} \pflabel{istype}
%               \item $\uetsmenv{\Delta}{\Psi, \xuetsmbnd{\tsmv}{\tau'}{\eparse}}$ \BY{Definition \ref{def:ueTLM-def-ctx-formation-UP} on \pfref{uetsmenv1}, \pfref{istype} and \pfref{eparse}}\pflabel{uetsmenv3}
%             %  \item \uetsmenv{\Delta}{\Psi} \BY{assumption}\pflabel{uetsmenv1}
%             %  \item \istypeU{\Delta}{\tau'} \BY{Lemma \ref{lemma:type-expansion-U} to \pfref{expandsTU}} \pflabel{istype}
%             %  \item \uetsmenv{\Delta}{\Psi, \xuetsmbnd{\tsmv}{\tau'}{\eparse}} \BY{Definition \ref{def:ueTLM-def-ctx-formation} on \pfref{uetsmenv1}, \pfref{istype} and \pfref{eparse}}\pflabel{uetsmenv3}
%               \item $\hastypeU{\Delta}{\Gamma}{e}{\tau}$ \BY{IH, part 1(b)(i) on \pfref{expandsU}}
%             \end{pfsteps*}
%             \resetpfcounter
%           \item[\text{(\ref{rule:eana-implicite})}] We have:
%             \begin{pfsteps*}
%               \item $\ue=\autsmap{b}{\tsmv}$ \BY{assumption}
%               \item $\uPsi = \uPsi', \uShyp{\tsmv}{x}{\tau}{\eparse}$ \BY{assumption}
%               \item $\encodeBody{b}{\ebody}$ \BY{assumption}
%               \item $\evalU{\eparse(\ebody)}{\inj{\lbltxt{Success}}{\ecand}}$ \BY{assumption}
%               \item $\decodeCondE{\ecand}{\ce}$ \BY{assumption}
%               \item $\cana{\emptyset}{\emptyset}{\esceneUP{\uDelta}{\uGamma}{\uPsi}{\uPhi}{b}}{\ce}{e}{\tau}$ \BY{assumption}\pflabel{cvalidE}
%             %  \item \uetsmenv{\Delta}{\Psi} \BY{assumption} \pflabel{uetsmenv}
%               \item $\emptyset \cap \Delta = \emptyset$ \BY{finite set intersection identity} \pflabel{delta-cap}
%               \item ${\emptyset} \cap \domof{\Gamma} = \emptyset$ \BY{finite set intersection identity} \pflabel{gamma-cap}
%               \item $\hastypeU{\emptyset \cup \Delta}{\emptyset \cup \Gamma}{e}{\tau}$ \BY{IH, part 2(b)(i) on \pfref{cvalidE}, \pfref{delta-cap}, and \pfref{gamma-cap}} \pflabel{penultimate}
%               \item $\hastypeU{\Delta}{\Gamma}{e}{\tau}$ \BY{definition of finite set union over \pfref{penultimate}}               
%              \end{pfsteps*} 
%              \resetpfcounter
%           \item[\text{(\ref{rule:eana-lit})}] We have:
%             \begin{pfsteps*}
%               \item $\ue=\auelit{b}$ \BY{assumption}
%               \item $\uPsi=\uASI{\uA}{\Psi, \xuetsmbnd{x}{\tau}{\eparse}}{\uI \uplus \designate{\tau}{a}}$ \BY{assumption}
%               \item $\encodeBody{b}{\ebody}$ \BY{assumption}
%               \item $\evalU{\ap{\eparse}{\ebody}}{\inj{\lbltxt{Success}}{\ecand}}$ \BY{assumption}
%               \item $\decodeCondE{\ecand}{\ce}$ \BY{assumption}
%               \item $\cana{\emptyset}{\emptyset}{\esceneUP{\uDelta}{\uGamma}{\uASI{\uA}{\Psi, \xuetsmbnd{x}{\tau}{\eparse}}{\uI \uplus \designate{\tau}{a}}}{\uPhi}{b}}{\ce}{e}{\tau}$ \BY{assumption} \pflabel{cvalidE}
%               \item $\emptyset \cap \Delta = \emptyset$ \BY{finite set intersection identity} \pflabel{delta-cap}
%               \item ${\emptyset} \cap \domof{\Gamma} = \emptyset$ \BY{finite set intersection identity} \pflabel{gamma-cap}
%               \item $\hastypeU{\emptyset \cup \Delta}{\emptyset \cup \Gamma}{e}{\tau}$ \BY{IH, part 2(a)(i) on \pfref{cvalidE}, \pfref{delta-cap}, and \pfref{gamma-cap}} \pflabel{penultimate}
%               \item $\hastypeU{\Delta}{\Gamma}{e}{\tau}$ \BY{definition of finite set union over \pfref{penultimate}}
%             \end{pfsteps*}
%             \resetpfcounter
%           \item[\text{(\ref{rule:eana-defuptsm})}] We have:
%             \begin{pfsteps*}
%               \item $\ue=\audefuptsm{\utau'}{\eparse}{\tsmv}{\ue'}$ \BY{assumption}
%               \item $\expandsTU{\uDelta}{\utau'}{\tau'}$ \BY{assumption} \pflabel{expandsTU}
%             %  \item \hastypeU{\emptyset}{\emptyset}{\eparse}{\aparr{\tBody}{\tParseResultExp}} \BY{assumption}\pflabel{eparse}
%               \item $\eana{\uDelta}{\uGamma}{\uPsi}{\uPhi, \uPhyp{\tsmv}{x}{\tau'}{\eparse}}{\ue'}{e}{\tau}$ \BY{assumption}\pflabel{expandsU}
%             %  \item \uetsmenv{\Delta}{\Psi} \BY{assumption}\pflabel{uetsmenv1}
%             %  \item \istypeU{\Delta}{\tau'} \BY{Lemma \ref{lemma:type-expansion-U} to \pfref{expandsTU}} \pflabel{istype}
%             %  \item \uetsmenv{\Delta}{\Psi, \xuetsmbnd{\tsmv}{\tau'}{\eparse}} \BY{Definition \ref{def:ueTLM-def-ctx-formation} on \pfref{uetsmenv1}, \pfref{istype} and \pfref{eparse}}\pflabel{uetsmenv3}
%               \item $\hastypeU{\Delta}{\Gamma}{e}{\tau}$ \BY{IH, part 1(b)(i) on \pfref{expandsU}}
%             \end{pfsteps*}
%             \resetpfcounter
%           \item[\text{(\ref{rule:eana-implicitp})}] We have:
%             \begin{pfsteps*}
%               \item $\ue=\auimplicitp{\tsmv}{\ue}$ \BY{assumption}
%               \item $\uPhi=\uASI{\uA \uplus \vExpands{\tsmv}{x}}{\Phi, \xuptsmbnd{a}{\tau'}{\eparse}}{\uI}$ \BY{assumption}
%               \item $\eana{\uDelta}{\uGamma}{\uPsi}{\uASI{\uA \uplus \vExpands{\tsmv}{x}}{\Phi, \xuptsmbnd{a}{\tau'}{\eparse}}{\uI \uplus \designate{\tau}{a}}}{\ue}{e}{\tau}$ \BY{assumption} \pflabel{esyn}
%               \item $\hastypeU{\Delta}{\Gamma}{e}{\tau}$ \BY{IH, part 1(b)(i) on \pfref{esyn}}
%             \end{pfsteps*}
%             \resetpfcounter
%         \end{byCases}
%       \item We induct on the assumption. There is one case.
%         \begin{byCases}
%           \item[\text{(\ref{rule:rana})}] We have:
%             \begin{pfsteps*}
%               \item $\urv=\aumatchrule{\upv}{\ue}$ \BY{assumption}
%               \item $r=\aematchrule{p}{e}$ \BY{assumption}
%               \item $\patExpands{\uGG{\uA'}{\pctx}}{\uPhi}{\upv}{p}{\tau}$ \BY{assumption} \pflabel{patExpands}
%               \item $\eana{\uDelta}{\uGG{{\uA}\uplus{\uA'}}{\Gcons{\Gamma}{\pctx}}}{\uPsi}{\uPhi}{\ue}{e}{\tau'}$ \BY{assumption} \pflabel{expandsUP}
%               \item $\patType{\pctx}{p}{\tau}$ \BY{Theorem \ref{thm:typed-pattern-expansion-B}, part 1 on \pfref{patExpands}}\pflabel{patType}
%               \item $\hastypeU{\Delta}{\Gcons{\Gamma}{\pctx}}{e}{\tau'}$ \BY{IH, part 1(b)(i) on \pfref{expandsUP}} \pflabel{hasType}
%               \item $\ruleType{\Delta}{\Gamma}{\aematchrule{p}{e}}{\tau}{\tau'}$ \BY{Rule (\ref{rule:ruleType}) on \pfref{patType} and \pfref{hasType}}
%             \end{pfsteps*}
%             \resetpfcounter
%         \end{byCases}
%     \end{enumerate}
%   \end{enumerate}
%   \item In the following, let $\uDelta=\uDD{\uD}{\Delta_\text{app}}$ and $\uGamma=\uGG{\uG}{\Gamma_\text{app}}$ and $\escenev=\esceneUP{\uDelta}{\uGamma}{\uPsi}{\uPhi}{b}$.
%   \begin{enumerate}
%     \item \begin{enumerate}
%       \item We induct on the assumption.
%         \begin{byCases}
%           \item[\text{(\ref{rule:csyn-var})}] We have:
%             \begin{pfsteps*}
%               \item $e=x$ \BY{assumption}
%               \item $\Gamma=\Gamma', \Ghyp{x}{\tau}$ \BY{assumption}
%               \item $\hastypeU{\Delta}{\Gamma', \Ghyp{x}{\tau}}{x}{\tau}$ \BY{Rule (\ref{rule:hastypeUP-var})}
%             \end{pfsteps*}
%             \resetpfcounter 
%           \item[\text{(\ref{rule:csyn-asc})}] We have:
%             \begin{pfsteps*}
%                \item $\ce=\aceasc{\ctau}{\ce'}$ \BY{assumption}
%                \item $\Delta \cap \Delta_\text{app}=\emptyset$ \BY{assumption} \pflabel{delta-disjoint}
%                \item $\domof{\Gamma} \cap \domof{\Gamma_\text{app}}=\emptyset$ \BY{assumption} \pflabel{gamma-disjoint}
%                \item $\cvalidT{\Delta}{\tsfrom{\escenev}}{\ctau}{\tau}$ \BY{assumption}\pflabel{expandsTU}
%                \item $\canaX{\ce'}{e}{\tau}$ \BY{assumption}\pflabel{eanaX}
%                \item $\istypeU{\Delta \cup \Delta_\text{app}}{\tau}$ \BY{Lemma \ref{lemma:candidate-expansion-type-validation} on \pfref{expandsTU}}\pflabel{istype}
%                \item $\hastypeU{\Delta}{\Gamma}{e}{\tau}$ \BY{IH, part 2(b)(i) to \pfref{eanaX}, \pfref{delta-disjoint}, \pfref{gamma-disjoint} and  \pfref{istype}}
%              \end{pfsteps*}
%              \resetpfcounter
%           \item[\text{(\ref{rule:csyn-let}) through (\ref{rule:csyn-match})}] In each of these cases, we apply:
%             \begin{itemize}
%               \item Lemma \ref{lemma:candidate-expansion-type-validation} to or over all ce-type validation premises.
%               \item The IH, part 2(a)(i) to or over all synthetic ce-expression validation premises.
%               \item The IH, part 2(a)(ii) to or over all synthetic ce-rule validation premises.
%               \item The IH, part 2(b)(i) to or over all analytic ce-expression validation premises.
%             \end{itemize}
%             We then derive the conclusion by applying Rules (\ref{rules:hastypeUP}), Rule (\ref{rule:ruleType}), Lemma \ref{lemma:weakening-UP},  the identification convention and exchange as needed.
%           \item[\text{(\ref{rule:csyn-splicede})}] We have:
%             \begin{pfsteps*}
%               \item $\ce=\acesplicede{m}{n}{\ctau}$ \BY{assumption}
%               \item $\parseUExp{\bsubseq{b}{m}{n}}{\ue}$ \BY{assumption}
%               \item $\esyn{\uDelta}{\uGamma}{\uPsi}{\uPhi}{\ue}{e}{\tau}$ \BY{assumption} \pflabel{expands}
%             %  \item $\uetsmenv{\Delta_\text{app}}{\Psi}$ \BY{assumption} \pflabel{uetsmenv}
%               \item $\Delta \cap \Delta_\text{app}=\emptyset$ \BY{assumption} \pflabel{delta-disjoint}
%               \item $\domof{\Gamma} \cap \domof{\Gamma_\text{app}}=\emptyset$ \BY{assumption} \pflabel{gamma-disjoint}
%               \item $\hastypeU{\Delta_\text{app}}{\Gamma_\text{app}}{e}{\tau}$ \BY{IH, part 1(a)(i) on \pfref{expands}} \pflabel{hastype}
%               \item $\hastypeU{\Dcons{\Delta}{\Delta_\text{app}}}{\Gcons{\Gamma}{\Gamma_\text{app}}}{e}{\tau}$ \BY{Lemma \ref{lemma:weakening-UP} over $\Delta$ and $\Gamma$ and exchange on \pfref{hastype}}
%             \end{pfsteps*}
%             \resetpfcounter
%         \end{byCases}
%       \item We induct on the assumption. There is one case.
%         \begin{byCases}
%           \item[\text{(\ref{rule:crsyn})}] We have:
%             \begin{pfsteps*}
%               \item $\crv=\acematchrule{p}{\ce}$ \BY{assumption}
%               \item $r=\aematchrule{p}{e}$ \BY{assumption}
%               \item $\patType{\pctx}{p}{\tau}$ \BY{assumption} \pflabel{patType}
%               \item $\csyn{\Delta}{\Gcons{\Gamma}{\pctx}}{\esceneUP{\uDelta}{\uGamma}{\uPsi}{\uPhi}{b}}{\ce}{e}{\tau'}$ \BY{assumption} \pflabel{cvalidE}
%               \item $\Delta \cap \Delta_\text{app} = \emptyset$ \BY{assumption}\pflabel{delta-disjoint}
%               \item $\domof{\Gamma} \cap \domof{\pctx} = \emptyset$ \BY{identification convention}\pflabel{gamma-disjoint1}
%               \item $\domof{\Gamma_\text{app}} \cap \domof{\pctx} = \emptyset$ \BY{identification convention}\pflabel{gamma-disjoint2}
%               \item $\domof{\Gamma} \cap \domof{\Gamma_\text{app}} = \emptyset$ \BY{assumption}\pflabel{gamma-disjoint3}
%               \item $\domof{\Gcons{\Gamma}{\pctx}} \cap \domof{\Gamma_\text{app}} = \emptyset$ \BY{standard finite set definitions and identities on \pfref{gamma-disjoint1}, \pfref{gamma-disjoint2} and \pfref{gamma-disjoint3}}\pflabel{gamma-disjoint4}
%               \item $\hastypeU{\Dcons{\Delta}{\Delta_\text{app}}}{\Gcons{\Gcons{\Gamma}{\pctx}}{\Gamma_\text{app}}}{e}{\tau'}$ \BY{IH, part 2(a)(i) on \pfref{cvalidE}, \pfref{delta-disjoint} and \pfref{gamma-disjoint4}}\pflabel{hastype}
%               \item $\hastypeU{\Dcons{\Delta}{\Delta_\text{app}}}{\Gcons{\Gcons{\Gamma}{\Gamma_\text{app}}}{\pctx}}{e}{\tau'}$ \BY{exchange of $\pctx$ and $\Gamma_\text{app}$ on \pfref{hastype}}\pflabel{hastype2}
%               \item $\ruleType{\Dcons{\Delta}{\Delta_\text{app}}}{\Gcons{\Gamma}{\Gamma_\text{app}}}{\aematchrule{p}{e}}{\tau}{\tau'}$ \BY{Rule (\ref{rule:ruleType}) on \pfref{patType} and \pfref{hastype2}}
%             \end{pfsteps*}
%             \resetpfcounter
%         \end{byCases}
%     \end{enumerate}
%     \item  \begin{enumerate}
%       \item We induct on the assumption.
%         \begin{byCases}
%           \item[\text{(\ref{rule:cana-subsume})}] We have:
%             \begin{pfsteps*}
%               \item $\csynX{\ce}{e}{\tau}$ \BY{assumption} \pflabel{esyn}
%               \item $\hastypeU{\Delta}{\Gamma}{e}{\tau}$ \BY{IH, part 2(a)(i) on \pfref{esyn}}
%             \end{pfsteps*}
%           \item[\text{(\ref{rule:cana-let}) through (\ref{rule:eana-match})}] In each of these cases, we apply:
%             \begin{itemize}
%               \item Lemma \ref{lemma:candidate-expansion-type-validation} to or over all ce-type validation premises.
%               \item The IH, part 2(a)(i) to or over all synthetic ce-expression validation premises.
%               \item The IH, part 2(a)(ii) to or over all synthetic ce-rule validation premises.
%               \item The IH, part 2(b)(i) to or over all analytic ce-expression validation premises.
%             \end{itemize}
%             We then derive the conclusion by applying Rules (\ref{rules:hastypeUP}), Rule (\ref{rule:ruleType}), Lemma \ref{lemma:weakening-UP},  the identification convention and exchange as needed.
%           \item[\text{(\ref{rule:cana-splicede})}] We have:
%             \begin{pfsteps*}
%               \item $\ce=\acesplicede{m}{n}{\ctau}$ \BY{assumption}
%               \item $\parseUExp{\bsubseq{b}{m}{n}}{\ue}$ \BY{assumption}
%               \item $\eana{\uDelta}{\uGamma}{\uPsi}{\uPhi}{\ue}{e}{\tau}$ \BY{assumption} \pflabel{expands}
%               \item $\istypeU{\Delta \cup \Delta_\text{app}}{\tau}$ \BY{assumption} \pflabel{istype}
%             %  \item $\uetsmenv{\Delta_\text{app}}{\Psi}$ \BY{assumption} \pflabel{uetsmenv}
%               \item $\Delta \cap \Delta_\text{app}=\emptyset$ \BY{assumption} \pflabel{delta-disjoint}
%               \item $\domof{\Gamma} \cap \domof{\Gamma_\text{app}}=\emptyset$ \BY{assumption} \pflabel{gamma-disjoint}
%               \item $\hastypeU{\Delta_\text{app}}{\Gamma_\text{app}}{e}{\tau}$ \BY{IH, part 1(b)(i) on \pfref{expands}, \pfref{delta-disjoint}, \pfref{gamma-disjoint} and \pfref{istype}} \pflabel{hastype}
%               \item $\hastypeU{\Dcons{\Delta}{\Delta_\text{app}}}{\Gcons{\Gamma}{\Gamma_\text{app}}}{e}{\tau}$ \BY{Lemma \ref{lemma:weakening-UP} over $\Delta$ and $\Gamma$ and exchange on \pfref{hastype}}
%             \end{pfsteps*}
%             \resetpfcounter
%         \end{byCases}
%       \item We induct on the assumption. There is one case.
%         \begin{byCases}
%           \item[\text{(\ref{rule:crana})}] We have:    
%             \begin{pfsteps*}
%                 \item $\crv=\acematchrule{p}{\ce}$ \BY{assumption}
%                 \item $r=\aematchrule{p}{e}$ \BY{assumption}
%                 \item $\patType{\pctx}{p}{\tau}$ \BY{assumption} \pflabel{patType}
%                 \item $\cana{\Delta}{\Gcons{\Gamma}{\pctx}}{\esceneUP{\uDelta}{\uGamma}{\uPsi}{\uPhi}{b}}{\ce}{e}{\tau'}$ \BY{assumption} \pflabel{cvalidE}
%                 \item $\istypeU{\Delta \cup \Delta_\text{app}}{\tau'}$ \BY{assumption} \pflabel{istype}
%                 \item $\domof{\Gamma} \cap \domof{\Gamma_\text{app}} = \emptyset$ \BY{assumption}\pflabel{gamma-disjoint3}
%                 \item $\Delta \cap \Delta_\text{app} = \emptyset$ \BY{assumption}\pflabel{delta-disjoint}
%                 \item $\domof{\Gamma} \cap \domof{\pctx} = \emptyset$ \BY{identification convention}\pflabel{gamma-disjoint1}
%                 \item $\domof{\Gamma_\text{app}} \cap \domof{\pctx} = \emptyset$ \BY{identification convention}\pflabel{gamma-disjoint2}
%                 \item $\domof{\Gcons{\Gamma}{\pctx}} \cap \domof{\Gamma_\text{app}} = \emptyset$ \BY{standard finite set definitions and identities on \pfref{gamma-disjoint1}, \pfref{gamma-disjoint2} and \pfref{gamma-disjoint3}}\pflabel{gamma-disjoint4}
%                 \item $\hastypeU{\Dcons{\Delta}{\Delta_\text{app}}}{\Gcons{\Gcons{\Gamma}{\pctx}}{\Gamma_\text{app}}}{e}{\tau'}$ \BY{IH, part 2(b)(i) on \pfref{cvalidE}, \pfref{delta-disjoint}, \pfref{gamma-disjoint4} and \pfref{istype}}\pflabel{hastype}
%                 \item $\hastypeU{\Dcons{\Delta}{\Delta_\text{app}}}{\Gcons{\Gcons{\Gamma}{\Gamma_\text{app}}}{\pctx}}{e}{\tau'}$ \BY{exchange of $\pctx$ and $\Gamma_\text{app}$ on \pfref{hastype}}\pflabel{hastype2}
%                 \item $\ruleType{\Dcons{\Delta}{\Delta_\text{app}}}{\Gcons{\Gamma}{\Gamma_\text{app}}}{\aematchrule{p}{e}}{\tau}{\tau'}$ \BY{Rule (\ref{rule:ruleType}) on \pfref{patType} and \pfref{hastype2}}
%               \end{pfsteps*}
%               \resetpfcounter

%         \end{byCases}
%     \end{enumerate}
%   \end{enumerate}
% \end{enumerate}

% We must now show that the induction is well-founded. All applications of the IH are on subterms except the following.  

% \begin{itemize}
% \item The only cases in the proof of part 1 that invoke the IH, part 2 are Case (\ref{rule:esyn-apuetsm}) in the proof of part 1(a)(i) and Case (\ref{rule:eana-lit}) in the proof of part 1(b)(i). The only cases in the proof of part 2 that invoke the IH, part 1 are Case (\ref{rule:csyn-splicede}) in the proof of part 2(a)(i) and Case (\ref{rule:cana-splicede}) in the proof of part 2(b)(i). We can show that the following metric on the judgements that we induct on is stable in one direction and strictly decreasing in the other direction:
% \begin{align*}
% \sizeof{\esyn{\uDelta}{\uGamma}{\uPsi}{\uPhi}{\ue}{e}{\tau}} & = \sizeof{\ue}\\
% \sizeof{\eana{\uDelta}{\uGamma}{\uPsi}{\uPhi}{\ue}{e}{\tau}} & = \sizeof{\ue}\\
% \sizeof{\csyn{\Delta}{\Gamma}{\esceneUP{\uDelta}{\uGamma}{\uPsi}{\uPhi}{b}}{\ce}{e}{\tau}} & = \sizeof{b}\\
% \sizeof{\cana{\Delta}{\Gamma}{\esceneUP{\uDelta}{\uGamma}{\uPsi}{\uPhi}{b}}{\ce}{e}{\tau}} & = \sizeof{b}
% \end{align*}
% where $\sizeof{b}$ is the length of $b$ and $\sizeof{\ue}$ is the sum of the lengths of the ueTLM literal bodies in $\ue$,
% \begin{align*}
% \sizeof{\ux} & = 0\\
% \sizeof{\auasc{\utau}{\ue}} & = \sizeof{\ue}\\
% \sizeof{\auletsyn{\ux}{\ue}{\ue'}} & = \sizeof{\ue} + \sizeof{\ue'}\\
% \sizeof{\auanalam{\ux}{\ue}} & = \sizeof{\ue}\\
% \sizeof{\aulam{\utau}{\ux}{\ue}} &= \sizeof{\ue}\\
% \sizeof{\auap{\ue_1}{\ue_2}} & = \sizeof{\ue_1} + \sizeof{\ue_2}\\
% \sizeof{\autlam{\ut}{\ue}} & = \sizeof{\ue}\\
% \sizeof{\autap{\ue}{\utau}} & = \sizeof{\ue}\\
% \sizeof{\auanafold{\ue}} & = \sizeof{\ue}\\
% \sizeof{\auunfold{\ue}} & = \sizeof{\ue}\\
% %\end{align*}
% %\begin{align*}
% \sizeof{\autpl{\labelset}{\mapschema{\ue}{i}{\labelset}}} & = \sum_{i \in \labelset} \sizeof{\ue_i}\\
% \sizeof{\aupr{\ell}{\ue}} & = \sizeof{\ue}\\
% \sizeof{\auanain{\ell}{\ue}} & = \sizeof{\ue}\\
% %\sizeof{\aucase{\labelset}{\utau}{\ue}{\mapschemab{\ux}{\ue}{i}{\labelset}}} & = \sizeof{\ue} + \sum_{i \in \labelset} \sizeof{\ue_i}\\
% \sizeof{\aumatchwithb{n}{\ue}{\seqschemaX{\urv}}} & = \sizeof{\ue} + \sum_{1 \leq i \leq n} \sizeof{r_i}\\
% \sizeof{\audefuetsm{\utau}{\eparse}{\tsmv}{\ue}} & = \sizeof{\ue}\\
% \sizeof{\auimplicite{\tsmv}{\ue}} & = \sizeof{\ue}\\
% \sizeof{\autsmap{b}{\tsmv}} & = \sizeof{b}\\
% \sizeof{\auelit{b}} & = \sizeof{b}\\
% \sizeof{\audefuptsm{\utau}{\eparse}{\tsmv}{\ue}} & = \sizeof{\ue}\\
% \sizeof{\auimplicitp{\tsmv}{\ue}} & = \sizeof{\ue}
% \end{align*}
% and $\sizeof{r}$ is defined as follows:
% \begin{align*}
% \sizeof{\aumatchrule{\upv}{\ue}} & = \sizeof{\ue}
% \end{align*}

% Going from part 1 to part 2, the metric remains stable:
% \begin{align*}
%  & \sizeof{\esyn{\uDelta}{\uGamma}{\uPsi}{\uPhi}{\autsmap{b}{\tsmv}}{e}{\tau}}\\
% =& \sizeof{\eana{\uDelta}{\uGamma}{\uPsi}{\uPhi}{\auelit{b}}{e}{\tau}}\\
% =& \sizeof{\cana{\emptyset}{\emptyset}{\esceneUP{\uDelta}{\uGamma}{\uPsi}{\uPhi}{b}}{\ce}{e}{\tau}}\\
% =&\sizeof{b}\end{align*}

% Going from part 2 to part 1, in each case we have that $\parseUExp{\bsubseq{b}{m}{n}}{\ue}$ and the IH is applied to the judgements $\esyn{\uDelta}{\uGamma}{\uPsi}{\uPhi}{\ue}{e}{\tau}$ and $\eana{\uDelta}{\uGamma}{\uPsi}{\uPhi}{\ue}{e}{\tau}$, respectively. Because the metric is stable when passing from part 1 to part 2, we must have that it is strictly decreasing in the other direction:
% \[\sizeof{\esyn{\uDelta}{\uGamma}{\uPsi}{\uPhi}{\ue}{e}{\tau}} < \sizeof{\csyn{\Delta}{\Gamma}{\esceneUP{\uDelta}{\uGamma}{\uPsi}{\uPhi}{b}}{\acesplicede{m}{n}{\ctau}}{e}{\tau}}\]
% and
% \[\sizeof{\eana{\uDelta}{\uGamma}{\uPsi}{\uPhi}{\ue}{e}{\tau}} < \sizeof{\cana{\Delta}{\Gamma}{\esceneUP{\uDelta}{\uGamma}{\uPsi}{\uPhi}{b}}{\acesplicede{m}{n}{\ctau}}{e}{\tau}}\]
% i.e. by the definitions above, 
% \[\sizeof{\ue} < \sizeof{b}\]

% This is established by appeal to Condition \ref{condition:body-subsequences}, which states that subsequences of $b$ are no longer than $b$, and the following condition, which states that an unexpanded expression constructed by parsing a textual sequence $b$ is strictly smaller, as measured by the metric defined above, than the length of $b$, because some characters must necessarily be used to delimit each literal body.
% \begin{condition}[Expression Parsing Monotonicity]\label{condition:body-parsing-B} If $\parseUExp{b}{\ue}$ then $\sizeof{\ue} < \sizeof{b}$.\end{condition}

% Combining Conditions \ref{condition:body-subsequences} and \ref{condition:body-parsing-B}, we have that $\sizeof{\ue} < \sizeof{b}$ as needed.
% \item In Case (\ref{rule:eana-subsume}) of the proof of part 1(b)(i), we apply the IH, part 1(a)(i), with $\ue=\ue$. This is well-founded because all applications of the IH, part 1(b)(i) elsewhere in the proof are on strictly smaller terms.
% \item Similarly, in Case (\ref{rule:cana-subsume}) of the proof of part 2(b)(i), we apply the IH, part 2(a)(i), with $\ce=\ce$. This is well-founded because all applications of the IH, part 2(b)(i) elsewhere in the proof are on strictly smaller terms.
% \end{itemize}
% \end{proof} 



%\renewcommand{\baselinestretch}{1.0}\normalsize
% By default \bibsection is \chapter*, but we really want this to show
% up in the table of contents and pdf bookmarks.
\renewcommand{\bibsection}{\chapter*{\bibname}\addcontentsline{toc}{chapter}{Bibliography}}
% \renewcommand{\bibpreamble}{\todolater{List conference abbreviations.}\\
% \todolater{Remove extraneous nonsense from entries.}}
\bibliographystyle{plainnat}
\bibliography{research} %your bib file


\end{document}

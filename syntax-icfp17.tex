%% For double-blind review submission
\documentclass[acmlarge,review,anonymous]{acmart}\settopmatter{printfolios=true}
%% For single-blind review submission
%\documentclass[acmlarge,review]{acmart}\settopmatter{printfolios=true}
%% For final camera-ready submission
%\documentclass[acmlarge]{acmart}\settopmatter{}

%% Note: Authors migrating a paper from PACMPL format to traditional
%% SIGPLAN proceedings format should change 'acmlarge' to
%% 'sigplan,10pt'.


%% Some recommended packages.
\usepackage{booktabs}   %% For formal tables:
                        %% http://ctan.org/pkg/booktabs
\usepackage{subcaption} %% For complex figures with subfigures/subcaptions
                        %% http://ctan.org/pkg/subcaption

%% Cyrus packages
\usepackage{microtype}

%% Listings
\usepackage{listings}
\lstset{tabsize=2, 
basicstyle=\ttfamily\fontsize{7.5pt}{1em}\selectfont, 
commentstyle=\itshape\ttfamily\color{gray}, 
stringstyle=\ttfamily\color{red},
mathescape=false,escapechar=!,
numbers=left, numberstyle=\scriptsize\color{mygray}, language=ML,moredelim=[il][\sffamily]{?},showspaces=false,showstringspaces=false,xleftmargin=0pt, numbersep=3pt, morekeywords=[1]{tyfam,opfam,let,fn,val,def,casetype,objtype,metadata,of,*,datatype,new,toast,syntax,module,where,import,for,ana,syn,opcon,tycon,metasignature,metamodule,metasig,metamod,static,at,by,tycase,mod,macro,match,pattern,in,patterns,expressions,implicit,forall,exptsm,pattsm},deletekeywords={double},classoffset=0,
aboveskip=2pt,belowskip=0pt,
moredelim=**[is][\color{red}]{SSTR}{ESTR},
moredelim=**[is][\color{OliveGreen}]{SHTML}{EHTML},
moredelim=**[is][\color{purple}]{SCSS}{ECSS},
moredelim=**[is][\color{brown}]{SSQL}{ESQL},
moredelim=**[is][\color{orange}]{SCOLOR}{ECOLOR},
moredelim=**[is][\color{magenta}]{SPCT}{EPCT}, 
moredelim=**[is][\color{gray}]{SNAT}{ENAT}, 
moredelim=**[is][\color{blue}]{SURL}{EURL},
moredelim=**[is][\color{SeaGreen}]{SQT}{EQT},
moredelim=**[is][\color{Periwinkle}]{SGRM}{EGRM},
moredelim=**[is][\color{YellowGreen}]{SID}{EID},
moredelim=**[is][\color{Sepia}]{SUS}{EUS},
deletestring=[d]{"},
}
\newcommand{\liv}[1]{\lstinline{#1}}
\newcommand{\li}[1]{\lstinline{#1}}


\makeatletter\if@ACM@journal\makeatother
%% Journal information (used by PACMPL format)
%% Supplied to authors by publisher for camera-ready submission
\acmJournal{PACMPL}
\acmVolume{1}
\acmNumber{1}
\acmArticle{1}
\acmYear{2017}
\acmMonth{1}
\acmDOI{10.1145/nnnnnnn.nnnnnnn}
\startPage{1}
\else\makeatother
%% Conference information (used by SIGPLAN proceedings format)
%% Supplied to authors by publisher for camera-ready submission
\acmConference[PL'17]{ACM SIGPLAN Conference on Programming Languages}{January 01--03, 2017}{New York, NY, USA}
\acmYear{2017}
\acmISBN{978-x-xxxx-xxxx-x/YY/MM}
\acmDOI{10.1145/nnnnnnn.nnnnnnn}
\startPage{1}
\fi


%% Copyright information
%% Supplied to authors (based on authors' rights management selection;
%% see authors.acm.org) by publisher for camera-ready submission
\setcopyright{none}             %% For review submission
%\setcopyright{acmcopyright}
%\setcopyright{acmlicensed}
%\setcopyright{rightsretained}
%\copyrightyear{2017}           %% If different from \acmYear


%% Bibliography style
\bibliographystyle{ACM-Reference-Format}
%% Citation style
%% Note: author/year citations are required for papers published as an
%% issue of PACMPL.
\citestyle{acmauthoryear}   %% For author/year citations



\begin{document}

%% Title information
\title[Reasonably Programmable Syntax]{Reasonably Programmable Syntax}         %% [Short Title] is optional;
                                        %% when present, will be used in
                                        %% header instead of Full Title.
% \titlenote{with title note}             %% \titlenote is optional;
                                        %% can be repeated if necessary;
                                        %% contents suppressed with 'anonymous'
% \subtitle{Subtitle}                     %% \subtitle is optional
% \subtitlenote{with subtitle note}       %% \subtitlenote is optional;
                                        %% can be repeated if necessary;
                                        %% contents suppressed with 'anonymous'


%% Author information
%% Contents and number of authors suppressed with 'anonymous'.
%% Each author should be introduced by \author, followed by
%% \authornote (optional), \orcid (optional), \affiliation, and
%% \email.
%% An author may have multiple affiliations and/or emails; repeat the
%% appropriate command.
%% Many elements are not rendered, but should be provided for metadata
%% extraction tools.

%% Author with single affiliation.
\author{First1 Last1}
\authornote{with author1 note}          %% \authornote is optional;
                                        %% can be repeated if necessary
\orcid{nnnn-nnnn-nnnn-nnnn}             %% \orcid is optional
\affiliation{
  \position{Position1}
  \department{Department1}              %% \department is recommended
  \institution{Institution1}            %% \institution is required
  \streetaddress{Street1 Address1}
  \city{City1}
  \state{State1}
  \postcode{Post-Code1}
  \country{Country1}
}
\email{first1.last1@inst1.edu}          %% \email is recommended

%% Author with two affiliations and emails.
\author{First2 Last2}
\authornote{with author2 note}          %% \authornote is optional;
                                        %% can be repeated if necessary
\orcid{nnnn-nnnn-nnnn-nnnn}             %% \orcid is optional
\affiliation{
  \position{Position2a}
  \department{Department2a}             %% \department is recommended
  \institution{Institution2a}           %% \institution is required
  \streetaddress{Street2a Address2a}
  \city{City2a}
  \state{State2a}
  \postcode{Post-Code2a}
  \country{Country2a}
}
\email{first2.last2@inst2a.com}         %% \email is recommended
\affiliation{
  \position{Position2b}
  \department{Department2b}             %% \department is recommended
  \institution{Institution2b}           %% \institution is required
  \streetaddress{Street3b Address2b}
  \city{City2b}
  \state{State2b}
  \postcode{Post-Code2b}
  \country{Country2b}
}
\email{first2.last2@inst2b.org}         %% \email is recommended


%% Paper note
%% The \thanks command may be used to create a "paper note" ---
%% similar to a title note or an author note, but not explicitly
%% associated with a particular element.  It will appear immediately
%% above the permission/copyright statement.
% \thanks{with paper note}                %% \thanks is optional
                                        %% can be repeated if necesary
                                        %% contents suppressed with 'anonymous'


%% Abstract
%% Note: \begin{abstract}...\end{abstract} environment must come
%% before \maketitle command
\begin{abstract}
Programmers, 
like ``pen-and-paper'' mathematicians, often seek to define ``syntactic sugar'' that  lowers the syntactic cost of common idioms. 
Unfortunately, the most direct mechanisms available to programmers, e.g. \li{camlp4} \cite{ocaml-manual} and Sugar* \cite{erdweg2011sugarj,erdweg2013framework}, do not support modular reasoning about determinism (i.e. separately defined forms can conflict syntactically with one another), and they obscure the type and binding structure of the program. We argue therefore that these mechanisms are unreasonable for programming ``in the large''.

This paper formally introduces \emph{parametric typed syntax macros} (pTSMs), which give library providers programmatic control over both the parsing and expansion of expressions and patterns of \emph{generalized literal form} at a specified type or parameterized family of types. Expansion is \emph{strictly hygienic}, meaning that expansions can refer to external bindings only via spliced terms or explicit module parameters, and expansions do not reveal internal bindings to spliced terms or to the remainder of the program. 
% Partial parameter application lowers the syntactic cost of this strict style. 
This design ensures that clients are able to reason about the binding structure of the program while holding the expansion of the program abstract. The system needs only convey to clients a \emph{segmentation} of each literal that gives the locations of the spliced terms. The system can determine this {segmentation} automatically and convey it to the client via syntax highlighting, or by presenting the client with a context-free grammar (without revealing the semantic actions associated with each production.) 
We argue that this mechanism occupies a ``sweet spot'' in the design space, in that it captures many common syntactic idioms while avoiding the problem of syntactic conflicts by construction and supplying clients with clear abstract reasoning principles.


% \emph{Typed syntax macros (TSMs)}, proposed in a recent short paper by Omar et al. \cite{sac15}, give library providers programmatic control over the parsing and expansion of only terms of {(generalized) literal form}. This appears to occupy a ``sweet spot'' in that it captures many common syntactic idioms while avoiding the problem of conflict. TSMs also maintain a reasonable type and binding discipline.% In particular, clients can use any combination of TSMs in a program without needing to consider conflicts between them, and the language validates each expansion that a TSM generates to maintain 1) a \emph{type discipline} (meaning clients can determine the type of an unexpanded expression without examining its expansion directly); 
% %and 2) a \emph{hygienic binding discipline}.
% %, meaning that the expansion cannot make any assumptions about bindings at the application site, nor  introduce ``hidden bindings'' into subterms. 

% TSMs have only been described minimally -- it is not clear how they should be adapted for integration into languages like ML that support {pattern matching}, {parameterized datatypes}, {modules} and abstract types. Moreover, the prior work makes several simplifying assumptions related to binding that are impractically restrictive.% bno mechanism for binding values for use across TSM definitions has been described and the hygiene mechanism makes giving the expansions that they generate access to ``helper functions'' awkward.

% This paper gives a complete account of TSMs that addresses these deficiencies. In particular, we 1) integrate TSMs with pattern matching; 2) introduce a distinct static phase of evaluation, which gives TSM definitions access to libraries; and 3) introduce type and module parameters, which serve two purposes: they allow for TSMs that operate uniformly at a parameterized family of types (rather than only at a single type), and they give expansions explicit, hygienic access to libraries. Support for partial application  of parameters lowers the syntactic cost of this explicit approach. 

% Put succinctly, we design a programming language in the ML tradition with a \emph{reasonably} programmable syntax.
\end{abstract}


%% 2012 ACM Computing Classification System (CSS) concepts
%% Generate at 'http://dl.acm.org/ccs/ccs.cfm'.
% \begin{CCSXML}
% <ccs2012>
% <concept>
% <concept_id>10011007.10011006.10011008</concept_id>
% <concept_desc>Software and its engineering~General programming languages</concept_desc>
% <concept_significance>500</concept_significance>
% </concept>
% <concept>
% <concept_id>10003456.10003457.10003521.10003525</concept_id>
% <concept_desc>Social and professional topics~History of programming languages</concept_desc>
% <concept_significance>300</concept_significance>
% </concept>
% </ccs2012>
% \end{CCSXML}

% \ccsdesc[500]{Software and its engineering~General programming languages}
% \ccsdesc[300]{Social and professional topics~History of programming languages}
%% End of generated code


%% Keywords
%% comma separated list
% \keywords{keyword1, keyword2, keyword3}  %% \keywords is optional


%% \maketitle
%% Note: \maketitle command must come after title commands, author
%% commands, abstract environment, Computing Classification System
%% environment and commands, and keywords command.
\maketitle


\section{Introduction}

Text of paper \ldots


%% Acknowledgments
\begin{acks}                            %% acks environment is optional
                                        %% contents suppressed with 'anonymous'
  %% Commands \grantsponsor{<sponsorID>}{<name>}{<url>} and
  %% \grantnum[<url>]{<sponsorID>}{<number>} should be used to
  %% acknowledge financial support and will be used by metadata
  %% extraction tools.
  This material is based upon work supported by the
  \grantsponsor{GS100000001}{National Science
    Foundation}{http://dx.doi.org/10.13039/100000001} under Grant
  No.~\grantnum{GS100000001}{nnnnnnn} and Grant
  No.~\grantnum{GS100000001}{mmmmmmm}.  Any opinions, findings, and
  conclusions or recommendations expressed in this material are those
  of the author and do not necessarily reflect the views of the
  National Science Foundation.
\end{acks}


%% Bibliography
\bibliography{../../papers/research}


%% Appendix
\appendix
\section{Appendix}

Text of appendix \ldots

\end{document}
